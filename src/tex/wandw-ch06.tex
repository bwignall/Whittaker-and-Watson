\chapter{The Theory of Residues; Application to the Evaluation of Definite Integrals} 

6"1. Residues. - '-  

If the function /( r) has a pole of order m at 2 = a, then, by the definition 
of a pole, an equation of the form 

where   (z) is analytic near and at a, is true near a. 

The coefficient a i in this expansion is called the residue of the function 
f z) relative to the pole a. 

Consider now the value of the integral 1 f z)dz, where the path of 

J a 

integration is a circle* a, whose centre is the point a and whose radius p is so 
small that    z) is analytic inside and on the circle. 

r m f dz r 

We have f z)dz= S a\ r , w+   z)dz. 

J a' r = l J a\ 2 ~ (i) J a. 

Now [ (f) z)dz = by § 52 ; and (putting z-a = pe' ) we have, if r l, 

gd-r) ie' 
1 -r 



JA -aV Jo p'-e"  '  Jo 



,,  z - ay J p''e 
But, when ?• = 1, we have 



27r 

= 0. 





r \ d  p" 

Hence finally I f(z) dz = 2 



idd = 27rz. 



Now let C be any contour, containing in the region interior to it a number 
of poles a,h,c,... of a function f z), with residues a\ i, 6\ i, c\ i, ... respec- 
tively : and suppose that the function f z) is analytic throughout C and its 
interior, except at these poles. 

Surround the points a, b,c, ... by circles a, /3, 7, ... so small that their 
respective centres are the only singularities inside or on each circle ; then the 
function f z) is analytic in the closed region bounded by C, a, /3, 7, .... 

* The existence of such a circle is implied in the definition of a pole as an isolated 
singularity. 



112 THE PROCESSES OF ANALYSIS [CHAP. Yl 

Hence, by § 5-2 corollary 3, 

! f z)dz=\ f z)dz+\ f z)dz+... 

J C   a J /3 

= 27rm\ i + 27n'6\ i + — 

Thus we have the theorem of residues, namely that if f z) he analytic 
throughout a contour C and its interior except at a number of poles inside the 
contour, then 

f f z)dz = 27ri R, 
J c 

where XR denotes the sum of the residues of the function f (z) at those of its 
poles which are situated within the contour G. 

This is an extension of the theorem of § 5*21. 

Note. If a is a simple pole oi f z) the residue of f z) at that pole is lim   z-a)f z) . 
6 "2. The evaluation of definite integrals. 

We shall now apply the result of § 6*1 to evaluating various classes 
of definite integrals ; the methods to be employed in any particular case may 
usually be seen from the following typical examples. 

6"21. The evaluation of the integrals of certain periodic functions taken 
between the limits and 27r. 



An integral of the type 



2n 



R (cos e, sin 6) dd, 



where the integrand is a rational function of cos 6 and sin 6 finite on the 
range of integration, can be evaluated by writing e'  = z ; since 

cos   - 1 (  + z-% sin   = — . (2 -  -1), 

the integral takes the form I S z)dz, where 8 z) is a rational function of z 

J c 
finite on the path of integration C, the circle of radius unity whose centre is 

the origin. 

Therefore, by § 61, the integral is equal to  wi times the sum of the residues 
of S  z) at those of its poles which are inside that circle. 

Example 1. If <p < 1, 

p-  de   f dz 

jo 1 -2/>cos +/>2 j f;i l-pz) z-p)' 

The only pole of the integrand inside the circle is a simple pole at p ; and the residue 
there is 

lim * -r 



, i  pz) z-p)  •(l- 2)  



6 '2-6 -22] THE THEORY OF RESIDUES 113 

de 277 



Hence 



jo 1  



  Example 2. If <p < 1, 

j 1 - '2p cos 2 + 52 -J  iz\ 2' ' 2' ' ) (1 -jt)22) (1 \ p5-2) 

= 27r2i?, 

(26+1)2 

where 2  denotes the snm of the residues of . . ,  sr -s   at its poles inside C ; these 

4s-"' (1 — pz )  z  — p) 

1 1 l+p  + p* ( 3+l)2 ( 3+XY2 

poles are 0, -p , p  ; and the residues at them are  -o ,  C —  , ,   "C,-. ;tt ; 

and hence the integral is equal to 

Tr l-p+p ) 

Example 3. If n be a positive integei", 

/ 2t 2ir /"2t 

I e '99cos(n -sin )o?(9= — , / e* "  sin (?i<9- sin  )rf  = 0. 

Example 4. If a > 6 > 0, 

pT de 2na pT 0?  \  7r(2a + 6) 

i  (  + 6cos )2 (a2\ t2)l' jo  a + hcoii 6f a  a + hf' 

622. T/ie evaluation of certain types of integrals taken between the limits 

— 00 and + 00 . 

We shall now evaluate I Q (x) dx, where Q  z) is a function such that 

J — X 

(i) it is analytic when the imaginary part of z is positive or zero (except at a 
finite number of poles), (ii) it has no poles on the real axis and (iii) as | 2  |—   00 , 
zQ z) >Q uniformly for all values of arg  such that O arg  Tr; provided 
that (iv) when x is real, xQ x)—>0, as ic— > + x , in such a way* that 

I Q  x) dx and 1 Q  x) dx both converge. 

Jo J -00 

Given e, we can choose p  (independent of arg2 ) such that \ zQ z) \ < ejir 
whenever |  ; | > po and   arg z  ir. 

Consider I Q(z)dz taken round a contour C consisting of the part of the 

real axis joining the points + p (where p > po) and a semicircle F, of radius p, 
having its centre at the origin, above the real axis. 

Then, by § 6"1, I Q (z) dz — 27ri'ER, where 2i2 denotes the sum of the 

J c 

residues of Q z) at its poles above the real axisf. 

* The condition xQ (x)   is not in itself sufficient to secure the convergence of I Q (x) dx ; 

consider Q (x) = (x log .t)~i. 

t Q(z) has no poles above the real axis outside the contour. 

W. M. A. 8 



114 THE PROCESSES OF ANALYSIS [CHAP. VI 

Therefore I f ' Q (z) dz - 27riSi? ' = I Q (z) dz . 

\ J -p \ J T 

In the last integral write z = pe , and then 

11 Q(z)dz\ = \ rQ (pe* ) pe 'idd 

\ j r I 1 Jo 

< Wei'JT)de 
Jo 

= e, 

by § 4-62. 

Hence lim 1 Q  z) dz =  iri R. 

p- -oo J -p 

But the meaning of | Q (x) da; is lim I Q (x) dx ; and since 
lim 1 Q (a;) rf  and lim Q (x) dx both exist, this double limit is the 

(r- -oo Jo   p- x> J -p 

same as lim Q (x) dx. 

p- oo -J — p 

Hence we have proved that 

Q(x)dx= 2'7rilR. 



This theorem is particularly useful in the special case when Q(x) is a 
rational function. 

[Note. Even if condition (iv) is not satisfied, we still have 

I  Q x) + Q(-a;) dx=\ im T Q x)dx = 2Tri2R.] 

./ p f:o J —P 

Exaynple 1. The only pole of  z + ) ~  in the upper half plane is a pole at   =    with 



3 

residue there -  ttt i- Therefore 
lb 



/"  dx \  3 

Example 2. If a > 0, 6 > 0, shew that 

r= - x*dx \  IT 
•'-• (a + bx )  I6aib  

Example 3. By integrating / e- - dz round a parallelogram whose corners are 
- R, R, R + ai, - R + ai and making R- qc , shew that, if X > 0, then 

/ e-'' ' coii 2Xax)dx = e- <>-'' I e->'x-dx = 2X-h e- < - j e-< dx. 

6221. Certain infinite integrals involving sines and cosines. 

If Q(z) satisfies the conditions (i), (ii) and (iii) of § 6-22, and ni > 0, then 
Q (z) e""'  also satisfies those conditions. 



THE THEORY OF RESIDUES 



115 



6-221, 6-222] 

Hence I  Q (a;) e' ''' + Q  - x) e'"''''] dx is equal to  ttiIR', where 2E' 

means the sum of the residues of Q  z) e   at its poles in the upper half plane ; 
and so 

(i) If Q  x) is an even function, i.e. if Q (— x) — Q (x), 

/• CO 

I Q (x) cos (jyix) dx = ttiSR. 

Jo 

(ii) If Q (x) is an odd function, 

Q (x) sin (inx) dx = ttSR'. 



6'222. Jordan's lemma*. 

The results of § 6'221 are true if Q  z) be subject to the less stringent 
condition Q(z)—>0 uniformly when O arg Tr as \ z:—>oo in place of the 
condition zQ z)- 0 uniformly. 

To prove this we require a theorem known as Jordan's lemma, viz. 

If Q(z) >0 iDiiformly with regard to argz as j  — ►x v)hen O arg Tr, 
and if Q (z) is analytic luhen both \ z'.>c a constant) and   arg z  w, then 

liin (\ e'"" Q (z) dz) = 0, 

where T is a semicircle of radius p above the real axis with centre at the origin. 

Given e, choose po so that \ Q z)\ <  ejir when \ z  p,  and   arg z   tt; 
then, if p > po. 



 el'Tr)pe-' p ''">dd 



e"""  Q  z) dz 
But |emipeose| = i  and so 

e""  Q  z) dz 



= (2€/7r)   ' pe-' '> i" f/6'. 



Now sin 6   2 /7r, whenf "S     tt, and so 



e' '  Q  z) dz 



< (2e/7r) '" pe'-' p l' dd 
Jo 



= (2e/7r).(7r/2m) 
< e/m. 



\  g-2mp0/7r 



* Jordan, Cours d'Aiiahjse, ii. (1894), pp. 285, 286. 

t This inequality appears obvious when we draw the graphs )/ = sin.r, y = 2xjir; it may be 
proved by shewing that (sin 6)1\$ decreases as 6 increases from to iir. 

8—2 



116 THE PROCESSES OF ANALYSIS [CHAP. VI 



Hence lim | e'"'' Q (z) dz = 0. 



This result is Jordan's lemma. 

Now 

[" |e *  Q (x) + e-' *  Q (- a;)] da; = 2TrilR' - I e""'  Q (z) dz, 

Jo J T 

and, making p—>oo, we see at once that 

 e"    Q (x) + e-'"''  Q (- x)] dx = l-rri R, 

Jo 

which is the result corresponding to the result of § 6"221. 

Example 1. Shew that, if a > 0, then 

/"" eos r T ,\  ?r \   
'o x  + a  2a 





Example 2. Shew that, if a   0, 6 0, then 

cos 2ax — cos 'ihx 



i: 



dx=n  h-a) 



X'- 

(Take a contour consisting of a large semicircle of radius p, a small semicircle of 
radius 8, both having their centres at the origin, and the parts of the real axis joining their 
ends ; then make p-  cxj , S- 0.) 

Example 3. Shew that, if 6 > 0, m   0, then 

I ,— ; — T TZTi cos mxclx = — j ~  Sb'  - a  — mb (36- + a )|. 
Jo  x  + o' y 46-'   

Example 4. Shew that, if  ' > 0, a > 0, then 

/" ""   sin ax , , , 
I — 5 — ,, ax=i7re~'"K 

Example 5. Shew that, if m   0, a > 0, then 



/ 



sin mx , tt Tre""*" / 2 
.r(j??- + a2)2 2a'' 4a  \ a 



(Take the contour of example 2.) 

Example 6. Shew that, if the real part of s be positive, 

[  e-*-e-'')j=\ ogz. 
[We have 

Ji>   ' t s o,p  [Js i Js t J 

lim -! I — dt— I - — du 



a- .0, p-*-x I./ 5   J 5c 

= lim I / " —-dt-] —- dt\ , 
since i~' e~' is analytic inside the quadrilateral whose corners are 8, 8z, pz, p. 



6"23, 6*24] THE THEORY OF RESIDUES 117 

Now I t~  e~ dt- -0 a,s, p- cc when (2)>0; and 

/ ~ t-  e- dt = \ ogz- I " t-  l-e-f) dt- \ ogz, 

since i''   1 -e-*)- l as t- 0.] 

6'23. Principal values of integrals. 

It was assumed in §§ 6'22, 6 '221, 6*222 that the function Q  x) had no poles on the real 
axis ; if the function has a finite number of simple poles on the real axis, we can obtain 
theorems corresponding to those already obtained, except that the integrals are all principal 
values (§ 4'5) and 2/2 has to be replaced by S/  +  S/s'q, where 2  means the sum of 
the residues at the poles on the real axis. To obtain this result we see that, instead of 
the former contour, we have to take as contour a circle of radius p and the portions of the 
real axis joining the points 

-p, a-8i; a + hi,h-b~2\ b + 8o, c-8 , ... 

and small semicircles above the real axis of radii Sj, So, ... witli centres a, b, c, ..., where 
a, b, c, ... are the poles of Q (2) on the real axis ; and then we have to make Sj, 8. , ... - 0 ; 
call these semicircles yj, y.,, Then instead of the equation 

I Q z)dz+ I Q (2) dz = 27ri2R, 

we get F j Q (2) dz + -2 Ynn j Q (2) dz+ j Q (2) dz = -Irri  It. 

Let a' be the residue of Q z) at a ; then writing z = a + 8ie'  on y  we get 

j Q (2) dz= (  Q a + 8i e' ) Sj e'6 id0. 
But Q  a + 81 6 )816' - a uniformly as 8,- 0; and therefore lim | Q z)dz= -iria' ; 

P y Q z)dz+( Q z)dz = 27ri2R + ni2R,,, 



we thus get 



-p 
and hence, using the arguments of i  (5-22, we get 



P j Q x) dx = 2m  2R +  2 /?o) 



UA 



The reader will see at once that the theorems of   6"221, 6-222 have precisely similar 
generalisations. 

The process employed above of inserting arcs of small circles so as to diminish the area 
of the contour is called indenting the contour. 

r "  
.  6"24. Evaluation of integrals of the fovTii \ af'~ Q x)dx. 

J 

Let Q x) be a rational function of x such that it has no poles on the 
positive part of the real axis and x Q x)—¥  both when a;— >0 and when 



118 



THE PROCESSES OF ANALYSIS 



[chap. VI 



Consider l(— zY~'  Q z)dz taken round the contour G shewn in the figure, 



consisting of the arcs of circles of radii 
p, S and the straight lines joining their 
end points ; (— zY"'  is to be interpreted 
as 

exp  ( a-1) log (- z)] 
and 

log (-  ) = log •  ' +   arg (- z), 

where — tt   arg (—  )   tt ; 

with these conventions the integrand is 
one-valued and analytic on and within 
the contour save at the poles of Q  z). 

Hence, if S?' denote the sum of the 
residues of (— zY~'  Q (z) at all its poles, 




[ (- zy-' Q (z) dz = 27ri'Zr. 



On the small circle write — z= Se , and the integral along it becomes 
— I (— zyQ z)id6, which tends to zero as 8— >0. 



On the large semicircle write — z = pe' , and the integral along it becomes 
— I  — sT Q (2) idO, which tends to zero as p— > x . 

On one of the lines we write — z = xe"\ on the other — z = xe~' ' and 
(-zy~  becomes a; - e±'"-i>'''. 

Hence 

lim r [w''-' e- <"-'' ' * Q (x) - x -'e'< - > "' Q (x)] dx = l-Trilr ; 

(S- .0, p oc ) J 5 

and therefore I af ~'  Q (x) dx = tt cosec (air) Sr. 

.'0 

Corollary. If Q x) have a number of simple poles on the positive part 
of the real axis, it may be shewn by indenting the contour that 

P j a;"~  Q (x) dx — tt cosec (citt) S?' — tt cot ((/tt) 2r', 
. 

where 1r' is the sum of the residues of z"'-  Q (z) at these poles. 
Example 1. If < a < 1, 

I dx=ir cosec arr, P I dx=iv cot air. 

Jii  + X jn  x 



6'3, 6*31] THE THEORY OF RESIDUES 119 

Example 2. If < 2 < 1 and - tt < n < tt, 



  



fz-l  gi(2— l)a 

, — .'di=—. . (Minding.) 



Example 3. Shew that, if - 1 < s < 3, then 



/; 



  dx - -'  



/: 



(Euler.) 



   x' 'f ' ~4cos 7r2' 
Example 4. Shew that, if — 1 < p < 1 and - tt < X < tt, then 

x~P dx \  TT sin X 
1 +2x cos X + X- sin pn sin X 

6 "3. Cauchy's integral. 

 V'e shall next discuss a class of contour-integrals which are sometimes found useful 
in analytical investigations. 

Let Cbe a contour in the 2-plane, and let/(s) be a function analytic inside and on C. 
Let (2) be another function which is analytic inside and on G except at a finite luimber 
of poles ; let the zeros of <  z) in the interior* of C be a,, 02, ..., and let their degrees of 
multiplicity be rj, ro, ... ; and let its poles in the interior of C be 6j, ho, ..., and let their 
degrees of multiplicity be 5, , §2)  ••• 

Then, by the fundamental theorem of residues, - — . I fiz) dz is equal to the sum 

 2ni J c' (f> z) 

of the residues of--—   ' at its poles inside C. 
<i> z) 

Now— —- - can have singularities only at the poles and zeros of 0(s). Near one 

of the zeros, say otj , we have 

0(2) = iI(2-ai)'-i-|-5(2-ai)'-. + i-f-.... 
Therefore < ' (2) = Ar   z - a- Yi -1 + 5 (r, + 1 ) (s - ai) -i + . . . , 

and /(2)=/( i) + (2- i)/'( i) + .... 

Therefore P  - '  ] is analytic at a, . 

Thus the residue of j   - , at the point 2 =  ,, is rif ai). 
Similarly the residue at 2 = 61 is — Si/(6j); for near 2 = 61, we have 

( (2) = C(2-6,j- . + Z>(2-6,)-*, + l-l-..., 

and f z)=f bi) +  z-b,)f' b,) + ..., 

so f!'  + '-  is analytic at i . 
0(2) 2-61 

Hence i J/ '  f  clz = 2r,fia,) - 2s,f b,), 

the summations being extended over all the zeros and poles of (f) (2). 

6'31. The mimher of roots of an equation contained loithin a contour. 
The result of the preceding paragraph can be at once applied to find how many roots of 
an equation cj)  z) = lie within a contour C. 

For, on putting /(2) = 1 in the preceding result, we obtain the resvilt that 



z) 



27rilc  z)' ' 

is equal to the excess of the number of zeros over the number of poles of (2) contained in 

the interior of C, each pole and zero being reckoned according to its degree of multiplicity. 

* (f) ( ) must not have any zeros or poles on C. 



120 THE PROCESSES OF ANALYSIS • [CHAP. VI 

Example 1, Shew that a polynomial <  (2) of degree m has m roots. 

Let (  (2) = ao2™ + ais'"~  + . .. +  , , ( o=t=0)- 

rpj gj ) \  >  og'"~  + •••+    -1 

( (2) ao2'  + ...+ ,  

Consequently, for large values of | 2 | , 

Thus, if C be a circle of radius p whose centre is at the origin, we have 

27rt J c (p  ) 27r  \ / c 2 27ri /   \ z / 2itI J c V J 

But, as in § 6-22, j (\ \   dz O 

as p- ao ; and hence as (f) z) has no poles in the interior of C, the total number of 
zeros of (f) (2) is 

lim -— . I  - ( 2 = TO. 

Example 2. If at all points of a contour C the inequality 

l fcS*|>l o +  i2 + ... + ai-i~ * ~  +  fc+i2' i + ... + a, 2'"j 
is satisfied, then the contour contains /(• roots of the equation 
a , 2'" +  , \ ! 2'  - 1 + . . . +  ! 2 +  o = 0. 
For write / (2) = a z''' + a   \  1 2"'- - 1 + . . . + ai2 + ao . 

Then /(2) =  ,2 i   -- + ...+a..,i2  -fa,-i2> -' + ...+ao\ 

= a,2' (l + tO, 
where | f/"!  a < 1 on the contour, a being independent* of 2. 
Therefore the number of roots of f z) contained in C 

27 jc/(2) 2injc\ s l+U dzj 

But I — — 2T!-i; and, since | C/'|<1, we can expand (1 + U)~'  in the uniformly cou- 
  vergent series 

Therefore the number of roots contained in C is equal to i: 
Example 3. Find how mauy I'oots of the equation 

26 + 62+10=0 
lie in each quadrant of the Argaud diagram. (Clare, 1900.) 

* I C/ 1 is a continuous function of z on C, and so attains its upper bound (§ 3-62). Hence its 
upper bound a must be less than 1. 



> 



6 "4] THE THEORY OF RESIDUES 121 

6 "4. Connexion between the zeros of a function and the zeros cf its derivate. 

Macdonald* has shewn that if fiz) he a fvMction of z analytic throughout the interior of 
a single closed contour C, defined hy the equation \ f z) i = 3  where M is a constant, then the 
number of zeros of f z) in this region exceeds the number of zeros of the derived function 
f  z) in the same region hy unity. 

On C \ etf z) = 3fe'e ; then at points on C 

/  =*'"-|-  '•" = ' "" 'S-(I) - 

Hence, by § 6'31, the excess of the number of zeros oi f z) over the number of zeros 
of/'(s) inside + C is 

27   j c f z) ~ 27ri j cf (2) 27rt j c U ' / dz) ' • 

Let s be the arc of C measured from a fixed point and let y  he the angle the tangent to 
C makes with Ox ; then 

1 f (dH /de\ , 1 r de'] 

-2 ija [dJ  I dz)  = - 2 ?rS dz]c 

I r, d0 , dzl 

= -2 ir ds- ''Sdsjc- 

dB 
Now log , is purely real and its initial value is the same as its final value ; and 

dz 
log =i\ j/ ; hence the excess of the number of zei'os oi f z) over the number of zeros of 

/' (s) is the change in y rj'iiT in describing the curve C ; and it is obvious J that if C is any 
ordinary curve, >//  increases by 27r as the point of contact of the tangent describes the 
curve C ; this gives the required result. 

Example 1. Deduce from Macdonald's result the theorem that a polynomial of degree 
n has n zeros. 

Example 2. Deduce from Macdonald's result that if a function/ (2), analytic for real 
values of z, has all its coefficients real, and all its zeros real and different, then between 
two consecutive zeros oi f z) there is one zero and one only of/' (2). 



KEFERENCES. 

M. C. Jordan, Cours dJ Analyse, 11. (Paris, 1894), Ch. vi. 

E. GouRSAT, Cours d' Analyse (Paris, 1911), Ch. xiv. 

E. LiNDELOF, Le Calcid des Residus (Paris, 1905), Ch. il. 

* Froc. London Math. Soc. xxix. (1898), pp. 576, 577. 

t /' ( 2) does not vanish on C unless C has a node or other singular point ; for, if f= <p + i\ p, 

where <A and xL are real, since i J- - /- , it follows that if /'(2) = at any point, then 
  ax dy 

  ,   , ~,   all vanish ; and these are sufficient conditions for a singular point on 
ox dy ox ay 

X For a formal proof, see Proc. London Math. Soc. (2), xv. (1916), pp. 227-242. 



122 



THE PROCESSES OF ANALYSIS 



[chap. VI 



Miscellaneous Examples. 

1. A function (f) (2) is zero when 2 = 0, and is real when z is real, and is analytic when 
2 I   1 ; if f(x, y) is the coefficient of i in  x + iy), prove that if - 1 <  < 1, 



'27r 



1 - 'ix cos 



  - / aoa e, sin 6) de = n4> ( ). 



(Trinity, 1898.) 



2. By integrating  -  — round a contour formed by the rectangle whose corners are 
e"' ' — 1 



0, R, B + i, i (the rectangle being indented at and and making R- xi , shew that 

(Legendre.) 



/. 



sin ax , 1 e" + 1 
ax = - 



3ttX 



1 



4 e -l 



1 
2a 



3. By integi-ating log (-2) Q z) round the contour of   6-24, where Q z) is a rational 
function such that 2 (2) 0 as ; 2 |- 0 and as ] 2 j -  x , shew that if Q (2) has no poles 

on the positive part of the real axis, I Q (.r) dx is equal to minus the sum of the 

.' 
residues of log  -z)Q (2) at the poles of Q (z) ; where the imaginary part of log ( - 2) lies 
between ± tt. 

4. Shew that, if a > 0, 6 > 0, 



f dv 

I e cos6xgii  (a sin hx)-  =-|7r(e -l). 



5. Shew that 



/ 



a sin 2x 



- — -, xdx = - TT log (1 +0), ( - 1< a < 1; 

l-2acos2.r + a2 4 b\ t j, \ 



-7rlog(l + a-i),  a?>r, 



6. Shew that 



(Cauchy.) 



f sva.<i>iX sintboX sind) .r snia r , "  . . , 

/ 2\ !\   --=— ... — - --  COS ai r . . . COS a, a7 a.r = — m, (Do ... (  , 

J Q X X X X '•I  -   

if < i) 02)  •• ni oi) a-i) •••o-m be real and a be positive and 

 >1< 1 | + |< 2i+---+l0nl + lai|+ ...+|a J. 

(Stormer, Acta Math, xix.) 

7. If a point 2 describes a circle C of centre a, and if /(2) be analytic throughout 

C and its interior except at a number of poles inside C, then the point u=f z) will 

describe a closed curve y in the w-plane. Shew that if to each element of y be attributed 

a mass proportional to the corresponding element of C, the centre of gravity of y is the 

flz) 
point r, where r is the sum of the residues of • — -!- at its poles in the interior of C. 



(Amigues, Noiiv. Ann. de Math. (.3), xii. (1893), pp. 142-148.) 



8. Shew that 



9. Shew that 



/ 



dx 



7r 2a + b) 



-   (. 2 + 62)  x  + a2)2 <ia?h (a + 6)2 " 



/: 



dx 



IT 1.3...(2?i-3) 1 



(  + 6 2)   n h l.2...(H-l) a"" ' 



THE THEORY OF RESIDUES 123 



10. If Fn  z)= n n (1 - 2' P), shew that the series 

m = l ]i=l 

fiz)=- I  "( '' ") 



1=2 (2"??~"-l)%""  

is an analytic function when z is not a root of any of the equations 2" = %" ; and that the 
sum of the residues of f z) contained in the ring-shaped space inchided between two 
circles whose centres are at the origin, one having a small radius and the other having 
a radius between n and n + 1, is equal to the number of prime numbers less than n + l. 
(Laurent, A' oiiv. Aim. de Math. (3), xviii. (1899), pp. 234-241.) 

11. If -4 and B represent on the Argand diagram two given roots (real or imaginary) 
of the equation /(j) = of degree n, with real or imaginary coefficients, shew that there is 
at least one root of the equation/'  z) = within a circle whose centre is the middle point 

of J  and whose radius is lAB cot - . (Grace, Proc. Camb. Phil. Soc. xi.) 

n 



12. Shew that, if 0<i'<l, 



= s —   lim 2 



[Consider / 



g 2v-l)ziri  2 

round a circle of radius   + 5 ; and make /i- -x .] 



s\ nirz z — x 

(Kronecker, Journal fiir Math, cv.) 
1 3. Shew that, if m > 0, then 

* sin" mt 



L 



dt 



1: 



°2 g  '"--'T<"- >- "'2i '"'- >'-- ""'"3V''"" ('- ''>'- + •   • 
Discuss the discontinuity of the integral at m = 0. 

14. If A + B + C+ ...=0 and a, b, c, ... are positive, shew that 

 cosa.r+ficos6.r+... +  cos a; , ., di 7, i-i /. 
dx= -A log a - 5 log b- ,..- A log a:. 

'*' 

(Wolstenholme.) 

/•gX(*+ ) • • 1 

15. By considering I -j r dt taken round a rectangle indented at the origin, shew 

that, if k > 0, 

I lim I -. — -T dt=ni+ lim PI — dt, 

p aa J -P  ' + '  p x J -p t 

and thence deduce, by using the contour of § 6*222 example 2, or its reflexion in the real 
axis (according as a;   or .v < 0), that 



1 /"p px(k + ti) 

Um - , . dt = 2, 1 or 0, 



k + ti 



p- .oo   J -P 

according as x>0, x = or x < 0. * 

[This integral is known as Cauchy's discontinuous factw.'] 
1 6. Shew that, if < a < 2, 6 > 0, /  > 0, then 

= l-,  -l a—br 



.r -i sin ( aTT -607) -5—  -  = i7rr  



124 THE PROCESSES OF ANALYSIS [CHAP. VI 

17. Let  ; > and let 2 e-n-' f = ylr t). 

 l=-oo 

r g-Z TTt 

By considering / -  — -dz round a rectangle whose corners are ± JV+ )±i, where 
N is an integer, and making N-*- qo , shew that 

By expanding these integrands in powers of e~ ' , e-""  respectively and integrating 
term-by-term, deduce from § 6"22 example 3 that 

 TTty J ---= 

Hence, hy putting t = l shew that 

v.(o= -H(i/o. 

(This result is due to Poisson, Journal de VEcole foly technique, xii. (cahier xix), (1823), 
p. 420 ; see also Jacobi, Journal fiir Math, xxxvi. (1848), p. 109 [Oes. Werke, ii. (1882), 
p. 188].) • 

18. Shew that, if i;>0, 



2 e 

J( = - 00 



- ntTTt - 2nnat = t ~   e'"''-  \ \  + ll 2 c - "' W< COS 2n7ra i 



(Poisson, Mem. de I'Acad. des Sci. vi. (1827), p. 592 ; Jacobi, Journal fur Math. ill. 
(1828), pp. 403-404 [Ges. Werke, i. (1881), pp. 264-265] ; and Landsberg, Journal fur 
Math. CXI. (1893), pp. 234-253 ; see also § 21-51.) 

