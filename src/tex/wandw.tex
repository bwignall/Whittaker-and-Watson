\documentclass{book}

\usepackage{amsfonts}
\usepackage{amsmath}
\usepackage{amsthm}
\usepackage{hyperref}
\usepackage{ifthen}
\usepackage{makeidx}
\usepackage{mathrsfs}
\usepackage{shorttoc}

\renewcommand{\indexname}{General Index}
\makeindex

\usepackage[T1]{fontenc}
% \usepackage{lmodern}
% \usepackage{mathpazo}
\usepackage{courier}
% \usepackage{cmbright}
% \usepackage{ccfonts}
% \usepackage{fouriernc}

\usepackage{fancyhdr}
\pagestyle{fancy}
\usepackage{titlesec}

\renewcommand{\partmark}[1]{ \markboth{#1}{} }

\fancyhf{}
\fancyhead[CE]{ \nouppercase{\leftmark} }
\fancyhead[CO]{ \nouppercase{\rightmark}}

%%%%%%%%%%%%%%%%%%%%%%%%%%%%%%%%%%%%%%%%%%%%%%%%%%


% Use (A),(B),... for equations, not (1),(2),...
\numberwithin{equation}{subsection}
\renewcommand{\theequation}{\Alph{equation}}

\newcommand{\half}{ \frac{1}{2} }

\newcommand{\eps}{\epsilon}

%%%% Document Sectioning %%%%
% [Shorted title for Contents]{Chapter #}{Section #}{Title}
\newcommand{\Section}[4][]{%
  \ifthenelse{\equal{#1}{}}{ % Use chapter title in Contents
    \section*{\centering\normalfont #4}
    }{ % Use shortened title in Contents
      \section*[#1]{\centering\normalfont #4}
    }}
\newcommand{\Subsection}[4]{\subsection*{\centering\normalfont #4}}
\newcommand{\Subsubsection}[5]{\subsubsection*{\centering\normalfont #5}}

%%%% Named Operators %%%% 
\DeclareMathOperator{\ArcCos}{arc\,cos}
\DeclareMathOperator{\ArcCosec}{arc\,cosec}
\DeclareMathOperator{\ArcCot}{arc\,cot}
\DeclareMathOperator{\ArcSec}{arc\,sec}
\DeclareMathOperator{\ArcSin}{arc\,sin}
\DeclareMathOperator{\ArcTan}{arc\,tan}
\DeclareMathOperator{\cosec}{cosec}
\DeclareMathOperator{\sech}{sech}
\DeclareMathOperator{\cosech}{cosech}

\DeclareMathOperator{\argcosh}{arg\,cosh}
\DeclareMathOperator{\argcoth}{arg\,coth}
\DeclareMathOperator{\argsinh}{arg\,sinh}
\DeclareMathOperator{\argtanh}{arg\,tanh}

\newcommand{\arccosec}{\ArcCosec}
\newcommand{\arccot}{\ArcCot}
\newcommand{\arcsec}{\ArcSec}

\renewcommand{\arccos}{\ArcCos}
\renewcommand{\arcsin}{\ArcSin}
\renewcommand{\arctan}{\ArcTan}

\DeclareMathOperator{\Cis}{Cis}
\DeclareMathOperator{\Log}{Log}
\DeclareMathOperator{\sgn}{\textit{sgn}\,}
\DeclareMathOperator{\am}{am}

\DeclareMathOperator{\Real}{\mathbf{R}}
\DeclareMathOperator{\Imag}{\mathbf{I}}
\renewcommand{\Re}{\Real}
\renewcommand{\Im}{\Imag}


\newcommand{\hardY}{\mathsf{Y}} %TODO:improve %17.6

\newcommand{\absval}[1]{\left| #1 \right|}
\newcommand{\thebrace}[1]{\left\{ #1 \right\} }
\newcommand{\thebracket}[1]{\left[ #1 \right] }
\newcommand{\theparen}[1]{\left( #1 \right) }

\newcommand{\wandwsectionsymbol}{\S}
\newcommand{\wandwsubsectionsymbol}{\ensuremath{\cdot}}
\newcommand{\hardchapterref}[1]{\wandwsectionsymbol #1}
\newcommand{\hardsectionref}[2]{\wandwsectionsymbol #1\wandwsubsectionsymbol#2}
\newcommand{\hardsubsectionref}[3]{\wandwsectionsymbol #1\wandwsubsectionsymbol#2#3}
\newcommand{\hardsubsubsectionref}[4]{\wandwsectionsymbol
#1\wandwsubsectionsymbol#2#3#4}

\newcommand{\miscexamples}{\textsc{Miscellaneous Examples}}

% Add W&W-style end-of-line example citations
%\newcommand{\addexamplecitation}[1]{\hfill (\textrm{#1 TODO})}
\newcommand{\addexamplecitation}[1]{%
  \pagebreak[0]%
  \hfil\allowbreak\null\nobreak\hfill\nobreak\mbox{(#1)}%
  \pagebreak[1]%
}

\newcommand{\dd}{\mathbf{d}}
\newcommand{\dmeasure}{\, \dd}
\newcommand{\littleo}{o}

\newcommand{\corollary}{Corollary}

\theoremstyle{remark}
\newtheorem{wandwexample}{Example}
%\numberwithin{wandwexample}{subsubsection}

% Chapter 17 (Bessel functions)
\newcommand{\besselFont}[1]{\mathscr{#1}}
\newcommand{\besI}{\besselFont{I}}
\newcommand{\besJ}{\besselFont{J}}
\newcommand{\besK}{\besselFont{K}}
\newcommand{\besY}{\besselFont{Y}}

%%%%%%%%%%%%%%%%%%%%%%%%%%%%%%%%%%%%%%%%%%%%%%%%%%
%% Names
%%%%%%%%%%%%%%%%%%%%%%%%%%%%%%%%%%%%%%%%%%%%%%%%%%
\newcommand{\Fejer}{Fej\'er}
\newcommand{\Lame}{Lam\'e}
\newcommand{\Poincare}{Poincar\'e}
\newcommand{\Schlafli}{Schl\"afli}
\newcommand{\Schlomilch}{Schl\"omilch}



%%%%%%%%%%%%%%%%%%%%%%%%%%%%%%%%%%%%%%%%%%%%%%%%%%
%% Begin document
%%%%%%%%%%%%%%%%%%%%%%%%%%%%%%%%%%%%%%%%%%%%%%%%%%
\begin{document}
\begin{center}
\bfseries
\LARGE A COURSE of\\[12pt]
\Huge MODERN ANALYSIS\\[12pt]
\normalsize AN INTRODUCTION TO THE GENERAL THEORY OF\\
\normalsize INFINITE PROCESSES AND OF ANALYTIC FUNCTIONS;\\
\normalsize WITH AN ACCOUNT OF THE PRINCIPAL\\
\normalsize TRANSCENDENTAL FUNCTIONS

\par
\vfil

\normalfont
\normalsize BY \\
\bigskip
\Large E.~T. WHITTAKER, Sc.D., F.R.S.\\
\footnotesize PROFESSOR OF MATHEMATICS IN THE UNIVERSITY OF EDINBURGH\\
\bigskip
\footnotesize AND\\
\bigskip
\Large G.~N. WATSON, Sc.D., F.R.S.\\
\footnotesize PROFESSOR OF MATHEMATICS IN THE UNIVERSITY OF BIRMINGHAM
\vfil\vfil
\textsf{THIRD EDITION}
\vfil\vfil\vfil
\Large Cambridge \\
at the University Press \\
1920
\end{center}
\clearpage




%%%%%%%%%%%%%%%%%%%%%%%%%%%%%%%%
\null\vfill
\begin{center}
\textit{First Edition} 1902 \\
\textit{Second Edition} 1915 \\
\textit{Third Edition} 1920
\end{center}
\vfill\clearpage
%\shorttoc{Contents}{0}
\tableofcontents
\part{The Processes of Analysis}
\chapter{Complex Numbers}

\Section{1}{1}{Rational numbers.}

The idea of a set of numbers is derived in the first instance from the
consideration of the set of positive* integral numbers, or positive
integers; that is to say, the nnmbL-rs 1. 2, 8, 4, .... Positive
integers have many properties, which will be found in treatises on the
Theory of Integral Numbers; but at a very early stage in the
development of mathematics it was found that the operations of
Subtraction and Division could only be performed among them subject to
inconvenient restrictions; and consequently, in elementary
Arithmetic, classes of numbers are constructed such that the
operations of subtraction and division can always bB performed among
them.

To obtaiti a class of numbers among which the operation of subtraction
can be performed without restraint wc construct the class of integers,
which consists of the class of positive f integers (+1, +2, +3, ...)
and of the class of negative integers (-1, -2, -3, ...) and the number
0.

To obtain a class of numbers among which the operations both of sub-
traction and of division can be performed freely:):, we construct the
class of rational numbers. Symbols which denote members of this class
are |, 3,

0, -Y--

We have thus introduced three classes of numbers, (i) the signless
integers, (ii) the integers, (iii) the rational numbers.

It is not part of the scheme of this work to discuss the construction
of the class of integers or the logical foundations of the theory of
I'ational numbers .

The extension of the idea of number, which has just been described,
was not effected without some opposition from the more conservative
mathematicians. In the latter half of the eighteenth century, Maseras
(1731-1824) and Frend (1757-1841) published works on Algebra,
Trigonometry, etc., in which the use of negative numbers was
disallowed, although Descartes had used them imrestrictedly more than
a hundred years before.

* Strictly speakinpr, a more appropriate epithet would be, not
positive, but signless.

t In the strict sense.

:*: With the exception of division by the rational number 0.

§ Such a discussion, defining a rational number as an ordered
number-pair of iute ers in a similar manner to that in which a complex
number is defined in \hardsectionref{1}{3} as an ordere i number-pair of real numbers,
will be found in Hobsou's Functions of a Real Variable, §§ 1-12.

1-2

%
% 4
%

A rational number x may be represented to the eye in the following
manner :

If, on a straight line, we take an origin and a fixed segment OPi (Pi
being on the right of 0), we can measure from a length OFx such that
the ratio OPx/OPi is equal to x; the point P is taken on the right or
left of according as the number x is positive or negative. We may
regard either the point P or the displacement OP (which Avill be
written OPx) as repre- senting the number x.

All the rational numbers can thus be represented by points on the
line, but the converse is not true. For if we measure off on the line
a length OQ equal to the diagonal of a square of which OPi is one
side, it can be proved that Q does not correspond to any rational
number.

Points oil the line which do not represent rational numbers may be
said to represent irrational numbers; thus the jjoint Q is said to
rej resent the irrational number, /2 = l 414213.... But while such
an explanation of the existence of irrational numbers satisfied the
mathematicians of the eighteenth century and may still be sufficient
for those whose interest lies in the applications of mathematics
rather than in the logical upbuilding of the theory, yet from the
logical standpoint it is improper to introduce geometrical intuitions
to supply deficiencies in arithmetical arguments; aud it was shewn by
Dedekind in 1858 that the theory of irrational numbers can be
established on a purely arithmetical basis without any appeal to
geometry.

\Section{1}{2}{Dedekind's* theory of irrational numbers.}

The geometrical property of points on a line which suggested the
starting point of the arithmetical theory of irrationals was that, if
all points of a line are separated into two classes such that every
point of the first class is on the right of every point of the second
class, there exists one and only one point at which the line is thus
severed.

Following up this idea, Dedekind considered rules by which a
separationf or section of all rational numbers into two classes can be
made, these classes (which will be. called the Z-class and the
P-class, or the left class and the right class) being such that they
possess the following properties :

(i) At least one member of each class exists.

(ii) Every member of the X-class is less than every member of the
P-class.

It is obvious that such a section is made b ' an rational number x;
and X is either the greatest number of the Z-class or the least number
of the

* The theory, though elaborated in 1858, was not published before the
appearance of Dede- kind's tract, Stetigkeit und irrationale Zahlen,
Brunswick, 1872. Other theories are due to Weitrstrass [see von
Dantscher, Die Weierstrasa'sche Theorie der irrationulen Zahlen
(Leipzig, 1908)] and Cantor, Math. Ann. v. (1872), pp. 123-130.

t This procedure formed the basis of the treatment of irrational
numbers by the Greek mathematicians in the sixth aud fifth centuries
b.c. The advance made by Dedekind consisted in observing that a purely
arithmetical theory could be built ap on it. *

%
% 5
%

i?-class. But sections can be made in which no rational number x plays
this part. Thus, since there is no rational number* Avhose square is
2, it is easy to see that we may form a section in which the i?-c]ass
consists of the positive rational numbers whose squares exceed 2, and
the Z-class consists of all other rational numbers.

Then this section is such that the i?-class has no least member and
the Z-class has no greatest member; for, if x be any positive
rational fraction,

and 2 are in order of magnitude; and therefore given any member x of
the Zr-class, we can always find a greater member of the Z-class, or
given any member x of the 7?-class, we can always find a smaller
member of the /i- class, such numbers being, for instance, y and \ j,
where y' is the same function of x' as y of x.

If a section is made in which the i -chiss has a least member Jo, or
if the /> class has a greatest member J,, the section determines a
rational-real number, which it is convenient to denote by the samef
symbol Ao\ or A .

If a section is made, such that the i -class has no least member and
the Z-class has no greatest member, the section determines an
irrational-real number I.

If X, y are real numbers (defined by sections) we say that x is
greater than 3/ if the Z-class defining x contains at least two§
members of the i -class defining y.

Lot a,, ... be real numbers and let .4j, i?,. ... be any members of
the corresponding Z-classes while A., B... are any members of the
corresponding i -classes. The classes of which A, Ao, ... are
respectively members will be denoted by the symbols A ), A. ), ....

Then the sum (written a + ) of two real numbers a and /3 is defined as
the real number (rational or irrational) which is determined by the
Z-class A + B,) and the E-class A. + B.,).

It is, of course, necessary to prove that these classes determine a
section of the rational numbers. It is evident that A i + 5, < A, +
B., and that at least one member of each of the classes Ai + B ),
A.,-\-B.,) exists. It remains to prove that there is, at most, one
rational

* For if piq be such a number, this fraction being in its lowest
terms, it may be seen that (2'if-i')/(p-9) is another such number, and
0<p-q<q, so that pjq is not in its lowest terms. The contradiction
implies that such a rational number does not exist.

+ This causes no confusion in practice.

X B. A, W. Russell defines the class of real numbers as actualhj being
the class of all L-classes; the class of real numbers whose i classes
have a greatest member corresponds to the class of rational numbers,
and though the rational-real number x which corresponds to a rational
number .r is conceptually distinct from it, no confusion arises from
denoting both by the same symbol.

§ If the classes had only one member in common, that member might be
the greatest member of the JL-class of .r and the least member of the
i -class of y.

%
% 6
%

number which is greater than every Ai + B and less than every A., +
B.,; suppose, if possible, that tliere are two, .v and y (?/>.r). Let
cfj be a member of (.Ij) and let a-> be a member of (Ao); and let lY
be the integer next greater than (a2- i)/ i(y-. ) - Take the last of

the numbers ai-|- ((,,\ flj), (where m=0, 1, ... lY), which belongs to
(A ) and the first of

them which belongs to (Ao); let these two numbers be Cj, c,. Then

 2 -' i = Jr ao - i) < i (3/ - )- Choose c/i, do in a similar maimer
from the classes defining /3; then

C2 + cL - f 1 - c?i <y - A'. But C2 + d., i Ci+di x, and therefore C2
+ c?2-Ci-c/i > i/-.?-; we have therefore arrived at a contradiction by
supjjosing that two rational numbers a;, y exist belonging neither to
( Jj + B ) nor to ( 2 + 2)-

If every rational number belongs either to the class (J 1 + 1) or to
the class ( 2+ 2), then the classes (Jj + i), ( 2 + 2) define an
irrational number. If one rational number.)- exists belonging to
neither class, then the Z-class formed by x and (Jj+ j) and the
i2-class ( 2 + -52) define the rational-real number x. In either
ca.se, the number defined is called the sum a + /3.

The difference a-/3 of two real numbers is defined by the Z-class
(J1-Z2) and the Z'-class ( 2- 1).

The product of two positive real numbers a, /3 is defined by the 7
class A B-i) and the Z-class of all other rational numbers.

The reader will see without difficulty how to define the product of
negative real num- bers and the quotient of two real numbers; and
further, it may be shewn that real numbers may be combined in
accordance with the associative, distributive and commuta- tive laws.

The aggregate of rational-real and irrational-real numbers is called
the aggregate of real numbers; for brevity, rational-real numbers and
irrational- real numbers are called rational and irrational numbers
respectively.

\Section{1}{3}{Complex numbers.}

We have seen that a real number may be visualised as a displacement
along a definite straight line. If, however, P and Q are any two
points in a plane, the displacement PQ needs two real numbers for its
specification; for instance, the differences of the coordinates of P
and Q referred to fixed rectangular axes. Ifthe coordinates of P be (
, 77) and those oi Q( + cc,7 +i/), the displacement PQ may be
described by the symbol [x, y]. We are thus led to consider the
association of real numbers in ordered* pairs. The natural definition
of the sum of two displacements [x, y\ [x, y'] is the displacement
which is the result of the successive applications of the two
displacements; it is therefore convenient to define the sum of two
number-pairs by the equation

[.c, y] + [x', y'] = [x + x\ y -f y'].

The order of the two terms distiuguishes the ordered number-pair [.r,
ij] from the ordered number-pair [?,r].

%
% 7
%

The product of a number-pair and a real, number x is then naturally
defined by the equation

x X \ x, y\ = \ x'x, x'y\

We are at liberty to define the product of two number-pairs in any
convenient manner; but the only definition, which does not give rise
to results that are merely trivial, is that symbolised by the equation

[x, y] X [x, ij] = [xx' - i/t/', xy + xy\

It is then evident that

[x, 0] X [x, y ] = [xx, xy' = xx [x, y']

and [0, y] x [x, y] = [- yy', x'y] = y x [- y', x'].

The geometrical interpretation of these results is that the effect of
multiplying by the displacement [x, 0] is the same as that of
multiplying by the real number x; but the effect of multiplying a
displacement by [0, y] is to multiply it by a real number y and turn
it through a right angle.

It is convenient to denote the number-pair [x, y] by the compound
symbol x + iy; and a number-pair is now conveniently called (after
Gauss) a complex number; in the fundamental operations of Arithmetic,
the complex number x+ iO may be replaced by the real number x and,
defining i to mean + il, we have i- = [0, 1] x [0, 1] = [- 1, 0]; and
so r may be replaced by - 1.

The reader will easily convince himself that the definitions of
addition and multiplication of numbei--pairs have been so framed that
we may perform the ordinary operations of algebra with complex numbers
in exactly the same way as with real numburs, treating the symbol i as
a number and replacing the product it by - I wherever it occurs.

Thus he will verify that, if a, b, c are complex numbers, we have

a + b = b + a,

ab = b(i,

(a + b) +c = a +(b -\-c),

ab .c = a.bc,

a b -\- c)= ab + ac,

and if ab is zero, then either a or 6 is zero.

It is found that algebraical operations, direct or inverse, when
applied to complex numbers, do not suggest numbers of any fresh type;
the complex number will therefore for our purposes be taken as the
most general type of number.

The introduction of the complex number has led to many important
developments in mathematics. Functions which, when real variables only
are considered, appear as essentially distinct, are seen to be
connected when complex variables are introduced :

%
% 8
%

thus the circular functions are found to be expressible in terms of
exponential functions of a complex argument, by the equations

TODO

Again, many of the most important theorems of modern analysis are not
true if the numbers concerned are restricted to be real; thus, the
theorem that every algebraic equation of degree n has % roots is true
in general only when regarded as a theorem concerning complex numbers,

Hamilton's quaternions furnish an example of a still fiu'ther
extension of the idea of number. A quaternion

iv+xi+1/j + zk

is formed from four real numbers w, .i; y, z, and four number- units
1, i, j, l\ in the same way that the ordinary complex number x-\-iy
might be regarded as being formed from two real numbers x, y, and two
number-units 1, i. Quaternions however do not obey the commutative law
of multiplication.

\Section{1}{4}{The modulus of a complex number.}

Let X + iy be a complex number, x and y being real numbers. Then the
positive square root of x- 4- y- is called the modulus of x + iy), and
is

written

x + iy.

Let us consider the complex number which is the sum of two given
complex numbers, x + iy and u + iv. We have

TODO

The modulus of the sum of the two numbers is therefore

\ \ {X + Uf + ((/ + V) 2,

or [ x- + y-) + ur + V-) -+ 2 xu + ijv)] .

But

 \ x-\-iy\-\-\ u + iv\ ] -= [ x- + /y"') + (w- + v' yf

= ( 2 4- y ) + ( <2 +;2 + 2 x' + t/'f tC + V') '

= (x- + y-) + (u + tj2) + 2 (xu + yv)- -h (xv - yu)-], and this
latter expression is greater than (or at least equal to) x- + y' ) +
(u" + V-) + 2 (xu + yv). We have therefore

TODO

i.e. the moduhis of the sum of two complex numbers cannot be greater
than the sum of their moduli; and it follows by induction that the
modulus of the sum of any number of complex numbers cannot be greater
than the sum of their moduli.

%
% 9
%

Let us consider next the complex number which is the product of two
given complex numbers, x + iy and u + iv; we have

 pG 4- iy) u + iv) = xu - yv) + i xv + yu), and therefore

TODO

The modulus of the product of two complex numbers (and hence, by in-
duction, of any number of complex numbers) is therefore equal to the
product of their moduli.

\Section{1}{5}{The Argand diagram.}

We have seen that complex numbers may be represented in a geometrical
diagram by taking rectangular axes Ox, Oy in a plane. Then a point P
whose coordinates referred to these axes are x, y may be regarded as
representing the complex number x + iy. In this way, to every point of
the plane there corresponds some one complex number; and, conversely,
to every possible complex number there corresponds one, and only one,
point of the plane.

The complex number x + iy may be denoted by a single letter* z. The
point P is then called the representative point of the number z; we
shall also speak of the number z as being the ajfx of the point P.

If we denote (jf + /) by r and choose 6 so that r cos 6 = x, r sin 9
=y, then r and 6 are clearly the radius vector and vectorial angle of
the point P, referred to the origin and axis Ox.

The representation of complex numbers thus afforded is often called
the Argand diagram .

By the definition already given, it is evident that r is the modulus
of z. The angle 6 is called the argument, or amplitude, or phase, of
z.

We write 6 - arg z.

From geometrical considerations, it appears that (although the modulus
of a complex number is unique) the argument is not unique;; if be a
value of the argument, the other values of the argument are comprised
in the expression 2mr + d where a is any integer, not zero. The
principal value of arg 2 is that which satisfies the inequality TODO.

* It is convenient to call .r and y the real and imaginary parts of z
respectively. We fre- quently write x = R(~), y = I(z).

t .J. E. Argand published it in 1806; it had however previously been
used by Gauss, and by Caspar Wessel, who discussed it in a memoir
presented to the Danish Academy in 1797 and published by that Society
in 1798-9.

X See the Appendix, § A-521.

%
% 10
%

If P, and P2 are the representative points corresponding to values z-
and Z.2 respectively of z, then the point which represents the value
z- + z. is clearly the terminus of a line drawn from P, equal and
parallel to that which joins the origin to P. 2.

To find the point which represents the complex number z Zo, where z
and Zo are two given complex numbers, we notice that if

z = 1 (cos 6 + i sin 6 ),

z.2, = ?-2 (cos f 2 + * sin o) then, by multiplication,

Z\ Z% = n 2 [cos ((9i + 2) + sin ( j + 6. ]. The point which
represents the number z z has therefore a radius vector measured by
the product of the radii vectores of Pj and P2, and a vectorial angle
equal to the sum of the vectorial angles of Pj and Pg.

REFERENCES.

Tke gical foundations of the theory of number. u . X. Whitehead axd B.
A. W. Russell, Principia Mathematica (1910-1913). 'B. A. W. Russell,
Introduction to Mathematical Philosophy (1919).

On irrational numbers.

R. Dedekixd, Stetigkeit unci irrationale Zahlen. (Brunswick, 1872.)
''V. VON Dantscher, Vorlesungen ueber die Weierstrass'sehe Theorie der
irrationalen

Zahlen. (Leipzig, 1908.) E .W. HoBSON, Functions of a Real Variable
(1907), Ch. i. T.'*?. I'a. Bromwich, Theory of Infinite Series (1908),
Appendix i.

On omplex numbers.

H. Hankel, Theorie der complexen Zahlen-systeme. (Leipzig, 1867.) O.
Stolz, Yorlesungen iiber allgemeine Arithmetik, II. (Leipzig, 1886.)
G. H. Hardy, A course of Pure Mathematics (1914), Ch. in.

Miscellaneous Examples.

1. Shew that the representative points of the complex numbers 1 + 4 2
4-7?', 34-10 are coHinear.

2. Shew that a paralwla can be drawn to pass through the
representative points of the complex numbers

2 + V, 44-4t, 6 + 9 8+ 16 10 + 2.5i.

3. Determine the nth. roots of unity by aid of the Argand diagram;
and shew that the number of primitive roots (roots the powers of each
of which give all the roots) is the number of integers (including
unity) less than n and prime to it.

Prove that if j, 2? 3,  he the arguments of the primitive roots, 2
cosjo =0 when

p m \& positive integer less than -7 ., where a, b, c, ... k are the
difterent constituent

n ( - 'f n

primes of 71; and that, when p-, - -?, 2 cos p6=,-,, where /x is
the number of

(toe, m iC ccoc,,, fc

the constituent primes. \addexamplecitation{Math. Trip. 1895}

\chapter{The Theory of Convergence} 

\Section{2}{1}{The definition* of the limit of a sequence.}

Let Zi, Zo, z-i, ... be an unending sequence of numbers, real or
complex. Then, if a number I exists such that, corresponding to every
positive f number e, no matter how small, a number ??o can be found,
such that

\ Zn-l'.< e

for all values of n greater than o. the sequence (z,,) is said to tend
to the limit I as n tends to infinity.

Symbolic forms of the statement;]: ' the limit of the sequence Zn), as
n tends to infinity, is / ' are :

lim Zn=l, \ unzn=l, 2,1- * I " -> 

If the sequence be such that, given an arbitrary number N (no matter
how large), we can find 7?o such that |,i | > iV for all values of n
greater than /lo, we say that '\ Zn\ tends to infinity as n tends to
infinity,' and we write

kn i -> 

In the corresponding case when -x,i>N when n> n we say that i/;,j - *
- oc,

If a sequence of real numbers does not tend to a limit or to co or to
- x, the sequence is said to oscillate.

211. Definition of the phrase ' of the order of

If ( ) and zn) are two sequences such that a number n exists such
that i (Ku/ n) I < jK" whenever n > n, where K is independent of n,
we say that n is ' of the order of Zn, and we write§

thus 15!i±l = 0fl

1 + n" \ n-

If lim( / ) = 0, we write n = o Zn).

* A definition equivalent to this was first given by John Wallis in
1655. [Opera, i. (1695), p. 382.]

t The number zero is excluded from the class of positive numbers.

J The arrow notation is due to Leathern, Camb. Math. Tracts, No. 1.

§ This notation is due to Bachmann, Zahlenthcorie (1894), p. 401, and
Landau, Primzahlen, I. (1909), p. 61.

%
% 12
%

\Section{2}{2}{The limit of an increasing sequence.}

Let (a7) be a sequence of real numbers such that Xn+i' Xn for all
values of w; then the sequence tends to a limit or else tends to
infinity (and so it does not oscillate).

Let X be any rational-real number; then either :

(i) Xn X for all values of n greater than some number /?o depending on
the value of x.

Or (ii) Xn < X for every value of n.

If (ii) is not the case for any value of x (no matter how large), then

Xn- OO.

But if values of x exist for which (ii) holds, we can divide the
rational numbers into two classes, the Z-class consisting of those
rational numbers x for which (i) holds and the i -class of those
rational numbers x for which (ii) holds. This section defines a real
number a, rational or irrational.

And if € be an arbitrary positive number, a- e belongs to the i-class
which defines a, and so we can find n such that Xn>oi - \ e whenever n
> n; and a + e is a member of the i?-class and so Xn<a + e.
Therefore, whenever n > n,

TODO

Therefore Xn -> a.

Corollary. A decreasing sequence tends to a limit or to - oo . Example
1. If lim4:, =, lims,' = r, then lim(, + 3,') = + '. For, given
e, we can find n and ' such that

(i) when TODO, (ii) when TODO. Let TODO be the greater of n and n';
then, when TODO

and this is the condition that Urn z,n + Sm')='i + i'-

Example 2. Prove simikrly that lim(s,- 2,,/) =;-/", l[m z,z,') =
ll', and, if /' + 0,

lim zjz,') = l/l'.

Example 3. If < x < 1, x" 0. For if A- = (l+a)-i, a > and

TODO

by the binomial theorem for a positive integral index. And it is
obvious that, given a positive number f, we can choose no such that
TODO

2'21. Limit-points and the Bolzano- Weierstrass* theorem.

Let (xn) be a sequence of real numbers. If any number G exists such

* This theorem, frequently ascribed to Weierstrass, was proved by
Bolzano, Ahh. der k. bohmischen Ges. der Wiss. v. (1817). [Eeprinted
in Klassiker der E.mkten Wiss., No. 153.] It seems to have been known
to Cauchy.

1-

%
% 13
%

that, for every positive value of e, no matter how small, an unlimited
number of terms of the sequence can be found such that

G - 6 < Xn < G + e,

then G is called a limit-point, or cluster-point, of the sequence.

Bolzano's theorem is that, if X '<cp, ivhere X, p are independent of
$n$, then the sequence TODO Juts at least one limit-point.

To prove the theorem, choose a section in which (i) the i -class
consists of all the rational numbers which are such that, if A be any
one of them, there are only a limited number of terms Xn satisfying
Xn>A; and (ii) the Z-class is such that there are an unlimited number
of terms a-',i such that x a for all members a of the Z-class.

This section defines a real number G; and, if e be an arbitrary
positive number, G - e and G + e are members of the L and R classes
respectively, and so there are an unlimited number of terms of the
sequence satisfying

G- e< G - he Xn G + €<G- - e, and so G satisfies the condition that it
should be a limit-point.

2'211. Definition of 'the greatest of tlie limits.'

The number G obtained in\hardsubsectionref{2}{2}{1} is called ' the greatest of the limits
of the sequence xn)' The sequence x ) cannot have a limit-point
greater than G] for if G' were such a limit-point, and e = i ((r' -
(r), G' - e is a member of the jR-class defining G, so that there are
only a limited number of terms of the sequence which satisfy Xn>G' -
€. This condition is incon- sistent with G' being a limit-point. We
write

G= \ imxn. The ' least of the limits,' L, of the sequence (written lim
x ) is defined to be

- lim (- Xn).

2-22. CaUCHY's* theorem on the necessary and sufficient CON- DITION
FOR THE existence OF A LIMIT.

We shall now shew that the necessary and sufficient condition for the
existence of a limiting value of a sequence of numbers z, Zn, z-i,
... is that, corresponding to any given positive number e, hoiuever
small, it shall be possible to find a number n such that

for all positive integral values of j)- This result is one of the most
important and fundamental theorems of analysis. It is sometimes called
the Principle of Convergence.

* Analyse Algebrique (1821), p. 125.

%
% 14
%

First, we have to shew that this condition is necessary, i.e. that it
is satisfied whenever a limit exists. Suppose then that a limit I
exists; then \hardsectionref{2}{1}) corresponding to any positive number e, however
small, an integer n can be chosen such that

for all positive values of p; therefore

  i Zn+p -l\ + \ Zn-l\ <,

which shews the necessity of the condition

I ii+p ~ -2'rt I < > and thus establishes the first half of the
theorem.

Secondly, we have to prove* that this condition is suficient, i.e.
that if it is satisfied, then a limit exists.

(I) Suppose that the sequence of real numbers xn) satisfies Cauchy's
condition; that is to say that, corresponding to any positive number
e, an integer n can be chosen such that

for all positive integral values of p.

Let the value of n, corresponding to the value 1 of e, be 7n.

Let Xj, pi be the least and greatest of a;i, a-g, ... av; then

Xi-1 <Xn< p, + l,

for all values of n; write Xj - I = X, pi + 1 = p.

Then, for all values of n, X < Xn < p- IVierefore by the theorem of §
2'21, the sequence (xn) has at least one liniit-point G.

Further, there cannot be more than one limit-point; for if there were
two, G and H (H < G), take e < I G - H). Then, by hypothesis, a number
n exists such that j Xn+p - Xn | < e for every positive value of p.
But since G and H are limit-points, positive numbers q and r exist
such that

I G - X, +q \ < €, \ H - Xn+r \ < l-nen | Cr X- q | -j- | Xn q X | -|-
| X n+r I ~r I Xji f. Ji ] < '±6.

But, by\hardsectionref{1}{4}, the sum on the left is gi-eater than or equal to j G -
H .

Therefore G - H < 4e, which is contrary to hypothesis; so there is
only one limit- point. Hence there are only a finite number of terms
of the sequence outside the interval G -, (j + S), where 8 is an
arbitrary positive number;

* This proof is given by Stolz and Gmeiner, Theoretische Arithmetik,
ii. (191)2), p. 144.

%
% 15
%

for, if there were an unlimited number of such terms, these would have
a limit-point which would be a limit-point of the given sequence and
which would not coincide with G; and therefore G is the limit of x ).

(II) Now let the sequence Zn) of real or complex numbers satisfy
Cauchy's condition; and let Zn = Xn -f- iyn, where Xn and yn are real
; then for all values of n and p

I n-irp - i I n+p 2'n i, | yn+p ~ yn \ | 2'n+p n\-

Therefore the sequences of real numbers x ) and (yn) satisfy Cauchy's
condition; and so, by (I), the limits of (.r) and y ) exist.
Therefore, by\hardsectionref{2}{2} example 1, the limit of (2) exists. The result is
therefore established.

\Section{2}{3}{Convergence of an infinite series.}

Let Wj, ?/2, u.i, ... Kit, ... be a sequence of numbers, real or
complex. Let the sum

Ml 4- Uo + . . . -I- tin

be denoted by Sn-

Then, if *S,i tends to a limit S as /; tends to infinity, the
infinite series

"i + Hi + 3 -f- <4 + . . . is said to he convergent, or to converge to
the sum S. In other cases, the infinite series is said to be
divergent. When the series converges, the expression S-Sn, which is
the sum of the series

" +l+ " +2+ Un+-,+ ...,

is called the remainder after n terms, and is frequently denoted by
the symbol R .

The sum Un+ + 11 +.. + ...-\- Un+p

will be denoted by Sn,p.

It follows at once, by combining the above definition with the results
of the last paragraph, that the necessary and sufficient condition for
the convergence of an infinite series is that, given an arbitrary
positive number e, we can find n such that I /S'\ | < e for every
positive value of p.

Since Un+i = n,l, it follows as a particular case that lim Un+i = - in
other words, the 7?th term of a convergent series must tend to zero as
// tends to infinity. But this last condition, though necessary, is
not sufficient in itself to ensure the convergence of the series, as
appears from a study of the series

In this series, Sn,n =, +, + - +  + . 

' n+ i n + 2 n + 6 2n

The expression on the right is diminished by writing (2?i)~ in place
of each term, and so Sn,,1 > 

%
% 16
%

Therefore S n+i = 1 + /Si, i + 2, 2 + 'S14, 4 + >S'8\ g + >Si6, le + 
  + S.n 

> ( + 3) -> X;

so the series is divergent; this result was noticed by Leibniz in
1673.

There are two general classes of problems which we are called upon to
investigate in connexion with the convergence of series :

(i) We may arrive at a series by some formal process, e.g. that of
solving a linear differential equation by a series, and then to
justify the process it will usually have to be proved that the series
thus formally ob-' tained is convergent. Simple conditions for
establishing convergence in such circumstances are obtained in §§
2'31-2"61.

(ii) Given an expression S, it may be possible to obtain a development

S= X i(,n + Rn, valid for all values of n; and, from the definition
of a limit,

00

it follows that, if we can prove that Rn - 0, then the series 2 u,,
converges

m = l

and its sum is S. An example of this problem occurs in § 54.

Infinite series were used* by Lord Brouncker in Phil. Trans. 11.
(1668), pp. 645-649, and the expressions convergent and disergent were
introduced by James Gregory, Professor of Mathematics at Edinburgh, in
the same year. Infinite series were used systematically by Newton in
1669, De anali/si per aequat. num. term, inf., and he investigated the
con- vergence of hypergeometric series (§ 14- 1) in 1704. But the
great mathematicians of the eighteenth century used infinite series
freely without, for the most part, considering the question of their
convergence. Thus Euler gave the sum of the series

1 1 1,, .. ... + -,+ - + -+l+2 + 2- + r'+ a)

as zero, on the ground that

2 +,2 +,3 + ... (6)

\-z

1, 1 1 3

and 1+- -t- -, + ... = - - (c).

z z~ z- V

The eiTor of course arises from the fact that the series h) converges
only when | 2 j < 1, and the series (c) converges only when | s j > 1,
so the series (a) never converges.

For the history of researches on convergence, see Pringsheim and Molk,
Encyclope'die des Sci. Math., i. (1) and Keifi", Geschichte der
unendlichen Reihen (Tiibingen, 1889).

2*301. AheVs inequality.

I "* Let fn fn+ for all integer values of n. Then 2 /

A is the greatest of the sums

1 ! i, i ! + 2 1 > I o-i + 0.2 + a, !, . . ., i ! + rto +   . + am

  Af, ivhere

* See also the note to\hardsectionref{2}{7}.

t Journal fiir Math. i. (1826), pp. 311-339. A particular case of the
theorem of\hardsubsectionref{2}{3}{1}, Corollary (i), also appears iu that memoir.

%
% 17
%

For, writing Ui + ao + . . . + a = Sn, we have

rii

t ilnfn = 5i/i + S.2 - Si)fo + (Ss - S jfs + . + (s - Sm-i)fm 11 = 1

- Si (/i -J 2) + S-2. (/2 ~y 3) + . . . + S i\ i \ Jm-i ~Jm) + Smfm-

Since /i - /o,/2 -fs,  are not negative, we have, when n = 2, 3,
... m,

I Sn-i \ (fn-i -fn) (fn-i - fn) ', alsO j S,n \ fm < fm,

and so, summing and using\hardsectionref{1}{4}, we get

n=l I

Corollari). If i, a ... ?Pi, Wj, ... are any numbers, real or complex,

2 a w \ <,A\ 2 i M' + 1 - w, i + 1 u-

'/'

where J is the greatest of the sums

2 a

n=l

, p = \, 2, ... m).

\addexamplecitation{Hardy.}

2"31. DiHchlet' s* test for convergence.

Let

Z < K, luhere K is independent of p. Then, if fn >fn\-\ >

and lim/ = O-f*, ((nfn converges.

M = l

For, since lim i = 0, given an arbitrary positive number e, we can
find m such that +! < e/2/i .

Then

m + q

m + q

2 an\ i X a j + ! X cirt t < 2/1', for all positive values of; so

that, by Abel's inequality, we have, for all positive values of j;;,

where A <2K.

I m+p

Therefore 2 cinfn

I n = m+l

m+p Z CLnJn - Jin+i >

n=m+l

< 2Kfm+i < e; and so, by\hardsectionref{2}{3}, S cinf 71 converges.

Corollary (i). Jfte 's test for convergence. If 2 converges and the
sequence (m ) is

n = l

nionotouic (i.e.,i ?t + i always or else M Wrt + i always) and j?(
|<k, where k is independent of /(, then 2 i/ converges.

For, by\hardsectionref{2}{2}, tends to a limit u; let |w- |=/ . Then i- 0 steadily;
and

therefore 2 ' converges. But, if (m ) is an increasing sequence, /
=w-m, and so

rt=i

2 li-n Un converges; therefore since 2 ua converges, 2 converges. If
(?<J is

 =1 ii=\ n=\

a decreasing sequence / = e<,i -, and a similar proof holds.

* Journal de Math. (2), vii. (1862), pp. 253-255. Before the
publication of the 2nd edition of Jordan's Cours d' Analyse (1893),
Dirichlet's test and Abel's test were frequently jointly described as
the Dirichlet-Abel test, see e.g. Pringsheim, Math. Ann. xxv. (1885),
p. 423.

t In these circumstances, we say j -0 steadily.

W. M. A. 2

%
% 18
%

Corollary (ii). Taking a = (-) -i in Dirichlet's test, it follows
that, if / /; i and lim / = 0, /i -ft +/3 -/i + . .  converges.

p

Example 1. Shew that if 0< <27r, I 2 sin (9 <coseci(9; and deduce
that, if

f - Q steadily, 2 / sin nO converges for all real values of 6, and
that 2 / cos nd converges

n=l "=i

if 6 is not an even multiple of tt.

Example 2. Shew that, if fn- 0 steadily, 2 (-)"/' cos (9 converges if
6 is real and

n = \

not an odd multiple of tt and 2 -)"-fnS\ nne converges for all real
values of 6. [Write 7r + for in example 1.]

2'32. Absolute and conditional convergence.

QO

In order that a series X Un of real or complex terms may converge, it
is

n=l

sufficient (but not necessary) that the series of moduli S Un \ should

n = \

00

converge. For, if <Tn,p = Un+i, + | Un+2 i +    + | Un+p \ and if
2 | w | converges,

71 = 1

\ ve can find n, corresponding to a given number e, such that cTn p <
e for all

values of jj. But ] Sn,p cr \ < e, and so S converges.

  = i

we see that t - 9 + o~4+--- converges, though \hardsectionref{2}{3}) the series of
moduli

The condition is not necessary; for, writing i = 1/n in\hardsubsectionref{2}{3}{1},
corollary (ii), see that

i\ -4- + +...is known to diverge.

1 2 O -i

In this case, therefore, the divergence of the series of moduli does
not entail the divergence of the series itself.

Series, which are such that the series formed by the moduli of their
terms are convergent, possess special properties of great importance,
and are called absolutely convergent series. Series which though
convergent are not abso- lutely convergent (i.e. the series themselves
converge, but the series of moduli diverge) are said to be
conditionally convergent.

" 1 2"33. The geometric series, and the series 2 - .

w = l

The convergence of a particular series is in most cases investigated,
not by the direct consideration of the sum Sn p, but (as will appear
from the following articles) by a. comparison of the given series with
some other series which is known to be convergent or divergent. We
shall now investigate the convergence of two of the series which are
most frequently used as standards for comparison.

%
% 19
%

(I) The geometric series. The geometric series is defined to be the
series z + z-- z + z'+ .... Consider the series of moduli

l+\ z\ + \ z ' + \ z' + ...\ for this series Sn,p =\ z\'' ' + \ z, "+-
+ ...->r z v

l-\ z\ P

= izr+

l-\ z\ '

Hence, if .g- < 1, then S,i,p< : -; for all values of j), and, by §
22,

example 3, given any positive number e, we can find n such that

[ |n+i l\ |2i|-i<e.

Thus, given e, we can find n such that, for all values of p, Sa,p<€.
Hence, by\hardsubsectionref{2}{2}{2}, the series

is convergent so long as 2 | < 1, and therefore the geometric series
is absolutely convergent if\ z\ < .

When z ' 1, the terms of the geometric series do not tend to zero as n
tends to infinity, and the series is therefore divergent.

TT VV .11111

(II) Ihe series + - + - + - + - + ....

" 1 Consider now the series,i = 2 -, where s is greater than 1.

7n = l i*

112 1 'e have 2" + 3 - < 2*- = 2 i '

11114 1

- I 1 1 - < - =

4 5*' 6' 7* 4 4*-i ' and so on. Thus the sum of 2 -1 terms of the
series is less than 1 J\ J\ J\ 1 1

] s-i 2*~i 4*~i 8 ~  I 2(2>-i) (s-1) 1 - 2 ~* ' and so the sum of
ani/ number of terms is less than (1 - 2 ~*)~ Therefore

n

the increasing sequence S m~ cannot tend to infinity; therefore, by §
2'2,

w = l

=0 1 .

the series S - is convergent if s>\ \ and since its terms are all real
and

M = 1 'i

positive, they are equal to their own moduli, and so the series of
moduli of the terms is convergent; that is, the convergence is
absolute.

2-2

%
% 20
%

If s = 1, the series becomes

1 + 1 + 1 + 1 + ...,

which we have already shewn to be divergent; and when 5 < 1, it is a
fortiori divergent, since the effect of diminishing s is to increase
the terms of the

< 1 .

series. The series S - is therefore divergent if s 1. n = l ''

2'34. The Comparison Theorem.

We shall now shew that a series; i + i, + 2/3+ ... is absolutely
con- vergent, provided that \ u \ is always less than G \ vn\, ivhere
C is some number independent of n, and v,i is the nth term of another
series which is known to be absolutely convergent.

For, under these conditions, we have

I ' ?i+i I + I 'i' i+2 1 +  .  + I Un p I < C I j Vn+i I + I f,iJ-2
: + ... + 1 V +p \ \,

where n and p are any integers. But since the series Si',i is
absolutely convergent, the series S | Vn \ is convergent, and so,
given e, we can find n such that

I " n+i I + i /i+2 I + .-..+ I Vn- p I < e/ C,

for all values oi p. It follows therefore that we can find /; such
that

1 Un+i I + I Wn+2 i + . . + 1 Un+p \ < e,

for all values of p, i.e. the series S | Un \ is convergent. The
series %Un is therefore absolutely convergent.

Corollary. A series is absolutely convergent if the ratio of its th
term to the nih. term of a series which is known to be absolutely
convergent is less than some number indej)endent of n.

Example 1. Shew that the series

COS,Z +- 2COS22 + .-J5COS 32 + -T, COS42+...

li" o" 4"

is absolutely convergent for all real values of z.

iCOS TliZ 1

- 5- -, . The moduli of n I ji-

the terms of the given series are therefore less than, or at most
equal to, the corresponding

terms of the series

n 1 1 1

1 + 2-2 + .3- + 4 2+-' .

which by\hardsubsectionref{2}{3}{3} is absolutely convergent. The given series is
therefore absolutely convergeut.

Example 2. Shew that the series
$$
TODO
$$
where 2 = e'", (?i=l, 2, 3,

is convergent for all values of 2, which are not on the circle [ ] =
1.

%
% 21
%

The geometric representation of complex numbers is helpful in
discussing a question of this kind. Let values of the complex number z
be represented on a plane; then the numbers Z\ t z, Zz, ... will
give a sequence of points which lie on the circumference of the circle
whose centre is the origin and whose radius is unity; and it can be
shewn that every point on the circle is a limit-point \hardsubsectionref{2}{2}{1}) of the
points z . --

For these special values z-, of, the given series does not exist,
since the denomi- nator of the nth term vanishes when 2 =,,. For
simplicity we do not discuss the series for any point z situated on
the circumference of the circle of radius unity.

Suppose now that [zj + l. Then for all values of 7i, | z - 2 (1 - [2 '
|>c~i, for

some value of c; so the moduli of the terms of the given series are
less than the corre- sponding terms of the series

c c c c

which is known to be ab.solutely convergent. The given series is
therefore absolutely convergent for all values of z, except those
which are on the circle | 2 | = 1.

It is interesting to notice that the area in the i-plane over which
the series converges is divided into two parts, between which there is
no intercommunication, by the circle

1 1 = 1.

Example 3. Shew that the series

SsinJ-f 4sin |-f8 sin -f- ... -l-2"sin .f- + ...

 J .J it i o

converges absolutely for all values of 2.

Since* lim 3" sin (2/8") = 2, we can find a number /, independent of
n (but depending on 2), such that | 3" sin (2/3") \ < i:; and
therefore

2 sin < U) .

3" I V3.

  /2\" Since 2 i' i ) converges, the given series converges
absolutely.

235. Gauchy's test for absolute convergence']'.

i/' lim M ]"" < 1, S ? i converges absolutely. >i -*. * jj = 1

For Ave can find m such that, when n ni, j m | '" p < 1, where p is
independent of u. Then, when n > ni, '. Un \ < p' ] and since 2 p"
converges,

n = >n+l

it follows from\hardsubsectionref{2}{3}{4}! that 2 Un (and therefore 2 m,J converges ab-
solute I '.

[Note. If lim \ u, \ \ ' >\, u does not tend to zero, and, by\hardsectionref{2}{3}, 2
m does not converge.]

* This is evident from results proved iu the Appendix. t Analyse
Algehrique, pp. 132-135.

%
% 22
%

2-36. D'Alemhert's* ratio test for absolute convergence.

We shall now shew that a series

?/i+ U2+ u-i + 1/4+ ... is absolutely convergent, provided that for
all values of n greater than some fixed value r, the ratio ' is less
than p, tvhere p is a positive number independent of n and less than
unity.

For the terms of the series

j Ur+i i + i Ur+-2 I + i Ur+i \ + ...

are respectively less than the corresponding terms of the series

which is absolutely convergent when p < 1; therefore S Un (and hence

  = / + !

the given series) is absolutely convergent.

A particular case of this theorem is that if lim (Un+Jun) \ = l <1,
the

series is absolutely convergent.

For, by the definition of a limit, we can find r such that

I ! \ 4 i < i (1 - 0, when n > r,

and then P±-' <l(l+l)<l,

when 11 > r.

[Note. If lim \ u, + l i<,, >1, % does not tend to zero, and, by §
2-3, 2 ?<,, does not

H = l

converge.]

Example 1. If 1 c |<1, shew that the series

/t=i converges absolutely for all values of z.

[For w + i/?i,i = c( + i)--''-e c-" + ie - 0, as ??- x, if;Cj<l.]

Example 2. Shew that the senes

a-6, (a- 6) (a- 26) a-h) a-2h)ia- h) . Z+-2J-.-+ 3j + j z +...

converges absolutely if \ z\ < \ b~' .

[For ±1 = - '1-z- -hz, as k- oo; so the condition for absolute
convergence is Un n + \ ' '

\ hz\ < \,. Q.\ z\ < h- .'\

* Opuscules, t. V. (1768), pp. 171-182.

%
% 23
%

Example 3. Shew that the series 2 - - \, converges absolutely if
i2|<l.

[For, when;2|<1, | 2 -(l 4-n-i)" I (1 + -!)"- I 2" i 1 + 1 + + ... -
1>1, so the moduli of the terms of the series are less than the
corresponding terms of the series 2 n Is""! I; but this latter series
is absolutely convergent, and so the given series con- verges
absolutely.]

2'37. A general theorem on series for luhicli lim |- =1.

It is obvious that if, for all values of n greater than some fixed
value r, I i/-,i+i I is greater than | Un\, then the terms of the
series do not tend to zero as

?i - > X, and the series is therefore divergent. On the other hand,
if i j

is less than some number which is itself less than unity and
independent of n (when n > r), we have shewn in\hardsubsectionref{2}{3}{6} that the series
is absolutely con- vergent. The critical case is that in which, as n
increases, - ' tends to the value unity. In this case a further
investigation is necessary.

We shall now shew that* a series u + u.. + u.i+ .. .,dn which lim -" =
1 will be absolutely convergent if a positive number c exists such
that

For, compare the series S ] Un j with the convergent series Ivn, where
and y1 is a constant; we have

Vn \ n + 1/ V nj n \ n

. \'Vn+i T I 1

and hence we can find m such that, when n > m,

u,

By a suitable choice of the constant A, we can therefore secure that
for all values of n we shall have

i Un I < V,v

As Si'n is convergent, 2 | Un j is also convergent, and so Sm,i is
absolutely convergent.

* This is the second (D'Alembert's theorem given in\hardsubsectionref{2}{3}{6} being the
first) of a hierarchy of theorems due to De Morgan. See Chrystal,
Algebra, Ch. xxvi. for an historical account of these theorems.

24

THE PROCESSES OF ANALYSIS

[chap. II

Corollary. If then the series is absolutely convergent if J.i < - 1

= 1 H - + - ], where i is independent of n,

n \ n-J

00 / " 1 \

Example. Investigate the convergence of 2 '' exp ( - -2 - ), when r>k
and when n=l \ 1 "v

r<k.

2'38. Convergence of the hijpergeometric series.

The theorems which have been given may be illustrated by a discussion
of the convergence of the hypergeometric series

  a.h a(a+l)6(6+l) . a(a+ l)(a + 2) 6(6 + 1)(6 + 2)

  + 17 + 1.2.c(c + l) " 1.2.3.c(c + l)(c + 2) " '

which is generally denoted (see Chapter XIV) by F a, b; c; z).

If c is a negative integer, all the terms after the (1 - c)th have
zero

denominators; and if either a or 6 is a negative integer the series
will

terminate at the (1 - a)th or (1 - 6)th term as the case may be. We
shall

suppose these cases set aside, so that a, h, and c are assumed not to
be

negative integers.

  In this series

i Un+i i ( "+ n - l)(b + n - 1)1 II

, - 7 -, V <s I * >

\ Ufi \ I n c + n - l) I

as ?? - > 00 .

We see therefore, by\hardsubsectionref{2}{3}{6}, that the series is absolutely convergent
ivhen \ z\ < l, and divergent ivhen | | > 1.

When U I = 1, we have *

1 +

0-1]

ii+'

n

a + b-

-c-1

6-1

n \ n /

Let a, b, c be complex numbers, and let them be given in terms of
their real and imaginary parts by the equations

a = a + ia", 6 = 6' + ib", c = c + ic". Then we have

 ' + 6' - c' - 1 + z (a" + 6" - c")

Un

1 +

= 1 +

a +b' - c

ly fa" + b"-c'

+ 0(1

\ n'

+

  

By\hardsubsectionref{2}{3}{7}, Corollary, a condition for absolute convergence is

a' + b' -c' < 0. * The symbol (l/>i-) does not denote the same
function of n throughout. See\hardsubsectionref{2}{1}{1}.

%
% 25
%

Hence ivhen \ z\ = \, a sufficient condition* for the absolute
convergence of the hyper geometric series is that the real part of a +
b - c shall be negative.

\Section{2}{4}{Effect of changing the order of the terms in a series.}

In an ordinary sum the order of the terms is of no importance, for it
can be varied without affecting the result of the addition. In an
infinite series, however, this is no longer the casef, as will appear
from the following example.

T.,11111111

Let 2 = H-3-2+5 + 7-| + 9 + n-o+---

1 cr -. 1 1 1 1 1 .

and,S'=l-2 + 3-4 + 5-6 + --->

and let 2,i and Sn denote the sums of their first n terms. These
infinite series are formed of the same terms, but the order of the
terms is different, and so - and Sn are quite distinct functions of n.

Let

11 1 I . a 0-w = j; + 2 +   . + > SO that bn = (T-.n - a

Then

 11 111 1

-3n-i +3 +  + 4,i\ i 2~4  2),

1 1 n 2 -'* 2 """

= (0"4/i - CT-jn) + 2 iti ~ n)

= n "r 9 211 

Making n->-y:, we see that

S = S + lS;

and so the derangement of the terms of /i' has altered its sum.

Example. If in the series

-. 1 1 1

1-2+3-4+---

the order of the terms be altered, so that the ratio of the number of
positive terms to the number of negative terms in the first n terms is
ultimately a-, shew that the sum of the series will become log (2a).
\addexamplecitation{Manning.}

2"41. The fundamental projierty of absolutely convergent series. We
shall shew that the sum of an absolutely convergent series is not
affected by changing the order in which the terms occur.

Let 8 = uy Un + 3 + ...

* The coudition is also necessary. See Bromwicb, Infinite Stnies, pp.
202-204.

t We say that the series S !' consists of the terms of S m,j in a
different order if a law

)! = 1 )t = l

is given by which corresponding to each positive integer x) we can
find one (and only one) integer q and vice versa, and Vq is taken
equal to Up. The result of this section was noticed by Dirichlet,
Berliyier Abh. (1837), p. 48, Journal de Math. iv. (1839), p. 397. See
also Cauchy, Resumes analytiques (Turin, 1833), p. 57.'

%
% 26
%

be an absolutely convergent series, and let S' be a series formed by
the same terms in a different order.

Let e be an arbitrary positive number, and let n be chosen so that

I I I I I I '

for all values of p.

Suppose that in order to obtain the first n terms of S we have to take
m terms of S'; then if k > m,

 k = n + terms of S with suffices greater than n, so that

>SV - S = Sn - S + terms of S with suffices greater than n.

Now the modulus of the sum of any number of terms of S with suffices
greater than n does not exceed the sum of their moduli, and therefore
is less

than 2 e-

Therefore | >S a;' - 'S' j < Sn- S\ + €.

But j - >Sf I < lim 11 Un+i I + I Un+-2 I +    + i Uri+p |

p-*-x

1 Therefore given e we can find m such that

\ Si:-s\ < €

when k > m; therefore /S,,/- > S, which is the required result.

If a series of real terms converges, but not absolutely, and if >S be
the sum of the first p positive terms, and if an be the sum of the
first n negative terms, then 8p->cc, cr - >- oo; and lim (>Sp + cr,j)
does not exist unless we are given some relation between p and n. It
has, in fact, been shewn by Riemann that it is possible, by choosing a
suitable relation, to make lim(>Sf 4- (7n) equal to cnii/ given real
number*.

\Section{2}{5}{Double series.}

Let ii i,n be a number determinate for all positive integral values of
m and ??; consider the array

 ],l) Wj O) '**I,3j  ' ' 2,1 J '2,2) *'2,3)   ';i, 1 ) h, 2 ) 3,
3 )   

* Ges. Werke, p. 221.

t A complete theory of double series, on which this account is based,
is given by Pringsh\&iui, Miinchcner Sitzunysberichte, xxvii. (1807),
pp. 101-152. See further memoirs by that writer, Math. Ann. liii.
(1900), pp. 289-321 and by London, ibid. pp. 322-370, and also
Bromwich, Infinite Series, which, m addition to an account of
Pringsheim's theory, contains many develop- ments of the subject.
Other important theorems are given by Bromwich, Proc. London Math.
Sac. (2), I. (1904), pp. 176-201.

%
% 27
%

Let the sum of the terms inside the rectangle, formed by the first m
rows of the first n columns of this array of terms, be denoted by
S,n,n-

If a number *S' exists such that, given any arbitrary positive number
e, it is possible to find integers m and n such that

whenever both /a > ru and v > n, we say* that the double series of
luliich the general element is u,,. converges to the sum S, and we
write

lira *S' = S.

 fi,V

If the double series, of which the general element is ] w, |, is
convergent, we say that the given double series is absolutely
convergent.

Since w = (/SV, - >SV, -i) - ('S' -], -'Sm-i, -iX it is easily seen
that, if the double series is convergent, then

lim Uf, = 0.

Stolz necessary and suffi cientf condition for convergence. A
condition for convergence which is obviously necessary (see\hardsubsectionref{2}{2}{2}) is
that, given e, we can find m and n such that | S +p, y - *%, / < e
whenever fi > m and v > n and p, a may take any of the values 0, 1, 2,
.... The condition is also sufficient; for, suppose it satisfied;
then, when fM> m + n, >S' +p, +p - S; < e.

Therefore, by\hardsubsectionref{2}{2}{2}, S has a limit *S'; and then making p and a tend
to infinity in such a way that p, + p = v + a, v,'e see that \ S - 8,,
, e when- ever p > m and v> n; that is to say, the double series
converges.

Corollary, An absolutely convergent double series is convergent. For
if the double series converges absolutely and if ty,t n he the sum of
m rows of n columns of the series of moduli, then, given f, we can
find fx such that, when p>ra>fi and q>/i>fi, i,j,q - t,n,n< - But \ Sp
q-S, n\ \ ip,q-tm,H and so \ Sp g-S, n\ < e when jc ??i>/x, q>n>fi;
and this is the condition that the double series should converge.

\Subsection{2}{5}{1}{Methods of summing double series.}
TODO

Let us suppose that S u, converges to the sum S . Then S >S' is
called the sum by rows of the double series; that is to say, the sum
by rows

OC/X\ 30/00\

is 5 ( S li, ). Similarly, the sum by columns is defined as 2 ( 2
V,*')- That these two sums are not necessarily the same is shewn b the
example

Su V =, in which the sum by rows is - 1, the sum by columns is + 1;

' ' p + v - -

and S does not exist.

* This definition is practically due to CsiU.chy, Analyse Algehrique,
p. 540. t This condition, stated by Stolz, Math. Ann. xsiv. (1884),
pp. 157-171, appears to have been first proved by Pringsheim.

J These methods are due to Cauchy.

%
% 28
%

Pringsheim's theorem* : If S exists and the sums by rows and columns
exist, then each of these suyns is equal to S.

For since S exists, then we can find m such that

I *S /i, p - S < e, if yu. > 7u, V > m.

And therefore, since lim >Sf exists, Mvn 8, ) - S %e; that is to say,

  Sp - S \ i e when fx > m, and so (§ 222) the sum by rows converges
to S. In like manner the sum by cohimns converges to S.

2"52. Absolutely convergent double series.

We can prove the analogue of \hardsubsectionref{2}{4}{1} for double series, namely that if
the terms of an absolutely convergent double series are taken in any
order as a simple series, their sum tends to the same limit, provided
that every term occurs in the summation.

Let cr be the sum of the rectangle of fx rows and v columns of the
double series whose general element is | m, |; and let the sum of
this double series be cr. Then given e we can find m and n such that
o- - cr < e whenever both fi > m and v> n.

Now suppose that it is necessary to take iV terms of the deranged
series (in the order in which the terms are taken) in order to include
all the terms of >S j/-fi,3/+i, and let the sum of these terms be ty

Then a' - 'Sj/-t-i, j/+i consists of a sum of terms of the type Up,q
in which p > m, q >n whenever M > m and M > n; and therefore

I h' - Sm+1,3T+1 O" - 0-j/+i\ 3/+1 < 2 f-

Also, S- Sji+i ijj i consists of terms iip q in which j) > m, q> n;
therefore I S - Sjfi+ijf+i I <r - o-jf+ M+i < 2 ' therefore | S-ty j <
e; and, corresponding to any given number e, we can find X; and
therefore ty- S.

Example 1. Prove that in an absolutely convergent double series, 2 !<
(, exists, and

H = l

thence that the sums by rows and columns respectively converge to S.

[Let the sum of fi rows of v columns of the series of moduli be t,
and let t be the sum of the series of moduli.

Then 2 \ ?i, \ < t, and so 2 m,;, converges; let its sum be 6;
then

\ bi\ + \ b.i\ + ... + \ b \ lim t t,

and so 2 b converges absolutely. Therefore the sum by rows of the
double series

exists, and similarly the sum by columns exists; and the required
result then follows from Pringsheim's theorem.]

* Loc. cit. p. 117.

%
% 29
%

Example 2. Shew from first principles that if the terms of au
absokitely convergent double series he arranged in the order

 M + ( 2,l + l,2) + ( :i,l + 2,2 + "l,3') + (*/4,l + --- + '<l,4) +
---5

tliis series converges to S*.

2"53. CaucJnj's theorem* on the multiplication of absolutely
convergent series.

We shall now shew that if two series

S = u + V. + (/3 + . . . and T =Vi + Vo + V3+ ...

are absolutely convergent, then the series

P = U Vi + 1 2 1 + "i V-i + . . .,

formed by the products of their terms, written in any order, is
absolutely con- vergent, and has for sum ST.

Let Sn = Ui + U., + . . . + Un,

Tn=V, + V.,+ ...+ Vn.

Then ST = lim S,, lini T = lim (SnTn)

by example 2 of\hardsectionref{2}{2}. Now

SnTn = Ui I'l + U>Vi + . . . 4- HnVi + l/il'2+ U.,V. + ... + UnV.

+

But this double series is absolutely convergent; for if these terms
are replaced by their moduli, the result is a Tn, where

<7" = 1 "1 I + I /2 I +    + 1 n I, rn = \ Vi \ + \ v.,\ + ...+\
Vn\,

and cTnTn is known to have a limit. Therefore, by \hardsectionref{2}{2}TODO:verifyref, if the
elements of the double series, of which the general term is u,nVn, be
taken in any order, their sum converges to ST.

Example. Shew that the series obtained by multiplying the two series 2
22 23 4,111

i + + + 2;. + 24 + -' 1 + 1 + + + -'

and rearranging accoi'ding to powers of z, converges so long as the
representative point of z lies in the ring-shaped region bounded by
the circles \ z\ = l and | a | = 2.

\Section{2}{6}{Fower-Seriesf.TODO}

A series of the type

Uo + aiZ + a z- + a-iZ" - ...,

in which the coefficients a, a,a2, a, ... are independent of z, is
called a series proceeding according to ascending powers of z, or
briefly a poiver-series.

* Analyse Algebrique, Note vii.

t The results of this section are due to Cauchy, Analyse Algebrique,
Ch. ix.

  30 THE PROCESSE>S OF ANALYSIS [CHAP. II

We shall now shew that if a power-series converges for any value z of
z, it ivill he absolutely convergent for all values of z whose
representative points are luithin a circle luhich passes through z and
has its centre at the origin.

00

For, if z be such a point, we have j | < i 'o | . Now, since S, V
converges,

anZo must tend to zero as ?i->oo, and so we can find M (independent
of n) such that

I 0.nZ,'' \ < M.

Thus i anz'' \ < M \ .

00

Therefore every term in the series S | n " | is less than the
corresponding term in the convergent geometric series

2i)/ -

the series is therefore convergent; and so the power-series is
absolutely convergent, as the series of moduli of its terms is a
convergent series; the result stated is therefore established.

Let lim | a | ~ "* = r; then, from\hardsubsectionref{2}{3}{5}, 2 anZ converges absolutely
when

00

\ z\ < r\ if \ z r, anZ" does not tend to zero and so 2 a z diverges
\hardsectionref{2}{3}).

The circle \ z\=r, which includes all the values of z for which the

power-series

tto + aiZ + a.iZ"- + ag -f- . . .

converges, is called the circle of convergence of the series. The
radius of the circle is called the radius of convergence.

In practice there is usually a simpler way of finding r, derived from
d'Alembert's test \hardsubsectionref{2}{3}{6}); r is lim ( / + 1) if this limit exists.

A power-series may converge for all values of the variable, as
happens, for instance, in the case of the series*

z z

which represents the function sin z; in this case the series
converges over the whole r-plane.

On the other hand, the radius of convergence of a power-series may be
zero; thus in the case of the series

I + 1\ z + 2\ z- - Z\ z + 4>\ z' - ...

we have

U

= n\ z

* The series for c, sin z, cos z and the fundamental properties of
these functions and of log z will be assumed throughout. A brief
account of the theory of the functions is given in the Appendix.

%
% 31
%

which, for all values of n after some fixed value, is greater than
unity when z has any value different from zero. The series converges
therefore only at the point z = 0, and the radius of its circle of
convergence vanishes.

A power-series may or may not converge for points which are actually
on the periphery of the circle; thus the series

z z- z'-' z* -'- + p + 25 + 3 + 4i +   '

whose radius of convergence is unity, converges or diverges at the
point z = 1 according as s is greater or not greater than unity, as
was seen in\hardsubsectionref{2}{3}{3}.

Corollary. If ( ) be a sequence of positive terms such that lim(a +i/a
) exists, this limit is equal to lim '"'.

261. Convergence of series derived from a poiuer-series.

Let cify + i2 + a.iZ- + a z + a r* + . . .

be a power-series, and consider the series

ai + 1a.,z - a z" + 4a42-' + . . ., which is obtained by
differentiating the power-series term by term. We shall now shew that
the deHved series has the same circle of convergence as the original
series.

For let 3 be a point within the circle of convergence of the
power-series; and choose a positive number 7'i, intermediate in value
between \ z\ and r the

X

radius of convergence. Then, since the series S n i" converges
absolutely, its

terms must tend to zero as n - > x; and it must therefore be possible
to find a positive number M, independent of, such that ! I < Mr ~ for
all values of n.

 X

Then the terms of the series 2 ?i j a | 1 2- 1 "~ are less than the
corre- sponding terms of the series

 =i

M 71:2,"-'

n =i n"~'

But this series converges, by\hardsubsectionref{2}{3}{6}, since \ z\ < r- . Therefore, by\hardsubsectionref{2}{3}{4}, the series

 ncin \ z\''-

X

converges; that is, the series 2 la "" converges absolutely for all
points z

n = l

X

situated within the circle of convergence of the original series 2
OnZ'' . When

n =

I I > ? anZ does not tend to zero, and a fortiori na z '' does not
tend to zero; and so the two series have the same circle of
convergence.

%
% 32
%

Corollary. The series 2 - - - > obtained by integrating the original
power-series term by term, has the same circle of convergence as 2 a
z""-.

n=0

\Section{2}{7}{Infinite Products.}

We next consider a class of limits, known as infinite products.

Let 1 + ai, 1 + a.2, 1 + as,    be a sequence such that none of its
members vanish. If, as n oo, the product

(1 + ai) (1 + 2) (1 + as)   . (1 + a ) (which we denote by Tin)
tends to a definite limit other than zero, this limit is called the
value of the infinite product

n = (l + a0(l + a,)(l + a3)...,

and the product is said to be convergent *. It is almost obvious that
a necessary condition for convergence is that lim cin = 0, since lim
Un-i = lim Un + 0.

00

The limit of the product is written II (1 4- cin).

n-l m ( rn \

Now n (l+a ) = exp- S log(l +a,,)K

w=l I n-\ . J

andf exp lim, ] = lim exp ii,n]

if the former limit exists; hence a sufficient condition that the
product

00

should converge is that 2 log(l + a ) should converge when the
logarithms

n = l

have their principal values. If this series of logarithms converges
absolutely, the convergence of the product is said to be absolute.

The condition for absolute convergence is given by the following
theorem : in order that the infinite product

(l+a,)(l+a2)(l + a3)... may he absolutely convergent, it is necessary
and sufiicient that the series

tti + 02 + as + . . . sJiould be absolutely convergent.

For, by definition, 11 is absolutely convergent or not according as
the

series

log (1 + Oi) + log (1 + Oo) + log (1 + ag) + ...

is absolutely convergent or not.

* The convergence of the product in which rt,i\ i= - l/n was
investigated by WaUis as early as 1655.

t See the Appendix, § A-2.

%
% 33
%

Now, since lim a = 0, Ave can find m such that, when n > ?n, | a | < |
; and then

  "' log (1 + Un) - 1

2- + 2 +    - 2 

<2 . + 23+--- =

And thence, when n> m, - h 5 therefore, by the comparison

theorem, the absolute convergence of 2 log (1 + ) entails that of Sa
and

conversely, provided that a 4= - 1 for any value of n.

This establishes the result*.

If, in a product, a finite number of factors vanish, and if, when
these are suppressed, the resulting product converges, the original
product is said to converge to zero. But such

a product as n (!- "') is said to diverge to zero.

n=2

Corollary. Since, if Sn- 'l, exp ( S' )- -exp, it follows from\hardsubsectionref{2}{4}{1}
that the factors of an absolutely convergent product can be deranged
without aftecting the value of the product.

Example 1. Shew that if n (1 + ) converges, so does 2 log (1 +a ), if
the logarithms

H=l )l=l

have their principal values.

Example 2. Shew that the infinite product

sin z sin \ z sin \ z sin z z ' \ z ' \ z ' \ z '" is absolutely
convergent for all values of z.

[For (sin-j /(-) can be written in the form 1 - |, where | X |<X- and
/ i.s inde-

l endent of n; and the series 2 - is aVjsolutely convergent, as is
seen on comparing

it with 2 - . The infinite product is therefore absohitely
convergent.] w = l '

2"71. Some examples of infinite products. Consider the infinite
product

  - ) -m-£)

which, as will be proved later (§ 7 '5), represents the function z sin
z.

In order to find whether it is absolutely convergent, we must consider
the

series 2 -, or -- S - : this series is absolutelv convergent, and so
the

product is absolutely convergent for all values of z. Now let the
product be written in the form

* A discussion of the convergence of infinite products, in which the
results are obtained without making use of the logarithmic function,
is given by Pringsheim, Math. Ann. xxxm. (1889), pp. 119-154, and also
by Bromwich, Infinite Series, Ch. vi.

W. M. A. 3

%
% 34
%

The absolute convergence of this product depends on that of the series

z z z z IT IT 27r 27r

But this series is only conditionally convergent, since its series of
moduli

\ z\ \ z\ \ z\ \ z\ IT IT 27r 27r

is divergent. In this form therefore the infinite product is not
absolutely

convergent, and so, if the order of the factors [ 1 + - ] is deranged,
there is

a risk of altering the value of the product.

Lastly, let the same product be written in the form

in which each of the expressions

1 + ) e mn

miTj

is counted as a single factor of the infinite product. The absolute
convergence of this product depends on that of the series of which the
(2?n - l)th and (2m)th terms are

1 + e mn - 1.

But it is easy to verify that

V mTTJ \ m-/

and so the absolute convergence of the series in question follows by
comparison

with the series

111111

l + l+2 + 2, + 3. + 3, + 4. + p+....

The infinite product in this last form is therefore again absolutely

convergent, the adjunction of the factors e '*"" having changed the
con- vergence from conditional* to absolute. This result is a
particular case of the first part of the factor theorem of Weierstrass
\hardsectionref{7}{6}).

Example 1. Prove that n ](l - ) e" is absolutely convergent for all
values of

n=i l.\ c-f-?i/ )

z, if c is a constant other than a negative integer.

For the infinite product converges absolutely with the series

n=i t\ c + nj J

%
% 35
%

Now the general term of this series is

But 2 - converges, and so, by\hardsubsectionref{2}{3}{4}, 2 ](l je"-!- converges
absolutely,

n=l n=\ l\ C + %/ J

and therefore the product converges absolutely.

Example 2. Shew that n jl-H--] z~'\ converges for all points z
situated

outside a circle whose centre is the origin and radius unity.

For the infinite product is absolutely convergent provided that the
series

=o / ] yn

2 1 S-"

is absolutely convergent. But lim (l -- ) =e, so the limit of the
ratio of the (w + l)th

term of the series to the Jith term is -; there is therefore absolute
convergence when

z

1

Example 3. Shew that

- < 1, i.e. when ] 2 | > 1. 1.2.3...(m-l)

2 2 ""U'

(2+l)(2 + 2)...(2 + 7ft-l)

tends to a finite limit as ??j- -x, unless 2 is a negative integer.

For the expression can be written as a product of which the nth factor
is

2 + 7i \ n ) ~\ n) V / I

This product is therefore absolutely convergent, provided the series

* 1 is absolutely convergent; and a comparison with the convergent
series 2 - shews that

this is the case. When 2 is a negative integer the expression does not
exist because one of the factors in the denominator vanishes.

Example 4. Prove that For the given product

,|o..(,-i)(i-i)(i.|)...(.-, )(i- )(i.£)

(,77 \ 2 3 "''2 ' 2k-l 2k k/

--log 2 .

= e sin 2.

= lim

X 2 1 - - e

 -27'

1-

2/C7

,2kn

1 +

2 \ -1-

klT

= lim e"-V 2+3"-+2A-i 2k) Ji\ \ en fi + \ e~ (l- e (1+ e <!-...,

3-2

%
% 36
%

since the product whose factors are

1 - - ) e

is absohitely convergent, and so the order of its factors can be
altered.

Since log2 = l-HJ-i + *---M

this shews that the given product is equal to

--logs . e " sin 2.

\Section{2}{8}{Infinite Determinants.}

Infinite series and infinite products are not by any means the only
known cases of limiting processes which can lead to intelligible
results. The researches of G. W. Hill in the Lunar Theory* brought
into notice the possibilities of infinite determinants.

The actual investigation of the convergence is due not to Hill but to
Poincare, Bull, de la Soc. Math, de France, xiv. (1886), p. 87. We
shall follow the exposition given by H. von Koch, Acta Math, xvi,
(1892), p. 217.

Let Aik be defined for all integer values (positive and negative) of
i, k, and denote by

the determinant formed of the numbers Aik i,k = - m, ... +m); then if,
as m - cc, the expression D,n tends to a determinate limit D, we
shall say that the infinite determinant

[- >i-J?,i-=-< ...+oo

is convergent and has the value D. If the limit D does not exist, the
deter- minant in question will be said to be divergent.

T he elements An, (where i takes all values), are said to form the
principal diagonal of the determinant D; the elements Aik, (where i is
fixed and k takes all values), are said to form the 7'ow i; and the
elements A c, (where k is fixed and i takes all values), are said to
form the column k. Any element A-iy; is called a diagonal or a
non-diagonal element, according as = A; or i \$ k. The element udo.o
is called the origin of the determinant.

2'81. Convergence of an infinite determinant.

We shall now shew that an infinite determinant converges, provided the
product of the diagonal elements converges absolutely, and the sum of
the non-diagonal elements converges absolutely.

For let the diagonal elements of an infinite determinant I) be denoted
by l+a, and let the non-diagonal elements be denoted by ajj., i=¥k),
so that the determinant is

* Reprinted in Acta Mathematica, viii. (1886), pp. 1-36. Infinite
determinants had previously occurred in the researches of Fiirstenau
on the algebraic equation of the 7ith degree, Darstellung der reellen
Wurzeln alyebraincher Gleiclnmgen durch Determinanten der
Coeffizienten (Marburg, 1860). Special types of infinite determinants
(known as continuants) occur in the theory of infinite continued
fractions; see Sylvester, Math. I'apers, i, p.~504 and in, p.~249

2-8-2'82]

THE THEORY OF CONVERGENCE

37

Then, since the series  2

i,k=-

is convergent.

Now form the products

m / m \ m / m

P, = n 1+ 2 au-, P,= n 1+ 2

i"=-ni\ fc= - m / i = -n( \ fc=-m

then if, in the expansion of P i, certain terms are replaced by zero
and certain other terms have their signs changed, we shall obtain i),
; thus, to each term in the expansion of 2>, there corresponds, in
the expansion of P,, a term of equal or greater modulus. Now An + p -
An represents the sum of those terms in the determinant i), + p which
vanish when the numbers TODO are replaced by zero; and to each of
these terms there corresponds a term of equal or greater modulus in Pm
+ p- m-

Hence

B

,-D \ < R,

-P.

Therefore, since P,n tends to a limit as ni-*-cc, so also Z), tends
to a limit. This establishes the proposition.

2"82. The rearrangement Theorem for convergent infinite determinants.

We shall now shew that a determinant, of the convergent form already
co)isidered, remains convergent when the elements of any row are
replaced by any set of elements whose moduli are all less than some
fixed positive mimber.

Replace, for example, the elements

-" 0, - >i

'0

A,

of the row through the origin by the elements

.../!\,,... jiQ ... ix,n ...

which satisfy the inequality

I M>- I < Mj where /x is a positive number; and let the new values of
Z) i ' nd D be denoted by Dm and D'. Moreover, denote by /* / and P'
the products obtained by suppressing in P,n and P the factor
corresponding to the index zero; we see that no terms of 2) ' can
have a greater modulus than the corresponding term in the expansion of
nP '; and consequently, reasoning as in the last article, we have

which is sufficient to establish the result stated.

Example. Shew that the necessary and sufficient condition for the
absolute conver- gence of the infinite determinant

lim 1 a, ...

 2

a., ... 1 as ...

is that the series

shall be absolutely convergent.

,0 ... /3, 1

ai/3i + ao/32 + 03/33 + ...

(von Koch.)

38

THE PROCESSES OF ANALYSIS

[chap. II

REFERENCES. Convergent series.

A. Pringsheim, Math. Ann. xxxv. (1890), pp. 297-394.

T. J. I'a. Bromwich, Theory of Infinite Series (1908), Chs. Ii, ill,
iv.

Conditionally convergent series.

G. F. B. Riemann, Ges. Math. Werke, pp. 221-225. A. Pringsheim, Math.
Ami. xxii. (1883), pp. 455-503.

Double series.

A. Pringsheim, MUnchener Sitzungsherichte, xxvii. (1897), pp. 101-152.

    Math. Ann. hill. (1900), pp. 289-321.

G. H. Hardy, Proc. London Math. Soc. (2) i. (1904), pp. 124-128.

Miscellaneous Examples.

1. Evaluate litn (e """ ), lim (?i~ ogn) when a>0, b>0.

2. Investigate the convergence of

3. Investigate the convergence of

(l.3...2n-l 4/1 + 3

\addexamplecitation{Trinity, 1904.}

- 1 . . . . \addexamplecitation{Peterhouse, 1906.}

 =i\ 2. 4, ..2% 2>i + 2j '

4. Find the range of vakies of z for which the series

2sin2s-4sin- 2 + 8sin''s-... + (-)™ + i2 sin2' s+... is convergent.

5. Shew that the series

1 l\ \ 1 1\

Z 2+1 2 + 2 2 + 3

is conditionally convergent, except for certain exceptional values of
z; but that the series

11 11

1 Jl\ 1

Z 2+1 Z + p-\ Z+p Z+p + l

+

+ ..

2 + 2p + -l 2 + 2 + 5- in which (p + q) negative terms always follow p
positive terms, is divergent. \addexamplecitation{Simon.}

6. Shew that

l-i - 1 + 1-1- 1+1- =ilof 2

7. Shew that the series

is convergent, although

8. Shew that the series is convergent although

1111

1" 23 3" 4

 2n + l/W2n- a>-

a + /3- + a3 + /3< + ...

\addexamplecitation{Trinity, 1908.} (l<a</3)

\addexamplecitation{Ceskro.}

(0<a</3<l)

\addexamplecitation{Cesaro.}

THE THEORY OF CONVERGENCE

9. Shew that the series

   z"-i (H-%-i) -l

39

 =l(3"-l) 3 -(l+ ->)

converges absokitely for all values of z, except the values

Z-(\ j gtkKilm

(a = 0, 1; k = 0, 1, ... m-\; i = l, 2, 3, ...).

10. Shew that, when s> 1,

I i= + i r + j-j- i Lii

and shew that the series on the right converges when <s< 1.

(de la Vallee Poussin, Mem. de VAcad. de Belgique, liii. (1896), no.
6.)

11. In the series whose general term is

   = y - yi'"'+l', (0<9<1< )

where v denotes the number of digits in the expression of n in the
ordinary decimal scale of notation, shew that

lim u = q,

and that the series is convergent, although lim Mn+i/wn==c 

12. Shew that the series

where !? v = j' ""'"""', (0<5'<1)

is convergent, although the ratio of the (?t + l)th term to the nth is
greater than unity

when n is not a triangular number. \addexamplecitation{Ceskro.}

13. Shew that the series

2

 =o( <' + )*' where w is real, and where w + nY is understood to mean
e iog(w + n) the logarithm being taken in its arithmetic sense, is
convergent for all values of s, when 1 x) is positive, and is
convergent for values of s whose real part is positive, when x is real
and not an integer.

14. If Un>0, shew that if 2?< converges, then lim nu, = 0, and that,
if in addition

M,i w,i + i, then lim (?i?< ) = 0.

15. If

shew that

m - n m + n - 1) !

"MLn - om+n

2m+n m\ n\ '

am,o = 2-'", ao, =-2-, ao,o=0,

n!=0 i=0 / =0 \ m=0

(m, n>0)

\addexamplecitation{Trinity, 1904.}

16. By converting the series

1+:

16g2 24g3

+ .2 "f" 1 l3 +    ?

l-q 1+ 2 l-jS (in which | g- 1 < 1), into a double series, shew that
it is equal to

1 +

X2 +

2 8(73

  qf (1+ 2)2 (1\ 3)2

\addexamplecitation{Jacobi.}

%
% 40
%

17. Assuming that sin2 = 3 n (I--5-,),

shew that if ? - qo and 3t-*-co iu such a way that lim (in\ n) = k,
where k is finite, then

hm n' 1 + - =F'',

the prime indicating that the factor for which r = is omitted. (Math.
Trip., 1904.)

18. If Uq-=Ui = U2 = 0, and if, when n>\,

W2 -i- -

 7l'

1 1 1

si 11 n n y/n

then n (1 +?< ) converges, though 2 and 2 2 are divergent.

n=0 n=0 ?i=0

\addexamplecitation{Math. Trip. 1906.}

19. Prove that

n - 1 - - exp 2

where k is any positive integer, converges absolutely for all values
of z.

20. If 2 a,i be a conditionally convergent series of real terms, then
n (l4-,i) con-

n=l n=l

verges (but not absolutely) or diverges to zero according as 2 a 2
converges or diverges.

11=1

\addexamplecitation{Cauchy.}

21. Let 2 dn be an absolutely convergent series. Shew that the
infinite determinant

11=1

A C)-

(C- 4)2- 0

~e.

-00

42- 0

- 8

42- 0

- 4

   42- 0

42- 0

42- 0 -

-e,

(c- 2)2- 0

22- 0

-01

-02

2- -00

- 3

    22- 0

22- 0

22- 0

-( 2

02- 0

02- 0

-0,

02- 0

-02

'" 02- 0

02- 0

-e.

-02

2'-0o

- 1

22- 0

(c + 2)2- 22- 0

 0 ~ 1

- 2 -3,

22- 0 -

- 4

-6,

- 2

-

(c + 4)2- o

  42- 0

42- 0

42- 0

42- 0

42- 0

converges; and shew that the equation

is equivalent to the equation

A(c) =

sin2 Ittc = A (0) sin2 W(9o-- .

(Hill; see\hardsubsectionref{19}{4}{2}.)


\chapter{Continuous Functions and Uniform Convergence} 

3"1. The dependence of one complex number on another.

The problems with which Analysis is mainly occupied relate to the
dependence of one complex number on another. If z and are two complex
numbers, so connected that, if z is given any one of a certain set of
values, corresponding values of can be determined, e.g. if is the
square of z, or if = 1 when z is real and = for all other values of z,
then is said to be a function of z.

This dependence must not be confused with the most important case of
it, which will be explained later under the title o analytic
functionality.

If f is a real function of a real variable z, then the relation
between ( and 2, which may be written

can be visualised by a curve in a plane, namely the locus of a point
whose coordinates referred to rectangular axes in the plane are (s,
(). No such simple and convenient geometrical method can be found for
visualising an equation

considered as defining the dependence of one complex number f = | +
i'7 on another complex number z = x + i)/. A representation strictly
analogous to the one already given for real variables would require
four-dimensional space, since the number of variables , fj, X, y is
now four.

One suggestion (made by Lie and Weierstrass) is to use a
doubly-manifold system of lines in the quadruply-manifold totality of
lines in three-dimensional space.

Another suggestion is to represent and;; separately by means of
surfaces

A third suggestion, due to Heffter*, is to write

then draw the surface r = r x, y) — which may be called the
modular-surface of the function— and on it to express the values of 6
by surface-markings. It might be possible to modify this suggestion in
various ways by representing 6 by curves drawn on the surface r=r (.r,
y).

3'2. Continuity of functions of real variables.

The reader will have a general idea (derived from the graphical
represen- tation of functions of a real variable) as to what is meant
by continuity.

* Zeitschrift fur Math, tind Phys. xlix. (1899), p. 235.



42 THE PROCESSES OF ANALYSIS [cHAP. Ill

We now have to give a precise definition which shall embody this vague
idea.

Jjetf x) be a function of x defined when a x b.

Let x be such that a x b. If there exists a number I such that,
corresponding to an arbitrary positive number e, we can find a
positive number rj such that

\ f a:)-l\ < e,

whenever \ x — Xi\ < rj, x x , and a x b, then I is called the limit
of /(a;)

It may happen that we can find a number 1+ (even when I does not
exist) such that \ f(x) — l+\ < € when x- < x < Xi + rj. We call Z+
the limit of f x) when X approaches x- from the right and denote it
by/(a;i + 0); in a similar manner we define f x — 0) if it exists.

If f(xi + 0), f xi), f x- — 0) all exist and are equal, we say that
/(a-) is continuous at x; so that \ if x) is continuous at x- , then,
given e, we can find 7] such that

\ f x) -f(x,) I < e,

whenever \ x — Xi\ < rj and a x h.

If 1+ and l\ exist but are unequal, f x) is said to have an ordinary
discontinuity* at Xi] and if 1+ = /\ 4=/( i), f ) is said to have a
removable discontinuity at x- .

lif x) is a complex function of a real variable, and \ if x) = g(x) +
i h (x) where g (x) and h (x) are real, the continuity of f(x) at x
implies the con- tinuity of g (x) and of A (x). For when \ f(x) —f x \
< e, then | g (x) —g(xi) \ < e and I h (x) — h xi)\ < e; and the
result stated is obvious.

Example. From 2*2 examples 1 and 2 deduce that if f x) and cf) (x) are
con- tinuous at Xi , so are f x) + x), f x) x x) and, if (.rj) =t= 0,
f x)/(l) (.*•).

The popular idea of continuity, so far as it relates to a i-eal
variable f x) depending on another real variable x, is somewhat
different from that just considered, and may perhaps best be expressed
by the statement "The function f x) is said to depend con- tinuously
on X if, as x passes through the set of all values intermediate
between any two adjacent values Xi and X2, f x) passes through the set
of all values intermediate between the corresponding values /( j)
and/(.r2)."

The question thus arises, how far this popular definition is
equivalent to the precise definition given above.

Cauchy shewed that if a real function f x), of a real variable . •,
satisfies the precise definition, then it also satisfies what we have
called the popular definition; this result

* If a function is said to have ordinary discontinuities at certain
points of an interval it is implied that it is continuous at all other
points of the interval.



3*2 1] CONTINUOUS FUNCTIONS AND UNIFORM CONVERGENCE 43

vill be proved in 3"63. But the converse is not true, as was shewn by
Darboux. This fact may be iUustrated by the following example*.

Between x= — I and x= +1 (except at x=0), let f x) = sin —; and
let/(0)=0.

It can then be proved that/ .r) depends continuously on x near .r=0,
in the sense of the popular definition, but is not continuous at a; =
in the sense of the precise definition.

Example. If f x) be defined and be an increasing function in the range
(a, b), the limits /(.r + 0) exist at all points in the interior of
the range.

[If f x) be an increasing function, a section of rational numbers can
be found such that, if a, A be any members of its Z-cla.ss and its
-clas.s, a<f x + h) for evei-y positive value of h and A ' f x- h) for
some po.sitive value of h. The number defined by this section is/(a; +
0).]

3"21. Simple curves. Continua.

Let X and y be two real functions of a real variable t which are
continuous for every value of t such that a- t %h. We denote the
dependence of x and y on t by writing

.' = •(0. 3/ = i/(0- a. t%h)

The functions x (t), y (t) are supposed to be such that they do not
assume the same pair of values for any two different values of t in
the range a < t < b.

Then the set of points with coordinates (x, y) corresponding to these
values of t is called a simple curve. If

X (a) = X h), y (a) = y (b), the simple curve is said to be closed.

Example. The circle x + 7/-= 1 is a simple closed curve; for we may
write t

x=coiit, y = smt. (0 27r)

A two-dimensional continuum is a set of points in a plane possessing
the following two properties :

(i) If (x, y) be the Cartesian coordinates of any point of it, a
positive number 8 (depending on x and y) can be found such that every
point whose distance from x, y) is less than S belongs to the set.

(ii) Any two points of the set can be joined by a simple curve
consisting entirel '' of points of the set.

Example. The points for which .r-+3/-<l form a continuum. For if P be
any point inside the unit circle such that OP=r<\, we may take 8=1-?-;
and any two points inside the circle may be joined by a straight line
lying wholly inside the circle.

The following two theorems | will be assumed in this work; simple
cases of them appear ob\ dous from geometrical intuitions and,
generally, theorems of a similar nature will be taken for granted, as
formal proofs are usually extremely long and difficult.

* Due to Mansion, Mathesis, (2) xix. (1899), pp. 129-131.

t For a proof that the sine and cosine are continuous functions, see
the Appendix, § A-41. + Formal proofs will be found in Watson's
Complex Integration and Cauchy's Theorem. (Cambridge Math. Tracts, No.
15.)



44 THE PROCESSES OF ANALYSIS [CHAP. Ill

(I) A simple closed curve divides the plane into two continua (the '
interior ' and the ' exterior '). '

(II) If P be a point on the curve and Q be a point not on the curve,
the angle between QP and Ox increases by + 27r or by zero, as P
describes the curve, according as Q is an interior point or an
exterior point. If the increase is + 27r, P is said to describe the
curve ' counterclockwise.'

A continuum formed by the interior of a simple curve is sometimes
called an open two-dimensional region, or briefly an open region, and
the curve is called its boundary; such a continuum with its boundary
is then called a closed two-dimensional region, or briefly a closed
region or domain.

A simple curve is sometimes called a closed one-dimensional region; a
simple curve with its end-points omitted is then called an open
one-dimensional region.

3"22. Continuous functions of complex variables.

Lety"( ) be a function of defined at all points of a closed region
(one- or two-dimensional) in the Argand diagram, and let z be a point
of the region.

Then f z) is said to be continuous at z , if given any positive number
e, we can find a corresponding positive number 77 such that

\ f z)-f z,)\ < e, whenever \ z — Zx\ < r) and 2- is a point of the
region.

3'3. Series of variable tei-ms. Uniformity of convergence. Consider
the series

  1 + a;2 (1+ ieO' (1 + T

This series converges absolutely (§ 2*33) for all real values of x. If
8n ( ) be the sum of n terms, then ' >•

' " ">= +" -(TT ' /

and so lim Sn x) = \ \ x'; xrf 0)

but Sn 0) = 0, and therefore lim Sn (0) = 0.

M-*-00

Consequently, although the series is an absolutely convergent series
of continuous functions of x, the sum is a discontinuous function of
x. We naturally enquire the reason of this rather remarkable
phenomenon, which was investigated in 1841-1848 by Stokes*, Seidelf
and Weierstrassj, who shewed that it cannot occur except in connexion
with another phenomenon, that of non-uniform convergence, which will
now be explained.

* Cnmb. Phil. Trans, viii. (1847), pp. 533-583. [Collected Papers, i.
pp. 236-313.] t Mi'mchener Abhandlungen, v. (1848), p. 381. X Ges.
Math. U'erke, i. pp. 67, 75.



3-22-3'3l] CONTINUOUS FUNCTIONS AND UNIFORM CONVERGENCE 45

Let the functions u (z), Wg (z), ... be defined at all points of a
closed region of the Argand diagram. Let

Sn (z) = U, (z) + M2 (2) + ...+ tin Z). '

oc

The condition that the series 2 Un z) should converge for any
particular

n = l

value of z is that, given e, a number n should exist such that

I n+p z) — Sn (z)\ < €

for all positive values of jo, the value of 7i of course depending on
e.

Let n have the smallest integer value for which the condition is
satisfied. This integer will in general depend on the particular value
of z which has been selected for consideration. We denote this
dependence by writing n (z) in place of ?i. Now it mag happen that we
can find a number N,

INDEPENDENT OF Z, SUch that

n z)<N "' '"

for all values of z in the region under consideration.

If this number N exists, the series is said to converge uniformly
throughout the region.

If no such number iV exists, the convergence is said to be
non-uniform*.

Uniformity of convergence is thus a property depending on a whole set
of values of z, whereas previously we have considered the convergence
of a series for various particular values of z, the convergence for
each value being con- sidered without reference to the other values.

We define the phrase ' uniformity of convergence near a point z ' to
mean that there is a definite positive number 8 such that the series
converges uniformly in the domain common to the circle \ z — z \ \ h
and the region in which the series converges.

3'31. On the condition for uniformity of convergence' .

If Rn,p z) = Un+i z) + iin+2 (z) + ... + Un+p (z), WO have Seen that
the

necessary and sufficient condition that S Un (z) should converge
uniformly

in a region is that, given any positive number e, it should be
possible to choose N INDEPENDENT OF z (but depending on e) such that

I Rn, p(z)\ < €

for ALL positive integral values of p.

* The reader who is unacquainted with the concept of uniformity of
convergence will find it made much clearer by consultinf; Bromwich,
Iiijinite Series, Ch. vii, where an illuminating account of Osgood's
graphical investigation is given.

t This section shews that it is indifferent whether uniformity of
convergence is defined by means of the partial remainder Rj p(z) or by
iJ (2). Writers differ in the definition taken as fundamental.



46 THE PROCESSES OF ANALYSIS [CHAP. Ill

If the condition is satisfied, by § 2-22, 8n z) tends to a limit, S
z), say for each value of under consideration; and then, since e is
independent of p,

and therefore, when n > iV,

S (2) - Sn (Z) = I lim i v, p 2) - E v. n-N (z), /

and so \ S(z)-S iz)\ < 2€.

Thus (writing e for e) a necessary condition for uniformity of
convergence is that \ S z) — Sn (z) \ < e, whenever n>N and N is
independent of z; the condition is also sufficient; for if it is
satisfied it follows as in § 2-22 (I) that I R rp z) 1 < 2e, which, by
definition, is the condition for uniformity.

Example 1. Shew that, if x be real, the sum of the series

X X X

TODO

is discontinuous at .r=0 and the series is non-uniformly convergent
near ' = 0.

The sum of the first n terms is easily seen to be 1; so when x — Q
the

nx+i

sum is; when a* 4=0, the sum is 1.

1 ' The value of RnXx) = S x)-Sn x) is — -- if x O; so when x is
small, say

:r=one-hundred-milliouth, the remainder after a million terms is — or
l-TTyTj

100 + 1 the first million terms of the series do not contribute one
per cent, of the sum. And in

general, to make < e, it is necessary to take

° nx + 1

7i>-(--l

X \ e

Corresponding to a given e, no number N exists, independent of x, such
that n<N for all values of x in any interval including x =; for by
taking x sufficiently small we can make n greater than any number N
which is independent of x. There is therefore non- uniform convergence
near ,r = 0.

Example 2. Discuss the series

x n. n±V)x'''-\ ]



nZx + n x \ + 7i- \ fx y in which x is real.

r. , , , . nx (n- ) X , . X ,

The 7ith term can be written , 5—5 - , —; , . — q , so o (x) = n ,
and

l+n\ v l + n + \ y X'' l+x''

  W ] +(,i + 1)2 2 •

[Note. In this example the sum of the series is not discontinuous at =
0.] But (taking 6<i, and 4=0), \ Rn x)\ < e if e-i(?i + l) \ x\ < l-ir
n+\ f x; i.e. if ?i+l>| €-i + Ve-2-4 |.T|-l or ?t + l<J e-i-v'e = 3-4
|:p|- .



3*32] CONTINUOUS FUNCTIONS AND UNIFORM CONVERGENCE 47

Now it is not the case that the second inequality is satisfied for all
values of n greater than a certain value and for all values of x; and
the first inequality gives a value of n(x) which tends to infinity as
x- 0; so that, corresponding to any interval containing the point =0,
there is no number N' independent of x. The series, therefore, is
non-uniformly convergent near :i, = 0.

The reader will observe that n x) is discontinuous at .r = 0; for n
x)- 'x> as .r- -0, but n(0) = 0.

3'32. Connexion of discontinuity luith non-uniform convergence.

We shall now shew that if a semes of continuous functions of z is
uniformly convergent for all values of z in a given closed domain, the
sum is a continuous function of z at all points of the domain.

For let the series be f z) = it, z) + Wo (2 ) + • . . + Un z)+ ...— Sn
(z) + Rn z), where Rn (z) is the remainder after n terms.

Since the series is uniformly convergent, given any positive number e,
we

can find a corresponding integer n independent of z, such that | R ( )
| < e

for all values of z within the domain.

Now n and e being thus fixed, we can, on account of the continuity of
Sn (•2 ), find a positive number rj such that

\ Sn(z)-S z')\ < l€,

whenever \ z — z' \ < r . We have then

l/( ) -/V) I = 1 [Sni ) - Sn z')] \ + lRn(2)- Rn ) \ < ! Sn z) -
Sn(z') I + I Rniz) \ + | Rn z') |

' < 6,

which is the condition for continuity at z. Example 1. Shew that near
x = the series

Ui x) + U<i x) + M3 ( 0 + • • • ,

1 1

where i(vP)=a;, (a;) =;p "~* — :p "~ ,

and real values of x are concerned, is discontinuous and non-uniformly
convergent.

In this example it is convenient to take a slightly different form of
the test; we shall shew that, given an arbitrarily small number f, it
is possible to choose values of x, as small as we please, depending on
n in such a way that | R [x) \ is not less than e for any value of n,
no matter how large. The reader will easily see that the existence of
such values of x is inconsistent with the condition for uniformity of
convergence. i\

The value of S ix) is .r" "-i; as n tends to infinity, S x) tends to
1, 0, or - 1, accord- ing as x is positive, zero, or negative. The
series is therefore absolutely convergent for all values of or, and
has a discontinuity at .t' = 0.



48 THE PROCESSES OF ANALYSIS [cHAP. Ill

1

In this series R x) = l-x ''-\ x > 0); however great n may be, by
taking* x = e- (2" - 1) we can cause this remainder to take the value
l-e' which is not arbitrarily small. The series is therefore
non-uniformly convergent near .r = 0.

Example 2. Shew that near z = the series

 il H-(l+2) -l l+(l + 2)"

is non-uniformly convergent and its sum is discontinuous. The nth term
can be written

l-(l+3)" 1-(1+Z)"-

l + (H-3) H-(l+z) -i' so the sum of the first n terms is , — — . Thus,
considering real values of z greater

than - 1, it is seen that the sum to infinity is 1, 0, or — 1,
according as z is negative, zero,

or positive. There is thus a discontinuity at 2 = 0. This
discontinuity is explained by the

fact that the series is non-uniformly convergent near 2=0; for the
remainder after n terms

in the series when z is positive is

-2

r+rr+2)"'

and, however great n may be, by taking z = ~, this can be made
numerically greater

2

than - , which is not arbitrarilv small. The series is therefore
non-uniformlv con- 1-1-e'

vergent near 2 = 0.

3'33. The distinction between absolute and uyiiform convergence.

The uniform convergence of a series in a domain does not necessitate
its absolute convergence at any points of the domain, nor conversely.
Thus

the series S vz r converges absolutely, but (near z = 0) not uniformly
;

(1 + ")"

while in the case of the series

 =i z + n ' the series of moduli is

1

5", =



 =i \ n + Z'\ \ which is divergent, so the series is only
conditionally convergent; but for all real values of z, the terms of
the series are alternately positive and negative and numerically
decreasing, so the sum of the series lies between the sum of its first
n terms and of its first (n -f 1) terms, and so the remainder after n
terms is numerically less than the nth. term. Thus we only need take a
finite number (independent of z) of terms in order to ensure that for
all real values of z the remainder is less than any assigned number e,
and so the series is uniformly convergent.

Absolutely convergent series behave like series with a finite number
of terms in that we can multiply them together and transpose their
terms.

* This value of x satisfies the condition i a; | < 5 whenever 2rt - 1
> log 5~i.



3'33-3"341] CONTINUOUS functions and uniform convergence 49

Uniformly convergent series behave like series with a finite number of
terms in that they are continuous if each term in the series is
continuous and (as we shall see) the series can then be integrated
term by term.

334. A condition, due to Weierstrass* , for uniform convergence.

A sufficient, though not necessary, condition for the uniform
convergence of a series may be enunciated as follows : —

If, for all values of z within a domain, the moduli of the terms of a
series *S' = Ui (z) + i<2 ( ) + W3 ( ) + • • • are respectively less
than the corresponding terms in a convergent series of positive terms

where M is independent of z, then the series S is uniformly convergent
in this region. This follows from the fact that, the series T being
convergent, it is always possible to choose n so that the remainder
after the first n terms of T, and therefore the modulus of the
remainder after the first n terms of *Si, is less than an assigned
positive number e; and since the value of n thus found is independent
of z, it follows (§ 3-31) that the series S is uni- formly convergent
; by § 234, the series S also converges absolutely.

Example. The .scries

1 ., 1 ,

cos Z + , CO.S- 2+57, CO.S-* Z- ...

is uniformly convergent for all re<il values of r, because the moduli
of its terms are not greater than the corresi)onding terms of the
convergent series

I i whose terms are positive constants.

3 •341 . Uniformity of convergence of infinite products t.

A convergent product n 1 + ?< (z) is said to converge uniformly in a
domain of values

M=l

of z if, given e, we can tind m independent of z such that

n 1 + (z) - n i+u z) \ < €

for all positive integral values of p.

The only condition for imiformity of convergence which will be used in
this work is that the product converges uniformly if | m (s) | < J/
where J/ is independent of 2 and

2 J/ convei-ges.

n = l

* Abhandlungen aus der Funktionenlehre, p. 70. The test given by this
condition is usually described (e.g. by Osgood, Annals of Mathematics,
iii. j[1889), p. 130) as the M-test.

t The definition is, effectively, that given by Osgood,
Funktionentheorie, p. 462. The condition here given for uuiformity of
convergence is also established in that work.

W. M. A. 4



50 THE PROCESSES OF ANALYSIS [CHAP. Ill

To prove the validity of the condition we observe that n (l + J/ )
converges (§ 2-7),

M = l

and so we can choose m such that

Vl+p VI

n l + M,, - n l + M <€;

n = l )( = 1

and then we have

m+p m I I m p ni+p ~] I

n l+un (.-) - n 1 +t< (s) =1 n 1 + (z) n i +u (z) - 1

11=1 M=l I I n=l \ \ n=m+i J I

m r ni+2> ~]

 n(i + 14) n i+i/ -i

m=l L n = m+l J

and the choice of ?h is independent of z.

3 "35. Hardy's tests for uniform convergence*.



The reader will see, from § 2-31, that if, in a given domain.



p



2 a z) k where a (2) is



real and k is finite and independent of and 2, and if / (2) t + i (s)
and fn )-

uniformly as w - - oo , then 2 a z) f £) converges uniformly. t=i

Also that if where k is independent of 2 and 2 a (2) converges
uniformly, then 2 a,j £) w (s) con-

Ii = l M = l

verges uniformly. [To prove the latter, observe that m can be found
such that

 7n + l(2), m + l(2)+ 7 + 2(2), •••, m + 1 ( ) + m + 2 ( )+ • • + m +
p (2)

are numerically less than e\ k; and therefore (§ 2-301)

2 a (2) M (s) < e <,n+i (2)//(-< e, n=jn+l I

and the choice of e and hi is independent of 2.]



° cos nQ "" sin % 2 , 2

7t=l '* n = l 'i



Example 1. Shew that, if S>0, the series

converge uniformly in the range

S (9 27r - S.

Obtain the corresponding result for the series

  (-)"cos?i<9 ' ( - ) sin n6

2i , 2 ,

n=l n n=\ n

by writing O + n for .

Example 2. If, when a a,' 6, | co,i (.1;) | < -j and 2 | < + i (.r) —
co (.r) | <j('2, where

)(=i

 •1, k.> are independent of n and .r, and if 2 a is a convergent
series independent of x,

n=i

then 2 a,tC > (:*;) converges uniformly when a .r / . (Hardy.)

n = l

* Proc. London Math. Soc. (2) iv. (1907), pp. 247-265. These results,
which are generalisa- tions of Abel's theorem (§ 3-71, below), though
well known, do not appear to have been published before 1907. From
their resemblance to the tests of Dirichlet and Abel for convergence,
Bromwich proposes to call them Dirichlet's and Abel's tests
respectively.



3"35, 3"4] CONTINUOUS FUNCTIONS AND UNIFORM CONVERGENCE 51

[For we can choose m, independent of .v, such that corollary, we have



m+p I

2 a < e, and then, l)y § 2-301

n=m+l I



m+p I

2 ttnOin ( •) I < ( 'l + ' '2) f •]



n=m+l I

3 "4. Discussion of a particular double senes.

Let (1)1 and w. be any constants whose ratio is not purely real; and
let a be positive.

The series 2; r- , in which the summation extends over

all positive and negative integral and zero values of in and n, is of
great importance in the theory of Elliptic Functions. At each of the
points z = — 2mcoi — 2/10)2 the series does not exist. It can be shewn
that the series converges absolutely for all other values of if a > 2,
and the convergence is uniform for those values of z such that; 2 +
2niQ)i + 2nco2 8 for all integral values of m and n, where 8 is an
arbitrary positive number.

Let S' denote a summation for all integral values of m and n, the term
for which 7n = n = being omitted.

Now, if ni and n are not both zero, and if \ z + 2m(o + 2nco.2\ " 8 >
for all integral values of m and n, then we can find a positive number
C. de- pending on B but not on 2, such that

I 1 !



(z + 2w\&)i + 2 a)a)''



 2m(Oi + 2710)2)"



Consequently, by § 3"34, the given series is absolutely and uniformly*
convergent in the domain considered if

2' 1

I mcoi + no)2 1 " converges.

To discuss the convergence of the latter series, let

0), = CTj 4- 1/3, , 0)0 = Qfo + z'/So ,

where a , a.,, /3i, /Sa are real. Since co../ coi is not real, a /S. —
ou/3i 4= 0. Then the series is

2'

 (a,m + aojiy- + /3,m + /SoTi) *

This converges (§ 2"5 corollary) if the series

S = ' i — -

(m-+n2)

converges; for the quotient of corresponding terms is

The reader will easily define uniformity of convergence of double
series (see § 3-5).

4—2



52 THE PROCESSES OF ANALYSIS [cHAP. Ill

where /z = njm. This expression, qua function of a continuous real
variable jx, can be proved to have a positive minimum* (not zero)
since ofi/3o — ao/3i =|=; and so the quotient is always greater than
a positive number K (independent of/x).

We have therefore only to study the convergence of the series S. Let

PI 1

V V S'



 'p,q



00 00 1

 4 S S' - , .

m = n=0 (m + 71 )2 *

Separating Sp g into the terms for which m = n, m > n, and m < n, re-
spectively, we have

pi p m-l 1 q n-1 1

IS. = S - I- s S - + s s .

'% 1 ml

But S r- <



n=o ni" + n ) "- (m ) "



m°



Therefore IS t -J— + S — + i .

But these last series are known to be convergent if a — 1 > 1. So the
series S is convergent if a > 2. The original series is therefore
absolutely and uni- formly convergent, when a > 2, for the specified
range of values of z.

Example. Prove that the series

1

2



(mj + 7112 + . . . + my?y

in which the summation extends over all positive and negative integral
values and zero values of mj, m2, ... wi , except the set of
simultaneous zero values, is absolutely convergent if fi>ir.
(Eisenstein, Journal fur Math, xxxv.)

3'5. The concept of uniformity.

There are processes other than that of summing a series in which the
idea of uniformity is of importance.

Let e be an arbitrary positive number; and let f z, ) be a function of
two variables z and , which, for each point z oi a, closed region,
satisfies the inequality \ f z, ) | < 6 when t, is given any one of a
certain set of values which will be denoted by ( z); the particular
set of values of course depends on the particular value of z under
consideration. If a set ( )o can be found such that every member of
the set ( )o is a member of all the sets ( j), the function f z, ) is
said to satisfy the inequality uniformly for all points z of

* The reader will find no difficulty in verifying this statement; the
minimum value in question is given by

K "' = h W + a + . + r- (a,-/3,r-+(a2 + ,3i)=l (ai + /32)2+ (a - iP ]-



3"5, 3'6] CONTINUOUS FUNCTIONS AND UNIFORM CONVERGENCE 53

the region. And if a function (f> z) possesses some property, for
every positive value of e, in virtue of the inequality \ f z, ):<e,(f)
(z) is then said to possess the property uniformly.

In addition to the uniformity of convergence of series and products,
we shall have to consider uniformity of convergence of integrals and
also uniformity of continuity; thus a series is uniformly convergent
when \ R, z)\ <e, t( = 0 assuming integer values in- dependent of z
only.

Further, a function f z) is continuous in a closed region if, given e,
we can find a I ositive number r/ such that 1/(2 + 2) —fi ) \ <
whenever

0<\ C \ < r and 2 + f is a point of the region.

The function will be uniformly continuous if we can find a positive
number; inde- pendent of z, such that rjKr and \ f z + C)~f ) i <
whenever

0<UI<'7 and 2 + f is a point of the region, (in this case the set (f)o
is the set of points whose moduli are less than r)).

We shall find later (§ 3-61) that continuity involves uniformity of
continuity; this is in marked contradistinction to the fact that
convergence does not involve uniformity of convergence.

36. The modified Heine-Borel theorem.

The following theorem is of great importance in connexion with
properties of uniformity; we give a proof for a one-dimensional
closed region*.

Given (i) a straight line CD and (ii) a latv by which, corresponding
to each point f P of CD, we can determine a closed interval I P) of
CD, P being an interior point of I (P).

Tlien the line CD can be divided into a finite number of closed
intervals Ji, Jo, ... Jk, such that each interval Jr contains at least
one point not an end point) Pr, such that no point of Jr lies outside
the interval I (Pr) associated (by means of the given law) luitli that
point Pr§.

A closed interval of the nature just described will be called a
suitable interval, and will be said to satisfy condition A).

If CD satisfies condition A ), what is required is proved. If not,
bisect CD; if either or both of the intervals into which CD is
divided is not- suitable, bisect it or them||.

* A formal proof of the tlieorem for a two-dimensional region will be
found in Watson's Complex Integration and Cauchy s Theorem (Camb.
Math. Tracts, No. 15).

t Examples of such laws associating intervals with points will be
found in §§ 3'61, 5'13.

t Except when P is at C or D, when it is an end point.

§ This statement of the Heine-Borel theorem (which is sometimes called
the Borel-Lebesgue theorem) is due to Baker, Proc. London Math. Soc.
(2) i. (1904), p. 24. Hobson, The Theonj of Functions of a Real
Variable (1907), p. 87, points out that the theorem is practically
given in Goursat's proof of Cauchy's theorem Trans. American Math.
Soc. i. (1900), j). 14); the ordinary form of the Heine-Borel theorem
will be found in the treatise cited.

II A suitable interval is not to be bisected; for one of the parts
into which it is divided might not be suitable.



54 THE PROCESSES OF ANALYSIS [CHAP. Ill

This process of bisecting intervals which are not suitable either will
terminate or it will not. If it does terminate, the theorem is proved,
for CD will have been divided into suitable intervals.

Suppose that the process does not terminate; and let an interval,
which can be divided into suitable intervals by the process of
bisection just described, be said to satisfy condition (B).

Then, by hypothesis, CD does not satisfy condition B); therefore at
least one of the bisected portions of CD does not satisfy condition
B). Take that one which does not (if neither satisfies condition B)
take the left-hand one); bisect it and select that bisected part
which does not satisfy condition B). This process of bisection and
selection gives an unending sequence of intervals 5o, Si, S2, ... such
that :

(i) The length of s is 2-" Ci).

(ii) No point of s,i+i is outside Sn-

(iii) The interval s does not satisfy condition (-4).

Let the distances of the end points of s from G be Xn, yn\ then Xn <
a? +i < 2 i+i Hn- Therefore, by § 2*2, x and yn have limits; and, by
the condition (i) above, these limits are the same, say; let Q be the
point whose distance from C is . But, by hypothesis, there is a number
hq such that every point of CD, whose distance from Q is less than 5q,
is a point of the associated interval /(Q). Choose n so large that t
CDk 8q; then Q is an internal point or end point of Sn and the
distance of every point of Sn from Q is less than Sq. And therefore
the interval 5 satisfies condition (A), which is contrary to condition
(iii) above. The hypothesis that the process of bisecting intervals
does not terminate therefore involves a contradiction; therefore the
process does terminate and the theorem is proved.

In the two-dimensional form of the theorem* the interval CD is
replaced by a closed two-dimensional region, the interval I P) by a
circlet with centre P, and the interval Jj. by a square with sides
parallel to the axes.

3'61. Uniformity of continuity.

From the theorem just proved, it follows without difficulty that if a
function f(x) of a real variable x is continuous when a x b, then f(x)
is unifurmly continuous throughout the range a x b.

For let e be an arbitrary positive number; then, in virtue of the
con- tinuity of f x), corresponding to any value of x, we can find a
positive number S , depending on x, such that

1/( 0 -/( O I < e

for all values of x' such that \ x' — x\ < Sx-

* The reader will see that a proof may be constructed on similar lines
by drawing a square circumscribing the region and carrying out a
process of dividing squares into four equal squares.

t Or the portion of the circle which lies inside the region.

:J: This result is due to Heine; see Journal fiir Math. lxxi. (1870),
p. 361, and lxxiv. (1872), p. 188.



3'61, 3'62] CONTINUOUS functions and uniform convergence 55

Then by § 3"G we can divide the range (a, b) into a finite number of
closed intervals with the property that in each interval there is a
number Xi such

that \ fioc') — f xx) \ < -€, whenever x lies in the interval in which
a;, lies.

Let So be the length of the smallest of these intervals; and let f ,
|' be any two numbers in the closed range (a, 6) such that | — ' | <
o- Then , f ' lie in the same or in adjacent intervals; if they lie
in adjacent intervals let , be the common end point. Then we can find
numbers x , Xo, one in each interval, such that

\ f )-f .)\ < \ e, ,/(?o)-/(- 0 < f>

' /(r ) -/( e) \ < \ \ , fit) -fu i < 5 6,

so that

i/(i) -/(r) = , /(B -fM] - [fit) -fM]

- /(r)-/( -.) + /( o)-/( .) i

< 6.

If , ' lie in the same interval, we can prove similarly that

i/( )-/(r)i<2 -

In either case we have shewn that, for (iny number in the range, we
have

\ f( )-f +0 <e whenever + is in the range and —Bo< < Bq, where So is
independent of . The uniformity of the continuity is therefore
established.

Corollary (i). From the two-dimensional form of the theorem of § 3"6
we can prove that a function of a complex variable, continuous at all
jwints of a closed region of the Argand diagram, is uniformly
continuous throughout that region.

Corollary (ii). A function f x) which is continuous throughout the
range a x b is hounded in the range; that is to say we can find a
number k independent of x such that \ f x) ! <K for all points x in
the range.

[Let n be the number of parts into which the range is divided.

Let , i, 2> ••• In-ij be their end points ] then if x be any point of
the rth interval we can find numbers Xi, x-i, ... Xn such that

l/( )-/(- 'i)|< , l/(.i-i)-/'(li)|<i , l/( i)-/(' 2)|<ie, l/(. 2)-/(
2)|<if,...

••• \ f ryi)-f x)\ < h. Therefore \ f a)-f x) |< ire, and so

which is the required result, since the right-hand side is independent
of x."]

The corresponding theorem for functions of complex variables is left
to the reader.

3'62. A real function, of a real variable, continuous in a closed
interval, attains its upper bound.

Let f x) be a real continuous function of x when a x b. Form a section
in which the i?-class consists of those numbers r such that r >f x)



56 THE PROCESSES OF ANALYSIS [cHAP. Ill

for all values of x in the range (a, h), and the X-class of all other
numbers. This section defines a number a such that f x) a., but, if h
be any positive number, values of x in the range exist such that
f(x)>a — 8. Then a is called the upper bound oi f x); and the theorem
states that a number x' in the range can be found such thai f x) = a.

For, no matter how small h may be, we can find values of x for which
|/( ) — aj" >'8~; therefore | /(. ) - a| |~ is not bounded in the
range; therefore (§ 3'61 cor. (ii)) it is not continuous at some point
or points of the range; but since | f x) — a | is continuous at all
points of the range, its re- ciprocal is continuous at all points of
the range (§ 3*2 example) except those points at which f(x) = a',
therefore f x) = a at some point of the range; the theorem is
therefore proved.

Corollary (i). The lower bound of a continuous function may be defined
in a similar manner; and a continuous function attains its lower
bound.

Corollary (ii). If /( ) be a function of a complex variable continuous
in a closed region, | f z) \ attains its upper bound.

3'63. A real function, of a. real variable, continuous in a closed
interval, attains all values between its upper and loiuer bounds.

Let 31, m be the upper and lower bounds off x); then we can find
numbers X, -v, by § 362, such that/( ) = M,f x) = m; let //. be any
number such that m< fjb< M. Given any positive number e, we can (by §
3-61) divide the range (x, x) into 'A finite number, r, of closed
intervals such that

l/(.r,'-')-/(*2' ')|<6,

where a i""*, iCjC* are any points of the rth interval; take -i**"',
x./'' to be the end points of the interval; then there is at least
one of the intervals for which /(*'!<' ') - f ,f(x.J ) — /x have
opposite signs; and since

|[/( ,<'-')-/z - /( ,"-))- j|<6,

it follows that j /(aa""') — /ii\ < e.

Since we can find a number i'''* to satisfy this inequality for all
values of 6, no matter how small, the lower bound of the function \
f(x)-fA,\ is zero; since this is a continuous function of x, it
follows from § 3-62 cor. (i) that/( ) — yLt vanishes for some value of
. .

3'64. The fluctuation of a function of a real variable*.

Let/(;) be a real bounded function, defined when a x b. Let

a Xi X2 ... : Xn b. Then I /(a) -f(x,) \ + \ f(x,) -f(x,) | + ... +
l/C J -f(b) | is called the fluctuation oi f x) in the range (a, b)
for the set of subdivisions x , X2, ... Xn.

The terminology of tliis section is partly that of Hobson, The Theory
of Functions of a Real Variable (1907) and partly that of Young, The
Theory of Sets of Points (190(5).



3'63-37l] CONTINUOUS FUNCTIONS AND UNIFORM CONVERGENCE 57

If the fluctuation have an upper bound FJ*, independent of n, for all
choices of iCi, j-o, ... Xn, then f x) is said to have limited total
fluctuation in the range (a, h). Fa!' is called the total fluctuation
in the range.

Example 1. If f x) be monotonic* in the range (a, b), its total
fluctnation in the range is|/(a)-/(6)|. '>i

Example 2. A fimction with limited total fluctuation can be expressed
as the differ- ence of two positive increasing monotonic functions.

[These function.s may be taken to be | Fa' +fix) , h Fa'-f x) .]

Example 3. If f x) have limited total fluctuation in the range a, b),
then the limits f x±0) exist at all points in the interior of the
range. [See § 3*2 example.]

Example 4. li f x\ g x) have limited total fluctuation in the range
(a, b) so has f x)g x).

[For \ f x')g x')-f x)g x)\ \ \ f :>f). \ g x')-g x)\ + \ g ) \ f
')-fi. )l and so the total fluctuation of f x) g x) cannot exceed g .
FJ'+f. G K where g are the upper bounds of |/( ) |, \ g (x) \ .]

- 3'7. Uniformity of convergence of power series. Let the power
.series

ao + tti;-!- ... +an2"+ ••• '

converge absolutely when z = Zo.

Then, if | | | 'o | , , a " | | a Zo" | .

00 CO

But since S | anZj" ' converges, it follows, by § 3-34, that S ttn "
converges

w = ' 71 =

uniformly with regard to the variable z when \ z \ Zq.

Hence, by § 3"32, a power series is a continuous function of the
variable throughout the closed region formed by the interior and
boundary of any circle concentric with the circle of convergence and
of smaller radius (§ 2"6).

  3'71. Abel's theoreni-'r on continuity up to the circle of
convergence.

00

Let S a,j " be a po ver series, whose radius of convergence is unity,
and

M =

cc

let it be such that S a converges; and let a; 1; then Abel's theorem

>i =

(OC \ 00

S '*] = S an. . n=0 J M=0

For, with the notation of § 3-35, the function x satisfies the
conditions

00

laid on u, x), when 0 a; l; consequently /(a:;) = 2 On " converges
uni-

M =

* The function is monotonic if f x)-f x')\ \ x-x') is one-signed or
zero for all pairs of different values of .r and x' .

t Journal fiir Math. i. (1826), pp. 311-339, Theorem iv. Abel's proof
employs directlj' the arguments by which the theorems of § 3-32 and §
3-35 are proved. In the case when S | a I converges, the theorem is
obvious from § 3-7.



58 THE PROCESSES OF ANALYSIS [CHAP. Ill

formly throughout the range a;: 1; it is therefore, by § 3"32, a
continuous function of x throughout the range, and so lim f x)=f l),
which is the

a;-*l-0

theorem stated.

3'72. Abel's theorem* on multiplication of series.

This is a modification of the theorem of § 2"53 for absolutely
convergent series.

Let Cn = ttohn + i \& \ i + . . . + an 0 •

Then the convergence of " an, S bn and 2 c is a sufficient condition
that

M=0 n=0 w=0

 00 \ / 00 \ 00

For, let ,

 (a;)= 2 an B(x)= 2 6na ", C(a.')-= i CnX''. n=0 n=0 M=0

Then the series for A (x), B x), C(x) are absolutely convergent when I
I < 1, (§ 2-6); and consequently, by § 2-53,

A(x)B(x)=C x) when < < 1; therefore, by § 2-2 example 2,

  lim A x)] lim B(x) = lim C(x)\

a; .l-0 a;- -l-0 x -l-O

provided that these three limits exist; but, by § 3-71, these three
limits are

00 00 30

2 a , 'Z bn, % Cn', and the theorem is proved.

n=0 w = w=0

3'73. Power series which vanish identically.

If a convergent poiver seynes vanishes for all values of z such that \
z\ r- , where r > 0, then all the coefficients in the power series
vanish.

For, if not, let a be the first coefficient which does not vanish.

Then am + '7/t+i + m+2' "+ ... vanishes for all values of z (zero
excepted) and converges absolutely when \ z\ \ r<r; hence, if s = a
+i + a,n+2Z + . . ., we have

00

I S I Z, I (ljn-|-7i I T , >j = l

and so we can findf a positive number S r such that, whenever | | S,

I , 2 \ L I 1 ' I

I ( m+i + ( f"m+2 + • • • 1 2' '" I '

and then | a + s | | m | - ! •' ' > i a.m\, and so a, + s =f when \ z
\ < S.

* Journal filr Math. i. (1826), pp.;:ili-339, Theorem vi. This is
Abel's original proof. In some text-books a more elaborate proof, by
the use of Cesaro's sums (§ 8-43), is given.



372,373] CONTINUOUS FUNCTIONS AND UNIFORM CONVERGENCE 59

We have therefore arrived at a contradiction by supposing that some
coefficient does not vanish. Therefore all the coefficients vanish.

Corollary 1. "We may 'equate corresponding coefficients' in two power
series whose sums are equal throughout the region 1 |<S, where 8>0.

Corollary 2. We may also equate coefficients in two power series which
are proved equal only when z is real.



REFERENCES.

T. J. 1'a. Bromwich, Theory of Infinite Series (1908), Ch. vii.

E. GouRSAT, Cours d'Analyse (Paris, 1910, 1911), Chs. i, xiv.

C J. DE LA Vall e Poussin (Louvain and Paris, 1914), Cours d Analyse
Infinitesimale, Introduction and Ch. viii.

G. H. Hardy, A course of Pure Mathematics (1914), Ch. v.

VV. F. Osgood, Lehrbiich der Funktionentheorie (Leipzig, 1912), Chs.
ii, iii.

G. N. Watson, Complex Integration and Cauchy's Theorem (Camb. Math.
Tracts, No. 15), (1914), Chs. I, II.



Miscellaneous Examples.

1. Shew that the series



 =l(l-2 )(l-2 l)

is equal to ,-, —. when 1 2 I < 1 and is equal to ,, -, when 1 2 1 >
1.

  zy z z)-

Is this fact connected with the theory of uniform convergence ?

2. Shew that the series

2sini + 4sinl + ... + 2"sin + ...

converges absolutely for all values of z z = excepted), but does not
converge uniformly near 2=0.

3. If Un x)=-2 n-\ f .re-'"-''' ' + n xe''' '''',

shew that 2 ?< x) does not converge uniformly near x=0. (Math. Trip.,
1907.)

n = \

4. Shew that the series —pr j- + —r —... is convergent, but that its
square (formed

by Abel's rule)

T 2+1,73 + 2; U'4' vW -

is divergent.

5. If the convergent series 5=— — — +—-—+... (r>0) be multiplied by
itsel,

the teiTus of the product being arranged as in Abel's result, shew
that the resulting series diverges if r | but converges to the sum s \
i r>. (Cauchy and Cajori.)



60 THE PROCESSES OF ANALYSIS [CHAP. Ill

6. If the two conditionally convergent series

2 - and 2



n=i W =i n'

where r and s lie between and 1, be multiplied together, and the
product an-anged as in Abel's result, shew that the necessary and
sufficient condition for the convergence of the resulting series is r
+ s >1. (Cajori.)

7. Shew that if the series 1 - 3 + 5 - t + • • •

be multiplied by itself any number of times, the terms of the product
being arranged as in Abel's result, the resulting series converges.
(Cajori.)

8. Shew that the qth. power of the series

 ! sin d+a sin 2 + ... +a,i sin n6 + ... is convergent whenever 5- (1
- r)< 1, r being the greatest number satisfying the relation

for all values of n.

9. Shew that if 6 is not equal to or a multiple of 2n, and if %0) Ui,
u , ...he a sequence such that u - O steadily, then the series 2i/ cos
nd + a) is convergent.

Shew also that, if the limit of t< is not zero, but ?i is still
monotonic, the sum of the

S

series is oscillatory if - is rational, but that, if - is irrational,
the sum may have any value

TT TT

between certain bounds whose difference is a cosec 6, where a= lim u -

(Math. Trip., 1896.)


%
% 61
%
\chapter{The Theory of Riemann Integration} 

\Section{4}{1}{The concept of integration.}

The reader is doubtless familiar with the idea of integration as the
operation inverse to that of differentiation; and he is equally well
aware that the integral (in this sense) of a given elementary function
is not always expressible in terms of elementary functions. In order
therefore to give a definition of the integral of a function which
shall be always available, even though it is not practicable to obtain
a function of which the given function is the differential
coefficient, we have recourse to the result that the integral* o f x)
between the limits a and h is the area bounded by the curve y =f(x),
the axis of cc and the ordinates x = a, x = b. We proceed to frame a
formal definition of integration with this idea as the starting-point.

\Subsection{4}{1}{1}{Upper and lower integrals'f.}

Let f(x) be a bounded function of x in the range a, b). Divide the
interval at the points Xi,Xo, ... Xn-iia x - x ... Xn-i b). Let U, L
be the bounds of /(a;) in the range (a, b), and let Ur, L be the
bounds of f(x) in the range (xr-i, Xr), where Xq = a, Xn=b.

Consider the sums:|:

Sn = U, (iCi - a) +Uo Xn-X,)+ ...+ Un (b - Xa-,), Sn = L, X - a) + Z/2
( 2 - i) +    + Ln (b - Xn-i).

Then U(b- a) Sn s., L b- a).

For a given n, Sn and s are bounded functions of x, x, ... n-i- Let
their lower and upper bounds § respectively be S, Sn, so that Sn, s
depend only on n and on the form of f x), and not on the particular
way of dividing the interval into n parts.

* Defined as the (elementary) function whose differential coefficient
is/(x).

t The following procedure for establishing existence theorems
concerning integrals is based on that given by Goursat, Cours d'
Analyse, i. Ch. iv. The concepts of upper and lower integrals are due
to Darboux, Ann. de I'Ecole norm. sup. (2) iv. (1875), p. 64.

+ The reader will find a figure of great assistance in following the
argument of this section. Sn and s represent the sums of the areas of
a number of rectangles which are respectively greater and less than
the area bounded by y=f(x), x-a, x-h and ?/ = 0, if this area be
assumed to exist.

§ The bounds of a function of n variables are defined in just the same
manner as the bounds of a function of a single variable \hardsubsectionref{3}{6}{2}).

%
% 62
%

Let the lower and upper bounds of these functions of n be S, s. Then

Sn > S, Sn S.

We proceed to shew that s is at most equal to S; i.e. S' s.

Let the intervals (a, x ), x-, x., ... be divided into smaller
intervals by new points of subdivision, and let

a,y,y, ... yu-i, yk = i), yk+i,  yi-i, yi = 2), yi+1,  Vm-i, h be
the end points of the smaller intervals; let U L,.' be the bounds of
/( ) in the interval y,-i, yr)-

m m

Let T,, = X yr- yr-i) UJ, t,n = t y,. - yr-i) L,!.

r=\ r=l

Since Ui, U.2, ... Uk do not exceed JJi, it follows without difficulty
that

Now consider the subdivision of (a, b) into intervals by the points
x-y, X2, ... Xn-\, and also the subdivision by a different set of
points a? x, ... x'n'-i- Let S'n',s'n' be the sums for the second
kind of sub- division which correspond to the sums *S, Sn for the
first kind of subdivision. Take all the points x, ... Xn-i] x, ...
x'n'\ y as the points y, y., ... y .

Then 8n T, t,, Sn,

and S'n' T n tm s'n' .

Hence every expression of the type Sn exceeds (or at least equals)
every expression of the type s'n'; and therefore S cannot be less
than s.

[For \ i S<s and s - S - 27] we could find an Sn and an s'n' such that
g - S<r], s - s'n'<'n and so s'n'>Sn, which is impossible.]

The bound S is called the upper integral off(x), and is written 1 f(x)
dx;

J a

s is called the lower integral, and written I / x) dx.

J a

If S = s, their common value is called the integral of f(x) taken
between the limits* of integration a and b.

The integral is written I f x)dx. .

ra rb

We define | f x)dx, when a< b, to mean - I f(x)dx. Exam'ple 1, I /
(*') + (*-') dx = \ f(x) dx+ l (x) dx.

J a J a J a

Example 2. By means of example 1, define the integral of a continuous
complex function of a real variable.

* ' Extreme values ' would be a more appropriate term but ' limits '
has the sanction of custom. 'Termini' has been suggested by Lamb,
Infinitesimal Calculus (1897), p. 207.

%
% 63
%

\Subsection{4}{1}{2}{Riemann's condition of integrability*.}

A function is said to be ' integrable in the sense of Riemann ' if
(with the notation of\hardsubsectionref{4}{1}{1}) Sn and s,i have a common limit (called
the Riemann integral of the function) when the number of intervals
x,.\ -, x ) tends to infinity in such a way that the length of the
longest of them tends to zero.

The necessa7-y and sufficient condition that a hounded function should
he integrahle is that S - Sn should teiid to zero luhen the numher of
intervals (xr-i, Xr) tends to infinity in such a way that the length
of the longest tends to zero.

The condition is obviously necessary, for if S and s,i have a common
limit n - Sn - ► as 71 - > 30 . And it is sufficient; for, since Sn
S' s s,i, it follows that if lim (Sn - Sn) = 0, then

lim *S' = lim s = S = s.

Note. A continuous function f(x) is 'integrable.' For, given e, we can
find 8 such that \ f(af) - f x")\ < (l b-a) whenever \ x' - x"\ < 8.
Take all the intervals Xg\ i, x-g) less than 8, and then Ug- Lg<€/ b -
a) and so >S' - s <e; therefore <S' - s - -0 under the circumstances
.specified in the condition of integrahility.

Corollary. If <S' and s have the same limit S for one mode of
subdivision of (a, h) into intervals of the specified kind, the limits
of .S' and of s for any other such mode of subdivision are both 8.

Example I. The product of two integi'able functions is an integrable
function.

Example 2. A function which is continuous except at a finite number of
ordinary discontinuities is integrable.

\ li f(x) have an ordinary discontinuity at c, enclose c in an
interval of length S,; given f, we can find 8 so that j f x')-f x) \
< e when i x' -x \ <8 and x, x' are not in this interval.

Then *S' -s,j f (6-a-8i) + /'Si, where k is the greatest value of \ f
x')-f x)\, when X, x' lie in the interval.

When Si- 0, y(-- |/(c + 0)-/(c-0), and hence lim (.S' -s )=0.]

 -*

Example 3. A function with limited total fluctuation and a finite
number of ordinary discontinuities is integrable. (See\hardsubsectionref{3}{6}{4} example
2.)

\Subsection{4}{1}{3}{A general theorem on integration.}

Let /(a;) be integrable, and let e be any positive number. Then it is
possible to choose S so that

n rh

t xp - Xj,\ i)f(x'p\ i) - f x)dx < 6,

p=\ J a

provided that Xy - Xp\ i- h, Xp\ i x'p\ i%Xp.

* Biemann (Ges. Math. Werke, p. 239) bases his definition of an
integral on the limit of the sum occurring in § -i-lS; but it is then
difficult to prove the uniqueness of the limit. A more general
definition of integration (which is of very great importance in the
modern theory of Functions of Real Variables) has been given by
Lebesgue, Annali di Mat. (3) vn. (1902), pp. 231-359. See also his
Lecons sur V integration (Paris, 1904).

%
% 64
%

To prove the theorem we observe that, given e, we can choose the
length of the longest interval, B, so small that S - Sn < e.

w

Also Sn> S ( - \ i)/(a?' -i) Sn,

p = l

  a

Therefore

  rb

2 xp - Xp\ )f(a;'p\,) - f(x) dx

9 = 1 J a

 s.,

  f-Jn ' n

< 6.

As an example* of the evaluation of a definite integral directly from
the theorem of this section consider I " - j, where X<1.

Jo (1- 2)2

Take S= - arc sin X and let,?,= sin s8, (0 <s8 <h tt), so that

 s+i-A's=2 sin |S cos (s+ ) 8< S;

also let Xg = sin (s + i) 8.

, P . c - - s-i sin sS - sin (5 - 1)S

Then 2 - '- = 2 \,.

s=i(i\ y2 \ j)i .=1 cos(s-A)S

= 2/9 sin 2 S

= arc sin X. sin |S/(JS) .

By taking p sufficiently large we can make

P dx I Xg-Xs-i

Jo (l\,,.2)i,s=l(l\ . '2 \ j)4

arbitrarily small.

We can also make arc sin X . < -j-| 1

arbitrarily small.

That is, given an arbitrary number f, we can make

P dx

/ i~

<e

arc sin X

by taking p sufficiently large. But the expression now under
consideration does not

depend on p; and therefore it must be zero; for if not we could take
c to be less than it,

and we should have a contradiction.

rx f g That is to say I '- - =arc sin X.

Jo (i\ .' )2

Example 1. Shew that

X 2x (n - l)x

I+COS- + COS f-...+cos -

,. 71 n n sm

Iim - - - . - - - - - - = .

n-*-'x> *'' -

Example 2. If f(x) has ordinary discontinuities at the points aj, 02,
..., then

fb ( fu -S, fa.,-S., [b 1

f x)dx = \ \ m\ \ + +...+ f x)dx\,

J a \ J a J a, +6, J ax + <c J

where the limit is taken by making 81, S2, ... §, ei, t i  f* tend
to +0 independently. * Netto, Zeitschriftfilr Math, und Phys. xl.
(1895).

%
% 65
%

Example 3. If /( ) is integrable when i x i and if, when Oj a < 6 < 6i
, we write

/ f x)dx = <i> a, b), and if/(6 + 0) exists, then

lim < (,\& + fi)-> K\&) ( o)

Deduce that, i f(x) is continuous at a and b,

d\

da

jjix) dx= -f a), -I jjix) dx=f b). Bxample 4. Prove by differentiation
that, if (f> x) is a continuous function of ./; and

dx

-y- a continuous function of t, then

at

fxi l ft fir

 < x)dx=\ \ x)'j dt.

Example 5. If /' x) and < ' x) are continuous when a x b, shew from
example 3 that

r / (x) <i> o;) dx + J'l 4>' x)f x) dx=f b) < (b) -f a) cf> (a).

Example 6. If/(.r) is integrable in the range (a, c) and 6 c, shew
that I f(x) dx

J a is a continuous function of b.

\Subsection{4}{1}{4}{J/c ?i Fa we Theorems.}

The two following general theorems are frequently useful.

(I) Let U and L be the upper and lower bounds of the integrable
function /(.r) in the range (a, b).

Then from the definition of an integral it is obvious that

J' U-f x)] dx, j ' fj(x) L) dx

are not negative; and so

U b-a)- l f x)dx L b-a).

This is known as the First Mean Value Theorem.

li' f(x) is contimious we can find a number | .such that a- b and such
that/( ) has any given value lying between U and L \hardsubsectionref{3}{6}{3}). Therefore
we can find | such that

rf x)dx = b-a)f \$).

J a

If F(x) has a continuous differential coefficient F' (x) in the range
(a, 6), we have, on

writing F' (x) for f(x),

F b)-F d) = b-a)F' )

for some value of such that a b.

Example. lif x) is continuous and ( x)' 0, shew that can be found such
that

' fix) cf> (x) dx =f (I) / % (x) dx.

a J a.

W. M. A.

%
% 66
%

(11) Let /(.? ) and 4> .v) be integrable in the range (o, b) and let
(a-) be a positive decreasing function of .r. Then Bonnefs* form of
the Second Mean Value Theorem is that a number exists such that a | 6,
and

 " f x)ci> x)dx 4> a) \ \ f x)dx. \ y'

J a J a

For, with the notation of §§ 4'1-4'13, consider the sum

p ♦ .;S'= 2 Xs-x,\ i)f x,\ )(i> x,\ ).

s=l

Writing x - x y) f x,\ i) = a,\ i, Xs-i) = 4>s-\, o + i + --- + 08 =
s, 'e have

Each term in the summation is increased by writing b for 6g\ i and
decreased by writing b for ftg\ i, if b, b be the greatest and least
of 6o, 6i, ... 6p\ i; and so b(j)(, S b(Po-

m

Therefore S lies between the greatest and least of the sums ( ) xq) 2
(xg-Xg\ i)f Xg\ i)

s=l

where m = l, 2, 3, ... p. But, given e, we can find 8 such that, when
Xg-Xg\ i<d,

p f p I

2 x, - x,\ i)f Xs i) (t> (.r,\ i) - I f(x) (f) (x) dx < e,

S=l J 0 I

m Cxra [

< (:ro) 2 X, -Xs-i)/ (X, \ i) - < ( o) / / ( ) < *- < f, s=l a I

and so, writing a, b for a'q, . p, we find that / f x)(j> x)dx lies
between the upper and

J a'

lower bounds ott < (a) I ' f .v)d.v±2e, where j may take all values
between a and ?>.

Let C and L be the upper and lower bounds of </> (a) j f(x) dx.

J a

fh

Then U+ 2e I /'(.r) < ( 0 dx' L-2e for a jjositive values of e;
therefore

r I f (x) cfi (x) dx L.

Since (j) a) I \ f(x) dx qua function of j takes all values between
its upper and lower J

bounds, there is some value, .say, of |i for which it is equal to I f
(x) (f> (x) dx. This proves the Second Mean Value Theorem.

E.rarnple. By writing ( x) -(f> b)\ in place of (f) (.v) in Bonnet's
form of the mean value theorem, .shew that if < (x) is a monotonic
function, then a number | exists such that a \$ b and

\ f x)(l> x)dx = 4> a) j f x)dx + 4) b) j f x)dx.

\addexamplecitation{Du Bois Reymond.}

* Journal de Math. xiv. (1849), p. 249. The proof given is a modified
form of an iuvestigatiou due to Holder, Gdtt. Nach. (1889), pp. 38-47.

+ By § 413 example 6, since /(.r) is bounded, I ' f(x) d.v is a
continuous function of fj.

%
% 67
%

\Section{4}{2}{Differentiation of integrals containing a parameter.}

The equation* f x, a)dx=\ dx is true if f(x, a) possesses a

ace J (I (I vCL

Riemann integral with respect to x and fa. = \ is a continuous
function of hoth-f the variables x and a.

For I- '/(., ) <fa = lim f .Aj;.\ ° + A)-/( . )

eta j a h-f-O .' a h

if this limit exists. But, by the first mean value theorem, since / is
a continuous function of a, the second integrand is fa x, a + 6h),
where

But, for any given e, a number 8 independent of x exists (since the
con- tinuity of fa is uniform]: with respect to the variable x) such
that

\ fa x, a) -fa (x, a) I < e/(6 - a), whenever | a' - a | < S.

Taking j A | < S we see that ! 6h | < 8, and so whenever \ h < 8,

[H x, a + h) -fix, a) J i *,,,, [ ., m x w m

 - -! y - J \ ' / dx- \ /a (, a) dx I fa (x, a + Oh) - / (x, a) \ dx

< e.

Therefore by the definition of a limit of a function \hardsectionref{3}{2}), lim
i'f(-. + h)-f(, .a)

h O J a h

I"*

exists and is equal to fadx.

J a

Example 1. If a, b be not constant.s but functions of a with
continuous differential coefficients, shew that

 |'/(.- a)d.v=f b, a) -/(a, ) +/ fj:.:

Example 2. If /(.r, a) is a continuous function of both variables, / f
x, a)dx is a

J a continuous function of a.

* This formula was given by Leibniz, without specifying the
restrictions laid on/(.r, a).

t (p x, y) is defined to be a continuous function of both variables
if, given e, we can find 5 such that | 4> x', y') - (/> (x, y)\ < €
whenever (x' - x)' + (y' - y)' -<S. It can be shewn by\hardsectionref{3}{6} that if
(x, y) is a continuous function of both variables at all points of a
closed region in a Cartesian diagram, it is uniformly continuous
throughout the region (the proof is almost identical with that of §
3-61). It should be noticed that, if (.r, y) is a continuous function
of each variable, it is not necessarily a continuous function of both
; as an example take

 [x,y)=. p!, 0(0,0) 1;

this is a continuous function of x and of y at (0, 0), but not of both
x and y.

X It is obvious that it would have been sufficient to assume that /
had a Riemann integral and was a continuous function of a (the
continuity being uniform with respect to x), instead of assuming that
/ was a continuous function of both variables. This is actually done
by Hobson, FuJictions of a Real Variable, p. 599.

5-2

%
% 68
%

\Section{4}{3}{Double integrals and repeated integrals.}

'Letf x, y) be a function which is continuous with regard to both of
the variables x and y, when a x h, a y - /3,

By\hardsectionref{4}{2} example 2 it is clear that

j 1 1 /( ' y) dy\ dx, j U J x, y) dx\ dy both exist. These are called
repeated integrals.

Also, as in\hardsubsectionref{3}{6}{2}, f(x, y), being a continuous function of both
variables, attains its upper and lower bounds.

Consider the range of values of x and y to be the points inside and on
a rectangle in a Cartesian diagram; divide it into nv rectangles b -
lines parallel to the axes.

Let Z7 i,, L,n, be the upper and lower bounds of f x, y) in one of
the smaller rectangles whose area is, say, Ayn,,j.', and let

71 V n V

- w - m,n -"-m,!!- n,v y -i - j,n - m,/u h, I'- j/t = 1 jj. = 1 ) = 1
;u. = 1

Then *S', >?, \,, and, as in\hardsubsectionref{4}{1}{1}, we can find numbers h,.-, s,,,
which are the lower and upper bounds of Sn,v, V" respectively, the
values of Sn,v, Sn,v depending only on the number of the rectangles
and not on their shapes; and n, s, . We then find the lower and upper
bounds S and s) respectively of,, Sn,v qua functions of n and v;
and S v S s s,, as in §411.

Also, from the uniformity of the continuity of f(x, y), given e, we
can find B such that

'J m,iJ. - Hij/i '" f>

(for all values of m and /i) whenever the sides of all the small
rectangles are less than the number h which depends only on the form
of the function f x, y) and on e.

And then >S', - Sa, < e (6 - a) (/3 - a),

and so S - s < e h - a) (/3 - a).

But S and s are independent of e, and so S = s.

The common value of S and s is called the double integral of f x, y)
and is written

f(x, y) (dxdy).

It is easy to shew that the reijeated integrals and the double
integral are all equal when f :c, y) is a continuous function of both
variables.

%
% 69
%

For let Y j, A, be the uppei- and lower bounds of

as .V varies between x -i and a:, .

Then 2 Y, (x, - x, -i) > \ f x,y)dy\ dx A, (a;,,, - x,,,\ .

"1 = 1 <' \ 3 a. ) i = l

But* 2 Ura,,.i y.-y,.-i) \,n .\ n 2 Z,,m (y/x - m-i)-

M = l /n = l

Multiplying these last inequalities by x -Xm x, using the preceding
inequalities and summing, we get

2 2 r .M,. / ]/ f x,y)dy\ dx 2 2 L,, A,;

 ( = 1 pi = l J a \ J a. ) m = l /x = l

and so, proceeding to the limit,

'S' £ [jy x,y)dy dx s.

But ' =s=l lfix,y) dxdy),

and so one of the repeated integrals is equal to the double integral.
Similarly the other repeated integral is equal to the double integral.

Corollary. If/(.v, y) be a continuous function of both variables,

\Section{4}{4}{Infinite integrals.}

If lini 1 /(.r)c?j;j exists, we denote it by f x)dx; and the limit in
question is called an infinite integral;. Examples.

,. r d'x,. (I \ \ 1

  j ( ' + a')' " b V 2 (62 + a') + 2a2y' 2a '

(3) By integrating by parts, shew that / t"e~ dt = n. \addexamplecitation{Euler.}

J

Similarly we define / f(x)dx to mean lim / f(x)dx, if this limit
exists; and

J -X a- - -o J a

/f x)dx is defined as / f(x)dx+l f .v)dx. In this last definition the
choice -00 J -<x>' J a

of a is a matter of indifference.

* The upper bound of f x, y) in the rectangle --i,, is not less than
the upper bound of /(.r, y) on that portion of the line .r = | which
lies in the rectangle.

t This phrase, due to Hardy, Proc. London Math. Soc. xxxiv. (1902), p.
16, suggests the analogy between an infinite integral and an infinite
series.

%
% 70
%

\Subsection{4}{4}{1}{Infinite integrals of continuous functions. Conditions for convergence.}

A necessary and sufficient condition for the convergence of f x)dx is

J a

that, corresponding to any positive number e, a positive number X
should exist such that f(x) dec \ < e whenever

The condition is obviously necessary; to prove that it is sufficient,
suppose

ra+n

it is satisfied; then, ii n X -a and n be a positive integer and Sn =
f(so),

. a

we have j Sn+p - Sn\ < €.

Hence, by\hardsubsectionref{2}{2}{2}, Sn tends to a limit, S; and then, if > a + n,

S-i f x)da;\ \ S-\'' ''f(a;)dx + t f(x)da;\

J a ' -a \ J a+n I

<26;

and so lim f(x) dx = S; so that the condition is sufficient.

\Subsection{4}{4}{2}{Uniformity of convergence of an infinite integral.}

The integral f x, a) dx is said to converge uniformly with regard to a

J a

in a given domain of values of a if, corresponding to an arbitrary
positive number e, there exists a number X independent of a such that

J /( > a)dx\ \ <€

for all values of a in the domain and all values of x X.

The reader will see without difficulty on comparing §§ 2'22 and 3'31
with\hardsubsectionref{4}{4}{1} that a necessary and sufficient condition that f x, a) dx
should

.' a

converge uniformly in a given domain is that, corresponding to any
positive number e, there exists a number X independent of a such that

I f(x, a)dx \ < e

\ J x' I

for all values of a in the domain whenever x" x X.

\Subsection{4}{4}{3}{Tests for the convergence of an infinite integral.}

There are conditions for the convergence of an infinite integral
analogous to those given in Chapter II for the convergence of an
infinite series.

The following tests are of special importance.

%
% 71
%

(I) Absolutely convergent integrals. It may be shewn that f x)dx

J a

certainly converges if \ f(a;) | dx does so; and the former integral
is then said to be absolutely convergent. The proof is similar to that
of\hardsubsectionref{2}{3}{2}.

Example. The comparison test. If \ f(x) g x) and / g x) dx converges,
then / f(x) dx converges absolutely.

[Note. It was observed by Dirichlet* that it is not necessary for the
convergence of I f x)dx that f x)-a'0 as x- cc : the reader may see
this by considering the function

/( ) = ( n x n + l- n + l)-' ),

f(x) = n + iyin + l-x) x- n+l) + n+l)- n + l -(n + l)- x n + l),

where n takes all integral values.

For / f(x)dx increa.'ses with and / f x)dx=l n+l)-; whence it follows

without difficulty that / f x)dx converges. But when a- = n + l -i
(?n-l)-2, y'(.t>) =; and so f(x) does not tend to zero.]

(II) The Maclaurin-Cauchyf test. Tf/(A-)>0 and/(x')-*0 steadily,

TODO

f x) dx and S /( ) converge or diverge together. 1 M = l

fm + l

For A -- /( 0 > fix) dx f m + 1 ),

J m n fn+] n+1

and SO 2 f(m) l f x)dx' 2 /(m).

  m = l J 1 wi=2

The first inequality shews that, if the series converges, the
increasing sequence / f x)dx converges \hardsectionref{2}{2}) when - -oo through
integral values, and hence it follows

fx'

without difficulty that / f(.v)dx converges when .r'-*-x; also if the
integral diverges, so does the series.

The second shews that if the series diverges so does the integral, and
if the integral converges so does the series \hardsectionref{2}{2}).

(III) Bertrand'sX test. 1 f(x) = 0 x ~' ), f x)dx converges when X <;
and \ if x) - x~' loga; " ), | f(x) dx converges when X, < 0.

. a

These results are particular cases of the comparison test given in
(I).

* Dirichlet's example was/(.r) = sin .r'-; Journal fiir Math. xvii.
(1837), p. 60. t Maclaurin Flit.vions, i. pp. 289, 290) makes a verbal
statement practically equivalent to this result. Cauchy's result is
given in his Oeuvrcs (2), vii. p. 269. X Journal de Math. vii. (1842),
pp. 38, 39.

%
% 72
%

(IV) Chartiers test for integrals involving periodic functions. If /(
) - steadily as x and if < x) dx is bounded as x <x>,

1 ' a

then f(x) (x) dx is convergent.

J a

For if the upper bound of I (.r) dx \ he A, we can choose X such that
f x)<e/2A

\ J a

when .X > A'; and then by the second mean vahie theorem, when .v" .v'
A', we have I /" " f(x) 6 (x) dx =\ f x') f(x) dx =f x') \ (f) x)dx-
cf) x) dx 2Af x') < f,

I \ / a;'  I J x' \ J a J a

which is the condition for convergence.

I ** Sill

Example I. I dt' converges.

J *'

Example 2. I a; - 1 sin x - ax) dx converges.

\Subsubsection{4}{4}{3}{1}{Tests for uniformity of convergence of an infinite integral f.}

(I) De la Vallee Poussins test . The reader will easily see by using

("00

the reasoning of\hardsubsectionref{3}{3}{4} that f x, a) dx converges uniformly with
regard

to a in a domain of values of a if \ f x, a) | < fi x), where fM x) is
independent

fee r "

of a and /jl (x) dx converges. [For, choosing X so that fx(x)dx<e

rx" when x' x' X, we have f x, a)dx < e, and the choice of X is inde-

J x'

pendent of a.]

/oo

Example. \ x' ~' e~''dx converges uniformly in any interval A, B) such
that

(II) The method of change of variable. This may be ilkist rated by an
example.

Consider / '- dx where a is real.

y " sin ax, /" " sin y,

We have / - 7- ' =, ~ 7~ 

] 3c' X J ax' y

   dy converges we can find Y such that / - - dy <e when y" y' Y.

y J y y

dx

< 6 whenever | a ' | F; if | a | S > 0, we therefore get

I /"*" sin ax, \ I dx \ < f

I y a;' -' I

* Journal de Math, xviii. (1853), pp. 201-212. It is remarkable that
this test for conditionally convergent integrals should have been
given some years before formal definitions of absolutely convergent
integrals.

t The results of this section and of\hardsubsectionref{4}{4}{4} are due to de la Valine
Poussin, Ann. de la Soc. Scientifique de Bruxelles, xvi. (1892), pp.
150-180.

X This name is due to Osgood.

%
% 73
%

when .'. " .// X= Y/8; and this choice of X is independent of a. So
the convergence is uniform when a S > and 'when a - 8 < 0.

Example. I j/ sm \& a:: )d > dx is uniformly convergent in any range
of real values of a. (de la Vallee Poussin.)

2- i sin zdz does not exceed a constant inde-

!

l endent uf a and .v since / z-i sin z dz converges.] J

(III) T/ie method of integration by parts.

If / / (x, a) dx <p x, a)+ X (* > ) d-''

and if ( f.r, a)-*-0 uniformly as x -X3 and /;( (.r, a)o?jp converges
uniformly with regard

J <t

to a, then obviously / f x, a) dx converges uniformly with regard to
a.

(IV) The method of decomposition.

Example. |J c\ os.r sin a |J sin (a +l) . |J sin (o - l)a. .

loth of the latter integrals converge uniformly in any closed domain
of real values of a from which the points a= ± 1 are excluded.

\Subsection{4}{4}{4}{Theorems concerning uniformly convergent infinite integrals.}
(I)
Let f x, a) dx converge uniformly luhen a lies in a domain S.

. a

'Then, if f x, a) is a continuous function of both variables ivJien x
a and a lies in S, f x, a)dx is a continuous function* of a.

J a

I r*

For, given e, we can find X independent of a, such that ' I f(x, a)dx
<e whenever X.

Also we can find 8 independent of x and a, such that \ f x,a)-f(x,a')\
< el X-a) whenever a - a.' < B.

That is to say, given e, we can find 8 independent of a, such that

f x,a.')dx-\ f x,a)dx \$ f x,a)-f x,a')]dx\

. a J a \ J a 1

+ I f x, a') dx +\ I fix, a) dx

\ Jx \ Jx

<3e,

whenever | a' - a | < S; and this is the condition for continuity.

* This result is due to Stokee. His statement is that the integral is
a continuous function of a if it does not ' converge infinitely
slowly.'

5

%
% 74
%

(II) If f x, a) satisfies the same conditions as in (I), and if a.
lies in S when A <a<B, then

I \ f x, (x)dx\ doi= \ fix, a)da[dx.

For, by\hardsectionref{4}{3},

Therefore

If

A [J a £ A

f x, a) dx r da= \ \ \ f(x, a) day dx. \ I f x, a) dx[ da- I \ i f x,
a) da)- dx

< f eda<e B-A),

J A

for all sufficiently large values of .

But, from §§ 2'1 and 4"41, this is the condition that

lim I \ i fix, a)da\ dx

should exist, and be equal to

f x, a) dx\ da. Corollary. The equation -r- \ rb .v, a)dx=l ~ dx is
true if the integral on the

da J a . J a va

right converges uniformly and the integrand is a continuous function
of both variables, when x' a and a lies in a domain >S', and if the
integral on the left is convergent.

Let A be a point of S, and let S=f x, a), so that, by\hardsubsectionref{4}{1}{3} example
3,

va

/ f x, a) da = (.r, a) - x, A).

Then / J / /(.r, a) day dx converges, that is / 4> x, a)-( ) x, A) dx
converges,

and therefore, since / cf) x, a)dx converges, so does i x, A) dx. J a
J a

I (f) x, a) dx \=j- / 0 (*-', n) - (- j -'1 ) 

Then

da

d da

/ \ j f x,a)daydx\

= T I \ l f(x.a)dx\ da daj A [J a- ' ' J

= fjix,a)dx=fy dx, which is the required result; the change of the
order of the integrations has been justified above, and the
differentiation of / with regard to a is justified by\hardsubsectionref{4}{4}{4} (I) and §
4-13 example 3.

%
% 75
%

\Section{4}{5}{Imjiroper integrals. Principal values.}

If I /(x) - >cc as X - a + 0, then lim f(x) dx may exist, abd is

i- + O J a+\&

written simply I f(x) dx; this limit is called an improper integral.

J a

If \ f(x) I - 00 as a; - > c, where a< c <b, then

/e-5 rb

lim I /( ) c?j: + lim I /(- O c?

S +O J a S' +oJ C+\&'

may exist; this is also written I f(x)dx, and is also called an
improper

J a

integral; it might however happen that neither of these limits exists
when 8, S' - > independently, but

lim ]/ f x)dx+i f(x)dxy exists; this is called 'Cauchy's principal
value of I f x)dx' and is written

J a

for brevity P I f(x) dx.

J a

Results similar to those of §§ 4-4-4-44 may be obtained for improper
integrals. But all that is required in practice is (i) the idea of
absolute convergence, (ii) the analogue of Bertrand's test for
convergence, (iii) the analogue of de la Vallee Poussin's test for
uniformity of convergence. The construction of these is left to the
reader, as is also the consideration of integrals in which the
integrand has an infinite limit at more than one point of the range of
integration*.

Examples. (1) / x - cos .v <ilr is an improper integral. J

- b

(2) r ./" (1 -.rf " dx is an improper integral if <X < 1, </i< 1. It
does not converge for negative values of X and /x.

dx is the principal value of an improper mtegral when

1 -A'

0<a<l. . 4-51. The inversion of the order of integration of a certain
repeated integral. ?r General conditions for the legitimacy of
inverting the order of integration when the

integrand is not continuous are difficult to obtain.

The following is a good example of the difficulties to be overcome in
inverting the order of integration in a repeated improper integral.

* For a detailed discussion of improper integrals, the reader is
referred either to Hobson's or to Pierpont's Functions of a Real
Variable. The connexion between infinite integrals and improper
integrals is exhibited by Bromwich, Infinite Series, § 164.

%
% 76
%

Let f x,y) he a continuous function of both variables, and let 0<X 1,
0</x l, < v < 1; then

This integral, which was first employed by Dirichlet, is of
importance in the theory of integral equations; the investigation
which we shall give is due to W. A. Hurwitz*.

Let x''~ i/' ~ (1 -x-yy~' f x,y) = (li x,y); and let M be the upper
bound of \ f x,y) |. Let S be any positive number less than .

Draw the triangle whose sides ave x = b, y = b, x+y = l-b\ at all
points on and inside this triangle ( x, y) is continuous, and hence,
by\hardsectionref{4}{3} corollary,

Now r~''dx 11 "'' ct> X, y) dy =l'~' dx |P"' < (, V) l + f]"' hdx+j'
'* Ldx,

where /i = / </> x, y) dy, L= (f> x, y) dy.

Jo J i-a;-6

But I /i I < r i/.r - y - 1 (1 - .r - y)" - 1 o y

since (l- -?/r-i<(l- '-S)''- .

Therefore, writing x = i\ -S)a'i, we havet

T"" /i dr I J/S',x- 1 p~ ./ - 1 ( 1 - .  - S)" - 1 c

  \& I Jo

 i/r -1 (1 - bt '- C x, - (1 -A-i)""' dx

The reader will prove similarly that Ly- 0 as 8- 0.

Hence I / o?a' i / 4) x,y)dy\= hm / t/ j / 4> x,y)dy\

= lim /

1-25 C fl-x-S

dx J: i ( ) x, y) dy

= lim

5H..0

fl-2S ( fi-y-S ]

* Annals of Mathematics, ix. (1908), p. 183.

t I a.-i ~ (l-.ri)''- dxi = £(A, ) exists if 0, j'>0 \hardsectionref{4}{5}example2).

t The repeated integral exists, and is, in fact, absolutely
convergent; for

/"i ri

writing 2/ = (1- a.-) s; and/ il/ - (1 - .r)' + ''-i cZx . I ' s' -1
(1 -s)''" £?s exists. And since the

 " fl-e - ' fl-ZS

integral exists, its value which is lim I may be written lim I

5, e O J S S O J S

%
% 77
%

by what has beeu already proved; but, by a precisely similar jjiece
of work, the last integral is

We have consequently proved the theorem in question.

Corollary. Writing = a + h-a) x, rj = h - b-a)y, we see that, if 4>
i\$j ) i con- tinuous,

//I f (|-; ~' ib-vf' iv-\$r'' a, V) dv

-J/V fli -af-Hb-rir-Hn- )"-' c >, rj)d Y

This is called Dirichlet's formula.

[Note. What are now called infinite and improper integrals Avere
defined by Cauchy, Leco?iS sur le calc. inf. 1823, though the idea of
infinite integrals seems to date from Maclaurin (1742). The test for
convergence was employed by Chartier (1853). Stokes (1847)
distinguished between 'essentially' (absolutely) and non-essentially
convergent integrals though he did not give a formal definition. Such
a definition was given by Dirichlet in 1854 and 1858 (see his
I'orlesiingen, 1904, p. 39). In the early part of the nineteenth
century improper integrals received more attention than infinite
integrals, probably because it was not fully realised that an infinite
integral is really the Iwiit of an integi'al.]

\Section{4}{6}{Complex inteyration*TODO.}

Integration with regard to a real variable x may be regarded as
integration along a particular path (namely part of the real axis) in
the Argand diagram. \ \ Qtf z), (= 7 -f iQ), be a function of a
complex variable z, which is continuous along a simple curve i in the
Argand diagram.

Let the equations of the curve be

x = x (t), tj = y it) (a t b).

Let X (a) + iy a) = Zq, x (b) + iy (b) = Z.

Then if-f x(t), y(t) have continuous difierential coefficients J Ave
define z f z)dz taken along the simple curve AB to mean

/

dx . dy Mt dt

 F + iQ)( + i' ]dt.

The 'length' of the curve AB will be defined as I \/ (--f) +( ) ' 

It obviously exists if -7-, -~ are continuous; we have thus reduced
the  dt dt

discussion of a complex integral to the discussion of four real
integrals, viz.

! \ A-' />!- />

dt

* A treatment of complex integration based on a different set of ideas
and not making so many assumptions concerning the curve AB will be
found in Watson's Complex Integration and Cauchy's Theorem.

f This assumption will be made throughout the subsequent work.

X Cp.\hardsubsectionref{4}{1}{3} example 4.

%
% 78
%

By\hardsubsectionref{4}{1}{3} example 4, this definition is consistent with the definition
of an integral when AB happens to be part of the real axis.

Examples, l f(z) dz= - l " f(z) dz, the paths of integration being the
same (but in opposite directions) in each integral.

/:- /.w:f-s-4f--(4'-4) *\

\Subsection{4}{6}{1}{The fundamental theorem of complex integration.}

From\hardsubsectionref{4}{1}{3}, the reader will easily deduce the following theorem :

Let a sequence of points be taken on a simple curve z Z; and let the
first n of them, rearranged in order of magnitude of their parameters,
be called i<">, gC", . . . z <"' (iTo"*' = z +i'"' = Z); let their
parameters be j"*', J"', . . . <'", and let the sequence be such that,
given any number h, we can find N such that, when n > N, +i<"' - <"' <
B, for r = 0, 1, 2, ...,n; let,."" be any point whose parameter lies
between < < '', r+i*"'; then we can make

I (,+i'") - Zr fiC/- ) - I \ f(z) dz

arbitrarily small by taking n sufficiently large.

\Subsection{4}{6}{2}{An upper limit to the value of a complex integral.}

Let M be the upper bound of the continuous function \ f(z) .

Then jJVw,.j £:/(.)sj(| + 4y)],,

 Ml, where I is the ' length ' of the curve z yZ.

That is to say, I f(z) dz cannot exceed Ml.

\Section{4}{7}{Integration of infinite series.}

We shall now shew that if S z) = u z) + ii. z)- ... is a uniformly
con- vergent series of continuous functions of z, for values of z
contained within some region, then the series

I III z) dz + I lu z) dz + ..., J c J c

(where all the integrals are taken along some path C in the region) is
con- vergent, and has for sum I S (z) dz.

J c

%
% 79
%

For, writing

>Sf Z) = Ml Z) + 2 ( ) +    + Ihi Z) + Rn Z),

we have

I S z) dz =1 u z)dz + ... -\ Un (z) dz + I R,, (z) dz.

J c J c J c J c

Now since the series is uniformly convergent, to every positive number
e there corresponds a number r independent of z, such that when n r we
have I Rn (z) I < f> for all values of z in the region considered.

Therefore if I be the length of the path of integration, we have (§
4'62)

f Rn z) J C

<el

dz

Therefore the modulus of the difference between / S z) dz and

J c n r

S I Um (z) dz can be made less than any positive number, by giving n
any

m = lJ c

sufficiently large value. This proves both that the series 2 Um z)dz
is

OT = 1 J c

convergent, and that its sum is / S(z)dz.

J c

Corollary. As in\hardsubsectionref{4}{4}{4} corollary, it may be shewn that*

0? ",, " d, .

if the series on the right converges uniformly and the series on the
left is convergent.

Example 1. Consider the series

" 'ix n n +1) 11x x -1 cos x" ?i 1+ 2 sin2 x ] 1 + n + 1)2 sin- .r '
in which x is real.

The Jith term is

2.rw cos x 2x (n + 1) cos x

1 + 71 sin"- X- 1 + (n + 1 )2 sin- x ' and the sum of n terms is
therefore

2x cos x' 2x(n + l) cos x

1+Sin2 2~ l + (7i+l)2sin2a;2*

Hence the series is absolutely convergent for all real values of x
except ± /(mrr) where ?>i = 1, 2, . . .; but

r> .,\ 2x(n+l)cosx -'"''-l+(K + l)2sin2, :2'

and if n be any integer, by taking x = n + l)~' this has the limit 2
as n <X) . The series is therefore non-uniformly convergent near x=0.

* - - ' means lira " where h- 0 along a definite simple curve; this
definition

is modified slightly in\hardsubsectionref{5}{1}{2} in the case when/(z) is an analytic
function.

%
% 80
%

Now the sum to infinity of the series is - - -, and so the integral
from to,r of ' 1 + sm-' x

the sum of the series is arc tan sin r . On the other hand, the sum of
the integrals from

to .V of the first n terms of the series is

arc tan sin .r - arc tan ( + !) sin x',

and as /i- X this tends to arc tan sin x-] - hir.

Therefore the integral of the sum of the series difiers from the sum
of the integi-als of the terms by tt.

Example 2. Discuss, in a similar manner, the series

== 2e .p l- (e-l) + e" + U'2

for real values of x.

Example 3. Discuss the series

7(1 + 11-2 + U3+...,

where

Ui = ze-'\ Un=nze-'"' - ( - 1) se-l"-!), for real values of z.

The sum of the first n terms is ?i2e~" ", so the sum to infinity is
for all real values of z. Since the terms Un are real and ultimately
all of the same sign, the convergence is absolute.

In the series

/ Uidz+ I ti.2dz+ I v.3dz + ...,

the sum of Ji. terms is -g (1 - e"" ), and this tends to the limit |
as n tends to infinity; this is not equal to the integral from to 2
of the sum of the series 2?<n-

The explanation of this discrejjancy is to be found in the
non-uniformity of the convergence near 2 = 0, for the remainder after
n terms in the series Ui + 112 + ...is - -aze~ "; and by taking z =
7i~' we can make this equal to e' '"-, which is not arbitrarily small;
the series is therefore non-uniformly convergent near z = 0.

Example 4. Compare the values of

/\ 2 tin[ dz and 2 / Undz, U=l J n=lj

where

2n z 2 n- l) z

"l+n'z )\ og n + l) l + n + -iyh'- log n + 2)'

\addexamplecitation{Trinity, 1903.}

REFERENCES.

G. F. B. Riemann, Ges. Math. Werke, pp. 239-241.

P. G. Lejeuxe-Dirichlet, Yorlesungen. (Brunswick, 1904.)

F. G. Meyer, Bestimmte Integrale. (Leipzig, 1871.)

E. GoDRSAT, Cours d Analyse (Paris, 1910, 1911), Chs. iv, xiv.

C. J. DE LA Vall e Poussix, Cours d' Analyse In fiyiite'stmale (Payis
and Louvaiu, 1914),

Ch. VI. E. W. HoBSON, Functions of a Real Variable (1907), Ch. v. T.
J. I'a. Bromwich, Theory of Infinite Series (1908), Appendix ill.

%
% 81
%

Miscellaneous Examples.

1. Shew that the integrals

I sin x )dx, I cos (.r-) dx, I x exp ( - x sin- x) dx Jo Jo Jo

converge. \addexamplecitation{Dirichlet and Du Bois Eeymoud.}

2. If a be real, the integral

f °° cos (ax),

Jo 1+ is a continuous function of a. \addexamplecitation{Stokes.}

3. Discuss the uniformity of the convergence of j x sin x - ax) dx.

Jo

3 /.rsin x -ax)dx= -f - -l-ir-j) cos (x- -ax)

/"/I a\,,,, 1 fsin(x -ax), ~\

- JU + : j " ' -" ) ' '+3°"j - x - ""-J

(de la Vallee Poussin.)

4. Shew that / ex ) [-e'< x -nx)]dx converges unifonuly in the range -
hr, hir) of values of a. \addexamplecitation{Stokes.}

r " x' dx

5. Discu.ss the convergence of I, - . when u, p, p are positive.

* Jo l+JT" |smjp|P - )/- r

(Hardy, Messenger, xxxi. (1902), p. 177.)

6. Examine the convergence of the integrals

Jo V 2 l-e'J .r ' jo X"

\addexamplecitation{Math. Trip. 1914.}

7. Shew that / - exists.

J " x' (sin x)

8. Shew that I .r -"e"" ' sin 2.rc/.r converges if a >0, a >0. (Math.
Trip. 1908.)

J a

9. If a series (7(2)= 2 (c - (? + i)sin (2i/+l) Tri:, (in which C(, =
0), converges uniformly

v=0

TT . . . . C

in an interval, shew that g z) -. is the derivative of the series/
(2)= 2 - sin 2vTrz.

sm irZ v=i V

(Lerch, Ann. de VEc. norm. sup. (3) xii. (1895), p. 351.)

10. Shew that r r... r |L i tff2- n d r r r dx,dx,...dx

converge when a>hi' and a~i + /3~i + ...+X~' < 1 respectively. (Math.
Trip. 1904.)

11. Iff x, ) be a continuous function of both.r andy in the ranges (a
x b), (a ?/ 6) except that it has ordinary discontinuities at points
on a finite number of curves, with continuously turning tangents, each
of which meets any line parallel to the coordinate axes

fb

only a finite number of times, then I f x, y) dx is a continuous
function of y.

/a, - Sj Ca.2-\&2 [b

+ 1 +...+ I fC 'j y + h)-f(x, y)]dx, where the numbers

Sj, \hardsectionref{2}{5}  fi, f2?  are so chosen as to exclude the
discontinuities ot f x, y + h) from the range of integration; Oj, 02,
... being the discontinuities off x, y).] \addexamplecitation{Bocher.}

W. M. A. 6


\chapter{The Fundamental Properties of Analytic Functions; 
Taylor's, Laurent's and Liouville's Theorems} 

51. Property of the elementary functions.

The reader will be already familiar with the term elementary function,
as used (in text-books on Algebra, Trigonometry, and the Differential
Calculus) to denote certain analytical expressions* depending on a
variable z, the symbols involved therein being those of elementary
algebra together with exponentials, logarithms and the trigonometrical
functions; examples of such

expressions are

1 • §

Z-, e~, iogz, arcsm '-.

Such combinations of the elementary functions of analysis have in
common a remarkable property, which will now be investigated.

Take as an example the function e .

Write e'=f z).

Then, if 2 be a fixed point and if z' be any other point, we have

f z')-f z) \ e~ -e' \, e' '- ' - 1 z -z z - z z - z

f, z' - z (z -z)-

and since the last series in brackets is uniformly convergent for all
values of it follows \hardsectionref{3}{7}) that, as z'->z, the quotient

z - z tends to the limit e, uniformly for all values of arg z - z).
This shews that tlie limit of

f z:)-f z)

z - z

is in this case independent of the path by which the point z tends
towards coincidence witJt z.

It wall be found that this property is shared by many of the
well-known elementary functions; namely, that iif(z) be one of these
functions and h. be

* The reader will observe that this is uot the sense in which the term
function is defined \hardsectionref{3}{1}) in this work. Thus e.g. .r - hj and | z \
are functions of z (-x + iy) in the sense of\hardsectionref{3}{1}, but are not
elementary functions of the type under consideration.

5 1-5 "12] FUNDAMENTAL PROPERTIES OF ANALYTIC FUNCTIONS 83 any complex
number, the limiting value of

exists and is independent of the mode in luhich h fends to zero.

The reader will, however, easily prove that, ii f(z)=x -iy, where z =
x- iy, then lim' - - - - IXJ- is not independent of the mode in which
A- >0.

5'11. Occasional failure of the jwoperty.

For each of the elementary functions, however, there will be certain
points z at which this property will cease to hold good. Thus it does
not hold for the function l/( - a) at the point z = a, since

 i Qh\ z - a+h z-a

does not exist when z = a. Similarly it does not hold for the
functions log z

and z at the point z = Q.

These exceptional points are called singular points or singulaHties of
the function f z) under consideration; at other points f z) is said
to be analytic.

The property does not hold good at any point for the function \ z.

5'12. Cauchy's* definition of an analytic function of a complex
variable.

The property considered in § b\ \ will be taken as the basis of the
definition of an analytic function, which may be stated as follows.

Let a two-dimensional region in the -plane be given; and let w be a
function of z defined uniquely at all points of the region. Let z, z-
hz be values of the variable z at two points, and u, u + Bu the
corresponding values

of u. Then, if, at any point z within the area, - tends to a limit
when 8x->0,

By-*0, independently (where 8z = 8x + iBy), u is said to be a function
of z which is monogenic or analytic j" at the point. If the function
is analytic and one-valued at all points of the region, we say that
the function is analytic throughout the region.

We shall frequently use the word ' function ' alone to denote an
analytic function, as the functions studied in this work will be
almost exclusively analytic functions.

* See the memoir cited in § 5 "2.

t The words ' regular ' and ' holomorphic ' are sometimes used. A
distinctiou has been made by Borel between ' monogenic ' and '
analytic ' functions in the case of functions with an infinite number
of singularities. See\hardsubsectionref{5}{5}{1}.

X See\hardsectionref{5}{2} cor. 2, footnote.

6-2

%
% 84
%

In the foregoing definition, the function u has been defined only
within a certain region in the -plane. As will be seen subsequently,
however, the function u can generally be defined for other values of 2
not included in this region; and (as in the case of the elementary
functions already discussed) may have singularities, for which the
fundamental property no longer holds, at certain points outside the
limits of the region.

We shall now state the definition of analytic functionality in a more
arithmetical form.

Let f z) be analytic at z, and let e be an arbitrary positive number;
then we can find numbers I and h, h depending on e) such that

z -z I

whenever \ z' - z\ < h.

Vi f z) is analytic at all points of a region, I obviously depends on
z; we consequently write 1 = f z).

Hence /(/) = f z) + - z) f z) + v z'- z\

where v is a function of z and z such that ! v < e when \ z -z\ < t>.

Example 1. Find the points at which the following functions are not
analytic :

z - \ (i) 2 . (ii) cosec2 (2 = /itt, w any integer). (iii) - - - - -
(3 = 2,3).

1

(iv) ez (j = 0). (V) z- ) zf (2 = 0,1).

Example 2. If z = x - iy, f z) = u ii\ where u, v, x, y are real and /
is an analytic function, shew that

8m "bv cu cv /T - s

 :r =, : =-; . (Kiemann.)

ex oy oy ex

5*13. An application of the modified Heine-Borel theorem.

Let f(z) be analytic at all points of a continuum; and on any point z
of

the boundar ' of the continuum let numbers f (z), S (S depending on z)
exist

such that

\ f(z')-f(z)- z'-z)Mz)<e,z'-z\

whenever \ z - z < 8 and z is a point of the continuum or its
boundary.

[We write /i iz) instead of /' z) as the differential coefficient
might not exist when c approaches z from outside the boundary so
that/j (2) is not necessarily a unique derivate.]

The above inequality is obviously satisfied for all points z of the
continuum as well as boundary points.

Applying the two-dimensional form of the theorem of\hardsectionref{3}{6}, we see that
the region formed by the continuum and its boundary can be divided
into a jinite number of parts (squares with sides parallel to the axes
and their

%
% 85
%

interiors, or portions of such squares) such that inside or on the
boundary of any j art there is one point z such that the inequality

fiz') - f z,) - (/ - z,)f, z,) .<e\ z'-z, \ is satisfied by all points
z inside or on the boundary of that part.

5*2. CaUCHY'S theorem* on the integral OF A FUNCTION ROUND A CONTOUR.

A simple closed curve C in the plane of the variable z is often called
a contour; ii A, B, D he points taken in order in the
counter-clockwise sense along the arc of the contour, and if f z) be a
one-valued continuousf function of z (not necessarily analytic) at all
points on the arc, then the integral

f f z)dz or f f z)dz

taken round the contour, starting from the point A and returning to A
again, is called the integral of f z) taken along the contour. Clearly
the value of the integral taken along the contour is unaltered if some
point in the contour other than A is taken as the starting-point.

We shall now prove a result due to Cauchy, which may be stated as
follows. If fiz) is a function of z, analytic at all points on and
inside a contour G, then

I f z)dz = 0.

For divide up the interior of C by lines parallel to the real and
imaginary

axes in the manner of § .5"13; then the interior of Cis divided into
a number

of regions whose boundaries are squares Cj, C, ... Cm and other
regions

whose boundaries D,, D,, ... Dy are portions of sides of squares and
parts

of G; consider

Mr N r

X f(z)dz+ S f z)dz,

n = lJ(,C ) n=lJ(.D )

each of the paths of integration being taken counter-clockwise; in
the complete sum each side of each square appears twice as a path of
integration, and the integrals along it are taken in opposite
directions and consequently cancel §; the only parts of the sum which
survive are the integrals oi f z)

* Memoire sur les integrates definies prises entre des limites
imarjinaires (1825). The proof here given is that due to Goursat,
Trans. American Math. Soc. i. (1900), p. 14.

t It is sufficient for f z) to be continuous when variations of z
along the arc only are considered.

:J: It is not necessary that f(z) should be analytic on C (it is
sufficient that it be continuous on and inside C), but if /( ) is not
analytic on C, the theorem is much harder to prove. This proof merely
assumes that /' [z) exists at all points on and inside C. Earlier
proofs made more extended assumptions; thus Cauchy's proof assumed the
continuity of f' z). Eiemann's proof made an equivalent assumption.
Goursat's first proof assumed that f z) was uniformly differentiable
throughout C.

§ See § 4G, example.

/

J (On)

%
% 86
%

taken along a number of arcs which together make up G, each arc being

taken in the same sense as in I f\ z)dz; these integrals therefore
just make

hc)'

up f z)dz.

Now consider 1 f z)dz. With the notation of\hardsubsectionref{5}{1}{2},

J (Cn)

f z) dz = [ f z,) + ( - z,)f' (.-,) + z- z,) v] dz

-' (C )

= /( i) - V ( 01 f dz +/' (z,) I zdz+l (z- z,) vdz.

\ f? = Mc = 0, f zdz

by the examples of\hardsectionref{4}{6}, since the end points of C coincide. Now let
In be the side of Cn and An the area of C . Then, using\hardsubsectionref{4}{6}{2},

TODO

But

l\ \ l n

= 0,

I

! (C )

In like manner

< e V2 . f \ dz\= eln \/2 . 4,, = 4e sj'2.

J Cn

I f z)dz\ \ i \ \ {z-z )vdz\

J (Dn) I J (Dn)

(Dn)

  4>e (An + In Xi)\ \ '2,

Avhere An is the area of the complete square of which Dn is part, In
is the side of this square and \ n is the length of the part of G
which lies inside this square. Hence, if \ be the whole length of G,
while I is the side of a square which encloses all the squares Gn and
Dn,

f(z)dz k S f z)dz + t \ f z)dz

|J(0 I n = \'J(Cn) I n = \ \ J Dn) |

( M N iV )

<4eV2 An+ S An' + l tXn\ (m = 1 m = 1 n=l )

< 4e V2 . I' + X).

Now e is arbitrarily small, and I, \ and 1 f z)dz are independent of
e.

•I iC)'

It therefore follows from this inequality that the only value which I
f(z) dz can have is zero; and this is Cauchy's result.

%
% 87
%

Corollary . If there are two jjaths z AZ and Zq,BZ from 2o to Z, and
if /(z) is a function of z analytic at all points on these curves and
throughout the domain enclosed by

these two paths, then / f z) dz has the same value whether the path of
integration is

Z(iAZ (jv ZqBZ. This follows from the fact that ZqAZBzq is a contour,
and so the integral taken round it (which is the difference of the
integrals along zqAZ and ZoBZ) is zero.

Thus, if /(2) be an analytic function of z, the value of / f z)dz is
to a certain extent

J AB

independent of the choice of the arc AB, and depends only on the
terminal points A and B. It must be borne in mind that (his is only
the case whenf z) is an analytic function in the sense of\hardsubsectionref{5}{1}{2}.

Corollary 2. Suppose that two simple closed curves (7 and Cj are
given, such that Co completely encloses Cj, as e.g. would be the case
if Cq and Ci were confocal ellipses.

Suppose moreover that/ (2) is a function which is analytic* at all
points on Cq and Cj and throughout the ring-shaped region contained
between Cq and C . Then by drawing a network of intersecting lines in
this ring-shaped space, we can shew, exactly as in the theorem just
proved, that the integral

//(

dz

is zero, ivhere the integration is taken round the whole boundary of
the nng-shaped space; this boundary consisting of two curves Co and C
, the one described in the counter-clochvise direction aiid the other
described in the qlockwise direction.

Corollary 3. In general, if any connected region be given in the
3-plane, bounded by any number of simple closed curves Co, Ci, Co,
..., and if /(z) be any function of z which is analytic and one-valued
everywhere in this region, then

I

f z)dz

is zero, where the integral is taken round the whole boundary of the
region; this boundary consisting of the curves Co, Ci, ..., each
described in such a sense that the region is kept either uhvays on the
right or always on the left of a person walking in the sense in
question round the boundary.

An extension of Cauchy's theorem I f z) dz = 0, to curves lying on a
cone whose vertex

is at the origin, has been made by Ravut (N'ouv. Annales de Math. (3)
xvi. (1897), pp. 365-7). Morera, Moid, del 1st. Lombardo, xxii.
(1889), p. 191, and Osgood, Bull.

Amer. Math. Soc. II. (1896), pp. 296-302, have shewn that the property
f z)dz =

may be taken as the property defining an analytic function, the other
properties being deducible from it. (See p. 110, example 16.)

Example. A ring-shaped region is bounded by the two circles | s | = 1
and | z | = 2 in the 2-plane. Verify that the value of I -, where the
integral is taken round the boundary of this region, is zero.

* The phrase 'analytic throughout a region' implies one-valuedness (§
5-12); that is to say that after z has described a closed path
surrounding Co, f z) has returned to its initial value. A function
such as log z considered in the region 1 | | 2 will be said to be '
analytic at all points of the region. '

  

%
% 88
%

For the boundary consists of the circumference ]2| = 1, described in
the clockwise direction, together with the circumference |s| = 2,
described in the counter-clockwise direction. Thus, if for points on
the iirst circumference we write 2 = e, and for points on the second
circumference we write z = '2e<'t>, then 6 and cp are real, and the
integral becomes

jo e' jo 2e'*

5'21. 2'he value of an analytic function at a 'point, expressed as an
integral taken round a contour enclosing the point.

Let C he a contour within and on which /( ) is an analytic function of
z. Then, if a be any point within the contour,

z - a is a function of z, which is analytic at all points within the
contour G except the point z = a.

Now, given e, we can find 3 such that

\ fi z)-f a)- z-a)f' a)\ \ \ z-a\

whenever | - a | < S; with the point a as centre describe a circle 7
of radius 7' < 8, r being so small that 7 lies wholly inside C

Then in the space between 7 and G f z)\ \ {z - a) is analytic, and so,
by\hardsectionref{5}{2} corollary 2, we have

f z)dz\ [f z)dz

c z - a Jy z - a where I and I denote integrals taken
counter-clockwise along the curves

G and 7 respectively.

But, since 1 2 - a | < S on 7, we have

r f(z) dz\ f f a)+iz- a)f (a) + v(z-a) J y z - a J y z - a

where \ v\ < e; and so

r m y j f dz f ( ) dz -\ vdz.

J C Z-a \ lyZ -a - Jy Jy

Now, if z be on 7, we may write

z - a = re, where r is the radius of the circle 7, and consequently

"2t ire dO

jyZ - a Jo

= I I de = 2771, y j, re'" Jo

and dz= ire' dd = 0;

also, by\hardsubsectionref{4}{6}{2},

vdz

< e . 27rr.

i

%
% 89
%

r f(z)d2 \ 2 if( a) I = f €dz\ \ 2irre.

J c z - a \ J y 1

( r ( 5: 1 n.7: i /'

Thus

c z - a

But the left-hand side is independent of e, and so it must be zero,
since e is arbitrary; that is to say

    2771 ] c z - a

This remarkable result expresses the value of a function f z\ (which
is analytic on and inside (7) at any point a within a contour C, in
terms of an integral which depends only on the value of/( ) at points
on the contour itself.

Corollary. If f z) is an analytic one-valued function of 2 in a
ring-shaped region bounded by two curves C and C", and a is a point in
the region, then

•' 2ni J c -(i "tti c' Z - a

where C is the outer of the ciu-ves and the integi-als are taken
counter-clockwise.

5"22. The derivates of an analytic function f z).

The function/' (2'), which is the limit of

f z + h)-f z) h

as h tends to zero, is called the derivate of / z). We shall now shew
that f' z) is itself an analytic function of z, and consequently
itself possesses a derivate.

For if be a contour surrounding the point a, and situated entirely
within the region in which f z) is analytic, we have

f(a+h)-f(a)

f (a) = hm --,

h o n

,, 0 2TTih [J c z-a-h (j z-a )

= Km -I- \ /( ) dz

h Q liri J c z-a) z- a~h)

Itti J c z - af h 27ri J c z - ay z-a- h)

Now, on C, f z) is continuous and therefore bounded, and so is (z -
a)~; while we can take | h less than the upper bound of U - a |.

\

90

Therefore

(z - a)- z - a - h) Then, if I be the length of C,

h f f z)dz

[chap. V

is bounded; let its upper bound be K.

THE PROCESSES OF ANALYSIS

lim

<lim !A|(27r)- 7 : = 0,

I h Q lirij c z -af z-a- h)

and consequently f (a) = - . |, .

   Ziri J c z - a)- '

a formula which expresses the value of the derivate of a function at a
point as an integral taken along a contour enclosing the point.

From this formula we have, if the points a and a + h are inside C,

f(a + h)-f'(a) J\ r f z)dz 1 1

h 27ri J c h

(z - a- hy z - ay

c (z - a - hy z - ay

f(z) dz

= 9:Z;f f )dz.

iTTl J c

... +hAh,

ZTTi J c \ Z - ay

and it is easily seen that J./ is a bounded function of z when \ h\ <
\ z - a.

Therefore, as h tends to zero, A~' /'( - + A) - /' (a) tends to a
limit, namely

J\ /• f(z)dz 27ri J c z - ay '

Since /' (a) has a unique differential coefficient, it is an analytic
function of a; its derivate, which is represented by the expression
just given, is denoted by/" (a), and is called the second derivate of
/(a).

Similarly it can be shewn that /"(a) is an analytic function of a,
possessing a derivate equal to

2 f f(z)dz\ 27ri Jc z-ay'

this is denoted by f" (a), and is called the third derivate of /(a).
And in general an nth derivate/*"' (a) of /(ct) exists, expressible by
the integral

r. n r f(z)dz P 2'rTiJc z-aT+ '

and having itself a derivate of the form

(n + 1)! r f z)dz .

27ri Jciz- a)"+ ' the reader will see that this can be proved by
induction without difficulty.

 '

%
% 91
%

A function which possesses a first derivate with respect to the
complex variable z at all points of a closed two-dimensional region in
the •-plane therefore possesses derivates of all orders at all points
inside the region.

5'23. Caiichys inequality for f " (a).

Let f(z) be analytic on and inside a circle G with centre a and radius
?\ Let M be the upper bound oif z) on the circle. Then, by\hardsubsectionref{4}{6}{2},

!/""( )i;4/c; 'i'' '

M.n\

Example. l f z) is analytic, z = x- iij and V- =;, + -2, .shew that

V2 log 1/(2) 1 = 0; andy2|/(e)|>0 unle.ss/(2) = or/' z) = Q. (Trinity,
1910.)

5"3. Analytic functions represented by uniformly convergent series.

 X)

Let X fn (z) be a series such that (i) it converges uniformly along a
=o

contour C, (ii) / (z) is analytic throughout C and its interior.

00 Then 2 fni ) converges, and the sum* of the series is an analytic

n =

function throughout C and its interior.

00

For let a be any point inside C; on C, let S fn z) = (z).

Then -. - dz = j- \ fn( )\

27ri J c z - a 27ri j c Im=o ] z -a

 .o \ 2mlc 2-a \ '

00

by* § 47. But this last series, by\hardsubsectionref{5}{2}{1}, is S fiid)', the series
under

n =

consideration therefore converges at all points inside C; let its sum
inside G (as well as on C) be called (z). Then the function is
analytic if it ha,s a unique differential coefficient at all points
inside G.

But if a and a + h be inside G,

 (a + h)- (a) \ J f ( ) (

A 27rt J c z - a) z - a - h)'

and hence, as in\hardsubsectionref{5}{2}{2}, lim W a- h) - a)] A~ ] exists and is equal to

7t-*0 * Since | 2 - a |~i is bounded when a is fixed and z is on C,
the uniformity of the convergence of S / z)l[z - a) follows from that
of 2 / [z).

n=0 n=0

%
% 92
%

  - : 7 - dz; and therefore <I> (z) is analytic inside G. Further, by
27rt J c z- a)

transforming the last integral in the same way as we transformed the
first

00 ao

one, we see that <!>' (a) = S / ' (a), so that 2 fn (a) may be '
differentiated

n=0 n=0

term by term.'

If a series of analytic functions converges only at points of a curve
which is not closed nothing can be inferred as to the convergence of
the derived series*.

'COS 7hJ7

Thus 2 ( - )" - 2 - converges uniformly for real values of x \hardsubsectionref{3}{3}{4}).
But the derived

H = l 'i

  sin ij series 2 ( - )""* converges non-uniformly near A' = (2m+1)
tt, (m any integer); and

H = l 'ii

the derived series of this, viz. 2 ( - )"~ cos n.v, does not converge
at all.

(1=1

Corollary. By\hardsectionref{3}{7}, the sum of a power series is analytic inside its
circle of con- vergence.

531. Analytic functions represented hy integrals.

Let f t, z) satisfy the following conditions when t lies on a certain
path of integration (a, h) and z is any point of a region >S' :

(i) f and ~ are continuous functions of t.

   dz

(ii) / is an analytic function of z.

df * (in) The continuity of ~- qua function of z is uniform with
respect to

the variable t.

rb .

Then I f(t, z)dt is an analytic function of z. For, by\hardsectionref{4}{2}, it has
the

, !* f* dt\ t, z) . unique derivate - - - - dt.

5*32. Analytic functions represented hy infinite integrals.

From\hardsubsectionref{4}{4}{4} (II) corollary, it follows that I f (t, z) dt is an
analytic

J a

function of z at all points of a region >S' if (i) the integral
converges, (ii) f t, z) is an analytic function of z when t is on the
path of integration and z is on S,

(iii) - : ' is a continuous function of both variables, (iv) - - - dt

dz J a OZ

converges uniformly throughout 8.

For if these conditions are satisfied f t, z) dt has the unique
derivate

J a

J a

dz

* This might have been anticipated as the main theorem of this section
deals with uniformity of convergence over a two-dimensional region.

5 -3 1-5 -4] Taylor's, Laurent's and liouville's theorems 93

A case of very great importance is afforded by the integral I e~' /(0
dt,

Jo where /(t) is continuous and \ f t)\ < Ke' where K, r are
independent of t; it is obvious from the conditions stated that the
integral is an analytic function of z when R z) r, > r. [Condition
(iv) is satisfied, by\hardsubsubsectionref{4}{4}{3}{1} (I),

r

since I fe"""''*' rf converges.]

Jo

5-4. Taylor's Theorem*.

Consider a function f z), which is analytic in the neighbourhood of a
point z = a. Let (7 be a circle with a as centre in the -plane, which
does not have any singular point of the function f z) on or inside it
; so that f z) is analytic at all points on and inside C. Let z = a +
h be any point inside the circle C. Then, by\hardsubsectionref{5}{2}{1}, we have

    Itti J c z-a-h

 -rriJc Xz-a iz-af ' ' ' z - ay- ' z - a)'' ' z - a - h)]

f z) But when z is on C, the modulus of - - -j is continuous, and so,

z - a - h

by\hardsubsectionref{3}{6}{1} cor. (ii), will not exceed some finite number M. Therefore,
by\hardsubsectionref{4}{6}{2},

1 f f(z)dz.h''+

27riJc z-ay'+' z-a-h) " 27r [rJ '

where R is the radius of the circle C, so that 'IttR is the length of
the path of integration in the last integral, and R = \ z - a\ for
points z on the cir- cumference of C.

The right-hand side of the last inequality tends to zero as ?? - > oo
. We have therefore

/(a + /0=/(a) + / /'(a)-h|,/"(a)+...-f |/-'(a) + ..., which we can
write

f(z)=f a) + (z - a)/' (a) + ~ ~ff" a) + ... + ri /< ) (a) + ....

This result is known as Taylors Theorem; and the proof given is due
to Cauchy. It follows that the radius of convergence of a poiuer
series is always

* The formal expansion was first published by Dr Brook Taylor (1715)
in his Methodus Incrementonim.

%
% 94
%

at least so large as only just to exclude from the interior of the
circle of con- vergence the nearest singularity of the function
represented by the series. And by\hardsectionref{5}{3} corollary, it follows that the
radius of convergence is not larger than the number just specified.
Hence the radius of convergence is just such as to exclude from the
interior of the circle that singularity of the function which is
nearest to a.

At this stage we may introduce some terms which will be frequently
used.

If f a) = 0, the function f(z) is said to have a zero at the point z =
a. If at such a point f (a) is different from zero, the zero of f(a)
is said to be simple; if, however,/' a),f" a), .../"*"'' (a) are all
zero, so that the Taylor's expansion of f z) at z = a begins with a
term in (z - a)", then the function f z) is said to have a zero of the
nth. order at the point z = a.

Example 1. Find the function / (s), which is analytic throughout the
circle C and its interior, whose centre is at the origin and whose
radius is unity, and has the value

a - cos 6 . sin 6

a -2acos6 + l a -2acos0 + l

(where a> 1 and 6 is the vectorial angle) at points on the
circumference of C.

[We have

f z) dz

/( )(0) = .f - - •' ' 2mjc z"

n ! /"St

27nJo

\ n\ n e~"i9d6 \ n \ f dz f d"" 1 "1

~2ffjo a-e<9 ~2iriJcz''(a-z)~~\ \ ds a-zJ

e- 'O.idd. -7- - ~;r ., (puttuig z = e*e) a- -2a cos + 1 ' ° '

Therefore by Maclaurin's Theorem*,

)l=0 "

or/(2) = (a-2)~' for all points within the circle.

This example raises the interesting question, Will it still be
convenient to define f z) as (a-2)~ at points outside the circle ?
This will be discussed in\hardsubsectionref{5}{5}{1}.]

Example 2. Prove that the arithmetic mean of all values of 2"" 2 cv,
for points z on

the circumference of the circle \ z\ = l, is a, if Sc? " is analytic
throughout the circle and its interior.

/ (") (0) [Let 2 v2''=/(2), so that a, =;- . Then, writing z = e',
and calling C the circle

K=0 "

277 jo 2" ~ 2ni j c 2"* ~ n\ """ -'

z'i * The re8ult/ 2) =/(0) +2/' (0) + -/" (0) + ..., obtained By
putting a = in Taylor's Theorem,

is usually called Maclaurin's Theorem; it was discovered by Stirling
(1717) and published by Maclaurin (1742) in his Fluxions.

%
% 95
%

Example 3. Let $f(z) = z^{r}$; then $f(z+h)$ is an analytic function
of $h$ when $\absval{h} < \absval{z}$ for all values of $r$; and so
$(z + h)^{r} = z^{r} + rz^{r-1} h + \frac{ r (r-1) }{2} z^{r-2} h^{2}
+ \cdots, $ this series converging when $\absval{h} < \absval{z}$.
This is the binomial theorem.\index{Binomial theorem}

Example 4. Prove that if h is a positive constant, and (1 - 2zh- h?) ~
in expanded in the form

\ + hP z) + h''P.2 z) + h P z) + (A),

(where P (2) is easily seen to be a polynomial of degree n in z), then
this series converges so long as z is in the interior of an ellipse
whose foci are the points z = \ and 2= -1, and whose semi-major axis
is (A + A"').

Let the series be first regarded as a function of A. It is a power
series in A, and therefore converges so long as the point A lies
within a circle in the /; -plane. The centre of this circle is the
point A = 0, and its circumference will be such as to pass through
that

singularity of (1 - 2zh- h' )~ which is nearest to A = 0.

But 1 - 22A -1- A2 = A - 2 + (22 - 1 )5>. /i \ 2 \ ( 2 \ 1 )ij,

so the singularities of (1 - 22A-|-A-)~2 are the points h=z - z' - ) '
and h=z + z - ) . [These singularities are branch points (see\hardsectionref{5}{7}).]

Thus the series (A) converges so long as | A | is less than both
|2-(22-l) | and |2-f(22\ l) |.

Draw an ellipse in the 2-plane passing through the point 2 and having
its foci at +L Let a be its semi-major axis, and 6 the eccentric angle
of 2 on it.

Then 2 = a cos -f i (a - 1 ) sin 9,

which gives 2 ± (22 - 1 )i = a + (a2 \ i) (cos + 1 sin 6),

so i2±(22-l)i | = a + (a2\ i)4.

Thus the series (A) converges so long as A is less than the smaller of
the numbers a-|-(a2- 1) and a- a - 1)2, i.e. so long as A is less than
a-(a2\ i)5. But A = a - (a2- 1) when a = |(A-f-A~i).

Therefore the series (A) converges so long as 2 is within an ellipse
whose foci are 1 and - 1, and whose semi-major axis is h h + h~ ).

5'41. Forms of the remainder in Taylor's series.

Let f x) be a real function of a real variable; and let it have
continuous differential coefficients of the first n orders when a x a
+ h.

If O i l, we have

fl (71-1 hm •) /i n -f\ n-\

It li. 7! (1 - ') "/'"" ( + *> = ifr /'"' ( + "'> - ''/' ( + "')•

Integrating this between the limits and 1, we have

n-l hm /•! hn /I \ f\ n-i

f(a + h)=f a)+ -,f (a)+ ] ', f'Ha + th)dt.

ni = im: Jo n-L)l

Let Rn = J~iyi ] \ l - 0' - V'"* ( + ih) dt;

and let j-j be a positive integer such that p n.

%
% 96
%

Then Rn = r - f ' ( " y~" ' ( " O"" /'"* ( + th) dt. Let U, L be the
upper and lower bounds of (1 - )' -p/("' (a + th). Then

f ' X (1 - 0 "' dt<\ \ t)P-' . (1 - O' -P/"'' (a + th) dt<[ U(l- t)P-'
dt. Jo Jo Jo

Since (1 - t)' ~P / '" (a + th) is a continuous function it passes
through all

values between U and L, and hence we can find 6 such that 1, and

[ I- ) -y <' ' (a + th) dt = -1(1- ey-Pf " (a + Oh). Jo

Therefore R = .J \ y;(1 - f- /*"' (a + 6h).

A" Writing p = n, we get Rn = - /"" (a + Oh), which is Lagrange s form
for

A" Me remainder; and writing j9 = 1, we get Rn = - \, y, (1 -
)''-'/''" (a + A),

which is Cauchys form for the remainder. Taking n = \ in this result,
we get

f a h)-f a) = hf a + 6h) \ i f x) is continuous when a x a + h; this
result is usually known as the First Mean Value Theorem (see also §
4-14).

Darboux gave in 1876 Journal de Math. (3) ii. p. 291) a form for the
remainder in Taylor's Series, which is applicable to complex variables
and resembles the above form given by Lagrange for the case of real
variables.

55. The Process of Continuation.

Near every point P, Zq, in the neighbourhood of which a function f z)
is analytic, we have seen that an expansion exists for the function as
a sei'ies of ascending positive integral powers of z - Zq), the
coefficients in which involve the successive derivates of the function
at z .

Now let A be the singularity of f z) which is nearest to P. Then the
circle within which this expansion is valid has P for centre and PA
for radius.

Suppose that we are merely given the values of a function at all
points of the circumference of a circle slightly smaller than the
circle of convergence and concentric with it together with the
condition that the function is to be analytic throughout the interior
of the larger circle. Then the preceding theorems enable us to find
its value at all points within the smaller circle and to determine the
coefficients in the Taylor series proceeding in powers of z - Zq. The
question arises, Is it possible to define the function at points
outside the circle in such a way that the function is analytic
throughout a larger domain than the interior of the circle ?,

5*6] Taylor's, Laurent's and liouvtlle's theorems 97

In other words, given a potver series which converges and represents a
function only at poiiits within a circle, to define hy means of it the
values of the function at points outside the circle.

For this purpose choose any point Pi within the circle, not on the
line PA. We know the value of the function and all its derivates at
Pj, from the series, and so we can form the Taylor series (for the
same function) with Pi as origin, which will define a function
analytic throughout some circle of centre Pj. Now this circle will
extend as far as the singularity* which is nearest to Pi, which may or
may not be A; but in either case, this- new circle will iisuall '!
lie partly outside the old circle of convergence, and for jjoints in
the region which is included in the new circle but not in the old
circle, the new series may he used to define the values of the
function, although the old series failed to do so.

Similarly we can take any other point Po, in the region for which the
values of the function are now known, and form the Taylor series with
P as origin, which will in general enable us to define the function at
other points, at which its values were not previously known; and so
on.

This process is called continuation . By means of it, starting from a
representation of a function by any one power series we can find any
number of other power series, which between them define the value of
the function at all points of a domain, any point of which can be
reached from P without passing through a singularity of the function;
and the aggregate § of all the power series thus obtained constitutes
the analytical expression of the function.

It is important to know whether continuation by two different paths
fBQ, PB'Q will give the same final power series; it will be seen that
this is the case, if the function have no singularity inside the
closed curve PBQB'P, in the following way : Let P be any point on PBQ,
inside the circle C' with centre P; obtain the continuation of the
function with Pi as origin, and let it converge inside a circle Ci;
let P be any point inside both circles and also inside the curve
PBQB'P; let S, Si, Si be the power series with P, Pi, Pi as origins;
then|| *S'i = S'i' over a certain domain which will contain Pi, if Pi'
be taken sufficiently near Pi; and hence Si will be the continuation
of Si; for if Ti were the continuation of Si, we have Ti = Si over a
domain containing Pj, and so \hardsubsectionref{3}{7}{3}) corresponding coefficients in i
and Ti are the same. By carrying out such a process a sufficient
number of times, we deform the path PBQ into the path PB'Q if no
singular point is inside PBQB'P. The reader will convince himself by
drawing a figure that the process can be carried out in a finite
number of steps.

* Of the function defined by the new sei'ies.

+ The word ' usually ' must be taken as referring to the cases which
are likely to come under the reader's notice while studying the less
advanced parts of the subject.

X French, prolongement; German, Fortsetzung.

§ Such an aggregate of power series has been obtained for various
functions by M. J. M. Hill, by purely algebraical processes, Proc.
London Math. Soc. xxxv. (1903), pp. 388-416.

II Since each is equal to S.

W. M. A. 7

%
% 98
%

Example. The series

1, Z 22 3

a a a" a* represents the function

/('-) = -- a - z

only for points z within the circle | I = | a | .

But any number of other power series exist, of the type

1 z-h z-hf z-hf

a-b' a-hf" a-bf' a-bf '-' '

if b/a is not real and positive these converge at points inside a
circle which is partly inside and partly outside | s | = | a j; these
series represent this same function at points outside this circle.

5-501. .On functions to which the continuation-process cannot be
applied.

It is not always possible to carry out the process of continuation.
Take as an example the function /(2) defined by the power series

which clearly converges in the interior of a circle whose radius is
unity and w hose centre is at the origin.

Now it is obvious that, as -1-0, /(2) +qc; the point +1 is therefore
a singularity of/ (2).

But /(2)=22+/( 2)

and if z-- 0, f(z )- x and so /(s)- x, and hence the points for which
z- = l are singularities oi f z); the point 2= - 1 is therefore also
a singularity oif z). Similarly since

we see that if 2 is such that 2* = 1, then z is a singularity of/ (2)
; and, in general, any root of any of the equations

22=1, 2* = 1, 28 = 1, 2l =l, ...,

is a singularity of f z). But these points all lie on the circle | 2 |
= 1; and in any arc of this circle, however small, there are an
unlimited number of them. The attempt to carry out the process of
continuation will therefore be frustrated by the existence of this
imbroken front of singularities, beyond which it is impossible to
pass.

In such a case the function f z) cannot be continued at all to points
2 situated outside the circle i 2 | = 1; such a function is called a
lacuvary f miction, and the circle is said to be a limiting circle for
the function.

551. The identity of tivo functions. .,

The two series

1 + 2 + ' + 2' + . . .

and - 1 + ( - 2) - ( - 2y- + ( - 2) - (2 - 2 + ...

do not both converge for any value, of z, and are distinct expansions.

Nevertheless, we generally say that they represent the same function,
on the

strength of the fact that they can both be represented by the same
rational

1 expression .

/ > -" J DEPARTMENT OF

V MATMEMATIC 5501, 5"51] Taylor's, Laurent's and liouville's THEOREl|
k)l)B R A R y

This raises the question of the identity of two functions. When can
two different expansions be said to represent the same function ?

We might define a function (after Weierstrass), by means of the last
article, as consisting of one power series together with all the other
power series which can be derived from it by the process of
continuation. Two different analytical expressions will then define
the same function, if they represent power series derivable from each
other by continuation.

Since if a function is analytic (in the sense of Cauchy,\hardsubsectionref{5}{1}{2}) at
and near a point it can be expanded into a Taylor's series, and since
a convergent power series has a unique differential coefficient (§
5'3), it follows that the definition of Weierstrass is really
equivalent to that of Cauchy.

It is important to observe that the limit of a combination of analytic
functions can represent different analytic functions in different
parts of the plane. This can be seen by considering the series

5(' + j)\!,('"i)(TT7.-r;i )

The sum of the first n + 1 terms of this series is

1 / 1\ 1

z \ zJ' I + z' '

The series therefore converges for all values of z (zero excepted) not
on the circle j 2 | = 1. But, as w - > oo, | 2:'* | - > or j £•" i -
oo according as | j is less or greater than unity; hence we see that
the sum to infinity of the series is

z when \ z\ < 1, and - when | j > 1. This series therefore represents
one

function at points in the interior of the circle | j = 1, and an
entirely different function at points outside the same circle. The
reader will see from\hardsectionref{5}{3} that this result is connected with the
non-uniformity of the convergence of the series near | j = 1.

It has been shewn by Borel* that if a region C is taken and a set of
points S such that points of the set S are arbitrarily near every
point of 6', it may be possible to define a function which has a
unique differential coefficient (i.e. is monogenic) at all points of C
which do not belong to *S'; but the function is not analytic in C in
the sense of Weierstrass.

Such a function is

.,, 00 w n exp(-expw*) f z)= 2 2 2; . . n=ip=oq=o z- p + qi)jn

* Proc. Math. Congress, Cambridge (1912), i. pp. 137-liJ8. Leqons sur
les fonctions mono- genes (1917). The functions are not monogenic
strictly in the sense of\hardsectionref{5}{1} because, in the example quoted, in
working out f z + h) -f(z)]jli, it must be su Dposed that R z + h) and
I z + ]i) are not botli rational fractions.

%
% 100
%

5-6. Laurent's Theorem.

A very important theorem was published in 1843 by Laurent*; it
relates to expansions of functions to which Taylor's Theorem cannot be
applied.

Let G and C be two concentric circles of centre a, of which C is the
inner; and let f z) be a function which is analytic i* at all points
on G and G' and throughout the annulus between G and C. Let a + A be
any point in this ring-shaped space. Then we have \hardsubsectionref{5}{2}{1} corollary)

• ZTTt j cz - a - h liTi j c z - a - h

where the integrals are supposed taken in the positive or
counter-clockwise direction round the circles.

This can be written

We find, as in the proof of Taylor's Theorem, that

f(z)dz.h- r f(z)dz(z-ar+

c(z-a)"+' z-a-h) Jc z-a-h)h +'

tend to zero as n- cc; and thus we have

/(a + h) = cio + a h + ajr + ... + -~ + j + ...,

where + a = - / K, and 6 = 5- . I z - ay'-'f(z) dz.

This result is Laurent's Theorem; changing the notation, it can be
expressed in the following form: If f(z) be analytic on the concentric
circles G and G' of centre a, and throughout the annulus between them,
then at any point z of the annidus f z) can he expanded in the form

f z) = tto -h i z -a) + a.2 z-ay- ...+ J -f y: y • • • •,

where a = . j a.nd b = . [ (t - aT' f t) dt.

An important case of Laurent's Theorem arises when there is only one
singularity within the inner circle G', namely at the centre a. In
this case the circle G' can be taken as small as we please, and so
Laurent's expansion is valid for all points in the interior of the
circle C, except the centre a.

* Comptes Rendus, xvii. (1843), pp. 348-349. t See\hardsectionref{5}{2} corollary 2,
footnote.

X We cannot write a- = fW (a)ln ! as in Taylor's Theorem since /(2) is
not necessarily analytic inside C.

5*6] Taylor's, Laurent's and liouville's theorems 101

Example 1. Prove that

e ' = J x)+zJ x) + z' J x) + ...-irz-J,, x)- ...

1 /"St

where Jn (•*) = s~ I ' ~ " ) •

 TT J

[For the function of z under consideration is analytic in any domain
which does not include the point z=Q; and so by Laurent's Theorem,

g2V z) = a,, + axZ + aoz'-+... + - ->r + ...,

where n = s - I - n and G = i, - . I e z dz,

and where 6' and C" are any circles with the origin as centre. Taking
C to be the circle of

radius unity, and writing = e', we have

1 Z"' "" 1 /' -'f

 -=--, / e:' sin .e- ' irf = jr- / COS (n6 - X H\ n 0) dB, 27ri. Jo
27r y

I sin (/i - i'sin ) tZ vanishes, as may be seen by writing in-cf) for
9. ThiLS

suice

a = J (A'), and \& = ( -)", since the function expanded is unaltered
if -z • l)e written for z, so that 6 = ( - )" Ai(.:i ), and the proof
is complete.]

Example 2. Shew that, in the annulus defined by|a|<|3|<i6|, the
function

r bz i

I'

\ \ {z-a) b-z) can be expanded in the form

*"+.?,* '(? + 6-.) c. * 1.3. ..(2 -1). 1.3... (2i + 2>i-l) /a\' " ' =
,!o 2 .ll l + n)l [b)

The function is one- valued and analytic in the annulus (see\hardsectionref{5}{7}),
for the branch-points 0, a neutralise each other, and so, by Laurent's
Theorem, if C denote the circle \ z\=r, where | o ' < /• < | 6 |, the
coefficient of 2" in the required expansion is

1 f dz ( bz 1

27riJ c '- \ \ {z-a) b-z) ' Putting z = re', this becomes

1 [-' .,,• 1.3... (2i-l) ?*<*•" 1.3... (2i-l)a' -''

the series being absolutely convergent and uniformly convergent with
regard to 6.

The only terms which give integrals different from zero are those for
which k = l + n. So the coefficient of z" is

(2 -1) 1 . 3 ...(2 + 2/i-l) a \ Sn

1 r 'T * 1

2ir J 1=0

2Kl\ 2* + ™.(/+m) ! ¥*"• b" '

Similarly it can be shewn that the coefficient of - is S a'\

%
% 102
%

Example 3. Shew that

Z Z''

1 rsT

where = '" '' "> '''" cos ( - v) sin (9 - ?i(9 o?,

ZTT / 1 /"-"

and i = 5- / e'" + ")'=°' cos (i'-?Osi" - '<9 < <9-

ZTry

5"61. T ie nature of the singularities of one-valued functions.

Consider first a function f z) which is analytic throughout a closed
region 8, except at a single point a inside the region.

Let it be possible to define a function z) such that (i) z) is analj
tic throughout S,

(ii) when a, /( ) = < ( ) + - 4-A . + ...+ "

z - a (z - a)- z - ay-

Then f(z) is said to have a 'pole of order n at a'; and the terms

h, -T + ... + -. are called the principal part of f(z) near a.

z - a. (z - a) (z - aY r r j \ /

By the definition of a singularity \hardsubsectionref{5}{1}{2}) a pole is a singularity.
If n = 1,

the singularity is called a simple pole.

Any singularity of a one-valued function other than a pole is called
an essential singularity.

If the essential singularity, a, is isolated (i.e. if a region, of
which a is an interior point, can be found containing no singularities
other than a), then a Laurent expansion can be found, in ascending and
descending powers of a valid when dk>\ z - a h, where A depends on the
other singularities of the function, and 8 is arbitrarily small. Hence
the ' principal part ' of a function near an isolated essential
singularity consists of an infinite series.

It should be noted that a pole is, by definition, an isolated
singularity, so that all singularities which are not isolated (e.g.
the limiting point of a sequence of poles) are essential
singularities.

There does not exist, in general, an expansion of a function valid
near a non-isolated singularity in the way that Laurent's expansion is
valid near an isolated singularity.

Corollary. If f z) has a pole of order n at a, and z) = z - aYf z) z
a), i\ r a)= lim z-a)' f z), then y\ r z) is analytic at a.

Example 1. A function is not bounded near an isolated essential
singularity.

[Prove that if the function were bounded near z=a, the coefficients of
negative powers of 2 - a would all vanish.]

5-61, 5 -62] Taylor's, Laurent's and liouville's theorems 103

c z\

Example 2. Find the singularities of the function e ~ 7 e - 1 .

At 2 = 0, the numerator is analytic, and the denominator has a simple
zero. Hence the function has a simple pole at 2 = 0.

Similarly there is a simple pole at each of the points mria ( = + 1,
+2, +3, ...); the denominator is analytic and does not vanish for
other values of z.

A.t z = a, the numerator has an isolated singularity, so Laurent's
Theorem is applicable, and the coefficients in the Laurent expansion
may be obtained from the quotient

c c

z- a 2 I (z - aV

Ml+'--" + ...)-l

which gives an expansion involving all positive and negative powers of
z - a). So there is an essential singularity at 2 = a.

Example 3. Shew that the function defined by the series

I wg"- (l +)t- )"-l

 =1 (2 -l) 2 -(l+/i-l)

has simple poles at the points 2 = (1 + ~i)e- ''' / ( =0, 1, 2, ... n
- \; ?i = l, 2, 3, ...). y (Math. Trip. 1899.)

5'62. The 'point at infinity.'

The behaviour of a function /C ') as | ' - oo can be treated in a
similar way to its behaviour as z tends to a finite limit.

If we write z = -,, so that large values of z are represented by small

values of z' in the '-plane, there is a one-one correspondence between
z and z, provided that neither is zero; and to make the
correspondence complete it is sometimes convenient to say that when z
is the origin, z is the * point at infinity.' But the reader must be
careful to observe that this is not a definite point, and any
proposition about it is really a proposition concerning the point / =
0.

Let/(2 ) = 4> z'). Then < z') is not defined at z = 0, but its
behaviour near z = is determined by its Taylor (or Laurent) expansion
in powers of z \ and we define < (0) as lim (/) if that limit exists.
For instance

the function (/) may have a zero of order m at the point 2' =; in
this case the Taylor expansion of ( ') will be of the form

and so the expansion of f z) valid for sufficiently large values of |
.0 j will be of the form

In this case,/(0) is said to have a zero of order m at ' infinity.'

%
% 104
%

Again, the function (f)(2') may have a pole of order m at the point z'
= 0; in this case

and so, for sufficiently large values of \ z\, f(z) can be expanded in
the form

N P

f z) = Az"' + Bz'''-' + Cz'"-- +...+Lz + M+- + - + ....

In this case,/(2 ) is said to have a pole of order m at ' infinity. '
Similarly f z) is said to have an essential singularity at infinity,
if z) has an essential singularity at the point / = 0. Thus the
function e' has an

essential singularity at infinity, since the function e~' or

1 \ 1\ 1

has an essential singularity at z = 0.

Example. Discuss the function represented by the series

2 -, :; 5-5, ( >1).

Z 1

The function represented by this series has singularities at 2=- and
2= - i

 n=\, 2, 3, ...), since at each of these points the denominator of one
of the terms in the series is zero. These singularities are on the
imaginary axis, and have 3 = as a limiting point; so no Taylor or
Laurent expansion can be foi med for the function valid throughout any
region of which the origin is an interior point.

For values of z, other than these singularities, the series converges
absolutely, since the limit of the ratio of the (?i + l)th term to the
?ith is lim (?i+ l)~i a~ = 0. The function is

an even function of z (i.e. is unchanged if the sign of z be changed),
tends to zero as I 2 I - Qc, and is analytic on and outside a circle
C of radius greater than unity and centre at the origin. So, for
points outside this circle, it can be expanded in the form

h + j + b+

where, by Laurent's Theorem,

'" 27nJ c =o ! a-2 + 22

This double .series converges absolutely when | 2 | > 1, and if it be
rearranged in powers of 2 it converges uniformly.

Since the coefficient of 2 ~ Ms 2 -; and the only term which
furnishes a non-

H=o n !

zero integral is the term in z~, we have

(\ )fc-ia-2A- dz

b'k • 1 - .

Itti J c =o, fo n\ a2*

5-63, 5-64] Taylor's, Laurent's and liouville's theorems 105

Therefore, when | 2 | > 1, the function can be expanded in the form

ill

e"' e * e

The function has a zero of the second order at infinity, since the
expansion begins with a term in z~' .

5-63. Liouville's Theorem*.

Let f z) he analytic for all values of z and let \ f z)\ < K for all
values of z, where K is a constant (so that \ f z) is hounded as | r |
- > x ). Then f z) is a constant.

Let z, z' be any two points and let C be a contour such that z, z are
inside it. Then, by\hardsubsectionref{5}{2}{1},

take C to be a circle whose centre is z and whose radius is p 2 | / -
s |; on

C write \ \ = z pe' \ since jf - /| 2P when is on C it follows from §
4-62 that

= 2\ z -z\ Kp-K Make p- cc keeping z and z' fixed; then it is obvious
that/(/) -f(z) = 0; that is to say, f(z) is constant.

As will 1)0 seen in the next article, and again frequently in the
latter half of this volume (Chapters xx, xxi and xxii), Liouville's
theorem furnishes short and convenient proofs of some of the most
important results in Analysis.

564. Functions with no essential singularities.

We shall now shew that the only one-valued functions luhich have no
singidarities, except poles, at any jjoint (including oo ) are
rational functions.

For let f(z) be such a function; let its singularities in the finite
part of the plane be at the points Cj, c-j, ... Ck'. and let the
principal part \hardsubsectionref{5}{6}{1}) of its expansion at the pole Cr be

+ 1 ::; + . . . + - 7 -

Z - Cr (Z - Crf '" (Z - CyJ

Let the principal part of its expansion at the pole at infinity be

a- z + a-iZ + ... + anZ \

if there is not a pole at infinity, then all the coefficients in this
expansion

will be zero.

* This theorem, which is really due to Cauchy, Comptes Rendus, xix.
(1844), pp. 1377, 1378, was given this name by Borchardt, Journal fvr
Math, lxxxviii. (1880), pp. 277-310, who heard it in Liouville's
lectures in 1847.

%
% 106
%

Now the function

has clearly no singularities at the points c, c.j, ... Ck, or at
infinity; it is therefore analytic everywhere and is bounded as \ z -
>cc, and so, by Liou\ dlle's Theorem, is a constant; that is,

/(.) = + .. + a..= + . . . +,,.. + I l i + ( . + • • + (i J •

where C is constant; f(z) is therefore a rational function, and the
theorem is established.

It is evident from Liouville's theorem (combined with § 361 corollary
(ii)) that a function which is analytic everywhere (including oc ) is
merely a constant. Functions which are analytic everywhere except at
oc are of considerable importance; they are known as integral
functions*. Examples of such functions are e, sin z, e . From\hardsectionref{5}{4}
it is apparent that there is no finite radius of convergence of a
Taylor's series which represents an integral function; and from the
result of this section it is evident that all integral functions
(except mere polynomials) have essential singularities at oo .

5"7. Many-valued functions.

In all the previous work, the functions under consideration have had a
unique value (or limit) corresponding to each value (other than
singularities) of .

But functions may be defined which have more than one value for each
value of z; thus if 2- = r (cos 6 + i sin 6), the function 2- has the
two values

r* (cos 1(9 + t sin (9), r jcos \ 0 + lir) + i sin \ (6 + 27r)|;

and the function arc tan x (x real) has an unlimited number of values,
viz. Arc tan x + mr, where - tt < Arc tan x <- 'tt and n is any
integer; further

examples of many- valued functions are log z, z, sin z~).

Either of the two functions which z represents is. however, analytic
except at = 0, and we can apply to them the theorems of this chapter;
and the two functions are called ' branches of the many-valued
function z'-.' There will be certain points in general at which two or
more branches coincide or at which one branch has an infinite limit;
these points are called ' branch-points.' Thus z- has a branch-point
at; and, if we consider the change in z as z describes a circle
counter-clockwise round 0, we see that 6

* Vxench, fonction entilre; Germa,n, game Funktion.

57] TAYLORS, Laurent's and ltouvtlle's theorems 107

increases by 27r, r remains unchanged, and either branch of the
function passes over into the other branch. This will be found to be a
general characteristic of branch-points. It is not the purpose of this
book to give a full discussion of the properties of many-valued
functions, as we shall always have to consider particular branches of
functions in regions not containing branch- points, so that there will
be comparatively little difficulty in seeing whether or not Cauchy's
Theorem may be applied.

Thus we cannot apply Cauchy's Theorem to such a function as z when the
path of iutegi-ation is a circle surrounding tlie origin; but it is
permissible to apply it to one of

the branches of z when the path of integration is like that shewn in §
6 '24, for through- out the contour and its interior the function has
a single definite value.

Example. Prove that if the different values of a, corresponding to a
given value of z, are represented on an Argand diagram, the
representative points will be the vertices of an equiangular polygon
inscribed in an equiangular spiral, the angle of the spiral being
independent of a.

\addexamplecitation{Math. Trip. 1899.}

The idea of the different braiickes of a function helps us to
understand such a paradox as the following.

Consider the function y = ' i

for which ~=x log x).

AVhen x is negative and real, is not real. But if x is negative and of
the form

P (where p and q are positive or negative integers), y is real.

Z'j -f- 1

If therefore we draw the real c\ irve

we have for negative values of ./ a set of conjugate points, one point
corresponding to each rational value of x with an odd denominator;
and then we might think of proceeding to form the tangent as the limit
of the chord, just as if the curve were continuous; and

thus --, when derived from the inclination of the tangent to the axis
of x, would appear

to be real. The question thus arises, Why does the ordinary process of
differentiation

give a non-real value for - ? The explanation is, that these conjugate
points do not all

arise from the same branch of the function y = x' . We have in fact •

y - 5

where k is any integer. To each value of k corresponds one branch of
the function y. Now in order to get a real value of y when x is
negative, we have to choose a suitable value for k : and this value of
k varies as we go from one conjugate point to an adjacent one. So the
conjugate points do not represent values of y arising from the same
branch of the

function y=x, and consequently we cannot expect the value of - when
evaluated

for a definite branch to be given by the tangent of the inclination to
the axis of x of the line joining two arbitrarily close members of the
series of conjugate points.

%
% 108
%

EEFEREXCES.

E. GouRSAT, Cours dJ Analyse, ii. (Paris, 1911), Chs. xiv and xvi.

J. Hadamard, La Serie de Taylor et son prolongement analytique
(Scientia, 1901).

E. LiNDELOF, Le Calmd des Residus (Paris, 1905).

C. J. DE LA Vallee Poussin, Covrs d' Analyse Infinitesimale, i. (Paris
and Louvain, 1914), Ch. X.

E. BoREL, Lecons stir les Fonctions Entieres (Paris, 1900).

G. N. Watson, Complex Integration and Cauchy's Theorem (Camb. Math.
Tracts, no. 15, 1914).

Miscellaneous Examples.

1. Obtain the expansion

/(.)=/( ) + 2 1 -/ (- + -3-3-r/ ( -2-j + -25751- ' \ \ r-] ' and
determine the circumstances and range of its vaUdity.

2. Obtain, under suitable circumstances, the expansion

+ .... (Corey, Ann. of Math. (2), i. (1900), p. 77.)

3. Shew that for the series

  1

)i=o -r~

the region of convergence consists of two distinct areas, namely
outside and inside a circle of radius unity, and that in each of these
the series represents one function and represents it completely.

(Weierstrass, Berliner Monatsherichte., 1880, p. 731; Ges. Werke, 11.
(1895), p. 227.)

4. Shew that the function

2 s"'

tends to infinity as 2- -exp i-niplm !) along the radius through the
point; where m is any integer and p takes the values 0, 1, 2, ..'.
(ni I - 1).

Deduce that the function cannot be continued beyond the unit circle.

(Lerch, Sitz. BOhm. Acad., 1885-6, pp. 571-582.)

5. Shew that, if z-- 1 is not a positive real number, then

-". •.s:)"<'-")- /:'=-('-'- '-**-

(Jacobi and Scheibner.)

Taylor's, Laurent's and liouville's theorems 109

6. Shew that, if s - 1 is not a positive real number, then

,,,, m, m(m+ ) to (m + 1) ... (to + ?i - 1)

(1-2) - =l+-2 + - 2 "" +-+ -1 .

n : J

(Jacobi and Scheibner.)

7. Shew that, if z and 1-2 are not negative real numbers, then

  Jo m+l [ m + 3 (m+3) ... (m + 27i- 1) J

+ "') (m+l)(w + 3)...(m + 27i-l)Jo ' -

(Jacobi and Scheibner.)

8. If, in the expansion of (a + i2 + a22 )'" by the multinomial
theorem, the remainder after n terms be denoted by R (2), so that

(a4- i2 + a2S-)'" = o + - i2 + -'-l2-"" + --- + n-iS"'"' + (s), shew
that

/4(.) = ( + a.. + ..r j, - (a + .M+a, r " ' •

9. If (ao + i2 + a22 )""'"' /'(ao + i< + 2 -)'"o?<

y

(Scheibner.)

be expanded in ascending powers of 2 in the form

Jl2 + .4222+...,

shew that the remainder after n-\ terms is

(ao + ai2 + a22 )~'"" I (ao + a, + a2 )'" o .i-(2?'i + ? + 1) 2 -i
""'c C

(Scheibner*.)

10. Shew that the series

where X (2)= - 1 +2- f, + I",- - + (- )", '

and where (2) is analytic near 2 = 0, is convergent near the point 2 =
; and shew that if the sum of the series be denoted by/(2), then/(2)
satisfies the differential equation

/'(--)=/( )-0(4

(Pincherle, Rend, dei Lincei (5), v. (1896), p. 27.)

11. Shew that the arithmetic mean of the squares of the moduli of all
the values of the series "2 0, on a circle |2| = r, situated within
its circle of convergence, is equal

 i=0

to the sum of the squares of the moduli of the separate terms.

(Gutzmer, Math. Ann. xxxii. (1888), pp. .596-600.)

* The results of examples 5, 6 and 7 are special cases of formulae
contained in Jacobi's dis- sertation (Berlin, 182.5) published in bis
Ges. Werke, in. (1884), pp. 1-44. Jacobi's formulae were generalised
by Scheibner, Lelpziger Berichte, xlv. (1893), pp. 432-443,

%
% 110
%

12. Shew that the series

2 e-2(am) -m-l m = l

converges when | 2 | < 1; and that, when a > 0, the function which it
represents can also be represented when | s ! < 1 by the integral

/ay f" e-"" dx n-J Jo ex\ 2 x '

and that it has no singularities except at the point z=l.

(Lerch, Monatshefte fiir Math, und Phys. viii.) 13. Shew that the
series

2

\ \,, 2 r z\ zj \

 z + z )-V- 2 ( i\ 2 \ 2y'zi)(2v + iv'zif' iv-2v'z- i) 2v + 2v'z-'
xf]'

in which the summation extends over all integral values of v, v',
except the combination (i' = 0, v' = 0), converges absolutely for all
values of z except purely imaginary values; and that its sum is + 1
or - 1, according as the real part of z is positive or negative.

(Weierstrass, Berliner Monatsherichte, 1880, p. 735.)

14. Shew that sin \ u ( + -)[ can be expanded in a series of the type

a, + aiZ + aoy-+...+- + j, + ...,

in which the coefficients, both of s" and of z~'\ are

1 / 2t

-- I sm (2m cos ) cos Ji o? . •2ir J ) •

n=i n-z' + a-

shew that/ (2) is finite and continuous for all real values of z, but
cannot be expanded as a Maclaurin's series in ascending powers of z;
and explain this apparent anomaly.

[For other cases of failure of Maclaurin's theorem, see a posthumous
memoir by Cellerier, Bidl. des Set. Math. (2), xiv. (1890), pp.
145-599; Lerch, Journal fiir Math. cm. (1888), pp. 126-138;
Pringsheim, Math. Ann. XLii. (1893), pp. 153-184; and Du Bois
Reymond, Miinehener Sitzungsberichte, vi. (1876), p. 235.]

16. If f z) be a continuous one- valued function of z throughout a
two-dimensional region, and if

   f z)dz =

h

for all closed contours C lying inside the region, then f z) is an
analytic function of z throughout the interior of the region.

[Let a be any point of the region and let

F z)=\'f z)dz

J a

It follows from the data that F z) has the unique derivate f z). Hence
F z) is analytic \hardsectionref{5}{1}) and so \hardsubsectionref{5}{2}{2}) its derivate /(z) is also
analytic. This important converse of Cauchy's theorem is due to
Morera, Rendiconti del R. 1st. Lomhardo Milano), xxil. (1889), p.
191.]


\chapter{The Theory of Residues; Application to the Evaluation of Definite Integrals} 

6"1. Residues. - '-  

If the function /( r) has a pole of order m at 2 = a, then, by the definition 
of a pole, an equation of the form 

where   (z) is analytic near and at a, is true near a. 

The coefficient a i in this expansion is called the residue of the function 
f z) relative to the pole a. 

Consider now the value of the integral 1 f z)dz, where the path of 

J a 

integration is a circle* a, whose centre is the point a and whose radius p is so 
small that    z) is analytic inside and on the circle. 

r m f dz r 

We have f z)dz= S a\ r , w+   z)dz. 

J a' r = l J a\ 2 ~ (i) J a. 

Now [ (f) z)dz = by § 52 ; and (putting z-a = pe' ) we have, if r l, 

gd-r) ie' 
1 -r 



JA -aV Jo p'-e"  '  Jo 



,,  z - ay J p''e 
But, when ?• = 1, we have 



27r 

= 0. 





r \ d  p" 

Hence finally I f(z) dz = 2 



idd = 27rz. 



Now let C be any contour, containing in the region interior to it a number 
of poles a,h,c,... of a function f z), with residues a\ i, 6\ i, c\ i, ... respec- 
tively : and suppose that the function f z) is analytic throughout C and its 
interior, except at these poles. 

Surround the points a, b,c, ... by circles a, /3, 7, ... so small that their 
respective centres are the only singularities inside or on each circle ; then the 
function f z) is analytic in the closed region bounded by C, a, /3, 7, .... 

* The existence of such a circle is implied in the definition of a pole as an isolated 
singularity. 



112 THE PROCESSES OF ANALYSIS [CHAP. Yl 

Hence, by § 5-2 corollary 3, 

! f z)dz=\ f z)dz+\ f z)dz+... 

J C   a J /3 

= 27rm\ i + 27n'6\ i + — 

Thus we have the theorem of residues, namely that if f z) he analytic 
throughout a contour C and its interior except at a number of poles inside the 
contour, then 

f f z)dz = 27ri R, 
J c 

where XR denotes the sum of the residues of the function f (z) at those of its 
poles which are situated within the contour G. 

This is an extension of the theorem of § 5*21. 

Note. If a is a simple pole oi f z) the residue of f z) at that pole is lim   z-a)f z) . 
6 "2. The evaluation of definite integrals. 

We shall now apply the result of § 6*1 to evaluating various classes 
of definite integrals ; the methods to be employed in any particular case may 
usually be seen from the following typical examples. 

6"21. The evaluation of the integrals of certain periodic functions taken 
between the limits and 27r. 



An integral of the type 



2n 



R (cos e, sin 6) dd, 



where the integrand is a rational function of cos 6 and sin 6 finite on the 
range of integration, can be evaluated by writing e'  = z ; since 

cos   - 1 (  + z-% sin   = — . (2 -  -1), 

the integral takes the form I S z)dz, where 8 z) is a rational function of z 

J c 
finite on the path of integration C, the circle of radius unity whose centre is 

the origin. 

Therefore, by § 61, the integral is equal to  wi times the sum of the residues 
of S  z) at those of its poles which are inside that circle. 

Example 1. If <p < 1, 

p-  de   f dz 

jo 1 -2/>cos +/>2 j f;i l-pz) z-p)' 

The only pole of the integrand inside the circle is a simple pole at p ; and the residue 
there is 

lim * -r 



, i  pz) z-p)  •(l- 2)  



6 '2-6 -22] THE THEORY OF RESIDUES 113 

de 277 



Hence 



jo 1  



  Example 2. If <p < 1, 

j 1 - '2p cos 2 + 52 -J  iz\ 2' ' 2' ' ) (1 -jt)22) (1 \ p5-2) 

= 27r2i?, 

(26+1)2 

where 2  denotes the snm of the residues of . . ,  sr -s   at its poles inside C ; these 

4s-"' (1 — pz )  z  — p) 

1 1 l+p  + p* ( 3+l)2 ( 3+XY2 

poles are 0, -p , p  ; and the residues at them are  -o ,  C —  , ,   "C,-. ;tt ; 

and hence the integral is equal to 

Tr l-p+p ) 

Example 3. If n be a positive integei", 

/ 2t 2ir /"2t 

I e '99cos(n -sin )o?(9= — , / e* "  sin (?i<9- sin  )rf  = 0. 

Example 4. If a > 6 > 0, 

pT de 2na pT 0?  \  7r(2a + 6) 

i  (  + 6cos )2 (a2\ t2)l' jo  a + hcoii 6f a  a + hf' 

622. T/ie evaluation of certain types of integrals taken between the limits 

— 00 and + 00 . 

We shall now evaluate I Q (x) dx, where Q  z) is a function such that 

J — X 

(i) it is analytic when the imaginary part of z is positive or zero (except at a 
finite number of poles), (ii) it has no poles on the real axis and (iii) as | 2  |—   00 , 
zQ z) >Q uniformly for all values of arg  such that O arg  Tr; provided 
that (iv) when x is real, xQ x)—>0, as ic— > + x , in such a way* that 

I Q  x) dx and 1 Q  x) dx both converge. 

Jo J -00 

Given e, we can choose p  (independent of arg2 ) such that \ zQ z) \ < ejir 
whenever |  ; | > po and   arg z  ir. 

Consider I Q(z)dz taken round a contour C consisting of the part of the 

real axis joining the points + p (where p > po) and a semicircle F, of radius p, 
having its centre at the origin, above the real axis. 

Then, by § 6"1, I Q (z) dz — 27ri'ER, where 2i2 denotes the sum of the 

J c 

residues of Q z) at its poles above the real axisf. 

* The condition xQ (x)   is not in itself sufficient to secure the convergence of I Q (x) dx ; 

consider Q (x) = (x log .t)~i. 

t Q(z) has no poles above the real axis outside the contour. 

W. M. A. 8 



114 THE PROCESSES OF ANALYSIS [CHAP. VI 

Therefore I f ' Q (z) dz - 27riSi? ' = I Q (z) dz . 

\ J -p \ J T 

In the last integral write z = pe , and then 

11 Q(z)dz\ = \ rQ (pe* ) pe 'idd 

\ j r I 1 Jo 

< Wei'JT)de 
Jo 

= e, 

by § 4-62. 

Hence lim 1 Q  z) dz =  iri R. 

p- -oo J -p 

But the meaning of | Q (x) da; is lim I Q (x) dx ; and since 
lim 1 Q (a;) rf  and lim Q (x) dx both exist, this double limit is the 

(r- -oo Jo   p- x> J -p 

same as lim Q (x) dx. 

p- oo -J — p 

Hence we have proved that 

Q(x)dx= 2'7rilR. 



This theorem is particularly useful in the special case when Q(x) is a 
rational function. 

[Note. Even if condition (iv) is not satisfied, we still have 

I  Q x) + Q(-a;) dx=\ im T Q x)dx = 2Tri2R.] 

./ p f:o J —P 

Exaynple 1. The only pole of  z + ) ~  in the upper half plane is a pole at   =    with 



3 

residue there -  ttt i- Therefore 
lb 



/"  dx \  3 

Example 2. If a > 0, 6 > 0, shew that 

r= - x*dx \  IT 
•'-• (a + bx )  I6aib  

Example 3. By integrating / e- - dz round a parallelogram whose corners are 
- R, R, R + ai, - R + ai and making R- qc , shew that, if X > 0, then 

/ e-'' ' coii 2Xax)dx = e- <>-'' I e->'x-dx = 2X-h e- < - j e-< dx. 

6221. Certain infinite integrals involving sines and cosines. 

If Q(z) satisfies the conditions (i), (ii) and (iii) of § 6-22, and ni > 0, then 
Q (z) e""'  also satisfies those conditions. 



THE THEORY OF RESIDUES 



115 



6-221, 6-222] 

Hence I  Q (a;) e' ''' + Q  - x) e'"''''] dx is equal to  ttiIR', where 2E' 

means the sum of the residues of Q  z) e   at its poles in the upper half plane ; 
and so 

(i) If Q  x) is an even function, i.e. if Q (— x) — Q (x), 

/• CO 

I Q (x) cos (jyix) dx = ttiSR. 

Jo 

(ii) If Q (x) is an odd function, 

Q (x) sin (inx) dx = ttSR'. 



6'222. Jordan's lemma*. 

The results of § 6'221 are true if Q  z) be subject to the less stringent 
condition Q(z)—>0 uniformly when O arg Tr as \ z:—>oo in place of the 
condition zQ z)- 0 uniformly. 

To prove this we require a theorem known as Jordan's lemma, viz. 

If Q(z) >0 iDiiformly with regard to argz as j  — ►x v)hen O arg Tr, 
and if Q (z) is analytic luhen both \ z'.>c a constant) and   arg z  w, then 

liin (\ e'"" Q (z) dz) = 0, 

where T is a semicircle of radius p above the real axis with centre at the origin. 

Given e, choose po so that \ Q z)\ <  ejir when \ z  p,  and   arg z   tt; 
then, if p > po. 



 el'Tr)pe-' p ''">dd 



e"""  Q  z) dz 
But |emipeose| = i  and so 

e""  Q  z) dz 



= (2€/7r)   ' pe-' '> i" f/6'. 



Now sin 6   2 /7r, whenf "S     tt, and so 



e' '  Q  z) dz 



< (2e/7r) '" pe'-' p l' dd 
Jo 



= (2e/7r).(7r/2m) 
< e/m. 



\  g-2mp0/7r 



* Jordan, Cours d'Aiiahjse, ii. (1894), pp. 285, 286. 

t This inequality appears obvious when we draw the graphs )/ = sin.r, y = 2xjir; it may be 
proved by shewing that (sin 6)1\$ decreases as 6 increases from to iir. 

8—2 



116 THE PROCESSES OF ANALYSIS [CHAP. VI 



Hence lim | e'"'' Q (z) dz = 0. 



This result is Jordan's lemma. 

Now 

[" |e *  Q (x) + e-' *  Q (- a;)] da; = 2TrilR' - I e""'  Q (z) dz, 

Jo J T 

and, making p—>oo, we see at once that 

 e"    Q (x) + e-'"''  Q (- x)] dx = l-rri R, 

Jo 

which is the result corresponding to the result of § 6"221. 

Example 1. Shew that, if a > 0, then 

/"" eos r T ,\  ?r \   
'o x  + a  2a 





Example 2. Shew that, if a   0, 6 0, then 

cos 2ax — cos 'ihx 



i: 



dx=n  h-a) 



X'- 

(Take a contour consisting of a large semicircle of radius p, a small semicircle of 
radius 8, both having their centres at the origin, and the parts of the real axis joining their 
ends ; then make p-  cxj , S- 0.) 

Example 3. Shew that, if 6 > 0, m   0, then 

I ,— ; — T TZTi cos mxclx = — j ~  Sb'  - a  — mb (36- + a )|. 
Jo  x  + o' y 46-'   

Example 4. Shew that, if  ' > 0, a > 0, then 

/" ""   sin ax , , , 
I — 5 — ,, ax=i7re~'"K 

Example 5. Shew that, if m   0, a > 0, then 



/ 



sin mx , tt Tre""*" / 2 
.r(j??- + a2)2 2a'' 4a  \ a 



(Take the contour of example 2.) 

Example 6. Shew that, if the real part of s be positive, 

[  e-*-e-'')j=\ ogz. 
[We have 

Ji>   ' t s o,p  [Js i Js t J 

lim -! I — dt— I - — du 



a- .0, p-*-x I./ 5   J 5c 

= lim I / " —-dt-] —- dt\ , 
since i~' e~' is analytic inside the quadrilateral whose corners are 8, 8z, pz, p. 



6"23, 6*24] THE THEORY OF RESIDUES 117 

Now I t~  e~ dt- -0 a,s, p- cc when (2)>0; and 

/ ~ t-  e- dt = \ ogz- I " t-  l-e-f) dt- \ ogz, 

since i''   1 -e-*)- l as t- 0.] 

6'23. Principal values of integrals. 

It was assumed in §§ 6'22, 6 '221, 6*222 that the function Q  x) had no poles on the real 
axis ; if the function has a finite number of simple poles on the real axis, we can obtain 
theorems corresponding to those already obtained, except that the integrals are all principal 
values (§ 4'5) and 2/2 has to be replaced by S/  +  S/s'q, where 2  means the sum of 
the residues at the poles on the real axis. To obtain this result we see that, instead of 
the former contour, we have to take as contour a circle of radius p and the portions of the 
real axis joining the points 

-p, a-8i; a + hi,h-b~2\ b + 8o, c-8 , ... 

and small semicircles above the real axis of radii Sj, So, ... witli centres a, b, c, ..., where 
a, b, c, ... are the poles of Q (2) on the real axis ; and then we have to make Sj, 8. , ... - 0 ; 
call these semicircles yj, y.,, Then instead of the equation 

I Q z)dz+ I Q (2) dz = 27ri2R, 

we get F j Q (2) dz + -2 Ynn j Q (2) dz+ j Q (2) dz = -Irri  It. 

Let a' be the residue of Q z) at a ; then writing z = a + 8ie'  on y  we get 

j Q (2) dz= (  Q a + 8i e' ) Sj e'6 id0. 
But Q  a + 81 6 )816' - a uniformly as 8,- 0; and therefore lim | Q z)dz= -iria' ; 

P y Q z)dz+( Q z)dz = 27ri2R + ni2R,,, 



we thus get 



-p 
and hence, using the arguments of i  (5-22, we get 



P j Q x) dx = 2m  2R +  2 /?o) 



UA 



The reader will see at once that the theorems of   6"221, 6-222 have precisely similar 
generalisations. 

The process employed above of inserting arcs of small circles so as to diminish the area 
of the contour is called indenting the contour. 

r "  
.  6"24. Evaluation of integrals of the fovTii \ af'~ Q x)dx. 

J 

Let Q x) be a rational function of x such that it has no poles on the 
positive part of the real axis and x Q x)—¥  both when a;— >0 and when 



118 



THE PROCESSES OF ANALYSIS 



[chap. VI 



Consider l(— zY~'  Q z)dz taken round the contour G shewn in the figure, 



consisting of the arcs of circles of radii 
p, S and the straight lines joining their 
end points ; (— zY"'  is to be interpreted 
as 

exp  ( a-1) log (- z)] 
and 

log (-  ) = log •  ' +   arg (- z), 

where — tt   arg (—  )   tt ; 

with these conventions the integrand is 
one-valued and analytic on and within 
the contour save at the poles of Q  z). 

Hence, if S?' denote the sum of the 
residues of (— zY~'  Q (z) at all its poles, 




[ (- zy-' Q (z) dz = 27ri'Zr. 



On the small circle write — z= Se , and the integral along it becomes 
— I (— zyQ z)id6, which tends to zero as 8— >0. 



On the large semicircle write — z = pe' , and the integral along it becomes 
— I  — sT Q (2) idO, which tends to zero as p— > x . 

On one of the lines we write — z = xe"\ on the other — z = xe~' ' and 
(-zy~  becomes a; - e±'"-i>'''. 

Hence 

lim r [w''-' e- <"-'' ' * Q (x) - x -'e'< - > "' Q (x)] dx = l-Trilr ; 

(S- .0, p oc ) J 5 

and therefore I af ~'  Q (x) dx = tt cosec (air) Sr. 

.'0 

Corollary. If Q x) have a number of simple poles on the positive part 
of the real axis, it may be shewn by indenting the contour that 

P j a;"~  Q (x) dx — tt cosec (citt) S?' — tt cot ((/tt) 2r', 
. 

where 1r' is the sum of the residues of z"'-  Q (z) at these poles. 
Example 1. If < a < 1, 

I dx=ir cosec arr, P I dx=iv cot air. 

Jii  + X jn  x 



6'3, 6*31] THE THEORY OF RESIDUES 119 

Example 2. If < 2 < 1 and - tt < n < tt, 



  



fz-l  gi(2— l)a 

, — .'di=—. . (Minding.) 



Example 3. Shew that, if - 1 < s < 3, then 



/; 



  dx - -'  



/: 



(Euler.) 



   x' 'f ' ~4cos 7r2' 
Example 4. Shew that, if — 1 < p < 1 and - tt < X < tt, then 

x~P dx \  TT sin X 
1 +2x cos X + X- sin pn sin X 

6 "3. Cauchy's integral. 

 V'e shall next discuss a class of contour-integrals which are sometimes found useful 
in analytical investigations. 

Let Cbe a contour in the 2-plane, and let/(s) be a function analytic inside and on C. 
Let (2) be another function which is analytic inside and on G except at a finite luimber 
of poles ; let the zeros of <  z) in the interior* of C be a,, 02, ..., and let their degrees of 
multiplicity be rj, ro, ... ; and let its poles in the interior of C be 6j, ho, ..., and let their 
degrees of multiplicity be 5, , §2)  ••• 

Then, by the fundamental theorem of residues, - — . I fiz) dz is equal to the sum 

 2ni J c' (f> z) 

of the residues of--—   ' at its poles inside C. 
<i> z) 

Now— —- - can have singularities only at the poles and zeros of 0(s). Near one 

of the zeros, say otj , we have 

0(2) = iI(2-ai)'-i-|-5(2-ai)'-. + i-f-.... 
Therefore < ' (2) = Ar   z - a- Yi -1 + 5 (r, + 1 ) (s - ai) -i + . . . , 

and /(2)=/( i) + (2- i)/'( i) + .... 

Therefore P  - '  ] is analytic at a, . 

Thus the residue of j   - , at the point 2 =  ,, is rif ai). 
Similarly the residue at 2 = 61 is — Si/(6j); for near 2 = 61, we have 

( (2) = C(2-6,j- . + Z>(2-6,)-*, + l-l-..., 

and f z)=f bi) +  z-b,)f' b,) + ..., 

so f!'  + '-  is analytic at i . 
0(2) 2-61 

Hence i J/ '  f  clz = 2r,fia,) - 2s,f b,), 

the summations being extended over all the zeros and poles of (f) (2). 

6'31. The mimher of roots of an equation contained loithin a contour. 
The result of the preceding paragraph can be at once applied to find how many roots of 
an equation cj)  z) = lie within a contour C. 

For, on putting /(2) = 1 in the preceding result, we obtain the resvilt that 



z) 



27rilc  z)' ' 

is equal to the excess of the number of zeros over the number of poles of (2) contained in 

the interior of C, each pole and zero being reckoned according to its degree of multiplicity. 

* (f) ( ) must not have any zeros or poles on C. 



120 THE PROCESSES OF ANALYSIS • [CHAP. VI 

Example 1, Shew that a polynomial <  (2) of degree m has m roots. 

Let (  (2) = ao2™ + ais'"~  + . .. +  , , ( o=t=0)- 

rpj gj ) \  >  og'"~  + •••+    -1 

( (2) ao2'  + ...+ ,  

Consequently, for large values of | 2 | , 

Thus, if C be a circle of radius p whose centre is at the origin, we have 

27rt J c (p  ) 27r  \ / c 2 27ri /   \ z / 2itI J c V J 

But, as in § 6-22, j (\ \   dz O 

as p- ao ; and hence as (f) z) has no poles in the interior of C, the total number of 
zeros of (f) (2) is 

lim -— . I  - ( 2 = TO. 

Example 2. If at all points of a contour C the inequality 

l fcS*|>l o +  i2 + ... + ai-i~ * ~  +  fc+i2' i + ... + a, 2'"j 
is satisfied, then the contour contains /(• roots of the equation 
a , 2'" +  , \ ! 2'  - 1 + . . . +  ! 2 +  o = 0. 
For write / (2) = a z''' + a   \  1 2"'- - 1 + . . . + ai2 + ao . 

Then /(2) =  ,2 i   -- + ...+a..,i2  -fa,-i2> -' + ...+ao\ 

= a,2' (l + tO, 
where | f/"!  a < 1 on the contour, a being independent* of 2. 
Therefore the number of roots of f z) contained in C 

27 jc/(2) 2injc\ s l+U dzj 

But I — — 2T!-i; and, since | C/'|<1, we can expand (1 + U)~'  in the uniformly cou- 
  vergent series 

Therefore the number of roots contained in C is equal to i: 
Example 3. Find how mauy I'oots of the equation 

26 + 62+10=0 
lie in each quadrant of the Argaud diagram. (Clare, 1900.) 

* I C/ 1 is a continuous function of z on C, and so attains its upper bound (§ 3-62). Hence its 
upper bound a must be less than 1. 



> 



6 "4] THE THEORY OF RESIDUES 121 

6 "4. Connexion between the zeros of a function and the zeros cf its derivate. 

Macdonald* has shewn that if fiz) he a fvMction of z analytic throughout the interior of 
a single closed contour C, defined hy the equation \ f z) i = 3  where M is a constant, then the 
number of zeros of f z) in this region exceeds the number of zeros of the derived function 
f  z) in the same region hy unity. 

On C \ etf z) = 3fe'e ; then at points on C 

/  =*'"-|-  '•" = ' "" 'S-(I) - 

Hence, by § 6'31, the excess of the number of zeros oi f z) over the number of zeros 
of/'(s) inside + C is 

27   j c f z) ~ 27ri j cf (2) 27rt j c U ' / dz) ' • 

Let s be the arc of C measured from a fixed point and let y  he the angle the tangent to 
C makes with Ox ; then 

1 f (dH /de\ , 1 r de'] 

-2 ija [dJ  I dz)  = - 2 ?rS dz]c 

I r, d0 , dzl 

= -2 ir ds- ''Sdsjc- 

dB 
Now log , is purely real and its initial value is the same as its final value ; and 

dz 
log =i\ j/ ; hence the excess of the number of zei'os oi f z) over the number of zeros of 

/' (s) is the change in y rj'iiT in describing the curve C ; and it is obvious J that if C is any 
ordinary curve, >//  increases by 27r as the point of contact of the tangent describes the 
curve C ; this gives the required result. 

Example 1. Deduce from Macdonald's result the theorem that a polynomial of degree 
n has n zeros. 

Example 2. Deduce from Macdonald's result that if a function/ (2), analytic for real 
values of z, has all its coefficients real, and all its zeros real and different, then between 
two consecutive zeros oi f z) there is one zero and one only of/' (2). 



KEFERENCES. 

M. C. Jordan, Cours dJ Analyse, 11. (Paris, 1894), Ch. vi. 

E. GouRSAT, Cours d' Analyse (Paris, 1911), Ch. xiv. 

E. LiNDELOF, Le Calcid des Residus (Paris, 1905), Ch. il. 

* Froc. London Math. Soc. xxix. (1898), pp. 576, 577. 

t /' ( 2) does not vanish on C unless C has a node or other singular point ; for, if f= <p + i\ p, 

where <A and xL are real, since i J- - /- , it follows that if /'(2) = at any point, then 
  ax dy 

  ,   , ~,   all vanish ; and these are sufficient conditions for a singular point on 
ox dy ox ay 

X For a formal proof, see Proc. London Math. Soc. (2), xv. (1916), pp. 227-242. 



122 



THE PROCESSES OF ANALYSIS 



[chap. VI 



Miscellaneous Examples. 

1. A function (f) (2) is zero when 2 = 0, and is real when z is real, and is analytic when 
2 I   1 ; if f(x, y) is the coefficient of i in  x + iy), prove that if - 1 <  < 1, 



'27r 



1 - 'ix cos 



  - / aoa e, sin 6) de = n4> ( ). 



(Trinity, 1898.) 



2. By integrating  -  — round a contour formed by the rectangle whose corners are 
e"' ' — 1 



0, R, B + i, i (the rectangle being indented at and and making R- xi , shew that 

(Legendre.) 



/. 



sin ax , 1 e" + 1 
ax = - 



3ttX 



1 



4 e -l 



1 
2a 



3. By integi-ating log (-2) Q z) round the contour of   6-24, where Q z) is a rational 
function such that 2 (2) 0 as ; 2 |- 0 and as ] 2 j -  x , shew that if Q (2) has no poles 

on the positive part of the real axis, I Q (.r) dx is equal to minus the sum of the 

.' 
residues of log  -z)Q (2) at the poles of Q (z) ; where the imaginary part of log ( - 2) lies 
between ± tt. 

4. Shew that, if a > 0, 6 > 0, 



f dv 

I e cos6xgii  (a sin hx)-  =-|7r(e -l). 



5. Shew that 



/ 



a sin 2x 



- — -, xdx = - TT log (1 +0), ( - 1< a < 1; 

l-2acos2.r + a2 4 b\ t j, \ 



-7rlog(l + a-i),  a?>r, 



6. Shew that 



(Cauchy.) 



f sva.<i>iX sintboX sind) .r snia r , "  . . , 

/ 2\ !\   --=— ... — - --  COS ai r . . . COS a, a7 a.r = — m, (Do ... (  , 

J Q X X X X '•I  -   

if < i) 02)  •• ni oi) a-i) •••o-m be real and a be positive and 

 >1< 1 | + |< 2i+---+l0nl + lai|+ ...+|a J. 

(Stormer, Acta Math, xix.) 

7. If a point 2 describes a circle C of centre a, and if /(2) be analytic throughout 

C and its interior except at a number of poles inside C, then the point u=f z) will 

describe a closed curve y in the w-plane. Shew that if to each element of y be attributed 

a mass proportional to the corresponding element of C, the centre of gravity of y is the 

flz) 
point r, where r is the sum of the residues of • — -!- at its poles in the interior of C. 



(Amigues, Noiiv. Ann. de Math. (.3), xii. (1893), pp. 142-148.) 



8. Shew that 



9. Shew that 



/ 



dx 



7r 2a + b) 



-   (. 2 + 62)  x  + a2)2 <ia?h (a + 6)2 " 



/: 



dx 



IT 1.3...(2?i-3) 1 



(  + 6 2)   n h l.2...(H-l) a"" ' 



THE THEORY OF RESIDUES 123 



10. If Fn  z)= n n (1 - 2' P), shew that the series 

m = l ]i=l 

fiz)=- I  "( '' ") 



1=2 (2"??~"-l)%""  

is an analytic function when z is not a root of any of the equations 2" = %" ; and that the 
sum of the residues of f z) contained in the ring-shaped space inchided between two 
circles whose centres are at the origin, one having a small radius and the other having 
a radius between n and n + 1, is equal to the number of prime numbers less than n + l. 
(Laurent, A' oiiv. Aim. de Math. (3), xviii. (1899), pp. 234-241.) 

11. If -4 and B represent on the Argand diagram two given roots (real or imaginary) 
of the equation /(j) = of degree n, with real or imaginary coefficients, shew that there is 
at least one root of the equation/'  z) = within a circle whose centre is the middle point 

of J  and whose radius is lAB cot - . (Grace, Proc. Camb. Phil. Soc. xi.) 

n 



12. Shew that, if 0<i'<l, 



= s —   lim 2 



[Consider / 



g 2v-l)ziri  2 

round a circle of radius   + 5 ; and make /i- -x .] 



s\ nirz z — x 

(Kronecker, Journal fiir Math, cv.) 
1 3. Shew that, if m > 0, then 

* sin" mt 



L 



dt 



1: 



°2 g  '"--'T<"- >- "'2i '"'- >'-- ""'"3V''"" ('- ''>'- + •   • 
Discuss the discontinuity of the integral at m = 0. 

14. If A + B + C+ ...=0 and a, b, c, ... are positive, shew that 

 cosa.r+ficos6.r+... +  cos a; , ., di 7, i-i /. 
dx= -A log a - 5 log b- ,..- A log a:. 

'*' 

(Wolstenholme.) 

/•gX(*+ ) • • 1 

15. By considering I -j r dt taken round a rectangle indented at the origin, shew 

that, if k > 0, 

I lim I -. — -T dt=ni+ lim PI — dt, 

p aa J -P  ' + '  p x J -p t 

and thence deduce, by using the contour of § 6*222 example 2, or its reflexion in the real 
axis (according as a;   or .v < 0), that 



1 /"p px(k + ti) 

Um - , . dt = 2, 1 or 0, 



k + ti 



p- .oo   J -P 

according as x>0, x = or x < 0. * 

[This integral is known as Cauchy's discontinuous factw.'] 
1 6. Shew that, if < a < 2, 6 > 0, /  > 0, then 

= l-,  -l a—br 



.r -i sin ( aTT -607) -5—  -  = i7rr  



124 THE PROCESSES OF ANALYSIS [CHAP. VI 

17. Let  ; > and let 2 e-n-' f = ylr t). 

 l=-oo 

r g-Z TTt 

By considering / -  — -dz round a rectangle whose corners are ± JV+ )±i, where 
N is an integer, and making N-*- qo , shew that 

By expanding these integrands in powers of e~ ' , e-""  respectively and integrating 
term-by-term, deduce from § 6"22 example 3 that 

 TTty J ---= 

Hence, hy putting t = l shew that 

v.(o= -H(i/o. 

(This result is due to Poisson, Journal de VEcole foly technique, xii. (cahier xix), (1823), 
p. 420 ; see also Jacobi, Journal fiir Math, xxxvi. (1848), p. 109 [Oes. Werke, ii. (1882), 
p. 188].) • 

18. Shew that, if i;>0, 



2 e 

J( = - 00 



- ntTTt - 2nnat = t ~   e'"''-  \ \  + ll 2 c - "' W< COS 2n7ra i 



(Poisson, Mem. de I'Acad. des Sci. vi. (1827), p. 592 ; Jacobi, Journal fur Math. ill. 
(1828), pp. 403-404 [Ges. Werke, i. (1881), pp. 264-265] ; and Landsberg, Journal fur 
Math. CXI. (1893), pp. 234-253 ; see also § 21-51.) 


%
% 125
%
\chapter{The Expansion of Functions in Infinite Series}

\Section{7}{1}{A formula due to Darboux*TODO.  Journal de Math. (3), ii. (1876), p. 271.}

Let $f(z)$ be analytic at all points of the straight line joining $a$
to $z$, and let $\phi(t)$ be any polynomial of degree $n$ in $t$.

Then if $0 \leq t \leq 1$, we have by differentiation
\begin{align*}
  \frac{\dd }{\dd t}
  \sum_{m=1}^{n} (-)^{m} (z-a)^{m} \phi^{(n-m)}(t) f^{(m)}(a + t(z-a))
  \\
  =
  -(z-a) \phi^{(n)}(t) f'(a + t(z-a))
  +
  (-)^{n} (z-a)^{n+1} \phi(t) f^{(n+1)}(a + t (z-a)).
\end{align*}

Noting that $\phi^{(n)}(t)$ is constant $= \phi^{(n)}(0)$, and integrating
between the limits $0$ and $1$ of $t$, we get
\begin{align*}
  \phi^{(n)}(0) \thebrace{ f(z) - f(a) }
  \\
  =&
  \sum_{m=1}^{n} (-)^{m-1} (z-a)^{m}
  \thebrace{
    \phi^{(n-m)}(1) f^{(m)}(z) - \phi^{(n-m)}(0) f^{(m)}(a)
  }
  \\
  &
  + (-)^{n} (z-a)^{n+1}
  \int_{0}^{1} \phi(t) f^{(n+1)}(a + t(z-a)) \dmeasure t,
\end{align*}
which is the formula in question.

Taylor's series may be obtained as a special case of this by writing
$\phi(t) = (t-1)^{n}$ and making $n\rightarrow\infty$.

%\begin{smalltext}
Example. By substituting $2n$ for $n$ in the formula of Darboux, and
taking $\phi(t) = t^{n} (t-1)^{n}$, obtain the expansion (supposed convergent)
$$
f(z) - f(a)
=
\sum_{n=1}^{\infty}
\frac{ (-)^{n-1} (z-a)^{n} }{2^{n} n!}
\thebrace{ f^{(n)}(z) + (-)^{n-1} f^{(n)}(a) },
$$
and find the expression for the remainder after $n$ terms in this
series.
%\end{smalltext}
\Section{7}{2}{The \Bernoulli an numbers and the \Bernoulli an polynomials.}

The function $\half z \cot \half z$ is analytic when $\absval{z} <
2\pi$,
and, since it is an even function of $z$, it can be expanded into a
Maclaurin series, thus
$$
\half z \cot \half z
=
1
- B_{1} \frac{z^{2}}{2!}
- B_{2} \frac{z^{4}}{4!}
- B_{3} \frac{z^{6}}{6!}
\cdots
;
$$
then $B_{n}$ is called the $n$th \emph{\Bernoulli an
  number}\footnote{TODO These numbers were introduced by Jakob BernouUi in bis Ars
Conjectandi, p. 97 (published posthumously, 1713).}.
It is found that\footnote{TODO Tables of the first sixty-two \Bernoulli an numbers have been given by
Adams, Brit. A.is. ReiJorts, 1877.}
$$
B_{1} = \frac{1}{6},
\quad
B_{2} = \frac{1}{30},
\quad
B_{3} = \frac{1}{42},
\quad
B_{4} = \frac{1}{30},
\quad
B_{5} = \frac{5}{66},
\quad
\ldots.
$$

%
% 126
%
These numbers can be expressed as definite integrals as follows:

We have, by example TODO:2 (p. TODO:122) of Chapter TODO:VI,
\begin{align*}
  \int_{0}^{\infty}
  \frac{\sin px \dmeasure x}{e^{\pi x} - 1}
  =&
  -\frac{1}{2p} + \frac{i}{2} \cot ip
  \\
  =&
  -\frac{1}{2p}
  +
  \frac{1}{2p} \thebrace{
    1
    + B_{1} \frac{(2p)^{2}}{2!}
    - B_{2} \frac{(2p)^{4}}{4!}
    + \cdots
    }.
\end{align*}

Since
$$
\int_{0}^{\infty}
\frac{x^{n} \sin \theparen{px + \half n \pi}}{e^{\pi x} - 1}
\dmeasure x
$$
converges uniformly (by de la Vall\'ee Poussin's test) near
$p=0$ we
may, by \hardsubsectionref{4}{4}{4} corollary, differentiate both
sides of this equation any number of times and then put $p = 0$; doing
so and writing $2t$ for $x$, we obtain
$$
B_{n}
=
4n
\int_{0}^{\infty}
\frac{t^{2n-1} \dmeasure t}{e^{2\pi t} - 1}.
$$
%\begin{remark}
A proof of this result, depending on contour integration, is given by
Carda, Monatshefte fur Math, v.nd Phys. v. (1894), pp. 321-4.
%\end{remark}

TODO:fixexample
Example. Shew that
$$
B_{n}
=
\frac{2n}{\pi^{2n} (2^{2n}-1)}
\int_{0}^{\infty}
\frac{x^{2n-1} \dmeasure x}{\sinh x}
> 0.
$$

Now consider the function $t \frac{e^{zt}-1}{e^{t}-1}$, which may be
expanded into a
Maclaurin series in powers of $t$ valid when $\absval{t} < 2\pi$.

\emph{The \Bernoulli an polynomial\footnote{TODO The name was given by
    Raabe, Journal filr Math. xlii. (1851), p.348.} of order $n$} is
defined to be the coefficient of $\frac{t^{n}}{n!}$
in this expansion. It is denoted by $\phi_{n}(z)$, so that
$$
t \frac{e^{zt}-1}{e^{t}-1}
=
\sum_{n=1}^{\infty}
\frac{\phi
_{n}(z) t^{n}}{n!}.
$$

This polynomial possesses several important properties. Writing $z+1$
for $z$ in the preceding equation and subtracting, we find that
$$
t e^{zt}
=
\sum_{n=1}^{\infty}
\thebrace{
  \phi_{n}(z+1) - \phi_{n}(z)
}
\frac{t^{n}}{n!}.
$$

On equating coefficients of $t^{n}$ on both
sides of this equation we obtain
$$
n z^{n-1}
=
\phi_{n}(z+1) - \phi_{n}(z),
$$
which is a difference-equation satisfied by the function $\phi_{n}(z)$.

%
% 127
%

An explicit expression for the \Bernoulli an polynomials can be obtained
as follows. We have
$$
e^{zt} - 1
=
zt
+ \frac{z^{2}t^{2}}{2!}
+ \frac{z^{3}t^{3}}{3!}
+ \cdots,
$$
and
$$
\frac{t}{e^{t}-1}
=
\frac{t}{2i} \cot \frac{t}{2i} - \frac{t}{2}
=
1
- \frac{t}{2}
+ \frac{B_{1} t^{2}}{2!}
- \frac{B_{2} t^{4}}{4!}
+ \cdots.
$$

Hence
$$
\sum_{n=1}^{\infty}
\frac{\phi_{n}(z) t^{n}}{n!}
=
\thebrace{
  zt
  + \frac{z^{2} t^{2}}{2!}
  + \frac{z^{3} t^{3}}{3!}
  + \cdots
}
\thebrace{
  1
  - \frac{t}{2}
  + \frac{B_{1} t^{2}}{2!}
  - \frac{B_{2} t^{4}}{4!}
  + \cdots
}.
$$

From this, by equating coefficients of $t^{n}$
(\hardsubsectionref{3}{7}{3}), we have
$$
\phi_{n}(z)
=
z^{n}
- \half n z^{n-1}
+ \binomialcoeff{n}{2} B_{1} z^{n-2}
- \binomialcoeff{n}{4} B_{2} z^{n-4}
+ \binomialcoeff{n}{6} B_{3} z^{n-6}
- \cdots,
$$
the last term being that in $z$ or $z^{2}$ and
$\binomialcoeff{n}{2}, \binomialcoeff{n}{4},\ldots$  being the
binomial coefficients; this is the Maclaurin series for the $n$th
\Bernoulli an polynomial.
%\begin{Remark}
When $z$ is an integer, it may be seen from the difference-equation that
$$
\phi_{n}(z)/n
=
1^{n-1}
+ 2^{n-1}
+ \cdots + (z-1)^{n-1}.
$$

The Maclaurin series for the
expression on the right was given by \Bernoulli.

%\begin{Remark}
\begin{wandwexample}
  Shew that, when $n > 1$,
  $$
  \phi_{n}(z) = (-)^{n} \phi_{n}(1-z).
  $$
\end{wandwexample}
%\end{Remark}
\Subsection{7}{2}{1}{The Euler-Maclaurin expansion.}

In the formula of Darboux \hardsectionref{7}{1}) write $\phi_{n}(t)$
for $\phi(t)$, where $\phi_{n}(t)$ is the $n$th \Bernoulli an polynomial.

Differentiating the equation
$$
\phi_{n}(t+1) - \phi_{n}(t) = n t^{n-1}
$$
$n - k$ times, we have
$$
\phi_{n}^{(n-k)}(t+1) - \phi_{n}^{(n-k)}(t)
=
n (n-t) \cdots k t^{k-1}
$$
Putting $t=0$ in this, we have
$\phi_{n}^{(n-k)}(1) = \phi_{n}^{(n-k)}(0).$

Now, from the Maclaurin
series for $\phi_{n}(z)$, we have if $k > 0$
\begin{align*}
  \phi_{n}^{(n-2k-1)}(0) = 0,
  &
  \quad
  \phi_{n}^{(n-2k)}(0) = \frac{n!}{(2k)!} (-)^{k-1} B_{k},
  \\
  \phi_{n}^{(n-1)}(0) = -\half n!,
  &
  \quad
  \phi_{n}^{(n)}(0) = n!.
\end{align*}

Substituting these values of $\phi_{n}^{(n-k)}(1)$ and
$\phi_{n}^{(n-k)}(0)$ in Darboux's
result, we obtain the Euler-Maclaurin sum
formula\footnote{TODO A history of the formula is given by Barnes, Proc. London Math. Soc.
(2), iii. (1905), p. 253. It was discovered by Euler (1732), but was
not published at the time. Euler communicated it (June 9, 1736) to
Stirling who replied (April 16, 1738) that it included his own theorem
(see \hardsubsectionref{12}{3}{3}) as a particular case, and also that the more general
theorem had been discovered by Maclaurin; and Euler, in a lengthy
reply, waived his claims to priority. The theorem was published by
Euler, Comm. Acad. Imp. Petrop. vi. (1732), [Published 1738], pp.
68-97, and by Maclaurin in 1742, Treatise on Fluxions, p. 672. For
information concerning the correspondence between Euler and Stirling,
we are indebted to Mr C. Tweedie.},

%
% 128
%
\begin{align*}
(z-a) f'(a)
=&
f(z) - f(a)
- \frac{z-a}{2} \thebrace{ f'(z) - f'(a) }
\\
&
+ \sum_{m=1}^{n-1} \frac{ (-)^{m-1} B_{m} (z-a)^{2m}}{(2m)!}
\thebrace{f^{(2m)}(z) - f^{(2m)}(a)}
\\
&
-\frac{(z-a)^{2n+1}}{(2n)!}
\int_{0}^{1} \phi_{2n}(t) f^{(2n+1)}\thebrace{ a + (z-a) t }
\dmeasure t.
\end{align*}

In certain cases the last term tends to zero as
$n \rightarrow \infty$, and we can
thus obtain an infinite series for $f(z) - f(a)$.

If we write $\omega$ for $z - a$ and $F(x)$ for $f'(x)$, the last formula becomes
\begin{align*}
  \int_{a}^{a+\omega} F(x) \dmeasure x
  =
  &
  \half \omega \thebrace{ F(a) + F(a + \omega) }
  \\
  &
  + \sum_{m=1}^{n-1}
  \frac{(-)^{m} B_{m} \omega^{2m}}{(2m)!}
  \thebrace{F^{(2m-1)}(a+\omega) - F^{(2m-1)}(a)}
  \\
  &
  + \frac{\omega^{2n+1}}{(2n)!}
  \int_{0}^{1} \phi_{2n}(t) F^{(2n)}(a + \omega t) \dmeasure t.
\end{align*}

Writing $a + \omega, a + 2\omega, \ldots, a + (r-1) \omega$
for $a$ in this result and adding up, we get
\begin{align*}
\int_{a}^{a + r\omega} F(x) \dmeasure x
=
\omega
 &
\thebrace{
  \half F(a) + F(a+\omega) + F(a+2\omega)
  + \cdots + \half F(a + r\omega)
}
\\
&
+ \sum_{m=1}^{n-1}
\frac{(-)^{m} B_{m} \omega^{2m}}{(2m)!}
\thebrace{
  F^{(2m-1)}(a + r\omega)
  -
  F^{(2m-1)}(a)
}
+ R_{n} ,
\end{align*}
where
$$
R_{n}
=
\frac{\omega^{2n+1}}{(2n)!}
\int_{0}^{1} \phi_{2n}(t)
\thebrace{ \sum_{m=0}^{r-1} F^{(2n)}(a + m\omega + \omega t)}
\dmeasure t.
$$

This last formula is of the utmost importance in connexion with the
numerical evaluation of definite integrals. It is valid if $F(x)$ is
analytic at all points of the straight line joining
$a$ to $a + r \omega$.

%\begin{Remark}
\begin{wandwexample}
  If $f(z)$ be an odd function of $z$, shew that
  $$
  z f'(z)
  =
  f(z)
  +
  \sum_{m=2}^{n}
  (-)^{m}
  \frac{B_{m-1} (2z)^{2m-2}}{(2m-2)!}
  f^{(2m-2)}(z)
  -
  \frac{2^{2n} z^{2n+1}}{(2n)!}
  \int_{0}^{1}
  \phi_{2n}(t)
  f^{(2n+1)}(-z + 2zt)
  \dmeasure t.
  $$
\end{wandwexample}
\begin{wandwexample}
  Shew, by integrating by parts, that the remainder after $n$
  terms of the expansion of $\half z \cot \half z$ may be written in the form
  $$
  \frac{ (-)^{n+1} z^{2n+1} }{ (2n)! \sin z }
  \int_{0}^{1} \phi_{2n}(t) \cos (zt) \dmeasure t.
  $$
\addexamplecitation{Math. Trip. 1904.}
\end{wandwexample}
%\end{Remark}
\Section{7}{3}{\Burmann's theorem*TODO * Memoires de VInstitut, ii. (1799), p. 13. See also Dixon, Proc.
London Math. Soc. xxxiv. (1902), pp. 151-153..}

We shall next consider several theorems which have for their object
\emph{the expansion of one function in powers of another function.}

%
% 129
%

Let $\phi(z)$ be a function of $z$ which is analytic in a closed
region $S$ of which $a$ is an interior point; and let
$$
\phi(a) = b.
$$

Suppose also that $\phi'(a) \neq 0$. Then Taylor's theorem furnishes the
expansion
$$
\phi(z) - b
=
\phi'(a) (z-a)
+ \half \phi''(a) (z-a)^{2}
+ \cdots,
$$
and if it is legitimate to revert this series we obtain
$$
z - a
=
\frac{1}{\phi'(a)}
\thebrace{ \phi(z) - b }
-
\half \frac{\phi''(a)}{\thebrace{\phi'(a)}^{3}}
\thebrace{ \phi(z) - b }^{2}
+ \cdots,
$$
which expresses $z$ as an analytic function of the variable
$\thebrace{ \phi(z) - b }$,
for sufficiently small values of $\absval{z-a}$. If then $f(z)$ be
analytic near $z = a$, it follows that $f(z)$ is an analytic function
of $\thebrace{ \phi(z) - b }$
when $\absval{z - a}$ is sufficiently small, and so there will be an
expansion of the form
$$
f(z)
=
f(a)
+ a_{1} \thebrace{ \phi(z) - b }
+ \frac{a_{2}}{2!} \thebrace{ \phi(z) - b }^{2}
+ \frac{a_{3}}{3!} \thebrace{ \phi(z) - b }^{3}
+ \cdots.
$$

The actual coefficients in the expansion are given by the following
theorem, which is generally known as \emph{\Burmann's theorem}.

\emph{Let $\psi(z)$ be a function of $z$ defined by the equation
$$
\psi(z) = \frac{z-a}{ \phi(z) - b };
$$
then an analytic function $f(z)$ can, in a certain domain of values of
$z$, be expanded in the form
$$
f(z)
=
f(a)
+
\sum_{m=1}^{n-1}
\frac{ \thebrace{\phi(z)-b}^{m} }{m!}
\frac{\dd^{m-1}}{\dd a^{m-1}}
\thebracket{
  f'(a) \thebrace{\psi(a)}^{m}
}
+
R_{n},
$$
where
$$
R_{n}
=
\frac{1}{2 \pi i}
\int_{a}^{z}
\int_{\gamma}
\thebracket{
  \frac{\phi(z) - b}{\phi(t) - b}
}^{n-1}
\frac{ f'(t) \phi'(z) }{\phi(t) - \phi(z)}
\dmeasure t \dmeasure z,
$$
and $\gamma$ is a contour in the $t$-plane, enclosing the points $a$
and $z$ and such that, if $\zeta$ be any point inside it, the equation
$\phi(t) = \phi(\zeta)$ has no roots on or inside the contour
except\footnote{It is assumed that such a contour can be chosen if
  $\absval{z - a}$ be sufficiently small;
  see\hardsubsectionref{7}{3}{1}.} a simple root $t=\zeta$.
}

To prove this, we have
\begin{align*}
  f(z) - f(a)
  = &
  \int_{a}^{z} f'(\zeta) \dmeasure \zeta
  = \frac{1}{2 \pi i} \int_{a}^{z} \int_{\gamma} TODO
\end{align*}

%
% 130
%

But, by\hardsectionref{4}{3},
\begin{align*}
  &
  \frac{1}{1 \pi i}
  \int_{a}^{z} \int_{\gamma}
  \thebracket{ \frac{\phi(\zeta) - b}{\phi(t) - b} }^{m}
  \frac{f'(t)\phi'(\zeta)}{\phi(t) - b}
  \dmeasure t \dmeasure \zeta
  \\
  &=
  \frac{ [\phi(z) - b]^{m+1} }{2 \pi i (m+1)}
  \int_{\gamma} \frac{f'(t)}{[\phi(t) - b]^{m+1}} \dmeasure t
  \\
  &=
  \frac{ [\phi(z) - b]^{m+1} }{2 \pi i (m+1)}
  \int_{\gamma}
  \frac{ f'(t) \thebrace{\psi(t)}^{m+1} }{ (t-a)^{m+1} }
  \dmeasure t
  \\
  &=
  \frac{ [\phi(z) - b ]^{m+1} }{ (m+1)! }
  \frac{ \dd^{m} }{ \dd a^{m} }
  \thebracket{ f'(a) \thebrace{\psi(a)}^{m+1} }.
\end{align*}
Therefore, writing
$m - 1$ for $m$,
\begin{align*}
  f(z) = f(a) +
  &
  \sum_{m=1}^{n-1}
  \frac{ [\phi(z) - b]^{m} }{m!}
  \frac{\dd^{m-1}}{\dd a^{m-1}}
  [ f'(a) \thebrace{ \psi(a) }^{m} ]
  \\
  &
 + \frac{1}{2 \pi i}
  \int_{a}^{z} \int_{\gamma}
  \thebracket{
    \frac{\phi(\zeta) - b}{\phi(t) - b}
  }^{n-1}
  \frac{f'(t) \phi'(\zeta)}{\phi(t) - \phi(\zeta)}
  \dmeasure t \dmeasure \zeta.
\end{align*}

If the last integral tends to zero as $n \rightarrow \infty$, we may write the
right-hand side of this equation as an infinite series.
\begin{wandwexample}
Prove that
$$
z
=
a
+ \sum_{n=1}^{\infty}
\frac{(-)^{n-1} C_{n} (z-a)^{n} e^{n (z^{2} - a^{2})}}{n!},
$$
where
$$
C_{n}
=
(2na)^{n-1}
- \frac{n(n-1)(n-2)}{1!} (2na)^{n-3}
+ \frac{n^{2}(n-1)(n-2)(n-3)(n-4)}{2!} (2na)^{n-5}
- \cdots.
$$
To obtain this expansion, write
$$
f(z) = z,
\quad
\phi(z) - b = (z-a) e^{z^{2} - a^{2}},
\quad
\psi(z) = e^{a^{2} - z^{2}}
$$
in the above expression
of \Burmann's theorem; we thus have
$$
z
=
a
+ \sum_{n=1}^{\infty}
\frac{1}{n!}
(z-a)^{n}
e^{n(z^{2} - a^{2})}
\thebrace{
  \frac{\dd^{n-1}}{\dd z^{n-1}}
  e^{n(a^{2} - z^{2})}
}_{z=a}.
$$

But, putting $z = a + t$,
\begin{align*}
  \thebrace{
    \frac{\dd^{n-1}}{\dd z^{n-1}}
    e^{n(a^{2} - z^{2})}
  }_{z=a}
  =&
  \thebrace{
    \frac{\dd^{n-1}}{\dd t^{n-1}}
    e^{-n(2at + t^{2})}
  }_{t=0}
  \\
  =&
  TOOD
\end{align*}
The highest value of $r$ which gives a term in the summation is
$r = n-1$.
Arranging therefore the summation in descending indices $r$,
beginning with $r = n-1$, we have
\begin{align*}
  \thebrace{
    \frac{\dd^{n-1}}{\dd z^{n-1}}
    e^{n(a^{2} - z^{2})}
  }_{z=a}
  =&
  TODO
  \\
  =& (-)^{n-1} C_{n},
\end{align*}
which gives the required result.
\end{wandwexample}
\begin{wandwexample}
  Obtain the expansion
  $$
  z^{2}
  =
  \sin^{2} z
  + \frac{2}{3} \half \sin^{4} z
  + \frac{2 \cdot 4}{3 \cdot 5} \frac{1}{3} \sin^{6} z
  + \cdots.
  $$
\end{wandwexample}
%
% 131
%
\begin{wandwexample}
Let a line $p$ be drawn through the origin in the $z$-plane,
perpendicular to the line which joins the origin to any point $a$. If
$z$ be any point on the $z$-plane which is on the same side of the line
$p$ as the point $a$ is, shew that
$$
\log z
=
\log a
+ 2 \sum_{m=1}^{\infty}
\frac{1}{2m+1} \theparen{\frac{z-a}{z+a}}^{2m+1}.
$$
\end{wandwexample}
\Subsection{7}{3}{1}{\Teixeira's extended form of \Burmann's theorem.}

In the last section we have not investigated closely the conditions of
convergence of \Burmann's series, for the reason that a much more
general form of the theorem will next be stated; this generalisation
bears the same relation to the theorem just given that Laurent's
theorem bears to Taylor's theorem: viz., in the last paragraph we
were concerned only with the expansion of a function in \emph{positive}
powers of another function, whereas we shall now discuss the expansion
of a function in \emph{positive and negative} powers of the second
function.

The general statement of the theorem is due to
\Teixeira\footnote{TODO Journal f\"ur Math, cxxii. (1900), pp.
  97-123.}, whose exposition we shall follow in this section.

Suppose (i) that $f(z)$ is a function of $z$ analytic in a ring-shaped
region $A$, bounded by an outer curve $C$ and an inner curve $c$;
(ii) that $\theta(z)$ is a function analytic on and inside $C$,
and has only one zero a
within this contour, the zero being a simple one;
(iii) that $x$ is a given point within $A$;
(iv) that for all points $z$ of $C$ we have
$$
\absval{\theta(x)} < \absval{\theta(z)},
$$
and for all points $z$ of $c$ we have
$$
\absval{\theta(x)} > \absval{\theta(z)}.
$$

The equation
$$
\theta(z) - \theta(x) = 0
$$
has, in this case, a single root $z = x$ in the interior of $C$, as is
seen from the equation\footnote{The expansion is justified
  by\hardsectionref{4}{7}, since
  $\sum_{n=1}^{\infty} \thebrace{\theta(x)/\theta(z)}^{n}$
  converges uniformly when $z$
  is on $C$.}
\begin{align*}
  \frac{1}{2 \pi i}
  \int_{C} \frac{\theta'(z)}{\theta(z) - \theta(x)} \dmeasure z
  =&
  \frac{1}{2 \pi i}
  \thebracket{
    \int_{C} \frac{\theta'(z)}{\theta(z)} \dmeasure z
    +
    \theta(x)
    \int_{C} \frac{\theta'(z)}{ \thebrace{\theta(z)}^{2}}
    \dmeasure z
    + \cdots
  }
  \\
  =&
  \frac{1}{2 \pi i}
  \int_{C} \frac{\theta'(z)}{\theta(z)} \dmeasure z
\end{align*}
of which the left-hand and right-hand members represent respectively
the number of roots of the equation considered
(\hardsubsectionref{6}{3}{1}) and the
number of the roots of the equation $\theta(z) = 0$ contained within
$C$.

Cauchy's theorem therefore gives
$$
f(x)
=
\frac{1}{2 \pi i}
\thebracket{
  \int_{C} \frac{f(z) \theta'(z)}{\theta(z) - \theta(x)} \dmeasure z
  -
  \int_{c} \frac{f(z) \theta'(z)}{\theta(z) - \theta(x)} \dmeasure z
}.
$$

%
% 132
%
The integrals in this formula can be expanded, as in Laurent's
theorem, in powers of $\theta(x)$, by the formulae
\begin{align*}
  TODO
\end{align*}

We thus have the formula
$$
TODO
$$
where
$$
TODO
$$

Integrating by parts, we get, if $n \neq 0$,
$$
TODO
$$

This gives a development of $f(x)$ in positive and negative powers of
$\theta(x)$, valid for all points $x$; within the ring-shaped space $A$.

If the zeros and poles of $f(z)$ and $\theta(z)$ inside $C$ are known,
$A_{n}$ and $B_{n}$ can be evaluated by\hardsubsectionref{5}{2}{2} or
by \hardsectionref{6}{1}.

\begin{wandwexample}
  Shew that, if $\absval{x} < 1$, then
  $$
  x
  =
  \half \theparen{ \frac{2x}{1+x^{2}} }
  +
  \frac{1}{2 \cdot 4} \theparen{ \frac{2x}{1+x^{2}} }^{3}
  +
  \frac{1\cdot 3}{2 \cdot 4 \cdot 6} \theparen{ \frac{2x}{1+x^{2}} }^{5}
  +
  \cdots.
  $$

  Shew that, when $\absval{x} > 1$, the second member represents $x^{-1}$.
\end{wandwexample}
\begin{wandwexample}
If $S^{(m)}_{2n}$ denote the sum of all combinations of the numbers
$$
2^{2}, 4^{2}, 6^{2}, \ldots, (2n-2)^{2},
$$
taken $m$ at a time, shew that
$$
\frac{1}{z}
=
\frac{1}{\sin z}
+
\sum_{n=0}^{\infty}
\frac{(-)^{n+1}}{ (2n+2)! }
\theparen{
  \frac{1}{2n+3}
  - \frac{S^{(1)}_{2(n+1)}}{2n+1}
  + \cdots
  + \frac{ (-)^{n} S^{(n)}_{2(n+1)}}{3}
}
(\sin z)^{2n+1}
$$
the expansion being valid for all values of $z$ represented by points
within the oval whose equation is $\absval{\sin z} = 1$ and which contains the
point $z = 0$. \addexamplecitation{\Teixeira.}
\end{wandwexample}

\Subsection{7}{3}{2}{Lagrange's theorem.}

Suppose now that the function $f(z)$ of \hardsubsectionref{7}{3}{1}
is analytic at all points in the interior of $C$, and let
$\theta(x) = (x - a) \theta_{1}(x)$. Then $\theta_{1}(x)$ is
analytic and not zero on or inside $C$ and the contour $c$ can be
dispensed with; therefore the formulae which give $A_{n}$ and
$B_{n}$ now become, by\hardsubsectionref{5}{2}{2} and \hardsectionref{6}{1},
\begin{align*} % TODO: multiline?
  A_{n}
  =&
  \frac{1}{2\pi in} \!
  \int_{C} \frac{f'(z)}{(z-a)^{n} \theta_{1}^{n}(z)} \dmeasure z
  = \frac{1}{n!} \frac{\dd^{n-1}}{\dd a^{n-1}}
  \thebrace{
    \frac{f'(a)}{\theta_{1}^{n}(a)
    }
  }
  \quad (n \geq 1),
  \\
  A_{0}
  =&
  \frac{1}{2\pi i} \!
  \int_{C} \frac{f(z) \theta'(z)}{\theta_{1}(z)}
  \frac{\dmeasure z}{z-a}
  =
  f(a),
  \\
  B_{n}
  =&
  0.
\end{align*}

%
% 133
%

The theorem of the last section accordingly takes the following form,
if we write $\theta_{1}(z) = 1 / \phi(z)$:

\emph{Let $f(z)$ and $\phi(z)$ be functions of $z$ analytic on and inside a
contour $C$ surrounding a point $a$, and let $t$ be such that the inequality
$$
\absval{t \phi(z) } < \absval{z - a}
$$
is satisfied at all points $z$ on the perimeter of $C$;
then the equation
$$
\zeta = a + t \phi(\zeta),
$$
regarded as an equation in $\zeta$, has one root in the interior
of $C$; and further any function of $\zeta$ analytic on and inside
$C$ can be expanded as a power series in $t$ by the formula
$$
f(\zeta)
=
f(a)
+
\sum_{n=1}^{\infty}
\frac{t^{n}}{n!}
\frac{\dd^{n-1}}{\dd a^{n-1}}
\thebracket{
  f'(a) \phi^{n}(a)
}.
$$
}
This result was published by Lagrange\footnote{TODO Mem. de VAcad. de Berlin, xxiv.; Oeuvres, iii. p. 25.} in 1770.
\begin{wandwexample}
Within the contour surrounding $a$ defined by the inequality
$\absval{z (z - a)} > \absval{a}$, where
$\absval{a} < \half \absval{ a }$, %TODO: verify
the equation
$$
z - a - \frac{a}{z} = 0
$$
has one root $\zeta$, the expansion of which is given by Lagrange's theorem
in the form
$$
\zeta
=
a
+
\sum_{n=1}^{\infty}
\frac{(-)^{n-1} (2n-2)!}{n! (n-1)! a^{2n-1}} a^{n}
%TODO: verify
$$

Now, from the elementary theory of quadratic equations, we know that
the equation
$$
z - a - \frac{a}{z} = 0
%TODO: verify
$$
has two roots, namely $TODO$ and $TODO$; and our
expansion
\emph{represents the former\footnote{The latter is outside the given
    contour.} of these only}---an example of the need for
care in the discussion of these series.
\end{wandwexample}
\begin{wandwexample}
  If $y$ be that one of the roots of the equation
  $$
  TODO
  $$
  which tends to $1$ when $z \rightarrow 0$, shew that
  $$
  TODO
  $$
  so long as $\absval{z} < \frac{1}{4}$.
\end{wandwexample}
\begin{wandwexample}
If $x$ be that one of the roots of the equation
$$
x = 1 + y x^{a}
$$
which tends to $1$ when $y \rightarrow 0$, shew that
$$
TODO
$$
the expansion being valid so long as
$$
\absval{y}
<
\absval{
  (a-1)^{a-1} a^{-a}
}.
$$
\addexamplecitation{McClintock.}
\end{wandwexample}

%
% 134
%

\Section{7}{4}{The expansion of a class of functions in rational fractions*'.}
Consider a function $f(z)$, whose only singularities in the finite
part of the plane are simple poles $a_{1},a_{2},a_{3},\ldots$, where
$\absval{a_{1}} \leq \absval{a_{2}} \leq \absval{a_{3}} \leq \cdots$;
let $b_{1},b_{2},b_{3},\ldots$, be the residues at these
poles, and let it be possible to choose a sequence of circles $C_{m}$ (the
radius of $C_{m}$ being $R_{m}$) vvith centre at $O$, not passing through any
poles, such that $\absval{f(z)}$ is bounded on $C_{m}$. (The function
$\cosec z$ may
be cited as an example of the class of functions considered, and we
take $R_{m} = (m + \half)\pi$.) Suppose further that
$R_{m} \rightarrow \infty$ as $m \rightarrow \infty$ and that
the upper bound\footnote{Which is a function of $m$.}
of $\absval{f(z)}$ on $C_{m}$ is itself bounded
as\footnote{Of course $R_{m}$ need not (and frequently must not) tend to infinity
  continuously; e.g. in the example taken
  $R_{m} = (m+\half)z$, where $m$ assumes only integer values.}
$m\rightarrow\infty$; so
that, for all points on the circle $C_{m}$, $\absval{f(z)} < M$, where $M$ is
independent of $m$.

Then, if $x$ be not a pole of $f(z)$, since the only poles of the
integrand are the poles of $f(z)$ and the point $z = x$, we have, by\hardsectionref{6}{1},
$$
\frac{1}{2 \pi i} \int_{C_{m}} \frac{f(z)}{z-x} \dmeasure z
=
f(x) + \sum_{r} \frac{b_{r}}{a_{r}-x}.
$$
where the summation extends over all poles in the interior of C, .

But
\begin{align*}
  \frac{1}{2 \pi i}
  \int_{C_{m}} \frac{f(z)}{z-x} \dmeasure z
  =&
  \frac{1}{2 \pi i}
  \int_{C_{m}} \frac{f(z)}{z} \dmeasure z
  +
  \frac{x}{2 \pi i}
  \int_{C_{m}} \frac{f(x)}{z(z-x)} \dmeasure z
  \\
  =&
  f(0) + \sum_{r} \frac{b_{r}}{a_{r}}
  +
  \frac{x}{2 \pi i}
  \int_{C_{m}} \frac{f(z)}{z(z-x)} \dmeasure x
\end{align*}
if we suppose the function $f(z)$ to be analytic at the origin.

Now as $m \rightarrow \infty$,
$\int_{C_{m}} \frac{f(z)}{z(z-x)} \dmeasure z$ is
$\bigo(R_{m}^{-1})$, and so tends to zero as
$m$ tends to infinity.

Therefore, making $m \rightarrow \infty$, we have
$$
0
=
f(x) - f(0)
+
\sum_{n=1}^{\infty}
b_{n} \theparen{
  \frac{1}{a_{n}-x} - \frac{1}{a_{n}}
}
-
\lim_{m\rightarrow\infty}
\frac{x}{2 \pi i}
\int_{C_{m}} \frac{f(z)}{z(z-x)} \dmeasure x,
$$
\ie
$$
f(x) = f(0)
+
\sum_{n=1}^{\infty}
b_{n}
\thebrace{
  \frac{1}{x-a_{n}} + \frac{1}{a_{n}}
  %TODO:consistent with above?
}
$$
which is an expansion of $f(x)$ in rational fractions of $x$; and the
summation extends over \emph{all} the poles of $f(x)$.

%\begin{smalltext}
If $\absval{a_{n}} < \absval{a_{n+1}}$ this series converges uniformly
throughout the region given by $\absval{x} < a$, where $a$ is any constant
(except near the points $a_{n}$).
For if $R_{m}$ be the radius of the circle which encloses the points
$\absval{a_{1}}, \ldots, \absval{a_{n}}$,
the modulus of the remainder of the terms of the series after the first $n$ is
$$
\absval{ \frac{x}{2 \pi i}
  \int_{C_{m}} \frac{f(z)}{z(z-x)} \dmeasure z
}
< \frac{Ma}{R_{m}-a},
$$
by\hardsubsectionref{4}{6}{2}; and, given
$\eps$, we can choose $n$ \emph{independent} of $x$
such that $Ma/(R_{m}-a) < \eps$.

* Mittag-Leffler, Acta Soc. Scient. Fennicae, xi. (1880), pp. 273-293.
See also Acta Math. iv. (1884), pp. 1-79.

%
% 135
%

The convergence is obviously still uniform even if
$\absval{a_{n}} \leq \absval{a_{n+1}}$
provided the terms of the series are grouped so as to combine the
terms corresponding to poles of equal moduli.

If, instead of the condition $\absval{f(z)} < M$, we have the
condition $\absval{ z^{-p} f(z) } < M$,
where $M$ is independent of $m$ when $z$ is on $C_{m}$, and $p$ is
a positive integer, then we should have to
expand $\int_{C} \frac{f(z)}{z-x} \dmeasure z$ by writing
$$
\frac{1}{z-x}
=
\frac{1}{z}
+ \frac{x}{z^{2}}
+ \cdots
+ \frac{x^{p+1}}{z^{p+1}(z-x)},
$$
and should obtain a similar but somewhat more complicated expansion.
\begin{wandwexample}
Prove that
$$
\cosec z
=
\frac{1}{z}
+
\sum (-)^{n}
\theparen{\frac{1}{z-n\pi} + \frac{1}{n\pi}}
$$
the summation extending to all positive and negative values of $n$.

To obtain this result, let $\cosec z - \frac{1}{z} = f(z)$.
The singularities of this function are at the
points $z=n\pi$, where $n$ is any positive or negative integer.

The residue of $f(z)$ at the singularity $n\pi$ is therefore
$(-)^{n}$, and the reader will easily see that $\absval{f(z)}$ is
bounded on the circle $\absval{z} = (n + \half) \pi$ as
$n \rightarrow \infty$.

Applying now the general theorem
$$
f(z)
=
f(0)
+
\sum c_{n} \thebracket{ \frac{1}{z-a_{n}} + \frac{1}{a_{n}}  },
$$
where $c_{n}$ is the residue at the singularity $a_{n}$, we have
$$
f(z)
=
f(0)
+
\sum (-)^{n} \thebrace{ \frac{1}{z-n\pi} + \frac{1}{n\pi}  }.
$$

But
$$
f(0)
=
\lim_{z \rightarrow 0} \frac{z - \sin z}{z \sin z} = 0.
$$

Therefore
$$
\cosec z
=
\frac{1}{z}
+
\sum (-)^{n} \thebracket{ \frac{1}{z-n\pi} + \frac{1}{n\pi}  },
$$
which is the required result.
\end{wandwexample}
\begin{wandwexample}
If $0<a<1$, shew that
$$
\frac{e^{az}}{e^{z}-1}
=
\frac{1}{z}
+
\sum_{n=1}
\frac{2z \cos 2na\pi - 4n\pi \sin 2na\pi}{z^{2}+4n^{2}\pi^{2}}.
$$
\end{wandwexample}
\begin{wandwexample}
Prove that
$$
\frac{1}{2 \pi x^{2} (\cosh x - \cos x)}
=
\frac{1}{2 \pi x^{4}}
-
\frac{1}{e^{\pi}-e^{-\pi}}
\frac{1}{\pi^{4} + \frac{1}{4} x^{4}}
+
\frac{2}{e^{2 \pi}-e^{-2 \pi}}
\frac{1}{(2 \pi)^{4} + \frac{1}{4} x^{4}}
-
\frac{3}{e^{3 \pi}-e^{-3 \pi}}
\frac{1}{(3 \pi)^{4} + \frac{1}{4} x^{4}}
+
\cdots.
$$

The general term of the series on the right is
$$
\frac{(-)^{r} r}{(e^{r \pi}-e^{-r \pi})
  \thebrace{(r \pi)^{4} + \frac{1}{4} x^{4}}}
$$
which is the residue at each of the four singularities
$r, -r, ri, -ri$ of the function
$$
\frac{\pi z}{(\pi^{4}z^{4} + \frac{1}{4} x^{4}) (e^{\pi z} - e^{-\pi
    z}) \sin \pi z}.
$$

%
% 136
%

The singularities of this latter function which are not of the type
$r, -r, ri, -ri$ are at the five points
$$
0,
\frac{(\pm 1 \pm i) x}{2 \pi}.
$$
At $z=0$ the residue is
$$
\frac{2}{\pi x^{4}};
$$
at each of the four points
$\frac{(\pm 1 \pm i) x}{2 \pi}$, the residue is
$$
\thebrace{
  2 \pi x^{2} (\cos x - \cosh x)
}^{-1}
$$

Therefore
\begin{align*}
  4
  \sum_{r=1}^{\infty}
  \frac{(-)^{r} r}{e^{r\pi} - e^{-r\pi}}
  \frac{1}{(r\pi)^{4} + \frac{1}{4} x^{4}}
  +
  \frac{2}{\pi x^{4}}
  -&
  \frac{2}{ \pi x^{2} (\cosh x - \cos x)}
  \\
  &
  =
  \frac{1}{2 \pi i}
  \lim_{n \rightarrow \infty}
  \int_{C}
  \frac{\pi z}{
    (\pi^{4}z^{4} + \frac{1}{x^{4}})
    (e^{\pi z}-e^{-\pi z})
    \sin \pi z
  }
  \dmeasure z,
\end{align*}
where $C$ is the circle whose radius is $n + \half$, ($n$ an integer),
and whose centre is the origin. But, at points on $C$, this integrand is
$\bigo( \absval{z}^{-3} )$; the limit of the integral round $C$ is
therefore zero.

From the last equation the required result is now obvious.
\end{wandwexample}
\begin{wandwexample}
  Prove that
  $$
  \sec x
  =
  4 \pi
  \theparen{
    \frac{1}{\pi^{2} - 4 x^{2}}
    -
    \frac{3}{9 \pi^{2} - 4 x^{2}}
    +
    \frac{5}{25 \pi^{2} - 4 x^{2}}
    -
    \cdots
  }.
  $$
\end{wandwexample}
\begin{wandwexample}
  Prove that
  $$
  \cosech x
  =
  \frac{1}{x}
  -
  2x
  \theparen{
    \frac{1}{\pi^{2} + x^{2}}
    -
    \frac{1}{4\pi^{2} + x^{2}}
    +
    \frac{1}{9 \pi^{2} + x^{2}}
    -
    \cdots
  }.
  $$
\end{wandwexample}
\begin{wandwexample}
  Prove that
  $$
  \sec x
  =
  4 \pi
  \theparen{
    \frac{1}{\pi^{2} + 4 x^{2}}
    -
    \frac{3}{9\pi^{2} + 4 x^{2}}
    +
    \frac{5}{25 \pi^{2} + 4 x^{2}}
    -
    \cdots
  }.
  $$
\end{wandwexample}
\begin{wandwexample}
  Prove that
  $$
  \coth x
  =
  \frac{1}{x}
  +
  2x
  \theparen{
    \frac{1}{\pi^{2} + x^{2}}
    +
    \frac{1}{4 \pi^{2} + x^{2}}
    +
    \frac{1}{9 \pi^{2} + x^{2}}
    +
    \cdots
  }
  $$
\end{wandwexample}
\begin{wandwexample}
  Prove that
  $$
  \sum_{m=-\infty}^{\infty}
  \sum_{n=-\infty}^{\infty}
  \frac{1}{ (m^{2}+a^{2}) (n^{2}+b^{2}) }
  =
  \frac{\pi^{2}}{ab} \coth \pi a \coth \pi b.
  $$
\addexamplecitation{Math. Trip. 1899.}
\end{wandwexample}
\Section{7}{5}{The expansion of a class of functions as
infinite products.}

The theorem of the last article can be applied to the expansion of a
certain class of functions as infinite products.

For let $f(z)$ be a function which has simple zeros at the
points\footnote{These being the only zeros of $f(z)$; and $a \neq 0$.}
$a_{1}, a_{2}, a_{3}, \ldots$, where
$\lim_{n \rightarrow \infty} \absval{a_{n}}$ is infinite; and let
$f(z)$ be analytic for all values of $z$.

Then $f'(z)$ is analytic for all values of $z$
\hardsubsectionref{5}{2}{2}), and so $\frac{f'(z)}{f(z)}$ can have
singularities only at the points $a_{1}, a_{2}, a_{3}, \ldots$.

Consequently, by Taylor's theorem,
$$
f(z)
=
(z-a_{r}) f'(a_{r})
+
\frac{ (z-a_{r})^{2} }{2} f''(a_{r})
+
\cdots
$$
and
$$
f'(z)
=
f'(a_{r})
+
(z-a_{r}) f''(a_{r})
+
\cdots.
$$

/ /
%
% 137
%
It follows immediately that at each of the points $a_{r}$, the
function
$\frac{f'(z)}{f(z)}$ has a simple pole, with residue $+1$.

If then we can find a sequence of circles $C_{m}$ of the nature described
in \hardsectionref{7}{4}, such that $\frac{f'(z)}{f(z)}$ is bounded on
$C_{m}$ as $m \rightarrow \infty$, it follows, from the
expansion given in \hardsectionref{7}{4}, that
$$
\frac{f'(z)}{f(z)}
=
\frac{f'(0)}{f(0)}
+
\sum_{n=1}^{\infty}
\thebrace{
  \frac{1}{z-a_{n}}
  -
  \frac{1}{a_{n}}
}.
$$

Since this series converges uniformly when the terms are suitably
grouped (\hardsectionref{7}{4}), we may integrate term-by-term
(\hardsectionref{4}{7}). Doing so, and taking the exponential of each
side, we get
$$
f(z)
=
c
e^{ z \frac{f'(0)}{f(0)} }
\prod_{n=1}^{\infty}
\thebrace{
  \theparen{ 1 - \frac{z}{a_{n}} }
  e^{ \frac{z}{a_{n}} }
},
$$
where $c$ is independent of $z$.

Putting $z = 0$, we see that $f(0) = c$, and thus the general result
becomes
$$
f(z)
=
f(0)
e^{ z \frac{f'(0)}{f(0)} }
\prod_{n=1}^{\infty}
\thebrace{
  \theparen{ 1 - \frac{z}{a_{n}} }
  e^{ \frac{z}{a_{n}} }
}.
$$

This furnishes the expansion, in the form of an infinite product, of
any function $f(z)$ which fulfils the conditions stated.
\begin{wandwexample}
  Consider the function
  $f(z) = \frac{\sin z}{z}$, which has simple zeros at
  the points $r \pi$, where $r$ is any positive or negative integer.

  In this case we have
  $$
  f(0) = 1,
  \quad
  f'(0) = 0,
  $$
  and so the theorem gives immediately
  $$
  \frac{\sin z}{z}
  =
  \prod_{n=1}^{\infty}
  \thebrace{
    \theparen{1 - \frac{z}{n \pi}}
    e^{ \frac{z}{n \pi} }
  }
  \thebrace{
    \theparen{1 + \frac{z}{n \pi}}
    e^{ -\frac{z}{n \pi} }
  };
  $$
  for it is easily seen that the condition concerning the behaviour of
  $\frac{f'(z)}{f(z)}$ as $\absval{z} \rightarrow \infty$ is fulfilled.
\end{wandwexample}
\begin{wandwexample}
  Prove that
  \begin{align*}
  \thebrace{
    1 + \theparen{ \frac{k}{x}  }^{2}
  }
  \thebrace{
    1 + \theparen{ \frac{k}{2\pi - x}  }^{2}
  }
  &
  \thebrace{
    1 + \theparen{ \frac{k}{2\pi + x}  }^{2}
  }
  \thebrace{
    1 + \theparen{ \frac{k}{4\pi - x}  }^{2}
  }
  \thebrace{
    1 + \theparen{ \frac{k}{4\pi + x}  }^{2}
  }
  \\
  =&
  \frac{\cosh x - \cos x}{1-\cos x}.
  \end{align*}
\addexamplecitation{Trinity, 1899.}
\end{wandwexample}
\Section{7}{6}{The factor theorem of Weierstrass*.}

The theorem of\hardsectionref{7}{5} is very similar to a more general theorem in
which the character of the function $f(z)$, as
$\absval{z} \rightarrow \infty$, is not so
narrowly restricted.

* Berliner Alh. (1876), pp. 11-60; Math. Werke, 11. (1895), pp.
77-124.
%
% 138
%
Let $f(z)$ be a function of $z$ with no essential singularities (except at
`the point infinity'); and let the zeros and poles of $f(z)$ be at
$a_{1}, a_{2}, a_{3}, \ldots$, where
$0 < \absval{a_{1}} \leq \absval{a_{2}} \leq \absval{a_{3}} \ldots$.
Let the zero\footnote{We here regard a pole as being a zero of
  negative order.} at $a_{n}$ be of (integer) order $m_{n}$.

If the number of zeros and poles is unlimited, it is necessary that
$\absval{a_{n}} \rightarrow \infty$, as $n \rightarrow \infty$;
for, if not, the points $a_{n}$ would have a limit
point\footnote{From the two-dimensional analogue
  of\hardsubsectionref{2}{2}{1}.},
which would be an essential singularity of $f(z)$.

We proceed to shew first of all that it is possible to find
polynomials $g(z)$ such that
$$
\prod_{n=1}^{\infty}
\thebracket{
  \thebrace{
    \theparen{
      1
      -
      \frac{z}{a_{n}}
    }
    e^{a_{n}(z)}
  }^{m_{n}}
}
$$
converges for all\footnote{Provided that z is not at one of the points
  for which m is negative.} finite values of $z$.

Let $K$ be any constant, and let $\absval{z} < K$; then, since
$\absval{a_{n}} \rightarrow \infty$, we can
find $N$ such that, when $n > N$, $\absval{a_{n}} > 2K$.

The first $N$ factors of the product do not affect its
convergence\footnote{Provided that z is not at one of the points
  for which m is negative.} % TODO:duplicated-footnote
consider any value of $n$ greater than $N$, and let
$$
g_{n}(z)
=
\frac{z}{a_{n}}
+
\half \theparen{ \frac{z}{a_{n}} }^{2}
+
\cdots
+
\frac{1}{k_{n}-1}
\theparen{ \frac{z}{a_{n}} }^{k_{n} - 1}.
$$
Then
\begin{align*}
  \absval{
    -
    \sum_{m=1}^{\infty}
    \frac{1}{m}
    \theparen{
      \frac{z}{a_{n}}
    }^{m}
    +
    g_{n}
  }
  =&
  \absval{
    \sum_{m=k_{n}}^{\infty}
    \frac{1}{m}
    \theparen{
      \frac{z}{a_{n}}
    }^{m}
  }
  \\
  <&
  \absval{ \frac{z}{a_{n}} }^{k_{n}}
  \sum_{m=0}^{\infty}
  \absval{ \frac{z}{a_{n}} }^{m}
  \\
  <&
  2
  \absval{ (K a_{n}^{-1})^{k_{n}}  },
\end{align*}
since
$ \absval{ z_{n} a_{n}^{-1} } < \half$.

Hence
$$
\thebrace{
  \theparen{
    1
    -
    \frac{z}{a_{n}}
  }
  e^{g_{n}(z)}
}^{m_{n}}
=
e^{u_{n}(z)},
$$
where
$$
\absval{ u_{n}(z) }
\leq
2
\absval{
  m_{n}
  (K a_{n}^{-1})^{k_{n}}
}.
$$

Now $m_{n}$ and $a_{n}$ are given, but $k_{n}$ is at our disposal;
since $K a_{n}^{-1} < \half$, we
choose $k_{n}$ to be the smallest number such that
$2 \absval{m_{n} (K a_{n}^{-1})^{k_{n}}} < b_{n}$,
where
$\sum_{n=1}^{\infty} b_{n}$ is any convergent series\footnote{E.g. we
  might take $b_{n} = 2^{-n}$.} of positive terms.

Hence
$$
\prod_{n = N+1}^{\infty}
\thebracket{
  \thebrace{
    \theparen{
      1
      -
      \frac{z}{a_{n}}
    }
    e^{g_{n}(z)}
  }^{m_{n}}
}
=
\prod_{n = N+1}^{\infty}
e^{u_{n}(z)},
$$
where
$ \absval{ u_{n}(z) } < b_{n} $;
and therefore, since $b_{n}$ is independent of $z$, the
product converges absolutely and uniformly when
$\absval{z} < K$, except near the points $a_{n}$.
%
% 139
%

Now let
$$
F(z)
=
\prod_{n=1}^{\infty}
\thebracket{
  \thebrace{
    \theparen{
      1
      -
      \frac{z}{a_{n}}
    }
    e^{g_{n}(z)}
  }^{m_{n}}
}.
$$

Then, if $f(z) \div F(z) = G_{1}(z)$, $G_{1}(z)$ is an integral
function \hardsubsectionref{5}{6}{4}) of $z$ and has no zeros.

It follows that
$\frac{1}{G_{1}(z)} \frac{\dd}{\dd z} G_{1}(z)$
is analytic for all finite values of $z$; and so, by Taylor's theorem,
this function can be expressed as a series $\sum_{n=1}^{\infty} n
b_{n} z^{n-1}$ converging everywhere; integrating, it follows that
$$
G_{1}(z) = c e^{G(z)},
$$
where $G(z) = \sum_{n=1}^{\infty} b_{n} z^{n}$ and $c$ is a constant;
this series converges everywhere, and so $G(z)$ is an integral
function.

Therefore, finally,
$$
f(z)
=
f(0)
e^{G(z)}
\prod_{n=1}^{\infty}
\thebracket{
  \thebrace{
    \theparen{
      1
      -
      \frac{z}{a_{n}}
    }
    e^{g_{n}(z)}
  }^{m_{n}}
}
$$
where $G(z)$ is some integral function such that $G(0) = 0$.
%\begin{smalltext}
[Note. The presence of the arbitrary element $G(z)$ which occurs in
this formula for $f(z)$ is due to the lack of conditions as to the
behaviour of $f(z)$ as $\absval \rightarrow \infty$]

\corollary. If $m_{n} =1$, it is
sufficient to take $k_{n} = n$, by\hardsubsectionref{2}{3}{6}.
%\end{smalltext}
\Section{7}{7}{The expansion of a class of periodic functions in a series of
cotangents.}

Let $f(z)$ be a periodic function of $z$, analytic except at a certain
number of simple poles; for convenience, let $\pi$ be the period of
$f(z)$ so that $f(z) = f(z + \pi)$.

Let $z = x + iy$ and let $f(z) \rightarrow l$ uniformly with respect
to $x$ as $y \rightarrow +\infty$,
when $0 \leq x \leq \pi$; similarly let $f(z) \rightarrow l'$
uniformly as $y \rightarrow -\infty$.

Let the poles of $f(z)$ in the strip $0 < x \leq \pi$ be at
$a_{1}, a_{2}, \ldots, a_{n}$; and
let the residues at them be $c_{1}, c_{2}, \ldots, c_{n}$.

Further, let $ABCD$ be a rectangle whose corners are\footnote{If any
  of the poles are on $x = \pi$, shift the rectangle slightly to
  the right; $\rho, \rho'$ are to be taken so large that
  $a_{1}, a_{2}, \ldots, a_{n}$ are
  inside the rectangle.}
$-i\rho$, $\pi - i\rho$, $\pi + i\rho'$, and $i\rho'$ in order.

Consider
$$
\frac{1}{2 \pi i}
\int f(t) \cot (t-z) \dmeasure t
$$
taken round this rectangle; the residue of the integrand at $a_{r}$ is
$c_{r} \cot (a_{r}-z)$, and the residue at $z$ is $f(z)$.

Also the integrals along $DA$ and $CB$ cancel on account of the
periodicity of the integrand; and as $\rho \rightarrow \infty$,
the integrand on $AB$ tends
uniformly to $l' i$, while as $\phi' \rightarrow \infty$
the integrand on $CD$ tends uniformly
to $-li$; therefore
$$
\half (l' - l)
=
f(z)
+
\sum_{r=1}^{n}
c_{r}
\cot (a_{r} - z).
$$
%
% 140
%

That is to say, we have the expansion
$$
f(z)
=
\half (l' - l)
+
\sum_{r=1}^{n}
c_{r}
\cot (z - a_{r}).
$$
\begin{wandwexample}
  \begin{align*}
    \cot (x - a_{1})
    \cot (x - a_{2})
    \cdots
    \cot (x - a_{n})
    =&
    \sum_{r=1}^{n}
    \cot (a_{r} - a_{1})
    \cdots
    *
    \cdots
    \cot (a_{r} - a_{n})
    \cot (x - a_{r})
    +
    (-)^{\half n},
    \\
    \textrm{or}
    =&
    \sum_{r=1}^{n}
    \cot (a_{r} - a_{1})
    \cdots
    *
    \cdots
    \cot (a_{r} - a_{n})
    \cot (x - a_{r}),
  \end{align*}
according as $n$ is even or odd; the `$*$' means that the factor
$\cot (a_{r} - a_{r})$ is omitted.
\end{wandwexample}
\begin{wandwexample}
Prove that
\begin{align*}
  \frac{
    \sin (x - b_{1})
    \sin (x - b_{2})
    \cdots
    \sin (x - b_{n})
  }{
    \sin (x - a_{1})
    \sin (x - a_{2})
    \cdots
    \sin (x - a_{n})
  }
  =&
  \frac{
    \sin (a_{1} - b_{1})
    \cdots
    \sin (a_{1} - b_{n})
  }{
    \sin (a_{1} - a_{2})
    \cdots
    \sin (a_{1} - a_{n})
  }
  \cot (x - a_{1})
  \\
  &
  +
  \frac{
    \sin (a_{2} - b_{1})
    \cdots
    \sin (a_{2} - b_{n})
  }{
    \sin (a_{2} - a_{1})
    \cdots
    \sin (a_{2} - a_{n})
  }
  \cot (x - a_{2})
  \\
  &
  +
  \cdots
  \\
  &
  +
  \cos (a_{1} + a_{2} + \cdots + a_{n}
  - b_{1} - b_{2} - \cdots - b_{n}).
\end{align*}
\end{wandwexample}
\Section{7}{8}{Borel's theorem.}
\footnote{TODO Lemons sur les series divergentes (1901), p. 94. See also the
memoirs there cited.}

Let $f(z) = \sum_{n=0}^{\infty} a_{n} z^{n}$ be analytic when
$\absval{z} \leq r$, so that, by
\hardsubsectionref{5}{2}{3},
$\absval{ a_{n} r^{n} } < M$
where $M$ is independent of $n$.

Hence, if
$\phi(z) = \sum_{n=0}^{\infty} \frac{a_{n} z^{n}}{n!}$,
$\phi(z)$ is an integral function, and
$$
\absval{ \phi(z) }
<
\sum_{n=0}^{\infty} \frac{M \absval{z^{n}} }{ r^{n} \cdot n!}
=
M e^{\absval{z}/r},
$$
and similarly
$\absval{ \phi^{(n)}(z) } < M e^{\absval{z}/r}/r^{n}$.

Now consider
$f_{1}(z) = \int_{0}^{\infty} e^{-t} \phi(zt) \dmeasure t$;
this integral is an analytic function
of $z$ when $\absval{z} < r$, by\hardsubsectionref{5}{3}{2}.

Also, if we integrate by parts,
\begin{align*}
  f_{1}(z)
  =&
  \thebracket{- e^{-t} \phi(zt) }_{0}^{\infty}
  +
  z \int_{0}^{\infty} e^{-t} \phi'(zt) \dmeasure t
  \\
  =&
  \sum_{m=0}^{n}
  z^{m}
  \thebracket{ - e^{-t} \phi^{(m)}(zt) }_{0}^{\infty}
  +
  z^{n+1}
  \int_{0}^{\infty}
  e^{-t} \phi^{(n+1)}(zt) \dmeasure t.
\end{align*}

But $\lim_{t \rightarrow 0} e^{-t} \phi^{(m)}(zt) = a_{m}$; and,
when $\absval{z} < r$,
$\lim_{t \rightarrow \infty} e^{-t} \phi^{(m)}(zt) = 0$.

Therefore
$$
f_{1}(z) = \sum_{m=0}^{n} a_{m} z^{m} + R_{n},
$$
%
% 141
%
where
\begin{align*}
  TODO
\end{align*}

Consequently, when $\absval{z} < r$,
$$
f_{1}(z) = \sum_{m=0}^{\infty} a_{m} z^{m} = f(z);
$$
and so
$$
f(z) = \int_{0}^{\infty} e^{-t} \phi(zt) \dmeasure t,
$$
where
$
\phi(z) = \sum_{n=0}^{\infty} \frac{a_{n} z^{n}}{n!};
$
is called \emph{Borel's function} associated with
$\sum_{n=0}^{\infty} a_{n} z^{n}$.

If
$S = \sum_{n=0}^{\infty} a_{n}$
and
$\phi(z) = \sum_{n=0}^{\infty} \frac{a_{n} z^{n}}{n!}$
and if we can establish the relation
$S = \int_{0}^{\infty} e^{-t} \phi(t) \dmeasure t$,
the series $S$ is said \hardsubsectionref{8}{4}{1}) to be
'\emph{summable (B)}; so that the
theorem just proved shews that a Taylor's series representing an
analytic function is summable (B).

\Subsection{7}{8}{1}{Borel's integral and analytic continuation.}
We next obtain Borel's result that his integral represents an analytic
function in a more extended region than the interior of the circle
$\absval{z} = r$.

TODO:figure

This extended region is obtained as follows: take the singularities
$a,b,c,\ldots$ of $f(z)$ and through each of them draw a line perpendicular
to the line joining that singularity to the origin. The lines so drawn
will divide the plane into regions of which one is a polygon with the
origin inside it.

\emph{Then Borel's integral represents an analytic function}
(which, by\hardsectionref{5}{5}
and\hardsectionref{7}{8}, is obviously that defined by $f(z)$
and its continuations)
\emph{throughout the interior of this polygon.} The reader will observe that
this is the first actual formula obtained for the analytic
continuation of a function, except the trivial one of
\hardsectionref{5}{5}, example.

For, take any point $P$ with affix $\zeta$ inside the polygon; then
the circle on $OP$ as diameter has no singularity on or inside
it\footnote{The reader will see this from the figure; for if there were such a
singularity the corresponding side of the polygon would pass between
$O$ and $P$; \ie,
 $P$ would be outside the polygon.}; and
consequently we can draw a slightly
%
% 142
%
larger concentric circle\footnote{The differeuce of the radii of the
  circles being, say, $\delta$.} $C$ with no singularity on or inside
it. Then, by\hardsectionref{5}{4},
$$
a_{n}
=
\frac{1}{2 \pi i}
\int_{C} \frac{f(z)}{z^{n+1}} \dmeasure z,
$$
and so
$$
\phi(\zeta t)
=
\frac{1}{2 \pi i}
\sum_{n=0}^{\infty}
\frac{\zeta^{n} t^{n}}{n!}
\int_{C} \frac{f(z)}{z^{n+1}} \dmeasure z;
$$
but
$\sum_{n=0}^{\infty} \frac{\zeta^{n} t^{n}}{n!} \frac{f(z)}{z^{n+1}}$
converges uniformly \hardsubsectionref{3}{3}{4}) on $C$ since
$f(z)$ is bounded and $\absval{z} \geq \delta > 0$, where
$\delta$ is independent of $z$; therefore, by\hardsectionref{4}{7},
$$
\phi(\zeta t)
=
\frac{1}{2 \pi i}
\int_{C} z^{-1} f(z) \exp(\zeta t z^{-1}) \dmeasure z,
$$
and so, when $t$ is real,
$\absval{\phi(\zeta t)} < F(\zeta) e^{\lambda t}$,
where $F(\zeta)$ is bounded in any closed region lying wholly
\emph{inside} the polygon and is independent of $t$;
and $\lambda$ is the greatest value of the real part of
$\zeta / z$ on $C$.

If we draw the circle traced out by the point $z/\zeta$, we see that
the real part of $\zeta/z$ is greatest when $z$ is at the extremity of the
diameter through $\zeta$, and so the value of $\lambda$ is
$ \absval{\zeta} \cdot \thebrace{\absval{\zeta} + \delta}^{-1} < 1$.

We can get a similar inequality for $\phi'(\zeta t)$ and hence,
by\hardsubsectionref{5}{3}{2},
$\int_{0}^{\infty} e^{-t} \phi(\zeta t) \dmeasure t$
is analytic at $\zeta$ and is obviously a one-valued function of
$\zeta$.

This is the result stated above.

\Subsection{7}{8}{2}{Expansions in series of inverse factorials.}

A mode of development of functions, which, after being used by
Nicole\footnote{TODO Mem de VAcad. des Sci. (Paris, 1717); see Tweedie, Proc. Edin. Math.
  Soc. xxxvi. (1918).}
and
Stirling\footnote{TODO Methodus Dijferentialis (Londou, 1730).}
in the eighteenth century, was systematically
investigated by
\Schlomilch\footnote{TODO Compendium der h'dheren AnalysU. More recent investigations are due
to Kluyver, Nielsen and Pincherle. See Comptes liendiis, cxxxiii.
(1901), cxxxiv. (1902), Annales de I'Ecole norm, sup. (3), XIX.,
XIII., xxiii., JRendiconti del Lincei, (5), xi. (1902), and Palermo
Rendiconti, xxxiv. (1912). Properties of functions defined by series
of inverse factorials have been studied in an important memoir by
Norlund, Acta Math, xxxvii. (1914), pp. 327-H87.}
in 1863, is that of expansion in a series
of inverse factorials.

To obtain such an expansion of a function analytic when
$\absval{z} > r$, we let
the function be
$f(z) = \sum_{n=0}^{\infty} a_{n} z^{-n}$, and use the formula
$f(z) = \int_{0}^{\infty} z e^{-tz} \phi(z) \dmeasure t$,
where $\phi(t) = \sum_{n=0}^{\infty} a_{n} t^{n} / n!$;
this result may be obtained in the same way as
that of\hardsectionref{7}{8}.
Modify this by writing
$e^{-t} = 1 - \xi$, $\phi(t) = F(\xi)$;
then
$$
f(z)
=
\int_{0}^{1}
z (1 - \xi)^{z-1} F(\xi) \dmeasure \xi.
$$

Now if $t = u + iv$ and if $t$ be confined to the strip
$-\pi < v < \pi$, $t$ is a
one-valued function of $\xi$ and $F(\xi)$ is an analytic function of
$\xi$; and $\xi$ is
restricted so that $-\pi < \arg (1-\xi) < \pi$. Also the interior of the
circle $\absval{\xi} = 1$ corresponds
%
% 143
%
to the interior of the curve traced out by the point
$t = - \log \theparen{2 \cos \half \theta} + \half i \theta$,
(writing $\xi= \exp \thebrace{i (\theta + \pi) }$ ); and inside this curve
$$
\absval{t} - R(t)
\leq
\sqrt{ \thebrace{R(t)}^{2} + \pi^{2} }
-
R(t)
\rightarrow
0,
$$
as $R(t) \rightarrow \infty$.

It follows that, when $\absval{\xi} \leq 1$,
$\absval{F(\xi)} < M e^{r\absval{t}} < M_{1} \absval{e^{rt}}$,
where $M_{1}$ is independent of $t$; and so
$F(\xi) < M_{1} \absval{(1-\xi)^{-r}}$.

Now suppose that $0 \leq \xi < 1$; then, by
\hardsubsectionref{5}{2}{3},
$\absval{F^{(n)}(\xi)} < M_{2} n! \rho^{-n}$
where $M_{2}$ is the upper bound of
$\absval{F(z)}$ on a circle with centre $\xi$ and
radius $\rho < 1 - \xi$.

Taking $\rho = \frac{n}{n+1} (1-\xi)$ and observing
that\footnote{ $(1 + x^{-1})^{x}$ increases with $x$;
  for $\frac{1}{1-y} > e^{y}$, when $y < 1$, and so
  $\log \theparen{\frac{1}{1-y}} > y$. That is to
  say, putting $y^{-1} = 1+x$,
  $
  \frac{\dd}{\dd x} x \log (1 + x^{-1})
  =
  \log (1 + x^{-1})
  -
  \frac{1}{1+x}
  >
  0
  $.
}
$(1 + n^{-1})^{n} < e$
we find that
\begin{align*}
  \absval{F^{(n)}}
  <&
  M_{1}
  \thebracket{
    1
    -
    \thebrace{
      \xi
      +
      \frac{n}{n+1} \xi
    }
  }^{-r}
  \cdot
  n!
  \thebrace{
    \frac{n (1-\xi)}{n+1}
  }^{-n}
  \\
  <&
  M_{1} e (n+1)^{r} n! (1-\xi)^{-r-n}.
\end{align*}

Remembering that, by\hardsectionref{4}{5}, $\int_{0}^{1}$ means
$\lim_{\epsilon \rightarrow 0} \int_{0}^{1-\epsilon}$, we
have, by repeated integrations by parts,
\begin{align*}
  f(z)
  =&
  \lim_{\epsilon \rightarrow +0}
  \thebracket{
    -(1 - \xi)^{z} F(\xi)
  }_{0}^{1-\epsilon}
  +
  \int_{0}^{1 - \epsilon}
  (1 - \xi)^{z} F'(\xi) \dmeasure \xi
  \\
  =&
  \thebracket{
    -(1 - \xi)^{z} F(\xi)
  }_{0}^{1 - \epsilon}
  +
  \frac{1}{z + 1}
  \thebracket{
    -(1 - \xi)^{z+1} F'(\xi)
  }_{0}^{1 - \epsilon}
  \\
  &
  \quad
  \quad
  +
  \frac{1}{z + 1}
  \int_{0}^{1 - \epsilon}
  (1 - \xi)^{z+1} F''(\xi) \dmeasure \xi
  \\
  =&
  \cdots
  \\
  =&
  b_{0}
  +
  \frac{b_{1}}{z+1}
  +
  \frac{b_{2}}{(z+1)(z+2)}
  +
  \cdots
  +
  \frac{b_{n}}{(z+1)(z+2)\cdots(z+n)}
  +
  R_{n},
\end{align*}
where
\begin{align*}
  b_{n}
  =&
  \lim_{\epsilon \rightarrow 0}
  \thebracket{
    -(1-\xi)^{z+n} F^{(n)}(\xi)
  }_{0}^{1 - \epsilon}
  \\
  =&
  F^{(n)}(0),
\end{align*}
if the real part of $z+n-r-n>0$, \ie if $\Re(z) > r$;
further
\begin{align*}
  \absval{R_{n}}
  \leq &
  \frac{1}{ \absval{ (z+1)(z+2)\cdots(z+n)  }}
  \lim_{\epsilon \rightarrow 0}
  \int_{0}^{1 - \epsilon}
  \absval{ (1-\xi)^{z+n} F^{(n+1)}(\xi) }
  \dmeasure \xi
  \\
  <&
  \frac{ M_{1} e (n+2)^{r} n!}{
    \absval{(z+1)(z+2)\cdots(z+n)} \Re(z-r)}
  \\
  <&
  \frac{ M_{1} e (n+2)^{r} n!}{
    \absval{ (r+1+\delta)(r+2+\delta) \cdots (r+n+\delta) \delta }
    },
\end{align*}
where $\delta = \Re(z - r)$.
%
% 144
%

Now
$$
\prod_{m=1}^{n}
\thebrace{
  \theparen{
    1
    +
    \frac{r + \delta}{m}
  }
  e^{-\frac{r+\delta}{m}}
}
$$
tends to a limit \hardsubsectionref{2}{7}{1}) as
$n \rightarrow \infty$, and so $\absval{R_{n}} \rightarrow \infty$
if
$ (n+2)^{r} e^{-(r+\delta) \sum_{1}^{n} 1/m} $
tends to zero; but
$$
\sum_{m=1}^{n} 1/m
>
\int_{1}^{n+1} \frac{\dmeasure x}{x}
=
\log(n+1),
$$
by\hardsubsectionref{4}{4}{3} (II), % TODO:insertref
and $(n + 2)^{r} (n+1)^{-r-\delta} \rightarrow \infty$ when
$\delta > 0$; therefore $R_{n} \rightarrow 0$ as
$n \rightarrow \infty$, and so, when $\Re(z) > r$,
we have the convergent expansion
$$
f(z)
=
b_{0}
+
\frac{b_{1}}{z+1}
+
\frac{b_{2}}{(z+1)(z+2)}
+
\cdots
\frac{b_{n}}{(z+1)(z+2)\cdots(z+n)}
+
\cdots.
$$
\begin{wandwexample}
Obtain the same expansion by using the results
$$
\frac{1}{(z+1)(z+2)\cdots(z+n+1)}
=
\frac{1}{n!}
\int_{0}^{1} u^{n} (1-u)^{z} \dmeasure z,
$$
$$
\int_{C}
\frac{ f(t) \dmeasure t }{z - t}
=
\int_{C} \dmeasure t
\int_{0}^{1}
f(t) (1-u)^{z-t-1} \dmeasure u.
$$
\end{wandwexample}
\begin{wandwexample}
Obtain the expansion
$$
\log\theparen{1 + \frac{1}{z}}
=
\frac{1}{z}
-
\frac{a_{1}}{z(z+1)}
-
\frac{a_{2}}{z(z+1)(z+2)}
-
\cdots,
$$
where
$$
a_{n}
=
\int_{0}^{1}
t (1-t) (2-t) \cdots (n-1-t) \dmeasure t,
$$
and discuss the region in which it converges.
\addexamplecitation{Schlomilch.}
\end{wandwexample}

%TODO
REFERENCES. E. Goursat, Cours d' Analyse (Paris, 1911), Chs. xv, xvi.
E. BoREL, Lecons sur les series divergentes (Paris, 1901). T. J. I'a.
Bromwich\footnote{The expansions considered by Eromwich are obtained by elementary
methods, i.e. without the use of Cauchy's theorem.} Theory of Infinite
Series (1908), Chs. viii, x, xi. 0.
Schlomilch, Compendium der hoheren Analysis, ii. (Dresden, 1874).

\begin{wandwmiscexamples}
  \begin{wandwmiscexample}
    If $y - x - \phi(y) = 0$, where $\phi$ is a given function of
    its argument, obtain the expansion
    $$
    f(y)
    =
    f(x)
    +
    \sum_{m=1}^{\infty}
    \frac{1}{m!}
    \thebrace{\phi(x)}^{m}
    \theparen{
      \frac{1}{1-\phi'(x)}
      \frac{\dd}{\dd x}
    }^{m}
    f(x)
    $$
    where $f$ denotes any analytic function of its argument, and discuss
    the range of its validity.
    \addexamplecitation{TODO:Levi-Civitk, Bertd. dei Lincei, (5), xvl
      (1907), p. 3.}
  \end{wandwmiscexample}
  \begin{wandwmiscexample}
    Obtain (from the formula of Darboux or otherwise) the expansion
    $$
    f(z) - f(z)
    =
    \sum_{n=1}^{\infty}
    \frac{(-)^{n-1} (z-a)^{n}}{n! (1-r)^{n}}
    \thebrace{
      f^{(n)}(z) - r^{n} f^{(n)}(a)
    };
    $$
    find the remainder after $n$ terms, and discuss the
    convergence of the series.
  \end{wandwmiscexample}
  % 145
  %
  \begin{wandwmiscexample}
    Shew that
    \begin{align*}
      f(x+h) - f(x)
      =&
      \sum_{m=1}^{n}
      (-)^{m-1}
      \frac{1 \cdot 3 \cdot 5 \cdots (2m-1)}{(m!)^{2}}
      \frac{h^{m}}{2^{m}}
      \thebrace{
        f^{(m)}(x+h)
        -
        (-)^{m} f^{(m)}(x)
      }
      \\
      & \quad
      +
      (-)^{n} h^{n+1}
      \int_{0}^{1}
      \gamma_{n}(t)
      f^{(n+1)}(x + ht) \dmeasure t,
    \end{align*}
    where
    $$
    \gamma_{n}(t)
    =
    \frac{x^{n+\half} (1-x)^{n + \half}}{(n!)^{2}}
    \frac{\dd^{n}}{\dd x^{n}}
    \thebrace{
      x^{-\half} (1-x)^{-\half}
    }
    =
    \frac{1}{\pi n!}
    \int_{0}^{1}
    (x-z)^{n}
    z^{-\half}
    (1-z)^{-\half}
    \dmeasure z,
    $$
    and shew that $\gamma_{n}(x)$ is the coefficient of
    $n! t^{n}$ in the expansion of
    $\thebrace{ (1-tx)(1+t-tx) }^{-\half}$
    in ascending powers of $t$.
  \end{wandwmiscexample}
  \begin{wandwmiscexample}
    By taking
    $$
    \phi(x+1)
    =
    \frac{1}{n!}
    \thebracket{
      \frac{\dd^{n}}{\dd u^{n}}
      \thebrace{
        \frac{(1-r)e^{xu}}{1 - r e^{-u}}
      }
    }_{n=0}
    $$
    in the formula of Darboux, shew that
    \begin{align*}
      f(x+h) - f(x)
      =&
      -
      \sum_{m=1}^{n}
      a_{m}
      \frac{h^{m}}{m!}
      \thebrace{
        f^{(m)}(x+h) - \frac{1}{r} f^{(m)}(x)
      }
      \\
      & \quad
      +
      (-)^{n} h^{n+1}
      \int_{0}^{1}
      \phi(t) f^{(n+1)}(x+ht) \dmeasure t,
    \end{align*}
    where
    $$
    \frac{1-r}{1 - r e^{-u}}
    =
    1
    -
    a_{1} \frac{u}{1!}
    +
    a_{2} \frac{u^{2}}{2!}
    -
    a_{3} \frac{u^{3}}{3!}
    +
    \cdots.
    $$
  \end{wandwmiscexample}
  \begin{wandwmiscexample}
    Shew that
    \begin{align*}
      f(z) - f(a)
      =&
      \sum_{m=1}^{n}
      (-)^{m-1}
      \frac{2 B_{m} (2^{2n} - 1)(z-a)^{2m-1}}{2m!}
      \thebrace{
        f^{(2m-1)}(a)
        +
        f^{(2m-1)}(z)
      }
      \\
      &
      \hfill
      \frac{(z-a)^{2n+1}}{2n!}
      \int_{0}^{1}
      \psi_{2n}(t)
      f^{(2n+1)}\thebrace{
        a + t(z-a)
      }
      \dmeasure t,
    \end{align*}
    where
    $$
    \psi_{n}(t)
    =
    \frac{2}{n+1}
    \thebracket{
      \frac{\dd^{n+1}}{\dd u^{n+1}}
      \theparen{
        \frac{u e^{tu}}{e^{u} + 1}
      }
    }_{u=0}.
    $$
  \end{wandwmiscexample}
  \begin{wandwmiscexample}
    Prove that
    \begin{align*}
      &
      f(z_{2}) - f(z_{1})
      =
      C_{1} (z_{2} - z_{1}) f'(z_{2})
      +
      C_{2} (z_{2} - z_{1})^{2} f''(z_{1})
      -
      C_{3} (z_{2} - z_{1})^{3} f'''(z_{2})
      \\
      &
      -C_{4} (z_{2}-z_{1})^{4} f^{\textrm{iv}}(z_{1})
      +
      \cdots
      +
      (-)^{n} (z_{2} - z_{1})^{n+1}
      \int_{0}^{1}
      \thebrace{
        \frac{\dd^{n}}{\dd u^{n}}
        \theparen{
          e^{tu} \sech u
        }
      }_{u=0}
      f^{(n+1)}(z_{1} + t z_{2} - t z_{1})
      \dmeasure t;
    \end{align*}
    in the series plus signs and minus signs occur in pairs, and the last
    term before the integral is that involving
    $(z_{2}-z_{1})^{n}$, also $C_{n}$ is the
    coefficient of $z^{n}$ in the expansion of
    $\cot\theparen{\frac{\pi}{4} - \frac{z}{2}}$
    in ascending powers of $z$. \addexamplecitation{Trinity, 1899.}
  \end{wandwmiscexample}
  \begin{wandwmiscexample}
    If $x_{1}$ and $x_{2}$ are integers, and $\phi(z)$ is a function
    which is analytic and bounded for all values of $z$ such that
    $x_{1} \leq \Re(z) \leq x_{2}$, shew (by integrating
    $$
    \int \frac{\phi(z) \dmeasure z}{ e^{\pm 2 \pi i z} - 1 }
    $$
    round indented rectangles whose corners are
    $x_{1}$, $x_{2}$, $x_{2} \pm \infty i$, $x_{1} \pm \infty i$)
    that
    \begin{align*}
      &
      \half \phi(x_{1})
      + \phi(x_{1} + 1)
      + \phi(x_{1} + 2)
      + \cdots
      + \phi(x_{2} - 1)
      + \half \phi(x_{2})
      \hfill
      \\
      &
      \hfill
      \int_{x_{1}}^{x_{2}} \phi(z) \dmeasure z
      +
      \frac{1}{i}
      \int_{0}^{\infty}
      \frac{ \phi(x_{2}+iy) - \phi(x_{1}+iy)
        - \phi(x_{2}-iy) + \phi(x_{1}-iy)}{ e^{2 \pi y} - 1 }
      \dmeasure y.
    \end{align*}
    Hence, by applying the theorem
    $$
    4n
    \int_{0}^{\infty}
    \frac{y^{2n-1}}{e^{2 \pi y} - 1}
    \dmeasure y
    =
    B_{n}
    $$
    %
    % 146
    %
    where $B_{1}, B_{2}, \ldots$ are \Bernoulli's numbers, shew that
    $$
    \phi(1) + \phi(2) + \cdots + \phi(n)
    =
    C
    + \half \phi(n)
    + \int^{n} \phi(z) \dmeasure z
    +
    \sum_{r=1}^{\infty}
    \frac{(-)^{r-1} B_{r}}{2r!} \phi^{(2r-1)}(n),
    $$
    (where $C$ is a constant not involving $n$), provided that the
    last series converges.

    (This important formula is due to TODO Plana, Mem. della R, Accad. di
    Torino, xxv. (1820), pp. 403-418; a proof by means of contour
    integration was published by Kronecker, Journal fur Math. cv. (1889),
    pp. 345-348. For a detailed history, see Lindelof, Le Calcul des
    Residus. Some applications of the formula are given in Chapter xii.)
  \end{wandwmiscexample}
  \begin{wandwmiscexample}
    Obtain the expansion
    $$
    u
    =
    \frac{x}{2}
    +
    \sum_{n=2}^{\infty}
    (-)^{n-1}
    \frac{1 \cdot 3 \cdots (2n-3)}{n!}
    \frac{x^{n}}{2^{n}}
    $$
    for one root of the equation
    $x = 2u + u^{2}$ and shew that it converges so
    long as $\absval{x} < 1$.
  \end{wandwmiscexample}
  \begin{wandwmiscexample}
    If $S^{(m)}_{2n+1}$ denote the sum of all combinations of the numbers
    $$
    1^{2}, 3^{2}, 5^{2}, \ldots (2n-1)^{2},
    $$
    taken $m$ at a time, shew that
    $$
    \frac{\cos z}{z}
    =
    \frac{1}{\sin z}
    +
    \sum_{n=0}^{\infty}
    \frac{(-)^{n+1}}{(2n+2)!}
    \thebrace{
      \frac{2^{2(n+1)}}{2n+3}
      -
      S^{(1)}_{2(n+1)}
      \frac{2^{2n}}{2n+1}
      +
      \cdots
      +
      (-)^{n}
      S^{(n)}_{2(n+1)}
      \frac{2^{2}}{3}
    }
    \sin^{2n+1} z.
    $$
    \addexamplecitation{Teixeira.}
  \end{wandwmiscexample}
  \begin{wandwmiscexample}
    If the function $f(z)$ is analytic in the interior of that one of
    the ovals whose equation is $\absval{\sin z} = C$
    (where $C \leq 1$), which includes the origin, shew that $f(z)$
    can, for all points $z$ within this oval, be
    expanded in the form
    \begin{align*}
      f(z)
      =&
      f(0)
      +
      \sum_{n=1}^{\infty}
      \frac{ f^{(2n)}(0)
        + S^{(1)}_{2n} f^{(2n-2)}(0)
        + \cdots
        S^{(n-1)}_{2n} f''(0)
      }{2n!}
      \sin^{2n} z
      \\
      &
      \quad
      \sum_{n=0}^{\infty}
      \frac{
        f^{(2n+1)}(0)
        + S^{(1)}_{2n+1} f^{(2n-1)}(0)
        + \cdots
        + S^{(n)}_{2n+1} f'(0)
      }{(2n+1)!}
      \sin^{2n+1} z,
    \end{align*}
    where $S^{(m)_{2n}}$ is the sum of all combinations of the numbers
    $$
    2^{2}, 4^{2}, 6^{2}, \ldots, (2n-2)^{2},
    $$
    taken $m$ at a time, and $S^{(m)}_{2n+1}$ denotes the
    sum of all combinations of the numbers
    $$
    1^{2}, 3^{2}, 5^{2}, \ldots, (2n-1)^{2},
    $$
    taken $m$ at a time.
    \addexamplecitation{Teixeira.}
  \end{wandwmiscexample}
  \begin{wandwmiscexample}
    Shew that the two series
    $$
    2z
    + \frac{2 z^{3}}{3^{2}}
    + \frac{2 z^{5}}{5^{2}}
    + \cdots
    $$
    and
    $$
    \frac{2z}{1 - z^{2}}
    -
    \frac{2}{1 \cdot 3^{2}}
    \theparen{
      \frac{2z}{1 - z^{2}}
    }^{3}
    +
    \frac{2 \cdot 4}{3 \cdot 5^{2}}
    \theparen{
      \frac{2z}{1 - z^{2}}
    }^{5}
    -
    \cdots
    $$
    represent the same function in a certain region of the $z$ plane,
    and can be transformed into each other by \Burmann's theorem.

    \addexamplecitation{TODO Kapteyn, Nieuw Archief, (2), iii. (1897), p. 225.}
  \end{wandwmiscexample}
  \begin{wandwmiscexample}
    If a function $f(z)$ is periodic, of period $2 \pi$, and is
    analytic at all points in the infinite strip of the plane,
    included between the two branches of the curve
    $\absval{\sin z} = C$ (where $C > 1$),
    shew that at all points in the strip it can be expanded in
    an infinite series of the form
    \begin{align*}
      f(z)
      =&
      A_{0} + A_{1} \sin z + \cdots + A_{n} \sin^{n} z + \cdots
      \\
      &
      \hfill
      + \cos z
      ( B_{1} + B_{2} \sin z + \cdots + B_{n} \sin^{n-1} z + \cdots );
    \end{align*}
    and find the coefficients
    $A_{n}$ and $B_{n}$.
  \end{wandwmiscexample}
  %
  % 147
  %
  \begin{wandwmiscexample}
    If $\phi$ and $f$ are connected by the equation
    $$
    \phi(x) + \lambda f(x) = 0,
    $$
    of which one root is $a$,
    shew that
    $$
    TODO
    $$
    the general term being
    $
    (-)^{m}
    \frac{\lambda^{m}}{1! 2! \cdots m! (\phi')^{\half m(m+1)}}
    $
    multiplied by a determinant in which
    the elements of the first row are
    $\phi', (\phi^{2})', (\phi^{3})', \ldots, (\phi^{m-1})', (f^{m} F')$
    and each row is the differential coefficient of the preceding
    one with respect to $a$; and
    $F, f, F', \ldots$ denote
    $F(a), f(a), F'(a), \ldots$.

    (TODOWronski, Philosophie de la Technie, Section ii. p. 381. For proofs of
    the theorem see Cayley, Quarterly Journal, xil. (1873), Transon, Nouv.
    Ann. de Math. xill. (1874j, and C Lagrange, Brux. Mem. Couronnes, 4",
    xlvii. (1886), no. 2.)
  \end{wandwmiscexample}
  \begin{wandwmiscexample}
    If the function $W(a, b, x)$ be defined by the series
    $$
    W(a,b,x)
    =
    x
    + \frac{a-b}{2!} x^{2}
    + \frac{(a-b)(a-2b)}{3!} x^{3}
    + \cdots,
    $$
    which converges so long as
    $$
    \absval{x} < \frac{1}{\absval{b}},
    $$
    shew that
    $$
    \frac{\dd}{\dd x} W(a,b,x)
    =
    1
    +
    (a-b) W(a-b,b,x);
    $$
    and shew that if
    $$
    y = W(a,b,x)
    $$
    then
    $$
    x = W(b,a,y).
    $$

    Examples of this function are
    \begin{align*}
      W(1,0,x) =& e^{x} - 1, \\
      W(0,1,x) =& \log (1+x) \\
      W(a,1,x) =& \frac{(1+x)^{a} - 1}{a}
    \end{align*}
    \addexamplecitation{\Jezek}
  \end{wandwmiscexample}
  \begin{wandwmiscexample}
    Prove that
    $$
    \frac{1}{ \sum_{n=0}^{\infty} a_{n} x^{n}}
    =
    \frac{1}{a_{0}}
    +
    \sum_{1}^{\infty}
    \frac{ (-)^{n} x^{n} }{ n! a_{0}^{n+1} } G_{n},
    $$
    where
    $$
    TODO
    $$
    and obtain a similar expression for
    $$
    \thebrace{
      \sum_{n=0}^{\infty} a_{n} x^{n}
    }^{\half}
    $$
    \addexamplecitation{TODO Mangeot, Ann. de VEcole norm. sup. (3), xiv.}
  \end{wandwmiscexample}
  \begin{wandwmiscexample}
    Shew that
    $$
    \frac{1}{ \sum_{r=0}^{n} a_{r} x^{r} }
    =
    -
    \sum_{r=0}^{\infty}
    \frac{1}{r+1}
    \frac{\partial S_{r+1}}{\partial a_{1}} x^{r},
    $$
    %
    % 148
    %
    where $S_{r}$ is the sum of the $r$-th powers of the reciprocals of
    the roots of the equation
    $$
    \sum_{r=0}^{n} a_{r} x^{r} = 0.
    $$
    \addexamplecitation{TODO Gambioli, Bologna Memorie, 1892.}
  \end{wandwmiscexample}
  \begin{wandwmiscexample}
    If $f(z)$ denote the $n$th derivate of $f(z)$, and if
    $f_{-n}(z)$ denote that one of the $n$th integrals of
    $f(z)$ which has an $n$-ple zero at $z=0$,
    shew that if the series
    $$
    \sum_{n=-\infty}^{\infty} f_{n}(z) g_{-n}(x)
    $$
    is convergent it represents a function of $z + x$;
    and if the domain of convergence includes the origin in the
    $x$-plane, the series is equal to
    $$
    \sum_{n=0}^{\infty} f_{-n}(z+x) g_{n}(0).
    $$
    Obtain Taylor's series from this result, by putting $g(z) = 1$.
    \addexamplecitation{Guichard.}
  \end{wandwmiscexample}
  \begin{wandwmiscexample}
    Shew that, if $x$ be not an integer,
    $$
    TODO
    $$
    as $\nu \rightarrow \infty$, provided that all terms for which
    $m = n$ are omitted from the summation.
    \addexamplecitation{Math. Trip. 1895.}
  \end{wandwmiscexample}
  \begin{wandwmiscexample}
    Sum the series
    $$
    \sum_{n=-q}^{p}
    \theparen{
      \frac{1}{(-)^{n} x-a-n}
      +
      \frac{1}{n}
    },
    $$
    where the value $n = 0$ is omitted, and $p,q$ are
    positive integers to be increased without
    limit.
    \addexamplecitation{Math. Trip. 1896.}
  \end{wandwmiscexample}
  \begin{wandwmiscexample}
    If
    $
    F(x)
    =
    e^{\int_{0}^{x} x \pi \cot (x \pi) \dmeasure x}
    $, shew that
    $$
    F(x)
    =
    e^{x}
    \frac{
      \prod_{n=1}^{\infty} \thebrace{
        \theparen{1 - \frac{x}{n}}^{n}
        e^{x + \half \frac{x^{2}}{n}}
      }
    }{
      \theparen{1 + \frac{x}{n}}^{n}
      e^{-x + \half \frac{x^{2}}{n}}
    }
    $$
    and that the function thus defined satisfies the relations
    $$
    F(-x) = \frac{1}{F(x)},
    \quad
    F(x) F(1-x) = 2 \sin x \pi.
    $$
    Further, if
    $$
    \psi(z)
    =
    z
    + \frac{z^{2}}{2^{2}}
    + \frac{z^{3}}{3^{2}}
    + \cdots
    =
    - \int_{0}^{z} \log (1-t) \frac{\dmeasure t}{t},
    $$
    shew that
    $$
    F(x)
    =
    e^{\half \pi i x^{2}
      -
      \frac{1}{2 \pi i}
      \psi( 1 - e^{-2 \pi i x} )
    }
    $$
    when
    $$
    \absval{ 1 - e^{-2 \pi i x} } < 1.
    $$
    \addexamplecitation{Trinity, 1898.}
  \end{wandwmiscexample}
  \begin{wandwmiscexample}
    Shew that
    \begin{align*}
      &
      \thebracket{
        1 + \theparen{\frac{k}{x}}^{n}
      }
      \thebracket{
        1 + \theparen{\frac{k}{2 \pi - x}^{n}}
      }
      \thebracket{
        1 + \theparen{\frac{k}{2 \pi + x}}^{n}
      }
      \thebracket{
        1 + \theparen{\frac{k}{4 \pi - x}^{n}}
      }
      \thebracket{
        1 + \theparen{\frac{k}{4 \pi + x}}^{n}
      }
      \\
      &
      \hfill
      \frac{
        \prod_{g=1}^{\leq \half n}
        \sqrt{1 - 2 e^{-\alpha_{g}} \cos(x + \beta_{g}) + e^{-2\alpha_{g}}}
        \sqrt{1 - 2 e^{-\alpha_{g}} \cos(x - \beta_{g}) + e^{-2\alpha_{g}}}
      }{
        2^{\half n}
        (1 - \cos x)^{\half n}
        e^{-k \cos \pi n}
      }
    \end{align*}
    where
    $$
    \alpha_{g} = k \sin \frac{2g-1}{n} \pi,
    \quad
    \beta_{b} = k \cos \frac{2g-1}{n} \pi,
    $$
    and
    $$
    0 < x < 2 \pi.
    $$
    \addexamplecitation{Mildner.}
  \end{wandwmiscexample}
  %
  % 149
  %
  \begin{wandwmiscexample}
    If $\absval{x} < 1$ and $a$ is not a positive integer, shew that
    $$
    \sum_{n=1}^{\infty}
    \frac{x^{n}}{n - a}
    =
    \frac{2 \pi i x^{a}}{1 - e^{2 a \pi i}}
    +
    \frac{x}{1 - e^{2 a \pi i}}
    \int_{C} \frac{t^{a-1} - x^{a-1}}{t - x} \dmeasure t,
    $$
    where $C$ is a contour in the plane enclosing the points $0,x$.
    \addexamplecitation{TODO Lerch, Casopis, xxi. (1892), pp. 65-68.}
  \end{wandwmiscexample}
  \begin{wandwmiscexample}
    If $\phi_{1}(z), \phi_{2}(z), \ldots$ are any polynomials in $z$,
    and if $F(z)$ be any integrable function, and if
    $\psi_{1}(z), \psi_{2}(z), \ldots$ be polynomials
    defined by the equations
    \begin{align*}
      &
      \int_{a}^{b}
      F(x) \frac{ \phi_{1}(z) - \phi_{TODO}(x) }{z - x}
      \dmeasure x
      =
      \psi_{1}(z),
      \\
      &
      \int_{a}^{b}
      F(x) \phi_{1}(x)
      \frac{ \phi_{2}(z) - \phi_{2}(x) }{z - x}
      \dmeasure x
      =
      \psi_{2}(z),
      \\
      &
      \int_{a}^{b}
      F(x)
      \phi_{1}(x) \phi_{2}(x) \cdots \phi_{m-1}(x)
      \frac{\phi_{m}(z) - \phi_{m}(x)}{z-x}
      \dmeasure x
      =
      \psi_{m}(z),
    \end{align*}
    Shew that
    \begin{align*}
      &
      \hfill
      \int_{a}^{b}
      \frac{F(x) \dmeasure x}{z - x}
      =
      \frac{\psi_{1}(z)}{\phi_{1}(z)}
      +
      \frac{\psi_{2}(z)}{\phi_{1}(z) \phi_{2}(z)}
      +
      \frac{\psi_{3}(z)}{\phi_{1}(z) \phi_{2}(z) \phi_{3}(z)}
      +
      \cdots
      \hfill
      \\
      &
      +
      \frac{\psi_{m}(z)}{\phi_{1}(z) \phi_{2}(z) \cdots \phi_{m}(z)}
      +
      \frac{1}{\phi_{1}(z) \phi_{2}(z) \cdots \phi_{m}(z)}
      \int_{a}^{b}
      F(x) \phi_{1}(x) \phi_{2}(x) \cdots \phi_{m}(x)
      \frac{ \dmeasure x }{z - x}.
    \end{align*}
  \end{wandwmiscexample}
  \begin{wandwmiscexample}
    A system of functions $p_{0}(z), p_{1}(z), p_{2}(z), \ldots$
    is defined by the equations
    $$
    p_{0}(z) = 1,
    \quad
    p_{n+1}(z) = (z^{2} + a_{n} z + b_{n}) p_{n}(z),
    $$
    where $a_{n}$ and $b_{n}$ are given functions of $n$, which tend
    respectively to the limits $0$ and $-1$ as $n \rightarrow \infty$.

    Shew that the region of convergence of a series of the form
    $\sum e_{n} p_{n}(z)$ where
    $e_{1}, e_{2}, \ldots$ are independent of $z$, is a Cassini's oval
    with the foci $+1, -1$.

    Shew that every function $f(z)$, which is analytic on and inside the
    oval, can, for points inside the oval, be expanded in a series
    $$
    f(z) = \sum (c_{n} + z c'_{n}) p_{n}(z) % TODO: verify
    $$
    where
    $$
    c_{n}
    =
    \frac{1}{2 \pi i} \int (a_{n}+z) q_{n}(z) f(x) \dmeasure z,
    \quad
    c'_{n}
    =
    \frac{1}{2 \pi i} \int q_{n}(z) f(z) \dmeasure z,
    $$
    the integrals being taken round the boundary of the region, and the
    functions $q_{n}(z)$ being defined by the equations
    $$
    q_{0} = \frac{1}{z^{2} + a_{0} z + b_{0}},
    \quad
    q_{n+1}(z)
    =
    \frac{1}{z^{2} + a_{n+1} z + b_{n+1}}
    q_{n}(z).
    $$
    \addexamplecitation{TODO Pincherle, Rend, dei Lincei, (4), v. (1889), p. 8.}
  \end{wandwmiscexample}
  \begin{wandwmiscexample}
    Let $C$ be a contour enclosing the point $a$, and let $\phi(z)$ and
    $f(z)$ be analytic when $z$ is on or inside $C$. Let $\absval{t}$ be so small that
    $$
    \absval{ t \phi(z) } < \absval{ z - a }
    $$
    when $z$ is on the periphery of $C$. By expanding
    $$
    \frac{1}{2 \pi i}
    \int_{C}
    f(z)
    \frac{1 - t \phi'(z)}{z - a - t \phi(z)}
    \dmeasure z
    $$
    in ascending powers of $t$, shew that it is equal to
    $$
    f(a)
    +
    \sum_{n=1}^{\infty}
    \frac{ t^{n} }{n!}
    \frac{ \dd^{n-1} }{ \dd a^{n-1} }
    \thebracket{ f'(a) \thebrace{\phi(a)}^{n}
    }
    $$
    Hence, by using \hardsectionref{6}{3}, \hardsubsectionref{6}{3}{1}, obtain Lagrange's theorem.
  \end{wandwmiscexample}
\end{wandwmiscexamples}
\chapter{Asymptotic Expansions and Summable Series}

\Section{Simple example of an asymptotic expansion.}

Consider the function $f(x) = \int_{x}^{\infty} t^{-1} e^{x-t}
\dmeasure t$, where $x$ is real and positive, and the path of
integration is the real axis.

By repeated integrations by parts, we obtain
$$
f(x) = \frac{1}{x} - \frac{1}{x^{2}} + \frac{2!}{x^{3}} - \cdots +
\frac{ (-)^{n-1} (n-1)!}{x^{n}} + (-)^{n} n! \int_{x}^{\infty}
\frac{e^{x-t} \dmeasure t}{t^{n+1}}.
$$

In connexion with the function $f(x)$, we therefore consider the
expression
$$
u_{n-1} = \frac{ (-)^{n-1} (n-1)!}{x^{n}},
$$
and we shall write
$$
\sum_{m=0}^{n} u_{m} = \frac{1}{x} - \frac{1}{x^{2}} +
\frac{2!}{x^{3}} - \cdots + \frac{ (-)^{n} n!}{x^{n+1}} = S_{n}(x).
$$
Then we have $\absval{u_{m}/u_{m-1}} = mx^{-1} \rightarrow \infty$.
\emph{The series $\sum u_{m}$ is therefore divergent for all values of
$x$}. In spite of this, however, the series can be used for the
calculation of $f(x)$; this can be seen in the following way.

Take any fixed value for the number $n$, and calculate the value of
$S_{n}$. We have
$$
f(x) - S_{n}(x) = (-)^{n+1} (n+1)! \int_{x}^{\infty} \frac{e^{x-t}
\dmeasure t}{t^{n+2}},
$$
and therefore, since $e^{x-t} \leq 1$,
$$
\absval{ f(x) - S_{n}(x) } = (n+1)! \int_{x}^{\infty} \frac{e^{x-t}
\dmeasure t}{t^{n+2}} < (n+1)! \int_{x}^{\infty} \frac{\dmeasure
t}{t^{n+2}} = \frac{n!}{x^{n+1}}.
$$
For values of $x$ which are sufficiently large, the right-hand member
of this equation is very small. Thus, if we take $x \geq 2n$, we have
$$
\absval{ f(x) - S_{n}(x)} < \frac{1}{2^{n+1} n^{2}},
$$
which for large values of $n$ is very small. It follows therefore that
\emph{the value of the function $f(x)$ can he calculated with great
accuracy for large values of $x$, by taking the sum of a suitable
number of terms of the series $\sum u_{m}$}.

Taking even fairly small values of $x$ and $n$
$$
S_{5}(10) = 0.09152, \quad 0 < f(10) - S_{5}(10) < 0.00012.
$$
%
% 151
%

The series is on this account said to be an asymptotic expansion of
the function $f(x)$. The precise definition of an asymptotic expansion
will now be given.

\Section{Definition of an asymptotic expansion.} A divergent series
$$
A_{0} + \frac{A_{1}}{z} + \frac{a_{2}}{z^{2}} + \cdots +
\frac{A_{n}}{z^{n}} + \cdots,
$$
in which the sum of the first $(n + 1)$ terms is $S_{n}$, is said to
be an \emph{asymptotic expansion} of a function $f(z)$ for a given
range of values of $\arg$, if the expression $R_{n}(z) = z^{n}[f(z) —
S_{n}(z)]$ satisfies the condition
$$
\lim_{ \absval{z} \rightarrow \infty } R_{n}(z) = 0 \quad (\textrm{$n$
fixed}),
$$
even though
$$
\lim_{n \rightarrow \infty} \absval{R_{n}(z)} = \infty \quad
(\textrm{$z$ fixed}).
$$
When this is the case, we can make
$$
\absval{ z^{n} [f(z) - S_{n}(z)] } < \eps,
$$
where $\eps$ is arbitrarily small, by taking $\absval{z}$ sufficiently
large.

We denote the fact that the series is the asymptotic expansion of
$f(z)$ by writing
$$
f(z) \sim \sum_{n=0}^{\infty} A_{n} z^{-n}.
$$

The definition which has just been given is due to
\Poincare\footnote{TODO}. Special asymptotic expansions had, however,
been discovered and used in the eighteenth century by Stirling,
Maclaurin and Euler. Asymptotic expansions are of great importance in
the theory of Linear Differential Equations, and in Dynamical
Astronomy; some applications will be given in subsequent chapters of
the present work.

The example discussed in \hardsectionref{8}{1} clearly satisfies the
definition just given: for, when $x$ is positive, $\absval{x^{n} [f(x)
- S_{n}(x)]} < n! x^{-1} \rightarrow 0$ as $x \rightarrow \infty$.

%\begin{Remark} For the sake of simplicity, in this chapter we shall
for the most part consider asymptotic expansions only in connexion
with real positive values of the argument. The theory for complex
values of the argument may be discussed by an extension of the
analysis.

\Subsection{Another example of an asymptotic expansion.} As a second
example, consider the function $f(x)$, represented by the series
$$
f(x) = \sum_{k=1}^{\infty} \frac{c^{k}}{x+k},
$$
where $x > 0$ and $0 < c < 1$.

% 
% 152
%

The ratio of the $k$th term of this series to the $(k- l)$th is less
than $c$, and consequently the series converges for all positive
values of $x$. We shall confine our attention to positive values of
$x$. We have, when $x > k$,
$$
\frac{1}{x+k} = \frac{1}{x} - \frac{k}{x^{2}} + \frac{k^{2}}{x^{3}} -
\frac{k^{3}}{x^{4}} + \frac{k^{4}}{x^{5}} - \cdots.
$$

If, therefore, it were allowable\footnote{It is not allowable, since
$k>x$ for all terms of the series after some definite term.} to expand
each fraction $\frac{1}{x+k}$ in this way, and to rearrange the series
for $f(x)$ in descending powers of $x$, we should obtain the formal
series
$$
\frac{A_{1}}{x} + \frac{A_{2}}{x^{2}} + \cdots + \frac{A_{n}}{x^{n}} +
\cdots,
$$
where
$$
A_{n} = (-)^{n-1} \sum_{k=1}^{\infty} k^{n-1} c^{k}.
$$
But this procedure is not legitimate, and in fact $\sum_{n=1}^{\infty}
A_{n} x^{-n}$ diverges. We can, however, shew that it is an asymptotic
expansion of $f(x)$.

For let
$$
S_{n}(x) = \frac{A_{1}}{x} + \frac{A_{2}}{x^{2}} + \cdots +
\frac{A_{n}}{x^{n}} %TODO: verify correct subscript; book is
inconsistent
$$

Then S\^\{x)= i ('- -\^4 + \^lf + ... + izyi\^)

k = l\ \ xj J x + k'

so that TODO

Now TODO converges for any given value of n and is equal to C„, say;
and hence fc=i

TODO

Consequently TODO

Example. If TODO, where x is positive and the path of integration is
the

J X

real axis, prove that

TODO

[In fact, it was shewn by Stokes in 1857 that

TODO

the upper or lower sign is to be taken according as TODO. ]
%\end{Remark}
\Section{Multiplication of asymptotic expansions.}

We shall now shew that two asymptotic expansions, valid for a common
range of values of arg\^', can be multiplied together in the same way
as ordinary series, the result being a new asymptotic expansion.

TODO

For let TODO

% 
% 153
% 
and let Sn\{z) and Tn\{z) be the sums of their first (n + 1) terms; so
that, n being fixed,

f(z) - Sn (Z) = (Z-X <t> (Z) - Tn \{z) = (z\^).

Then, if C\^ = \^o\^m + \^iB,n-i + . . . + \^m\^o. it is obvious that*

SJz)Tn\{z)= i C,nZ--\^+0(z-).

But f\{z) Cf> (Z) = \{Sn (Z) + (Z-)] [Tn \{z) + (\^-»)\}

= Sn \{Z) Tn (Z) + (Z-\^)

TO =

This result being true for any fixed value of n, we see that

TODO

\Subsection{Integration of asymptotic expansions.}

We shall now shew that it is permissible to integrate an asymptotic
expansion term by term, the resulting series being the asymptotic
expansion of the integral of the function represented by the original
series.

For let TODO and let TODO.

TODO

Then, given any positive number e, we can find Xq such that TODO when
x>Xo,

and therefore

TODO

But TODO and therefore TODO.

On the other hand, it is not in general permissible t to
diflFerentiate an asymptotic expansion; this may be seen by
considering TODO.

\Subsection{Uniqueness of an asymptotic expansion.}

A question naturally suggests itself, as to whether a given series can
be

* See § 2-11; we use o (2~") to denote any function i/- (z) such that
3" i/- (2) -»- as | 2 , -*- x . t For a theorem concerning
differentiation of asymptotic expansions representing analytic
functions, see Kitt, Bull. American Math. Soc. xxiv. (1918), pp.
225-227.

%
% 154
%

the asymptotic expansion of several distinct functions. The answer to
this is in the affirmative. To shew this, we first observe that there
are functions L \{x) which are represented asymptotically by a series
all of whose terms are zero, i.e. functions such that lim x\^L \{x) =
for every fixed value of n. The

a; -*• 00

function e~\^ is such a function when x is positive. The asymptotic
expansion* of a function J\{x) is therefore also the asymptotic
expansion of

J\{x) + L(x).

On the other hand, a function cannot be represented by more than one
distinct asymptotic expansion over the whole of a given range of
values of z; for, if

TODO

then TODO

which can only be if TODO,

Important examples of asymptotic expansions will be discussed later,
in connexion with the Gamma-function (Chapter xii) and Bessel
functions (Chapter xvii).

\Section{Methods of 'summing' series.}

We have seen that it is possible to obtain a development of the form

f(x)= i A,,x~'" + R,,\{x),

m =

00

where Rn(x)—\^ co as ?i— > oo , and the series S) A.,nX~\^ does not
converge.

m =

We now consider what meaning, if any, can be attached to the ' sum '
of a non- convergent series. That is to say, given the numbers ao,
a\^, a\^, ..., we wish to formulate definite rules by which we can
obtain from them a

00 00

number 8 such that S = 'S, an if 2 a„ converges, and such that aS'
exists

TODO

when this series does not converge.

\Subsection{Borel's method of summation.} We have seen (§ 7-81) that

00 /"»

2 anz"" = e-\^(f) (tz) dt,

n=0 J

where (tz) = X " , the equation certainly being true inside; the
circle

00

of convergence of S a\^z'\^. If the integral exists at points z
outside this circle, we define the ' Borel sum ' of S a,i2" to mean
the integral.

* It has been shewn that when the coefficients in the expansion
satisfy certain inequaUties, there is only one analytic function with
that asymptotic expansion. See PJiil. Trans. 213, a, (1911), pp.
279-313.

t Borel, Le\^o)is stir les Series Diveryentes (1901), pp. 97-115.

% 
% 155
% 

Thus, whenever R(z)< I, the ' Bore] sum ' of the series S 2''\^ is

»=o

1 e-*e''dt = \{l-z)-\

J

If the ' Borel sum ' exists we say that the series is ' summable (B).'

\Subsection{Elder's* method of summation.} A method, practically due
to Euler, is suggested by the theorem of § 3*7l;

00 00

the ' sum ' of S a,i may be defined as lim S ctn\^'\^ when this limit
exists.

n=0 2:\^-1-0 n=0

Thus the ' sum ' of the series 1 — 1 + 1 — 1 + ... would be lim (1 — X
+ cc- — ...) = lim (1 + x)~\^ = i.

\Subsection{Cesdro's-f method of summation.}

Let Sn = «! + ao + . . . + «n; then if S = lim - (s, + Sj + • • • +
\^n) exists, we say that S Un is 'summable (C'l),' and that its sum
(CI) is S. It is necessary to establish the 'condition of
consistency;!:,' namely that S= 2 a„ when this series is convergent.

00 n

To obtain the required result, let TODO, then we have

m = l m = l

to prove that TODO.

Given e, we can choose n such that

so TODO

Then, if i\^ > n, we have

n+p

% a,

m=n-rl

< e for all values of p, and

TODO

Since TODO is a positive decreasing sequence, it follows from Abel's
inequality (§ 2'301) that

TODO

Therefore

TODO

* Instit. Cale. Diff. (1755). See Borel, loc. cit. Introduction, t
Bulletin des Sciences Math. (2), xiv. (1890), p. 114. + See the end of
§ 8-4.

%
% 156
%
Making v—\^x, we see that, if S be any one of the limit points (§
2'21) of S\^ , then

n I

S — % a„, \^ e. Therefore, since | s — 5,1 ! < e, we have

This inequality being true for every positive value of e we infer, as
in § 2"21, that S =\^s; that is to say S\^ has the unique limit s;
this is the theorem which had to be proved.

Example 1. Frame a definition of 'uuiforui .•summability (Cl) of a
series of variable terms.'

Example 2. TODO

\Subsubsection{TODO:Cesdrds general method of summation.}

A series TODO is said to be 'summable (Cr)' if TODO exists, where

It follows from § 8-43 example 2 that the 'condition of consistency'
is satisfied; in fact it can be proved* that if a series is summable
(C/) it is also summable \{Cr) when r>r'; the condition of consistency
is the particular case of this result when r = 0.

\Subsection{The method of summation of Rieszi.}

A more extended method of ' summing ' a series than the preceding is
by means of

hm 2 ( 1 - \^- cin ,

in which X,i is any real function of n which tends to infinity with n.
A series for which this limit exists is said to be 'summable \{Rr)
with sum-function X„.'

\Section{Hardy's convergence theorem.}

s summable ( a,i=0(l/?i),

Let S an he a series tuhick is summable \{G 1). Then if

w=l

the series TODO converges.

n=\

* Bromwich, Infinite Series, § 122. t Comptes Rendus, cxlix. (1910),
pp. 18-21.

X Proc. London Math. Sac. (2), viii. (1910), pp. 302-304. For the
proof here given, we are indebted to Mr Littlewood.

Let Sn = a I + a.\^ +. + cin; then since S Un is summable \{G 1), we
have

% 
% 157
% 

M = l

Si + So+ ... + Sn = n[s + (l)j,

where s is the sum ((71) of X «».

Let TODO and let

Sm-s = t,a, \{m = l, 2, ... n),

ti + to+ ... +tn = (Tn •

With this notation, it is sufficient to shew that, if j a„ | < Kn~\^,
where K is independent of n, and if On = n.o (1), then tn —> sis n —>
cc .

Suppose first that a\^, a\^, ... are real. Then, if tn does not tend
to zero, there is some positive number h such that there are an
unlimited number of the numbers t,i which satisfy either\^ (i) t\^ > h
or (ii) tn < —h. We shall shew that either of these hypotheses implies
a contradiction. Take the former*, and choose n so that tn > h.

Then, when r = 0, 1,2,

< K/n.

Now plot the points P,. whose coordinates are (r, tn+r) in a Cartesian
diagram. Since tn\^r+i-tn+r = an+r+i, the slope of the line PrPr+i is
less than = arc tan (K/n).

Therefore the points Pq, Pj, P.,, ... lie above the line y = h — xtan
6. Let Pk be the last uf the points P„, Pj, ... which lie on the left
of a;'= hcot 6, so that TODO.

Draw rectangles as shewn in the figure. The area of these rectangles
exceeds the area of the triangle bounded by y = h — x tan 6 and the
axes; that is to say

TODO

* The reader will see that the latter hypothesis involves a
contradiction by using arguments of a precisely similar character to
those which will be employed in dealing with the former hypothesis.

%
% 158
%

But I (Tn+k — 0"7i-i ! < I 0'7i+A; ! + I \^n-i \

= (n + k).o\{l) + \{n-l).o(l) = n.o\{l), since k \^ hnK~\^, and h, K
are independent of n.

Therefore, for a set of values of n tending to infinity,

\^h-K~\^n <n.o\{l), which is impossible since \^h'-K~\^ is )iot o (1)
as /i-> x .

This is the contradiction obtained on the hypothesis that lim tn> h >
0; therefore Urn tn \^ 0. Similarly, by taking the corresponding case
in which tn\^ — h, we arrive at the result lim tn \^ 0. Therefore
since lim tn > lim tn,

we have lim t\^ = lim tn = 0,

and so tn — > 0.

That is to say Sn —> s, and so 2 a,i is convergent and its sum is s.

If an be complex, we consider R (a„) and / (a\^) separately, and find

TODO

that S R\{an) and S /(c/„) converge by the theorem just proved, and so
TODO

The reader will see in Chapter ix that this result is of great
importance in the modern theory of Fourier series.

Corollary. If TODO be a function of TODO such that TODO

throughout a domain of values of TODO, and if TODO, where K is
independent of |,

2 a„ (\^) converges uniformly throughoxit the domain. n=i

For, retaining the notation of the preceding section, if \^n(\^) does
not tend to zero uniformly, we can find a positive number h
independent of n and | such that an infinite sequence of values of n
can be found for which t\^ (\^») > A or i„ (\^„) <-h for some point
TODO of the domain*; the value of \^„ depends on the value of n under
consideration.

We then find, as in the original theorem,

TODO

for a set of values of n tending to infinity. The contradiction
implied in the inequality shews that h does not exist, and so
\^rt(|)-9-0 uniformly.

* It is assumed that «„ (\^) is real; the extension to complex
variables cau be made as in the former theorem. If no such number h
existed, <„ (|) would tend to zero uniformly.

t It is essential to observe that the constants involved iu the
inequality do not depend on |,j. For if, say, K depended on \^„, K~\^
would really be a function of n and might be o (1) qua function of n,
and the inequality would not imply a contradiction.

%
% 159
%

REFERENCES.

H. PoiNCAR\^, Acta Mathematica, viii. (1886), pp. 295-344.

E. BoREL, Lecons sur les Series Divergentes (Paris, 1901).

T. J. Pa. Bromwich, Theory of Infinite Series (1908), Ch. xi.

E. W. Barnes, Phil. Trans, of the Royal Society, 206, a (1906), pp.
249-297.

G. H. Hardy and J. E. Littlewood, Proc. London Math. Soc. (2), xi.
(1913), pp. 1-16*.

G. N. Watson, Phil. Trans, of the Royal Society, 213, a (1911), pp.
279-313.

S. Chapman +, Proc. London Math. Soc. (2), ix. (1911), pp. 369-409.

Miscellaneous Examples.

/" e~\^ 1 2 ! 4 !

:; :idt~ - ~ + -\^ — ...

when X is real and positive.

2. Discuss the representation of the function

f(x)=\{'' \^<\^\{t)e\^-dt

(where x is supposed real and positive, and is a function subject to
certain general con- ditions) by means of the series

TODO

Shew that in certain cases (e.g. (\^(;) = e'") the series is
absolutely convergent, and represents TODO for large positive values
of $x$ but that in certain other cases the series is the asymptotic
expansion of $f(x)$.

3. Shew that

for large positive values of z.

TODO

(Legendre, Exerdces de Calc. Int. (1811), p. 340.) 4. Shew that if,
when x>0,

TODO

Shew also that/(.r) can be expanded into an absolutely convergent
series of the form

TODO \addexamplecitation{\Schlomilch.}

•'\^ ' A.=i(\^+l)(.r-i-2)...(\^ + \^-\}

5. Shew that if the series 1+0 + 0-1+04-1 + + 0-1 + ..., in which two
zeros precede each —1 and one zero precedes each +1, be 'summed' by
Ceskro's method, its sum is f. (Euler, Borel.)

6. Shew that the series 1-21 + 4! — ... cannot be summed by Borel's
method, but the series l+0-2! + + 4! + ... can be so summed.

* This paper contains many references to recent developments of the
subject. t A bibliography of the literature of summable series will be
found on p. 372 of this memoir. % First section done
%
% 160
%
\chapter{Fourier Series and Trigonometrical Series}
\Section{9}{1}{Definition of Fourier series}\footnote{Throughout
  this chapter (except in \hardsubsectionref{9}{1}{1}) it is supposed that all
  the numbers involved are real.}.
Series of the type
\begin{align*}
  \half a_{0}
  + (a_{1} + \cos x + b_{1} \sin x)
  + (a_{2} + \cos 2x + b_{2} \sin 2x)
  + \cdots
  \\
  &
  \hfill
  \half a_{0}
  +
  \sum_{n=1}^{\infty} ( a_{n} \cos n x + b_{n} \sin n x),
\end{align*}
where $a_{n}$, $b_{n}$ are independent of $x$, are of great importance in many
investigations. They are called \emph{trigonometrical series}.

If there is a function $f(t)$ such that
$\int_{-\pi}^{\pi} f(t) \dmeasure t$ exists as a Riemann
integral or as an improper integral which converges absolutely, and such that
$$
\pi a_{n} = \int_{-\pi}^{\pi} f(t) \cos nt \dmeasure t,
\quad
\pi b_{n} = \int_{-\pi}^{\pi} f(t) \sin nt \dmeasure t,
$$
then the trigonometrical series is called a \emph{Fourier series}.

%\begin{smalltext}
Trigonometrical series first appeared in analysis in connexion with
the investigations of Daniel Bernoulli on vibrating
strings\index{Strings, vibrations of}\index{Vibrations of!strings};
d'Alembert had previously solved the equation of
motion
$ \ddot{y} = a^{2} \frac{\dd^{2} y}{\dd x^{2}}$
in the form
$y = \half \thebrace{f(x+at) + f(x-at)}$, where $y=f(x)$ is
the initial shape of the string starting from rest;
and Bernoulli shewed that a formal solution is
$$
y
=
\sum_{n=1}^{\infty}
b_{n}
\sin \frac{n \pi x}{l}
\cos \frac{n \pi a t}{l},
$$
the fixed ends of the string being $(0,0)$ and $(l,0)$; and he asserted
that this was the most general solution of the problem. This appeared
to d'Alembert and Euler to be impossible, since such a series, having
period $2l$, could not possibly represent such a function
as\footnote{This function gives a simple form to the initial shape of the string.}
$c x (l-x)$ when $t = 0$.
A controversy arose between these mathematicians, of which
an account is given in Hobson's \emph{Functions of a Real Variable}.

Fourier, in his \emph{TODO Theorie de la Chaleur}, investigated a number of
trigonometrical series and shewed that, in a large number of
particular cases, a Fourier series \emph{actually converged to the sum $f(x)$}.
Poisson attempted a general proof of this theorem. TODO Journal de VEcole
poly technique, xil. (1823), pp. 404-509. Two proofs were given by
Cauchy, TODO Men. de VAcad. R. des Sci. vi. (1823, publi-hed 1826), pp.
603-612 Oeuvres, (1), n. pp. 12-19) and Exercices de Math. 11. (1827),
pp. 341-376 (Oeuvres, (2), vil. pp. 393-430); these proofs, which are
based on the theory of contour integration, are concerned with rather
particular classes of functions and one is invalid. The second proof
has been investigated by Harnack, TODO Math. Ann. xxxii. (1888), pp.
175-202.

%\end{smalltext}
%
% 161
%

In 1829, Dirichlet gave the first rigorous proof\footnote{TODO Journal J iir Math. iv. (1829), pp. 157-169.}
that, for a general class of functions, the Fourier series, defined as above, does
converge to the sum $f(x)$. A modification of this proof was given later
by Bonnet\footnote{TODO Meiuoires des Savants etramjers of the Belgian Academy, xxiii.
(1848-1850). Bonnet employs the second mean value theorem directly,
while Dirichlet's original proof makes use of arguments precisely
similar to those by which that theorem is proved. See \hardsubsectionref{9}{4}{3}.}.

The result of Dirichlet\index{Statement of Fourier's theorem, \Dirichlet's}
is that\footnote{The conditions postulated for $f(t)$
  are known as \emph{Dirichlet's conditions}; as will be seen in TODO §§ 9-2, 9 42, they are unnecessarily
  stringent.}
if $f(t)$ is defined and bounded in the
range $(-\pi, \pi)$ and if $f(t)$ has only a finite number of maxima and
minima and a finite number of discontinuities in this range and,
further, if $f(t)$ is defined by the equation
$$
f(t + 2 \pi) = f(t)
$$
outside the range $(-\pi, \pi)$, then, provided that
$$
\pi a_{n}
=
\int_{-\pi}^{\pi} f(t) \cos nt \dmeasure t,
\quad
\pi b_{n}
=
\int_{-\pi}^{\pi} f(t) \sin nt \dmeasure t,
$$
the series $TODO$ converges to the sum /( +
0)+/( - 0) .

Later, Riemann and Cantor developed the theory of trigonometrical
series generally, while still more recently Hurwitz, \Fejer\ and others
have investigated properties of Fourier series when the series does
not necessarily converge. Thus \Fejer has proved the remarkable
theorem that a Fourier series (even if not convergent) is 'summable
$(C1)$' at all points at which $f(x \pm 0)$ exist, and its sum $(C1)$ is
$\half \thebrace{ f(x+0) + f(x-0) }$,
provided that $\int_{-\pi}^{\pi} f(t) \dmeasure t$ is an absolutely convergent integral.
One of the investigations of the convergence of Fourier series which we shall give later
(\hardsectionref{9}{4}{2}) is based on this result.

For a fuller account of investigations subsequent to Riemann, the
reader is referred to Hobson's \emph{Functions of a Real Variable}, and to
TODO de la Vallee Poussin's Cours dWnalyse Infinitesimale.

\Subsection{9}{1}{1}{Nature of the region within which a
  trigonometrical series converges.}

Consider the series
$$
\half a_{0}
+
\sum_{n=1}^{\infty}
\theparen{
  a_{n} \cos nz
  +
  b_{n} \sin nz
},
$$
where $z$ may be complex. If we write
$e^{iz} = \zeta$, the series becomes
$$
\half a_{0}
+
\sum_{n=1}^{\infty}
\thebrace{
  \half (a_{n} - i b_{n}) \zeta^{n}
  +
  \half (a_{n} + i b_{n}) \zeta^{-n}
}
$$
This Laurent series will converge, if it converges at all, in a region
in which $a \leq \absval{\zeta} \leq b$, where $a,b$ are positive
constants.

But, if $z = x + iy$, $\absval{\zeta} = e^{-y}$, and so we get, as the
region of convergence of the trigonometrical series, the strip in the
$z$ plane defined by the inequality
$$
\log a \leq -y \leq \log b.
$$

The case which is of the greatest importance in practice is that in which
$a = b = 1$, and the strip consists of a single line, namely the real axis.

TODO Example 1
Let
$$
f(z)
=
\sin z
- \half \sin 2z
+ \frac{1}{3} \sin 3z
- \frac{1}{4} \sin 4z
+ \ldots,
$$
where $z = x + iy$.
%
% 161
%

Writing this in the form
$$
f(z)
=
-
\half
i
\theparen{
  e^{iz}
  - \half e^{2iz}
  + \frac{1}{3} e^{3iz}
  -
  \cdots
}
+
\half
i
\theparen{
  e^{-iz}
  - \half e^{-2iz}
  + \frac{1}{3} e^{-3iz}
  -
  \cdots
}
$$
we notice that the first series converges\footnote{The series
  \emph{do converge} if $y=0$, see \hardsubsectionref{2}{3}{1} example TODO 2}
only if
$y \geq 0$, and the second only if $y \leq 0$.

Writing $x$ in place of $z$ ($x$ being real), we see that by Abel's
theorem (\hardsubsectionref{3}{7}{1}),
\begin{align*}
  f(x)
  &=
  \lim_{r \rightarrow 1} \theparen{
    r \sin x
    - \half r^{2} \sin 2x
    + \frac{1}{3} r^{3} \sin 3x
    - \cdots
  }
  \\
  &=
  \lim_{r \rightarrow 1} \thebrace{
    - \half \theparen{
      r e^{ix}
      - \half r^{2} e^{2ix}
      + \frac{1}{3} r^{3} e^{3ix}
      - \cdots
    }
    +
    \half i \theparen{
      r e^{-ix}
      - \half r^{2} e^{-2ix}
      + \frac{1}{3} r^{3} e^{-3ix}
      - \cdots
    }
  }
\end{align*}

This is the limit of one of the values of
$$
- \half i \log (1 + r e^{ix})
+ \half i \log (1 + r e^{-ix}),
$$
and as $r \rightarrow 1$ (if $-\pi < x < \pi$), this tends to
$\half x + k\pi$, where $k$ is some integer.

Now $\sum_{n=1}^{\infty} \frac{(-)^{n-1} \sin nx}{n}$ converges uniformly
(\hardsubsectionref{3}{3}{5} example TODO 1) and is therefore continuous in
the range $-\pi + \delta \leq x \leq \pi - \delta$, where
$\delta$ is any positive constant.

Since $\half x$ is continuous, $k$ has the same value wherever $x$ lies in the
range; and putting $x=0$, we see that $k=0$.

\emph{Therefore, when $-\pi < x < \pi$,
  $$
  f(x) = \half x.
  $$
}

But, when $\pi < x < 3\pi$,
$$
f(x)
=
f(x - 2\pi)
=
\half (x - 2\pi)
=
\half x - \pi,
$$
and generally, if $(2n - 1) \pi < x < (2n + 1) \pi$,
$$
f(x) = \half x - n \pi.
$$

We have thus arrived at an example in which $f(x)$ is
not represented by a single analytical expression.

It must be observed that this phenomenon can only occur when the
strip in which the Fourier series converges is a single line.
For if the strip is not of zero breadth, the associated Laurent
series converges in an annulus of non-zero breadth and represents an
analytic function of $\zeta$ in that annulus; and, since
$\zeta$ is an analytic function of $z$, the Fourier series
represents an analytic function of $z$; such a series is given by
$$
r \sin x
- \half r^{2} \sin 2x
+ \frac{1}{3} r^{3} \sin 3x
- \cdots,
$$
where $0 < r < 1$; its sum is
$\arctan \frac{r \sin x}{1 + r \cos x}$, the $\arctan$ always
representing an angle between $\pm \half \pi$.

Example TODO
When $-\pi \leq x \leq \pi$,
$$
\sum_{n=1}^{\infty}
\frac{(-)^{n-1} \cos nx}{n^{2}}
=
\frac{1}{12} \pi^{2}
-
\frac{1}{4} x^{2}.
$$

The series converges only when $x$ is real; by
\hardsubsectionref{3}{3}{4} the convergence is then
absolute and uniform.

Since
$$
\half x
=
\sin x
- \half \sin 2x
+ \frac{1}{3} \sin 3x
- \cdots
\quad
(-\pi + \delta \leq x \pi - \delta,
\delta > 0),
$$
and this series converges uniformly, we may integrate
term-by-term from $0$ to $x$ (\hardsectionref{4}{7}),
and consequently
$$
\frac{1}{4} x^{2}
=
\sum_{n=1}^{\infty}
\frac{(-)^{n-1} (1 - \cos nx)}{n^{2}}
\quad
(-\pi + \delta \leq x \leq \pi - \delta).
$$
%
% 163
%

That is to say, when $-\pi + \delta \leq x \leq \pi - \delta$,
$$
C - \frac{1}{4} x^{2}
=
\sum_{n=1}^{\infty} \frac{(-)^{n-1} \cos nx}{n^{2}},
$$
where $C$ is a constant, at present undetermined.

But since the series on the right converges uniformly throughout the
range $-\pi \leq x \leq \pi$, its sum is a continuous function of $x$ in this
extended range; and so, proceeding to the limit when
$x \rightarrow \pm \pi$, we see
that the last equation is still true when $x = \pm \pi$.

To determine $C$, integrate each side of the equation \hardsectionref{4}{7} between
the limits $-\pi, \pi$; and we get
$$
2 \pi C - \frac{1}{6} \pi^{3} = 0.
$$

Consequently
$$
\frac{1}{12} \pi^{2} - \frac{1}{4} x^{2}
=
\sum_{n=1}^{\infty}
\frac{ (-)^{n-1} \cos nx }{ n^{2} }
\quad
(-\pi \leq x \leq \pi).
$$

Example TODO.
By writing $\pi - 2x$ for $x$ in example TODO2, shew that
$$
\sum_{n=1}^{\infty} \frac{\sin^{2} nx}{n^{2}}
=
\begin{cases}
  \half x (\pi - x)                         & 0 \leq x \leq \pi,   \\
  \half \thebrace{ \pi \absval{x} - x^{2}} & -\pi \leq x \leq \pi.
\end{cases}
$$

\Subsection{9}{1}{2}{Values of the coefficients in terms of the sum of a
trigonometrical series.}
Let the trigonometrical series
$
\half c_{0}
+
\sum_{n=1}^{\infty} (c_{n} \cos nx + d_{n} \sin nx)
$
be uniformly convergent in the range $(-\pi, \pi)$ and let its sum be $f(x)$.
Using the obvious results
\begin{align*}
  \int_{-\pi}^{\pi} \cos mx \cos nx \dmeasure x
  =&
  \begin{cases}
    0 & m \neq n \\
    \pi & m = n \neq 0,
  \end{cases}
  \\
  \int_{-\pi}^{\pi} \sin mx \sin nx \dmeasure x
  =&
  \begin{cases}
    0 & m \neq n \\
    \pi & m = n \neq 0,
  \end{cases}
  \\
  \int_{-\pi}^{\pi} \dmeasure x
  =& 2\pi,
\end{align*}
we find, on multiplying the equation
$
\half c_{0}
+
\sum_{n=1}^{\infty}
(c_{n} \cos nx + d_{n} \sin nx)
= f(x)
$
by\footnote{Multiplying by these factors does not destroy the uniformity of the
  convergence.}
$\cos nx$ or by $\sin nx$ and integrating
term-by-term\footnote{These were given by TODO Euler (with limits $0$ and $2\pi$),
  Nova Acta Acad. Petrop. xi. (1793).}
(\hardsectionref{4}{7}),
$$
\pi c_{n} = \int_{-\pi}^{\pi} f(x) \cos nx \dmeasure x,
\quad
\pi d_{n} = \int_{-\pi}^{\pi} f(x) \sin nx \dmeasure x.
$$

TODO Corollary. A trigonometrical series uniformly convergent in the range
$(-\pi, \pi)$ is a Fourier series.

TODO Note. Lebesgue has given a proof (TODO Series trigonometriques, p. 124) of a
theorem communicated to him by Fatou that the trigonometrical series
$\sum_{n=2}^{\infty} \sin nx / \log n$, which converges for all real values of $x$
(\hardsubsectionref{2}{3}{1} example TODO), is \emph{not} a Fourier series.

\Section{9}{2}{On Dirichlet's conditions and Fourier's theorem.
\index{Statement of Fourier's theorem, \Dirichlet's}}
A theorem, of the type described in \hardsectionref{9}{1}, concerning the
expansibility of a function of a real variable into a trigonometrical
series is usually described
%
% 164
%
as \emph{Fourier's theorem}. On account of the length and difficulty of a
formal proof of the theorem (even when the function to be expanded is
subjected to unnecessarily stringent conditions), we defer the proof
until TODO §§ \hardsubsectionref{9}{4}{2}, \hardsubsectionref{9}{4}{3}.
It is, however, convenient to state here certain
\emph{sufficient} conditions under which a function can be expanded into a
trigonometrical series.

\emph{Let $f(t)$ he defined arbitrarily when $-\pi \leq t \leq \pi$
  and defined\footnote{This definition frequently results in $f(t)$ not being
    expressible by a single analytical expression for all real values of $t$.
    Cf.\hardsubsectionref{9}{1}{1} example TODO:1.}
  for all other real values of $t$ by means of the equation
  $$
  f(t + 2\pi) = f(t),
  $$
  so that $f(t)$ is a periodic function with period $2\pi$.
}
\emph{
  Let $f(t)$ be such that
  $\int_{-\pi}^{\pi} f(t) \dmeasure t$ exists; and if this is an improper
  integral, let it be absolutely convergent.
}

\emph{
  Let $a_{n}, b_{n}$ be defined by the
  equations\footnote{The numbers $a_{n}, b_{n}$ are called
    \emph{the Fourier constants\index{Fourier constants}} of
    $f(t)$, and the symbols $a_{n}, b_{n}$ will be used in this sense throughout
    §§ TODO \hardsectionref{9}{2}--\hardsectionref{9}{5}.
    It may be shewn that the convergence and absolute convergence of the
    integrals defining the Fourier constants are consequences of the
    convergence and absolute convergence of
    $\int_{-\pi}^{\pi} f(t) \dmeasure t$.
    Cf. §§ TODO \hardsubsectionref{2}{3}{2}, \hardsectionref{4}{5}.}
  $$
  \pi a_{n} = \int_{-\pi}^{\pi} f(t) \cos nt \dmeasure t,
  \quad
  \pi b_{n} = \int_{-\pi}^{\pi} f(t) \sin nt \dmeasure t
  \quad
  (n=0,1,2,\ldots).
  $$
}

\emph{
  Then, if $x$ be an interior point of any interval $(a, b)$ in which
  $f(t)$ has limited total fluctuation, the series
  $$
  \half a_{0}
  +
  \sum_{n=1}^{\infty} (a_{n} \cos nx + b_{n} \sin nx)
  $$
  is convergent, and its sum\footnote{The limits $f(x \pm 0)$ exist,
    by \hardsubsectionref{3}{6}{4} example TODO:3.}
  is $\half \thebrace{ f(x+0) + f(x-0) }$.
  If $f(t)$ is continuous at $t=x$, this sum reduces to $f(x)$.
}

This theorem will be assumed in
TODO \hardsubsectionref{9}{2}{1}--\hardsubsectionref{9}{3}{2};
these sections deal with theorems concerning Fourier series which
are of some importance
in practical applications. It should be stated here that every
function which Applied Mathematicians need to expand into Fourier
series satisfies the conditions just imposed on $f(t)$, so that the
analysis given later in this chapter establishes the validity of all
the expansions into Fourier series which are required in physical
investigations.

The reader will observe that in the theorem just stated,
$f(t)$ is subject to less stringent conditions than those contemplated by
Dirichlet, and this decrease of stringency is of
considerable practical importance. Thus, so simple a series as
$\sum_{n=1}^{\infty} (-)^{n-1} (\cos nx) / n$
is the
expansion of the function\footnote{Cf. example TODO:6 at the end of the chapter (p. TODO:190).}
$\log \absval{2 \cos \half x}$; and this function
does not satisfy Dirichlet's condition of boundedness at $\pm \pi$.

It is convenient to describe the series
$\half a_{0} + \sum_{n=1}^{\infty} (a_{n} \cos nx + b_{n} \sin nx)$
as \emph{the Fourier series associated with $f(t)$}. This description must,
however, be
%
% 165
%
taken as implying nothing concerning the convergence of the series in
question.

\Subsection{9}{2}{1}{The representation of a function by Fourier series for ranges
other than $(-\pi,\pi)$.}

Consider a function $f(x)$ with an (absolutely) convergent integral,
and with limited total fluctuation in the range $a \leq x \leq b$.

Write
$x = \half (a + b) - \half (a-b) \pi^{-1} x',
\quad
f(x) = F(x')$.

Then it is known (\hardsectionref{9}{2}) that
$$
\half [F(x'+0) + F(x'-0)]
=
\half a_{0} + \sum_{n=1}^{\infty} (a_{n} \cos nx' + b_{n} \sin nx'),
$$
and so
\begin{align*}
  & \half \thebrace{ f(x+0) + f(x-0)}
  \\
  &
  \hfill
  =
  \half a_{0}
  +
  \sum_{n=1}^{\infty}
  \thebrace{
    a_{n} \cos \frac{n \pi (2x-a-b)}{b-a}
    +
    b_{n} \sin \frac{n \pi (2x-a-b)}{b-a}
  },
\end{align*}
where by an obvious transformation
\begin{align*}
  \half (b-a) a_{n} =& \int_{a}^{b} f(x) \cos \frac{n \pi (2x-a-b)}{b-a} \dmeasure x,
  \\
  \half (b-a) b_{n} =& \int_{a}^{b} f(x) \sin \frac{n \pi (2x-a-b)}{b-a} \dmeasure x
  .
\end{align*}
\Subsection{9}{2}{2}{The cosine series and the sine series.}
Let $f(x)$ be defined in the range $(0,l)$ and let it have an
(absolutely) convergent integral and also let it have limited
total fluctuation in that range.
\emph{Define} $f(x)$ in the range
$\wandwtypo{(0,-l)}{(-l,0)}$
by the equation
$$
f(-x) = f(x).\index{Even functions}
$$

Then
$$
\half \thebrace{ f(x+0) + f(x-0) }
=
\half a_{0}
+
\sum_{n=1}^{\infty} \thebrace{
  a_{n} \cos \frac{n \pi x}{l}
  +
  b_{n} \sin \frac{n \pi x}{l}
},
$$
where, by \hardsubsectionref{9}{2}{1},
\begin{align*}
  l a_{n}
  =&
  \int_{-l}^{l} f(t) \cos \frac{n \pi t}{l} \dmeasure t
  =
  2 \int_{0}^{l} f(t) \cos \frac{n \pi t}{l} \dmeasure t,
  \\
  l b_{n}
  =&
  \int_{-l}^{l} f(t) \sin \frac{n \pi t}{l} \dmeasure t
  = 0,
\end{align*}
so that when $-l \leq x \leq l$,
$$
\half \thebrace{ f(x+0) + f(x-0) }
=
\half a_{0} + \sum_{n=1}^{\infty} a_{n} \cos \frac{n \pi x}{l};
$$
this is called the \emph{cosine series}.

If, however, we define $f(x)$ in the range $(0,-l)$ by the equation
$$
f(-x) = -f(\wandwtypo{-}{}x),\index{Odd functions}
$$
%
% 166
%
we get, when $-l \leq x \leq l$,
$$
\half \thebrace{ f(x+0) + f(x-0) }
=
\sum_{n=1}^{\infty} b_{n} \sin \frac{n \pi x}{l},
$$
where
$$
l b_{n}
=
2 \int_{0}^{l} f(t) \sin \frac{n \pi t}{l} \dmeasure t;
$$
this is called the \emph{sine series}.

Thus the series
$$
\half a_{0}
+
\sum_{n=1}^{\infty} a_{n} \cos \frac{n \pi x}{l},
\quad
\sum_{n=1}^{\infty} b_{n} \sin \frac{n \pi x}{l},
$$
where
$$
\half l a_{n}
=
\int_{0}^{l} f(t) \cos \frac{n \pi t}{l} \dmeasure t,
\quad
\half l b_{n}
=
\int_{0}^{l} f(t) \sin \frac{n \pi t}{l} \dmeasure t,
$$
\emph{have the same sum when $0 \leq x \leq l$;}
but their sums are numerically
equal and opposite in sign when $0 \geq x \geq -l$.

%\begin{smalltext}
The cosine series was given by Clairaut, TODO Hist, de I'Acad. R. des Sci.
1754 [published, 1759], in a memoir dated July 9, 1757; the sine
series was obtained between 1762 and 1765 by Lagrange, Oeuvres, l. p.
553.
%\end{smalltext}

TODO Example 1. Expand $\half (\pi - x) \sin x$ in a cosine series in the range
$0 \leq x \leq \pi$.
[We have, by the formula just obtained,
$$
\half (\pi - x) \sin x
=
\half a_{0}
+ \sum_{n=1}^{\infty} a_{n} \cos nx,
$$
where
$$
\half \pi a_{n}
=
\int_{0}^{\pi} \half (\pi - x) \sin x \cos nx \dmeasure x.
$$

But, integrating by parts, if $n \neq 1$,
\begin{align*}
  &
  \int_{0}^{\pi} 2 (\pi - x) \sin x \cos nx \dmeasure x
  \\
  & \quad
  \int_{0}^{\pi} (\pi - x) \thebrace{
    \sin (n+1) x - \sin (n-1) x
  } \dmeasure x
  \\
  & \quad
  \thebracket{
    (x - \pi)
    \thebrace{
      \frac{ \cos (n+1) x }{n+1}
      -
      \frac{ \cos (n-1) x }{n-1}
    }
  }_{0}^{\pi}
  -
  \int_{0}^{\pi}
  \thebrace{
    \frac{ \cos (n+1) x }{n+1}
    -
    \frac{ \cos (n-1) x }{n-1}
  }
  \dmeasure x
  \\
  & \quad
  \pi
  \theparen{
    \frac{1}{n+1}
    -
    \frac{1}{n-1}
  }
  =
  \frac{-2\pi}{(n+1)(n-1)}
\end{align*}
Whereas if $n=1$, we get
$\int_{0}^{\pi} 2 (\pi - x) \sin x \cos x \dmeasure x = \half \pi$.

Therefore the required series is
$$
\half
+ \frac{1}{4} \cos x
- \frac{1}{1 \cdot 3} \cos 2x
- \frac{1}{2 \cdot 4} \cos 3x
- \frac{1}{3 \cdot 5} \cos 4x
- \cdots.
$$

It will be observed that it is only for values of $x$ between
$0$ and $\pi$ that the sum of this series is proved to be
$\half (\pi - x) \sin x$; thus for
instance when $x$ has a value between $0$ and $-\pi$,
the sum of the series is not
$\half (\pi - x) \sin x$, but $-\half (\pi + x) \sin x$; when $x$ has a value
between $\pi$ and $2 \pi$, the sum of the series happens to be again
$\half (\pi - x) \sin x$, but this is a mere coincidence arising from the special
function considered, and does not follow from the general theorem.]

TODO Example 2. Expand $\frac{1}{8} \pi x (\pi - x)$ in a sine series,
valid when $0 \leq x \leq \pi$.

[The series is $\sin x + \frac{\sin 3x}{3^{3}} + \frac{\sin 5x}{5^{3}} + \cdots.$]
%
% 167
%

TOOD Example 3. Shew that, when $0 \leq x \leq \pi$,
$$
\frac{1}{96}
\pi
(\pi - 2x)
(\pi^{2} + 2 \pi x - 2 x^{2})
=
\cos x
+ \frac{\cos 3x}{3^{4}}
+ \frac{\cos 5x}{5^{4}}
+ \cdots.
$$

[Denoting the left-hand side by $f(x)$, we have, on integrating by
parts and observing that $f'(0) = f'(\pi) = 0$,
\begin{align*}
  \int_{0}^{\pi} f(x) \cos nx \dmeasure x
  =&
  \frac{1}{n} \thebracket{f(x) \sin nx}_{0}^{\pi}
  -
  \frac{1}{n} \int_{0}^{\pi} f'(x) \sin nx \dmeasure x
  \\
  =&
  \frac{1}{n^{2}} \thebracket{f'(x) \cos nx}_{0}^{\pi}
  -
  \frac{1}{n^{2}} \int_{0}^{\pi} f''(x) \cos nx \dmeasure x
  \\
  =&
  -\frac{1}{n^{3}} \thebracket{f''(x) \sin nx}_{0}^{\pi}
  +
  \frac{1}{n^{3}} \int_{0}^{\pi} f'''(x) \sin nx \dmeasure x
  =&
  -\frac{1}{n^{4}} \thebracket{f'''(x) \cos nx}_{0}^{\pi}
  =
  \frac{\pi}{4 n^{4}} (1 - \cos n \pi).]
\end{align*}
TODO Example 4. Shew that for values of $x$ between $0$ and $\pi$,
$e^{s x}$ can be expanded in the cosine series
$$
\frac{2 s}{\pi}
\theparen{e^{s \pi} - 1}
\theparen{
  \frac{1}{2 s^{2}}
  + \frac{\cos 2x}{s^{2} + 4}
  + \frac{\cos 4x}{s^{2} + 16}
  + \cdots
}
-
\frac{2 s}{\pi}
\theparen{e^{s \pi} - 1}
\theparen{
  \frac{\cos x}{s^{2} + 1}
  + \frac{\cos 3x}{s^{2} + 9}
  + \cdots
},
$$
and draw graphs of the function $e^{s x}$ and of the sum of the series.

TODO Example 5. Shew that for values of $x$ between $0$ and $\pi$,
the function $\frac{1}{8} \pi (\pi - 2x)$ can
be expanded in the cosine series
$$
\cos x
+ \frac{\cos 3x}{3^{2}}
+ \frac{\cos 5x}{5^{2}}
+ \cdots,
$$
and draw graphs of the function $\frac{1}{8} \pi (\pi - 2x)$ and of the sum of the
series.

\Section{9}{3}{The nature of the coefficients in a Fourier series.}\footnote{TODO The analysis of this section and of \hardsubsectionref{9}{3}{1} is contained in Stokes'
great memoir, Camb. Phil. Tratis. VIII. (1849), pp. 538-583 [Math.
Papers, i. pp. 236-313].}

Suppose that (as in the numerical examples which have been discussed)
the interval $(-\pi, \pi)$ can bo divided into a finite number of ranges
$(-\pi, k_{1}), (k_{1}, k_{2}), \ldots, (k_{n}, \pi)$
such that throughout each range $f(x)$
and all its differential coefficients are continuous with limited
total fluctuation and that they have limits on the right and on the
left (\hardsectionref{3}{2}) at the end points of these ranges.

Then
$$
\pi a_{m}
=
\int_{-\pi}^{k_{1}} f(t) \cos mt \dmeasure t
+ \int_{k_{1}}^{k_{2}} f(t) \cos mt \dmeasure t
+ \cdots
+ \int_{k_{n}}^{\pi} f(t) \cos mt \dmeasure t.
$$

Integrating by parts we get
\begin{align*}
  &
  \pi a_{m}
  =
  \thebracket{
    m^{-1} f(t) \sin mt
  }_{-\pi}^{k_{1}}
  +
  \thebracket{
    m^{-1} f(t) \sin mt
  }_{k_{1}}^{k_{2}}
  +
  \cdots
  +
  \thebracket{
    m^{-1} f(t) \sin mt
  }_{k_{n}}^{\pi}
  \\
  & \hfill
  - m^{-1} \int_{-\pi}^{k_{1}} f'(t) \sin mt \dmeasure t
  - m^{-1} \int_{k_{1}}^{k_{2}} f'(t) \sin mt \dmeasure t
  \cdots
  - m^{-1} \int_{k_{n}}^{\pi} f'(t) \sin mt \dmeasure t,
\end{align*}
so that
$$
a_{m} = \frac{A_{m}}{m} - \frac{b_{m}'}{m},
$$
%
% 168
%
where
$$
\pi A_{m}
=
\sum_{r=1}^{n}
\sin m k_{r}
\thebrace{
  f(k_{r} - 0) - f(k_{r} + 0),
}
$$
and $b_{m}'$ is a Fourier constant of $f'(x)$.

Similarly
$$
b_{m} = \frac{B_{m}}{m} + \frac{a_{m}'}{m},
$$
where
$$
\pi B_{m}
=
-
\sum_{r=1}^{n}
\cos m k_{r}
\thebrace{
  f(k_{r} - 0)
  -
  f(k_{r} + 0)
}
-
\cos m \pi
\thebrace{
  f(\pi - 0)
  -
  f(-\pi + 0)
}
,
$$
and $a_{m}'$ is a Fourier constant of $f'(x)$.

Similarly, we get
$$
a_{m}' = \frac{A_{m}'}{m} - \frac{b_{m}''}{m},
\quad
b_{m}' = \frac{B_{m}'}{m} - \frac{a_{m}''}{m},
$$
where $a_{m}'', b_{m}''$ are the Fourier constants of
$f''(x)$ and
\begin{align*}
  \pi A_{m}'
  =&
  \sum_{r=1}^{n} \sin m k_{r} \thebrace{
    f'(k_{r}-0) - f'(k_{r}+0)
  },
  \\
  \pi B_{m}'
  =&
  - \sum_{r=1}^{n} \cos m k_{r} \thebrace{
    f'(k_{r}-0) - f'(k_{r}+0)
  }
  \\
  \hfill
  - \cos m \pi \thebrace{
    f'(\pi - 0) - f'(-\pi + 0)
  }.
\end{align*}

Therefore
$$
a_{m} =
\frac{A_{m}}{m}
- \frac{B_{m}'}{m^{2}}
- \frac{a_{m}''}{m^{2}},
\quad
b_{m} =
\frac{B_{m}}{m}
+ \frac{A_{m}'}{m^{2}}
- \frac{b_{m}''}{m^{2}},
$$

Now as $m \rightarrow \infty$, we see that
$$
A_{m}' = \bigo(1),
\quad
B_{m}' = \bigo(1),
$$
and, since the integrands involved in $a_{m}''$ and $b_{m}''$
are bounded, it is evident that
$$
a_{m}'' = \bigo(1),
\quad
b_{m}'' = \bigo(1).
$$

Hence if $A_{m}=0, B_{m}=0$, the Fourier series for $f(x)$ converges
absolutely and uniformly, by \hardsubsectionref{3}{3}{4}.

The necessary and sufficient conditions that
$A_{m} = B_{m} = 0$ for all values of $m$ are that
$$
f(k_{r} - 0) = f(k_{r} + 0),
\quad
f(\pi - 0) = f(-\pi + 0)
$$
that is to say that\footnote{Of course $f(x)$ is also subject to the conditions stated at the
beginning of the section.} $f(x)$ should be continuous for \emph{all} values of $x$.

\Subsection{9}{3}{1}{Differentiation of Fourier series.}
The result of differentiating
$$
\half a_{0}
+ \sum_{m=1}^{\infty} (a_{m} \cos mx + b_{m} \sin mx)
$$
term by term is
$$
\sum_{m=1}^{\infty} \thebrace{
  m b_{m} \cos mx
  -
  m a_{m} \sin mx
}.
$$
%
% 169
%

With the notation of \hardsectionref{9}{3}, this is the same as
$$
\half a_{0}'
+
\sum_{m=1}^{\infty} ( a_{m}' \cos mx + b_{m}' \sin mx),
$$
provided that $A_{m} = B_{m} = 0$ and
$\int_{-\pi}^{\pi} f'(x) \dmeasure x = 0$;
these conditions are satisfied if $f(x)$ is continuous for all values of
$x$.

Consequently sufficient conditions for the legitimacy of
differentiating a Fourier series term by term are that $f(x)$ should be
continuous for \emph{all} values of $x$ and $f'(x)$ should have only a finite
number of points of discontinuity in the range $(-\pi, \pi)$, both
functions having limited total fluctuation throughout the range.

\Subsection{9}{3}{2}{Determination of points of discontinuity.}

The expressions for $a_{m}$ and $b_{m}$ which have been found in
\hardsectionref{9}{3} can
frequently be applied in practical examples to determine the points
at which the sum of a given Fourier series may be discontinuous. Thus,
let it be required to determine the places at which the sum of the
series
$$
\sin x
+ \frac{1}{3} \sin 3x
+ \frac{1}{5} \sin 5x
+ \cdots
$$
is discontinuous.

\emph{Assuming} that the series is a Fourier series and not \emph{any}
trigonometrical series and observing that
$a_{m} = 0, b_{m} = (2m)^{-1}(1 - \cos m \pi)$, we get on considering the
formula found in \hardsectionref{9}{3},
$$
A_{m} = 0,
\quad
B_{m} = \half - \half \cos m \pi,
\quad
a_{m}' = b_{m}' = 0.
$$

Hence if $k_{1}, k_{2},\ldots$ are the places at which the analytic
character of the sum is broken, we have
$$
0
=
\pi A_{m}
=
\thebracket{
  \sin m k_{1} \thebrace{
    f(k_{1} - 0) - f(k_{1} + 0)
  }
  +
  \sin m k_{2} \thebrace{
    f(k_{2} - 0) - f(k_{2} + 0)
  }
  +
  \cdots
}.
$$
Since this is true for all values of $m$, the numbers
$k_{1}, k_{2}, \ldots$ must
be multiples of $\pi$; but
there is only one even multiple of $\pi$ in the range
$-\pi < x \leq \pi$, namely zero.
So $k_{1} = 0$,
and $k_{2}, k_{3}, \ldots$ do not exist.
Substituting $k_{1} = 0$ in the equation
$B_{m} = \half - \half \cos m \pi$, we have
$$
\pi (\half - \half \cos m \pi)
=
- \thebracket{
  \cos m \pi \thebrace{
    f(\pi - 0) - f(-\pi + 0)
  }
  + f(-0)
  - f(+0)
}.
$$

Since this is true for all values of $m$, we have
$$
\half \pi = f(+0) - f(-0),
\quad
\half \pi = f(\pi - 0) - f(-\pi + 0).
$$

This shews that, if the
series is a Fourier series, $f(x)$ has discontinuities at the points
$n \pi$ ($n$ any integer), and since $a_{m}' = b_{m}' = 0$, we should
expect\footnote{In point of fact
  $$
  f(x)
  =
  \begin{cases}
    -\frac{1}{4} \pi & -\pi < x < 0;\\
    \frac{1}{4} \pi & 0 < x < \pi.
  \end{cases}
  $$
} $f(x)$
to be constant in the open range $(-\pi, 0)$ and to be another constant
in the open range $(0, \pi)$.

\Section{9}{4}{\Fejer's theorem.}

We now begin the discussion of the theory of Fourier series by proving
the following theorem, due to \Fejer,\footnote{TODO:Math. Ann. lviii. (1904), pp. 51-69.}
concerning the summability of
the Fourier series associated with an arbitrary function, $f(t)$:

\emph{Let $f(t)$ be a function of the real variable $t$, defined arbitrarily
  when $-\pi \leq t < \pi$, and defined by the equation
  $$
  f(t + 2\pi) = f(t)
  $$
%
% 170
%
  for all other real values of $t$; and let
  $\int_{-\pi}^{\pi} f(t) \dmeasure t$
  exist and (if it is an improper integral)
  let it he absolutely convergent.
}

\emph{Then the Fourier series associated with the function
  $f(t)$ is summable\footnote{See \hardsubsectionref{8}{4}{3}.} ($C1$)
  at all points $x$ at which the two limits $f(x \pm 0)$ exist.}

And its sum ($C1$) is
$$
\half \thebrace{
  f(x + 0) + f(x-0)
}.
$$

Let $a_{n}, b_{n}, (n=0,1,2,\ldots)$ denote the Fourier constants
(\hardsectionref{9}{2}) of
$f(t)$ and let
$$
\half a_{0} = A_{0},
\hfill
a_{n} \cos n x + b_{n} \sin n x = A_{n}(x),
\hfill
\sum_{n=0}^{m} A_{n}(x) = S_{m}(x).
$$

Then we have to prove that
$$
\lim_{m \rightarrow \infty}
\frac{1}{m} \thebrace{
  A_{0}
  + S_{1}(x) + S_{2}(x) + \cdots + S_{m-1}(x)
}
=
\half \thebrace{
  f(x+0) + f(x-0)
},
$$
provided that the limits on the right exist.

If we substitute for the Fourier constants their values in the form of
integrals (\hardsectionref{9}{2}), it is easy to verify
that\footnote{It is obvious that, if we write $\lambda$ for $e^{i(x-t)}$ in the second line,
  then
  \begin{align*}
    m + &
    (m-1) (\lambda + \lambda^{-1})
    + (m-2) (\lambda^{2} + \lambda^{-2})
    + \cdots
    + (\lambda^{m-1} + \lambda^{1 - m})
    \\
    =&
    (1 - \lambda)^{-1} \thebrace{
      \lambda^{1-m}
      + \lambda^{2-m}
      + \cdots
      + \lambda^{-1}
      + 1
      - \lambda
      - \lambda^{2}
      - \cdots
      - \lambda^{m}
    } \\
    =&
    (1 - \lambda)^{-2} \thebrace{
      \lambda^{1 - m}
      - 2 \lambda
      + \lambda^{m+1}
    }
    =
    (\lambda^{\half m} - \lambda^{-\half m})^{2}
    /
    (\lambda^{\half} - \lambda^{-\half})^{2}.
  \end{align*}
  m + (m - 1) (X + X-i) + (hi - 2) (X2 + X-2) + . . . + (X' -i + Xi-' )

  = (l-X)-i \ i-™ + X2-' +...+X-i + l-X--X'-i-...-X' = (1 - X)-2 xi- v -
  2X + X' +i = (X '" - X~ '"f /(X - X" )
}
\begin{align*}
  A_{0} + \sum_{n=1}^{m-1} S_{n}(x)
  =&
  m A_{0}
  + (m - 1) A_{1}(x)
  + (m - 2) A_{2}(x)
  + \cdots
  + A_{m-1}(x)
  \\
  =&
  \frac{1}{\pi}
  \int_{-\pi}^{\pi} \thebrace{
    \half m
    + (m-1) \cos (x-t)
    + (m-2) \cos 2(x-t)
    + \cdots
    + \cos (m-1)(x-t)
  }
  f(t) \dmeasure t
  \\
  =&
  \frac{1}{2\pi}
  \int_{-\pi}^{\pi}
  \frac{\sin^{2} \half m (x-t)}{\sin^{2} \half (x-t)}
  f(t) \dmeasure t
  \\
  =&
  \frac{1}{2\pi}
  \int_{-\pi+x}^{\pi+x}
  \frac{\sin^{2} \half m (x-t)}{\sin^{2} \half (x-t)}
  f(t) \dmeasure t,
\end{align*}
the last step following from the periodicity
of the integrand.

If now we bisect the path of integration and write $x \mp 2\theta$ in place of
$t$ in the two parts of the path, we get
$$
A_{0}
+ \sum_{n=1}^{m-1} S_{n}(x)
=
\frac{1}{\pi}
\int_{0}^{\half \pi}
\frac{\sin^{2} m \theta}{\sin^{2} \theta}
f(x + 2\theta) \dmeasure \theta
+
\frac{1}{\pi}
\int_{0}^{\half \pi}
\frac{\sin^{2} m \theta}{\sin^{2} \theta}
f(x - 2\theta) \dmeasure \theta
$$

\emph{Consequently it is sufficient to prove that, as
  $m \rightarrow \infty$, then}
$$
\frac{1}{m}
\int_{0}^{\half\pi}
\frac{\sin^{2} m \theta}{\sin^{2} \theta}
f(x + 2\theta)
\dmeasure \theta
\rightarrow
\half \pi f(x + 0),
\quad
\frac{1}{m}
\int_{0}^{\half\pi}
\frac{\sin^{2} m \theta}{\sin^{2} \theta}
f(x - 2\theta)
\dmeasure \theta
\rightarrow
\half \pi f(x - 0),
$$
%
% 171
%

Now, if we integrate the equation
$$
\half
\frac{\sin^{2} m \theta}{\sin^{2} \theta}
=
\half m + (m-1) \cos 2\theta + \cdots + \cos 2(m-1)\theta,
$$
we find that
$$
\int_{0}^{\half \pi}
\frac{\sin^{2} m \theta}{\sin^{2} \theta}
\dmeasure \theta
=
\half \pi m,
$$
and so we have to prove that
$$
\frac{1}{m}
\int_{0}^{\half \pi}
\frac{\sin^{2} m \theta}{\sin^{2} \theta}
\phi(\theta)
\dmeasure \theta
\rightarrow
0
\quad
\textrm{as} %TODO: clean up; do more properly
\quad
m \rightarrow \infty
$$
where $\phi(\theta)$ stands in turn for each of the two functions
$$
f(x + 2\theta) - f(x + 0),
\quad
f(x - 2\theta) - f(x - 0).
$$
Now, given an arbitrary positive number $\eps$, we can choose $\delta$ so
that\footnote{On the assumption that $f(x \pm 0)$ exist.}
$$
\absval{\phi(\theta)} < \eps
$$
whenever $0 < \theta \leq \half \delta$. This choice of $\delta$ is obviously independent of $m$.

Then
\begin{align*}
  \absval{
    \frac{1}{m}
    \int_{0}^{\half \pi}
    \frac{\sin^{2} m\theta}{\sin^{2} \theta}
    \phi(\theta)
    \dmeasure \theta
  }
  &
  \leq
  \frac{1}{m}
  \int_{0}^{\half \delta}
  \frac{\sin^{2} m\theta}{\sin^{2} \theta}
  \absval{ \phi(\theta) }
  \dmeasure \theta
  +
  \frac{1}{m}
  \int_{\half\delta}^{\half\pi}
  \frac{\sin^{2} m\theta}{\sin^{2} \theta}
  \absval{ \phi(\theta) }
  \dmeasure \theta
  \\
  &
  <
  \frac{\eps}{m}
  \int_{0}^{\half \delta}
  \frac{\sin^{2} m\theta}{\sin^{2} \theta}
  \dmeasure \theta
  +
  \frac{1}{m \sin^{2} \half \delta}
  \int_{\half \delta}^{\half \pi}
  \absval{\phi(\theta)}
  \dmeasure \theta
  \\
  &
  \leq
  \frac{\eps}{m}
  \int_{0}^{\half \pi}
  \frac{\sin^{2} m\theta}{\sin^{2} \theta}
  \dmeasure \theta
  +
  \frac{1}{m \sin^{2} \half\delta}
  \int_{0}^{\half \pi}
  \absval{\phi(\theta)}
  \dmeasure \theta
  \\
  &
  =
  \half \pi \eps
  +
  \frac{1}{m \sin^{2} \half\delta}
  \int_{0}^{\half \pi}
  \absval{\phi(\theta)}
  \dmeasure \theta
\end{align*}

Now the convergence of
$\int_{-\pi}^{\pi} \absval{f(t)} \dmeasure t$
entails the convergence of
$$
\int_{0}^{\half \pi}
\absval{\phi(\theta)}
\dmeasure \theta,
$$
and so, given $\eps$ (and therefore $\delta$), we can make
$$
\half \pi \eps
\sin^{2} \half \delta
>
\int_{0}^{\half\pi}
\absval{ \phi(\theta) }
\dmeasure \theta,
$$
by taking $m$ sufficiently large.

Hence, by taking $m$ sufficiently large, we can make
$$
\absval{
  \frac{1}{m}
  \int_{0}^{\half \pi}
  \frac{\sin^{2} m \theta}{\sin^{2} \theta}
  \phi(\theta)
  \dmeasure \theta
}
<
\pi \eps,
$$
where $\eps$ is an arbitrary positive number; that is to say, from the
definition of a limit,
$$
\lim_{m \rightarrow \infty}
\frac{1}{m}
\int_{0}^{\half\pi}
\frac{\sin^{2} m\theta}{\sin^{2} \theta}
\phi(\theta)
\dmeasure \theta
=
0,
$$
and so \Fejer's theorem is established.
%
% 172
%

%TODO:Corollary
Corollary 1. Let $U$ and $L$ be the upper and lower bounds of $f(t)$ in any
interval $(a, b)$ whose length does not exceed $2\pi$, and let
$$
\int_{-\pi}^{\pi} \absval{f(t)} \dmeasure t = \pi A.
$$
Then, if $a + \eta \leq x \leq b - \eta$, where $\eta$ is any positive number, we have
\begin{align*}
  U
  -
  \frac{1}{m}
  \thebrace{
    A_{0}
    +
    \sum_{n=1}^{m-1} S_{n}(x)
  }
  &
  =
  \frac{1}{2 m \pi}
  \thebrace{
    \int_{-\pi + x}^{x - \eta}
    + \int_{x - \eta}^{x + \eta}
    + \int_{x + \eta}^{\pi + x}
  }
  \frac{\sin^{2} \half m (x-t)}{\sin^{2} \half (x-t)}
  \thebrace{
    U - f(t)
  }
  \dmeasure t
  \\
  &
  \geq
  \frac{1}{2 m \pi}
  \thebrace{
    \int_{-\pi + x}^{x - \eta}
    + \int_{x + \eta}^{\pi + x}
  }
  \frac{\sin^{2} \half m (x-t)}{\sin^{2} \half (x-t)}
  \thebrace{
    U - f(t)
  }
  \dmeasure t
  \\
  &
  \geq
  -\frac{1}{2 m \pi}
  \thebrace{
    \int_{-\pi + x}^{x - \eta}
    + \int_{x + \eta}^{\pi + x}
  }
  \frac{\absval{U} + \absval{f(t)}}{\sin^{2} \half \eta}
  \dmeasure t
\end{align*}
so that
$$
\frac{1}{m}
\thebrace{
  A_{0}
  + \sum_{n=1}^{m-1} S_{n}(x)
}
\leq
U +
\thebrace{
  \absval{U}
  + \half A
}
/
\thebrace{
  m \sin^{2} \half \eta
}.
$$
Similarly
$$
\frac{1}{m}
\thebrace{
  A_{0}
  + \sum_{n=1}^{m-1} S_{n}(x)
}
\geq
L -
\thebrace{
  \absval{L}
  + \half A
}
/
\thebrace{
  m \sin^{2} \half \eta
}.
$$
Corollary 2. Let $f(t)$ be continuous in the interval $a \leq t \leq b$. Since
continuity implies uniformity of continuity (\hardsubsectionref{3}{6}{1}), the choice of
$\delta$ corresponding to any value of $x$ in $(a, b)$ is independent of $x$, and the
upper bound of $\absval{f(x \pm 0)}$, i.e. of $\absval{f(x)}$, is also independent of $x$,
so that
\begin{align*}
  \int_{0}^{\half \pi} \absval{\phi(\theta)} \dmeasure \theta
  =&
  \int_{0}^{\half \pi}
  \absval{
    f(x \pm 2\theta) - f(x \pm 0)
  }
  \dmeasure \theta
  \\
  \leq
  &
  \half \int_{-\pi}^{\pi} \absval{f(t)} \dmeasure t
  + \half \pi \absval{f(x \pm 0)},
\end{align*}
and the upper bound of the last expression is independent of $x$.

Hence the choice of $m$, which makes
$$
\absval{
  \frac{1}{m}
  \int_{0}^{\half \pi}
  \frac{\sin^{2} m \theta}{\sin^{2} \theta}
  \phi(\theta)
  \dmeasure \theta
}
<
\pi \eps,
$$
is independent of $x$, \emph{and consequently
  $\frac{1}{m}\thebrace{
    A_{0} + \sum_{n=1}^{m-1} S_{n}(x)
  }$ tends to the limit $f(x)$, as $m \rightarrow \infty$,
  uniformly throughout the interval $a \leq x \leq b$}.
\Subsection{9}{4}{1}{The Riemann-Lehesgue lemmas.}
In order to be able to apply Hardy's theorem (\hardsectionref{8}{5}) to deduce the
convergence of Fourier series from \Fejer's theorem, we need the two
following lemmas :

%TODO
(I) Let $\int_{a}^{b} \psi(\theta) \dmeasure \theta$ exist and (if it is an improper integral) let it be
absolutely convergent. Then, as $\lambda \rightarrow \infty$,
$$
\int_{a}^{b} \psi(\theta) \sin (\lamba \theta) \dmeasure \theta
\quad
\textrm{is}
\quad
\littleo(1).
$$
%TODO
(II) If, further, $\psi(\theta)$ has limited total fluctuation in the range
$(a,b)$ then, as $\lambda \rightarrow \infty$,
$$
\int_{a}^{b} \psi(\theta) \sin (\lambda\theta) \dmeasure \theta
\quad
\textrm{is}
\quad
\bigo(1 / \lambda).
$$

%
% 173
%

Of these results (I) %TODO:fixref
was stated by W. R. Hamilton\footnote{TODOTrans. Dublin Acad. xix. (1843), p. 267.} and by
Riemann\footnote{TODO:Ges. Math. IVerke, p. 241. For Lebesgue's investigation see his
Series trigonometriques (1906), Ch. III.} in
the case of bounded fiuictions.
The truth of (II) %TODO:fixref
seems to have been
well known before its importance was realised; it is a generalisation
of a result established by Dirksen\footnote{TODO:Journal fUr Math. iv. (1829), p. 172.}
and Stokes (see \hardsectionref{9}{3}) in the
case of functions with a continuous differential coefficient.

The reader should observe that the analysis of this section remains
valid when the sines are replaced throughout by cosines.

%TODO:fixref
(I) It is convenient\footnote{For this proof we are indebted to Mr Hardy;
  it seems to be neater than the proofs given by other writers,
  %TODO
  e.g. de la Vallee Poussin,
  Cours cV Analyse Infinitesiniale, ii. (1912), pp. 140-141.}
to establish this lemma first in the case in
which $\psi(\theta)$ is bounded in the range $(a, b)$. In this case, let $K$ be
the upper bound of $\absval{\psi(\theta)}$, and let $\eps$ be an arbitrary positive
number. Divide the range $(a, b)$ into $n$ parts by the points
$x_{1}, x_{2}, \ldots x_{n-1}$, and form the sums $S_{n}, s_{n}$ associated with the function
$\psi(\theta)$ after the manner of \hardsectionref{4}{1}. Take $n$ so large that
$S_{n} - s_{n} < \eps$; this is
possible since $\psi(\theta)$ is integrable.

In the interval $(x_{r-1}, x_{r})$ write
$$
\psi(\theta) = \psi_{r}(x_{r-1}) + \omega_{r}(\theta),
$$
so that
$$
\absval{ \omega_{r}(\theta) } \leq U_{r} - L_{r},
$$
where $U_{r}$ and $L_{r}$ are the upper and lower bounds of
$\psi(\theta)$ in the interval
$(x_{r-1}, x_{r})$.

It is then clear that

By taking X sufficiently large (n remaining fixed after e has been
chosen), the last expression may be made less than 2e, so that
%TODO:alignment
\begin{align*}
  &
  \absval{
    \int_{a}^{b}
    \psi(\theta) \sin(\lambda \theta) \dmeasure \theta
  }
  \\
  & \quad
  =
  \absval{
    \sum_{r=1}^{n}
    \psi_{r}(x_{r-1})
    \int_{x_{r-1}}^{x_{r}}
    \sin (\lambda \theta) \dmeasure \theta
    +
    \sum_{r=1}^{n}
    \int_{x_{r-1}}^{x_{r}}
    \omega_{r}(\theta) \sin(\lambda \theta) \dmeasure \theta
  }
  \\
  & \quad
  \leq
  \sum_{r=1}^{n}
  \absval{
    \psi_{r}(x_{r-1})
  }
  \cdot
  \absval{
    \int_{x_{r-1}}^{x_{r}}
    \sin (\lambda \theta) \dmeasure \theta
  }
  +
  \sum_{r=1}^{n}
  \int_{x_{r-1}}^{x_{r}}
  \absval{ \omega_{r} (\theta)} \dmeasure \theta
  \\
  & \quad
  \leq n K \cdot (2 / \lambda)
  +
  (S_{n} - s_{n})
  \\
  & \quad
  < (2nK / \lambda)
  +
  \eps.
\end{align*}

By taking $\lambda$ sufficiently large ($n$ remaining fixed after $\eps$ has been
chosen), the last expression may be less than $2\eps$, so that
$$
\lim_{\lambda \rightarrow \infty}
\int_{a}^{b} \psi(\theta) \sin(\lambda \theta) \dmeasure \theta = 0,
$$
and this is the result stated.

When $\psi(\theta)$ is unbounded, if it has an absolutely convergent integral,
by \hardsectionref{4}{5}, we may enclose the points at which it is unbounded in a
finite\footnote{The \emph{finiteness} of the number of intervals is assumed in the
definition of an improper integral,\hardsectionref{4}{5}.} number
%
% 174
%
of intervals $\delta_{1}, \delta_{2}, \ldots, \delta_{p}$ such that
$$
\sum_{n=1}^{p}
\int_{\delta_{r}}
\absval{ \psi(\theta) } \dmeasure \theta
<
\eps.
$$
If $K$ denote the upper bound of $\absval{\psi(\theta)}$ for values of
$\theta$ outside these intervals, and if
$\gamma_{1}, \gamma_{2}, \ldots, \gamma_{p+1}$ denote the portions of the interval
$(a, b)$ which do not belong to $\delta_{1}, \delta_{2}, \ldots, \delta_{p}$
we may prove as before that
\begin{align*}
  \absval{\int_{a}^{b} \psi(\theta) \sin (\lambda \theta) \dmeasure \theta}
  & =
  \absval{
    \sum_{r=1}^{p+1}
    \int_{\gamma_{r}} \psi(\theta) \sin(\lambda \theta) \dmeasure \theta
    +
    \sum_{r=1}^{p}
    \int_{\delta_{r}} \psi(\theta) \sin(\lambda \theta) \dmeasure \theta
  }
  \\
  & \leq
  \absval{
    \sum_{r=1}^{p+1}
    \int_{\gamma_{r}} \psi(\theta) \sin(\lambda \theta) \dmeasure \theta
  }
  +
  \sum_{r=1}^{p}
  \int_{\delta_{r}}
  \absval{\psi(\theta) \sin(\lambda \theta)}
  \dmeasure \theta
  \\
  & <
  (2nK / \lambda) + 2 \eps.
\end{align*}

Now the choice of $\eps$ fixes $n$ and $K$, so that the last expression may be
made less than $3\eps$ by taking $\lambda$ sufficiently large. That is to say
that, even if $\psi(\theta)$ be unbounded,
$$
\lim_{\lambda \rightarrow \infty}
\int_{a}^{b} \psi(\theta) \sin(\lambda \theta) \dmeasure \theta
=
0,
$$
provided that $\psi(\theta)$ has an (improper) integral which is absolutely
convergent.

The first lemma is therefore completely proved.

%TODO
(II) When $\psi(\theta)$ has limited total fluctuation in the range $(a, b)$,
by %TODO: un-hardcode example number
\hardsubsectionref{3}{6}{4} example 2, we may write
$$
\psi(\theta) = \chi_{1}(\theta) - \chi_{2}(\theta),
$$
where $\chi_{1}(\theta), \chi_{2}(\theta)$ are positive increasing bounded functions.

Then, by the second mean-value theorem (\hardsubsectionref{4}{1}{4}) a number
$\xi$ exists such that $a \leq \xi \leq b$ and
\begin{align*}
  \absval{
    \int_{a}^{b} \chi_{1}(\theta) \sin(\lambda \theta) \dmeasure \theta
  }
  &=
  \absval{
    \chi_{1}(b) \int_{\xi}^{b} \sin(\lambda \theta) \dmeasure \theta
  }
  \\
  & \leq
  2 \chi_{1}(b) / \lambda.
\end{align*}

If we treat $\chi_{2}(\theta)$ in similar manner, it follows that
\begin{align*}
  \absval{
    \int_{a}^{b} \psi (\theta) \sin(\lambda \theta) \dmeasure \theta
  }
  &
  \leq
  \absval{
    \int_{a}^{b} \chi_{1}(\theta) \sin(\lambda\theta) \dmeasure \theta
  }
  +
  \absval{
    \int_{a}^{b} \chi_{2}(\theta) \sin(\lambda\theta) \dmeasure \theta
  }
  \\
  &
  \leq
  2 \thebrace{ \chi_{1}(b) + \chi_{2}(b) } / \lambda
  \\
  = &
  \bigO(1 / \lambda),
\end{align*}
and the second lemma is established.

Corollary. If $f(t)$ lie such that $\int_{-\pi}^{\pi} f(t)$ %TODO:fix typo in book (missing 'dt')?
exists and is an absolutely convergent integral,
the Fourier constants $a_{n}, b_{n}$ of $f(t)$ are $\littleo{l}$ as
$n \rightarrow \infty$;
and if, further, $f(t)$ has limited total fluctuation in the range
$(-\pi, \pi)$, the Fourier constants are $\bigo{1/n}$.

[Of course these results are not sufficient to ensure the convergence
of the Fourier series associated with $f(t)$; for a series, in which the
terms are of the order of magnitude of the terms in the harmonic
series \hardsectionref{2}{3}), is not necessarily convergent.]

%
% 175
%

\Subsection{9}{4}{2}{The proof of Fourier's theorem.}

We shall now prove the theorem enunciated in \hardsectionref{9}{2}, namely:

%TODO:emphasize paragraphs
Let $f(t)$ be a function defined arbitrarily when $-\pi \leq t < \pi$, and defined by the
equation $f(t + 2\pi) = f(t)$ for all other real values of $t$; and let
$\int_{-\pi}^{\pi} f(t) \dmeasure t$
exist and (if it is an improper integral) let it be absolutely
convergent. Let $a_{n}, b_{n}$ be defined by the equations
$$
\pi a_{n} = \int_{-\pi}^{\pi} f(t) \cos nt \dmeasure t,
\quad
\pi b_{n} = \int_{-\pi}^{\pi} f(t) \sin nt \dmeasure t.
$$
Then, if $x$ be an interior point of any interval $(a, b)$ within which
$f(t)$ has limited total fluctuation, the series
$$
\half a_{0}
+
\sum_{n=1}^{\infty} (
a_{n} \cos nx + b_{n} \sin nx
)
$$
is convergent and its sum is $half \thebrace{f(x+0) + f(x-0)}$.

It is convenient to give two proofs, one applicable to functions for
which it is permissible to take the interval $(a, b)$ to be the interval
$(-\pi+x, \pi + x)$, the other applicable to functions for which it is
not permissible.

%TODO:autonumbering
(I) When the interval $(a, b)$ may be taken to be $(-\pi + x, \pi + x)$,
it follows from \hardsubsectionref{9}{4}{1} (II) %TODO:ref
that $a_{n} \cos nx + b_{n} \sin nx$ is $\bigo{1/n}$ as
$n \rightarrow \infty$. Now by \Fejer's theorem (\hardsectionref{9}{4})
the series under consideration
is summable (Cl) %TODO
and its sum (C'l) %TODO
is\footnote{The limits $f(x \pm 0)$ exist, by \hardsubsectionref{3}{6}{4} example 3.%TODO
}
$\half \thebrace{f(x+0) + f(x-0)}$. Therefore.,
by Hardy's convergence theorem \hardsectionref{8}{5}), the series under consideration
is CONVERGENT %TODO:emph?
and its sum (by \hardsubsectionref{8}{4}{3}) is
$\half \thebrace{f(x+0) + f(x-0)}$.

(II) %TODO
Even if it is not permissible to take the interval $(a, b)$ to be
the whole interval $(-\pi + x, \pi + x)$, it is possible, by
hypothesis, to choose a positive number $\delta$, less than $\pi$,
such that $f(t)$
has limited total fluctuation in the interval $(x-\delta, x+\delta)$.
We now define an auxiliary function $g(t)$, which is equal to $f(t)$ when
$x - \delta \leq t \leq x + \delta$,
and which is equal to zero throughout the rest of the interval
$(-\pi + x, \pi + x)$; and $g(t + 2\pi)$ is to be equal to $g(t)$ for all real
values of $t$.

Then $g(t)$ satisfies the conditions postulated for the functions under
consideration in (I),%TODO:ref
namely that it has an integral which is
absolutely convergent and it has limited total fluctuation in the
interval $(-\pi + x, \pi + x)$; and so, if
$a_{n}^{(1)}, b_{n}^{(1)}$ denote the Fourier
constants of $g(t)$, the arguments used in (I) %TODO:addref
prove that the Fourier
series associated with $g(t)$, namely
$$
\half a_{0}^{(1)}
+
\sum_{n=1}^{\infty}
\theparen{
  a_{n}^{(1)} \cos nx
  +
  b_{n}^{(1)} \sin nx
},
$$
is convergent and has the sum
$\half \thebrace{g(x+0) + g(x-0)}$, and this is
equal to
$$
\half \thebrace{
  f(x+0) + f(x-0)
}.
$$
%
% 176
%

Now let $S_{m}(x)$ and $S_{m}^{(1)} (x)$ denote the sums of the first $m + 1$
terms of the Fourier series associated with $f(t)$ and $g(t)$ respectively. Then
it is easily seen that
\begin{align*}
  S_{m}(x)
  =&
  \frac{1}{\pi}
  \int_{-\pi}^{\pi} \thebrace{
    \half
    + \cos (x-t)
    + \cos 2(x-t)
    + \cdots
    + \cos m(x-t)
  }
  f(t) \dmeasure t
  \\
  =&
  \frac{1}{2\pi}
  \int_{-\pi}^{\pi}
  \frac{\sin (m+\half) (x-t) }{\sin \half (x-t)}
  f(t) \dmeasure t
  \\
  =&
  \frac{1}{2\pi}
  \int_{-\pi+x}^{\pi+x}
  \frac{\sin (m+\half) (x-t) }{\sin \half (x-t)}
  f(t) \dmeasure t
  \\
  =&
  \frac{1}{\pi}
  \int_{0}^{\half \pi}
  \frac{\sin (2m+1)\theta}{\sin \theta}
  f(x + 2\theta) \dmeasure \theta
  +
  \frac{1}{\pi}
  \int_{0}^{\half \pi}
  \frac{\sin (2m+1)\theta}{\sin \theta}
  f(x - 2\theta) \dmeasure \theta,
\end{align*}
by steps analogous to those given in \hardsectionref{9}{4}.

In like manner
$$
S_{m}^{(1)}(x)
=
\frac{1}{\pi}
\int_{0}^{\half \pi}
\frac{\sin (2m+1)\theta}{\sin \theta}
g(x + 2\theta) \dmeasure \theta
+
\frac{1}{\pi}
\int_{0}^{\half \pi}
\frac{\sin (2m+1)\theta}{\sin \theta}
g(x - 2\theta) \dmeasure \theta,
$$
and so, using the definition of $g(t)$, we have
\begin{align*}
  S_{m}(x) - S_{m}^{(1)}(x)
  =&
  \frac{1}{\pi}
  \int_{\half\delta}^{\half \pi} \sin (2m+1)\theta
  \frac{f(x+2\theta)}{\sin \theta}
  \dmeasure \theta
  \\
  &\quad
  \frac{1}{\pi}
  \int_{\half\delta}^{\half \pi} \sin (2m+1)\theta
  \frac{f(x-2\theta)}{\sin \theta}
  \dmeasure \theta.
\end{align*}

Since $\cosec$ is a continuous function in the range $(\half \delta, \half \pi)$, it follows
that $f(x \pm 2\theta) \cosec \theta$ are integrable functions with absolutely
convergent integrals; and so, by the Riemann-Lebesgue lemma of
\hardsubsectionref{9}{4}{1} (I), %TODO:add reference to lemma (the `(I)' here)
\emph{both the integrals on the right in the last equation tend to zero
as $m \rightarrow \infty$}.

That is to say
$$
\lim_{m \rightarrow \infty}
\thebrace{
  S_{m}(x) - S_{m}^{(1)}(x)
}
= 0.
$$

Hence, since
$$
\lim_{m \rightarrow \infty}
S_{m}^{(1)}(x)
=
\half \thebrace{
  f(x+0) + f(x-0)
},
$$
it follows also that
$$
\lim_{m \rightarrow \infty} S_{m}(x)
=
\half \thebrace{
  f(x+0) + f(x-0)
}.
$$

\emph{We have therefore proved that the Fourier series associated with
  $f(t)$, namely
  $
  \half a_{0}
  + \sum \theparen{
    a_{n} \cos nx
    +
    b_{n} \sin nx
  }
  $
  is convergent and its sum is}
$$
\half \thebrace{
  f(x+0) + f(x-0)
}
$$
\Subsection{9}{4}{3}{The Dirichlet-Bonnet proof of Fourier's theorem.}
It is of some interest to prove directly the theorem of \hardsubsectionref{9}{4}{2},
without making use of the theory of summability; accordingly we now
give a proof which is on the same general lines as the proofs due to
Dirichlet and Bonnet.
%
% 177
%

As usual we denote the sum of the first $m + 1$ terms of the Fourier
series by $S_{m}(x)$, and then, by the analysis of \hardsubsectionref{9}{4}{2}, we have
$$
S_{m}(x)
=
\frac{1}{\pi}
\int_{0}^{\half \pi}
\frac{\sin (2m+1)\theta}{\sin \theta}
\f(x + 2\theta)
\dmeasure \theta
+
\frac{1}{\pi}
\int_{0}^{\half \pi}
\frac{\sin (2m+1)\theta}{\sin \theta}
\f(x - 2\theta)
\dmeasure \theta.
$$

Again, on integrating the equation
$$
\frac{\sin (2m+1)\theta}{\sin \theta}
=
1
+ 2 \cos 2\theta
+ 2 \cos 4\theta
+ \cdots
+ 2 \cos 2m\theta,
$$
we have
$$
\int_{0}^{\half\pi}
\frac{\sin (2m+1)\theta}{\sin \theta}
\dmeasure \theta
=
\half\pi,
$$
so that
\begin{align*}
  TODO
\end{align*}

In order to prove that
$$
\lim_{m \rightarrow \infty}
S_{m}(x)
=
\half \thebrace{
  f(x+0) + f(x-0)
},
$$
it is therefore sufficient to prove that
$$
\lim_{m \rightarrow \infty}
\int_{0}^{\half\pi}
\frac{\sin (2m+1)\theta}{\sin \theta}
\phi(\theta) \dmeasure \theta
=
0,
$$
where $\phi(\theta)$ stands in turn for each of the functions
$$
f(x+2\theta) - f(x+0),
\quad
f(x-2\theta) - f(x-0).
$$
Now, by \hardsubsectionref{3}{6}{4} example 4, %TODO:replace ref
$\theta \phi(\theta) \cosec \theta$ is a function with limited
total fluctuation in an interval of which $\theta=0$ is an
end-point;\footnote{The other end-point is $\theta = \half (b-x)$
  or $\theta = \half (x-a)$, according as $\phi(\theta)$
  represents one or other of the two functions.} and so
we may write
$$
\theta
\phi (\theta)
\cosec \theta
=
\chi_{1}(\theta)
-
\chi_{2}(\theta),
$$
where $\chi_{1}(\theta), \chi_{2}(\theta)$ are bounded positive
increasing functions of $\theta$ such that
$$
\chi_{1}(+0) = \chi_{2}(+0) = 0.
$$

Hence, given an arbitrary positive number $\eps$, we can choose a positive
number $\delta$ such that
$$
0 \leq \chi_{1}(\theta) < \eps,
\quad
0 \leq \chi_{2}(\theta) < \eps
$$
whenever $0 \leq \theta \leq \half \delta$.

We now obtain inequalities satisfied by the three integrals on the
right of the obvious equation
\begin{align*}
  &
 \int_{0}^{\half\pi}
 \frac{\sin (2m+1)\theta}{\sin \theta}
 \phi(\theta) \dmeasure \theta
 =
 \int_{\half \delta}^{\half \pi}
 \sin (2m+1)\theta
 \frac{\phi(\theta)}{\sin \theta}
 \dmeasure \theta
 \\
 &
 \quad
 \int_{0}^{\half \delta}
 \frac{\sin (2m+1)\theta}{\theta}
 \chi_{1}(\theta) \dmeasure \theta
 -
 \int_{0}^{\half \delta}
 \frac{\sin (2m+1)\theta}{\theta}
 \chi_{2}(\theta) \dmeasure \theta
\end{align*}
%
% 178
%

The modulus of the first integral can be made less than $\eps$ by taking $m$
sufficiently large; this follows from
\hardsubsectionref{9}{4}{1} (i) %TODO:subref
since $\phi(\theta) \cosec \theta$
has an integral which converges absolutely in the interval
$(\half \delta, \half \pi)$.

Next, from the second mean-value theorem, it follows that there is a
number $\xi$ between $0$ and $\delta$ such that
\begin{align*}
TODO
\end{align*}

Since
$\int^{\infty} \frac{\sin t}{t} \dmeasure t$
is convergent, it follows that
$\absval{\int_{\beta}^{\infty} \frac{\sin u}{u} \dmeasure u}$
has an upper bound\footnote{The reader will find it interesting to prove that
  $\int_{0}^{\infty} \frac{\sin u}{u} \dmeasure u = \half\pi$.}
$B$ which is independent of $\beta$, and it is then clear that
$$
\absval{
  \int_{0}^{\half\delta}
  \frac{\sin (2m+1)\theta}{\theta}
  \chi_{1}(\theta)
  \dmeasure \theta
}
\leq
2 B \chi_{1}(\half\delta)
<
2 B \eps.
$$

On treating the third integral in a similar manner, we see that we can
make
$$
\absval{
  \int_{0}^{\half\delta}
  \frac{\sin (2m+1)\theta}{\sin \theta}
  \phi(\theta)
  \dmeasure \theta
}
<
(4B+1) \eps
$$
by taking $m$ sufficiently large; \emph{and so we have proved that}
$$
\lim_{m \rightarrow \infty}
\int_{0}^{\half\pi}
\frac{\sin (2m+1)\theta}{\sin \theta}
\phi(\theta)
\dmeasure \theta
=
0.
$$
But it has been seen that this is a sufficient condition for the limit
of $S_{m}(x)$ to be $\half \thebrace{f(x+0) + f(x-0)}$; and we have therefore
established the con- vergence of a Fourier series in the circumstances
enunciated in \hardsubsectionref{9}{4}{2}.

%\begin{smallfontnote}%TODO
Note. The reader should observe that in either proof of the
convergence of a Fourier \emph{series} the second mean-value theorem is
required; but to prove the summability of the series, the \emph{first}
mean-value theorem is adequate. It should also be observed that, while
restrictions are laid upon $f(t)$ throughout the range $(-\pi, \pi)$ in
establishing the \emph{summability} at any point $x$, the only additional
restriction necessary to ensure \emph{convergence} is a restriction on the
behaviour of the function in the \emph{immediate neighbourhood} of the point
$x$. The fact that the convergence depends only on the behaviour of the
function in the immediate neighbourhood of $x$ (provided that the
function has an integral which is absolutely convergent) was noticed
by Riemann and has been emphasised by Lebesgue,
%TODO:ref
Series Trigonometriques p: 60.
%\end{smallfontnote}%TODO

It is obvious that the condition\footnote{Due to Jordan, Comptes Rendus, xcii. (1881), p. 228.}
that $x$ should be an interior point
of an interval in which $f(t)$ has limited total fluctuation is merely a
\emph{sufficient} condition for the convergence of the Fourier series; and
it may be replaced by any condition which makes
$$
\lim_{m \rightarrow \infty}
\int_{0}^{\half\pi}
\frac{\sin (2m+1)\theta}{\sin \theta}
\phi(\theta)
\dmeasure \theta
=
0.
$$
%
% 179
%

Jordan's condition is, however, a natural modification of the
Dirichlet condition that the function $f(t)$ should have only a finite
number of maxima and minima, and it does not increase the difficulty
of the proof.

Another condition with the same efifect is due to Dini,
% TODO:ref
Sopra le Serie di Foiirier (Pisa, 1880),
namely that, if
$$
\Phi(\theta)
=
\thebrace{
  f(x+2\theta) + f(x-2\theta) - f(x+0) - f(x-0)
}
\theta,
$$
then
$\int_{0}^{a} \Phi(\theta) \dmeasure \theta$ should converge absolutely for some positive value of
$a$.

[If the condition is satisfied, given $\eps$ we can find $\delta$ so that
$$
\int_{0}^{\half\delta}
\absval{
  \Phi(\theta)
}
\dmeasure \theta
<
\eps,
$$
and then
$$
\absval{
  \int_{0}^{\half\delta}
  \sin (2m+1)\theta
  \frac{\theta}{\sin \theta}
  \Phi(\theta)
  \dmeasure \theta
}
<
\half \pi \eps
$$
the proof that
$
\absval{
  \int_{\half\delta}^{\half \pi}
  \frac{\sin (2m+1)\theta}{\sin \theta}
  \phi(\theta)
  \dmeasure \theta
}
<
\eps
$
for sufficiently large values of $m$
follows from the Riemann-Lebesgue lemma.]

A more stringent condition than Dini's is due to Lipschitz,
Journal fuer Math. LXiil. (1864), p. 296, %TODO:ref
namely $\absval{\phi(\theta) < C \theta^{k}}$, where $C$ and
$k$ are positive and independent of $\theta$.

For other conditions due to Lebesgue and to
de la Vallee Poussin, %TODO:proper accents
see the latter's
Cours d' Analyse In/lnitesimale, ii. (1912), pp. 149-150. %TODO:ref
It should be noticed that Jordan's condition differs in character from
Dini's condition; the latter is a condition that the series may
converge \emph{at a point}, the former that the series may converge
\emph{throughout an interval}.

\Subsection{9}{4}{4}{The uniformity of the convergence of Fourier series.}
Let $f(t)$ satisfy the conditions enunciated in \hardsubsectionref{9}{4}{2},
and further let it be continuous
(in addition to having limited total fluctuation) in
an interval $(a, b)$. \emph{Then the Fourier series associated with $f(t)$
  converges uniformly to the sum $f(x)$ at all points $x$ for which
  $a + \delta \leq x \leq b - \delta$, where $\delta$ is any positive number.}

Let $h(t)$ be an auxiliary function defined to be equal to $f(t)$ when
$a \leq t \leq b$ and equal to zero for other values of $t$ in the range
$(-\pi, \pi)$, and
let $\alpha_{n}, \beta_{n}$ denote the Fourier constants of $h(t)$.
Also let $S_{m}^{2}(x)$ denote the sum of the first $m + 1$ terms of the
Fourier series associated with $h(t)$.

Then, by \hardsectionref{9}{4} corollary 2,%TODO:ref
it follows that
$\half \alpha_{0} + \sum_{n=1}^{\infty} \theparen{\alpha_{n} \cos nx + \beta_{n} \sin nx}$
is \emph{uniformly} summable throughout the interval $(a+\delta, b-\delta)$;
and since
$$
\absval{
  \alpha_{n} \cos nx + \beta_{n} \sin nx
}
\leq
\sqrt{
  \alpha_{n}^{2} + \beta_{n}^{2}
},
$$
which is independent of $x$ and which, by \hardsubsectionref{9}{4}{1} (ii), %TODO:ref
is $\bigO(1/n)$, it
follows from \hardsectionref{8}{5} corollary that
$$
\half \alpha_{0} + \sum_{n=1}^{\infty} \theparen{\alpha_{n} \cos nx + \beta_{n} \sin nx}
$$
converges uniformly to the sum $h(x)$, which is equal to $f(x)$.

Now, as in \hardsubsectionref{9}{4}{2},
$$
S_{m}(x)
-
S_{m}^{(2)}(x)
=
\frac{1}{\pi}
\int_{\half (b-x)}^{\half \pi}
\frac{\sin (2m+1)\theta}{\sin \theta}
f(x + 2\theta)
\dmeasure \theta
+
\frac{1}{\pi}
\int_{\half (x-a)}^{\half \pi}
\frac{\sin (2m+1)\theta}{\sin \theta}
f(x - 2\theta)
\dmeasure \theta.
$$
%
% 180
%

As in \hardsubsectionref{9}{4}{1} we choose an arbitrary positive number
$\eps$ and then enclose the points at which $f(t)$ is unbounded in a set of intervals
$\delta_{1}, \delta_{2}, \ldots, \delta_{p}$, such
that
$\sum_{r=1}^{p} \int_{\delta_{r}} \absval{f(t)} \dmeasure t < \eps$.

If $K$ be the upper bound of $\absval{f(t)}$ \ outside these intervals, we then
have, as in \hardsubsectionref{9}{4}{1},
$$
\absval{
  S_{m}(x) - S_{m}^{(2)}(x)
}
<
\theparen{
  \frac{2nK}{2m + 1} + 2\eps
}
\cosec \delta,
$$
where the choice of $n$ depends only on $a$ and $b$ and the form of the
function $f(t)$. Hence, by a choice of $m$ \emph{independent} of $x$ we can make
$$
\absval{
  S_{m}(x) - S_{m}^{(2)}(x)
}
$$
arbitrarily small; so that $\absval{S_{m}(x) - S_{m}^{(2)}(x)}$
tends uniformly to zero. Since
$S_{m}^{(2)}(x) \rightarrow f(x)$
uniformly, it is then obvious that
$S_{m}(x) \rightarrow f(x)$
uniformly; and this is the result to be proved.

%\begin{notefont}???
NOTE. It must be observed that no general statement can be made about
uniformity or absoluteness of convergence of Fourier series. Thus the
series of \hardsubsectionref{9}{1}{1} example 1 %TODO:ref
converges uniformly except near $x = (2n+1)\pi$
but converges absolutely only when $x = n\pi$, whereas the series of
\hardsubsectionref{9}{1}{1} example 2 %TODO:ref
converges uniformly and absolutely for all real values
of $x$.
%\begin{notefont}???
\begin{wandwexample}
 If $\phi(\theta)$ satisfies suitable conditions in the range $(0, \pi)$,
shew that
\begin{align*}
  \lim_{m \rightarrow \infty}
  \int_{0}^{\pi}
  \frac{\sin (2m+1)\theta}{\sin \theta}
  \phi(\theta) \dmeasure \theta
  =&
  \lim_{m \rightarrow \infty}
  \int_{0}^{\half \pi}
  \frac{\sin (2m+1)\theta}{\sin \theta}
  \phi(\theta) \dmeasure \theta
  \\
  & \quad
  +
  \lim_{m \rightarrow \infty}
  \int_{0}^{\half \pi}
  \frac{\sin (2m+1)\theta}{\sin \theta}
  \phi(\pi - \theta) \dmeasure \theta
  \\
  &
  = \half \pi \thebrace{
    \phi(+0) + \phi(\pi - 0)
  }.
\end{align*}
\end{wandwexample}
\begin{wandwexample}
 Prove that, if $a > 0$,
 $$
 \lim_{m \rightarrow \infty}
 \int_{0}^{\infty}
 \frac{\sin (2n+1)\theta}{\sin \theta}
 e^{-a\theta}
 \dmeasure \theta
 =
 \half \pi \coth \half a \pi.
 $$
 \addexamplecitation{Math. Trip. 1894.}
 [Shew that
 \begin{align*}
   \int_{0}^{\infty}
   \frac{\sin (2n+1)\theta}{\sin \theta}
   e^{-a\theta} \dmeasure \theta
   =&
   \lim_{m \rightarrow \infty}
   \int_{0}^{m \pi}
   \frac{\sin (2n+1)\theta}{\sin \theta}
   e^{-a\theta} \dmeasure \theta
   \\
   =&
   \lim_{m \rightarrow \infty}
   \int_{0}^{\pi}
   \frac{\sin (2n+1)\theta}{\sin \theta}
   \thebrace{
     e^{-a \theta}
     + e^{-a(\theta + \pi)}
     + \cdots
     + e^{-a(\theta + m \pi)}
   } \dmeasure \theta
   \\
   =&
   \int_{0}^{\pi}
   \frac{\sin (2n+1)\theta}{\sin \theta}
   \frac{e^{-a \theta} \dmeasure \theta}{1 - e^{-a\pi}},
 \end{align*}
 and use example l.] %TODO:ref
\end{wandwexample}
\begin{wandwexample}
  Discuss the uniformity of the convergence of Fourier series
  by means of the Dirichlet-Bonnct integrals, without making use of the
  theory of summability.
\end{wandwexample}
%
\Section{9}{5}{The TODO-TODO* theorem concerning Fourier constants.}
\emph{Let $f(x)$ be bounded in the interval $(-\pi, \pi)$ and let
$\int_{-\pi}^{\pi} f(x) \dmeasure x$ exist, so
%TODO:move into section header
* Math. Ann. lvii. (1903), p. 429. Liapounoff discovered the theorem
in 1896 and published it in the Proceedings of the Math. Soc. of the
Univ. of Kharkov. See Comptes Rendw, cxxvi. (1898), p. 1024.
%
% 181
%
that the Fourier constants $a_{n}, b_{n}$ of $f(x)$ exist. Then the series
$$
\half a_{0}^{2}
+
\sum_{n=1}^{\infty} \theparen{
  a_{n}^{2} + b_{n}^{2}
}
$$
is convergent and its sum is\footnote{This integral exists by
  \hardsubsectionref{4}{1}{2} example 1. %TODO:ref
  A proof of the theorem has been given by de la Valine Poussin, %TODO:accent(s)
  in which the sole restrictions on $f(x)$ are that the (improper) integrals of
  $f(x)$ and $\thebrace{f(x)}^{2}$ exist in the interval $(-\pi, \pi)$. See his
  Cours d' Analyse Infinitesimale, ii. (1912), pp. 165-166.}
$$
\frac{1}{\pi}
\int_{-\pi}^{\pi}
\thebrace{
  f(x)
}^{2}
\dmeasure x.
$$}.

It will first be shewn that, with the notation of \hardsectionref{9}{4},
$$
\lim_{m \rightarrow \infty}
\int_{-\pi}^{\pi}
\thebrace{
  f(x)
  -
  \frac{1}{m}
  \sum_{n=0}^{m-1}
  S_{n}(x)
}^{2}
\dmeasure x
=
0.
$$

Divide the interval $(-\pi, \pi)$ into $4r$ parts, each of length $\delta$;
let the upper and lower bounds of $f(x)$ in the interval
$\thebrace{(2p-1)\delta - \pi, (2p+3)\delta - \pi}$
be $U_{p}, L_{p}$, and let the upper bound of $\absval{f(x)}$ in the
interval $(-\pi, \pi)$ be $K$. Then, by
\hardsectionref{9}{4} corollary 1, %TODO:ref
\begin{align*}
  \absval{
    f(x)
    -
    \frac{1}{m}
    \sum_{m=0}^{m-1} S_{n}(x)
  }
  <&
  U_{p} - L_{p} + 2K / \thebrace{m \sin^{2} \half\delta}
  \\
  <&
  2K \thebracket{
    1 + 1/\thebrace{m \sin^{2} \half\delta}
  }
\end{align*}
when $x$ lies between $2p\delta$ and $(2p+2)\delta$.

Consequently, by the first mean-value theorem,
$$
\int_{-\pi}^{\pi}
\thebrace{
  f(x)
  -
  \frac{1}{m}
  \sum_{n=0}^{m-1} S_{n}(x)
}^{2}
\dmeasure x
<
2K
\thebrace{
  1
  +
  \frac{1}{m \sin^{2} \half\delta}
}
\thebrace{
  2\delta
  \sum_{p=0}^{2r-1}
  (U_{p}-L_{p})
  +
  \frac{4Kr}{m \sin^{2} \half \delta}
}
$$

Since $f(x)$ satisfies the Rieraann condition of integrability
(\hardsubsectionref{4}{1}{2}), it follows that both
$4\delta \sum_{p=0}^{r-1} (U_{2p}-L_{2p})$
and
$4\delta \sum_{p=0}^{r-1} (U_{2p+1}-L_{2p+1})$ can be made arbitrarily
small by giving $r$ a sufficiently large value.
When $r$ (and therefore also $\delta$) has been given
such a value, we may choose $m_{1}$ so large that
$r / \thebrace{m_{1} \sin^{2} \half\delta}$ is
arbitrarily small. That is to say, we can make the expression on the
right of the last iiiecpiality arbitrarily small by giving $m$ any value
greater than a determinate value $m_{1}$. Hence the expression on the left
of the inequality tends to zero as $m \rightarrow \infty$.

But evidently
\begin{align*}
  &
  \int_{-\pi}^{\pi}
  \thebrace{
    f(x)
    -
    \frac{1}{m}
    \sum_{n=0}^{m-1} S_{n}(x)
  }^{2} \dmeasure x
  \\
  &\quad
  \int_{-\pi}^{\pi}
  \thebrace{
    f(x)
    -
    \sum_{n=0}^{m-1}
    \frac{m-n}{m}
    A_{n}(x)
  }^{2} \dmeasure x
  \\
  &\quad
  \int_{-\pi}^{\pi}
  \thebrace{
    f(x)
    -
    \sum_{n=0}^{m-1}
    A_{n}(x)
    +
    \sum_{n=0}^{m-1}
    \frac{n}{m}
    A_{n}(x)
  }^{2} \dmeasure x
  \\
  &\quad
  \int_{-\pi}^{\pi}
  \thebrace{
    f(x)
    -
    \sum_{n=0}^{m-1}
    A_{n}(x)
  }^{2} \dmeasure x
  +
  \int_{-\pi}^{\pi}
  \thebrace{
    \sum_{n=0}^{m-1}
    \frac{n}{m}
    A_{n}(x)
  }^{2} \dmeasure x
  \\
  &\quad\quad
  +
  2
  \int_{-\pi}^{\pi}
  \thebrace{
    f(x)
    -
    \sum_{n=0}^{m-1}
    A_{n}(x)
  }
  \cdot
  \thebrace{
    \sum_{n=0}^{m-1}
    \frac{n}{m}
    A_{n}(x)
  } \dmeasure x
  \\
  &\quad
    \int_{-\pi}^{\pi}
    \thebrace{
      f(x)
      -
      \sum_{n=0}^{m-1}
      A_{n}(x)
    }^{2} \dmeasure x
    +
    \frac{\pi}{m^{2}}
    \sum_{n=0}^{m-1}
    n^{2} (a_{n}^{2} + b_{n}^{2}),
\end{align*}
%
% 182
%
since
$$
\int_{-\pi}^{\pi}
f(x) A_{r}(x) \dmeasure x
=
\int_{-\pi}^{\pi}
\thebrace{
  \sum_{n=0}^{m-1}
  A_{n}(x)
}
A_{r}(x)
\dmeasure x
$$
when $r = 0,1,2,\ldots,m-1$.

Since the original integral tends to zero and since it has been proved
equal to the sura of two positive expressions, it follows that each of
these expressions tends to zero; that is to say
$$
\int_{-\pi}^{\pi}
\thebrace{
  f(x)
  -
  \sum_{n=0}^{m-1}
  A_{n}(x)
}^{2} \dmeasure x
\rightarrow
0.
$$

Now the expression on the left is equal to
\begin{align*}
  \int_{-\pi}^{\pi}
  \thebrace{
    f(x)
  }^{2} \dmeasure x
  -&
  2
  \int_{-\pi}^{\pi}
  \thebrace{
    f(x)
    -
    \sum_{n=0}^{m-1}
    A_{n}(x)
  }
  \cdot
  \thebrace{
    \sum_{n=0}^{m-1}
    A_{n}(x)
  }
  \dmeasure x
  \\
  &\quad\quad
  \int_{-\pi}^{\pi}
  \thebrace{
    \sum_{n=0}^{m-1}
    A_{n}(x)
  }^{2}
  \dmeasure x
  \\
  =&
  \int_{-\pi}^{\pi}
  \thebrace{
    f(x)
  }^{2}
  \dmeasure x
  -
  \int_{-\pi}^{\pi}
  \thebrace{
    \sum_{n=0}^{m-1}
    A_{n}(x)
  }^{2}
  \dmeasure x
  \\
  =&
  \int_{-\pi}^{\pi}
  \thebrace{
    f(x)
  }^{2}
  \dmeasure x
  -
  \pi
  \thebrace{
    \half a_{0}^{2}
    +
    \sum_{n=1}^{m-1}
    (a_{n}^{2} + b_{n}^{2})
  }
  ,
\end{align*}
so that, as $m \rightarrow \infty$,
$$
\int_{-\pi}^{\pi}
\thebrace{
  f(x)
}^{2}
\dmeasure x
-
\pi
\thebrace{
  \half a_{0}^{2}
  +
  \sum_{n=1}^{m-1}
  (a_{n}^{2} + b_{n}^{2})
}
\rightarrow
0.
$$
This is the theorem stated.

% TODO:formatting
Corollary. Parseval's theorem\footnote{Mem. par divers senium, i. (1805), pp. 639-648.%TODO:ref
  Parseval, of course, assumed the permissibility of integrating the trigonometrical series
  term-by-term.}. If $f(x), F(x)$ both satisfy the
conditions laid on $f(x)$ at the beginning of this section, and if
$A_{n}, B_{n}$ be the Fourier constants of $F(x)$, it follows by subtracting the pair
of equations which may be combined in the one form
$$
\int_{-\pi}^{\pi}
\thebrace{
  f(x) \pm F(x)
}^{2}
\dmeasure x
=
\pi
\thebrace{
  \half (a_{0} \pm A_{0})^{2}
  +
  \sum_{n=1}^{\infty}
  ((a_{n} \pm A_{n})^{2}
  +
  (b_{n} \pm B_{n})^{2})
}
$$
that
$$
\int_{-\pi}^{\pi}
f(x)F(x)
\dmeasure x
=
\pi
\thebrace{
  \half a_{0}A_{0}
  +
  \sum_{n=1}^{\infty}
  (a_{n}A_{n} + b_{n}A_{n})
}
$$
\Section{9}{6}{Riemann's theory of trigonometrical series.}
The theory of Dirichlet concerning Fourier series is devoted to series
which represent given functions. Important advances in the theory were
made by Riemann, who considered properties of functions defined by a
series of the type\footnote{Throughout \hardsectionref{9}{6}--\hardsubsubsectionref{9}{6}{3}{2} %TODO:cite multiple
  the letters $a_{n}, b_{n}$ do not necessarily denote Fourier constants.}
$\half a_{0} + \sum_{n=1}^{\infty} (a_{n} \cos nx + b_{n} \sin nx)$
where it is assumed that
$\lim_{n \rightarrow \infty} (a_{n} \cos nx + b_{n} \sin nx) = 0$.
We shall give the propositions leading
up to Riemann's theorem\footnote{The proof given is due to G. Cantor, Journal fiir
  Math, lxxii. (1870), pp. 130-142.} %TODO:ref
that if two trigonometrical series converge and
are equal
%
% 183
%
at all points of the range $(-\pi, \pi)$ with the possible exception of a
finite number of points, corresponding coefficients in the two series
are equal.

\Subsection{9}{6}{1}{Riemann's associated function.}
Let the sum of the series
$\half a_{0} + \sum_{n=1}^{\infty} (a_{n} \cos nx + b_{n} \sin nx)
=
A_{0} + \sum_{n=1}^{\infty} A_{n}(x)$,
at any point $x$ where it converges, be denoted by $f(x)$.

Let
$$
F(x) = \half A_{0} x^{2} - \sum_{n=1}^{\infty} n^{-2} A_{n}(x).
$$

\emph{Then, if the series defining $f(x)$ converges at all points of any
finite interval, the series defining $F(x)$ converges for all real
values of $x$.}

To obtain this result we need the following Lemma due to Cantor :

Cantor's lemma\footnote{Riemann appears to have regarded this result as obvious.
  The proof here given is a modification of Cantor's proof,
  Math. Ann. iv. (1871), pp. 139-143, and Journal fiir Math, lxxii. (1870), pp. 130-138.}.%TODO:ref
\emph{If $\lim_{n \rightarrow \infty} A_{n}(x) = 0$ for all values of $x$ such that $a \leq x \leq b$,
then $a_{n} \rightarrow 0, b_{n} \rightarrow 0$.
}

For take two points $x, x+\delta$ of the interval. Then, given $\eps$, we can
find $n_{0}$ such that\footnote{The value of $n_{0}$ depends on $x$ and on $\delta$.},
when $n > n_{0}$
$$
\absval{
  a_{n} \cos nx + b_{n} \sin nx
}
<
\eps,
\quad
\absval{
  a_{n} \cos n(x+\delta) + b_{n} \sin n(x+\delta)
}
<
\eps.
$$

Therefore
$$
\absval{
  \cos n\delta
  (a_{n} \cos nx + b_{n} \sin nx)
  +
  \sin n\delta
  (-a_{n} \sin nx + b_{n} \cos nx)
}
<
\eps.
$$

Since
$$
\absval{
  \cos n\delta
  (a_{n} \cos nx + b_{n} \sin nx)
}
<
\eps,
$$
it follows that
$$
\absval{
  \sin n\delta
  (-a_{n} \sin nx + b_{n} \cos nx)
}
<
2\eps,
$$
and it is obvious that
$$
\absval{
  \sin n\delta
  (a_{n} \sin nx + b_{n} \cos nx)
}
<
2\eps.
$$

Therefore, squaring and adding,
$$
\sqrt{a_{n}^{2} + b_{n}^{2}}
\absval{ \sin n\delta }
<
2\eps\sqrt{2}
$$

Now suppose that $a_{n}, b_{n}$ have not the unique limit $0$; it will be shewn
that this hypothesis involves a contradiction. For, by this
hypothesis, \emph{some} positive number $\eps_{0}$ exists such that there is an
unending increasing sequence $n_{1}, n_{2}, \ldots$ of values of $n$, for which
$$
\sqrt{a_{n}^{2} + b_{n}^{2}} > 4\eps_{0}.
$$

Now let the range of values of $\delta$ be called the interval $I_{1}$ of length
$L_{1}$ on the real axis.

Take $n_{1}'$ the smallest of the integers $n_{r}$ such that
$n_{1}' L_{1} > 2\pi$; then
$\sin n_{1}'y$ goes through all its phases in the interval $I$; call
$I_{2}$
that sub-interval\footnote{If there is more than one such sub-interval,
  take that which lies on the left.}
of $I_{1}$ in which $\sin n_{1}'y > 1/\sqrt{2}$; its length is
$\pi / (2n_{1}') = L_{2}$. Next take $n_{2}'$ the smallest of the integers
$n_{r} (> n_{1}')$
such that $n_{2}' L_{2} > 2\pi$, so that
$\sin n_{2}'y$ goes through all its phases in
the interval $I_{2}$; call $I_{3}$ that sub-interval%TODO:repeat previous footnote
of $I_{2}$ in which $\sin n_{2}'y > 1/\sqrt{2}$;
its length is $\pi / (2 n_{2}') = L_{3}$. We thus get a sequence of
decreasing intervals $I_{1}, I_{2},\ldots$ each contained in all the previous
ones. It is obvious from the definition of an irrational number that
there is a certain point $a$ which is not outside any of these
intervals, and $\sin na \geq 1/\sqrt{2}$ when
$n=n_{1}', n_{2}', \ldots (n_{r+1}' > n_{r}')$.
For these values of $n$,
$\sqrt{a_{n}^{2} + b_{n}^{2}} \sin na > 2\eps_{0} \sqrt{2}$.
But it has been shewn that
corresponding
%
% 184
%
to given numbers $a$ and $\eps$ we can find $n_{0}$ such that when
$n > n_{0}$,
$\sqrt{a_{n}^{2} + b_{n}^{2}} \sin na < 2\eps_{0} \sqrt{2}$;
since some values of $n_{r}'$ are greater than $n_{0}$, the
required contradiction has been obtained, because we may take
$\eps < \eps_{0}$; therefore
$a_{n} \rightarrow 0, b_{n} \rightarrow 0$.

Assuming that the series defining $f(x)$ converges at all points of a
certain interval of the real axis, we have just seen that
$a_{n} \rightarrow 0, b_{n} \rightarrow 0$. Then, for all real values of
$x$,
$\absval{a_{n} \cos nx + b_{n} \sin nx} \leq \sqrt{a_{n}^{2} + b_{n}^{2}} \rightarrow 0$,
and so, by \hardsubsectionref{3}{3}{4}, the
series
$\half A_{0} x^{2} - \sum_{n=1}^{\infty} n^{-2} A_{n}(x) = F(x)$
converges absolutely and uniformly for
all real values of $x$; therefore, (\hardsubsectionref{3}{3}{2}), $F(x)$ is continuous for all
real values of $x$.

\Subsection{9}{6}{2}{Properties of Riemann's associated function;
  Riemann's first lemma.}
It is now possible to prove Riemann's first lemma \emph{that if
  $$
  G(x, \alpha)
  =
  \frac{F(x + 2\alpha) + F(x - 2\alpha) - 2F(x)}{4 \alpha^{2}},
  $$
  then $\lim_{\alpha \rightarrow 0} G(x,\alpha) = f(x)$, provided that
  $\sum_{n=0}^{\infty} A_{n}(x)$ converges for the value
  of $x$ under consideration.}

Since the series defining $F(x), F(x \pm 2\alpha)$ converge absolutely, we may
rearrange them; and, observing that
\begin{align*}
  \cos n(x + 2\alpha) + \cos n(x - 2\alpha) - 2 \cos nx =& -4 \sin^{2} n\alpha \cos nx,\\
  \sin n(x + 2\alpha) + \sin n(x - 2\alpha) - 2 \sin nx =& -4 \sin^{2} n\alpha \sin nx,
\end{align*}
it is evident that
$$
G(x, \alpha)
=
A_{0}
+
\sum_{n=1}^{\infty}
\theparen{
  \frac{\sin n\alpha}{n\alpha}
}^{2}
A_{n}(x).
$$

It will now be shewn that this series converges uniformly with regard
to $\alpha$ for all values of $\alpha$, provided that
$\sum_{n=1}^{\infty} A_{n}(x)$ converges. The result
required is then an immediate consequence of \hardsubsectionref{3}{3}{2}:
for, if
$f_{n}(\alpha) = \theparen{
  \frac{\sin n\alpha}{n\alpha}
}^{2}, (\alpha \neq 0)$
and $f_{n}(0) = 1$, then $f_{n}(\alpha)$ is continuous for all values of $\alpha$, and so
$G(x,\alpha)$ is a continuous function of $\alpha$, and therefore, by \hardsectionref{3}{2},
$G(x, 0) = \lim_{\alpha \rightarrow \infty} G (x, \alpha)$.

To prove that the series defining $G(x, \alpha)$ converges uniformly, we
employ the test given in \hardsubsectionref{3}{3}{5} example 2. %TODO:ref
The expression corresponding to $\omega_{n}(x)$ is
$f_{n}(\alpha)$, and it is obvious that
$\absval{f_{n}(\alpha)} \leq 1$; it is therefore sufficient to shew
that
$\sum_{n=1}^{\infty} \absval{f_{n+1}(\alpha) - f_{n}(\alpha)} < K$,
where $K$ is independent of $\alpha$.

%\begin{smallfont}
In fact\footnote{Since $x^{-1} \sin x$ decreases as $x$ increases from $0$ to $\pi$.},
if $s$ be the integer such that
$a \absval{\alpha} \leq \pi < (s+1) \absval{\alpha}$,
when $\alpha \neq 0$ we have
$$
\sum_{n=1}^{s-1}
\absval{
  f_{n+1}(\alpha) - f_{n}(\alpha)
}
=
\sum_{n=1}^{s-1}
(
f_{n}(\alpha)
??? %TODO
f_{n+1}(\alpha)
)
=
\frac{ \sin^{2} \alpha }{\alpha^{2}}
-
\frac{ \sin^{2} s\alpha }{s^{2} \alpha^{2}}.
$$
%
% 185
%

Also
\begin{align*}
  \sum_{n=s+1}^{\infty}
  \absval{
    f_{n+1}(\alpha) - f_{n}(\alpha)
  }
  =&
  \sum_{n=s+1}^{\infty}
  \absval{
    \thebrace{
      \frac{\sin^{2} n\alpha}{\alpha^{2}}
      \theparen{
        \frac{1}{n^{2}}
        -
        \frac{1}{(n+1)^{2}}
      }
    }
    +
    \frac{\sin^{2} n\alpha - \sin^{2} (n+1)\alpha }{(n+1)^{2} \alpha^{2}}
  }
  \\
  \leq &
  \sum_{n=s+1}^{\infty}
  \frac{1}{\alpha^{2}}
  \theparen{
    \frac{1}{n^{2}}
    -
    \frac{1}{ (n+1)^{2} }
  }
  +
  \sum_{n=s}^{\infty}
  \frac{\absval{
      \sin^{2} n\alpha - \sin^{2} (n+1)\alpha
    }}{ (n+1)^{2} \alpha^{2} }
  \\
  \leq &
  \frac{1}{ (s^{2} + 1)^{2} \alpha^{2} }
  +
  \sum_{n=s+1}^{\infty}
  \frac{\absval{
      \sin \alpha \sin (2n+1)\alpha
    }}{ (n^{2} + 1)^{2} \alpha^{2} }
  \\
  \leq &
  \frac{1}{ (s^{2} + 1)^{2} \alpha^{2} }
  +
  \frac{\absval{\sin \alpha}}{\alpha^{2}}
  \sum_{n=s+1}^{\infty}
  \frac{1}{(n+1)^{2}}
  \\
  \leq &
  \frac{1}{\pi^{2}}
  +
  \frac{\absval{\sin \alpha}}{\alpha^{2}}
  \int_{s}^{\infty}
  \frac{\dmeasure x}{(x+1)^{2}}
  \\
  \leq &
  \frac{1}{\pi^{2}}
  +
  \frac{1}{(s+1) \alpha}.
\end{align*}

Therefore
\begin{align*}
  \sum_{n=1}^{\infty}
  \absval{
    f_{n+1}(\alpha) - f_{n}(\alpha)
  }
  \leq &
  \frac{\sin^{2} \alpha}{\alpha^{2}}
  -
  \frac{\sin^{2} s\alpha}{s^{2}\alpha^{2}}
  +
  \theparen{
    \frac{\sin^{2} s\alpha}{s^{2}\alpha^{2}}
    +
    \frac{\sin^{2} (s+1)\alpha}{(s+1)^{2}\alpha^{2}}
  }
  + \frac{1}{\pi^{2}} + \frac{1}{\pi}.
  \\
  \leq &
  1 + \frac{1}{\pi} + \frac{2}{\pi^{2}}.
\end{align*}

Since this expression is independent of $\alpha$, the result required has
been obtained\footnote{This inequality is obviously true when $\alpha = 0$.}.
%\end{smallfont}

Hence, if $\sum_{n=0}^{\infty} A_{n}(x)$ converges, the series defining $G(x,\alpha)$
converges uniformly with respect to $\alpha$ for all values of $\alpha$, and, as stated above,
$$
\lim_{\alpha \rightarrow \infty} G(x, \alpha)
= G(x,0)
= A_{0} + \sum_{n=1}^{\infty} A_{n}(x)
= f(x).
$$

\begin{wandwexample}
  If
  $H(x, \alpha, \beta) = \frac{F(x+\alpha+\beta) ? %TODO
    F(x+\alpha-\beta) ? %TODO
    F(x-\alpha+\beta) +
    F(x-\alpha-\beta)
  }{4 \alpha \beta}$
  shew that $H(x,\alpha,\beta) \rightarrow f(x)$ when $f(x)$ converges if
  $\alpha,\beta \rightarrow 0$ in such a
  way that $\alpha/\beta$ and $\beta/\alpha$ remain finite.
  \addexamplecitation{Riemann.}
\end{wandwexample}
\Subsubsection{9}{6}{2}{1}{Riemann's second lemma.}
\emph{With the notation of
\hardsectionref{9}{6}--\hardsubsectionref{9}{6}{2}, %TODO:refmultiple
if
$a_{n}, b_{n} \rightarrow 0$, then
$\lim_{\alpha \rightarrow 0} \frac{F(x+2\alpha)+F(x-2\alpha)-2 F(x)}{4\alpha}$ for all values of $x$.
}

For
$\frac{1}{4} \alpha^{-1} \thebrace{ F(x + 2\alpha) + F(x - 2\alpha) - 2 F(x)
  =
  A_{0} \alpha + \sum_{n=1}^{\infty} \frac{\sin^{2} n\alpha}{n^{2}\alpha} A_{n}(x)
}$; but by \hardsubsectionref{9}{1}{1} example 3, %TODO:example ref
if $\alpha > 0$,
$\sum_{n=1}^{\infty} \frac{\sin^{2} n\alpha}{n^{2}\alpha} = \half (\pi - \alpha)$; and so,
since
\begin{align*}
  &
  A_{0}(x) \alpha
  +
  \sum_{n=1}^{\infty}
  \frac{\sin^{2} n\alpha}{n^{2} \alpha} A_{n}(x)
  \\ &
  A_{0}(x) \alpha
  +
  \half (\pi - \alpha) A_{1}(x)
  +
  \sum_{n=1}^{\infty}
  \thebrace{
    \half (\pi - \alpha)
    -
    \sum_{m=1}^{n}
    \frac{\sin^{2} m\alpha}{m^{2} \alpha}
  } \thebrace {
    A_{n+1}(x) - A_{n}(x)
  },
\end{align*}
it follows from \hardsubsectionref{3}{3}{5} example 2, %TODO:ref example
that this series converges uniformly
with regard to $\alpha$ for all values of $\alpha$ greater than, or equal to,
zero\footnote{If we define $g_{n}(\alpha)$ by the equations
  $TODO$, and $TODO$, then $g_{n}(\alpha)$ is continuous when
  $\alpha \geq 0$, and $g_{n+1}(\alpha} \leq g_{n}(\alpha).
}
%
% 186
%

But lim la-i [F x - 2a) + F x - 2a) - 2F(.x)]

= lim

.4o (x)a + 7r-a.)A, (x) + 2 g (a) [An+i (x) - An ix)

w = l

and this limit is the value of the function when a = 0, by | 3"32;
and this value is zero since lim An(x) = 0. By symmetry we see that
lim = lim.

\Subsection{9}{6}{3}{Riemann's theorem* on trigonometrical series.}

Two trigonometrical series which converge and are equal at all j oints
of the range (- \pi, \pi), witli the possible exception of a finite
number of points, must have corresponding coefficients equal.

An immediate deduction from this theorem is that a function of the
type considered in \hardsubsectionref{9}{4}{2} cannot be represented bv auy trigonometrical
series in the range (- \pi, \pi) other than the Fourier series. This
fact was first noticed by Du Bois Reymond.

We observe that it is certainly possible to have other expansions of
(say) the form oq + 2 (a, cos i rax + /3, sin rax),

which represent /(a-) between - \pi and \pi; for write .r=2, and
consider a function c ( ), which is such that ( ( ) =/(2|) when -\ ir
< <\ tv, and (|) =g (|) when - \pi < | < - \pi, and when \pi < < \pi,
where g ( ) is any function satisfying the conditions of \hardsubsectionref{9}{4}{3}. Then
if we expand ( (|) in a Fourier series of the form

0,1 + 2 (a cos m + /3 i cos m ),

this expansion represents /(.r) when - Tr<x<ir; and clearly by
choosing the function g (|) in different ways an unlimited number of
such expansions can be obtained.

The question now at issue is, whether other series- proceeding in
sines and cosines of integral multiples of x exist, which differ from
Fourier's expansion and yet represent / (.r) between - \pi and \pi.

If possible, let there be two trigonometrical series satisfying the
given conditions, and let their difference be the trigonometrical
series

Ao+ t An x)=f(x). =1

Then f(x) = at all points of the range (- \pi, \pi) w th a finite number
of

exceptions; let i, o be a consecutive pair of these exceptional
points, and

let F (x) be Riemann's associated functiou. We proceed to establish a

lemma concerning the value of F(x) when j< x< fo-

\Subsubsection{9}{6}{3}{1}{Schwartz' lemma.}\footnote{TODO} In the range i<x< ., F(x) is a linear
function of x, if f (x) = in this range.

For if = 1 or if <9= -1

cl> ) = e F x)-Fi,)-l F i,)-F(i,) y,hHx-,),-x) is a continuous
function of x in the range i j;,, nd ( i) = ( ( 2) = 0.

* The proof we give is due to Gr. Cantor, Journal fit r Math, lxxii.
(1870), pp. 139-142. + Quoted by G. Cantor, Journal filr Math, i.xxii.
(1870).

%
% 187
%

If the first term of (f) (x) is not zero throughout the range* there
will be some point x=c at which it is not zero. Choose the sign of \$
so that the first term is positive at c, and then choose h so small
that (f) (c) is still positive.

Since cf) (x) is continuous it "attains its upper bound \hardsubsectionref{3}{6}{2}), and
this upper bound is positive since (f> (c) > 0. Let (x) attain its
upper bound at Cj, so that Ci + |i, Ci4= 2  Then, by Riemanifs first
lemma,

lini </'(gi + °) + < (gi- )-20(< i)\ .

But (c'l + a) (f) (c']), (f> (ci-a) - (p (0|), SO this limit must be
negative or zero.

Hence, by supposing that the first term of (x) is not everywhere zero
in the range ( j, 2)5 we have arrived at a contradiction. Therefore it
is zero; and consequently F(x) is a linear function of x in the range
1 < x < 2- li J lemma is therefore proved.

\Subsubsection{9}{6}{3}{2}{Proof of Riemann's Theorem.}
We see that, in the circumstances under consideration, the curve y = F
x) represents a series of segments of straight lines, the beginning
and end of each line corresponding to an exceptional point; and as F
x), being uniformly convergent, is a continuous function of x, these
lines must be connected.

But, by Riemann's second lemma, even if be an exceptional point,

a a a

Now the fraction involved in this limit is the difference of the
slopes of the two segments which meet at that point whose abscissa is
; therefore the two segments are continuous in direction, so the
equation y = F(x) represents a single line. If then we write F(x) =
cx- c, it follows that c and c' have the same values for all values of
. Thus

X

 A x" - ex - c' = S n~-A n (x), =i

the right-hand side of this equation being periodic, with period 27r.

The left-hand side of this equation must therefore be periodic, with
period

27r. Hence

 4,, = (), c = 0,

and - c' = - ir'-A,i(x).

>i = \

Now the right-hand side of this equation converges uniformly, so we
can multiply by cos nx or by sin nx and integrate.

This process gives

7rn~-a,i = - cj cos nxdx = 0,

7rn~ bn= - c j sin nxdx = 0. * If it is zero throughout the range, F
(x) is a linear function of z.

%
% 188
%
Therefore all the coefficients vanish, and therefore the two
trigonometrical series whose difference is J.o+ S An(x) have
corresponding coefficients equal. This is the result stated in \hardsubsectionref{9}{6}{3}.

\Section{9}{7}{Fourier s representation of a function by an integral.}*

It follows from § 9 "43 that, if / ai) be continuous except at a
finite number of discontinuities and if it have limited total
fluctuation in the range (- x, x ), then, if x be any internal point
of the range (- a, /3),

lim' r siii( + l)( - ) y( ) 11 iTT - sin 6 [f x + 26) +/( - W)].

Now let X be any real number, and choose the integer m so that \ = 27n
+ 1 + 277 where t; < 1.

Then sin X t- x) - sin 2m + 1 ) ( - x)] (t - x)-' f (t) dt

J -a.

= I 2 cos (2 i + 1 + t;) ( - x)] . sin rj t - x)] t - x)-'f t) dt

. -a

as ??i- >x by \hardsubsectionref{9}{4}{1} (ii), since (t - x)~' f (t) sin 7 t - x) has
limited total fluctuation.

Consequently, fi'om the proof of the Riemann-Lebesgue lemma of \hardsubsectionref{9}{4}{1},

it is obvious that if I \ f t)' dt and | 1/(0 i converge, thenf

Jo . -00

1" r f(t)di = h f( + )+f( - )]>

A-*-x J -X

 t-x)

and so

lim r \ I cos u (t - x) du\ f(t) dt = \pi f x + 0) +f(x - 0) .

To obtain Fourier's result, we must reverse the order of integration
in this repeated integral.

For any given value of \ and any arbitrary value of e, there exists a
number such that

r\ f(t)idt< el J p

* La Theorie Analytique de la Chaleur, Cb. ix.

t r°° . . . ("

I means the double limit lim I . If this limit exists, it is of course
equal

J -X p -  X, cr -* X J -p

I'-

to lim

p x J -p

%
% 189
%

writing cos u(t- ir) .f(t) = (f> (t, u), we have*

\ r\ I (f>(t, a) dul dt-l \ I (f> t, u) dt] du !

I ' ( J 7 ( . ) '

= j \ t (f) t, a) duldt+ j I (f> t, u)dul dt

-[ \ j <f) t, u) dt\ du - j \ j (f> t,ii)dtldu\ = j \ l 4>(t, u) dul
dt-\ \ I (fi t, u) dt\ du

  P [J ) . fi )

< r \ i''\ < t>(t, u)\ duldt+i'' j' (t>(t, u)[dtdu

< 2\ I \ f t)\ dt<€.

Since this is true for all values of e, no matter how small, we infer
that /oo rk r\ foD /- Qo r\ r\ r- x

=; similarly = .

J a J J J I) J i) Jo J . )

Hence | ir [f x + 0) +/(./; - 0) = lim | cos u (t - oc)f t) dtdu

=

A- . JO. -00

00;" 00

COS u t - oe)f t) dtdu. .' -

This result is known as Fourier s integral tlieorem. Example. Verify
Fourier's integral theorem dinx-tlj (i) fur the function

(ii) for the function defined by the equations

/(.r)=l, -\ < x< ); fix) = 0, (|x|>l). (Rayleigh.j

REFERENCES. G. F. B. Riemann, Ges. Math. Werke, pp. 213-250. E. W.
HoBSON, Functions of a Real Variable (1907), Ch. vii. H. Lebesgde,
Lemons sur les Series Trigonometriques. (Paris, 1906.) C. J. DE LA
VallJ e Poussix, Cours d: Analyse Infinitesimale, ir. (Louvain and
Paris,

1912), Ch. IV.

H. Burkhardt, Ena/clopddie der Math. WUg. ii. 1 (7). (Leipzig, 1914.)

G. A. Carse and G. Shearer, A course in Fourier's analysis and
periodogram analysis

(Edinburgh Math. Tracts, No. 4, 1915).

* The equation I / = I I is easilj' justified by § 4-.3, by
considering the ranges within

which /(x) is continuous.

t For a proof of the theorem when f(x] is subject to less stringent
restrictions, see Hobson, Functions of a Real Variable, %% 492-493.
The reader should observe that, although

Um I I exists, the repeated integral I \ I sin m (( -x) duj- /(<) dt
does not,

A-*oo y -00 7 7 -00 U J

%
% 190
%

Miscellaneous Examples.

1. Obtain the expansions

  ' l-2rcos2 + r2

(6) - log ( 1 - 2r cos z + r ) = - r cos J - '' cos 2 - - r cos 3 - .
. .,

,,, ?'sinz . 1 . 1 . \,

(c) arc tan, = rsins + -?' sin 22 + ?: /' sin 3s + ...,

   r cos 2 2 3

,, 2rsin2 . I,  1 c

(a) arc tan 3- = r sin z- -r sin 3 + - r° sm 02 + . . .,

1 - r'' 3 5

and shew that, when | r | < 1, thej- are convergent for all values of
z in certain strips parallel to the real axis in the -plane.

2. Expand x and x in Fourier sine series valid when - \pi < .r < \pi;
and hence find the value of the sum of the series

sin X - sin 2.r + -3 sin 3*' - - sin 4.r + . . ., for all values of
a.-. \addexamplecitation{Jesus, 1902.}

3. Shew that the function of x I'epresented by 2 ~i sin sin /2a, is
constant

n = \

(0 < X < 2a) and zero (2a < x < \pi), and draw a graph of the function.

\addexamplecitation{Pembroke, 1907.}

4. Find the cosine series representing /(.r) where

/ (x) = sin X + cos X (0 < . ' \pi ),

/(.r) = sin.r-cos.r (in- .r < \pi). \addexamplecitation{Peterhouse, 1906.}

5. Shew that

sin 3n- sinSjTA' sin Ttt., r -,

sin TTX + - - + - - + - - - + ... = Jtt [x],

where [x] denotes 4-1 or - 1 according as the integer next inferior to
x is even or uneven, and is zero if x is an integer. \addexamplecitation{Trinity, 1895.}

6. Shew that the expansions

log and

1

2 cos - X 2

log 2 .sin - X

= cos X-- cos 2x + - cos 3x

- COS X - - COS 2x - - cos 3x

are valid for all real A'alues of x, except multiples of \pi.

7. Obtain the expansion

"(-)"* cos m.r, o M / 1 \ 1 /  T  \

2 ~ - -yT-. = (cos X + cos 2x) log ( 2 cos -x\ +-,:?; (sin zx + sin x)
- cos x,

and find the range of values of . ' for which it is applicable.
\addexamplecitation{Trinity, 1898.}

8. Prove that, if < a- < 27r, then

sin X 2 sin 2x Z sin 3.:*; \ \pi sinh a (\pi - x) a' + Y' " a +2- "a T +
' ' " " 2 ' sinh aTr '

\addexamplecitation{Trinity, 1895.}

FOURIER SERIES

%
% 191
%

9. Shew that between the values -\pi and +n o( x the following
expansions hold

2 . / sin 2 sin 2x 3 sin Zx

sin in.v = - sin ran -. r - -, - + -= - - ...

IT \ V-m'' -l -m 32-7 2

2 . ( \ m cos X m cos 2.r m cos 3 ' cos mx = - sin ??l7r;;- + - - -r
+ - 5 -

TT \ 2m P - 7rt- 22-77l2 32-J/l2

e-mx g-mx 2 / 1 ??IC0S.¥ ?/l COS 2.r 7?2C0S3.1- \

10. Let X be a real variable V)etween and 1, and let n be an odd
number 3. Shew that

(-1)* = - + - 2 -tan - cos2OTn-:r, if is not a multiple of -, where s
is the gi-eatest integer contained in nx; but

 1 2 1 WITT

0= - + - 2 - tan - cos 2m7rx, n 7r, =i i n

if jp is an integer multiple of 1/n. \addexamplecitation{Berger.}

11. Shew that the sum of the series

X

 5 + 47r ~ 2 m~ sin mn cos 2m7rx

m = 1

is 1 when 0<x<l, and when fj<A'<l, and is -1 when l<.r<|.
\addexamplecitation{Trinity, 1901.}

12. If

shew that, when - 1 <x <\,

a" V (x)

ae"'

e" - 1 =o n

cos 2irx+

s47r.r cosB\pij: \ \ -i 2" ""*' " i-

22"

32

2/i !

sin 47ra; sin 6nx "2-" + 1 "*" "32

. -, 8IU 'iTTX Sin OTTX-, .,, -,. .,, .

sm27ra;+-j; +-; +. .. = (-)"+! \ \,, V,,, x).

22" 2 + l

2 + 1 :

\addexamplecitation{Math. Trip. 1896.}

13. If 7?i is an integer, shew that, for all real values of .v,

cos2' a;=2

1.3.5...(2m-

2.4.6 ...2m

1) fl, r,i (2 i + l

m m- )

cos 2x + - - ~ cos 4x

(ot4-1)(7 + 2)

m m-l) m - 2) * (m + l)(?\pi + 2)(w+3)

cos 6.r + ...V,

  \ ., 4 2.4.6...(27ft-2) (1 2w-l . f2m-l)(27n-3),,

r.ns2m-l\;p|\ \ - \ - \ \ \ \ J -!;; + ., COS 2x+-,,,, -;
C0S4a-+.

1.3.5 ...(2 i-l) (2 2/u + l

(2to + 1)(2 h-3)

/

14. A point moves in a straight line with a velocity which is
initially u, and which receives constant increments, each equal to u,
at equal intervals r. Prove that the velocity at any time t after the
beginning of the motion is

u ut V- 1 . 2miit 77 + - + - 2 - sin,

and that the di-stance traversed is

lit

( + ) + T7>-o32 2 -2 cos

2mTTt

\addexamplecitation{Trinity, 1894.}

%
% 192
%

15. If

X 2 = 1

f x)= 2 sin (6?i - 3) j; - 2 2 -sm 2n-l)x

ji=i '2n - 1 71=1 zn - i

3 v'3 f . sin 5.r sin 7.r sin ll r ]

shew that / ( + 0) =/ (\pi - 0) = - \pi,

and /G +0)-/(i7r-0)=-i7r, /(;:|7r+0)-/(37r-0)=*7r.

Observing that the last series is

6 sin (2ft- 1 ) TTsin 2n-l)x

draw the graph of /(* ) \addexamplecitation{Math. Trip. 1893.}

16. Shew that, when <.r < n,

 / N 2 /3/ 1 .1 1 Ti, \

f(,v) = '- I cos .r-- cos 0. '+- cos /.<--- cos 11. r,'-f... ' ' 3 \
/ 11 /

= sin 2x' + - sin 4.f + - sin 8.v + r sin lO.r + . . . 2 4 5

where / ) = i (0< < 7r),

f x) = (l7r<.r<g7r),

f x)=-\ v %7T<X<1t).

Find the sum of each series when .r = 0, \ ir, qtt, n, and for all
other values of x.

\addexamplecitation{Trinity, 1908.}

17. Prove that the locus represented by

2 2 - '' " '*'' ny =

n=l "

is two systems of lines at right angles, dividing the coordinate plane
into squares of area \pi'\ \addexamplecitation{Math. Trip. 1895.}

18. Shew that the equation

" ( - )" ~ sin ny cos nx \

n=i n ~

represents the lines i/= ±mir, (wi = 0, 1, 2, ...) together with a set
of arcs of ellipses whose

semi-axes are \pi and nlslZ, the arcs being placed in squares of area
27r" . Draw a diagram

of the locus. \addexamplecitation{Trinity, 1903.}

19. Shew that, if tlie point x,y, z) lies inside the octahedron
bounded by the planes

±x±y±z = Tr, then

",,, sin 7ia;sin 7jwsin wj 1

 j-Y- - =2-y-

\addexamplecitation{Math. Trip. 1904.}

20. Circles of radius a are diawn having their centres at the
alternate angular jjoints of a regular hexagon of side a. Sliew that
the equation of the trefoil formed by the outer arcs of the circles
can be put in the form

-nr- = -; + - cos 3(9- --= cos 6 + -- - cos9 - ..., 6s'3a 22.4 5.7 8.
10

the initial line being taken to pass through the centre of one of the
circles.

\addexamplecitation{Pembroke, 1902.}

2m . = 1 H sm

%
% 193
%

21. Draw the graph represented by

rr J 1 ( - )" COS nm6\ m [2 a=i 1 - nm)- J ' where m is an integer.
\addexamplecitation{Jesus, 1908.}

22. With each vertex of a regular hexagon of side 2a as centre the arc
of a circle of radius 2a lying within the hexagon is drawn. Shew that
the equation of the figure formed by the six arcs is

f-=6-3,.i + 2Si< -)-' j- f co.6 <>,

4a a=i 6n-l) bn+l)

the prime vector bisecting a petal. \addexamplecitation{Trinity, 1905.}

23. Shew that, if c> 0,

lim f

e ' cot A" sin 2ii + l)x . dx=- n tanh -c\pi.

\addexamplecitation{Trinity, 1894.}

24. Shew that

sin(2 + l) dx 1

lim r

IT O

oth 1.

\addexamplecitation{King's, 1901.}

sin X l+x' 2

25. Shew that, when - 1 < .r < 1 and a is real,

/ '°sin(2n + l) sin(H-.r)(9, 1 sinha.r

hm I -.;; 5 -r, d6= --TT .-, .

  .xjo sm a + 6- 2 smh a

\addexamplecitation{Math. Trip. 1905.}

26. Assuming the possibility of expanding f(x) in a uniformly
convergent series of

the form A -ainkx, where is a root of the equation /-cosa/t-Hft sin a
- = and the

  . . . *

summation is extended to all positive roots of this equation,
determine the constants Jj..

\addexamplecitation{Math. Trip. 1898.}

1 ""

27. If f(x) = -a + 2 a cosnx + b,t>im nx)

2,(=1

is a Fourier series, shew that, iif x) satisfies certain general
conditions,

a = - P i f (t) cos nt ta.n - t --, 6 = - / f t)smnttn,n-t - .

7T J ' 'It ""yO 'It

\addexamplecitation{Beau.}

W. M. A. 13

%
% 194
%
\chapter{Linear Differential Equations}

\Section{10}{1}{Linear Differential Equations\footnote{The analysis contained in this chapter is mainly theoretical; it
    consists, for the most part, of existence theorems. It is assumed that
    the reader has some knowledge of practical methods of solving
    differential equations; these methods are given in works exclusively
    devoted to the subject, such as
    Forsyth, A Treatise on Differential Equations (1914). %TODO:formatting
  }.
  Ordinary points and singular points.}

In some of the later chapters of this work, we shall be concerned with
the investigation of extensive and important classes of functions
which satisfy linear differential equations of the second order.
Accordingly, it is desirable that we should now establish some general
results concerning solutions of such differential equations.

The standard form of the linear differential equation of the second
order will be taken to be
\begin{equation}
  \frac{\dmeasure^{2} u}{\dmeasure z^{2}}
  + p(z) \frac{\dmeasure u}{\dmeasure z}
  + q(z) u
  = 0
\end{equation}
and it will be assumed that there is a domain $S$ in which both
$p(z), q(z)$ are analytic except at a finite number of poles.

Any point of $S$ at which $p(z), q(z)$ are both analytic will be called
an \emph{ordinary point} of the equation; other points of $S$ will be called
\emph{singular points}.

\Section{10}{2}{Solution\footnote{This method is applicable only to equations of the second order.
    For a method applicable to equations of any order, see
    Forsyth, Theory of Differential Equations, iv. (1902), Ch. I. %TODO:formatting
    } of a differential equation valid in the vicinity of an ordinary point.}

Let $b$ be an ordinary point of the differential equation, and let $S_{b}$ be
the domain formed by a circle of radius $r_{b}$, whose centre is $b$, and its
interior, the radius of the circle being such that every point of $S_{b}$ is
a point of $S$, and is an ordinary point of the equation.

Let $z$ he a variable point of $S_{b}$.

In the equation write
$u = v \exp \thebrace{
  - \half \int_{b}^{z} p(\zeta) \dmeasure \zeta
}$,
and it becomes
\begin{equation}
  \frac{\dmeasure^{2} v}{\dmeasure z^{2}}
  + J(z) v
  = 0
\end{equation}
where
$$
J(z)
=
q(z)
- \half \frac{\dmeasure p}{\dmeasure z}
- \frac{1}{4} \thebrace{p(z)}^{2}.
$$
%
% 195
%

It is easily seen (\hardsubsectionref{5}{2}{2}) that an ordinary point of equation (A) %TODO:ref
is also an ordinary point of equation (B). %TODO:ref

Now consider the sequence of functions $v_{n}(z)$, analytic in $S_{b}$,
defined by the equations
\begin{align*}
  v_{0}(z)
  =& a_{0} + a_{1} (z-b),
  \\
  v_{n}(z)
  =&
  \int_{b}^{z} (\zeta - z) J(\zeta) v_{n-1}(\zeta) \dmeasure \zeta,
  \quad
  (n=1,2,3,\ldots)
\end{align*}
where $a_{0}, a_{1}$ are arbitrary constants.

Let $M, \mu$ be the upper bounds of $\absval{J(z)}$ and $\absval{v_{0}(z)}$ in the domain
$S_{b}$. \emph{Then at all points of this domain}
$$
\absval{v_{n}(z)}
\leq
\mu M^{n} \absval{z-b}^{2n} / (n!).
$$

For this inequality is true when $n=0$; if it is true when $n=0,1,\ldots,m-1$,
we have, by taking the path of integration to be a straight
line,
\begin{align*}
  \absval{v_{m}(z)}
  &=
  \absval{
    \int_{b}^{z} (\zeta-z) J(\zeta) v_{m-1}(\zeta) \dmeasure \zeta
  }
  \\
  &\leq
  \frac{1}{(m-1)!}
  \int_{b}^{z}
  \absval{\zeta-z} \cdot \absval{J(\zeta)}
  \mu M^{m-1}
  \absval{\zeta-b}^{2m-2} \cdot \absval{\dmeasure \zeta}
  \\
  &\leq
  \frac{1}{(m-1)!}
  \mu M^{m} \absval{z-b}
  \int_{0}^{\absval{z-b}} t^{2m-2} \dmeasure t
  \\
  &<
  \frac{1}{m!} \mu M^{m} \absval{z - b}^{2m}.
\end{align*}

Therefore, by induction, the inequality holds for all values of $n$.

Also, since $\absval{v_{n}(z)} \leq \frac{\mu M^{n} r_{b}^{2n}}{n!}$ when $z$ is in $S_{b}$, and
$\sum_{n=0}^{\infty} \mu M^{n} r_{b}^{2n}/(n!)$ converges,
it follows (\hardsubsectionref{3}{3}{4}) that
$v(z) = sum_{n=0}^{\infty} v_{n}(z)$ is a series of analytic functions
uniformly convergent in $S_{b}$; while, from the definition of $v_{n}(z)$,
\begin{align*}
  \frac{\dmeasure}{\dmeasure z} v_{n}(z)
  &= -\int_{b}^{z} J(\zeta) v_{n-1}(\zeta) \dmeasure \zeta,
  \quad (n = 1,2,3,\ldots)
  \\
  \frac{\dmeasure^{2}}{\dmeasure z^{2}} v_{n}(z)
  &= -J(z) v_{n-1}(z);
\end{align*}
hence it follows (\hardsectionref{5}{3}) that
\begin{align*}
  \frac{\dmeasure^{2} v(z)}{\dmeasure z^{2}}
  =&
  \frac{\dmeasure^{2} v_{0}(z)}{\dmeasure z^{2}}
  +
  \sum_{n=1}^{\infty} \frac{\dmeasure^{2} v_{n}(z)}{\dmeasure z^{2}}
  \\
  =&
  - J(z) v(z).
\end{align*}

\emph{Therefore $v(z)$ is a function of $z$, analytic in $S_{b}$, which satisfies
the differential equation
$$
\frac{\dmeasure^{2} v(z)}{\dmeasure z^{2}} + J(z) v(z) = 0,
$$
%
% 196
%
and, from the value obtained for
$\frac{\dmeasure}{\dmeasure z} v_{n}(z)$, it is evident that
$$
v(b) = a_{0},
\quad
v'(b) = \thebrace{\frac{\dmeasure}{\dmeasure z}v(z)}_{z=b} = a_{1},
$$
where $a_{0}, a_{1}$ are arbitrary.}

\Subsection{10}{2}{1}{Uniqueness of the solution.}

If there were two analytic solutions of the equation for $v$,
say $v_{1}(z)$ and $v_{2}(z)$ such that
$v_{1}(b) = v_{2}(b) = a_{0}, v_{1}'(b) = v_{2}'(b) = a_{1}$, then,
writing $w(z) = v_{1}(z) - v_{2}(z)$, we should have
$$
\frac{\dmeasure^{2} w(z)}{\dmeasure z^{2}} + J(z) w(z) = 0.
$$

Differentiating this equation $n-2$ times and putting $z = b$, we get
$$
w^{(n)}(b) + J(b) w^{(n-2)}(b) + TODO
+ \cdots + J^{(n-2)}(b) w(b) = 0.
$$
Putting $n = 2, 3, 4, \ldots$ in succession, we see that all the
differential coefficients of $w(z)$ vanish at $b$; and so, by Taylor's
theorem, $w(z) = 0$; that is to say the two solutions
$v_{1}(z), v_{2}(z)$ are identical.

Writing
$$
u(z) = v(z) \exp \thebrace{
  - \half \int_{b}^{z} p(\zeta) \dmeasure \zeta
},
$$
we infer without difficulty that $u(z)$ is the only analytic solution
of (A) %TODO:ref
such that $u(b) = A_{0}, u'(b) = A_{1}$, where
$$
A_{0} = a_{0},
\quad
A_{1} = a_{1} - \half p(b) a_{0}.
$$

Now that we know that a solution of (A) %TODO:ref
exists which is analytic in $S_{b}$,
and such that $u(b), u'(b)$ have the arbitrary values
$A_{0}, A_{1}$, the simplest method of obtaining the
solution in the form of a Taylor's series is to assume
$u(z) = \sum_{n=0}^{\infty} A_{n} (z-b)^{n}$, substitute this
series in the differential equation and
equate coefficients of successive powers of $z-b$ to zero
(\hardsubsectionref{3}{7}{3}) to
determine in order the values of
$A_{2}, A_{3}, \ldots$ in terms of $A_{0}, A_{1}$.

%\begin{smallfont}
[Note. In practice, in carrying out this process of substitution, the
re;vder will find it much more simple to have the equation 'cleared of
fractions' rather than in the canonical form (A) %TODO:ref
of \hardsectionref{10}{1}. Thus the
equations in examples 1 and 2 %TODO:ref
below should be treated in the form in
which they stand; the factors
$1 - z^{2}, (z-2)(z-3)$
should \emph{not} be
divided out. The same remark applies to the examples of
\hardsectionref{10}{3}, \hardsubsectionref{10}{3}{2}.] %TODO:multiref
%\end{smallfont}

From the general theory of analytic continuation (\hardsectionref{5}{5}) it follows
that the solution obtained is analytic at all points of $S$ except at
singularities of the differential equation. The solution however is
\emph{not}, in general, 'analytic throughout $S$'
(\hardsectionref{5}{2} cor. 2, footnote), %TODO:refs
except at these points, as it may not be one-valued; i.e. it may not
return to the same value when $z$ describes a circuit surrounding one or
more singularities of the equation.
%
% 197
%

[The property that the solution of a linear differential equation is
analytic except at singularities of the coefficients of the equation
is common to linear equations of all orders.]

When two particular solutions of an equation of the second order are
not constant multiples of each other, they are said to form a
\emph{fundamental system}.

\begin{wandwexample}
Shew that the equation
$$
(1 - z^{2}) u'' - 2 z u' + \frac{3}{4} u = 0
$$
has the fundamental system of solutions
\begin{align*}
  TODO
\end{align*}

Determine the general coefficient in each series, and shew that the
radius of convergence of each series is $1$.
\end{wandwexample}
\begin{wandwexample}
Discuss the equation
$$
(z-2)(z-3) u'' - (2z-5) u' + 2u = 0
$$
in a manner similar to that of example 1.
\end{wandwexample}
%
\Section{10}{3}{Points which are regular for a differential equation.}
Suppose that a point $c$ of $S$ is such that, although $p(z)$ or $q(z)$ or
both have poles at $c$, the poles are of such orders that
$(z - c) p(z)$, $(z- c)^{2} q(z)$ are analytic at $c$.
Such a point is called a \emph{regular point}\footnote{The name
  `regular point' is due to
  Thome, Journal fiir Math. lxxv. (1873), p. 266. %TODO:ref
  Fuchs had previously used the phrase ' point of determinateness.'}
for the differential equation. Any poles of $p(z)$ or of $q(z)$ which are
not of this nature are called \emph{irregular points}. The reason for making
the distinction will become apparent in the course of this section.

If $c$ be a regular point, the equation may be written\footnote{Frobenius
  calls this the normal form of the equation.}
$$
(z-c)^{2} \frac{\dmeasure^{2} u}{\dmeasure z^{2}}
+ (z-c) P(z-c) \frac{\dmeasure u}{\dmeasure z}
+ Q(z-c) u = 0,
$$
where $P(z - c), Q(z - c)$ are analytic at $c$; hence, by Taylor's
theorem,
\begin{align*}
  TODO
\end{align*}
where $p_{0}, p_{1}, \ldots, q_{0}, q_{1}, \ldots$ are constants; and these series
converge in the domain $S_{c}$. formed by a circle of radius $r$ (centre $c$)
and its interior, where $r$ is so small that $c$ is the only singular
point of the equation which is in $S_{c}$.

Let us assume as - formal solution of the equation

u = z- cY

1 + S an (z - cr

where a, a, a.,, ... are constants to be determined.

%
% 198
%

Substituting in the differential equation (assuming that the
term-by-term differentiations and multiplications of series are
legitimate) we get

(z-cY

a(a-l)-l- S an a + 7i)(a + n-l)(z-cY

+ z-cYP z-c).

a+ 1 an(a + n)(z-cy'

+ z-cYQ z-c)

1+ 1 ttniz-cT

= 0.

Substituting the series for P(z - c), Q z - c), multiplying out and
equating to zero the coefficients of successive powers of z - c, we
obtain the following sequence of equations :

oi- + h-l)a + qo = 0,

, (a + iy + (po-'i-)(o( + l) + qo + ap, + q, = 0, a, (a + 2y + (po -
1) ( + 2 ) + o] + a, (a + l)p, + q + ap, + q, = 0,

an (a -1- nf + (p,-l) ci-]- n) + qo]

+ 2 Qn-m ( + n - m) p,n + ?, + dpn + qn = 0. m=l

The first of these equations, called the indicial equation*,
determines two values (which may, however, be equal) for a. The reader
will easily convince himself that if c had been an irregular point,
the indicial equation would have been (at most) of the first degi'ee;
and he will now appreciate the distinction made between regular and
irregular singular points.

Let a = pi, a = p.2 be the roots i* of the indicial equation F(a) = a?
+ po-l)oi + q, = 0; then the succeeding equations (when a has been
chosen) determine a, a, ..., in order, uniquely, provided that F a +
n) does not vanish when ?i = 1, 2, 3, ...; that is to say, if a = pi,
that p is not one of the numbers TODO; and, if a =
p.., that p is not one of the numbers p - - 1, p., + 2,

Hence, if the difference of the exponents is not zero, or an integer,
it is always possible to obtain two distinct series which formally
satisfy the equation.

Example. Shew that, if ra is not zero or an integer, the equation

is formally satisfied by two series whose leading terms are

.*-' jl+.

-+

.:,

16(H-to) 7' " i ' 16(l-m)

determine the coefficient of the general term in each series, and shew
that the series converge for all values of z.

* The name is due to Cayley, Quarterly Journal, xxi. (1886), p. 326.

t The roots pi, po of the indicial equation are called the expotients
of the differential equation at the point c.

%
% 199
%

\Subsection{10}{3}{1}{Convergence of the expansion o/§ lO'S.}

If the exponents TODO, p are not equal, let p- be that one whose real
part is not inferior to the real part of the other, and let
TODO
then
TODO.
Now, by \hardsubsectionref{5}{2}{3}, we can find a positive number M
such that \ pn\ < Mr-\ \ qn\ < Mr-'\ p.j n + qn i < Mr-'\ where M is
independent of n; it is convenient to take if 1. Taking a = pi, we
see that

aAJMl±,<

M

<

M

\ F p, + ) \ r s+1 since |5 + 1 1 >1.

If now we assume 1 a,,, I < i/"r-" when n=\,2, ...m- 1, we get

ttm. -

2 a -t (pi + vi-t)pi-it qt] + piPm + qr.

F(p, + m)

t I a, t \ piPt + qt] + \ piPm + q,a ! + s (m - 1 - I \ vt

m- 1

m\ 8 +in\

m,M' r-"'+ \ 2 m-t)\ M' r

']

m" 1 1 + smr |

Since 1 1 + snr j 1, because R (s) is not negative, we get

m + 1

a, <

2m

jpn,.-m < j,/' r-'",

and so, by induction, | a i < M' r'" for all values of ?2.

If the values of the coefficients corresponding to the exponent po be
a (iz, ... we should obtain, by a similar induction,

\ an\ < M'Wr-'\

where k is the upper bound of |1- s|~\ \ l - s\~\ 1 1 - |-s |~S ...;
this bound exists when s is not a positive integer.

We have thus obtained two formal series

Wi (z) = z - C)P'

 2 (2 ) = - CY

1+ S aniz-cr

M = 1

1+ ian(z-cY

The first, however, is a uniformly convergent series of analytic
functions when \ z-c\ < rM~\ as is also the second when \ z-c\ < rM-'
K-\ provided

%
% 200
%

in each case that arg (z - c) is restricted in such a way that the
series are one-valued; consequently, the formal substitution of these
series into the left-hand side of the differential equation is
justified, and each of the series is a solution of the equation;
provided always that pi - p., is not a positive integer or zero*.

With this exception, we have therefore obtained a fundamental S3 stem
of solutions valid in the vicinity of a regular singular point. And by
the theory of analytic continuation, we see that if all the
singularities in S of the equation are regular points, each member of
a pair of fundamental solutions is analytic at all points of S except
at the singularities of the equation, which are branch- points of the
solution.

\Subsection{10}{3}{2}{Derivation of a second solution in the case when
  the difference of the exponents is an integer or zero.}

In the case when p - po, = s is a positive integer or zero, the
solution Wo z) found in § lO'Sl may break downf or coincide with Wi
z).

If we write u = w z), the equation to determine f is

(z - cy -r + \ 2 z - cf '-- - + (z-c)Piz-c) = 0, . dz- [ u ( z) ) dz

of which the general solution is

 = A-rB\ ~ - - ry- exp - / - - dz. dz

J W, Z)Y i J 2-C ]

= A+B

(z-c)-P°

= A+Bj (z- c)-P - p g (z) dz,

where A, B are arbitrary constants and g z) is analytic throughout the
interior of any circle whose centre is c, which does not contain any
singu- larities of P ( - c) or singularities or zeros of z - c)"''i w
z); also g (c) = 1.

00

Let g z)= + S gn.(z-cy\

n = l

Then, Hsi O,

 =A+Br\ l+i(z-cf ( z-c)-'-'dz

J [ w = l J

= A+B --(z- c)- - 'i - (z - cy-' + gs log (z - c)

\ S n=l n

+ % - (z - cY-' n = s+l n -s

* If Pi -p2 is a positive integer, k does not exist; if p = p2, the
two solutions are the same.

t The coefficient a/ may be indeterminate or it may be infinite; in
the former case to-, (z) will be a solution containing two arbitrary
constants qq' and aj; the series of which Ug is a factor will be a
constant multiple of wi (z).

%
% 201
%

Therefore the general solution of the differential equation, which is
analytic at all points of C (c excepted), is

Awi z) + B [gsti\ (z) log (z-c) + w (z)], where, by \hardsubsectionref{2}{5}{3}, w (z) = z
- c) p- \ - - + S h,, z - c)"l,

the coefficients hn being constants.

When s = 0, the corresponding form of the solution is

Aw,(z) + B w, (z) log (z-c) + (z- cY X hn (z - c)"

L n = l \

The statement made at the end of \hardsubsectionref{10}{3}{1} is now seen to hold in the
exceptional case Avhen s is zenj or a positive integer.

In the special case when gg = 0, the second solution does not involve
a logarithm.

The solutions obtained, which are valid in the vicinity of a regular
point of the equation, are called regular integrals.

Integrals of an equation valid near a regular point c may be obtained
practically by first obtaining w- z), and then determining the
coefficients in

30

a function u\ (z) = S hn z - c)''-+'*, by substituting lu (z) log (z -
c) + n\ z) in

the left-hand side of the equation and equating to zero the
coefficients of the various powers of z - c in the resulting
expression. An alternative method due to Frobenius* is given by
Forsyth, Treatise on Differential Equations, pp. 243-258.

Example L Shew that integrals of the equation

<Pu \ du -

regular near z - are

Wi z) = l+ 2

and ..,(.) log.- 2 . ( \ + \ +...+-j.

Verify that these series converge for all vahies of z. Example 2. Shew
that integrals of the equation

,d u, -.sdu 1

regular near 2 = are

'°'W = + .!.( 2.4...2,. )-

and .,(.)log. + 4 j( \ - - j ( \ \ + \ \ ...\ \ J..

Verify that these series converge when | . i < 1 and obtain integrals
regular near 3= L * Journal fur Math, lxxvi. (1874), pp. 214-224.

%
% 202
%

Example 3. Shew that the hypergeometric equation

z z) :i + c-(a->rh + ) z -ahu = Q dz dz

is satisfied by the hypergeometric series of \hardsubsectionref{2}{3}{8}.

Obtain the complete sohition of the equation when c = .

\Section{10}{4}{Solutions valid for large values of $z$.}

Let = l/5i; then a solution of the differential equation is said to be
valid for ' large values of | | ' if it is valid for sufficiently
small values of Ui |; and it is said that ' the point at infinity is
an brdinary (or regular or irregular) point of the equation ' when the
point j = is an ordinary (or regular or irregular) point of the
equation when it has been transformed so that Zi is the independent
variable.

Since

we see that the conditions that the point 2= oc should be (i) an
ordinary point, (ii) a regular point, are (i) that 2z - z~p 2), z*q z)
should be analytic at infinity \hardsubsectionref{5}{6}{2}) and (ii) that zp z), z-q (z)
should be analytic at infinity.

Example 1. Shew that every point (including infinity) is either an
ordinary point or a regular point for each of the equations

z l-z)-jj ->r c- a-vh + ) z - -ahu = 0,

(1- 2), - -2s +?i(?i + l) = 0,

where a, \&, c, n are constants.

Exam-pie 2. Shew that every point except infinity is either an
ordinary point or a regular point for the equation

 S-"+4"- ( '- ') =°'

where n is a constant.

Example 3. Shew that the equation

has the two solutions

,, .,, d' u du .

2\ 1 i i 3.4.5.6 1

 " 3' 7 ' 277 7 ' 2. 4. 7. 9 z

the latter converging when | s | > 1.

\Section{10}{5}{Irregular singularities and confluence.}

Near a point which is not a regular point, an equation of the second
order cannot have two regular integrals, for the indicial equation is
at most of the first degree; there may be one regular integral or
there may be none. We shall see later (e.g.\hardsectionref{16}{3}) what is the nature
of the solution near

%
% 203
%

such points in some simple cases. A general investigation of such
solutions* is beyond the scope of this book.

It frequently happens that a differential equation may be derived from
another differential equation by making two or more singularities of
the latter tend to coincidence. Such a limiting process is called
confluence; and the former equation is called a confluent form of the
latter. It will be seen in § lO'G that the singularities of the former
equation may be of a more complicated nature than those of the latter.

\Section{10}{6}{The differential equations of mathematical physics.}

The most general differential equation of the second order which has
every point except a, a, as, a and oo as an ordinary point, these
five points being regular points with exponents a, ySr at ar r = 1,
2, 3, 4) and exponents fjbj, /j,2 at X, may be verified f to be

dz'' Vti z-ar )dz \ rZi z-ayf T (z-ar)

r = \

where A is such that;): / i and i., are the roots of

and B, C are constants.

The remarkable theorem has been proved by Klein§ and B6cher|| that all
the linear differential equations which occur in certain branches of
Mathematical Physics are confluent forms of the special equation of
this type in which the difference of the two exponents at each
singularity is |; a brief investigation of these forms will now be
given.

If we put y9,. = a,. +, (r = l, 2, 8, 4) and write in place of z, the
last written equation becomes

d i U h- Adu (4 ar(ar + h) 1 ? + 2 r+C ] \ Q

r=l

* Some elementary investigations are given in Forsyth's Differential
Equations (1914). Complete investigations are given in his Theory of
Differential Equations, iv. (1902).

t The coefficients of - and u must be rational or they would have an
essential singularity dz

4 4

at some point; the denominators of p z), q z) must be H z-a ), n z-a )
respectively;

r=l J-=l

putting p(z) and q (z) into partial fractions and remembering that p
z) = 0(z- ), q z) - 0(z~-) as 1 2 - 00, we obtain the required result
without difficulty.

4

X It will be observed that mi, fJ-i are connected by the relation
M1+M2+ 2 (a,. + /3 ) = 3.

j=i

§ Ueher lineare Differentialgleichungen der zweiter Ordnung (1894), p.
40; see also Vorlesung

Uber Lamd schen Funktionen.

II Ueber die Reihenentwickelungen der Potentialtheorie (1894), p. 193.

%
% 204
%

where (on account of the condition yu.., - f i = 2)

\ r=l

This differential equation is called the generalised \\Lame\\ equation.

It is evident, on writing i = ao throughout the equation, that the
confluence of the two singularities a, a., yields a singularity at
which the exponents a, /3 are given by the equations

a + /3 = 2 ( ! + oo), a = a, (oc, + h) + a, (a, + |) + D,

where D = (Aa - + 2Ba + C)/ (ai - a. ) (a - a ) .

Therefore the exponent-difference at the confluent singularity is not
, but it may have any assigned value by suitable choice of B and C. In
like manner, by the confluence of three or more singularities, we can
obtain one irregular singularity.

By suitable confluences of the five singularities at our disposal, we
can obtain six types of equations, which may be classified according
to (a) the number of their singularities with exponent-difference |,
(6) the number of their other regular singularities, (c) the number of
their irregular singu- larities, by means of the following scheme,
which is easily seen to be exhaustive*:

(a)

(b)

(c)

(I)

3

1

Lame

(H)

2

1

Mathieu

(III)

1

2

Les:eudre

(IV)

1

1

Bessel

(V)

1

1

Weber, Hermite

(VI)

1

Stokes t

These equations are usually known by the names of the mathematicians
in the last column. Speaking generally, the later an equation comes in
this scheme, the more simple are the properties of its solution. The
solutions of (II)-(YI) are discussed in Chapters xv-xix of this work,
and J of (I) in Chapter xxiii. The derivation of the standard forms of
the equations from the generalised \\Lame\\ equation is indicated by the
following examples :

* For instance the arrangement (a) 3, h) 0, (c) 1 is inadmissible as
it would necessitate six initial singularities.

t The equation of this type was considered by Stokes in his researches
on Diffraction, Camb. Phil. Trans, ix. (1856), pp. 168-182; it is,
however, easily transformed into a particular case of Bessel's
equation (example 6, below).

t For properties of equations of type (I), see the works of Klein and
Forsyth cited at the end of this chapter; also Todhunter, The
Functions of Laplace, \\Lame\\ and Bessel (1875).

%
% 205
%

Example . Obtain \Lame's equation

r=l

(where h and n are constants) by taking

n, =0 = 03 = 04 = 0, 8B = n('/i- l) a, 4(7=/i 4, and making 04 - x .

Example 2. Obtain the equation

dc' \ rc- y dc 4c (c- 1) ~ '

(where a and j are constants) by taking ai = 0, 0-2 = 1, and making 3
= c/4 -x. Derive Mathieu's equation \hardsectionref{19}{1})

-7 + (a + 16jcos 2i) =

by the substitution f = cos 2.

Example 3. Obtain the equation

 *4./K\ JLW, 1 f!L( L)\ \ !!!!\ l \ \ n

c r " (r c- 1/ c/c * I f f - ij c (c- 1) ~ '

by taking

 i = 2=l, 3 = a4 = 0, ai = a.j = a3 = 0, a4 = |.

Derive Legendre's equation (§§ 15-13, 15'5)

by the substitution ( = z~\

Example 4. By taking, = (/.2 = 0, 01 = 02 = 03 = 04 = 0, and making
a3 = a4-*-x, obtain the equation

Derive Bessel's equation (\hardsubsectionref{17}{1}{1})

d u

dz " dz

gd u du,,

by the substitution C=z' .

Example 5. By taking i=0, 01 = 02 = 03 = 04 = 0, and making (/o = 3 =
a4-*-x, obtain the equation

. dhi du,,,,,

f +i5 +i( +*-if) =o.

Derive Weber's equation \hardsectionref{16}{5})

d u,,, x

  + ( +*- -') =o

by the substitution f= 2.

Example 6. By taking 0 =0, and making 0 -9- x ( = 1, 2, 3, 4), obtain
the equation

  + ( if+Ci) = 0. By taking

u = B,C+C\ r-v, B,C+C\ = IB,z), shew that

 d v dv,,,.

%
% 206
%

Example 7. Shew that the general form of the generalised \\Lame\\ equation
is un- altered (i) by any homographic change of independent variable
such that qo is a singular point of the transformed equation, (ii) by
any change of dependent variable of the type

'll = z-ar) V.

Example 8. Deduce from example 7 that the various confluent forms of
the generalised \Lame\ equation may always be reduced to the forms given
in examples 1-6.

[Note that a suitable homographic change of variable will transform
any three distinct points into the points 0, 1, qo .]

\Section{10}{7}{Linear differential equations ivith three singularities.}

d?u, du, . \ Let d ' dz' '

have three, and only three singularities*, a, h, c; let these points
be regular points, the exponents thereat being a, a!; /3, /3'; 7,
7'.

Then p z) is a rational function with simple poles at a, b, c, its
residues at these poles being 1 - a - a', 1 - /3 - /S', 1 - 7 - 7';
and as 2 co, p z)- 2z- is z- ). Therefore

  z - a z - z - c

andf a + a' + + y8' + 7 + 7' = 1.

In a similar manner

,, [aoL a-h) a-c) B 'ih-c) h-a) yy' (c - a) c - b) l z - a z - b z - c

1

X

 z - a) z- b) z - g) and hence the differential equation is

d-'u U- o- ' 1-/3-/3 1-7-7 ) du

dz' [ z - a z - b z - c \ dz

  oia' a-b ) (a-c) /3 ' (b -c)(b-a) 77 (c -a)(c- b) \ z - a z - b z~c
I

= 0.

 z- a) z - b) z- c) This equation was first given by Papperitzj,

To express the fact that u satisfies an equation of this type (which
will be called Riemann's P-equation), Riemann§ wrote

fa b c \ u = p a /3 7 z[. W /3' 7 J

* The point at infinity is to be an ordinary point.

t This relation must be satisfied by the exponents.

+ Math. Ann. xxv. (1885), p. 21B.

§ Abh. d. k. Ges. d. Wiss. zu Gottingen, vii. (1857). It wiU be seen
from this memoir that, although Kiemann did not apparently construct
the equation, he must have inferred its existence from the
bypergeometric equation.

%
% 207
%

The singular points of the equation are placed in the first row with
the corresponding exponents directly beneath them, and the independent
variable is placed in the fourth column.

Example. Shew that the hypergeometric equation

d-u,, .,, . c?

,,, d-u,, 7,, . du,

is defined by the scheme

j' 00 1 \

pJ a 2-.

[l-o h c - a-h j

10 71. Transformations of Riemanns P-equation. The two transformations
which are typified by the equations

a b

z-h

(I) iv l) [v k) z\ = P a -k -k-l y- l

; (

b

c

7 W

y3

7

a'

/3'

i

a

b

c

(II) P<a

a'



7

IS'

7'

c

a' + k '-k-l y'+l

J (a /5' 7' J

(where z, a, 61, Cj are derived from z, a, b, c by the same
homographic transformation) are of great importance. They may be
derived by direct transformation of the differential equation of
Papperitz and Riemann by suitable changes in the dependent and
independent variables respectively; but the truth of the results of
the transformations may be seen intuitively when we consider that
Riemann's P-equation is determined uniquely by a knowledge of the
three singularities and their exponents, and (I) that if
$$
TODO
$$
IS'
$$
TODO
$$
then Ui = i- -jj (-37) satisfies a differential equation of the second

order with the same three singular points and exponents a + k, a +k;
l3 - k - l,j3' - k - l; y+ I, y' + I; and that the sum of the
exponents is 1,

Az + B Also (II) if we write z = j -, the equation in z- is a linear
equation

of the second order with singularities at the points derived from a,
b, c by this homographic transformation, and exponents a, a; fS, ';
y, y thereat.

%
% 208
%

\Subsection{10}{7}{2}{The connexion of Riemanns P-equation with the hypergeometric equation.}

B) means of the results of § lO'Tl it follows that

 a - a /3 + a + y 7' - 7 J

where x =

a' - a /3' + a + 7 7' - 7 (z - a)(c- b)

(z - h) c - a)'

Hence, by \hardsectionref{10}{7} example, the solution of Riemann's P-equation can

always be obtained in terms of the solution of the hypergeometric
equation

1 1,7 r n,' -1, z - a) c- ))

whose elements a, 0, c, x are a+ + 7, a-t-p+7, i+a - a, j - 7 -

 z-b) c - a)

respectively.

\Section{10}{8}{Linear differential equations tuith two singularities.}

If, in \hardsectionref{10}{7}, we make the point c a regular point, we must have

, ',, aa(a-b)(a-c) /3/3' (b - c) (b - a) l\ \ y=0, 77' = and - 4. 1
\ Lv 1 j g

divisible by - c, in order that p (z) and q (z) may be analytic at c.
Hence a + a + + jS' = 0, aa = yS/S', and the equation is d-ii j 1 - a
- a' 1 + a + a'] du atx a - by u dz- \ z - a z - b ) dz z- of z - b)-
'

of which the solution is

. fz - aY -D fz-( u=A J + B

\ z -b) \ z - bj

that is to say, the solution involves elementary functions only. When
a - a, the solution is

. fz-aY fz - aY fz-a

bJ'

REFEREXCES.

L. FuCHS, Journal fur Math. Lxvi. (1866), pp. 121-160.

L. W. Thomis, Journal fur Math. Lxxv. (1873), pp. 265-291, Lxxxvii.
(1879), pp. 222-349.

L. ScHLESiNGER, Handbuch der linearen Differentialgleichungen.
(Leipzig, 1895-1898.)

G. Frobenius, Journal fur Math. Lxxvi. (1874), pp. 214-235.

G. F. B. Riemann, Ges. Math. Werke, pp. 67-87.

F. C. Kleix, Ueber Uneai-e Differentialgleichungen der zineiter
Ordnung. (Gottingen, 1894.)

A. R. Forsyth, Theory of Differential Equations, iv. (1902).

T. Craig, Differential Equations. (New York, 1889.)

E. Goursat, Coxirs d\ inalyse, 11. (Paris, 1911.)

%
% 209
%

Miscellaneous Examples.

1. Shew that two solutions of the equation

are 2 - 3 7 2* + ..., 1- 2 + ..., and investigate the region of
convergence of these series.

2. Obtain integrals of the equation

d" 1,, ON

regular near 2 = 0, in the form

. i = "r + r6 + 1024 + -/'

3

 2 = Wll0g2-jg+....

3. Shew that the equation

has the solutions

TODO

and that these series converge for all values of z.

4. Shew that the equation

dZ tr=l Z-ttr ) dz \ r=i z - a,.Y r=l*- rj

where

2 (a, + 3,) = -2, 2 Z), = 0, 2 (a, A-+ari3r) = 0, 2 (a,2/) + 2a,a,i3,)
= 0,

r=I r=I r=l r=l

is the most general equation for which all points (including x ),
except aj, a-.,, ... a, are ordinary points, and the points a are
regular points with exponents a, r respectively.

\addexamplecitation{Klein.}

5. Shew that, if /iJ + y + /3' + y' =, then

(0 X 1 I i-1 X 1 ~|

P\ 0 y 22' p' 2/3 y 2-. \addexamplecitation{Riemann.}

h \& y J i y' 23' y J

[The dififerential equation in each case is

c/22 + 2- 1 dz r 22 . IJ,2\ l ""J

6. Shew that, if y + y' = and if co, or are the complex cube roots of
unity, tfeen rO X 1 \ rl CO 0)2 j

pJo y 23- = p'y y y 2-. \addexamplecitation{Riemann.}

U i y J ly y 7' >'

[The differential equation in each case is

d' u 222 du yy zu,

C£22 23\ l dz z -\ f -

W. M. A. 1

%
% 210
%

7. Shew that the equation

(l\ .2)g\ (2o + i)2 Vn( + 2a) =

is defined b\ \ the scheme

pj -n z

\ jf-a n + 2a - a j and that the equation

may be obtained from it by taking a = l and changing the independent
variable.

8. Discuss the solutions of the equation

dhi,,, dii (,1

\addexamplecitation{Halm.}

'-U,, -.du (, . 1 \ + ( + l+m) + (/i + l+ mj w =

valid near s = and those valid near s = oo . \addexamplecitation{Cunningham.}

 22

9. Discuss the solutions of the equation

valid near 2=0 and those valid near z=<x, . Consider the following
special cases :

(i) pi= -% (ii) /i - 5, (iii) /x + i/ = 3.
\addexamplecitation{Curzon.}

10. Prove that the equation '

s(l-2) +-(l-22) + (a2 + fe)w =

has two particular integrals the product of which is a single-valued
transcendental function. Under what circumstances are these two
pax'ticular integrals coincident ? If their product be F z), prove
that the particular integrals are

where C is a determinate constant.
\addexamplecitation{Lindemann; see \hardsectionref{19}{5}.}

l. Prove that the general linear differential equation of the third
order, whose singularities are 0, 1, oc, which has all its integrals
regular near each singularity (the exponents at each singularity being
1, 1, - 1), is

dhi (2 2 \ d \ \ \ 3 1 \ "1 du

d W ) dz V z z-\ y z- If] dz \ \ 3cos2a 3sin2a 1 1 \ n

+ p - w ) " ¥ ' " m '

where a n y ha 'e any constant value. \addexamplecitation{Math. Trip. 1912.}

\chapter{Integral Equations} 

11"1. Definition of an integral equation.

An integral equation is one which involves an unknown function under
the sign of integration; and the process of determining the unknown
function is called solving the equation*.

The introduction of integral equations into analysis is due to Laplace
(1782) who considered the equations

f x) = [e f (f) (t) dt, g x) = [ - 4> (t) dt

(where in each case (f) represents the unknown function), in connexion
with the solution of differential equations. The first integral
equation of which a solution was obtained, was Fourier's equation

f(.v) = l cos (xt) (f> (t) dt,

of which, in certain circumstances, a solution isf

2 f

(f)( x) = - I cos (ux)f(u) du,

TTJo

f x) being an even function of a;, since cos xt) is an even function.

Later, Abel+ was led to an integral equation in connexion with a
mechanical problem and obtained two solutions of it; after this,
Liouville investigated an integral equation which arose in the course
of his researches on differential equations and discovered an
important method for solving integral equations §, which will be
discussed in § 11 '4.

In recent years, the subject of integral equations has become of some
importance in various branches of Mathematics; such equations (in
physical problems) frequently involve repeated integrals and the
investigation of them naturally presents greater difficulties than do
those elementary equations which will be treated in this chapter.

To render the analysis as easy as possible, we shall suppose
throughout that the constants a, h and the variables x, y, are real
and further that

* Except in the case of Fourier's integral (§ 9'7) we practically
always need continuom solutions of integral equations.

t If this value of (p be substituted in the equation we obtain a
result which is, effectively, that of §9-7.

+ Solution de quelques problemes a Vaide d'integrales definies (1823).
See Oeuvres, i. pp. 11, 97.

§ The numerical computation of solutions of integral equations has
been investigated recently by Whittaker, Proc. Roijal Soc. xciv. (A),
(1918), pp. 367-383.

U-2

212 THE PROCESSES OF ANALYSIS [CHAP. XI

O', y, i>' also that the given function* K (x, y), which occurs under
the integral sign in the majority of equations considered, is a real
function of a; and y and either (i) it is a continuous function of
both variables in the range (a x b, a i/ b), or (ii) it is a
continuous function of both variables in the range a y x b and K (x,
y) = when y > x; in the latter case K x, y) has its discontinuities
regularly distributed, and in either case it is

easily proved that, iff(y) is continuous when a y b, f y) K x, y) dy
is a

J a

continuous function of x when a x b.

11 'll. An algebraical lemma.

The algebraical result which will now be obtained is of great
importance in Fredholm's theory of integral equations.

Let (a'i, 3/1, Zy), .V2, i/2, 22)) (• '35 3) 3) be three points at
unit distance from the origin. The greatest (numerical) value of the
volume of the parallelepiped, of which the lines joining the origin to
these points are conterminous edges, is +1, the edges then being
perpendicular. Therefore, if Xr + 7/ + Zr = l r = l, 2, 3), the upper
and lower bounds of the determinant

 2 2/2 22 I

• 3 3 23 I

are ±1.

A lemma due to Hadamardt generalises this result.

Let

 21? 0 22, ... a n

A

 nl) )!2' ••• < n

n

where a,. is real and 2 a" r = l ( i = l, 2, ... n); let A j. be the
cofactor of a r in D and let A be the determinant whose elements are A
j., so that, by a well-known theorem |,

Since Z) is a continuous function of its elements, and is obviously
bounded, the ordinary theory of maxima and minima is applicable, and
if we consider variations in

" dD

air ( '=!) 2, ... n) only, D is stationary for such variations if 2 -
8aij. = 0, where Saj,...

r=l VOlir

n

are variations subject to the sole condition 2 a.ir8air=0; therefore
§

r = l

n

but 2 airAir=I), and so X'2a\ r = D; therefore Air = Da.ir-

r=l

* Bocher in his important work on integral equations (Camb. Math.
Tracts, No. 10), always considers the more general case in which A'
(x, y) has discontinuities regularly distributed, i.e. the
discontinuities are of the nature described in Chapter iv, example 10.
The reader will see from that example that the results of this chapter
can almost all be generalised in this way. To make this chapter more
simple we shall not consider such generalisations.

+ Bulletin des Sci. Math. (2), xvii. (1893), p. 240.

J Burnside and Panton, Theory of Equations, ii. p. 40.

§ By the ordinary theory of undetermined multipliers.

11-11, 11-2] INTEGRAL EQUATIONS 213

Considering variations in the other elements of D, we see that D is
stationary for variations in all elements when Amr=Damj. (m=l, 2, ...
n; r=l, 2, ... n). Consequently A = D". D, and so I)' ' - D ~ . Hence
the maximum and minimum values of I) are +1.

Corollary. If a r be real and subject only to the condition | a I < j
since

2 a,,/( ii/) 2 1,

r=l

we easily see that the maximum value of | i9 | is (/i J/)" = ?i "i/".

11'2. Fredholms* equation and its tentative solution. An important
integral equation of a general type is

J a

where f(x) is a given continuous function, X is a parameter (in
general complex) and K (oo, ) is subject to the conditions f laid down
in § ll'l. K (x, ) is called the nucleusX of the equation.

This integral equation is known as Fredholms equation or the integral
equation of the second kind (see § 11 "3). It was observed by Volterra
that an equation of this type could be regarded as a limiting form of
a system of linear equations. Fredholm's investigation involved the
tentative carrying out of a similar limiting process, and justifying
it by the reasoning given below in § 11-21. Hilbert (Gottinger Nach.
1904, pp. 49-91) justified the limiting process directly.

We now proceed to write down the system of linear equations in
question, and shall then investigate Fredholm's method of justifying
the passage to the limit.

The integral equation is the limiting form (when 5-*-0) of the
equation

<i> x) =f x) + \ i K X, X,) (t> (.r,) 8, where x - Xq\ i = 8, XQ = a,
.r = b.

Since this equation is to be true when a x b, it is true when x takes
the values Xi, Xi, ... Xni and so

n

-\ 8 2 K x.p, x ) (.rg) + (f) (Xj,) =/ (Xj,) ip = l,2,... n).

q=l

* Fredholm's first paper On the subject appeared in the Ofversigt af
K. Vetenskaps-Akad. Forhandlingar (Stockholm), lvii. (1900), pp.
39-46. His researches are also given in Acta Blath. xxvii. (1903), pp.
365-390.

t The reader will observe that if K x, |) = |>x), the equation may be
written

<p (x) =f (x) +\ Tk (x, I) ( ) d|.

This is called an equation with variable upper limit.

1;. Another term is kernel; French noyau, German Kern.

214

THE PROCESSES OF ANALYSIS

[chap. XI

This system of equations for < (.rp), p=l, 2, ... 7i) has a unique
solution if the determinant formed by the coefficients of x ) does not
vanish. This determinant is

Z> (X) = 1 - X8 K (. 1,xi) -X8K .vi, x. ) ... -UK x, .v ) -\ 8K x.,
, A'l) 1 - XS K x.,, .r.,) ... - X S A' (.rg, . )

-\ 8K (.r, xi) - \ 8 K x, x.,) ... 1-X8K (x, x ) = 1 - X 2 8A' x,
.Vp) +- 2 S2 1 ''''' ' • '" " '

p = l 2 !p, q=l I K X, Xp) A [Xq, Xg)

X3 )i

'•;>> <I, r=l

+ ...

 "(" 'p' ' P' V" p " '9/ ' (." p> " r)

K Xq,Xp) K Xg,.Vq) K Xg,Xr)

lK x,.,Xp) K Xr,Xq) K Xr, Xr)

on expanding* in powers of X.

Making S-3-O, ti-s-qo, and writing the summations a-s integrations, we
are thus led to consider the series

/•6 x2 [ [

i)(X) = l-X| A' (li, 0 1 + 2] / j

 ( 2, 1) />:( 2, I2)

d idi;-...

Further, if Z) (.r, x,) is the cofactor of the term in I)n ) which
involves K x\, x J, the solution of the system of linear equations is

,, /( i)i) (.r, xi)+f x.2) Z) (av, .v.i) + ...+f xJD x, Xn)

Now it is easily seen that the appropriate limiting form to be
considered in association with Da Xf, x ) is D ); also that, if /n +
i',

Dn (AV, -iV) = U\ K x, x ) - XS 2

K(x,x ) K x,Xp)

A Xpi Xi,) A (A'p, Xp)

+,\ \ 8 2

 (• A'- V) Xf,.Vp) K x,Xg)

K Xp, .r ) K (Xp, .?7p) (Xp,; •g)

A' (.•Pg, X ) K (Xg, Xp) K Xg, Xg)

So that the limiting form for 8~ I) x ., x ) to be considered t is

D (.r,Xt,; X) = X A" x

>, Av)-X2 j

2! ja Ja A:( i,

'' I K x,x ) A'(.r, 1) I

K 2,X;) K -2, l) ( 2,6)

 Il' l2-----

Consequently we are led to consider the possibility of the equation
<P(x)=f x) + J \ [ d .x,; ) fii)d giving the solution of the integral
equation.

* The factorials appear because each determinant of s rows and columns
occurs s ! times as p, q, ... take all the values 1, 2, ... n, whereas
it appears only once in the original determinant for D (X).

t The law of formation of successive terms is obvious from those
written down.

] 1 -21] INTEGRAL EQUATIONS

Example 1. Shew that, in the case of the equation

  x) = x->r\ \ xii(l) y)dy, J

D ) = 1\, D (.r, y; X) = \ xy Zx

215

we have

and a solution is

0( ) =

3-X'

we have

Example 2. Shew that, in the case of the equation

<j> x) = x + \ I (xy+y )<f>(i/)dy,

J

D x, y; X) = X xy + ?/2) +X xy -,xy - If + y\ and obtain a sohition
of the equation.

11"21. Investigation of Fredholms solution.

So far the construction of the solution has been purely tentative; we
now start ah initio and verify that we actually do get a solution of
the equation; to do this we consider the two functions D ( ), D x,
y; ) arrived at in § 11*2.

We write the series, by which D ( -) was defined in § 11 '2, in the
form 1 + 2 so that

 =i n\

rb fb rb J a -1 a J a

d,d 2 ••• d n;

since K x, y) is continuous and therefore bounded, we have \ K x,y)\ <
M, where M is independent of x and y; since K x, y) is real, we may
employ Hadamard's lemma (§ 11*11) and we see at once that

Write n 'M'' (b - a)" = n ! 6; then

lim(6.WM=lim f-" - f|(l+jYf

u oo M oo (71+1)5 (\ nj )

since ( 1 +

The series 2 bnX"- is therefore absolutely convergent for all values
of X,;

w = l

and so (§ 2-34) the series 1+2 - ~ converges for all values of X and
there- fore (§ 5"64) represents an integral function of X.

Now Tfrite the series for D (x, y;X) in the form 2 - - '- - .

216 THE PROCESSES OF ANALYSIS [CHAP. XI

Then, by Hadamard's lemma § 1111),

and hence ' < Cn, where c is independent of a; and y and 2 Cn> ' is n\
=o

absohitely convergent.

Therefore D x, y; ) is an integral function of \ and the series for D
x, y\ \ )- \ K (x, y) is a uniformly convergent (§ 3-34) series of
continuous* functionsj|of x and y when a x b, a y b.

Now pick out the coefficient of A" (a;, y) in D(x, y;X); and we get

D x,y;X) \ D ) K x, y) + 2 (-) X +', where

Expanding in minors of the first column, we g t Q x, y) equal to the
integral of the sum of n determinants; writing | i, o, ... m-i, |,
,n, • • n-\ in place of fi, o, ... in the ??ith of them, we see that
the integrals of all the determinants + are equal and so

Qn x,y) = -n\ K x, y) Pnd d i • • • d n-i,

J a J a J a

where

Pn=\ K X, I), K X,,), ... K(x, | \ 0

i di.a K A ... A-(inf -o

It follows at once that

D(x,y\ X) = \ D(X)K(x,y)+\ ( D (x, \ X)K ly)dl

. a

Now take the equation

<i> )=fi ) + \' K ly)<f> y)dy,

J a

multiply by D (x,; X) and integrate, and we get

l'f(BD(,; )d

J a

= I % (I) 0 -. ?; ) ? - r [' (•' '; ) < ' ) > y* '

.'a . a J i(

the integrations in the repeated integral being in either order.

* It is easy to verify that every term (except possibly the first) of
the series for D (x, y; ) is a continuous function under either
hypothesis (i) or hypothesis (ii) of § 11-1.

h

t The order of integration is immaterial (§ 4-3).

t

11-21] INTEGRAL EQUATIONS 217

That is to say

J a

= r<j>( )D(w,; ) d -r D w,y; ) -\ D( ) K x,y)]<j>(y)dy

J a J a

= \ D ) \' K x,y)<i> y)dy

in virtue of the given equation.

Therefore if D X) 0 and if Fredholm's equation has a solution it can
be none other than

< X) =f x) +jy( ) d;

and, by actual substitution of this value of <f> x in the integral
equation, we see that it actually is a solution.

This is, therefore, the unique continuous solution of the equation if
i)(X)4=0.

Corollary. If we put /( ) = 0, the ' homogeneous ' equation

J a

has no continuous solution except (j) (.i')=0 unless D X) = 0.

Example 1. By expanding the determinant involved in \$ (.r, y) in
minors of its first row, shew that

D x,y; ) = \ D K)K x,y) + \ \ ' K x, )D,y; ) d .

J a

Example 2. By using the formulae

2)(X) = 1+ i ""fi, D x,y; ) = \ D X) K(x, y)+ i ( - )" " ' ' "/•'' \
/i=i 'I 11=1 " •

shew that f " i> (,; X) f l = - X .

Example 3. If K x, y) = 1 (y 0) i \ y) = (y > ' )-> shew that D ) = b-
a) X.

Example 4. Shew that, if K (.r, y)=fi (x) .f [y), and if

-•6

fi )f2 x)dx = A,

then

D ) = A\ D x,y; ) = \ h x)f., (y),

and the solution of the corresponding integral equation is

218 THE PROCESSES OF ANALYSIS [CHAP. XI

Example 5. Shew that, if

K x, y) =/i x) gi y) +f x) 2 y), then D (X) and D x, y; X) are
quadratic in X; and, more generally, if

II K x,y)= 2 f x)g,n ),

m = l

then i)(X) and D x, y, X) are polynomials of degree n in X.

11*22. Volterra's reciprocal functions.

Two functions K x, y), k x, y; X) are said to be reciprocal if they
are bounded in the ranges a- x, y - 1), any discontinuities they may
have are regularly distributed (§ ll'l, footnote), and if

K x,y)+k x,y] ) = \ \ k x, \ \ ) K, y)d .

J a

We observe that, since the right-hand side is continuous*, the surn of
two reciprocal functions is continuous.

Also, a function K x, y) can only have one reciprocal if Z) (X) 4=;
for if there were two, their difference k x, y) would be a continuous
solution of the homogeneous equation

h x,y; X) = X f >(-,(.r, )K k,y)di,

(where x is to be regarded as a parameter), and by § 11 '21 corollary,
the only continuous solution of this equation is zero.

By the use of reciprocal functions, Volterra has obtained an elegant
reciprocal relation between pairs of equations of Fredholm's type.

We first observe, from the relation

B x,y; ) = \ D X)K x,y) + \ \ D x, -X) K H y) d

proved in § 11 "21, that the value of A;(, y; X) is

-D x,y;X)l[\ D ) ], and from § 11*21 example 1, the equation

k x, y; ) + K x, y) = X f K x, ) k, y;X)d

J a M

is evidently true.

Then, if we take the integral equation

< > x)=f(x) + xl'K x, )<f>( )d,

J a

when a'> x h, we have, on multiplying the equation

J a * By example 10 at the end of Chapter iv.

11-22, 11 -23] INTEGRAL EQUATIONS 219

by k (x,; ) and integrating,

J a

= r k(x, : ) f( )d + xf' I' k(x, : ) K 1,)<f>(,)d,dl

J a J II J a

Reversing the order of integration* in the repeated integral and
making use of the relation defining reciprocal functions, we get

\ \ x, : ) 4>( )di

J a

= !'k(w, : ) f( )d +r K(x,,) + k w,,;X) < (|0 i,

J a J a

and so X f '(, X)/( )fZ = -X l K (x, ) <f> (,) d,

J a n

= -<P(x)+f x). Hence f(x) = ( ) + \ f ' (a;, f; ) /( ) d;

. a

similarly, from this equation we can derive the equation

4> x)=f(x) + \ f'K(x, )4>( )d,

J II

so that either of these equations with reciprocal nuclei may be
regarded as the solution of the other.

11"23. Homogeneous integral equations.

rb The equation < x) = X ) K x, ) </> ( ) d is called a homogeneous
integral

- n

equation. We have seen (§ 1121 corollary) that the only continuous
solution of the homogeneous equation, when D (X) 4= 0, is ( x) = 0.

The roots of the equation D (X) = are therefore of considerable
importance in the theory of the integral equation. They are called the
characteristic numbers of the nucleus.

It will now be shewn that, when D (X) = 0, a solution which is not
identically zero can be obtained.

Let+ X = Xo be a root m times repeated of the equation D (X) = 0.

Since D (X) is an integral function, we may expand it into the
convergent series

D (X) = Cm (X - Xo)" + c,n, (X - Xor +1 + . . . (m > 0, c, + 0).

* The reader will have no diflSculty in extending the result of § 4-3
to the integral under consideration.

t French valeurs caracteristiques, German Eigenicerthe.

J It will be proved in § 11-51 that, if K (x, ij) = K y, x), the
equation D (X) = has at least one root.

220 THE PROCESSES OF ANALYSIS [CHAP. XI

Similarly, since D x, y; X) is an integral function of \, there
exists a Taylor series of the form

by § 3'34 it is easily verified that the series defining g (cc, y), (n
= 1, 1 + 1, ...) converges absolutely and uniformly when a x b, a y -
b, and thence that the series for D (x, y; ) converges absolutely and
uniformly in the same domain of values of x and y.

But, by § 11-21 example 2,

L

i)(f.f;X)df=-X >

now the right-hand side has a zero of order m - 1 at A,o, while the
left-hand side has a zero of order at least I, and so we have m-\' I.

Substituting the series just given for D ( ) and D x,y\ X) in the
result of § 11'21 example 2, viz.

D x,y- ) = XD (X) K x, y) + \ \ ''K x, ) D I y; ) d

J a

dividing by (X, - XoY and making X -* X,o. we get

9i, y) = \ K x, I) gi (, y) d .

J a

Hence if y have any constant value, gi x, y) satisfies the homogeneous
integral equation, and any linear combination of such solutions,
obtained by giving y various values, is a solution.

Corollary. The equation

<l> G ) =/(. -) + Xo f ' iT (.r, ) < ( ) d

J a

has no solution or an infinite number. For, if ( x) is a solution, so
is x) + c gi (x, y),

y where Cy may be any function of y.

Example 1. Shew that solutions of

< x) = \ I cos'<(.r-|)( ( )c/

are </> (.r) = cos (?i - 2r) .r, and ( (.r) = sin (% - 2?-) .r; where
r assumes all positive integral values (zero included) not exceeding
hi.

Example 2. Shew that

  x) = \ y cos'' (.r-h I) ( ( ) o?|

has the same solutions as those given in example 1, and shew that the
corresponding values of X give all the roots of D (X) = 0.

ir3-ir4] INTEGRAL EQUATIONS 221

11*3. Integral equations of the first and second kinds. Fredholra's
equation is sometimes called an integral equation of the second kind;
while the equation

f x) = \ \ \ \ {x, ), )d

J a

is called the integral equation of the first kind.

In the case when K x, ) = Q ii > x, we may write the equations of the
first and second kinds in the respective forms

f x) = \ [ K x, ) )dl

J a

4> a )=f x) + xrK x, )cf>( )d .

J a

These are described as equations with variable upper limits.

11"31. Volterra's equation.

The equation of the first kind with variable upper limit is frequently
known as Volterra's equation. The problem of solving it has been
reduced by that writer to the solution of Fredholm's equation.

Assuming that K x, ) is a continuous function of both variables when
\$ X, we have

f x) = \ \ K x, )< )dl

J a

The right-hand side has a differential coefficient (§ 4*2 example 1)
if -T- exists and is continuous, and so

ox

f (x) = \ K (x, x) <p X) + Xj -cf> ) dl

This is an equation of Fredholm's type. If we denote its solution by
x), we get on integrating ft-om a to x,

f x)-f a)=\ \ K x, )<l> )dl

J a

and so the solution of the Fredholm equation gives a solution of
Volterra's equation if /(a) = 0.

The solution of the equation of the first kind with constant upper
limit can frequently be obtained in the form of a series *.

11"4. The Liouville- Neumann method of successive substitutions "f. A
method of solving the equation

cf> x)=f(x) + \ \ ' K(x, )cf ( )d,

J a

which is of historical importance, is due to Liouville.

* See example 7, p. 231; a solution valid under fewer restrictions is
given by Bocher. t Journal de Math. ii. (1837), iii. (1838). K.
Neumann's investigations were later (1870); see his Untersuchungen
ilber das logarithmische und Newton'sche Potential.

222 THE PROCESSES OF ANALYSIS [CHAP. XI

It consists in continually substituting the value of 4) oo) given by
the right-hand side in the expression < ( ) which occurs on the
right-hand side.

This procedure gives the series S x)=f x) + \ \ ' K x, aA?) + S X- f K
x, 0 [* K I,)

J a m = 2 J a J a

J a

Since | K x, y) \ and \ f x) \ are bounded, let their upper bounds be
M, M'. Then the modulus of the general term of the series does not
exceed

|\ [ ' l/' ir(6-a)'". The series for S x) therefore converges
uniformly when

\ \ \ < M-' h-a)-; and, by actual substitution, it satisfies the
integral equation.

If Kix, y) = when y>x\ we find by induction that the modukis of the
general term in the series for S x) does not exceed

X I * Jf'" M' x - a)>"l m l) \ X\'>' i/' M' (6 - ay>'jm !,

and so the series converges uniformly for all values of X; and we
infer that in this case Fredholm's solution is an integral function of
X.

It is obvious from the form of the solution that when | X. | < M~ (b -
a)~\ the reciprocal function k (x,; X) may be written in the form

k x, X)=.-K x, )- t X-- f' K(x, 0 f /l (I,, eO

m=2 J a J a

for with this definitipn of k x,; A,), we see that

S x)=f x)-\ \ ' k(x,; ) f( )d,

J a

so that k x, |;X) is a reciprocal function, and by § 11"22 there is
only one reciprocal function.

Write

K (X, I) = K, X, ), f K X, r) Kn (r, B r - J n+r (, ),

and then we have

-k x,;X)= S V K, +,(x, ),

m =

while f K (x, r ) Kn (r, e r = A .+ (*'. )>

J a

as may be seen at once on writing each side as an (m + n - l)-tuple
integral. The functions K, (x, f) are called iterated functions.

J a

ITO, ll'olj INTEGRAL EQUATIONS 223

11*5. Symmetric nuclei.

Let Ki x, y) = K (y, x); then the nucleus K x, y) is said to be
symmetric. The iterated functions of such a nucleus are also
symmetric, i.e. Kn, y) = Kn (y, x) for all values of n; for, if Kn
x, y) is symmetric, then

Kn+r (, y) = f K,, ) Kn I y) d = j K, (|, o:) K, y, ) d

J a 'a

= Kn(y, )K,i,x)d = Kn, y,x),

J a

and the required result follows by induction.

Also, none of the iterated functions are identically zero; for, if
possible, let Kp (x, y) = 0; let n be chosen so that 2"" <]) - 2",
and, since Kp (x, y) = 0, it follows that K n x, y) = 0. from the
recurrence formula.

But then = K n x, x) = I, \ i x, ) K -i (, ) d

J a

= f [Kn-.i . rr-d,

J a

and so K n-i i, ) - j continuing this argument, we find ultimately
that Ki (x, y) = 0, and the integral equation is trivial.

11"51. Schmidt's* theorem that, if tfoe nucleus is symmetric, the
equation J) ( ) = has at least one root.

To prove this theorem, let

Un = Kn (X, X) dx,

J a

SO that, when \ \ \ < M~ (b - a)~\ we have, by § 11-21 example 2 and §
114,

\ 1 dDiX)

D ) dx ri " •

Now since I I fjLKn+i (x, |) +,i\ i (x, )]- d dx

J a J a

for all real values of fi, we have

tl'U . + 2/JLU.>n + U,n-, > 0, and so U n+i U.n-- U n-, Uon-1 > 0.

Therefore U.,, Ui, ... are all positive, and if- JZ /C, = i, it
follows, by in- duction from the inequality ?7o +2 C/2,1-2 C 2 i'.
that Unn+n/U.n > v -

X

Therefore when j X- v~\ the terms of S Un'. ' ~ do not tend to zero;

and so, by § 5'4, the function, - -7- - has a singularity inside or
on the

* The proof given is due to Kneser, Palermo Bendiconti, xxii. (1906),
p. 236.

224 THE PROCESSES OF ANALYSIS [CHAP. XI

circle •\ \ = v~ \ but since D( ) is an integral function, the only
possible

sinerularities of -r., . - j- are at zeros of D(X); therefore D (X)
has a zero ° 1) (X,) d\

inside or on the circle \ \ \ = v~ .

[Note. By § 11-21, Z> (X) is either an integral function or else a
mere polynomial; in the latter case, it has a zero by § 6"31 example
1; the point of the theorem is that in the former case D (X) cannot
be such a function as e ', which has no zeros.]

11'6. Orthogonal functions.

The real continuous functions (jj (x), (f)o (x), . . . are said to be
orthogonal and normal* for the range a, b) if

If Ave are given n real continuous linearly independent functions u-
x), Uo x), ...Un(x), we can form n linear combinations of them which
are orthogonal.

For suppose we can construct m - 1 orthogonal functions </>!, ...
(f>m-i such that (f>p is a linear combination of u-, ii, ... Up
(where p = 1, 2, ... ni - I); we shall now shew how to construct the
function, such that c i, <p.2, ••• (f>m are all normal and
orthogonal.

Let (f>m (x) = Ci, m (f>i ( ) + C2, m < 2 ( ) + ••• + Cm-i <pm-i (i )
+ Urn ),

so that i( is a function of Mj, lu, ... w, . Then, multiplying by and
integrating,

i< ni x) (t>p oc) dx = Cp,, + I Um ) <j>p (x) dx p < m). Hence I i4>m
i ) 4>p oc) dx =

J a

if Cp m == - Uin (jc) <pp ( ) dx;

J a

a function i, (x), orthogonal to (f) (x), (f>. (x), . . . <pm-i (x),
is therefore con- structed.

rb

Now choose a so that a- I [i(f>r,i (x)]- dx=l;

J a

and take <p,n (x) = a. i<, . (x).

Then j </) (x) 4>p (x) dx |~ '

We can thus obtain the functions (/)i, c/).., ... in order.

* They are said to be orthogonal if the first equation only is
satisfied; the systematic study of such functions is due to Murphy,
Camb. Phil. Trans, iv. (1833), pp. 353-408, and v. (1835), pp.
113-148, 31.5-394.

ir6, ir6l] INTEGRAL EQUATIONS 225

The members of a finite set of orthogonal functions are linear! '
inde- pendent. For, if

a,(j), (x) + a.2(f>. (x)+ ... + an(f>n ( v) = 0,

we should get, on multiplying by (f>p (a;) and integrating, a =;
therefore all the coefficients a vanish and the relation is nugatory.

It is obvious that tt ~ cost/u;, tt ~ - sin mx form a set of normal
orthogonal functions foifthe range ( - tt, tt).

Example 1. From the functions 1, .r, x-, ... construct the following
set of functions which are orthogonal (but not normal) for the range (
- 1, 1) :

1, X, - i, x -'jx, x* - x- + - g, ....

Example 2. From the functions 1, x, x, ... construct a set of
functions

which are orthogonal (but not normal) for the range (a, h); where

[A similar investigation is given in >5 15-14.]

11 "61. The connexion of orthogonal functions with homogeneous
integral equations.

Consider the homogeneous equation

<l> x) = \,\% )K x, )dl

  a

where Xq is a real* characteristic number for K(x, ); we have already
seen how to construct solutions of the equation; let n linearly
independent solutions be taken and construct from them n orthogonal
functions (pi, < .\,, ... (/> .

Then, since the functions, are orthogonal, rr i <l>Uy)f K x, )cf>, (
)di\'dy = i f\ < p.n(y)fK(x, )<f>, )Ydi

J a |\ t = 1 J a J m = lJ a [\ J a J

and it is easily seen that the expression on the right may be written
in the form

rb

i If K(x. )4>, ( )d

H = 1 (. J a

>H = 1

on performing the integration with regard to y; and this is the same
as i r K x, y) cf>,n (y) dyi' K x, |),, ( ) rff

m. = \ J a J a

Therefore, if we write K for K x, y) and A for

i 4>Ay)\' K, )<l>.nX )dl

m=\ J a

* It will be seen immediately that the characteristic numbers of a
symmetric nucleus are all real.

W. M. A. 15

Therefore

226 THE PROCESSES OF ANALYSIS [CHAP. XI

rb rb

we have A?dy = \ KAdy,

J a J a

rb rb rb

and so 1 A-dij = K'dy - (K - Kfdy.

•la J a J a

a \ m = \ f o ) J a

and SO Xo~' <, ( ) [ [K x,y)Ydy.

w = 1 J a

Integrating, we get

n <: Xo'- I / [K x, y)Y-dydx.

J a J a

This formula gives an upper limit to the number, 7i, of orthogonal
functions corresponding to any characteristic number Xo-

These n orthogonal functions are called characteristic functions (or
auto- functions) corresponding to Xq.

Now let '"* (x), < <i' (x) be characteristic functions corresponding
to different characteristic numbers Xq, Xj.

Then </>' > (x) </><!' (x) = xJ K x, ) ( c) x) </)' ' ) d,

J a

and so

[ (f> ' >(x)(f) ' (x)dx = \,\ i K(x, )4> '> (x)cf)'' )d dx ...(1),

and similarly

[ (/)' ' (x) </) w (x) dx = \ of I K x, ) 0 " ( ) </)(!> (x) d dx

•la J a J a

= X [ [ K(lx)cf> '' ix)(f> ' ( )dxd ...(2),

J a •' a

on interchanging x and .

We infer from (1) and (2) that if \ \ and if K x, ) = K, x),

I 4> ' (x)(f> ' (x)dx=0,

J a

and so the functions < "'' (x), '" (x) are mutually orthogonal.

If therefore the nucleus be symmetric and if, corresponding to each
characteristic number, we construct the complete system of orthogonal
functions, all the functions so obtained will be orthogonal.

Further, if the nucleus be symmetric all the characteristic numbers
are real; for if Xq, Xj be conjugate complex roots and if* Uq x) = v
(x) + iw (x) be

* V (x) and w (x) being real.

117] INTEGRAL EQUATIONS 227"

a solution for the characteristic number X, then iii (x) = v (x) -
iiu (x) is a solution for the characteristic number \; replacing 0*''
(x), 'i* (x) in the equation

I ( 'o' (x) </>< ' (x) dx =

J a

by V (x) + iw (x), v x) - iw (x), (which is obviously permissible), we
get

f [\ v(x)Y+ \ w x)Y]dx = 0,

J a

which implies v (x) = w (x) = 0, so that the integral equation has no
solution except zero corresponding to the characteristic numbers Xq.
il this is

contrary to § 11 "23; hence, if the nucleus be symmetric, the
characteristic

numbers are real.

11 7. The development* of a symmetric nucleus.

Let < j x), (f)., (x), 3 (x), ... be a complete set of orthogonal
functions satisfying the homogeneous integral equation with symmetric
nucleus

cf>(x) = xfK(x, )<f>( )dl

J a

the corresponding characteristic numbers beingf Xj, X, X, -

XT J. 1, f' < n (* ) < i(v) • -r 1

Now supposel that the series - - - - is umiormly convergent

when a x b, a y %b. Then it luill he shewn that

K(x.y)=i M My). =i

For consider the symmetric nucleus

71 = 1 / n

If this nucleus is not identically zero, it will possess (§ 11 '51) at
least one characteristic number /j,.

Let ylr(x) be any solution of the equation

 jr x)=fJ,f H(x.. )ylri )dl

' a

which does not vanish identically.

Multiply by cf)n (x) and integrate and we get

ry!r(x)cf>,, x)dx = f,f Hxix, )- I "t if l l ( ) cj U ) dxd;

J a a ' a [ n = l 7i )

* This investigation is due to Schmidt, the result to Hilbert.

t These numbers are not all different if there is more than one
orthogonal function to each characteristic number.

X The supposition is, of course, a matter for verification with any
particular equation.

15-2

>t\

228 THE PROCESSES OF ANALYSIS [cHAP. XI

since the series converges uniformly, we may integrate term by term
and get

f ir (.v) 4> x) dx = \ \ ( ) (/> i )d - r < (I) f ( ) d

= 0.

Therefore t/ (x) is orthogonal to < i (x), < 2 ( ), ••• ', and so
taking the equation

J a [ n = l n )

rb

we have yjr (x) = y"- / K (x, )' ) d .

J a

Therefore /i, is a characteristic number of K (x, y), and so -v/r x)
must be a linear combination of the (finite number of) functions < n
(*') corresponding to this number; let

m

Multiply by (j>m (x) and integrate; then since -yfr (x) is orthogonal
to all the functions (j)n (x), we see that a, = 0, so, contrary to
hypothesis, yjr (x) = 0.

The contradiction implies that the nucleus H (x, y) must be
identically zero; that is to say, K (x, y) can be expanded in the
given series, if it is uniformly convergent.

Example. Shew that, if Xq be a characteristic number, the equation

4> x)=f .v) + \ J'' E x, ) )d

J a

certainly has no sokition . unless f x) is orthogonal to all the
characteristic functions corresponding to .

11 "71. The solution of Fredhohns equation by a series. Retaining the
notation of § 11 '7, consider the integral equation

   x) =f x) +\ f K (x, )< ) dl

-' a

where K x, ) is symmetric.

If we assume that ( ) can be expanded into a uniformly convergent

00

series 2 an4>n (1), we have

00 00 A

2 an(f)n x!)=f(x)+ 1 -an(f>n( ), ' w = l M=l n

SO that f x) can be expanded in the series

00 A -A

S an -\ (f)n x).

Hence if the function f x) can be expanded into the convergent series

< . " b \

2 bn< >n(ix), then the series 2 . " \ <f>n ), if it converges
uniformly in

n = l n-l n

the range a, b), is the solution of Fredhohns equation.

ir71-ir8l] INTEGRAL EQUATIONS 229

X

To determine the coefficients 6, we observe that S bn (f)n ( )
converges uni-

n=l

formly by § 3'35*; then, multiplying by (f)n(x) and integrating, we
get

J a

11 "8. Solution of AbeVs integral equation. This equation is of the
form

  -llw

d\$ 0<fi<l, a x b),

where/' (x) is continuous aud/(a)=0; we proceed to find a continuous
solution u (x)

Let (f)(x)= I u (I) o?|, and take the formula t

J a

7T n dx

sin/iTT ~ J z-xY~> x - y

multiply by u ( ) and integrate, and we get, on using Dirichlet's
formula (§ 4*51 corollary\

  f' f x)dx

Since the original expression has a continuous derivate, so has the
final one; therefore the continuous solution, if it exist, can be
none other than

"" ''V~dzj z-xy-> '

and it can be verified l)y substitution J that this function actually
is a solution.

11 "SI. Schlomilch's integral equation.

Let f x) have a continuous differential coefficient when - tt .r tt.
Then the equation

2 /"i" f x)=- (f) xsmd)de

  J n

has one solution with a continuoiis differential coefficient when - it
x it, namely

/ in < x)=f 0) + x ' f' xsm6)d6. Jo From § 4-2 it follows that

f (.v) = - sin Oct)' x sin (9) dd j

(so that we have (0)=/(0), ' (0) = |7r/' (0)).

* Since the numbers X, are all real we may arrange them in two sets,
one negative the other positive, the members in each set being in
order of magnitude; then, when ' X ! > X, it is evident that X /(X -
X) is a monotonic sequence in the case of either set.

t This follows from § 6"24 example 1, by writing (z - x)l(x - f ) in
place of x.

X For the details we refer to Bocher's tract.

§ Zeitschrift filr Math, und Phys. ii. (1857). The reader will easily
see that this is reducible to a case of Volterra's equation with a
discontinuous nucleus.

230 THE PROCESSES OF ANALYSIS [CHAP. XI

Write .r sin // for x\ and we have on multiplying by x and integrating

X j /' x sin \//-) d\ lr - - I < I sin ( ' (.i- sin 6 sin \//-) c? \
dyjr.

Change the order of integration in the repeated integral (§ 4'3) and
take a new variable x in iDlace of \ f defined by the equation sin; =
sin 6 sin yj/.

rr,. fi" .,, .s,, 2.V fi ( f (b' (x sin v) cos Y dv],

Then xj f'ixsn.i.)d = j \ j -~J -~ - \ de.

Changing the order of integration again (§ 4'51),
$$
TODO
$$
and so x \ f x sin y\ r) d = x \ </>' (.r sin x) cos;>( o?;

Jo J

= 0Gt-)-( (O).

Since (f) (0) =/ (0), we must have

(j) (.r) =/ (0)+x ' f (x sin;/.) c a/;

y

and it can be verified by substitution that this function actually is
a solution.

REFERENCES.

H. Bateman, Report to the British Association*, 1910.

M. BocHER, Introditction to Integral Equations (Cambridge Math.
Tracts, No. 10, 1909).

H. B. Heywood et M. Fr chet, Liquation de Fredkolm. (Paris, 1912).

V. Volte ra, Lego7is sur les equations integrales et les equations
integro-differentielles (Paris, 1913).

T. Lalesco, Introduction a la the'orie des equations integrales
(Paris, 1912).

I. Fredholm, Acta Mathematica, xxvii. (1903), pp. 365-390.

D. HiLBERT, Orundzilge einer allgemeinen Theorie der linearen
Integralgleichungen (Leipzig, 1912).

E. Schmidt, Math. Ann. lxiii. (1907), pp. 433-476.

E. GouRSAT, Cours d' Analyse, iii. (Paris, 1915), Chs. xxx-xxxill.

MiSCELT.ANEOUS EXAMPLES.

1. Shew that if the time of descent of a particle down a smooth curve
to its lowest point is independent of the starting point (the particle
starting from rest) the curve is a cycloid. (Abel.)

* The reader will find a more complete bibliugrapby in this Report
than it is possible to give here.

INTEGRAL EQUATIONS 231

2. Shew that, if/( ) is continuous, the sohition of

(f) (x) =f x) + \ j COS (2.rs) <f) (s) ds

/(.r) + X I f s)cos 2jcs)ds

'" *( >= Hrjxv .

assuming the legitimacy of a certain change of order of integration.

3. Shew that the Weber-Hermite functions

satisfy 4> x) = \ I e '*- (s) ds

for the characteristic values of X. (A. Milne.)

4. Shew that even periodic solutions (with period Stt) of the
diflferential equation

- - + ( 2 + >l-2 cos2 x) cf) (.r) = satisfy the integral equation

<f> x) = \ j e < oo'=< '(f, s)ds. (Whittaker; see § 19-21.)

5. Shew that the characteristic functions of the equation

are <f (x) = cos m.v, sin inx,

where \=m' and m is any integer.

6. Shew that <p (x) = / " - -f (|) d\$

has the discontinuous solution <f) (x) = Lr ~ . (Bocher.)

7. Shew that a solution of the integral equation with a symmetric
nucleus

f :v)=rK x, )<t>i\$)di

J a

is 0( )= 2 a \ \ (f) (x),

n=l

provided that this series converges uniformly, where X, 0 (x) are the
characteristic numbers and functions of K x, |) and 2 a 0 (x) is the
expansion off(x).

n=l

8. Shew that, if | A ] < 1, the characteristic functions of the
equation

'f' '' =LI[ i-2hcL i- Hh'' "

are 1, cos?n.r, sin mx, the corresponding characteristic numbers being
1, l/A'", Ijh'", where m takes all positive integral values.

\part{The Transcendental Functions}
%
% 235
%
\chapter{The Gamma Function}

\Section{12}{1}{Definitions of the Gamma-function. The Weierstrassian product.}

\index{Gamma-function}

Historically, the Gamma-functioni$^*$\footnote{The notation
  $\Gamma(z)$ was introduced by Legendre in
  1814.} % TODO symbol footnote
$\Gamma(z)$ was first defined by Euler as the limit of a product
(\hardsubsectionref{12}{1}{1}) from which can be derived the infinite
integral $\int_0^\infty t^{z - 1} e^{-t} \dmeasure t$; but in
developing the theory of the function, it is more convenient to define
it by means of an infinite product of Weierstrass' canonical form.

Consider the product \index{Weierstrassian product}\index{product for
  the Gamma-function}
$ze^{\gamma z} \prod_{n=1}^\infty \thebrace{\theparen{1 + \frac{z}{n}}
  e^{-\frac{z}{n}}}$ , where
$\gamma = \lim_{m \rightarrow \infty} \thebrace{ \frac{1}{1} + \half +
  \ldots + \frac{1}{m} - \log m} = 0 \cdot 5772157 \ldots$ .

%\begin{smalltext} % TODO 
[The constant $\gamma$ is known as Euler's or Mascheroni's constant
\index{Constant, Euler's or Mascheroni's}\index{Euler's constant}\index{Mascheroni's constant}; to
prove that it exists we observe that, if
\begin{displaymath}
u_n = \int_0^1 \frac{t}{n (n + t)} \dmeasure t = \frac{1}{n} -
\log\frac{n+1}{n} ,
\end{displaymath}
$u_n$ Is positive and less than $\int_0^1 \frac{\dmeasure t}{n^2} = \frac{1}{n^2}$;
therefore $\sum_{n=1}^\infty u_n$ converges, and 
\begin{displaymath}
  \lim_{m \rightarrow \infty} \thebrace{ \frac{1}{1} + \half 
      + \ldots + \frac{1}{m} - \log m} =
  \lim_{m \rightarrow \infty} \thebrace{\sum_{n=1}^mx u_n  - 
      \log \frac{m+1}{m}} =
  \sum_{n=1}^\infty u_n .
\end{displaymath}

The value of $\gamma$ has been calculated by J.~C.~Adams to 260 places of
decimals.]
%\end{smalltext}

The product under consideration represents an analytic function of $z$,
for all values of $z$; for, if $N$ be an integer such that $\absval{z} \le
\half  N$, we have$^\dagger$\footnote{Taking the principal value
  of $\log(1 + z/n)$.},% TODO symbol footnote
if n > N,
\begin{align*}
\absval{\log \theparen{1 + \frac{z}{n}} - \frac{z}{n}} 
& =   \absval{ -\half  \frac{z^2}{n^2} + \frac{1}{3} \frac{z^3}{n^3} -
  \ldots} \\
& \le \frac{|z|^2}{n^2} \thebrace{ 1 + \absval{\frac{z}{n}} +
      \absval{\frac{z^2}{n^2}} + \ldots} \\
& \le \frac{1}{4} \frac{N^2}{n^2} \thebrace{ 1 + \half +
      \frac{1}{2^2} + \ldots} \le \half  \frac{N^2}{n^2} .
\end{align*}

Since the series $\sum_{n = N+1}^\infty
\thebrace{\frac{N^2}{2n^2}}$ converges, it follows that,  when
$|z| \le \half N$,

%
% 236
%

$\sum_{n = N+1}^\infty \absval{\log \theparen{1 + \frac{z}{n}} -
  \frac{z}{n}}$ is an absolutely and uniformly convergent series of
analytic functions, and so it is an analytic function
\hardsectionref{5}{3}); consequently its exponential
$\prod_{n=N+1}^\infty \thebrace{\theparen{1 + \frac{z}{n}}
  e^{-\frac{z}{n}}}$ is an analytic function, and so
$ze^{\gamma z} \prod_{n=1}^\infty \thebrace{\theparen{1 +
    \frac{z}{n}}  e^{-\frac{z}{n}}}$ is an analytic function when
$|z| \le \half N$, where $N$ is any integer; that is to say, the
product is analytic for all finite values of $z$.

The Gamma-function was defined by
Weierstrass$^*$\footnote{\emph{Journal f\"ur Math.}
  \textsc{LI}. (1856). This formula for $\Gamma(z)$ had been obtained
  from Euler's formula \hardsubsectionref{12}{1}{1}) in 1848 by
  F. W. Newman, \emph{Cambridge and Dublin Math. Journal},
  \textsc{III}.  (1848), p. 60.} % TODO footnote/reference
by the equation
\begin{displaymath}
  \frac{1}{\Gamma(z)} = ze^{\gamma z} \prod_{n=1}^\infty
  \thebrace{\theparen{1 + \frac{z}{n}}  e^{-\frac{z}{n}}} ;
\end{displaymath}
\index{Weierstrassian product} from this equation \emph{it is apparent that
$\Gamma(z)$ is analytic except at the points
  $z=0, -1, -2, \ldots$, where it has simple poles}.

%\begin{smalltext} % TODO 
Proofs have been published by
H\"older$^\dagger$\footnote{\emph{Math. Ann.}  \textsc{XXVIII.}
  (1887), pp. 1-13.}, % TODO footnote/reference
Moore$^\ddagger$\footnote{\emph{Math. Ann.} \textsc{XLVIII.} (1897),
  pp. 70-74.}, % TODO footnote/reference
and Barnes$^\S$\footnote{\emph{Messenger of Math.} \textsc{XXIX}. (1900),
  pp. 122-128.} % TODO footnote/reference
of a theorem known to Weierstrass \index{Weierstrass' theorem on
  Gamma-functions} that the Gamma-function does not satisfy any
differential equation with rational coefficients.
%\end{smalltext}  

%Example 1.
\begin{wandwexample}
  Prove that

  \begin{displaymath}
    \Gamma(1) = 1, \qquad \Gamma'(1)=-\gamma,
  \end{displaymath}
  where $\gamma$ is Euler's constant.

  [Justify differentiating logarithmically the equation
  \begin{displaymath}
    \frac{1}{\Gamma(z)} = ze^{\gamma z} \prod_{n=1}^\infty
    \thebrace{\theparen{1 + \frac{z}{n}}  e^{-\frac{z}{n}}} 
  \end{displaymath}
  by \hardsectionref{4}{7}, and put $z = 1$ after the differentiations
  have been performed.]
\end{wandwexample}

%Example 2. 
\begin{wandwexample}
  Shew that
  \begin{displaymath}
    1 + \half + \frac{1}{3} + \ldots + \frac{1}{n} =
    \int_0^1 \frac{1 - (1 - t)^n}{t} \dmeasure t ,
  \end{displaymath}
  and hence that Euler's constant $\gamma$ is given by$^{||}$ \footnote{The
    reader will see later \hardsectionref{12}{2} example 4) that this
    limit may be written 
    \begin{displaymath}
      \int_0^1 \theparen{1 - e^{-t}} \frac{\dd t}{t} - 
      \int_1^\infty \frac{e^{-t} \dmeasure t}{t} .
    \end{displaymath}
  } % TODO footnote
  \begin{displaymath}
  \lim_{n \rightarrow \infty} \thebracket{\int_0^1 \thebrace{1 -
      \theparen{1-\frac{t}{n}}^n}  \frac{\dmeasure t}{t} - 
      \int_1^n \theparen{1-\frac{t}{n}}^n  \frac{\dmeasure t}{t}} .
  \end{displaymath}
\end{wandwexample}

%Example 3. 
\begin{wandwexample}
  Shew that
  \begin{displaymath}
    \prod_{n=1}^\infty \thebrace{\theparen{1 - \frac{x}{z + n}}
      e^{\frac{x}{n}}} = \frac{e^{\gamma x}\Gamma(z + 1)}{\Gamma(z - x + 1)} .
  \end{displaymath}
\end{wandwexample}

%
% 237
%

\Subsection{12}{1}{1}{Euler's formula for the Gamma-function.}

By the definition of an infinite product we have
\begin{align*}
  \frac{1}{\Gamma(z)} 
  &= z 
    \thebracket{\lim_{m \rightarrow \infty} 
    e^{\theparen{1 + \half + \ldots + \frac{1}{m} - \log m}z}
    \vphantom{\prod_{n=1}^m}} % \vphantom is here to make the brackets
                              % of the same size in both parts of the
                              % formula  
    \thebracket{\lim_{m \rightarrow \infty}
    \prod_{n=1}^m
    \thebrace{\theparen{1 + \frac{z}{n}}  e^{-\frac{z}{n}}}} \\
  &=z \lim_{m \rightarrow \infty} \thebracket{
    e^{\theparen{1 + \half + \ldots + \frac{1}{m} - \log m}z}
    \prod_{n=1}^m
    \thebrace{\theparen{1 + \frac{z}{n}}  e^{-\frac{z}{n}}}} \\
  &= z \lim_{m \rightarrow \infty} \thebracket{
    m^{-z} \prod_{n=1}^m \theparen{1 + \frac{z}{n}}} \\
  &= z \lim_{m \rightarrow \infty} \thebracket{
    \prod_{n=1}^{m-1} \theparen{1 + \frac{1}{n}}^{-z}
    \prod_{n=1}^m \theparen{1 + \frac{z}{n}}} \\
  &= z \lim_{m \rightarrow \infty} \thebracket{
    \prod_{n=1}^{m} \thebrace{
    \theparen{1 + \frac{z}{n}}
    \theparen{1 + \frac{1}{n}}^{-z}
    } \theparen{1 + \frac{1}{m}}^z} .
\end{align*}
Hence 
\begin{displaymath}
  \Gamma(z) = \frac{1}{z} \prod_{n=1}^{\infty} \thebrace{
    \theparen{1 + \frac{1}{n}}^{z}
    \theparen{1 + \frac{z}{n}}^{-1} }.
\end{displaymath}
This formula is due to Euler$^*$\footnote{It was given 
in 1729 in a letter to Goldbacb, printed in Fuss'
\emph{Corresp. Math.}}; % TODO footnote
it is valid except when $z = 0, -1, -2, \ldots$. 
\index{product for the Gamma-function}\index{Euler's product}

% Example. 
\begin{wandwexample*}
  Prove that
  \begin{displaymath}
    \Gamma(z) =
    \lim_{n \rightarrow \infty} 
    \frac{1 \cdot 2 \ldots % TODO if \cdot is a good choice here?
      (n-1)}{z(z+1) \ldots(z+n-1)} n^z . 
  \end{displaymath}
  \addexamplecitation{Euler.}
\end{wandwexample*}

\Subsection{12}{1}{2}{The difference equation satisfied by the
  Gamma-function.} 

\index{Difiference equation satisfied by the Gamma-function} We shall
now shew that the function $\Gamma(z)$ satisfies the difference
equation
\begin{displaymath}
  \Gamma(z + 1) = z \Gamma(z) .
\end{displaymath}
For, by Euler's formula, if $z$ is not a negative integer,
\begin{align*}
\Gamma(z + 1)/\Gamma(z) 
  &= 
    \frac{1}{z+1} \thebracket{\lim_{m \rightarrow \infty}
    \prod_{n=1}^{m} \frac{\theparen{1 + \frac{1}{n}}^{z+1}}{
    1 + \frac{z+1}{n}} }
    \div
    \thebracket{ \frac{1}{z} \lim_{m \rightarrow \infty}
    \prod_{n=1}^{m} \frac{\theparen{1 + \frac{1}{n}}^{z}}{
    1 + \frac{z}{n}} } \\
  &= 
    \frac{z}{z+1} \lim_{m \rightarrow \infty} \prod_{n=1}^{m} 
    \thebrace{\frac{\theparen{1 + \frac{1}{n}}(z+n)}{z + n + 1}} \\
  &= 
    z \lim_{m \rightarrow \infty} \frac{m+1}{z + m + 1} = z .
\end{align*}
This is one of the most important properties of the Gamma-function.
Since $\Gamma(1) = 1$, it follows that, if $z$ is a positive integer, 
$\Gamma(z) = (z - 1)!$.

%
% 238
%

%Example. 
\begin{wandwexample*}
  Prove that
  \begin{displaymath}
    \frac{1}{\Gamma(z+1)} + \frac{1}{\Gamma(z+2)} +
    \frac{1}{\Gamma(z+3)} +  \ldots =
    \frac{e}{\Gamma(z)} \thebrace{\frac{1}{z} - 
      \frac{1}{1!} \, \frac{1}{z + 1} + 
      \frac{1}{2!} \, \frac{1}{z + 2} 
      - \ldots } .
  \end{displaymath}
  [Consider the expression
  \begin{displaymath}
    \frac{1}{z} + \frac{1}{z(z + 1)} + 
    \frac{1}{z(z + 1)(z + 2)} + \ldots + 
    \frac{1}{z(z + 1) \ldots (z + m)} .
  \end{displaymath}
  It can be expressed in partial fractions in the form 
  $\displaystyle \sum_{n=0}^m \frac{a_n}{z + n}$, where
  \begin{displaymath}
    a_n = \frac{(-)^n}{n!} \thebrace{
      1 + \frac{1}{1!} + \frac{1}{2!} + \ldots +
      \frac{1}{(m - n)!}
    } =
    \frac{(-)^n}{n!} \thebrace{
      e - \sum_{r = m - n + 1}^\infty \frac{1}{r!}
    } .
  \end{displaymath}
  Noting that 
  $\displaystyle \sum_{r = m - n + 1}^\infty \frac{1}{r!} 
  < \frac{e}{(m-n+1)!}$, 
  prove that $\displaystyle \sum_{n=0}^m \frac{(-)^n}{n!} \, 
  \frac{1}{z+n}
  \thebrace{\sum_{r = m - n + 1}^\infty \frac{1}{r!}} 
  \rightarrow 0$ as
  $m \rightarrow \infty$ when $z$ is not a negative integer.]
\end{wandwexample*}

\Subsection{12}{1}{3}{The evaluation of a general class of infinite products.}

By means of the Gamma-function, it is possible to evaluate the general
class of infinite products of the form

n Un,

7t = l

where Un is any rational function of the index n.

For, resolving m into its factors, we can Avrite the product in the
form

* f (n -a )(n-a2) ...(n- ak) ]

n=i[ (n-b,).: n-bi) I'

and it is supposed that no factor in the denominator vanishes.

In order that this product may converge, the number of factors in the
numerator must clearly be the same as the number of factors in the
denominator, and also A 1; for, otherwise, the general factor of the
product would not tend to the value unity as n tends to infinity.

We have therefore k= I, and, denoting the product by P, we may \ vrite

 =i ] n-b ) ... n-bk)i ' The general term in this product can be
written

(- (- )(-r-(-r

a + a + ... +ak-b,- ...-bk . = 1 - - h Jin

n

where 4 is (h~-) when n is large.

In order that the infinite product may be absolutely convergent, it is
therefore necessary further \hardsectionref{2}{7}) that

tti + . . . + Ojk - 1 -   - /fc = 0.

%
% 239
%

We can therefore introduce the factor

exp /i-i ( ! + ... +ak-b,- ... - hk)

into the general factor of the product, without altering its value;
and thus we have

p= n

(i\ ), (i\ )e"...(i- *)."

"=N (- )--(- )-

But it is obvious from the Weierstrassian definition of the Gamma-
function that

,"|('" )''"]"- r(- )e-T''

and so P = r(- 6,)6,r(-6 (- .) TJ l-K) . a T -CH)...akV -ak), =
ir(l-a )

a formula which expresses the general infinite product P in terms of
the Gamma-function.

Example 1. Prove that

 " (a-i-6 + ) \ r(a + l)r (6-1-1) s=i (a+ ) (6-l-s) T a + b + ) '

Example 2. Shew that, if a = cos (27r/n) + i sin 27r jn\ then

/ \ / \ - -

 (i- j(i-|;)... = -r(-x )r(-aa- )...r(-a"-i:r ) -'.

\Subsection{12}{1}{4}{Connexion between the Gamma-function and the circular functions.}

We now proceed to establish another most important property of the
Gamma-function, expressed by the equation

r( )r(i-. )= .- .

sm TTZ We have, by the definition of Weierstrass \hardsectionref{12}{1}),

r(.)r(-.)=-in (i+i)r rnj(i-i);"

Z Sin TTZ

by \hardsectionref{7}{5} example 1. Since, by \hardsubsectionref{12}{1}{2},

r(i-z) = -zr(-z)

we have the result stated.

%
% 240
%

Corollary 1. If we assign to z the value \, this formula gives r( ) 2
= 7r; since, by the formula of Weierstrass, r ( ) is positive, we
have

Corollary "l. If r/.(s) = r'(s)/r(2), then i/ (1 -z)-- /r (0) = 7r cot
tts.

\Subsection{12}{1}{5}{The multiplication-theorem of Gauss* and Legendre.}
We shall next obtain the result

r(.)r(. + )r(. + )...r(. + 'i ) = (2.)n -. - -rM.

n) \ n For let ( (.) = iT;; \ .

Then we have, by Euler's formnla \hardsubsectionref{12}{1}{1} example),

 - 1 .2...(m- l).m - /'*

n'"- n lim

 (Z)

,. 1 . 2 ... (m - 1) . (nm)'*

n lim

\ im - 1)! m

nz (nz + 1 ) . . . (71Z + n7n - 1)

 n'ln - 1) ! nm)'" i(m- 1) !j wi" 'n

m oo (wm-1)!

It is evident from this last equation that <p (z) is independent of z.
Thus (j) (z) is equal to the value which it has when z = -; and so

Therefore < (z)Y = n| | T ( -) T (l -;:)|

tt' -i (27r)"-

. TT . 27r . (?i- l)7r

sm - sm - ... sm

n n n

Thus, since cj) (n~ ) is positive,

<,W = (2,r)4<"-'' -*,

i.e. r (.) r (. + ) . . . r (. + '-i i) = - (2,r)4 < - '> r ( .).

Corollary. Taking n = 2, we have

2 -ir(.)r( +i)=,r4r(24

This is called the duplication formula.

* Werke, iii. p. 149. The case in which n - 2 was given by Legendre.

%
% 241
%

Example. If (, o) = EMlM

shew that

D(p,,)b p + \ q)...B(p ',q) B (np, no) = a ~ i \ n /

\Subsection{12}{1}{6}{Expansions for the logarithmic derivates of the Gamma-function.}

We have r (2+l) -J = '>' n /("l + -") e~ ]- .

DifiFerentiating logarithmically \hardsectionref{4}{7}), this gives

l gJ iiiLi)= \ + !\ + --,,

rfs ''' 1 (2 + 1) '*"2(2 + 2) " 3(2 + 3) ""

Therefore, since log r (s + 1 ) = log 2 + r (2), we have

Iogr(2)=-y--.+ 2 2

dz ° ' z n=in(2 + n)'

d-. \,,. (f

Differentiating again,,logr(.+ l) = + 2 (2 + 2) + -

1 1

 (2+l>' (2 + 2)2 ""

These expansions are occasionally used in ai)plications of the theory.
\Section{12}{2}{Elder s expression of T z) as an injinite integral.}

The infinite integral I e~H ~ dt represents an analytic function of z
when*

Jo

the real part of z is positive \hardsubsectionref{5}{3}{2}); it is called the Eulerian
Integral of the Second Kindf. It will now be shewn that, when li z)>0,
the integral is equal to F (z). Denoting the real part of z by x, we
have x > 0. Now, if J

n(z,n)=j'Ul-yJ\'- dt,

we have U (z, n) = n l (1 - TyW-'hlr,

J )

if we write t= nr; it is easily shewn by repeated integrations by
parts that, when a; > and n is a positive integer,

f n 1 n; !

 1 - t)"t - dr = - T- (1 - Tf + - (I - Tf-'T'dr Jo \ J Jo S'.

n(n-l)...l fi,

z(5 + l)...( + n-l)Jo '* '

J n / \ 1 . 2 . . . ?i

and so 11 (2, n) = -. - n

z z ) ... z -n)

Hence, by the example of \hardsubsectionref{12}{1}{1}, 11 z, n) r( ) as /i -* x .

* If the real part of z is not positive the integral does not converge
on account of the singu- larity of the integrand at f = 0.

t The name was given by Legendre; see \hardsectionref{12}{4} for the Eulerian
Integral of the First Kind.

t The many-valued function < ~i is made precise by the equation *~i =
c(*-i)'°s', log r being purely real.

W. M. A. 16'

%
% 242
%

Consequently F ( ) = lim (1 ) t-~ dt.

And so, if T, (2) = I e-H'-'dt,

J (\ we have

ri( )-r( )= lim r ie-'-(l- y\ t'-'dt+j e-H'-'dt .

Now lim I e'U'-'dt = 0,

since I e~H ~ dt converges.

.'0

To shew that zero is the limit of the first of the two integrals in
the formula for Fj z) - F(z) we observe that

[To establish these inequalities, we proceed as follows : when < y <
1, from the series for e and (1 - 3/)" Writing tjn for y, we have

1 + - e-'

n

HJ-

and so 0 e- -fl--

71

= e-Ml-eM 1

<e-Ml- 1

-1)1

Now, if O-Sa l, (1 - a)" l - na by induction when ?ia<l and obviously
when 7ia 1; and, writing t /n for a, we get

and so* 0 e-*-(l--) %e-H ln,

which is the required result.]

From the inequalities, it follows at once that

r' - f 1 - - j ) t'-'dt ' I n-'e-ff'- 'dt

<n-'j e-H' - 'dt O, Jo

as n - X, since the last integral converges.

* This analysis is a modification of that given by Schloinilch,
Compendium der hoheren Analysis, ii. p. 243. A simple method of
obtaining a less precise inequality (which is sufficient for the
object required) is given by Bromwich, Infinite Series, p. 459.

%
% 243
%

Consequently Fi 2)= V (z) when the integral, by which Tj (z) is
defined, converges; that is to say that, when the real part of z is
positive,

J n And so, when the real part of z is positive, F (z) may be defined
either by this integral or by the Weierstrassian product. Example 1,
Prove that, when R (z) is positive,

Example 2. Prove that, if li z) > and R (s) > 0,

/ e-"=x -'cla.-= . Jo z

Example 3. Prove that, if R z)>Q and R is) > 1,

Example 4. From :; 12'1 example 2, by using the inequality

0%e- -( X fie- ln,

deduce that

l\ e- \ e-i/<

-at.

<

t

\Subsection{12}{2}{1}{Extension of the infinite integral to the case
  in which the argument of the Gamma-function is negative.}

The formula of the last article is ni> longer applicable when the real
part of z is negative. Cauchy* and Saalschiitzt have shewn, however,
that, for negative arguments, an analogous theorem exists. This can be
obtained in the following way.

Consider the function

r, z)=j\'- (e-'-\ + t-, + ... + -)>'- dt,

where k is the integer so chosen that -k>x>-k-l, x being the real part
of z. By partial integration we have, when s < - 1,

r,(.)=[f(.-.-n.<-il + ...+(-r.,)];

+.'j7''( -'-i+'--+(-''(j )*

The integrated part tends to zero at each limit, since x+k is negative
and x + k + l is positive : so we have

r, z) = lT, z+i).

The same proof applies when x lies between and -1, and leads to the
result

r z+i)=zr2 z) (0>.r>-i).

The last equation shews that, between the values and -1 of a*,

To z) = r(z).

* Exercices de Math. u. (1827), pp. 01-92.

t Zeitschrift fur Math, und Phys. xxxii. (1887), xxxiii. (1888).

16-2

%
% 244
%

The preceding equation then shews that To z) is the same as T (2) for
all negative values of Ji z) less than -1. Thus, for all negative
values of R(z), we have the result of Cauchy and Saalschiitz

where k is the integer next less than - B (z).

Example. If a function P (ft) be such that for positive values of /x
we have

J

- 1 -x

e~'-' dx,

and if for negative values of /x we define Pj (fi) by the equation

dx.

Pi(;u)=|\ i'--i(e- -l + .r-... + (-)*- + i,)

where k is the integer next less than - ju, shew that

/',W./'W-l + -... + (-)'--.-,. \addexamplecitation{Saalschiitz.}

\Subsection{12}{2}{2}{Hankers expression of TODO as a contour integral.}

The integrals obtained for T z) in
\hardsectionref{12}{2}, \hardsubsectionref{12}{2}{1} %TODO:cite multiple
are members of a
large class of definite integrals by which the Gamma-function can be
defined. The most general integral of the class in question is due to
Hankel*; this integral will now be investigated.

Let D be a contour which starts from a point p on the real axis,
encircles the origin once counter-clockwise and returns to p.

Consider 1 (-t)~~ e~hlt, when the real part of z is positive and z is
not an integer.

The many-valued function (- ty~ is to be made definite by the
convention that (- ty~ = e' ~ ' ° '"*' and log (- t) is purely real
when t is on the negative part of the real axis, so that, on D, - ir %
arg (- t) % it.

The integrand is not analytic inside D, but, by \hardsectionref{5}{2} corollary 1, the
path of integration may be deformed (without affecting the value of
the integral) into the path of integration which starts from p,
proceeds along the real axis to h, describes a circle of radius h
counter-clockwise round the origin and returns to p along the real
axis.

On the real axis in the first part of this new path we have arg (- ) =
- tt, so that - ty~' - e~ '' ~ H ~ (where log i is purely real); and
on the last part of the new path (- ty- = e " ' - ' t ' .

On the circle we write - i = 8e*; then we get

f -ty-'e-Ht-= \ e-'-(--i) '-ie-'f? +f"(8e' )--ie (cose+;sm< )gg,ej 7

+ ''e' ' -' H'-' e' dt = - 2i sin -jTz) j t'-' e-hit + ih' I " e<>e+
(cose+tsine) \

* Zeitschrift filr Math, und Phys. ix. (1864), p. 7.

%
% 245
%

This is true for all positive values of S p; now make S; then 8' and
I gize+s (COS 0+isin e 0 j ize iQ [ \ \ q integrand tends to its limit

J - TT J -IT

uniformly.

We consequently infer that

[ (-ty-'e-'dt = -21 sin 7r2)[''t'-'e-*dt.

J B Jo

This is true for all positive values of p; make p x, and let C be
the limit of the contour D.

Then f (- ty-'e- dt = - 2i sin ttz) f t'-'e- dt.

J c Jo

Therefore T ( ) = - r 4 I (- tf-'e-'dt.

'2i sm TTzJ c

Now, since the contour C does not pass through the point t = 0, there
is no need longer to stipulate that the real part of 2 is positive;
and

I (-ty~ e~ dt is a one-valued analytic function of z for all values of
2.

J c

Hence, by \hardsectionref{5}{5}, the equation, just proved when the real part 0/2 is
positive, persists for all values of z with the exception of the
values 0, ±1, +2

Consequently, for all except integer values of,

r(0) = -\ J - I -ty-'e-'dt.

This is Hankel's formula; if we write 1 - 2 for z and make use of
\hardsubsectionref{12}{1}{4}, we get the further result that

We shall write / for \, meaning thereby that the path of inte-

1 re C

gration starts at 'infinity' on the real axis, encircles the origin in
the positive direction and returns to the starting point.

Example 1. Shew that, if the real part of z be positive and if a be
any positive

constant, l - t)~ e~ dt tends to zero as p -ao, when the path of
integration is either of

the quadrants of circles of radius p + a with centres at - a, the end
points of one quadrant being p and - a + 1 (p -|- a), and of the other
p and -a - i p- a).

V

' (0 + )

%
% 246
%

Deduce that lim I " ' t)-' e- dt= lim I -t)-'e- dt,

p-*xy-a + 'P p- y J C

and hence, by writing t= -a-iu, shew that

\ \ ['

--- = -- I e "*- " (a + <)"* f M- r (s) ztt y \ x

[This formula was given by Laplace, Theorie Analytique des
Prohahilites (1812), p. 134, and it is substantially equivalent to
Hankel's formula involving a contour integral.]

Example 2, By taking a = 1, and putting t= -\ + i tan B in example 1,
shew that -J- = - ( "" cos (tan (9 - 2(9) cos - 2 (9c?<9.

r (2) 77 Jo

Example 3. By taking as contour of integration a parabola whose focus
is the origin, shew that, if a > 0, then

Y z)=-. - - e- ''(l + -) -Jcos 2a + (22-l)arctani; (/i!. sin nz J

\addexamplecitation{Bourguet, Acta Math, i.}

Example 4. Investigate the values of x for which the integral

2 r*

 - sin tdt

TT J

converges; for such values of x express it in terms of
Gamma-functions, and thence shew that it is equal to

\addexamplecitation{St John's, 1902.}

Example 5. Prove that I (log t)" ' dt converges when m > 0, and, bv
means

Jot

of example 4, evaluate it when m=l and when m = 2.
\addexamplecitation{St John's, 1902.}

\Section{12}{3}{Gauss' expression for the logarithmic derivate of the
  Gamma-function as an infinite integral*TODO.}

We shall now express the function -7- log V (z) = as an infinite

integral when the real part of z is positive; the function in
question is frequently written yjr (z). We first need a new formula
for 7.

Take the formula \hardsectionref{12}{2} example 4)

  Jo i J\ i 5 0 Us J t / a olJA J\& t J'

where A = l-e- since | - = log .- O as 8-*-0.

./a l-e"

 Yriting = 1 -e~" in the first of these integrals and then replacing u
by t we have

y= lim I r,dt- r '- dt] = r \ j~,-]] e- dt.

5-*o Us l-e Js t J Jo U-e ' t)

This is the formula for y which was required.

* Wtr'ke, HI, p. 159.

%
% 247
%

To get Gauss' formula, take the equation \hardsubsectionref{12}{1}{6})

r'( ) 1 .. /I 1 \

T z) ' z acm=iVm z-vm)

1 r

and write = e-'( +'">rf;

z - m Jo

this is permissible when m = 0, 1, 2, ... if the real part of z is
positive.

It follows that

 '=-7- I e- ? + lim S (e-""-e-(' + )')f <

.0 rt%\rightarrow \inftyJowi = l

r( )

-7+ lim = -I dt

1-

V 1 - e-'

W - lim 1 -e-(''+ '(/t

1-e-

Now, when < < 1, i is a bounded function of t whose limit as - [0 is
6nite;

I 1 - e 'I

and when t 1, - | < - J- < - .

Therefore we can find a number A' independent of t such that, on the
path of integration,

I 1 - ' I

andso I r\ zl~,-K ' Vdt\ < K[ e-i"*')' rf = /r(H + l)-' 0 as /i- -x .

I jo 1-e-' ./o

We have thus proved the formula

t< ) = s'" ' *'->=.C(T-r ')'"-

which is Gauss' expression of - z) as an infinite integral. It may be
remarked that this is the first integral which we have encountered
connected with the Gamma-function in which the integrand is a
single-valued function.

Writing <=log (1 -f-.i') in Gauss' result, we get, if A=(J - 1,

 -if=limf t'-,,U.

since < I - dt < j y =log - g- 0 as 8 0.

T'(z),. / " f 11 rf.r

Hence \ = uaj |,-.\ \ \ \ |\,

sothat ''<=)=rM/J e-'-(, §',

an equation due to Dirichlet*.

Werke, i. p. 275.

%
% 248
%

Example 1 . Prove that, if the real part of z is positive,

Example ± Shew that y=\ l -t)-' -e- ]t-' dt. \addexamplecitation{Dirichlet.}

\Subsection{12}{3}{1}{Binet's first expression for TODO in terms of an infinite integral.}

Binet* has given two expressions for logr( ) which are of
great

importance as shewing the way in which log V z) behaves as ] 2 j -* oo
. To

obtain the first of these expressions, we observe that, when the real
part of

z is positive,

r' /'-r \ L 1 \ r* (o-t o-tz ]

dt,

r( + i) Jo [ e -i

writing z + \ for 2 in \hardsectionref{12}{3}.

Now, by \hardsubsubsectionref{6}{2}{2}{2} example 6, we have

log2=) - r~ '

f" 1 and so, since (22 )" = e~ dt,

'Jo

we have

dz

logr(. + l) = l + log.-/"g-;+ - Je-*.

The integrand in the last integral is continuous as i -; and since

- - - + -f - z. is bounded as - 00, it follows without difficulty
that the

integral converges uniformly when the real part of z is posijiive; we
may consequently integrate from 1 to under the sign of integration
(\hardsubsectionref{4}{4}{4}) and we get-f-

iogr(.+i) = (. + |)iog.-.H-i+/;g- + p-- -

dt.

Since - - + - - i 7 continuous as i - by \hardsectionref{7}{2}, and since

log r (z + 1) = log + log r (z),

have

/ 1\ (" CI 1 1 ) e-*

iogr(.) = (.-2)iog.-. + i.-j |--- + - \ -- |- .

J (2 t e' - l)t

* Journal de I'Ecole Poly technique, xvi. (1839), pp. 123-143.

t Logr(2 + l) meaus the sum of the principal values of the logarithms
in the factors of the Weierstrassian product.

%
% 249
%

To evaluate the second of these integrals, let*

so that, taking z = in the last expression for logr(2:), we get

i log 7r = i + t/-/.

( + - ) - - dt, we have

t

r ' 1 \ dt

-jo V t e'- ) t

=/:r- '-HT

~jo 1 c/ V t ) t ' '2tj

* - rf<

t

= 2 + Uogi Consequently /=1- log(27r).

We therefore have Binet's result that, when the real part of z is
positive,

3-tZ

 -dt.

log r( ) = [z- \ ogz-z + \ log(27r) + J G ~ 7 + "'-t)

li z = x- iy, we see that, if the upper bound of ( ~ 7 " t \ i ) 7 ' *
' values of is K, then

logr( )-( -i)log + -|log(27r)'<i(:| "'"

so that, when x is large, the terms (z - \ og z - z \ o (2'jr) furnish
an

approximate expression for log V (z).

Example I. Prove that, when A' (a) > 0,

logr(.-)= r *i4.1-r+( -l) "j 7- \addexamplecitation{Malmsten.}

Example 2. Prove that, when R (z) > 0,

* This artifice is due to Priagsheim, Math. Ann. xxxi. (1888), p. 473.

%
% 250
%

Example 3. From the formula of \hardsubsectionref{12}{1}{4}, shew that, if <;r < 1,

2logrW-log.+logsin..-=/; |-' Ui=fl-'-(l-2.).-. f .

\addexamplecitation{Kummer.}

Example 4. By expanding sinh (| - .v) and 1 - 2x' in Fourier sine
series, shew from example 3 that, if < .v < 1,

00

log r (.p) = i log TT - - log sin -nx + 2 2 a,i sin 2?i7r,

, / " r 2?i7r e-n dt

" jo V + 4wV2 27rJ i! Deduce from example 2 of \hardsubsectionref{12}{3} that

"*" "" 2 " "*" 2"" + * g ''* )

(Kummer, Journal fiir Math. xxxv. (1847), p. 1.)

\Subsection{12}{3}{2}{Binet's second expression for log V z) in terms of an infinite integral.}

Consider the application of example 7 of Chapter vii (p. 145) to the
equation \hardsubsectionref{12}{1}{6})

The conditions there stated as sufficient for the transformation of a
series into integrals are obviously satisfied by the function ( )= -
r-, if the real part of z be positive; and we have

Since !5'(, + /i) I is easily seen to be less than K t/n, where Ki is
inde- pendent of t and n, it follows that the limit of the last
integral is 'zero.

Hence V;: log F ( ) = 2,- -I i - -, - -.

dz 2z- z Jo z + t'f e- - 1

I 2 I Since ~ - - does not exceed K (where K depends only on 8) when
the

I I*

real part of z exceeds 8, the integral converges uniformly and we may
integrate under the integral sign \hardsubsectionref{4}{4}{4}) from 1 to z.

We get

-I log r (.) = -;+ log . + - 2 /; . JL,

where C is a constant. Integrating again,

log r (.) = (. - 1) log . + (c - 1) . + c + 2/; '- dt.

where 6" is a constant.

%
% 251
%

Now, if z is real, arc tan tjz \$ tlz,

and so

logrW-( -i)log -(C-l).-C"|<?/" d(.

But it has been shewn in \hardsubsectionref{12}{3}{1} that

\ \ % z)-[z-- \ ogz- z- \ og 1'K) -0,

as 2 -* X through real values. Comparing these results we see that C =
0, 6"=ilog(27r).

Hence for all values of z whose real part is positive,

logrW = (.-l)log.-. + llog(2.) + 2|; -I rf*.

where arc tan u is defined by the equation

arc tan xi = |,

.'o l+<-

in which the path of integration is a straight line.

This is Binet's second expression for log V z).

Example. Justify differentiating with regard to z under the sign of
integration, so as to get the equation

12 33. The asymptotic expansion of the logarithm of the Gamma-
function (Stirling's series).

We can now obtain an expansion which represents the function log F z)
asymptotically \hardsectionref{8}{2}) for large values of \ z\, and which is used in
the calculation of the Gamma-function.

Let us assume that, if = x -f iy, then a-' S >; and we have, by
Binet's second formula,

log r ( ) = ( - ) log - + - log (27r) + </) ( ),

where (.) = 2/; c/.

Now

(\ )n-i 2,1-1 (\ y rt u n

arc

,,,,, t It' It' (\ )n-i 2n-i (\ )H rt

u + z' Substituting and remembering \hardsectionref{7}{2}) that

Jo e' ' -l ~4w'

%
% 252
%

where Bi, Bo, ... are Bernoulli's numbers, we have

  / X V (-)' "' Br 2 (-)'* r " f f U' ' du \ dt

.=1 2r (2?- - 1) '--l 271-1 j \ [ ! () 2 2 + 2 j g2,r< \ I

Let the upper bound * of - 1 for positive values of u be K,

j w + j

Then

ftl

f u' 'du] dt 1 r, I, r" 1 r, ) (

uow + ' j e-' '-li" ' ' Jo [Jo \ e'-'-l

Ji-z n+1

 4(n+l)(2M+l)| |2" Hence

2(-y r ( f* u l \ d KA+

z "-' Jo Vou' + z e -l 2 n + l) 2n + l)\ z p +i '

and it is obvious that this tends to zero uniformly as | | - oo if I
arg | tt - A, where tt > A > 0, so that K cosec 2 A.

Also it is clear that if ] arg r | Itt (so that Kz = l) the error in
taking the first 71 terms of the series

I (-y-'Br 1

rti 2r (2r - 1) z -' as an approximation to <f) (z) is numerically
less than the (n + l)th term. Since, if | arg z\ \ 7r - A,

-,211-1

!</> ( ) - i Z ' \ I < cosec 2 A .,

r r=i 2r(2r-l)l; 2(n + l)(2,

2(n + l)(2n + l)

-0,

as 00, it is clear that

B, B, B,

1.2.2 3. 4. 2=* ' 5.6.2 "* is the asymptotic expansion f \hardsectionref{8}{2}) of
(2). We see therefore that the series

is the asymptotic expansion of log V z) when | arg | tt - A.

-Kj 18 the lower bound of -!, .j and is consequentlj' equal to

4 2y2,

, 9, .n o or 1 as x2<m2 or x >y . t The development is asymptotic;
for if it converged when | 2 | p, by \hardsectionref{2}{6} we could find K, such that
B <(2;i-l)2nA>2"; and then the series 2 ~ "~' " " would define an
integral function; this is contrary to \hardsectionref{7}{2}.

%
% 253
%

This is generally known as Stirling s series. In \hardsectionref{13}{6} it will be
estab- lished over the extended range | arg tt - A.

In particular when z is positive (= x), we have

r u? du'] dt Bn+.

Jo Uo w + j e ' '-l

w + je ' '-l 2 n + l) 2n + l)af

Hence, when x>0, the value of < )(oc) always lies between the sm7i of
n terms and the sum ofn + 1 terms of the series for all values of n.

D g

In particular < (f> (x) < - -, so that <f> (./) = y where < < 1.

Hence T x) = x'- - e'' i2'rrf e ' 'l

Also, taking the exponential of Stirling's series, we get

\ x .r-, a f 1 1 139 571 /'I

r ix) = e X ~ (27r) -,1 - 288 ~ 51840 ~ 2488320 "*" V

This is an asymptotic formula for the Gamma-function. In conjunction
with the formula T x + l) = xr x), it is very useful for the purpose
of com- puting the numerical value of the function for real values of
x.

Tables of the function logioT (. ), correct to 12 decimal places, for
values of .v between 1 and 2, were constructed in this way by
Legendre, and published in his Exercices de Calcul Integral, ii. p.
85, in 1817, and his Traite des fonctions elliptiques (1826), p. 489.

It may be observed that V (x) has one minimum for positive values of
.>;, when .r= 1-4616321..., the value of log,or(.r) then being
1-9472391....

Example. Obtain the expansion, convergent when R z) > 0,

\ og,T z) = z-l)\ og, z- z+ \ og, 27r) + J (z),

where in which

- W- 5 |, + | + 2 (- +1) (. + 2) 3 (.-+1) (.- + 2) z + S) -

and generally

c = r (x+l) x + 2) ... x + n- ) 2x - ) xdx. \addexamplecitation{Binet.}

J

\Section{12}{4}{The Eulenan Integral of the First Kind.}

The name Eulerian Integral of the First Kind was given by Legendre to
the integral

B (p, q) = f xP-' (1 - x)'i-' dec,

J

which was first studied by Euler and Legendre*. In this integral, the
real parts of p and q are supposed to be positive; and xp~, (1 - x) ~
are to be understood to mean those values of e( ~ ) °°* and
e('v-i)'°e(i- =) which correspond to the real determinations of the
logarithms.

* Euler, Nov. Comvi. Petrop. xvi. (1772); Legendre, Exercices, i. p.
221.

%
% 254
%

With these stipulations, it is easily seen that B (p, q) exists, as a
(possibly improper) integral \hardsectionref{4}{5} example 2).

We have, on writing (1 - *) for w,

B p,q) = B q,p).

Also, integrating by parts,

Jo L i Jo PJo

so that B p,q + l) = B p+l,q).

Example 1. Shew that

B p,q)= £ip + l,q) + B p,q + l).

Example 2. Deduce from example 1 that

B P,<l + )= B p,q).

Example 3. Prove that if n is a positive integei*,

,,, 1 . 2 ... M

B p,n + ) = -

p p + l)... p + n)\ Example 4. Prove that

Example 5. Prove that

r (2) = lim n B z, n).

\Subsection{12}{4}{1}{Expression of the Eulerian Integral of the First
  Kind in terms of the Gamma function.}

We shall now establish the important theorem that

R/.,, x\ r(m)r(n)

First let the real parts of m and n exceed |; then

r (m) r (n) = e-* '"-i c a; x g-?' if'- dy. Jo Jo

On writing a;'- for .t, and y for y, this gives

fR rR

V (m) r (n) = 4 lim e"*' x'-""-' dx x g-?/' /-"-i dy

= 4 lim e-( '+2'') 2m-iy2n-i( ( \

Now, for the values of m and n under consideration, the integrand is
continuous over the range of integration, and so the integral may be
con- sidered as a double integral taken over a square Sji. Calling the
integrand

%
% 255
%

f x, y), and calling Qg the quadrant with centre at the origin and
radius R, we have, if Tji be the part of S, outside Qb,,

I f(x, y) dxdy - f oc,y) dxdy JJsr JJqr I

= 1 f(a:,y)dxdy

 JJ Tr

i \ f >y)ida dy-\ \ \ f,y)dxdy\

JJSr JJ Sir

  . ., .,

- 0 as R X,

since 1 1 |/(;, y) \ dxdy converges to a limit, namely

JJ Sr

Therefore

Imi

2 I e- '-, x""-' idxx2\ e' \ y-''-' \ dy. Jo' Jo

I f(x,y)dxdy= lim // f x,y)dxdy.

Changing to polar* coordinates x = rcos 0, y = r sin 6), we have f x,
y) dxdy = |

Oh

Hence

1 1 f x, y) dxdy = ( \ e-"" (r cos 0) '"-' (r sin 0)-''-' rdrd0.

J J Or J J

r (w) r ( ) = 4 e-r' r-('"+")-i dr cos=" -i sin ' - c? Jo Jo

= 2r ( i + ?z) cos '"-' sin ''- <9c? .

Jo

Writing cos- = u we at once get

r (m) r (?i) = r (?u + ) . z (w, ?i).

This has only been proved when the real parts of ni and n exceed |;
but it can obviously be deduced when these are less than | by TODO
example 2.

This result, discovered by Euler, connects the Eulerian Integral of
the First Kind with the Gamma-function.

Example 1. Shew that

[' (l-f-A-)' -i(l-.r) -i( =2P + 9-i Mli2).

  -I ' ' ip+q)

  * It is easily proved by the methods of \hardsubsectionref{4}{1}{1} that the areas A /x of
  \hardsectionref{4}{3} need not be rect- angles provided only that their greatest
diameters can be made arbitrarily small by taking the number of areas
sufficiently large; so the areas may be taken to be the regions
bounded by radii vectores and circular arcs.

%
% 256
%

Example 2. Shew that, if

j\ isj .v x + 1 2! .r + 2 3! .r + d

then

f x,y)=f y + \,x-\ \

where and y have such vakies that the series are convergent. (Jesus,
1901.)

Example 3. Prove that

j'J'j xy) l-xT- y> i yr- dxdy = - > 'j

\addexamplecitation{Math. Trip. 1894.}

\Subsection{12}{4}{2}{Evaluation of trigonometrical integrals in terms of the Gamma-function.}

We can now evaluate the integral cos" ~ x sm' ~ xdx, where m and n

Jo

are not restricted to be integers, but have their real parts positive.
For, writing cos o; = t, we have, as in \hardsubsectionref{12}{4}{1},

I'*" .,, IF am) ran)

cos' - cc sin" -1 xdx = f. ' - .

The well-known elementary formulae for the cases in which m and n are
integers can be at once derived from this result.

Example. Prove that, when \ k\ < .\,

fh CDS'" e sin' edd \ T (lm+ )T /i+ ) fh cos'" -'"Odd J \ l ksm 6)i ~
Tlh-a + hi+l) 7r J (l - k sin2 )i + *

\addexamplecitation{Trinity, 1898.}

\Subsection{12}{4}{3}{Pochhammers* extension of the Ealerian Integral of the First Kind.}

We have seen in \hardsubsectionref{12}{2}{2} that it is possible to replace the second
Eulerian integral for F(z) by a contour integral which converges for
all values of z. A similar process has been carried out by Pochhammer
for Eulerian integrals of the first kind.

Let P be any point on the real axis between and 1; consider the

integral

r(i+,o+, 1-, 0-) e-'"'<' + ) t - (1 - 0 -' dt = e a, /3).

J F

The notation employed is that introduced at the end of \hardsubsectionref{12}{2}{2} and
means that the path of integration starts from P, encircles the point
1 in the positive (counter-clockwise) direction and returns to P, then
encircles the origin in the positive direction and returns to P, and
so on.

, * Math. Ann. xxxv. (1890), p. 495.

%
% 257
%

At the starting-point the arguments of t and - t are both zero; after
the circuit (1 +) they are and l-ir; after the circuit (0 +) they are
Itt and 27r; after the circuit (1 - ) they are lir and and after the
circuit (0 - ) they are both zero, so that the final vahie of the
integrand is the same as the initial value.

It is easily seen that, since the path of integration may be deformed
in any way so long as it does not pass over the branch points 0, 1 of
the integrand, the path may be taken to be that shewn in the figure,
wherein the four parallel lines are supposed to coincide with the real
axis.

>

// the real parts of a. and y9 are positive the integrals round the
circles tend to zero as the radii of the circles tend to zero*; the
integrands on the paths marked a, h, c, d are

 a-Ig-'W(a-l) (1 \ )3-lg2.r<(8-l) a-l g2,r.(a-]) ( \ )fl-i

respectively, the arguments of t and 1 - nuiu being zero in each case.

Hence we may write e (a, ) as the sum of four (possibly improper)
integrals, thus :

e (a, /3) = e- '(''+ )

I t -' (1 - 0 "' + I t"-' (1 - 0 ~'e-''

 dt

C 1 /""

+ t"-' (1 - tf-' e -'t"* ) dt + -' (1 - tf-'e- ' dt . .' 1

Hence

€ (a, /3) = e- '(' - ) (1 - e---) (1 - e ' ) f -' (1 - tf-' dt

JO

, ., ., l (a)r(/9) = - -isin (a7r)sin (ott) ~

- 47r* "r(l-a)r(l-/: )r(a + )-

Now e (a, /S) and this last expression are analytic functions of a and
of /3 for all values of a and /S. So, by the theory of analytic
continuation, this equality, proved when the real parts of a and are
positive, holds for all values of a and l3. Hence for all values of a
and /3 lue have proved that

/ J -47r-

r i-cc)l\ l-id)l\ a+id)- ' The reader ought to have uo difficulty in
proving this.

AV. M. A.

17

%
% 258
%

\Section{12}{5}{Dirichlet's integral*TODO.}

We shall now shew how the repeated integral

/ = f f . . . I f t, + t,+ ...+ tn) i"'- o" - . . . tn' n-idt dt, . .
. dt

may be reduced to a simple integral, where/is continuous, a > (?* = 1,
2, ... n) and the integration is extended over all positive values of
the variables such that 1 + 4+ ... + tn \$1.

To simplify p'M f\ t + T+ ) t' -'T -'dtdT

Jo Jo

(where we have written t, T, a, j3 for t, t, cui, Og and X for 3 + 4
+ ... +tn), put t = T \ - v)/v; the integral becomes (if X. 0)

r~U' /(x + Tlv) l - vy-' V-''-' r-+3-i dvdT.

Jo J T/(l- )

Changing the order of integration \hardsubsectionref{4}{5}{1}), the integral becomes

( [ V( + T/v) l- vy-'v''-' T- -'dTdv. J J

Putting T = VTo, the integral becomes

I i /( + " 2) (1 - vy-' v -' r. + -' dr., dv J J

r (a)r(8) r -

r( + /3) Jo

Hence

  J|... j/(T. + 3+ ... +Qt/.+ -i 3" -i ... tn ' -'dr.dt, ... dtn,

r(,)r(,) r(e

the integration being extended over all positive values of the
variables such that T2 + ts + ... +tn- .

Continually reducing in this way we get

r((x, + a,+ ...+an) Jo- ' which is Dirichlet's result. Example 1.
Reduce

to a simple integral; the range of integration being extended over
all positive values of the variables such that

it being assumed that a, 6, c. a,, y, p, q, r are positive.
\addexamplecitation{Dirichlet.}

* Werkt, I. pp. 375, 391.

%
% 259
%

Example 2. Evaluate / / x yi dxdy,

m and n being positive and

' >0, 2/ 0, x>" + y " 1 . ( Pembroke, 1 907 . )

Example 3. Shew that the moment of inertia of a homogeneous ellipsoid
of unit density, taken about the axis of z, is

where a, 6, c are the semi-axes.

Example 4. Shew that the area of the epicycloid x-' +y =P is fn-r-.

REFERENCES. N. Nielsen, Handhnch der Theorie aer (Jamma-funktion*.
(Leipzig, 1906.) 0. ScHLOMiLCH, Compendium der ho her en Analysis, ll.
(Brun.swick, 1874,) E. L. LiNDELOF, Le Calcul des Residus, Ch. iv.
(Paris, 1905.) A. Pringsheim, Math. Ann. xxxi. (1888), pp. 455-481.

Miscellaneous Examples.

1. Shew that

\addexamplecitation{Trinity, 1897.}

2. Shew that

.,'i"i vr. rrp rTr.-i7i;""='"<"+"- '™'"' ' ' '>

3. Prove that

r'(i) T' (is)

f(i) ~ r(|) = S - (' ® " ' 2-)

4. Shew that

 r(i)! 32 52-1 72 92-1 IP . .

"16;; " = 3  5  r T  92  i -ZTi  \addexamplecitation{Trinity, 1891.}

5. Shew that

- f ( -a)(M+ + y ) a \ \ 1 .,, .,\,,

n \ - - --7-7 -; - r 1+, H- = - - sm (an) B (/3, y).

 =o I ( +/:<) ( +y) \ + l/j TT ' " '

\addexamplecitation{Trinity, 1905.}

6. Shew that ( ) ('Isf ' \addexamplecitation{Peterhouse, 1906.}

7. Shew that, if z = iC where is real, then

l l=\/(fSVc)- \addexamplecitation{Trinity, 1904.}

8. When x is positive, shew thatt

r (x) r (h) °° 2n ! 1

 . ., Tx = 2 -; - - - . \addexamplecitation{Math. Trip. 1897.}

* This work contains a complete bibliography.

t This and some other examples are most easily proved by the result of \hardsubsectionref{14}{1}{1}.

17-2

%
% 260
%

9. If a is positive, shew that

r( )r( +i) i ( -)" ( - 1) ((f - 2) ...(g-H) 1 r(2+ ) =o ! s+ '

10. If .V > and

JO

shew that

and

11. Shew that if X > 0,,c> 0, - tt <a <\ tt, then
$$
TODO
$$
\addexamplecitation{Euler.}

12. Prove that, if 6 > 0, then, when < s < 2,

/ - 7" dx = \ u\ f- cosec (J7rs)/r (s), y - and, when < s < 1,

/ * cosfto; \ J (a. .)/p (5), \addexamplecitation{Euler.}

\ / .

13. If < /i < 1, prove that

jj(l+.t'-cos -rf.r=r(,0 cos (f -1) - +,

\addexamplecitation{Peterhouse, 1895.}

14. By taking as contour of integration a parabola with its vertex at
the origin, derive from the formula

1 r** "")

2 sin an \ the result

1 f

r(a)= - -. / e- ''x"--' l+:c''-) [Ziim x + a?(.YCcot(-Jc)

2 sui utt J )

+ sin x + a - 2) arc cot ( - x) ] dx, the arc cot denoting an obtuse
angle.

(Bourguet, Acta Math. i. p. 367.)

15. Shew that, if the real part of a-n is positive and 2 l/a is
convergent, then

n = l

d' is convergent when m > 2, where \//-( ) (2) = -r log r (z). (Math.
Trip. 1907.)

16. Prove that

"jo 1'

d\ ogr z)\ /"e- g-e-g'

< 2

= r (l+a)-i-(l+a)- ---), Jo a

/ 1 .2 - 1 \ ] = / dx-y. \addexamplecitation{Legendre.}

Jo x-l

%
% 261
%

17. Prove that, when R z)>0,

\ ogr z)=r i- -xiz-l)] f- . \addexamplecitation{Bmet.}

18. Prove that, for all values of z except negative real values,

log r (2) = (3 - 1) log 2 - 2 + Uog (27r)

J 1 1 2 J; \ 3\ = J ]

19. Prove that, when (2) > 0,

-rlogr(2) = log2- / r j l-a.-+log.r .

dz ° ° Jo (l-.r)log.r

20. Prove that, when R (2) > 0,

c/2' °

21. If

shew that

/2+1 logr(Oo? = ?',

and deduce from \hardsubsectionref{12}{3}{3} that, for all values of 2 except negative real
value.s,

u = z log z - z + i log (2n-).

\addexamplecitation{Raabe, Jovnial fur Math, xxv.}

22. Prove that, for all wilues of 2 except negative real values,

sin 2nTrx

00 f dx' logr(2) = (2-i)log2-2 + Uog(27r)+ 2 / --

n=lj *+'

23. Prove that

(Bourguet*.

Bip,p)B p+h,p+h)= y \addexamplecitation{Binet.}

24. Prove that, when -(</<(,

r,,,1 /""" cosh (2n*) c?M

25. Prove that, when q>\,

B p, q) + Bip + \, q) + B p + 2, q) + ...=B p, q- ) .

26. Prove that, when p-a>0,

B p-ci,q) aq a(a + l) g(g + l)

 (i, ?) ' it;4-?'*' 1. 2. (/> + (/) (io + ? + l)

27. Prove that

B p, q)B p q, r) = B q, r)B q+r, p). \addexamplecitation{Euler.}

28. Shew that

n,,, .,, d.r r(a)r(b) 1

Jo " '' (.r+?>) + '' r(a + 6) (l+jo) jo"'

if a > 0, 6 > 0, p > 0. \addexamplecitation{Trinity, 1908.}

* This result is attributed to Bourguet by Stieltjes, Journal de Math,
(i), v. p. 432.

%
% 262
%

29. Shew that, if m > 0, n>0, then

n ( i+. )2 >-i (1 - -y n-i, \ r(m)r n)

and deduce that, when a is real and not an integer multiple of ijr,

'i /cos 6 + sin \ cos 2a

/'

\ i,r veos C - SUl

d6=.

and

30. Shew that, if a > 0, > 0,

/:

2 sin (tt cos a) '

\addexamplecitation{St John's, 1904.}

\addexamplecitation{Kummer.}

31 . Shew that, if a > 0, + 6 > 0,

fr(a)r(S) r(a + b)r 8)

r..r--(i- ) \,, lrw

Jo I-''*'' 5-0 I (a

l= / (a+6)- /'(a).

' + 8) T a + b + 8) Deduce that, if in addition a + c>0, a- b + c>0,

/l a-l (l\ b)(l\ . .c) r (t ).! ( ± \ +\ ' )

jo "(i-.r)(-log.r) * ~ *r(a + 6)r(a+c)" 32. Shew that, if a, b, c be
such that the integral converges,

fUl-x")(l-afi)(l-x'),, T b + c+l)T c + a + l)T a+b + l) ' a.v=los

JO (i-A-)(-log.r) '" '' °r(a + ' ) r b + l)r c + l)r a + h + c+l)' 33.
By the substitution cos = 1-2 tan <, shew that

(3-cos )5 4v/7r

\addexamplecitation{St John's, 1896.}

f sinP.v

34. Evaluate in terms of Gamma-functions the integral / ' dx, when /)
is a

J '*-"

fraction greater than unity whose numerator and denominator are both
odd integers.

[Shew that the integral is h I sin'' x\~+ 2 ( - )" ( 1 ) [ d.v.]

'- Jo l n=i \ x+mr x-mrj)

35. Shew that

\addexamplecitation{Clare, 1898.}

/:

93?-

2" + 2 r=o2/'!(n-r)

-a-mi-

36. Prove that

log Bip,g) log ( +i) 4- I ' \: '"\ \ l '" dv. \addexamplecitation{Euler.}

s v/', -/; °\ pq J j(, (1-V)l0g 7

37. Prove that, if p>0, p + s>0, then

B(p,p)

 '

s s~l), s(s-l)(s-2)(s-3)

 (P>P + )=- f7 ' l+ ( ) + 2.4.(2 +

fd i,-- - <--

38. The curve r"*=2'"~i a'"cos ??i is composed of ?n equal closed
loops. Shew that the length of the arc of half of one of the loops is

i~ a I (i cos x) ' dx, Jo

and hence that the total perimeter of the curve is

a iV

lm)\

%
% 263
%

39. Draw the straight line joiniug the points ±i, and the semicircle
of \ z\ = \ which lies on the right of this line. Let C be the contour
formed by indenting this figure at

- ?', 0, i. By considering / 2P- -i z + z- )p "-- dz, shew that, if p
+ q>l, q <l,

I " cosP*i--6 cos ip-q) 6 d6 =,, "!,, r.

Jo ip + q-l)2P i- B(p,q)

Prove that the result is true for all values of p and q such that p +
q>'l.

\addexamplecitation{Cauchy.}

40. If s is positive (not necessarily integral), and - in .r hn, shew
that

Mild draw graphs of the series and of the function cos*.i'.

41. Obtain the expansion

cos .f-2, r (*+i)[r(| +ia+i)r(i5- a+i) " r(h+?,a+i)raj- a+iy-]'

and find the values of x for which it is applicable. \addexamplecitation{Cauchy.}

42. Prove that, if /> > A,

22p-i r 2/>"- f 12 12.32 1 "|2

'"'''"=-A ""Wi'U + l i' 2(2yT3-) + 2.4.(2p+3)(2f+5)+ . '

\addexamplecitation{Binet.}

43. Shew that, if .c < 0, ./ + -- > 0, then

r(-.f) [-.f,(-..-)(l-.r) (-.r)(l- )(2-f ) 1 r(5) 1 5 - s(l+2) " z
l+z) 2 + z) j

and deduce that, when x + r> 0,

  Ina m +* ) \ f \ 4   -1), X '( -l)(- -2) \ f/-' r>) z z(z +
l) - z z + l) z + 2)

44. Using the result of example 43, prove that

logr(2 + a) = logr(0 + rtlogs

2z

fa

dt

a [\ \ { t)(2-t) ...(n-t)dt- f" t l-t) 2-t) ...(n-t)'

\ 5 .' .'o

nti n + l)z z+l) z + 2) ... z + n)

investigating the region of convergence of the .series.

(Binet, Journal de V Ecole polytechnique, xvi. (1839), p. 256.)

45. Prove that, if /> > 0, > 0, then

-. P-

B ip, q) = V- i (2 )* * "' '

%
% 264
%

where

II (p, g) = 2p - - T arc tan - - -. - -, -,

and p2 = 2 + q +pq-

46. If 6 =2*-'7r(l-i'r), F=2 - 7r(i-iA'),

and if the function F (,r) be defined by the equation

shew (1) that F o:) satisfies the equation

F x+ ) =xF x) +

r(l-A-)'

(2) that, for all positive integral values of x,

F x) = rix\

(3) that F(x) is analytic for all finite values of x,

1 (7 \ 2

(4) that "- "* '

47. Expand

F(x)= ~- r -r- log- 7

as a series of ascending powei's of a.

(Various evaluations of the coefficients in this expansion have been
given by Bourguet, Bull des Set. Math. v. (1881), p. 43; Bourguet,
Acta Math. ll. (1883), p. 261; Schlomilch,' Zeitschrift fiir Math,
und Phys. xxv. (1880), pp. 35, 351.)

48. Prove that the G-function, defined by the equation

G(z + ) = (2nf'e- ' + - n |(i+i)%-'+-'- /(2")|,

is an integral function which satisfies the relations

0 z + ) = V z)G z), (?(1) = 1,

(n !)'V6-' (n + 1) = 11 . 22 . 33 ... w". \addexamplecitation{Alexeiewsky.}

(The most important properties of the G-function are discussed in
Barnes' memoir, Quarterly Journal, xxxi.)

49. Shew that and deduce that

log Q J = / s cot Tvzdz-z log (27r).

50. Shew that

logrri + l)c <=i3log(2rr)-|3(2+i;+3logr(s + l)-logG'(s + l).

\chapter{The Zeta Function of Riemann}

13"1. Definition of the Zeta-f unction.

Let s = a + it where a and t are real*; then, if S > 0, the series

= 1 n=i n is a uniformly convergent series of analytic functions (§§
2"33, 3'34) in any domain in which <r 1 + 8; and consequently the
series is an analytic function of s in such a domam. The function is
called the Zeta-function; although it was known to Eulerf, its most
remarkable properties were not discovered before RiemannJ who
discussed it in his memoir on prime numbers; it has since proved to
be of fundamental importance, not onl - in the Theory of Prime
Numbers, but also in the higher theory of the Gamma-function and
allied functions.

1311. The generalised Zeta-function .

Many of the properties possessed by the Zeta-function are particular
cases of properties possessed by a more general function defined, when
cr' 1 + 8, by the equation

where a is a constant. For simplicity, we shall suppose || that < \$
1, and then we take arg(a + n) = 0. It is evident that (s, 1) = (s).

1312. The expression of s, a) as an infinite integral.

Since (a-F ??)-' T (s) = / af- e' '' ''''' do-., when arg.r = and o- >
(and a fortiori when a \ - S), we have, when o- 1 + 8, V s) s,a)= \ \
m :L | .-r*-' e" < + '* c a;

.v xjio 1-6- .'o l-e- "

* The letters o", t will be used in this sense throughout the chapter.

t Commentatioiies Acad. ScL Imp. Petropolitunae, ix. (1737), pp.
160-188.

+ Berliner MonaUherichte, 1659, pp. 671-680. Ges. Werke (1876), pp.
136-141.

§ The definition of this function appears to be due to Hurwitz,
Zeitschrift fiir Math, ttnd Phys. xxvii. (1882), pp. 86-101.

II When a has this range of values, the properties of the function
are, in general, much simpler than the corresponding properties for
other values of a. The results of\hardsubsectionref{13}{1}{4} are true for all values of a
(negative integer values excepted); and the results of §§ 13-12,
13-13, 13-2 are true when R (a) > 0.

%
% 266
%

Now, when ic' i), e l + x, and so the modulus of the second of these
integrals does not exceed

I "" A-'-2e-(- + ) rf = (N f aV- r (o- - 1),

-'0

which (when o- 1 + 8) tends to as JV x . Hence, when a l + 8 and arg x
= 0,

this formula corresponds in some respects to Euler's integral for the
Gamma- function.

13"13. The expression* of (s, a) as a contour integral. When a 1 + 8,
consider

(0+) / y-i g-a2

' 1 - e

the contour of integration being of Hankel's t3rpe \hardsubsectionref{12}{2}{2}) and not
containing the points + 2n7ri(n = l, 2, 3,,..) which are poles of the
integrand; it is supposed (as in\hardsubsectionref{12}{2}{2}) that, arg(- z)\ ir.

It is legitimate to modify the contour, precisely as in\hardsubsectionref{12}{2}{2}, whenf
o- 1 + B; and we get

'(0+) - A -ip-az r° 8-ip-az

Therefore

27n . ' 1 - e

Now this last integral is a one-valued analytic function of s for all
values of s. Hence the only possible singularities of s, a) are at the
singularities of r (1 - s), i.e. at the points 1, 2, 3, ..., and, with
the exception of these points, the integral affords a representation
of s, a) valid over the whole plane. The result obtained corresponds
to Hankel's integral for the Gamma- function. Also, we have seen that
s, a) is analytic when o- 1 -H 8, and so the only singularity of s, a)
is at the point .s = 1. Writing 6- = 1 in the

integral, we get

1 r(o+) e-,

ZTTl J 1 - e ' which is the residue at £; = of the integrand, and this
residue is 1.

Hence lim f \ = -l.

* il (1 -s)

* Given by Riemann for the ordinary Zeta-function.

+ If (7 1, the integral taken along any straight line up to the origin
does not converge.

%
% 267
%

Since T (1 - s) has a single pole at s = 1 with residue - 1, it
follows that the only singularity of s, a) is a simple pole with
residue + 1 at 5 = 1.

Example 1. Shew that, when R (s) > 0,

(1 \ 21 -*) /-(s) = i: \ 1 + 1 \ i . 1 2* 3' 4s ~

1 / * x ~

Exuraple 2. -Shew that, when R s)> 1,

Example 3. Shew that

where the contour does not include any of the points +rr ±877 ±57ri',
....

1314. Values of s, a) for special values of s.

In the special case when s is an integer (positive or negative),

is a one- valued function of z. We may consequently apply Cauchy's
theorem, so that - . -r: dz is the residue of the intesfrand at = 0,
that

is to say, it is the coefficient of z'" in V .

1 - e~

To obtain this coefficient we differentiate the expansion \hardsectionref{7}{2})

. e-" - 1 \ \ I (-)n< (a)2 e- -l n=\ n'.

term-by-term with regard to a, where (f)n iO denotes the Bernoullian
poly- nomial.

(This is obviously legitimate, by\hardsectionref{4}{7}, when | s j < tt, since - \,
can be expanded into a power series in z imiformly convergent with
respect to a.)

Then i ll-r Ml".

Therefore f s is zero or a negative integer (= - m), we have

  - m, a) = - </)', +o(a)/[(m -f- 1) ( i -f- 2)|. In the special case
when a = 1, if s = - m, then (5) is the coefficient

/ y, yji I Z

of z-~ in the expansion of - ' ' .

%
% 268
%

Hence, by\hardsectionref{7}{2},

 (-2m) = 0, l-2m) = (-y-BJ 2m) (m = 1, 2, S, ...),

r(0)=-|.

These equations give the value of (s) ivhen s is a negative integer or
zero.

13"15. TJie forniula* of Hurwitz for s, a) when cr< 0.

Consider - - -; - r-; - dz taken round a contour C consisting of

27n J c I - e~ °

a (large) circle of radius (2iV+l)7r, (iV an integer), starting at the
point

(2iV"+ l)7r and encircling the origin in the positive direction, arg
(- z) being

zero at z = -(2N+l) ir.

In the region between C and the contour 2NTr +7r; +), of which the
contour of § 13"lo is the limiting form, (- zy~' e~' (1 - e~ )~ is
analytic and one-valued except at the simple poles + 2Tri, + 47, ...,
± 2N7ri.

Hence

27riJc l-e~' 27ri j n+i) n l-e'- n=i

where i2, Rn are the residues of the integrand at 2n7ri, - 2 7
respectively. At the point at which - z = 2mre~ ', the residue is

(2n7r)'-le- '''('-l)e-2 "''',

and hence Rn + Rn = (27?7r) ~ 2 sin i ' " 2'n-an j .

Hence

1 /(0+) (\ )s-ig-a

27riJ(2iv'+i) l-e-

\ 2 sin STT cos 2'rran) 2 cos sir sin (27ra? ) " "(27rr,=i /? - " ' \
2 ) = n':i n'-'

+ H-  T -, dz.

2in J c I - e

Now, since < a 1, it is easy to see that we can find a number K
independent of N' such that | e~" (1 - e~ )~ \ < K when z is on C.

Hence

1 r ( -z\ \ ~' e~"' I 1 f""

-  1 -- . dz\ < K\, (2.Y + 1) -rrYe'"' I rf

27r

<ir (2iV-|-l)7r|' e-l*l as i\" X if cr < 0.

* Zeitschrift fill- Math, und Phys. xsvii. (1882), p. 95,

%
% 269
%

Making N y:, we obtain the result of Hurwitz that, if o- < 0,

 u. r(l-,v) ( /I cos(27ra ) /i \ 4 sin(27ra/i))

each of these series being convergent.

13151. Riemairii's relation between (s) and (l - s).

If we write a = 1 in the formula of Hurwitz given in\hardsubsectionref{13}{1}{5}, and
employ\hardsubsectionref{12}{1}{4}, we get the remarkable result, due bo Riemann, that

2 -* r (S) (S) cos (I STTJ = (1 \ s).

Since both sides of this equation are analytic functions of s, save
for isolated values of s at which they have poles, this equation,
proved when a < 0, persists (by\hardsectionref{5}{5}) for all values of s save those
isolated values.

Example 1. If m be a positive integer, shew that

C (2>n) = S- "*- 1 7r2"' BJ 2m) ! .

Example 2. Shew that r is)n~ C (s) is unaltered by replacing s by 1 -
s.

(Riemann.)

Example 3. Deduce from Riemann's relation that the zeros of f (5) at -
2, - 4, - 6, ... are zeros of the tirst order.

13"2. Hennites* fornuda for (s, a).

Let us apply Plana's theorem (example 7, p. 145) to the function (p
(z) =(a + z)~\ where arg ii + z) has its principal value.

Define the function q x, y) by the equation

(/ < '' y) = 9- - K + + wT' - ( + A' - iy)~']

= - Ha + A')- + if] ~ * sin \ s arc tan - - .

  X + a)

Since + arc tan - - does not exceed the smaller of tt and ', we X + a
' x+ a

have

\ q x,ij)\ \ (a + x)'+f-] *< 1 y-'; sinh JItt, 5 j j,

' q (X, y) : [ a + xf + f]- -''\ jsinh | | | . Using the first result
when y > a and the second when y < a it is

* Annali di Matemalica, (3),fv. (1901), pp. 57-72.

t If t>0, arc tan = :,<. I 5; and arc tan < / dt.

J 1-rt- J 1-i-t- Jo

%
% 270
%

evident that, if o- > 0, q (.v, y) (e-" - 1) dy is convergent when x
and

Jo

tends to as ic - X; also (a + x)~ dx converges if a > 1.

.0

Hence, if o- > 1, it is legitimate to make 'o - oo in the result
contained in the example cited; and we have

 (s,a) = la- +\ \ a+ocy dx + 'lj (a2 + 2)-i jsin fs arc tan 0j- J .

So

  s, a) = la-s + f +2j\ a + f)-y jsin (. arc tan )| - .

This is Hermite's formula*; using the results that, if y 0,

arc tan y/a y/a f y < dT j, arc tan y/a < 2 tt (y>: a7r],

we see that the integral involved in the formula converges for all
values of s. Further, the integral defines an analytic function of s
for all values of s.

To prove this, it is sufficient \hardsubsectionref{5}{3}{1}) to shew that the integral
obtained by differentiating under the sign of integration converges
uniformly; that is to say we have to prove that

/ - 1 log a- + \ y2) (a' +y-) - 5* sin i s arc tan -

dij

o ' y

- I (( +y ) * arc tan - cos f s arc tan - j

di/

~' y -I

converges uniformly with respect to s in any domain of values of s.
Now when s ! A, where A is any positive number, we have

1 (a2+y-) ~ * arc tan '- cos (s arc tan j < a- + i/' ')i cosh (|n-A);

since a-+f)

  J

,2 iA y±y\

converges, the second integral converges uniformly by\hardsubsubsectionref{4}{4}{3}{1} (I).

By dividing the path of integration of the first integral into two
parts (0, rra), Una, X ) and using the results

sin'lsarctau- I <sinh -, sin (s arc tan -) l<sinhJr7rA

V / 1 V /,

in the respective parts, we can simihxrly shew that the first integral
converges uniformly.

Consequently Hermite's formula is valid \hardsectionref{5}{5}) for all values of s,
and it is legitimate to differentiate under the sign of integration,
and the differentiated integral is a continuous function of s.

* The corresponding formula when = 1 had been previously giveu by
Jensen.

%
% 271
%

13'21. Deductions from Hermites formula. Writing s = in Hermite's
formula, we see that

Making s - 1, from the uniformity of convergence of the integral
involved in Hermite's formula we see that

li,, 1 (,, a)-- \ = lim + 1 + 2 r, -,, .

  i( ' *-lj,-*! s-\ 2a Jo a- + y-') e y-l)

Hence, by the example of\hardsubsectionref{12}{3}{2}, we have

hm|ri., )- - | = -j .

Further, differentiating* the formula for l s, a) and then making s -
0, we get

\ d, A y 1 \, a -*' log a a'-'

+ 2 - o log (a- + 2/-) . (a- + y-) ~ *' sin ( s arc tan -)

Jo i \ ci/

+ (a- + V-) ~ arc tan - cos ( s arc tan - ) [ - -

    a \ a)] e' y - 1

I i\, ' arc tan (Wa),

= ( a-,jlog - + 2J \ \ \ A rfy.

Hence, by § 1232,

These results had previously been obtained in a different manner by
Lerch -j*.

Corollary. lim k(s) 1 = 7> T (0) = -5 log(27r).

13"3. Euler's product for (s).

Let (T l+B; and let 2, 3, 5, .../>,... be the prime numbers in order.
Then, subtracting the series for 2~* (s) from the series for (s), we
get

 (.).(l-2-) = ~ +,+~ +, + ...,

* This was justified in § 13 -2.

t The formula for f (s, a) from which Lerch derived these results is
given in a memoir published by the Academy of Sciences of Prague. A
summary of his memoir is contained in the Jahrbuch iiber die
Fortschritte der Math. 1893-1894, p. 484.

%
% 272
%

all the terms of Sn~ for which n is a multiple of 2 being omitted;
then in like manner

all the terms for which n is a miiltiple of 2 or 3 being omitted; and
so on; so that

 (s) . (1 - 2-0(1 - 3-0  (1 -p-') = 1 + 2' -

the ' denoting that only those values of n (greater than p) which are
prime to 2, 3, ... j:? occur in the summation.

Now* i t'n-' I \$ I'n-'- \$ 2 n' - as ja co .

Therefore if a I + 8, the product (s) Yl (1 -p~ ) converges to 1,
tuhere

p the number p assumes the prime values 2, 3, 5, ... only.

But the product 11 (1 -p~ ) converges when a I + 8, for it consists of
p

some of the factors of the absolutely convergent product 11 (1 - n~ ).

Consequently we infer that (s) has no zeros at which a 1 + 8; for if

it had any such zeros, IT (1 -p~ ) would not converge at them. p

Therefore, ii a 1 + 8,

This is Euler's result.

13"31. Riemanns hypothesis concerning the zeros of (s).

It has just been proved that (s) has no zeros at which a >1.

From the formula (| 13*1 51)

y s) = 2 -i r(.9)|-isec ( STT Ul-s)

it is now apparent that the only zeros of (.5) for which a < are the
zeros

of r(s) - sec (2 '''') ) i-e- the points s = - 2, - -4, ....

Hence all the zeros of f (s) except those at - 2, - ]>, ... lie in
that strip of the domain of the complex variable s ivhich is defined
by - a 1.

It was conjectured by Riemann, but it has not yet been proved, that
all

the zeros of (.s) in this strip lie on the line o" = 2 ' " ' ile it
has quite recently

been proved by Hardy -f- that an infinity of zeros of (s) actually lie
on cr = : .

It is highly probable that Riemann's conjecture is correct, and the
proof of it would have far-reaching consequences in the theory of
Prime Numbers.

* The first term of S' starts with the prime next greater than p. t
Comptes Rendiif!, clviii. (1914), p. 1012; see p. 280.

%
% 273
%

13'4. Riemanns integral for (s). It is easy to see that, if cr > 0,

n' Hence, when a > 0,

   s) r ( ) TT - i = lim f 1 e- "''  x -'- dx.

X

Now, if OT x)= S e"""' '", since, by example 17 of Chapter vi (p.
124), 1 + 2ct x) = x~ ' 1 4- 2ts (I/*'), we have lim x -sr x) = 1;
and hence

 st(x)x ~ dx converges when a > .

Jo

Consequently, if a > 2, (s)r ('.7s')7r-'''"=lim ! (x) x -''' dx- i 1
e''''''' xi'-' dx'] .

V" / N o lJt) .0 M = .V+1 J

Now, as in\hardsubsectionref{13}{1}{2}, the modulus of the last integral does not exceed

Jo \ n = N+l j .'o l\ e-< V+l) x

.0

= 7r(iY+l)|-'|(iV H2;\ \ )7r l-i< rQ<r-l)

-*- as uV - - X, since a > 2. Hence, when cr > 2,

= ri-. + -i +x- -x;7(llx)\ x -'-' dx+ I t!r( )a:i -l(fa;

= - + 7 + f .rizTOr)a;-i'' + l(- )(;a;+[ nT(x)x '-Ux.

Consequently

r(5)r(L')7r-i*- - v -r,= I (x - -''> +x ')x-' (x)dx.

Now the integral on the right represents an analytic function of s for
all values of s, by\hardsubsectionref{5}{3}{2}, since on the path of integration

tn- x) < e-' * S e-"" e-"" (1 - e-"")-'.

;i=0

Consequently, by § o S, the above equation, proved when cr> 2,
persists for all values of 5.

w. M. A.  18

%
% 274
%

If now we put

s = l + it, ls(s- 1) (s) r ( s 7r- *- = 1(0, we have

  (t) = I - ff + j x-i (x) cos ( tlogx dx.

Since x~ -st x) log x\ cos U log x + - mr) dx

satisfies the test of\hardsubsectionref{4}{4}{4} corollary, we may differentiate any
number of times under the sign of integration, and then put = 0.
Hence, by Taylor's theorem, we have for all values* of

 (0= S a t''', =o

by considering the last integral ag i is obviously real. This result
is fundamental in Riemann's researches.

13*5. Inequalities satisfied by (s, a) when o-> 0.

We shall now investigate the behaviour of (s, a) as t - + oo, for
given values of cr.

When cr> 1, it is easy to see that, if N be any integer,

as, -)=U- + n)- - -,\ s)il aY- -LM' where

\ 1 J 1 \ 1 \ \ 1

/'"+! u-n,

fn

Now, when tr'> 0, l/ ( ) I ! I /

Jn (u + a)

" +] a-n

ai

fn+l J n (n

di

 n + af- ' = slin + a)-"-'. Therefore the series i" f (s) is a
uniformly convergent series of analytic functions

when cr >; so that 2 / (s) is an analytic function when <t> 0; and
consequently, bj'

;i=.V
\hardsectionref{5}{5}, the function ( (s, a) may be defined when (r>0 by the series

C (., a) = £ (a + )-.- (i\,)(; ).-. -j/. . Now let [t] be the
greatest integer in | < |; and take iV=[ ]. Then

|C )I 2 \ \ {a + nr \ + \ \ { l-sr'i[t] + ay- + 2 \ s\ \ {n + ar'' '

71=0 n=[t] [t] X

< 2 a + 7i)- + \ t mt] + ay ' + \ s\ 2 (n + a) ' \ n=0 i>=[t]

* In this particular piece of analysis it is convenieut to regard t as
a complex variable, defined by the equation s = + it; and then | (t)
is an integral function of t.

%
% 275
%

Using the Maclaurin-Cauchy sum formula \hardsubsectionref{4}{4}{3}), we get

r[t] r

Jo J[t]-l

Now when 8 a- 1 - S where S > 0, we have \ (s, a) \ <a- + l-a)- a +
[t]y-'' -a - + \ t [t] + af-'' + \ s\ < r-H[t]-l + a)-''. Hence f (s,
a) = 0 \ t p"* ), the constant implied in the symbol being independent
of s. But, when 1 - 8 cr l + S, we have

I C (, a) I = ( i i \'-n + / ( + A-)-'' dx

<0 \ t f-") + '-'+( + tf-"] I '' (a+x) - 1 dx,

since (a + x)~' a ~' a+x)-'>- when o- l, and (a+x)'" a+[t]) ~' (a +
x)- when o" 1, anoJ so

Cis, a) = 'Itr'' log \ t\ \ }. When 0- 1+8,

|C(, )| a~ + i (a + r'-* = 6'(l).

13'51. Inequalities satisfied by f (s, a) wlien cr 0.

We next obtain inequalities of a similar nature when <j h. In the case
of the function f (s) we use Riemaim's relation

C(s) = 2''7r -i r (1 -s) f (1 -s) sin (isTr). Now, when o- < 1 - 8, we
have, by § 1233,

r(l-s) = 0 e(*"*)'*' ( -*)-( -*' and ao

C (s) = [exp .V I | + ( -o--i01og|l-s|+iarctan /(l-o-) ]C(l-s).

Since arc tan i/(l - cr)= ±i7r4-0 ( ~ ), according as is positive or
negative, we see, from the results already obtained for f (s, a), that

i B) = 0 \ t\ \ -''\ i s).

In the case of the function (s, ), we have to use the formula of
Hurwitz \hardsubsectionref{13}{1}{5}) to obtain the generalisation of this result; we
have, when o- < 0,

i s,a): -i ±nY- V s)\ \ e ' ' Uiy- )- -''''" i-a s)\ where Ca (!-' )=
2

1 %'-

. - Hence (1 -e" ''"') f (l - ) = e2' + 2 e2 ' ' [/i'-i- (n- l) -i]

+ (S-1) i / T' /"" tt -2£;

since the series on the right is a uniformly convergent series of
analytic functions whenever o- l-S, this equation gives the
continuation of fa(l-*) over the range O o-: 1-S; so that, whenever cr
1 - S, we have

sin7raCa(l-s) 1 1+ 2 /i' ~' + (n-l)' -iH-|s-l I 2 /" n"-- dv..

1=2 re=iV+l ] n-X

18-2

276

THE TRANSCENDENTAL FUNCTIONS [cHAP. XIII

And obviously

Taking V=[ ], we obtain, as in\hardsectionref{13}{5},

Ca l-s)=0 \ tr) 8 a l-8) .

= 0 \ tf\ og\ t ) -8 (T<8).

C s)=0 ) a<-8).

Consequently, whether a is unity or not, we have the results

C s,a) = 0 \ t\ \ -'') (a 8)

= 0 \ t\ \ ) (8 0- 1-8)

= Oi\ t\ \ \ og\ t ) -8 a 8).

We may combine these results and those of\hardsectionref{13}{5}, into the single
formula

C(s,a) = 0(i<r" 'log| |), where*

r(o-)-i-(r, (o- O); 7-(o-) = A, (O a i); r(a) = l-(r, (*-\$<t 1);
r(cr) = 0, ( r l);

and the log | t \ may be suppressed except when - 8 o- S or when 1 - S
cr l + S.

13*6. The asymptotic expansion of log T z + a). From\hardsectionref{12}{1} example 3,
it follows that

\ aJ =i (.V a+nJ J T (z + a) Now, the principal values of the
logarithms being taken,

= 2

n=l

- az

  (-)"'"'

+ 2

(\ yn-i r

ji a + n)J 2 'ni (a - + w)"'J j,,' ! m a'' If I I < a, the double
series is absolutely convergent since

' ' - log 1 + -

 aA- n) ° V a+ n) a+ n

= 1 [\ /i(a + n)

converges.

Consequently

log

az

+

e-v r(a)

r ( + a) a =1 71 (a + n) ',,,=2 wi

1 1 772'

X / yn- 1

2 5 z'"' m,a).

Now consider;;; -; - -. tis, a) ds, the contour of integration being

ziri J c ssimrs

similar to that of\hardsubsectionref{12}{2}{2} enclosing the points 5 = 2, 3, 4, ... but
not the

points 1, 0, -1,-2, ...; the residue of the integrand at s = m(m 2) is

- z m, a); and since, as cr x (where s= a + it), s, a) = (1), the

integral converges if | 2 | < 1.

* It can be proved that t a) may be taken to be i (1 - a) when (t 1.
See Landau, Prim- zahlen, % 237.

%
% 277
%

Consequently

, e-y'r a) z az \ -rrz .. ..

I z + a) a =i n (a + m) Ztti q s sin its

Hence

- V a) V' a) 1 f -TTZ' .,,

  r ( + a) r (a) 27ri j c- 5 sin tts '

Now let D be a semicircle of (large) radius N with centre at s = f,
the semicircle lying on the right of the line <r = |. On this
semicircle (s, a)=0(l), !ir*| = |;<'e- ' '-g', and so the integrand
is* 5:; e-' i'f- ' s3l. Hence if | | < 1 and - tt + 8 arg tt - 8,
where S is positive, the integrand is \ zY e' *" ), and hence

t, s, a)ds 0

J j)S sm 7r5

as iV -* 00 . It follows at once that, if; arg z \ : 7r - 8 and j [ <
1,

, T(a) r'(a) i r + '' 'rrz',,,

log Y ~- -. =-z + r-. ~. (s, a) ds.

° 1 ( + a) 1 (a) 'Itti J s issimrs

But this integral defines an analytic function of for all values of 12
| if

j arg z] TT - 8.

Hence, by\hardsectionref{5}{5}, the above equation, proved when [ | < 1, persists for
all values of I I when ] arg zI' tt - 8.

Now consider I --. (s, a) ds, where n is a fixed integer and

-/ -n- ±iii sin TTS'

R is going to tend to infinity. By\hardsubsectionref{13}{5}{1}, the integrand is [z' e'
R"' ',"' where - n - - cr :\$ -; and hence if the upper signs be
taken, or if the lower signs be taken, the integral tends to zero as
i2 - x . Therefore, by Cauchy's theorem,

Via) T'ia) 1 f->'-h + i n

log \ \ = -z j -~- + -. - (s,a)ds+ X R,n,

where R is the residue of the integrand at s = - in. Now, on the new
path of integration

I s sin ITS \

where K is independent of z and t, and t (t) is the function defined
in\hardsubsectionref{13}{5}{1}.

* The constants implied in the symbol are independent of s and z
throughout.

%
% 278
%

Consequently, since j e~ - ',t]' - -i)dt converges, we have

when \ 2\ is large.

Now, when m is a positive integer, i?, = - - '- and so

- m

by\hardsubsectionref{13}{1}{4}, Rm = r '-,, where 6,/ (a) denotes the derivate of

m(m + l)(m + 2)

Bernoulli's polynomial.

Also Ro is the residue at s = of

and so i2 = f - - a j log + ' (0, a)

= (i - ) log + log r ( ) - log (27r), by\hardsubsectionref{13}{2}{1}.

And, using\hardsubsectionref{13}{2}{1}, R\, is the residue* at >Sf= of

\ ia\ s=\ ..,(i, >...),(i,s, g,+..,(l\ i:g,...).

XT 75 1 r' (a)

Hence R\ = - z\ oo'z+ z tt + z.

r(a)

Consequently, finally, if | arg z ir- S and | j is large, log r (z +
a) == (z + a -~ \ og z - z + l\ og 27r)

+ i (-)"'~ </>w+2( -), \ \ s

w=i w(77H- l)(m + 2) *

In the special case when a = 1, this reduces to the formula found
previously in\hardsubsectionref{12}{3}{3} for a more restricted range of values of arg 2.

The asymptotic expansion just obtained is valid when a is not
restricted by the inequality < a 1; but the investigation of it
involves the rather more elaborate methods which are necessary for
obtaining inequalities satisfied by (s, a) when a does not satisfy the
inequality 0<a%l. But if, in the formula just obtained, we write a=l
and then put z + a for z, it is easily seen that, when j arg ( + a) [
< tt - 8, we have

log r (2 + a + 1) = f + a + 2) log (z + a)-z-a + l ' + o l);

* Writings = 5+1.

%
% 279
%

subtracting log (z + a) from each side, we easily see that when both

I arg z + a) 7r- 8 and \ arg zI tt - 8, we have the asymptotic formula

logr( + a)=( + a- jlog - + . log(27r) + o(l),

where the expression which is o (1) tends to zero as \ Zi->X).

REFERENCES. G. F. B. RiEMANN, Ges. Werke, pp. 145-155. E. G. H.
Landau, Handbuch der Primzahlen. (Leipzig, 1909.) E. L. LiNDELOF, Le
Calcid des Residue, Ch. iv. (Paris, 1905.) E. W. Barnes, Messenger of
Mathematics, xxix. (1899), pp. 64-128. G. H. Hardy and J. E.
Littlewood, Acta Mathematical xli. (1917), pp. 119-196.

Miscellaneous Examples.

1. Shew that

(2 - 1 ) f (5) = - - + 2 / (i +y2)-* sin (s arc tan 2y)

(Jensen, D Intermediaire des Math. (1895), 'p. 346.)

2. Shew that

2 -i / * dv

C(s)= \ j-2 j (1+/)-** sin (5 arc tan y) - - .

(Jensen.)

3. Discuss the asymptotic expansion of \ ogG z + a), (Chapter xii
example 48) by aid of the generahsed Zeta-function. (Barnes.)

4. Shew that, if cr > 1,

p m=.l mp"

the summation extending over the prime numbers jd = 2, 3, 5,

(Dirichlet, Journal de Math. iv. (1839), p. 407.)

5. Shew that, if o-> 1,

where A (w) = when n is not a power of a prime, and A n) = \ ogp when
is a power of a prime p.

6. Prove that e~- -dx

log C (5) = 2 2

 \&

'(is) Jo

** r (is) 1

(Lerch, KraMw Rozprawy*, ll. See the Jahrbuch ilber die Fortschritte
der Math. 1893-1894, p. 482.

%
% 280
%

7. If 00

where | a; | < 1, and the real part of s is positive, shew that

and, if 5 < 1,

lim (1 - xy- (j) (s, A-) = r (1 - s).

(Appell, Comptes Rendus, lxxxvii.)

8. If X, a, and s be real, and < a < 1, and s > 1, and if

< (--' ' ).= ?,( : .' . .

shew that

and

(b (x, a, 1 - s) = tttVo

(Lerch, Acta Math, xi.)

9. By evaluating the residues at the poles on the left of the straight
line taken as contour, shew that, if k > 0, and | arg 3/ 1 < Att,

1 fk+cci

e-y= --.l rhi)y-''du,

and deduce that, if - > i,

9 - f ' ' (")  ('" )" " '') du = w x\

k - xi

and thence that, if a is an acute angle.

r 1 (0 = TT cos ia - W" 1 + 2 or (e -) . 1 t + t

(Hardy.)

10. By differentiating 2?i times under the integral sign in the last
result of example 9, and then making a - Jtt, deduce from example 17
on p. 124 that

r-*i- <*. (,)*= < -cos|

By taking n large, deduce that there is no number 0 such that | t) is
of fixed sign when t > to, and thence that f (s) has an infinity of
zeros on the line <r = .

(Hardy.)

[Hardy and Littlewood, P?'oc. London Math. Soc. xix. (1920), have
shewn that the number of zeros on the hue o- = i for which < < T' is
at least ( T) as - 00; if the

Kiemann hypothesis is true, the number is -- 7' log - " 7"+ (log 7;
see

Landau, Pnmzahlen, i. p. 370.]

\chapter{The Hypergeometric Function} 

141. The hypergeometric series.

We have already (§ 2*38) considered the hypergeometric series*

1 + - . , ( + l) ( + l) a(a + l)(a+2)6(6+l)(6 + 2) 3 l.c 1.2.c(c + l)
"* 1.2.3.c(c + l)(c + 2) " ••

from the point of view of its convergence. It follows from § 2-38 and
§ 5-3 that the series defines a function which is analytic when i 2 j
< 1.

It will appear later (§ 14-53) that this function has a branch point
at 2 = 1 and that if a cutf (i.e. an impassable barrier) is made from
+ 1 to + along the real axis, the function is analytic and one-valued
throughout the cut plane. The function will be denoted by F a, b;c;z).

Many important functions employed in Analysis can be expressed by
means of hypergeometric functions. Thus|

(1 +2)" = F(-n, : ;-z),

log l + z) = zF(l, l;2;-z),

e'=\ im F l,0;l;zl0).



Example. Shew that d



F a, b; c; z) = ~F a + \, 6 + 1; c + 1; 2).



1411. The valuel of F(a, b; c; I) when E(c-a-b)>0.

The reader will easily verify, by considering the coefficients of
.2;'* in the

* The name was given bj- Wallis in 1655 to the series whose Kth term
is a + b a + 2b ... a + n-l)b . Euler used the term hypergeometric in
this sense, the modern use of the term being apparently due to Kummer,
Journal fiir Math. xv. (1S36).

t The plane of the variable z is said to be cut along a curve when it
is convenient to consider only such variations in z which do not
involve a passage across the curve in question ; so that the cut may
be regarded as an impassable barrier.

X It will be a good exercise for the reader to construct a rigorous
proof of the third of these results.

§ This analysis is due to Gauss. A method more easy to i-emember but
more difficult to justify is given in § 14-6 example 2.



282 THE TRANSCENDENTAL FUNCTIONS [CHAP. XIV

various series, that if \$ ir < 1, then c c-l- 2c-a-b-l)x]F (a, b : c
; x) + (c - a) c - b) xF (a, b ; c + I ; x)

= c(c-l) l-x)F(a,b;c-l;x)

= C(C -l)jl+ (Un-Un-i)x'>'l,

where Un is the coefficient of " in i''(a, 6 ; c — 1 ; x).

Now make x—>l. By § 3*71, the right-hand side tends to zero if

1+2 (un — tin-i) converges to zero, i.e. if ?i,j— >0, which is the
case when =i

R c- a - 6) > 0.

Also, by § 2*38 and § 3'71, the left-hand side tends to

c(a+b-c)F a,b;c;l) + c-a)(c- b) F(a, b;c+l;l) under the same condition
; and therefore

Repeating this process, we see that

F(a, b: c;l) = i Yl ) —— -, ( F(a, b:c + 7n; 1)

  ' '\ n=o c + n) c-a-b + n)\ ' ' '

= ] lira n ) . , f [ lim F(a, b;c + m; 1),

if these two limits exist.

But ( 12'13) the former limit is tt r-Fr, rz, if c is not a negative

• T c — a) 1 (c— 6) °

integer; and, if Un(ci, b, c) be the coefficient oi x in F (a, b; c;
x), and

m > I c I , we have

00

\ F a, b] c + m; l) — l\ \ 2 \ un(a, b, c + m)\

n = l

00

  X Un (ja,, ib\, m — \ c )

n=l

' Q,b I "

< , 2 Mn (I a i -h 1, I 6 I + 1, 7?l -f- 1 - I C I).

m-\ c\ n=o

Now the last series converges, when ??i > ic| 4- a -f j l — 1, and is
a positive decreasing function of m; therefore, since [m — '\ c\ \ }~
—>0, we have

lim F(a, b: c + m ; 1)= 1 ;

and therefore, finally,

   , , r(c)r(c-a-6)



14*2, 14-3] THE HYPERGEOMETRIC FUNCTION 283

14"2. The differential equation satisfied by F (a, b; c: z).

The reader will verify without difficulty, by the methods of § lO'S,
that the hypergeometric series is an integral valid near = of the
hypergeometric

equation*

   ~ d? + (c - (a + + 1) - - \&' = ;

from § 10"3, it is apparent that every point is an 'ordinary point' of
this equation, with the exception of 0, 1, x , and that these are '
regular points.'

Example. Shew that an integral of the equation

i

z< F a- a, 6 + a; a- + 1; z).

14'3. Solutions of Riemann's P-equation by hypergeometric functions.

In § 10"72 it was observed that Riemann's differential equation f

dru j] -a -a' 1- -/3' l-y-y\ du dz- \ z— a z—b z — c ) dz

\ aa' (a-b)(a-c) jB' h - c) b - a ) \ 77 (c-a)(c-6) z— a z — b z — c

u



 z — a) z — b) z — c)



= 0.



by a suitable change of variables, could be reduced to a
hypergeometric equation ; and, carrying out the change, we see that a
solution of Riemann's equation is

fz-ay /z-c\ y 3 r./ 1 ' ( -a)(c-\&)]

provided that a — a' is not a negative integer ; for simplicity, we
shall, throughout thi section, suppose that no one of the exponent
differences a — a', /3 — , 7 — 7' is zero or an integer, as (§ 10"32)
in this exceptional case the general solution of the differential
equation may involve logarithmic terms ; the formulae in the
exceptional case will be found in a memoir :J: by Lindelof, to which
the reader is referred.

Now if a be interchanged with a', or 7 with 7', in this expression, it
must still satisfy Riemann's equation, since the latter is unaffected
by this change.

* This equation was given by Gauss.

t The constants are subject to the condition a + a' + + ' + y + y' =
l.

J Acta Soc. Sclent. Fennicae, xix. (1893). See also Klein's
lithographed Lectures, Veher die hypergeometrische Funktion (Leipzig,
1894).



284 THE TRANSCENDENTAL FUNCTIONS [CHAP. XIV

We thus obtain altogether four expressions, namely,

'z-aX /z-c\ y f . , , - . , , ,\ c-h) z-a)\



M, =



(z-ay /z-c\ y p , ry, -, > c-b) z-a)]

 ==(.— i) U— ft) J | + /3 + 7', o + +t; l+°-'''; (o-a)(.-6) j-

which are all solutions of the differential equation.

Moreover, the differential equation is unaltered if the triads (a, a',
a), (yS, y8', h), (y, y', c) are interchanged in any manner. If
therefore we make such changes in the above solutions, they will still
be solutions of the differential equation.

There are five such changes possible, for we may write

 b, c, a\, c, a, b], a, c, b], [c, b, a], [b, a, c]

in turn in place of [a, b, c], with corresponding changes of a, a, /S,
(3', 7, 7'.

We thus obtain 4 x = 20 new expressions, which with the original four
make altogether twenty-four particular solutions of Riemann's
equation, in terms of hypergeometric series.

The twenty new solutions may be written down as follows :



u. =



z-by fz-aY ( , , , a-c)(z-b)\ iP'j/8 + 7 + a, /3 + 7'+a; l + yS-/3'; )
, (

z-b\ \ ' [z-ay t q, , o' , ', . 1 , /D' o. (a-c)( -\&)]



Ua =



 s=('— ) ('- Y Fy+y + oc, '+y'+a-l + /3'-/3- ,. .

\ z - cj \ z - cj [ " a-b) z- c)

(z-b fz — a-y \ ( , \ , , , (a — c)(z — b)

\ z — cl \ z — c \ (a —b) z — c)

i /3+7+a, /3+7+a ; l + /3'-/3; ~ '-\

\ Z-C' \ Z-C' y I ' r- I ) r- r- ' ( \ J - c)\

(z-c\ y (z-b\ \ , o -. , b-a) z- c)]

" "=U- ) U- J |7 + ,7+a-+ ;l + y-7; \ ; \ |

U,,= l - )'( - r F\ y a + ',y c '; I + 7-7'; j H \ z-aj \ z-al [' ' "
' ' ' (l,-c)(z-a)\

   = (lllfy7l Yf f/'+o + ZS'. 7'+a+/3'; 1 + 7-7; P - \ z-aj \ z-a) ' "
' ' ' > (i,-c) z- a)



14-4] THE HYPERGEOMETRIC FUNCTION 285



Z ~ Cj



a,.=



MlQ —



FL + y + ',a + y' + '; I + a - a ; l' '\ \ ' ' ''\ \ , [ b-a) z-c)

/ -cy'/2-a\ j ' , /o' , 1, ' a-b)(z-c))

.T b) U— 6J |t+/3 + ,7+ +< : 1+7-7; \ ;( \ |,

\ z - a J \ z-a) [ ' (c-b)(z- a) '

— / i c — o)(z — a))

1 V - / \ z-aj ( ' " I r ' (c-6)( -a)J '

By writing 0, 1 - C, , B, 0, 6'- -5, a; for a, a', yg, S', y, y,

/ — AU — respectively, we obtain 24 solutions of the hypergeoraetric

equation satisfied hy F A, B; G ; x).

The existence of these 24 solutions was first shewn by Kuinmer*.

14"4. Relations betiueen particular solutions of the hyper geometric
equation.

It has just been shewn that 24 expressions involving hypergeometric
series are solutions of the hypergeometric equation ; and, from the
general theory of linear differential equations of the second order,
it follows that, if any three have a common domain of existence, there
must be a linear relation with constant coefficients connecting those
three solutions.

If Ave simplify n- , u.,, ih, ih] u , u g-, u i, W22 in the manner
indicated at

* Journal fUr Math. xv. (1836), pp. 39-83, 127-172. They are obtained
in a different manner in Forsyth's Treatise on Differential Equations,
Chap. vi.



' •22 = ,

\ z — aj \ z



286 THE TRANSCENDENTAL FUNCTIONS [CHAP. XIV

the end of § 14-3, we obtain the following solutions of the
hypergeometric equation with elements A, B, G, x:

y, =F A,B;C-x),

y = (- xy- 'FiA -C+\,B-G+1;2-C;x),

y,=(l-xr-''- F C-B,C-A;G;xl

y = (\ ccY-'' l - xf- - 'Fil -B,l-A;2-G;x),

y,r = F A,B; A + B - G + 1 ; 1 - x),

y = xf- -''F G-B, G-A; G-A-B-hl; l-x),

y = (\ xY 'FiA, A-G+\;A-B + \; x-%

y = (- x)- F B,B-G + l]B-A + l; x-'). If 1 arg(l — a;) I < TT, it is
easy to see from § 2-53 that, when j .r | < 1, the relations
connecting y , y.2, y y must be y y-i, y2= rj , by considering the
form of the expansions near c ; = of the series involved.

In this manner we can group the functions Wj, ... z/24 into six sets
of four*, viz. u-i, v-i, Mi3, Wis; Un, Hi, u , iiis; ii , u , M21, u
', Wg, u , U22, M24 j '"9? Wji, t<i7> ' 19 j Wjo, U12, u-18, '' ho,
such that members of the same set are constant multiples of one
another throughout a suitably chosen domain.

In particular, we observe that Ui, u , u s, u are constant multiples
of a function which (by §§ 5*4, 2"o3) can be expanded in the form



 z-aY\ l+ i en(2-a)4



when \ z — a\ is sufficiently small ; when arg z — a) is so restricted
that (z — a)* is one-valued, this solution of Riemann's equation is
usually written P'"*. And P'"'; P' ', P' '; P* ', P* * are defined in
a similar manner when



z — a



\ z — b\, \ z — c\ respectively are sufficiently small.



To obtain the relations which connect three members of separate sets
of solutions is much more difficult. The relations have been obtained
by elaborate transformations of the double circuit integrals which
will be obtained later in § 14-61 ; but a more simple and singularly
elegant method has recently been discovered by Barnes ; of his
investigation we shall give a brief account.

14"5. Barnes contour integrals for the hypergeometric function .

Consider r—. ~ -- p/ , . — — '(-zyds,

273-1 j\ xj r(c + s)

where |arg(— -z)] <7r, and the path of integration is curved (if
necessary) to ensure that the poles of r(a + 5)r(6 + 6'), viz. s = — a
— n, —h—n n = 0, 1, 2, . . .),

* The special formula

FiA,l;C;.) F( C-A,l;C; ),

which is derivable from the relation counectiug Mj with i3, was
discovered in 1730 by Stirling, Methodus Differential is, prop. vii.

t Piuc. London Math. Soc. (2), vi. (1908), pp. 141-177. References to
previous work on similar topics by Pincherle, Mellin and Barnes are
there given.



14-5]



THE HYPERGEOMETRIC FUNCTION



287



lie on the left of the path and the poles of r(— 5), viz. s = 0, 1, 2,
..., lie on the right of the path *.

From § 13'6 it follows that the integrand is

Oils j +''-c-i exp - arg - z) . I (s) - 7r\ I s)\ \ ]

as s— ►CO on the contour, and hence it is easily seen (| 5"32) that
the integrand is an analytic function of z throughout the domain
defined by the inequality I arg 2\ TT — S, where B is any positive
number.

Now, taking note of the relation F (— 5) F (1 + s) = — tt cosec s-rr,
consider

T(a+s)r(b+s) iri-zy



If IttiJ c



ds,



27rij(7 F(c + s) F(l +s) sinsTT

where C is the semicircle of radius iV + - on the right of the
imaginary axis

with centre at the origin, and N is an integer.

Now, by § 13"6, we have

r(a+ )F(6 + ) -n-i-zy . a+,\ ,\ ,) (- y T(c + s)r l+s) sinsTT ' sin
stt

as N—> 00 , the constant implied in the symbol being independent of
arg s when s is on the semicircle ; and, if 6 = f j\ \ + j e''* and j
| < 1, we have

(— zy cosec sir = exp j ( iV + .j j cos 6 log | | — ( lY + j sin arg
(— z)

-(iV + )7r|sm ||]

exp I i\ \ + . ) cos 6' log I I - (iV + ] 8 1 sin ( 1 11

exp|2- iV +. )logj

exp|-2- 8(i\ r + i)|



=



  I 6* k 7 TT,







1 I zii 1



Hence if log | | is negative (i.e. \ z\ < 1), the integrand tends to
zero sufficiently rapidly (for all values of 6 under consideration) to
ensure that



/.



'0 as iV— oo



Now



J -ooi [J ~xi J C J (iV+i) i)



by Cauchy's theorem, is equal to minus 2'Tri times the sum of the
residues of the integrand at the points s = 0, 1,2, ... N. Make N—>cc
, and the last

* It is assumed that a and b are such that the contour can be drawn,
i.e. that a and b are not negative integers (in which case the
hypergeometric series is merely a polynomial).



288 THE TRANSCENDENTAL FUNCTIONS [CHAB. XIV

three integrals tend to zero when | arg (— ) | tt — S, and \ z \ < 1,
and so, in these circumstances,

the general term in this summation being the residue of the integrand
at

S = 71.

Thus, an analytic function namely the integral under consideration)
exists throughout the domain defined by the inequality j arg | < tt,
and, when \ z\ < 1, this analytic function may he represented by the
sei'ies



V



r(a + n)r(b + n) r c + n) .n !



The symbol F(a, h; c; z) will, in future, be used to denote this
function divided by r( .)r(6)/r(c).

14'51. The continuation of the hyper geometric series.

To obtain a representation of the function F (a, b; c; z) in the form
of series convergent when j 2 | > 1, we shall employ the integral
obtained in § 14'5. If -D be the semicircle of radius p on the left of
the imaginary axis with centre at the origin, it may be shewn* by the
methods of § 14'o that

V a + s)V h + s)T -s)



Jd r c + s)



as p— x , provided that j arg — z) \ < tt, \ z \ > 1 and p—><xi in
such a way that the lower bound of the distance of D from poles of the
integi and is a positive number (not zero).

Hence it can be proved (as in the corresponding work of § 14"5) that,
when I arg —z)\ < 7r and \ z', > 1,

1 r r a + s)r(b + s)T -s)

\ V r(a+7i)r(l -c + a + n) sin ( c- a - n) tt /\ x- -h ~ rZo r (1 + w)
r (1 - 6 + ft + n) cos n-rr sin (b - ft - n) it

  T(b + n)r l-c + b + n) sin (c - 6 - ?i) tt f \

n=o r (1 + n) r (1 — ft + 6 + n) cos nir sin (ft — 6 — w) tt

the expressions in these summations being the residues of the
integrand at the points s = - a- n, s = - b — n respectively.

It then follows at once on simplifying these series that the analytic

* In considering the asymptotic expansion of the integrand when | s [
is large on the contour or on D, it is simplest to transform T (a +
s), V (b + s), T c + s) by the relation of § 12-14.



14-51, 14*52] THE HYPERGEOMETRIC FUNCTION 289

continuation of the series, by which the hypergeometric function was
originally defined, is given by the equation

r(c) ""' '- ' r(a-c) -(- r' - ( >i- + ;i- + )

1 0 — c) where arg (— z)\ < tt.

It is readily seen that each of the three terms in this equation is a
solution of the hypergeometric equation (see § 14"4).

This result has to be modified when a — 6 is an integer or zero, as
some of the poles of r a + s)r h + s) are double poles, and the
right-hand side then may involve logarithmic terms, in accordance with
1 43.

Corollary. Putting h = c, we see that, if | arg - z)\ < n,

r (a) z)-" = -. f "' r a+s) r ( -s) ( - zyds, 'ZniJ \ xt

where (1 - )~<'- -l as 2-*.0, and so the value of [ arg(l - z) \ which
is less than tt always has to be taken in this equation, in virtue of
the cut (sec i:; 14'1) from to -f-oo caused by the inequality | arg( -
z) \ < n.

14*52. Barnes' lemma that, if the 'path of integration is curved so
that the poles of T y — s)T b — s) lie on the right uf the path and
the poles of T (a + s) r (/3 + s) lie on the left*, then

Write / for the expression on the left.

If C be defined to be the semicircle of radius p on the right of the
imaginary axis with centre at the origin, and if p-s-oo in such a way
that the lower bound of the distance of C from the poles of r (y - s)
r (5 - s) is positive (not zero), it is readily seen that

T a + s)T \& + s)T y-s)T b-s) = - ''' '

= 0[..' - +>+ - exp -2 1/( )l J, as ] s |- -oo on the imaginary axis
or on C.

Hence the original integral converges ; and / - -Oasp-a-oc , when (a +
/3-|-y-|-S- 1)<0.

Thus, as in § 14'5, the integral involved in 7 is - 2ni times the sum
of the residues of the integrand at the poles oiT y — s)T b — s) ;
evaluating these residues we gett

/= i r( a + y+%)rO + y + ? ) TT I V a + b + n)T + b + n)



a=oV n + ) T + y-8 + 7l) sin(S-y)7r ,,=o T (?i+l) r (1 -f-S-y + w)
sin(y-S)7r*

* It is supposed that a, /3, y, 5 are such that no pole of the first
set coincides with any pole of the second set.

t These two series converge (§ 2-38).

W. M. A. 19



290 THE TRANSCENDENTAL FUNCTIONS [cHAP. XIV

And so, using the result of 12"14 freely, by § 14-11 :

\ 7rr(l-a-/3-y -5) r T (a + d) T (3 + 5) r(a + y ) rO + y) )

sin(y-8)7r |r(l-a-y)r(l-/3-y) T (1 - a -8)~r (1 - -8)J

r(a+y)r( + y)r(a + g)r( + a) f r , ro ,

- r(a+/J + y + 8)sm(a + + y + 5).sin(y-8). " ( + ) ' " ('

-sin (a + S) 77 sin 04-S)7r|.

But 2sin(fi+y)7rsiu(/3 + y)7r-2sin(a + S)7rsin (3 + S)7r

= cos (a — /3) TT - cos (a + /3 + 2y) tt - cos (a — jS) tt + cos (a +
+ 28) tt

= 2sin(y-8)7r sin(a+/3 + y + S)7r.

Therefore Vja + y)n ym-\ + ) V )

 (a + + y + S)

which is the required result ; it has, however, only been prt)ved when

 (o + /3 + y + 8-l)<0;

hut, )y the theory of analytic continuation, it is true throughout the
domain through which both sides of the equation are analytic functions
of, say, a: and hence it is true for all values of a, /ii, y, 8 for
which none of the poles of r (a + s) r (/3 + s), qua function of s,
coincide with any of the poles of r (y - s) r (8 - s).

Corollary. Writing s + k., a — k, \$-k, y + k, 8 + k in place of s, a,
/3, y, 8, we see that the result is still true when the limits of
integration are —; ,• + cx) i, where k is any real constant.

14 "53. The connexion between hypergeometrie functions of z and of \
—z. We hav§ seen that, if | arg ( - s) | < tt,

r(c) ' ' ' 2ni j \ x,- T c + s) "• '

= . \ \ r-.\ T a t)Y h-irt)T s-t)V c-a-h-t)dt\

'' TTi J \ x i l TTl J ~k—x i )

.T c-a)T c-b) ' by Barnes' leinma.

If k be so chosen that the lower bound of the distance between the s
contour and the t contour is positive (not zero), it may be shewn that
the order of the integrations* may be interchanged.

Carrying out the interchange, we see that if arg (1 -z) be given its
principal value,

T(C'-a)T c-h)T a)r b)F a, b; c; 2)/r (c)

= s— / r a + t)r b+t)r c-a-b~t) —. r s-t)ri-s)(~zYds dt

 T lJ-k-Jii 'ZtvI J -jci

1 f-k+oci

= r—. I r a + t)r b+t)r c~a-b-t)r -t) i-zy dt.

2ni J -A-K/

* Methods similar to those of § 4-51 may be used, or it may be proved
without much difficulty that conditions established by Bromwich,
Infinite Series, % 177, are satisfied.



14'53, 14 6] THE HYPERGEOMETRIC FUNCTION 291

Now, when I arg (1 -k) \ < Stt and | I -z\ < 1, this last integral may
be evaluated by the methods of Barnes' lemma (§ 14-52) ; and so we
deduce that

r c-a)r c-b)T a)rib)F a,b;c;z)

= r c)r a)r b)T c-a-b) F a, b ; a + b-c + l ; l-z)

+ r c)r c - a) r (c-b) r a + b - c) I -zy-'"-" F c- a, c-b; c-a-b + l;
z),

a result which shews the nature of the singularity of F a, b ; c ; z)
at z=.

This result has to be modified if c — a — 6 is an integer or zero, as
then

Y a-- t) V b->r t) T c - a -b - t)T -t)

has double poles, and logarithmic terms may appear. With this
exception, the result is valid when | arg —z)\ < tt, | arg (1 - 2) ; <
tt.

Taking | s | < 1, we may make 2 tend to a real value, and we see that
the result still holds for real values of 2 such that < 2 < 1.

14"6. Solution of Riemann's equation hy a contour integral.

We next proceed to establish a result relating to the expression of
the liypergeometric function by means of contour integrals.

Let the dependent variable u in Riemann's equation (§ 10"7) be
replaced by a new dependent variable defined by the relation

u = (z - aY z - bf (z - c)y I.

The differential equation satisfied by / is easily found to be

d f 1 + a a' l+ -§' 1-f - 7) dl dz- \ z — a z — h z — c ] dz

( g + /3 + 7) (g + + ry + 1) + Sa (g + /?' + 7' - 1)1 z — a) z — 0) z
— c)

which can be written in the form

Q(z) -[ X-2)Q' z)+R z)]f

+ li ( - 2) (X - 1 ) (/' (z) - (X-DR (z)] 1 = 0,

where / \=l-ot- -j = a' + ' + y',

iQ(z) z-a)(z-h) z-c),

(R(z) = X a' +/3 + ry) z-b) z-c).

It must be observed that the function / is not analytic at x , and
consequently the above differential equation in / is not a case of the
generalised hypergeometric equation.

We shall noiu sheiu that this differential equation can be satisfied
by an integral of the form

1=1 t- ay'+ +y- (t - 6)-+ '+y-i (t - c)-+ -y -1 z - )- - -y dt, J c

jirovlded that C, tJie contour of integration, is suitably cJiosen.

19—2



292 THE TRANSCENDENTAL FUNCTIONS [CHAP. XIV

For, if we substitute this value of / in the differential equation,
the con- dition* that the equation should be satisfied becomes

j (t- ay+ +y-' t - by+ '+y-' t - cY+p+y'-' z - t)-''- -y--Kdt = 0,

J c where

ir = (\ - 2) |q (z) + t- z) Q' (z) + l(t- zj Q" z)



+ (t-z) R(z) + (t-z)R'(z)]

= ( 2) Q (t) -it- zy] -it-z) [R (t) -a- zy s ( ' + /3 + 7)1

= - (1 + a + /3 + 7) ( - a) ( - 6) (i - c)

+ S (a + /9 + 7) ( - ) ( - c) t - z).

f dV It follows that the condition to be satisfied reduces to -7- dt =
0, where

J c dt

v=(t- ay+ +y (t - by+ '+y (t - cy+ +y (t - )-(i+-+p+v .

The integral / is therefore a solution of the differential equation,
when C is such that V resumes its initial value after t has described
C.

Now

V= t- ay'+ +y-' (t - by+ '+y-' (t - cy+ +y-' [z - t)-''- -y U,

where U = (t- a) t -b) t- c) (z - ty\

Now Z7 is a one-valued function of t ; hence, if C be a closed
contour, it must be such that the integrand in the integral / resumes
its original value after t has described the contour.

Hence finally any integral of the type

(z-ay(z-by(z - c)y \ (t-ay+y+''-\ t-b)y+''+ '-' t-cy+ y'-' z-ty''~ -y
dt,

JC

•where C is either a closed contour in the t-plane such that the
integrand resumes its initial value after t has described it, or else
is a simple curve such that V lias the same value at its termini, is a
solution of the differential equation of the general hypergeometric
function.

The reader is referred to the memoirs of Pochhammer, Math. An)i. xxxv.
(1890), pp. 495-526, and Hob.son, Phil. Trans. 187 a (1896), pp.
443-531, for an account of the methods by which integrals of this type
are transformed so as to give rise to the relations of § 14-51 and
14-53.

Example 1. To deduce a real definite integral which, in certain
circumstances, represents the hypergeometric series.

• The iliEferentiations imder the sign of integration are legitimate
(§ 4-2) if the path C does not depend on z a3d does not pass through
the points a, b, c, z ; if C be an infinite contour or if C passes
through the points a, b, c or z, further conditions are necessary.



14-61] THE HYPERGEOMETRIC FUNCTION 293

The hypergeometric series F a, b; c; z) is, as already shewn, a
solution of the differential equation defined by the scheme

r 00 1 \

P a zy

\ \ —c b c- a- b ] If in the integral

which is a constant multiple of that just obtained, we make 6 -x
(without paying attention to the validity of this process), we are led
to consider

/ t'''' t-\ y- - t-zydt.

Now the limiting form of V in question is

and this tends to zero at t= 1 and = cc , provided R (c) > R(b)>0.

We accordingly consider / t"~'= t — Vf-''-' t-z)~" dt, where z is not*
positive and greater tlum 1.

In this integral, write t = u~ ; the integral becomes

/ ? ''-' (1-jt)' -''-' (l-M2)- 0??<. .'

We are therefore led to expect that this integral may be a solution of
the differential equation for the hypergeometric series.

The reader will easily see that if R (c) > R b)> 0, and if arg u = arg
(1 - m) =0, while the branch of l-uz is specified by the fact that
(1-m2)- 1 as m- 0, the integral just found is

T b)r c-b)

  ( , \&; c;z .

This can be proved by expandingf (1 — s)-a in ascending powers of z
when | 3 | <1 and using § 12 "41.

Example 2. Deduce the result of § 14-11 from the preceding example.

14'61. Determination of an integral which represents i "\

We shall now shew how an integral which represents the particular
solution i ' (§ 14-3) of the hypergeometric difterential equation can
be found.

We have seen (§ 14-6) that the integral

l= z- df [z~bf z - cy ( t-a)P+y+- '-\ t-b)y+<'+P'- (t-c)'' +
+y'-Ht-z)-<'~ -ydt

satisfies the difl:erential equation of the hypergeometric function,
provided C is a closed contour such that the integrand resumes its
initial value after t has described C. Now the singularities of this
integrand in the plane are the points a, b, c, z; and after describing
the double circuit contour (§ 12'43) symbolised by (6 + , c-h, 6 — , c
— ) the integi-and returns to its original value.

* This ensures that the point t — ljz is not on the path of
integration, t The justification of this process by § 4'7 is left to
the reader.



294 THE TRANSCENDENTAL FUNCTIONS [CHAP. XIV

Now, if z lie iu a circle whose centre is , the circle not containing
either of the points h and c, we can choose the path of integration so
that t is outside this circle, and so \ z — a\ < i\ t — a\ for all
points t on the path.

Now choose arg(2 — a) to be numericall ' less than it and arg(i — 6),
ai*g(2-c) so that they reduce to* arg(a — 6), arg(a-c) when z- a\ fix
arg( — a), arg (< — 6), arg (< — c) at the point iV at which the path
of integration starts and ends ; also choose arg t - z) to reduce to
arg t — a) when z- a.

Then (z-bf = (a-bf |l +/3 (j ' ) + ...j- ,

(.-c)V=( -o)' l+ (i ?) + ...l, and since we can expand (t-z)~' ~°~'
into an absolutely and uuifornily convergent series

we may expand the integral into a series which converges absolutely.

Multiplying up the absolutely convergent series, we get a series of
integer powers of z — a multiplied by (s — a) . Consequently we must
have

We can define P '''\ P \ P '\ F y\ P ' by double circuit integrals in
a similar manner.

14'7. Relations hetiueen contiguous hypergeometric functions.

Let P z) be a solution of Riemann's equation with argument z,
singularities a, h, c, and exponents a, a', /3, j3', y, 7'. Further
let P(z) be a constant multiple of one of the six functions P * ,
P'"'', P' ', P' \ P<y , P't*. Let Pi+i,m-i(z) denote the function
which is obtained by replacing two of the exponents, I and m, in P 2)
hy I + 1 and m — 1 respectively. Such functions P;+i, i\ i (z) are
said to be contiguous to P (z). There are 6 x 5 = -30 contiguous
functions, since I and m may be any two of the six exponents.

It was first shewn by Riemannf that the function P(z) and any tivo of
its contiguous functions are connected hy a linear relation, the
coeffi,cients in which are polynomials in z.

There will clearly be 5 x 30 x 29 = 435 of these relations. To shew
how to obtain them, we shall take P z in the form

F z) = z- ay (z - hY (z - c)y f t- ay+y+' '- (t - 6)v+' + -

J c

(t - cY+ +y'-' (z - ty- -y dt,

where C is a double circuit contour of the type considered in § 14*61.

* The values of arg (a-b), arg (a - c) being fixed.

t Abh. der k. Ges. der Wiss. zu Gottingen, 1857; Gauss had previously
obtained 15 rehitions between contiguous hypergeometric functions.



147] THE HYPERGEOMETRIC FUNCTION 295

First, since the integral round G of the differential of any function
which resumes its initial value after i has described C is zero, we
have

= [ I- (t - ay+ +y (t - by ' y-Ht - cY+ +y'-' t - zy- - -y] dt. J c dt

On performing the differentiation by differentiating each factor in
turn,

we get

(a' + + y)P + (a + ' + y-l) Pa-+,, '-, + (a + (3 + y'-l) Pa'+i,y-i

\ (a + /3 + 7) p

Considerations of symmetry shew that the right-hand side of this
equation can be replaced by

These, together with the analogous formulae obtained by cyclical
inter- change* of (a, a, a) with (6, /3, ') and (c, 7, 7'), are six
linear relations connecting the hypergeometric function P with the
twelve contiguous functions

Pa+], '-l, -I p+i y'—i, -/y+i,a\ l, -ta-],v'-l5 -Lfi+i a'-l, y~\, '~\
i -I a.'+l, '— 1) -I a'4-],Y'\ i, i '4.),-y'\ i, -t j3'+l, a'-l > ''
y'+I, a'- 1 ) -*7'+l, '— 1-

Next, writing t — a = (t- h) + (6 — ), and usingf Pa'\ i to denote the
result of writing a' — 1 for a' in P, we have

P = P..\ :,,.+:- (6- )P.-:.

S-imilarly P = P \ i,y'+i + (c - a) Pa'-i-

Eliminating Pa-i from these equations, we have

 c-h)P + (a - c) Pa-\ :, vi + (b- a) Pa'-.,y-,, = 0. This and the
analogous formulae are three more linear relations con- necting P with
the last six of the twelve contiguous functions written above.

Next, writing ( — ) = ( — a) — ( — a), we readily find the relation

P = P +:, v'-i - ( - )" ' ( - f y ( - 0

Jc which gives the equations

( \ a)-i P - ( - 6)- P +,, Y--: = ( - h)- [P- z- c)- Py+, a'-i

= Kz - c)-' [P- z- a)-' Pa+i, \ i .

* The interchange is to be made only in the integrands ; the contour C
is to remain unaltered.

t Pa'-l is not a function of Riemann's type since the sum of its
exponents at a, h, c is not unity.



296 THE TRANSCENDENTAL FUNCTIONS [CHAP. XIV

These are two more linear equations between P and the above twelve
contiguous functions. *

We have therefore now altogether found eleven linear relations between
P and these twelve functions, the coefficients in these relations
being rational functions of 2. Hence each of these functions can be
expressed linearly in terms of P and some selected one of them ; that
is, between P and any two of the above functions there exists a linear
relation. The coefficients in this relation will be rational functions
of z, and therefore will become polynomials in z when the relation is
multiplied throughout by the least common multiple of their
denominators.

The theorem is therefore proved, so far as the above twelve contiguous
functions are concerned. It can, without difficulty, be extended so as
to be established for the rest of the thirty contiguous functions.

Corollary. If functions be derived from P by replacing the exponents
a, a, /3, /3', y, y

by a+p, a' + g, /3 + r, jS'+s, y + t, y' + x, where p, q, r, s, f, u
are integers satisfying the

relation

p- -q + r + s + t + u=0,

then between P and any two such functions there exists a linear
relation, the coefficients in which are polynomials in z.

This result can be obtained by connecting P with the two functions by
a chain of intermediate contiguous functions, writing down the linear
relations which connect them with P and the two functions, and from
these relations eliminating the intermediate contiguous functions.

Many theorems which will be established subsequently, e.g. the
recurrence-formulae for the Legendre functions (§ 15-21), are really
cases of the theorem of this article.

REFERENCES. C. F. Gauss, Ges. Werke, in. pp. 12.3-163, 207-229. E. E.
KuMMER, Journal fur Math. xv. (1836), pp. 39-83, 127-172. G. F. B.
RiEMANN, Ges. Math. Werke, pp. 67-84. E. Papperitz, Math. Ann. xxv.
(1885), pp. 212-221. S. PiNCHERLE, Rend. Accad. Lincei (4), iv.
(1888), pp. 694-700, 792-799. E. Y. Barxes, Proc. London Math. Soc.
(2), vi. (1908), pp. 141-177. Hj. Mellin, Acta Soc. Fennicae, xx.
(1895), No. 12.

Miscellaneous Examples.

1. Shew that

F a, h + \; c; z)-F a, b; c; z) = — F a + l, h + l; c + l; z).

2. Shew that if o is a negative integer while /3 and y are not
integers, then the ratio F(a, IB; a + /3 + l -y; \ - .v)-7-F a, ; y ;
A') is independent of x, and find its value.



THE HYPERGEOMETRIC FUNCTION 297

dP d P

3. If P iz) be a hypergeometrie function, express its derivates -,-
and -- linearly in

dP

terms of P and contiguous functions, and hence find the linear
relation between P, -j- ,

d'-P and -TV , i.e. verify that P satisfies the hypergeometrie
difterential equation.

4. Shew that i j, j ; 1 ; 4 (1 -z)) satisfies the hypergeometrie
equation satisfied by jP(|, I ; 1 ; z). Shew that, in the left-hand
half of the lemniscate \ z z) | = j, these two functions are equal ;
and in the right-hand half of the lemniscate, the former function is
equal to F \, | ; 1 ; 1 —2).

5. \ i Fu, =F a + ,h; c; .r), / \ =i (a - 1, h; c; x), determine the
15 linear relations with polynomial coefficients which connect F a, h;
c; x) with pairs of the six functions Fa , Fa-, F,,, F,\ , F, , F,\ .
(Gauss.)

6. Shew that the hypergeometrie equation

x(x-l/J - y-(a+ + l)x ' + a i/ =

is satisfied by the two integrals (suppo.sed convergent)

[\ \ -\ l-z)y- -' l-xz)-''dz J

and ['/-i(l-2)"-i' l-(l- )2 ~"rf2.

.' "

7. Shew that, for values of x between and 1, the solution of the
equation

is AF a,i ; i; \ --Ixy-l + B l-2x) F h a + l), h fi + l); |; l-2xy- ,

where A, B are arbitrary constants and F a, •,y;x) represents the
hypergeometrie series.

(Math. Trip. 1896.)

8. Shew that

Jim|F( ,/3,y,..)- J (-) ,;r(y-a)r(y- )r( )r( ) ' J

\ r(y-a- )r(y ) (y-a) (y- )

where h is the integer such that k R (a + — y)<k- .

(This specifies the manner in which the hj'pergeometric function
becomes infinite when .r- -l -0 provided that o-f |3-y is not an
integer.) (Hardy.)



9. Shew that, when i? (y - a - /S) < 0, then

-/3-y

- -l



T y)n'' -y



 "•( 4- -y)r(a)r( )

as '/i -x ; where S denotes the sum of the first n terms of the series
for F a, /3 ; y ; 1).

(M. J. M. Hill, Proc. London Math. Soe. (2), v.)



298 THE TRANSCENDENTAL FUNCTIONS [CHAP. XIV

10. Shew that, ii i/i, y-i be indeijeiident sokitions of



• 2- 1- -.



then the oreneral sohition of



is z = Ay - Byiy.i- Cyi', where J, B, C are constants.

(Appell, Comptes Rendus, xci.)

11. Deduce from example 10 that, if + ft + | = c,

 F(a h- r- r)\ \ = \ rW\ r(2c-l) - T 2a + 7>) F ia + b+n)T 2b+7i) ,,

'' ' '' r 2a)T 2b)T a + b)n=o nl T c + n)T 2c- 1+n) *' '

(Clausen, Journal fiir Math, iii.)

12. Shew that, if | | <i and j x \ —s) j <|,

F 2a, 2/3; a+/3 + *; x = F a, /3 ; a+ + l ; 4.r(l-.r) . (Kummer.)

13. Deduce from example 12 that

14. Shew that, if co = e ' ' and i (a) < 1,

i (a,3a-l; 2a; " - ) = 3 '' '* exp [K Sa - 1) | | ,

i ( ,3a-l; 2a; -co) =3 - exp W (1 -3a) M | .

(Watson, Quarterly Journal, xhi.)

15. Shew that

V 2 ) 2 -rj> -r\ j ..y V9; r(|)r('/i + f)

(Heymann, Zeitschrift fur Math, und Phys. XLiv.)

16. If il-xY+ -y F 2a,2 ;2y; x) = + Bx + C:'f- + Dx +..., shew that

i (a, /3; y+ ; ) i (y-a, y-/3 ; y+l; x)

y+l (7+l)(7 + i-) (y+*)(y + f)(7+|)

(Cayley, PAz7. Mag. (4), xvi. (1858), pp. 356-357. See also Orr, Camb.
Phil. Trans, xvii. (1899), pp. 1-15.)

17. If the function F a, ji, /3', y ; .*;, y) be defined by the
equation

i (a, ,/3', y; x,y) = \ - \ \ i' -\ \ ~xc)y-''-\ \ - ux)-\ l-uy)- '
du,

r (a) r (y - a) j

then shew that between F and any three of its eight contiguous
functions

i (a±l), F ii± ) , F ii'±l), F y±\ \

there exists a homogeneous linear equation, whose coefficients are
polynomials in x and y.

(Le Vavasseur.)



THE HYPERGEOMETRIC FUNCTION 299

18. If y - a - iS < 0, shew that, a.s x - 1-0,

and that, if -0-/3 = 0, the corresponding approximate formula is

(Math. Trip. 1893.)

19. Shew that, when \ x\ < 1,



/•(2 + ,0 + ,a:-,0-) 1 1 fl



= \ ,r/' sin OTT sin (y - a) TT . ' — " - "V (a, /3 ; y ; ),

where c denotes a point on the straight line joining the points 0, x,
the initial arguments of v — x and of v are the same as that of .>;,
and arg (1 — j/) -*-0 as v- 0.

(Pochhammer.)

20. If, when \ arg (1 -x)\ < 2n,

K x)=—r T -s)Ti ; + s)\ \ ' xYds,

LTTl J \ y, ,

and, when | arg.*; | < 27r,

l-lZl J \ K,-

by changing the variable s in the integral or otherwise, obtain the
following relations : K (.r) = A" (1 - ), if I arg (1 - :f) I <

K l-x) = K'(x), if|argj;|<7r.

K x) = l - .r)-4 A' ( - , if i arg (1 - .r) \ < n.

K l-x) x-h K f"-— , if I arg.r | <7r.



K' x) = x 2 / ' (Ijx), if I arg.r |'< tt.

A" (1 - a;) = (1 - X) - i A" ( \ J\ - \ ") , if j arg (1 - .x-) I < -
(Barnes.)

21. With the notation of the preceding example, obtain the following
results :

11=0 [ 'I- )

2 K' x)= - l f lt— j'- - log -4log 2 + 4 (1 - 1 + -- .

when \ x\ < l, | arg .i? | < tt ; and

iT ( ) = + i ( - ) - 5 a: (l/.r) + ( - ) - 2 ' (1/:??),

when j arg — x)\ < ir, the ambiguous sign being the same as the .sign
of / (x).

(Barnes.)



300 THE TRANSCENDENTAL FUNCTIONS [CHAP. XIV

22. Hypergeometric series in two variables are defined by the
equations F, a; , ';y; .r, y) = 2 "Xn f" '"J/'S

F, (a ; /3, ' ; y, y ; ,r, y) = 2 '-, x y .



-ts a,a, , ; y; x, y)= 2 , --,-\ x 'y"



 n( n Pm Pn ,( ??l . n i y i ,1



FUa, ;y,y';.,y) .XpV ' '"'" '' where a, = a(a + l). ..(a + m — 1), and
2 means 2 2 .

?n, n m=0 =0

Obtain the differential equations

c"F. d'F ?Ft ?F

-(l-)8 +y(l-)5j;5 + r-( +e+l)- 5;j'-* -. /-,=0,

d F d F dF;

.il-o:)- +y Hy- a + + l)x]j -a F, = 0,

and four similar equations, derived from these by interchanging x with
y and a, /3, y with a', 3', y when a', ', y occur in the corresponding
series.

(Appell, Comptes Rendus, xc.) 23. If a is negative, and if

a= —V + a,

where v is an integer and a is positive, shew that



r (x) r (a)






u D (-)"( -l)(a-2) ... (a-%) , where = — - - — ' — — " ' G( -n),



" i (•*/ = — . (Hermite, Joiirnal mr Math, xcii.)

x+n \ 1 . /



24. When a < 1, shew that



T x)T a-x) I \ l Rn



T a) n=\ X + n n=\ X — a — n''

where (-) a(a + l) ... (a + .-l)

7i !

25. When a > 1, and v and a are respectively the integral and
fractional parts of a, shew that

T x)T a-x) G x)p.,\ - (?( ')p, r(a) =i .r + ?i =i . r-a — n

-G x) + + ...+ f'-'X '\ \ X — a X-a-\ .r-a-i + lj'



THE HYPERGEOMETRIC FUNCTION 301



where (?( )=(l-? (l -- ) ... (l \ . \ )

V a/ V a+l/ \ a + v — lj

ajjj \ (-)"a(a+l)... (a + -l)



It, ;

(Hermite, Journal fiir Math, xcil.) 26. If

/• (T , \ r - '(jZ + + n-l) ., x x+l)(7/ + v + n-l)( + v + n)

./ (.r, y, .)- 1 - C, - - - ,) + C.3 3,( + 1) ( + ,)( . , + i) - ... ,

where n is a positive integer and C'i, C2, ... are binomial
coefficients, shew that

f (T y , r(y)rCy- -H/ )r(.r-n;)r(y+?0

./H .y. ) Y y-x)T y + n)V :6)V x-irV n)'

(Saalschiitz, Zeitschrift fur Math. xxxv. ; a number of similar
results are given by Dougall, Proc. Edinburgh Math. Soc. xxv.)

shew that, when liib + e - ,a — ) > 0, then

i (a,a-5+l, -c + l'; S, e ; l)=2- ..A P W " " — ' ' ' ' ' r(8-
a)r(f-ia)r( + ia)r(S + e-a-l)

(A. C. Dixon, P/'oc. London Math. Soc. xxxv.)

28. Shew that, if R (a) < .if, then

; f a(a+l)... (a-l-n- l)l ,. ,,, . r(l-3a)

(Morley, Proc. London Math. Soc. xxxiv.) 2!). If

[' / '....-I (1 \ .,.)>-iy-i (1 \ )A-. (1 -a,yy--J->'dxdi/ = B(i,J,
X-, m), J i> J i>

shew, by integrating with respect to jc, and also with respect to i
that B i,j, /•, m) is a symmetric function of i+J,j + i\ /• + + in,
/n+i.

Deduce that

F a,ld,y; fi, f ; l) r (8) F ( ) T (6 + 6 - a -/3-y)

is a symmetric function of 8, f , S + e — a - /i, 8 + e — - y, 8 + e -
y - a.

(A. ('. Dixon, Proc. London Math. Soc. (2), II. (1905), jjp. 8-16. For
a proof of a special case by Barnes' methods, see Barnes, Quarterly
Journal, XLI. (1910), pp. 136-140.)

30. If

x ~ il-x f'"- d" I

/ = i (- , a + ;.; y, .r) = - - y— - W ( x) \

shew that, when n is a large positive integer, and 0< .r < 1,

  = , - (sin, )i-V(cos0)V- -icos (2u + a)( -i,r(2y-l) + o( 1 ),

where .r = sin2 .

(This result is contained in the great memoir by Darboux, "Sur
I'approxi- mation des fonctions de trfes grands nombres," Journal de
Math. (3), iv. (1878), pp. 5-56, 377-416. For a systematic development
of hyper- geometric functions in which one (or more) of the constants
is large, see Camh. Phil. Trans, xxii. (1918), pp. 277-308.)


\chapter{Legendre Functions} 

\Section{15}{1}{Definition of Legendre polynomials.}

Consider the expression (1 -2 /i -|-/? )~ S when \'izh-h' \ < l, it
can be expanded in a series of ascending powers of ' .zh - Ii-. If, in
addition, I 2zh \ + \ h <l, these powers can be multiplied out and the
resulting series rearranged in any manner \hardsubsectionref{2}{5}{2}) since the expansion
of [1 - | 2zh \ + \ h\ \ ]]~ in powers of \ 2zh\ + \ h\' then
converges absolutely. In particular, if we rearrange in powers of h,
we get

(1 - 2zh + /r) -i = P, (z) + hP, (z) + h -P, (z) + ¥Ps z)+..., where

P, z) l, P, z) = z, P, z) = l Sz -n P, z) = l(5z -Szl

p (z) = I S5z' - Wz' + 3), P, (z) = I QSz' - 70z' + 1 5z), and
generally

 n z) - 2,, \,, 1 2 (2n - 1) 2.4. (2n - 1) (2n - 3) ' ' " j

,to 2".r!(7? -r)!(n-2r)!

r = 1 1

where ??i = ** 2 " - 1)' whichever is an integer.

If a, 6 and S be positive constants, ft being so small that 2ah + h' l
-8, the expansion of (1 - 2zh + /i ) ~ converges uniformly with
respect to z and k when j 2 j a, | A | 6.

The expressions TODO.    > which are clearly all
polynomials in z, are known as Legendre jwlynomials*, Pn z) being
called the Legendre polynomial of degree n.

It will appear later ( 15"2) that these polynomials are particular
cases of a more extensive class of functions known as Legendre
functions.

Example 1. By giving z special values in the expression (1 - izh + K )
" =-', shew that

* other names are Legendre coefficients and Zonal Harmonies. They were
iutrodueed into analysis in 1784 by Legendre, Memoires par divers
saians, x. (1785).

%
% 303
%

Example 2. From the expansion (1 - 2A cos + A ') '' = ( + \ he + Ji-e
' +..\

+ :

shew that

P (cos 6) = l-3---( -l) k cos '/i + J",!f'". 2 cos ( -2)5 " 2.4...
(2w) [ 2.(2?i - 1) '

1.3.(2n)(2?i-2),,, l 2 .4.(2.-l)(2 -3) ("- ) + -|- Deduce that, if 5
be a real angle,

!/> (cos5)|< 2. 4. ..2m r + 2.(2 -l)- + 2. 4Tl>3iy(2 rr3)- + -j

so that I P,, (cos 5) I 1. \addexamplecitation{Legendre.}

Example 3. Shew that, when z= -\,

TODO
(Ckre, 1905. )

15"11. Rodrig lies' * formula for the Legendre polynomials.

It is evident that, when n is an integer,

(In fin ( 71 ., t 1

  / w 'i! (2n-2r)!,.=0 r!(n-r)! (n-2r)! '

where w =;j ?i or (n - 1), the coefficients of negative powers of z
vanishing.

From the general formuhi for P,; (z) it follows at once that

this result is known as Rodrigues' formula.

Example. Shew that P (2) = has n real roots, all lying between + 1.

15'12. Schldfii'sf integral for Pn(z). 4, X 1

From the result of § loll combined with\hardsubsectionref{5}{2}{2},' it follows at once
that

where C is a contour which encircles the point z once
counter-clockwise; this result is called Schldfiis integral formula
for the Legendre polynomials.

* Corresp. sur VEcole poly technique, iii. (1814-1816), pp. 361-385. t
Schlafli, Ueber die zwei Heine'schen Kngelfunctionen (Bern, 1881).

%
% 304
%

15"13. Legendres differential equation.

We shall now prove that the function u = P ( ) is a solution of the
differential equation

, . d-u du,

which is called Legendre's differential equation for functions of
degree >l

For, substituting Schlafli's integral in the left-hand side, we have,
by\hardsubsectionref{5}{2}{2},

  ""~~dz ~ dV " " + 1) n ( )

, I (n + 1) r | ( - 1)"+ ),,

  2'7ri.2 ']cdt\ \ {t-zy'+-'] '

and this integral is zero, since ( - - 1)'*+ t - z)~"-~- resumes its
original value after describing C when n is an integer. The L gendre
polynomial therefore satisfies the differential equation.

The result just obtained can be written in the form

d\ [,,,, dPn z)\ dz

|(1 - z") 1 + n (n + 1) P. ( ) = 0.

It will be observed that Legendre's equation is a particular case of
Riemann's equation, defined by the scheme

-1 oc 1

71 + 1 z.

-n J

Example 1. Shew that the equation satisfied by "r is defined by the
scheme

P< -r 7i + r+l -r z. ( -n+r J

Example 2. If z = t], shew that Legendre's differential equation takes
the form

Shew that this is a hypergeometric equation.

Example 3. Deduce Schlafli's integral for the Legendre functions, as a
limiting case of the general hypergeometric integral of\hardsectionref{14}{6}. .

[Since Legendre's eqiiation is given by the scheme

P< 71+1 2>, I -71 J

15 13, 15*14] LEGENDRE FUNCTIONS

the integral suggested is

305

= 1, f -l) t-z)-''- dt,

taken round a contour C such that the integrand resumes its initial
value after describing it; and this gives Schlafli's integral.]

15-14. The integral properties of the Legendre polynomials. We shall
now shew that*

[' P.n z)Pn z)dz

.' -1

=

2

2n+l

(m :jfe n), m = n).

dJ u

Let [u].r denote -r; then, if r- n, [ z- - 1)",. is divisible by z -
1)""'';

and so, if / < n, [ z - 1)" vanishes when z=\ and when 2 = - 1.

Now, of the two numbers m, n, let m be that one which is equal to or
greater than the other.

Then, integrating by parts continually,

j' z"--i)%,, z -iy%dz

 ( -ir .\, ( --i) l

-f (z -lru- -lru dz -1 J -1

= (-)'" r (z- -i)"' (z -ir],,,,dz,

.' -1 since z" - l)'" t\ i, K ' - l)'" 7 -2.  vanish at both
limits.

Now, when m > n, [(2- - !)", -;- = 0, since differential coefficients
of (2- - 1)" of order higher than 2 vanish; and so, luhen ni is
greater than n, it follows from Rodrigues' formula that '

/:

P (z)Pn z)dz = 0.

/:

When ni = n, we have, by the transformation just obtained,

= (2n) if (1 - z-Y dz

= 2.(2w)! I (l-2-)"fZ Jo

y

= 2.(2n)! sin-"+' dcW Jo

= 2 . (2w)

, 2.4...(2n)

3.5...(2w+l)' These two results were given by Legendre in 1784 and
1789.

W. M. A.

20

%
% 306
%

where cos 6 has been written for z in the integral; hence, by
Rodrigues' formula,

r (P / M- 2.(2n)! (2".7i!) 2

J \ I n V 2* . n !)2 (2w + 1) ! 2/? + 1 "

We have therefore obtained both the required results.

It follows that, in the language of Chapter xi, the functions (?i +
|)2 P (3) are normal orthogonal functions for the interval ( - 1, 1).

Example 1. Shew that, if x> 0,

[ (cosh2.? -j)-4p (2)c s = 25(n + i)-ie-(2" + i)=:.

Example 2. If /= I P, z) P (z) dz, then .'

\addexamplecitation{Clare, 1908.}

(i) /-l/(2/i + l) (m=w),

(ii) 7=0 m - n even),

Y - '*'* + " ?2. ' i '

(iii) 7= . .. - '-~, -;,,-,' ' ., (?i = 2i/ + I, m = 2u.).

\ ' 2'" + "-i(%-?)0 (" + "i + l) (" !) (m

\addexamplecitation{Clare, 1902.}
\Section{15}{2}{Legendre functions.}

Hitherto we have supposed that the degree n of P z is a positive
integer; in fact, P ( ) has not been defined except when n is a
positive integer. We shall now see how P,j z) can be defined for
values of n which are not necessarily integers.

An analogy can be drawn from the theory of the Gamma-function. The
expression z ! as ordinarily defined (viz. as 2 (2 - 1 ) (2 - 2) ... 2
. 1 ) has a meaning only for positive integral values of z; but when
the Gamma-function has been introduced, z ! can be defined to be r (2
+ I), and so a function z ! will exist for values of z which are not
integers.

Referring to\hardsectionref{1}{5}"13, we see that the differential equation

is satisfied by the expression

''~27rijc 2-(t-zr+ ' ''

even when n is not a positive integer, provided that C is a contour
such that (t2\ i)n4-i \ -n-2 resumes its original value after
describing C.

Suppose then that n is no longer taken to be a positive integer.

The function (f - 1)"+' ( - 2 )~"~ has three singularities, namely the
points t = l, t = - 1, t = z; and it is clear that after describing a
circuit round the point = 1 counter-clockwise, the function resumes
its original value multiplied by e Trtfre-i-i) . while after
describing a/<5ircuit round the point t = z counter-clockwise, the
function resumes its original value multiplied by

%
% 307
%

  >nii-n-:>) \ jf therefore C be a contour enclosing the points t=l
and t = z, but not enclosing the point = -1, then the function ( -
1)"+ ( - 2 )-'*" will resume its original value after t has described
the contour C. Hence, Legendre's differential equation for functions
of degree n,

/I,x' * du,

is satisfied hy the expression

(1+, +) t -l)

"=2 -/.

  '2-'' t - zf ' *'

for all values of n; the many- valued functions will be specified
precisely by taking A on the real axis on the right of the point t=l
(and on the right of 2 if 5 be real), and by taking arg ( - 1) = arg(
4- 1) = and \ \ Yg t - z \ < 'Tr dX A.

''This expression will be denoted by F ( 2 ). ( nd will be termed the
Legeiidre function of degree n of the first kind.

We have thus defined a function P z), the definition being valid
whether n is an integer or not.

The function Pn z) thus defined is not a one-valued function of z;
for we might take two contours as shewn in the figure, and the
integrals along them would not be the same;

to make the contour integral unique, make a cut in the t plane from -
1 to - oo along the real axis; this involves making a similar cut in
the z plane, for if the cut were not made, then, as z varied
continuously across the negative part of the real axis, the contour
would not vary continuously.

It follows, by\hardsubsectionref{5}{3}{1}, that P z) is analytic throughout the cut plane.

15'21. The Recurrence Formulae.

We proceed to establish a group of formulae (which are really
particular cases of the relations between contiguous Riemann
P-functions which were shewn to exist in\hardsectionref{14}{7}) connecting Legendre
functions of different degrees.

If G be the contour of\hardsectionref{15}{2}, we have*

F M- JL\ [ ( '- y dt P'(z)-''- ( - dt

 n z) - 2, . j (TZTp-i ' ' " > - 2"+'7ri j c t - z)--'-'

WewriteP ' z)for-P (2).

20-2

%
% 308
%

Now - (t"--iy' ' 2(7i + l)t(t -l) \ (n + l) t -iy'+'

and so, integrating,

Therefore

2-+1 iri Jc t- zY '2 +- 7ri J c (t - zY ' 2 + Tri j t-zY+'

Consequently

i>,,w-.i'., .)= .jj;:-i 'rf* (A).

Differentiating*, we get

P'n+, (Z) - zP'n (Z) - Pn (Z) = nP, (z),

' ndso P'n,(z) - zP'n(z) = (n + 1) P,, 2) (I).

This is the first of the required formulae. Next, expanding the
equation

we find that

c t-zY Jc (t-zY J ( t-ZY '

Writing ( --1)4-1 for t- and (t - z) + z for t in this equation, we
get

 i c t-zY ]c (t-zY Jc t-zY-''

Using (A), Ave have at once

(n + 1) P +, (z) - zPn (z)] + 7iP \, (z) - nzP (z) = 0. That is to
say

(n + 1) P, i (z) - 2>i + 1) zPn (z) + nP \, U) = (II),

a relation f connecting three Legendre functions of consecutive
degrees. This is the second of the required formulae.

We can deduce the remaining formulae from (T) and (II) thus :
Differentiating (II), we have n + 1) [P' +, (z) - zP'n z)\ - n [zP\ z)
- P' \, ( )i - 2n + 1) P z) = 0. Using (I) to eliminate P'n+i (z),
and then dividing hyl n, we get

zP'n z)-P'n\, z) = nP,, z) (III).

* The process of differentiating under the sign of integration is
readily justified by\hardsectionref{4}{2}. t This relation was given in substance by
Lagrange in a memoir on Probability, Misc. Taurinemia, v. (1770-1773),
pp. 167-232.

X If n = 0, we have TODO, and the result (III) is
true but trivial.

/.

%
% 309
%

Adding (I) and (III) we get

P'n+A2)-P'n- z)==(2n+l)P,,(z) (IV).

Lastly, writing n -1 for n in (I) and eliminating P' \ i (z) between
the equation so obtained and (III), we have

(2-'-l)P'n z) = nzP, z)-nFn-,(z) (V).

The formulae (I) - (V) are called the recurrence formidae.

The above proof holds whether n is an integer or not, i.e. it is
applicable to the general Legendre functions. Another proof which,
however, only applies to the case when n is a positive integer (i.e.
is only applicable to the Legendre polynomials) is as follows :

Write V= l-2hz + h -)- .

Then, equating coefficients* of powei's of h in the expansions on each
side of the equation

 l-2hz + h )~ = z-h) V,

we have nP z)- 2n- ) zP \ i z) + n- ) P \ o z) = 0,

which is the formula (II).

Similarly, equating coeflficients* of powers of h in the expansions on
each side of the equation

 Th=' '- Tz

we have z ' -?- ' - " ',- i "' = nP (z\

dz

dP i ) dPr,\ y z)

dz

which is the formula (III). The others can be deduced from these.

Example 1. Shew that, for all values of,

\addexamplecitation{Hargreaves.}

jSz Pn'+P'n l)- PnPn l] = n + -i)P\ \,- 2n+ ) P .

Example 2.

If

M

hew that

dM x) dx

3f x) =

.( "( ' " " h Lo'

= nM \ i x) and / M x)dx = 0. \addexamplecitation{Trinity, 1900.}

Example 3. Prove that if m and n are integers such that m, both being
even

or both odd,

/ I dP (z) dP iz),,,

\ J -1 ~ - dz = m m + l). \addexamplecitation{Clare, 1898.}

Example 4. Prove that, if m, n are integers and m n,

P d P (z) d- P z) ( n-l)n n+ ) 7i- 2), .,

x n-(-) +m .

\addexamplecitation{Math. Trip. 1897.} * The reader is recommended to justify these
processes.

V

%
% 310
%

15'211. The expression of any polynomial as a series of Legendre
polynomials.

Let fn z) be a polynomial of degree n \ n z.

Then it is always possible to choose a,,, a-, ... so that

/ z) = aoPo z) + iPi (5) + . . . + anPn z),

for, on equating coefficients of z'\ z"-~', ... on each side, we
obtain equations which determine a, a \ i, ... uniquely in turn, in
terms of the coefficients of powers of z in fn (z).

To determine (/q, cti,  >i in the most simple manner, multiply the
identity by Pr z), and integrate. Then, by\hardsubsectionref{15}{1}{4},

when r = 0, 1, 2, ... n; when r > n, the integral on the left
vanishes.

Example 1. Given r" = aoPo (2) + < i A (2) + -.. +,ii'n (2), to
determine ao ii  ' 'n-

(Legendre, Exercices de Calc. Int. li. p. 352 )

[Equate coefficients of s" on both sides; this gives

""" i r 

Let In,m= I ~"'Pm ) d, SO that, by the result just given,

\ 2 "> + i(m!)

Now when n - ru is odd, 7,, is the integral of an odd function with
limits ± 1, and so vanishes; and 7, also vanishes when 71- in is
negative and even.

To evaluate /,, when n - m is a positive even integer, we have from
Legendre's equation

m m + 1) f z Pra z)dz=-( 2" \ (1 - z ) P, ' z) dz

= J znii-z )P i,) +nj' z- z )P,: z)dz

  n\ \ -- z )P, z) -nj'j, n- ) z-- - n + ) z")P, z)dz,

on integrating by parts twice; and so

m m + 1 ) 7,, = n (/ + 1 ) /,, -n 7i-l) / \ 2., Therefore

T \ n n-l)

n n- ) ... (m + l) j

 n - m) n-'2-m) ... 2. ( + ? + !)( + ? - 1) ... (27H + 3) by carrying
on the process of reduction.

%
% 311
%

Consequently !,,= -, - - - - - - .

and so,h = 0, when n - m is odd or negative,

(2m + l)2"'n\ \ {ln + m)l ' . . . -,

 m=~ri - TTT - . .i\, When n - m is even and positive.!

Example 2. Express cos 7id as a series of Legendre polynomials of cos
6 when n is an integer.

Example 3. Evaluate the integrals

j' zP, z)P, z)dz, j[ z'Pn(z)Pn liz)dz.

\addexamplecitation{St John's, 1899.}
Example 4. Shew that

j[ l- z ) P ' (z) ' dz= . \addexamplecitation{Trinity, 1894.}

Example 5. Shew that

n

nP (cos 6) 2 cos reP,,\,. (cos d).
\addexamplecitation{St John's, 1898.}

r = l

Example 6. If,. = / ( 1 - 22) /' (s) rfs, where m < /<, shew that

(?i-wi)(2 + 2m + l)M = 2n-' -i- \addexamplecitation{Trinity, 1895.}

15'22. Murphy's expression* of Pn z) as a hypergeometric function.

Since \hardsubsectionref{15}{1}{3}) Legendre's equation is a particular case of Riemann's
equation, it is to be expected that a formula can be obtained giving
Pn(z) in terms of hypergeometric functions. To determine this formula,
take the integral of\hardsectionref{15}{2} for the Legendre function and suppose that
1 1 - | < 2; to fix the contour C, let B be any constant such that <
5 < 1, and suppose that z is such that 1 1 - | 2 (1 - S); and then
take C to be the circle f*

\ l-t\ = 2-8. 1-z

Since

1-t

convergent series :|:

2-28 - - < 1, we may expand (t - z)~' ~ into the uniformly

(*-.)- -.=o-i)- i+( + i)£5j+*'i±iiii >(;;y+...|.

Substituting this result in Schlafli's integral, and integrating
term-by- term \hardsectionref{4}{7}), we get

p . . \ .2 ( z - ly (n + l)(n + 2) ...(n + r) r< +' +' (t' - 1)"

dt

 t-iy

rio 2-. r\ y [dr

* Electricity (1833). Murphy's result was obtained only for the
Legendre polynomials.

t This circle contains the points t = l, t = z.

X The series terminates if n be a negative integer.

l'('+ >"

%
% 312
%

by\hardsubsectionref{5}{2}{2}. Since arg (t+l) = when = 1, we get

= 2"-'- n (n-1)... (n -r + 1),

t=i

and so, when \ 1 - z \ 2 (1 - B) < 2, we have

p ., (n + l)(n + 2)... n + r).(-n)(l-n)...(r-l-n) (i i V"

   ",.=0 (H) V2 - 2 j

= F( n + l,-n; 1; - ).

This is the required expression; it supplies a reason \hardsubsectionref{14}{5}{3}) why
the cut from - 1 to - 00 could not be avoided in\hardsectionref{15}{2}.

Corollary. From this result, it is obvious that, for all values of n.

Note. When n is a positive integer, the result gives the Legendre
polynomial as a polynomial in 1 - s with simple coeificients.

Exam/pie 1. Shew that, if m be a positive integer,

f --'P.n.. \ \ r(2m + +2) /Trinitv 1907)

Example 2. Shew that the Legendre polynomial Pn (cos 6) is equal to
(-) F(?i + l, -n; \; cos 16), and to cos'' hS F - n, - n; 1; tan ).
\addexamplecitation{Murphy.}

15*23. Laplace's integrals* for P z).

We shall next shew that, for all values of n and for certain values of
z, the Legendre function Pn z) can be represented by the integral
(called Laplace's first integral)

- f "( + ( ' - 1)* cos < " d .

(A) Proof applicable only to the Legendre polynomials. When n is a
positive integer, we have, by\hardsubsectionref{15}{1}{2},

where G is any contour which encircles the point z counter-clockwise.
Take G to be the circle with centre z and radius | 2 - 1 | *, so that,
on G, t ~ z - z- - 1 ) e**, where may be taken to increase from - tt
to tt.

* Mecanique Celeste, Livre xi. Ch. 2. For the contour employed in this
section, and for some others introduced later in the chapter, we are
indebted to Mr J. Hodgkinaon.

%
% 313
%

Making the substitution, we have, for nil values of z,

= ir- f" + ( --l)*cos<f) c?<f>

Ztt J \

1 f" ' 1

= - [z + iz"- 1)5 COS 4)Y '

since the integrand is an even function of < . The choice of the
branch of the two- valued function z - 1)- is obviously a matter of
indifference.

(B) Proof applicable to the Legendre functions, where n is
um-estricted.

Make the same substitution as in (A) in Schlafli's integral defining
Pn z); it is, however, necessary in addition to verify that i = 1 is
inside the contour and = - 1 outside it, and it is also necessary that
we should specify the branch of [z + z" - 1) cos ", which is now a
many- valued function of < .

The conditions that t = \, t = -\ should be inside and outside G re-
spectively are that the distances of z from these points should be
less and greater than \ z -l\ \ . These conditions are both satisfied
if | 2: - 1 1 < | + 1 1, which gives R z) > 0, and so (giving arg z
its principal value) we must have

|arg |< TT.

Therefore P z) = [" [z + ( - l)i cos (j>Y ( <l>>

Ztt J \,r

where the value of arg 2 -I- ( ' - 1)2 cos <f) is specified by the
fact that it [being equal to arg(i'- - 1) - arg( - z)] is numerically
less than tt when t is on the real axis and on the right of z (see §
15 "2).

Now as (f) increases from -n to tt, 2 + z' -1) cos describes a
straight line in the

Argand diagram going from z- z'- 1)2 to z + z - l)~ and back again;
and since this line

does not pass through the origin*, arg [2 + (s"- - l)'- cos0 does not
change by so much as 77 on the range of integration.

Now suppose that the branch of z + z - l)'-* cos ( |" which has to be
taken is such that

it reduces to z" e'""-'" (where / is an integer) when ( = 7r.

Jnkni - Then P,. z) = -- z + (z - 1 )* cos < " d4>,

2.TT J -TT

where now that branch of the many-valued function is taken which is
equal to s" when

Now make 2- -l b ' a path which avoids the zeros of Pn z) ', since Pn
z) and the integral are analytic functions of z when | arg 2 | <hr, k
does not change as z describes the path. And so we get (r' ''' = 1.

* It only does so if is a pure imaginary; and such values of z have
been excluded.

%
% 314
%

Therefore, when | arg 2 | < - tt and n is unrestricted,

 n( ) = j[ [Z + Z' - If COS <1>Y d<f>,

where arg [z + (z- - 1)- cos cp] is to be taken equal to arg z when
(f> = tt. This expression for Pn (z), Avhich may, again, obviously be
written

- r [z + z- - 1)* cos </)]" d<p,

TT J

is known as Laplace's first integral for P ( ).

Corollary. From\hardsubsectionref{15}{2}{2} corollary, it is evident that, when | arg 2 |
<i7r,

dcj)

 cos "+i'

a result, due to Jacobi, Journal fiir Math. xxvi. (1843), pp. 81-87,
known as Laplace's second integral for P (z).

Example 1. Obtain Laplace's first integral by considering

2 A" j z + z-- ) cos ( " c?,

,1 = 1) J

and using\hardsubsectionref{6}{2}{1} example 1.

Example 2. Shew, by direct differentiation, that Laplace's integral is
a solution of Legendre's equation.

Example 3. If s < 1, | A | < 1 and

(l-2/icos(9 + A2)- = 2 bncosnd, 11=0

shew that = - T- jo (TT TO P? -"- " *-)

Example 4. When 2>1, deduce Laplace's second integral from his first
integral by the substitution

 S- (22 -1)5 cos IZ + Z - l)*COS( = l.

Example 5. By expanding in powers of cos<, shew that for a certain
range of values of z,

- I" z + (z'--l)hcoscl)]"d(t) = z F -hi,i,-in; 1; I-2-2). 7  - -

Example 6. Shew that Legendre's equation is defined by the scheme

r 00 1

p -hi i + hi M,

[ +hi -hi J

where 2 = (1 + " ).

15 -231. The Mehler-Dirichlet integral* for P (2).

Another expression for the Legendre function as a definite integral
may be obtained in the following way :

* Dirichlet, Journal fur Math. xvii. (1837), p. 35; Mehler, Math. Ann.
v. (1872), p. 141.

%
% 315
%

For all values of /*, we have, by the preceding theorem,

Pn z) = - 1 Z + COS (i> (22 - 1) dcp. TT J

In this integral, replace the variable by a new variable k, defined by
the equation

A = 2 + (s2 - 1 )Ti COS (j),

and we get . P (3) = - / h"- l-2kz + ffi) ' dh;

the path of integration is a straight line, arg /i is determined by
the fact that k = z when = |7r, and l--2hz + h' )~ = - i z- - l) sin
(f).

Now let z = cos d; then

Pn (cos 6) ' \ A" (1 - ihz + /i2) - hdh.

Now 6 being restricted .so that - tt <6 <\ it when n is not a positive
integer) the path of integration may be deformed* into that arc of the
circle |A| = 1 which passes through A = 1, and joins the points A =
e~* A = e'*, since the integrand is analytic throughout the region
between this arc and its chord t.

Writing h=e we get

1 [e J t+i) *

P (cos )= -d<i>,

  .' - (2cos</)--2cos )*

and so Pn (cos 6) = - ' i >

 .'0 2(cos0-cos )P

it is easy to see that the positive value of the square root is to be
taken.

This is known as Mehler's simplified form of Dirichlefs integral. The
result is valid for all values of n.

2 /"" \ (cos 6) = - \ = "'- "- -! 

Example 1. Prove that, when a is a positive integer,

2 /"" sin( + )0o?(/) 2(cos -cos</>)

(Write 7r-6 for 6 and tt - for ( in the result just obtained.)

Exa'mple 2. Prove that

1 / A"

P (cos ) = -,-  - T 5

27ri./ (A2\ 2Acos + l)4

the integral being taken along a closed path which encircles the two
points h=e, and a suitable meaning being assigned to the radical.

* If e be complex and R (cos ) > the deformation of the contour
presents slightly greater difficulties. The reader will easily modify
the analysis given to cover this case.

+ The integrand is not analytic at the ends of the arc but behaves
like (ft-e*' )~ near them; but if the region be indented (§ 6-'23) at
e*' and the radii of the indentations be made to tend to zero, we see
that the deformation is legitimate.

%
% 316
%

Hence (or otherwise) prove that, if 6 lie between Jtt and Itt,

,, 4 2.4...2 ico\&(n6 + d>) 1 cosM + 3( )\

P (cos ) = - ?r-T 77. - TT r H 1 -

77 3.5...(27i + l) 2gij yj 2(2?i + 3) (2 sin )

J 12.32 cos(% + 5( ) |-,

2.4.(2n + 3)(2?i+5) ( smO) I

' +

where < denotes \ d - jtt.

Shew also that the first few terms of the series give an approximate
value of P (cos 6) for all values of 6 between and n- which are not
nearly equal to either or it. And explain how this theorem may be used
to approximate to the roots of the equation P (cos ) = 0.

(See Heine, Kugelfunktionen, i. p. 178; Darboux, Comptes Rendus,
Lxxxii. (1876), pp. 365, 404.)

\Section{15}{3}{Legendre functions of tite second kind.}

We have hitherto considered only one solution of Legendre's equation,
namely P ( ). We proceed to find a second solution.

We have seen \hardsectionref{15}{2}) that Legendre's equation is satisfied by

j f -ir t-zy

dt,

taken round any contour such that the integrand returns to its initial
value after describing it. Let D be a figure-of-eight contour formed
in the following way : let z be not a real number between + 1; draw
an ellipse in the i-plane with the points + 1 as foci, the ellipse
being so small that the point = 2 is outside. Let A be the end of the
major axis of the ellipse on the right of = 1.

Let the contour D start from A and describe the circuits (1 -, - 1
+), returning to A (cf.\hardsubsectionref{12}{4}{3}), and lying wholly inside the ellipse.

Let 1 arg zl ir and let | arg (z - t) - > arg as - > on the contour.
Let arg(i + l) = arg(i-l) = at A.

Then a solution of Legendre's equation valid in the plane (cut along
the

real axis from 1 to - 00 ) is

1 r it' - 1)"

if n is not an integer.

When i (? -I- 1) > 0, we may deform the path of integration as in §
12-43, and get

(where arg(l - = arg(l -|- ) = 0); this will be taken as the
definition of Qn (z) when n is a positive integer or zero. When n is a
negative integer (= - m - 1) Legendre's differential equation for
functions of degree n is identical with that for functions of degree
m, and accordingly we shall take the two fundamental solutions to be
P. (z), Q,n z).

Qn (z) is called the Legendre function of degree n of the second kind.

%
% 317
%

15'31. Expansion of Qn (z) as a j)ower-series.

We now proceed to express the Legendre function of the second kind as
a power-series in z~' .

We have, when the real part of n + 1 is positive,

1 r

Suppose that \ z\ > 1. Then the integrand can be expanded in a series
uniformly convergent with regard to t, so that

- n- 1

dt

dt

where r = 2, the integrals arising from odd values of / vanishing.
Writing t- = u, we get without difficulty, from\hardsectionref{12}{4}'1,

The proof given above applies only when the real part of (n + 1) is
positive (see\hardsectionref{4}{5}); but a similar process can be applied to the
integral

Qn Z) = -. . \ l (f- - 1) Z - t)- dt,

4i sin UTT J D the coefficients being evaluated by writing 1 t- - 1)"
f dt in the form

J JD /(1-) / (-! + )

Jo Jo

and then, writing f- = u and using\hardsubsectionref{12}{4}{3}, the same result is
reached, so that the formula

TT r(/i + l) 1 /1,11,, . 3 1

is true for unrestricted values of n (negative integer values
excepted) and for all values* of z, such that [ 2: | > 1, j arg z\ <
tt.

Example 1. Shew that, when n is a positive integer,

* When n is a positive integer it is unnecessary to restrict the value
of arg z.

%
% 318
%

[It is easily verified that Legendre's equation can be derived from
the equation

by diflferentiating n times and writmg = 

Two independent solutions of this equation are found to be I

(32-1)" and (s2-l) /' (i;2-l)- -irft;.

It follows that |(2- - 1) j (v - 1) --1 dv

is a solution of Legendre's equation. As this expression, when
expanded in ascending powers of z~i, commences with a term in 2- -i,
it must be a constant multiple* of Q (z); and on comparing the
coetficient of -''-i in this expression with the coefficient of s~"~'
in the expansion of Q (2), as found above, we obtain the required
result.]

Example 2. Shew that, when n is a positive integer, the Legendre
function of the second kind can be expressed by the formula

Example 3. Shew that, when % is a positive integer,

t=0 t .[11- I) . J z

[This result can be obtained by applying the general integration
-theorem

to the preceding result.]

15'32. The recurrence- formulae for Qn )-

The functions P (2) and Q, z) have been defined by means of integrals
of precisely the same form, namely

[ t--iy' t-z)-''- dt,

taken round different contours.

It follows that the general proof of the recurrence-formulae for P z),
given in\hardsubsectionref{15}{2}{1}, is equally applicable to the function Qn (z); and
hence that the Legendre function of the second kind satisfies the
recurrence-formulae

(w -H 1) \$ + 1 (2) - (2 + 1 ) zQn z) + n Qn-l (z) = 0,

zQ'n z)-Q'n-l ) = nQ z), Q'n l z)-Qn~l )== 2ri+1)Q,, Z),

 z' -l)Q'n (2) = nzQ (2) - n\$ \ i (2). Example 1. Shew that

\$o(2) = 41ogj4|> i( ) = S logJ -l,

and deduce that Q> (2) = '2 (2) log - h,

* P (2) contains positive powers of z when n is an integer.

%
% 319
%

Example 2. Shew by the recurrence-formulae that, when n is a positive
integer*,

 1/ - i

where / \ i (z) consists of the positive (and zero) powers of z in the
expansion of

z+1 hPn (s) log in descending powers of z.

2-1

[This example shews the nature of the singulai'ities of Q ( ) at ± 1,
when n is an integer, which make the cut from - 1 to +1 necessary. For
the connexion of the result with the theory of continued fractions,
see Gauss, Werke, in. pp. 165-206, and Frobenius, Journal fur Math.
Lxxiii. (1871), p. 16.]

15 33. The Laplacian integral f for Legendre functions of the second
kind. It will now be proved that, when R n+ 1)> 0,

Qn ( ) = f " [2 + (2' - 1)* cosh 6>;- -i de, Jo

where arg \ z + z- - 1)* cosh 6] has its principal value when 6 = 0,
if n be not an integer.

First suppose that z>. In the integral of § lo*3, viz.

write TODO

so that the range (- 1, 1) of real values of t corresponds to the
range (- oo, oo ) of real values of 6. It then follows (as in\hardsubsectionref{15}{2}{3}
A) by straightforward substitution that

, Q z) = \ \ [z z - - 1 ) cosh (9 -"- (

= z + z"-- ) cos\ ie]-''-'de,

Jo

since the integrand is an even function of 0.

To prove the result for values of z not comprised in the range of real
values greater than 1, we observe that the branch points of the
integrand, qua function of z, are at the

points ±1 and at points where z + z -l) cosh vanishes; the latter are
the points at which z= ± coth d.

Hence Q (z) and I z+ z - 1)- cosh B -''- d\$ are botli analytic + at
all points of the Jo plane when cut along the line joining the points
z= ± I.

* If -1<3<1, it is apparent from these formulae that Q z + Oi) - Q
(z-Oi)= - 7riP (z). It is convenient to define Qn(z) for such values
of z to be J(? ( + Oi) + JQ (0-Oi). The reader will observe that this
function satisfies Legendre's equation for real values of z. t This
formula was first given by Heine; see his Kugelfunktionen, p. 147.
:J: It is easy to shew that the integral has a unique derivate in the
cut plane.

%
% 320
%

By the theory of analytic continuation the equation proved for
positive vahies of z - 1

persists for all values of z in the cut plane, provided that arg z+(z
- l) cosh is given a suitable value, namely that one which reduces to
zero when 2 - 1 is positive.

The integrand is one-valued in the cut plane [and so is Qn )] when n
is a positive

integer; but Arg z + z - l)2cosh increases by 27r as arg 2 does so,
and therefore if n be not a positive integer, a further cut has to be
made from 2= - 1 to z= - oc .

These cuts give the necessary limitations on the value of 2; and the
cut when n is not

an integer ensures that arg z + (2 - 1 ) = 2 arg z + 1 ) + (2 - 1)-)
has its principal value.

Example 1. Obtain this result for complex values of z by taking the
path of integration to be a certain circular arc before making the
substitution

 \ / (2 + 1) -(2-1) / (2+1)* + (2-1) '

where 6 is real.

Example 2. Shew that, if 2 > 1 and coth a = z,

Qn (2) = I 2 - (2- - 1)* cosh u "- du, where arg z - z -\ )5 cosh u] =
0. \addexamplecitation{Trinity, 1893.}

15*34. Neumann's* formula for Qn )i when n is an integer.

When n is a positive integer, and 2 is not a real number between 1 and
-1, the function Qn (2) is expressed in terms of the Legendre function
of the first kind by the relation

which we shall now establish.

When I 2 i > 1 we can expand the integi-and in the uniformly
conve|'gent series

wj=0 2

Consequently

 y \ i -y TO=o J -1

The integrals for which m - n is odd or negative vanish \hardsubsubsectionref{15}{2}{1}{1});
and so

  J -\ z - y m=o J -I

 1 I \ .,\, \, 2- n + 27n) \ (n + m) '. 2,=o' ni\ \ {2n + 2m+l)l

<0n ( !N2

(2ri + l)! 2 i + o, .,?i-t-i, + .,,*;

by\hardsubsectionref{15}{3}{1}. The theorem is thus established for the case in which
|2|>1. Since each side of the equation

represents an analytic function, even when | 2 | is not greater than
unity, provided that 2 is not a real number between - 1 and 4- 1, it
follows that, with this exception, the result is- true \hardsectionref{5}{5}) for all
values of 2.

* F. Neumann, Journal fur Mdth. xxxvn. (1818), p. 24.

%
% 321
%

The reader should notice that Neumann's formula apparently expresses
Qn z) as a one- valued function of z, whereas it is known to be many-
valued \hardsubsectionref{15}{3}{2} example 2). The reason for the apparent discrepancy is
that Neumann's formula has been established when the z plane is cut
from - 1 to +\, and Q z) is one-valued in the cut plane.

Example 1. Shew that, when - 1 \$ i2 (2) 1, | § (z) j | 7(2) |-'; and
that for other values of 2, | Q (2) | does not exceed the larger of |
2- 1 |~i, | 2-I- 1 |~'.

Example 2. Shew that, when n is a positive integer, Qn z) is the
coefficient of h" in the expansion of (1 - 2A2 -|- fi') ~ ai-c cosh \
- \ .

[For, when | A j is sufficiently small,

= (1 - 2A2+A2) - i arc cosh 1- --, ! .

This result has been investigated by Heine, Kugelfunktionen, I. p.
134, and Laurent, Journal de Math. (3), I. p. 373.]

\Section{15}{4}{Heine's* development of TODO t - z)~ as a series of
  Legendre polynomials in z.} 

We shall now obtain an expansion which will serve as the basis of a
general class of expansions involving Legendre polynomials.

The reader will readily prove by induction from the
recurrence-formulae (2m + 1) tq, (0 - ("i + 1) Q,+i t) - mQ,,\, (t)
= 0,

 2m + [) zP, z) - m + 1) P,+i z) - ml\ \, z) = 0, that

-- = I (2m + 1 ) P,, Z) Qrn (t) 4- " I [Pn, z) Q (t) ~ P (z) Q . (t)
.

Using Laplace's integrals, we have

Pn Az)Qn t)-Pn z)Qn+x t)

\ \ p r [z- iz- - l) C0S</)j

TT j J 1 -I- (t- -if- COsh ?< "+'

X [2 + ( - - 1 )- COS ( - \ t- (t- - ] )- cosh |~ ] d(f>du.

I z + z -lfcos(p

r*row consider | j-

] t + t -l) coshu

Let cosh a, cosh a be the semi-major axes of the ellipses with foci +
1 which pass through z and t respectively. Let be the eccentric angle
of z; then

2 - cosh (a + 10), I 2 + (2 - 1)' COS 1 = 1 cosh (a -f- id) ± sinh a +
id) cos (f) |

= cosh- a - sin 9 + (cosh a - cos B) cos'- ± 4 sinh a cosh a cos < 2 .
This is a maximum for real values of < when cos 0= + 1; and hence

I 2 ± (2 - 1)2 cos |2 2 cosh2 a -I + 2 cosh a (cosh a - l) =exp (2a).
Similarly \ t + (t - 1)2 cosh u \ exp a.

* Journal fur Math. XLir. (1851), p. 72. W. M. A. 21

%
% 322
%

Therefore

I P +i (z) Qn (t) - Pa 2) Qn+i (t) 1 TT-i exp n (a - a) T f * Vd<f>du,

J J

2 + (2- - Ip cos<f>

rhere \ V\ =

+ \ + ( --l) cosh u]

t + f-l) coshu

Therefore j Pn+i (z) Qn (0 ~ Pn ( ) Qn+i (0 j - > 0, as ?i -> 00,
provided a < a. And further, if t varies, a remaining constant, it is
easy to see that

the upper bound of I I Vd<j)du is independent of t, and so

Jo Jo

Pn+ z)Qn(t)-Pn )Qn+ (t)

tends to zero uniformly with regard to t.

Hence if the point z is in the interior of the ellipse which passes
through the point t and has the points ± 1 for its foci, then the
expansion

r =l 2n + ) Pn z)Qn t) t - z =o

is valid; and if the a variable point on an ellipse with foci ± 1 such
that z is a fixed point inside it, the expansion converges uniformly
with regard to t.

15"41. Neiiinanns* exj)ansion of an arbitrary f motion in a series of
Legendre polynomials.

We proceed now to discuss the expansion of a function in a series of
Legendre polynomials. The expansion is of special interest, as it
stands next in simplicity to Taylor's series, among expansions in
series of poljnaomials.

Let f 2) be any function which is analj tic inside and on an ellipse
C, whose foci are the points z ±. We shall shew that

f 2) = a,Po (z) + ttiA z) + aoPo 2) + ttaPa 2)+..., where a,,, Oj, a,
... are independent of 2, this expansion being valid for all points z
in the interior of the ellipse C.

Let t be any point on the circumference of the ellipse.

Then, since S (2n+ l)Pn(z) Qn(t) converges uniformly with regard to t,

w=0

/( > = 2 - l/t- f - 2 - !. .L<2" + 1> " < > " (') ' "> "

X

= % anPn(z), w=0

2n + 1 f where a = . I f(t) Qn (t) dt.

* K. Neumann, Ueher die Entwickeluny einer Funktion nach den
Kugelfunktionen (Halle, 1862). See also Thomd, Journal fur Math. lxvi.
(1866), pp. 337-343. Neumann also gives an ex- pansion, in Legendre
functions of both kinds, valid in the annulus bounded by two ellipses.

%
% 323
%

00

This is the required expansion; since 2 2n + 1) Pn z) Qn(t) may be
proved*

to converge uniformly with regard to z when z lies in any domain C"
lying wholly inside C, the expansion converges uniformly throughout C.

Another form for a can therefore be obtained by integrating, as in §
15-211, so that

an = (' + 2) j /( (  A form of this equation which is frequently
useful is

"' Irli/V*''' " - "" ' " ' '

which is obtained by substituting for Pn ( ) from Rodrigues' formula
and integrating by parts.

The theorem Avhich bears the same relation to Neumann's expansion as
Fourier's theorem beai's to the expansion of\hardsubsectionref{9}{1}{1} is as follows :

Let fit) he defined 'ivhen - 1 1, and let the integral of \ -t-y f t)
exist and he ahsohitely convergent; also let

a = n- h) r f t)P t)dt.

Then 2a Pn a;) is convergent and has the su?n i f ic+0) + f jc - 0) at
any point x, for which - 1< ar < 1, if any condition of the type
stated at the end of § 9 "43 is satisfied.

For a proof, the reader is referred to memoirs by Hobsont and
Burkhardt J.

Example 1. Shey that, if p ( 1) be the radius of convergence of the
series 2c s", then 2c P (2) converges inside an ellipse whose
semi-axes are (p+p~'), \ (p - p" ).

Example 2. If z=( \ \ /[2 = 5'' ~ JJ fj ±l), where y>x>\,

prove that /" ' = (a: + 1 ) (y - 1 ))i i P (.*:) Q,, (y).

[Substitute Laplace's integrals on the right and integrate with regard
to ( .] Example 3. Shew that

2V-) '" lEI)[vi-!l = Jo' '>  *" < '-

(Frobenius, Journal fur Math. LXXiil. (1871), p. 1.)
\Section{15}{5}{Ferrers associated Legendre functions TODO}
 We shall now
introduce a more extended class of Legendre functions. If m be a
positive integer and - 1 < < 1, n being unrestricted §, the functions

P- (Z) = (1 - Zf - " "1, Q rn (,) (1 \,.)i d "Qn( )

* The proof is similar to the proof in\hardsectionref{15}{4} that that convergence is
uniform with regard to t.

t Proc. London Math. Soc. (2), vi. (1908), pp. 388-395; (2), vii.
(1909), pp. 24-39.

X Miinchener Sitzungsberichte, xxxix, (1909), No. 10.

§ See p. 317, footnote. Ferrers writes r ' z) for P "' (z).

21 2

%
% 324
%

will be called Ferrers' associated Legendre functions of degi-ee 71
and order on of the first and second kinds respectively.

It may be shewn that these functions satisfy a differential equation
analogous to Legendre's equation.

For, differentiate Legendre's equation

(1 \ 2) \ 2 +,, (n + 1 ) y = - dz- dz

111 times and write v for - - . We obtain the equation

(1 - ), - 2 ( + 1 ) + ('i - '>0 (' + i + 1) = 0.

Write lu = (1 - 2'-) '" '. and we get

ox d"W dw f /, 1 X m ]

(1 - ) d.-- - 2 s + r ( + 1) - 1 = 0-

This is the differential equation satisfied by P,,"* (z) and Qj'' (z).
From the definitions given above, several expressions for tlie
associated Legendre fiinctious may be obtained.

Thus, from Schlafli's formula we have

J n-ri i J A

where the contour does not enclose the point = - 1.

Further, when n is a positive integer, we have, by Rodrigues' formula,

p in (2) = i i i '-

Example. Shew that Legendre's associated equation is defined by the
scheme / oc 1

f) im n + 1 Ui l-u \addexamplecitation{Olbricht.}

[ - im - H - hn j

15"51. The integral pro'perties of the associated Legendre functions.
The generalisation of the theorem of\hardsubsectionref{15}{1}{4} is the following : When
n, r, m, are positive integers and n > m, r > m, then

.1 I (ri n),

Pn'''(z)Pr"H2)dz\, .,

J 1 2 (n+ m)\

2?i + 1 (n - m)l

To obtain the first result, multiply the differential equations for P
"(z),. Pr" z) by P/" (z), Pn" (z) respectively and subtract; this
gives

+ (n- r) (n + r + 1) P ( ) Pn'" 2) = 0.

dz

%
% 325
%

On integrating between the limits - 1, +1, the result follows when n
and r are unequal, since the expression in square brackets vanishes at
each limit.

To obtain the second result, we observe that

squaring and integrating, we get

P " + z) = l- z-f - - + mz (1 -z"-)- P, - z);

  '"( )r\ o-..z.,./.. "( )

j'j P,r (z)Ydz=j' (l- )|' ' HV27n P -( )

dz

.' - 1 1-2 -

on integrating the first two terms in the first line on the right by
parts. If now Ave use the differential equation for Pn " (z) to
simplify the first integral in the second line, we at once get

j P '"+ (z)\' dz = (n - m) (71 + m + 1) I [Pn'" i )]' dz. By repeated
applications of this result we get

1 P "* z)]- dz = (n - m +l)(n- m + 2) . . . ??

.' -1

X n + m) (n + vi-l)... (n + I) \ P (z)Y dz,

J -1

and so f |P.-(.);.\&= 2 <i±!!i |.

j -1 2?l + 1 n - 7/t):

\Section{15}{6}{Hohsons definition of the associated Legendre functions.}

So far it has been taken for granted that the function (1- 2 ) '"
which occurs in Ferrers' definition of the associated functions is
purely real; and since, in the more elementary physical applications
of Legendre functions, it usually happens that - 1<2 <1, no
complications arise. But as we wish to consider the associated
functions as functions of a complex variable, it is undesirable to
introduce an additional cut in the 2 -plane by giving arg(l - z) its
principal value.

Accoi'dingly, in futm e, when z is not a real number such that -\ < z<
1, we shall follow Hobson in defining the associated functions by the
equations

p- (.) = (.' - i)i ' ''"'j;; ), Q.r (.) = iz'- i)i * >,

where ni is a positive integer, n is unrestricted and arg z, arg ( +
1), arg ( - 1) have their principal values.

%
% 326
%

When 111 is unrestricted, P;i'" (s) is defined by Hobson to be

f(T3; )( ) (- ...+i;i- .;i-J );

and Barnes has given a definition of § '" z) from which the formula

O m ( sm( + m) 7f T(n+m + ) T \ ) ( -1) "' " > sinWTT 2 + ir(7i + f) n
+ m + l

may be obtained.

Throughout this work we shall take m to be a positive integer.

15"61. Expression of Pn z) as an integral of Laplace s type.

If we make the necessary modification in the Schlafli integral of §
15"5, in accordance with the definition of\hardsectionref{15}{6}, we have

p, . (,) = (n + !)(' + 2);.. ( + - 0 (,, \ 1 )J,,. |- --- (,, \ 1)..
(, \,)\ \ -. rf. Write i = + (2- - 1)2 e'*, as in\hardsubsectionref{15}{2}{3}; then

where a is the value of when Hs at, so that

I arg z- - 1)- + a j < TT.

Now, as in\hardsubsectionref{15}{2}{3}, the integrand is a one- valued periodic function
of the real variable with period 27r, and so

Since 2 + ( - 1) cos ]" is an even function of </>, we get, on
dividing the range of integration into the parts (- tt, 0) and (0,
tt),

PrJ z) = - ' I [z + z - yf cos j'* cos m< d<f).

TT Jo

The ranges of validity of this formula, which is due to Heine,
(according as n is or is not an integer) are precisely those of the
formula of § 15 "23.

Example. Shew that, if | arg z I < tt,

n n- I) ... n-m-

Pn"' ) = (-)"

' + !) f cosm(f)d(f)

Jo 3 + (32\ i)icos< + i'

where the many-valued functions are specified as in § 15 23. 
\Section{15}{7}{The addition theorem for the Legendre polynomials*.}
Let z=xaf - x - 1)
(x' - Vp cosco, where x, x', a are unrestricted complex numbers.

* Legendre, Calc. Int. 11. pp. 262-269. An investigation of the
theorem based on physical reasoning will be given subsequently § 18
"4).

%
% 327
%

Then we shall shew that

Pn Z) = Pn X) Pn x') + 2 2 ( - ) ' "" PrT ( ) ?/" ( ') COS r.m.

m=i n + ni) \

First let R (of) > 0, so that +(- " ) <-OM< -9J j bounded function of
in the I af + x'' -l) cos(f> ! range < < < 27r. If M be its upper
bound and if j A j < i/""', then

i .r + ( .r2-l) cos( a,-0)

converges uniformly with regard to 0, and so \hardsectionref{4}{7})

i A' f '' + ( '-1) cos (0,-0 ) " p - A" :c + ( 2-l) cos(co-( ) "

 =0 j- r y + (y2\ l) COS0 + ' J -n n=i) .r' + (;ar:'2 - 1) COS 0 " + 1

\ /""

- -' y + (j7'2- l)i COH(f>-h x+ x--'l) COS <0-(f>)

Now, by a slight modification of example 1 of\hardsubsectionref{6}{2}{1}, it follows that

/"" d(f) \ 2v

J -n A + B COS ( + C sin ( (A-- B - C' ) ' where that value of the
radic;il is taken which makes

Therefore

r -

+ (.r'2 - l)i cos (f)-h x + x2 - 1) cos ( - (j))

2n

  a/ - hxf - of' \ 1 )i - A (a;2 - 1 )i cos a) 2 - [h ( 2 \ i )i in a,
2ji

27r (l-2> 0+A2)i '

and when h~ 0, tliis expression has to tend to 27r Po ( ') by j5
15-23. Expanding in powers of h and equating coefficients, we get

Now /' (2) is a polynomial of degree ti in cos, and can C(jnsequently
be expressed in

H

the form i.4o+ 2 J, cosjrtco, where the coefficients J,, Jj, ... A
are independent of w 

to determine them, we use Fourier's rule ( 9"12), and we get

1 / " A,n = I Pn (s) cos ma d(o

'"' J -n

\ 1 I" r f"" .V + x -l P cos (<a - </>) " cos Wlm, . "], 2 r2J\ Li-T
a;' +(a;'2- 1)4 cos (/> + J

\ \ 1\ /"' r / " .r + (. 2 - 1)4 cos (o) - 0) cos 7W<0 -]

2 -''' i -T L j -T y + (a/2\ l)4cos< + i J

- J f " r [" + (.r2 - 1 )4 cos /- " cos m (0 + >// ),. 1,, 27r2J\
;,Li-T ' + (a;'2-l)4cos0 +i J

on changing the order of integration, writing £ij = + >//' and
changing the limits for >// from +7r - to ±ir.

%
% 328
%

/ " i .

Now I x + x - l) cos-yj/ " sinm-\ j/d-\ lr = 0, since the integrand is
an odd function;

and so, by\hardsubsectionref{15}{6}{1},

111 [ cos mcf) . P (x),

7r n + m)l J -,r' + (,f'2 - 1 ) 3 cos ( + 1

Therefore, when | arg / | < 7r,

P ( ) = P ( 0i*nG ') + 2 2 (-)'nL p ' (. ')P (. '')cosma>.

 t=i (% + m; !

But this is a mere algebraical identity in x, x and cos <o (since n is
a positive integer)

and so is true independently of the sign of R x').

The result stated has therefore been proved.

The corresponding theorem with Ferrers' definition is

TODO

15 "71. The addition theorem for the Legendre functions.

Let .r, x' be two constants, real or complex, whose arguments are
numerically less than 77; and let ( '±1), (*''±1) be given their
princiiJal values; let co be real and let

z=xx' - x' - 1)2 x'' ~ 1)2 cos W. . Then loe shall sheiv that, if |
arg s | < tt for all valves of the real variable co, and n he not a
positive integer,

P z) = P X) P X') +-2 2 ( - )- jT :; ) n'" ( ') / n'" ( ') COS mo,.

Let cosh a, cosh a' be the semi-major axes of the ellipses with foci
+1 passing through X, x' respectively. Let, jS' be the eccentric
angles of x, x' on these ellipses so that

Let a + i =, a' + i/3' = ', so that .*' = cosh, .r'=:Cosh '.

Now as G) passes through all real values, R (z) oscillates between

R (xx') ±R x -l)i (a''2 - 1 ) 2 = cosh (a ± a') cos (/3 ± '), so that
it is necessary that ±/3' he acute angles positive or negative. Now
take Schlafli's integral

and write

\ e e~"" sinh cosh |g' - cosh | sinh \ \ '] + cosh g cosh l ' -e""
sinh sinh g'

cosh ig'+e sinh ' The path of ?:, as increases from - tt to tt, may be
shewn to be a circle; and the reader will verify that

\ 2 e' * ~" cosh Jg + s inh g sinh igcoshjf - e'" cos h |g sinh if

 ~ - " id, " ' '

cosh ig' + e sinh ig' \ 2 i " " " siuh ig + cos h g cosh igco |g' - /
" sinh g sinh i|' cosh Jg' + e sinh |g'

\ e cosh g' -|- sinh g' e'" sinh g sinh ig' + e *" sinh g cosh ig' -
cosh g sinlij'

cosh ig' + e'* sinh ig'

%
% 329
%

Since* j cosh ' | > | sinh i|' |, the argument of the denominators
does not change when < increases by 27r; for similar reasons, the
arguments of the first and third numerators increase by 27r, and the
argument of the second does not change; therefore the circle contains
the points i = l, t = z, and not t= - 1, so it is a possible contour.

Making these substitutions it is readily found that

J\ p x + (x -l) cos a>-(t>) '' cos 0 " + 1

2n J - y + ( '2-l)i

,

and the rest of the work follows the course of\hardsectionref{15}{7} except that the
general form of Fourier's theorem has to be employed.

Example. Shew that, if n be a positive integer,

Qn ' + (x -l ) (.r'2 - l)i cos a,J = \$ (x) /> x') + -2 2 \$ '" (x) P
- ' x) cos m<o,

when (o is real, R (.* ') 0, and | x - 1) x+ l)\ < \ (x- 1) ( + 1) |.

(Heiiie, Kiigelfunktionen; K. Neumann, Leipziger Abh. 1886.)

\Section{15}{8}{TODO.}

A function connected with the associated Legeudre function Pn'" (z) is
the function n" (2), which for integral values of n is defined to be
the coefficient of /< in the expansion of (1 - 2kz + h )~'' in
ascending powers of /t.

It is easily seen that C,," (2) satisfies the dift'erential equation

c?2y (2i/ + l) 2 o?y n(H + 2v) \ rf "*" z -l dz z -\ y '

For all values of n and i it may l)e shewn that we can define a
function, satisfying this equation, by a contour integral of the form

  > jc t-zY "'''

where C is the contour of v; 1 ")"2; this corresponds to Schlafli's
integral. The reader will easily prove the following results :

(I) When n is an integer

C''(z)= -2Yv v + ) ... v+ n- ) a\, h--dl. \ i n+.-h.

"' n\ \ {2n + 2v- ) 2n + 2v-2)... n + 2vy ' dz ' ' '

since P (s) = C - (s), Rodrigues' formula is a particular case of this
result.

(II) When r is an integer,

c*" -',z\= L, p (,\

n-,' ' (2r-l)(2r-3)...3.1 dz " ''

whence C*" (z) = - ~ Pj (2).

n-r ' (2r-l)(2/--3)...3. 1 "

The last equation gives the connexion between the functions C (2) and
P/ (2).

* This follows from the fact that cos/3'>0.

t This function has been studied by Gegenbauer, Wiener
Sitzungsberichte, lxx. (1874), pp. 434- 443; Lxxv. (1877), pp.
891-896; xcvii. (1888), pp. 259-316; en. (1893), p. 942.

%
% 330
%

(III) Modifications of the recurrence-formulae for /* (2) are the
following :

   = 2 0" + (4 >i67(2) = ( -H-2,/)2C"' (2)-2,.(l-22)(7''-l(2).

REFERENCES.

A. M. Legendre, Calcul Integral, 11. (Paris, 1817).

H. E. Heine* Handbuch der Kugelfimktionen (Berlin, 1878).

N. M. Ferrers, Spherical Harmonics (1877).

I. Todhunter, Functions of Laplace, \Lame\ and Bessel (1875).

L. ScHLAFLi, Ueber die zwei Heine schen Kugelfunhtionen (Bern, 1881).

E. W. HoBSON, Phil. Trans, of the Royal Society, 187 a (1896), pp.
443-531.

E. W. Barnes, Quarterly Journal, xxxix. (1908), pp. 97-204.

R. Olbricht, Stiidien ueber die Kugel- und Cylinder funhtionen (Halle,
1887). [Nova Acta Acad. Leop. lii. (1888), pp. 1-48.]

N. Nielsen, Theorie des fonctions metaspheriques (Paris, 1911).

Miscellaneous Examples f.

1. Prove that when n is a positive integer,

\addexamplecitation{Math. Trip. 1898.}

yi dP

2. Prove that z z )

I \ i as

is zero unless rn - n= + 1, and determine its value in these cases.

dz

\addexamplecitation{Math. Trip. 1896.}

3. Shew (by induction or otherwise) that when n is a positive integer,

(27t + l) P 2( ), = l\ p 2\ 2,(P,2 + p,2+... + p2 \,) + 2(PlPo + P2A
+ + -l ) \addexamplecitation{Math. Trip. 1899.}

4. Shew that

zP,: (z) = nP z) + 2n - 3) P \ 2 (z) + 2n - 7) P \ 4 (2) + .  . 

\addexamplecitation{Clare, 1906.}

5. Shew that

z'P " (z) = nin-l) Pn z)+ 2 (2n-4'/-+l) r 2n-2r + l)-2 P \ 2, z),
where p = hi or I (n-l). \addexamplecitation{Math. Trip. 1904.}

* Before studying the Legendre function P (z) in this treatise, the
reader should consult Hobson's memoir, as some of Heine's work is
incorrect.

t The functions involved in examples 1-30 are Legendre polynomials.

%
% 331
%

6. Shew that the Legendre polynomial satisfies the relation

 z -\ f = n n- ) n + ) n + 2) j dz A dz.

(Trin. Coll. Dublin.)

7. Shew that

Pap f \ n ! \ 7 - n n + ) \

j z + i -i( -(2 \ l)(2ri + l)(2 + 3)-

\addexamplecitation{Peterhouse, 1905.}

8. Shew that the values of ) a - z f P " (z) P ' z) dz are as follows
:

.' -1

(i) 8'rt (n + 1) when m - n is positive and even,

(ii) -2 ( 2\ i)(7i-2)/(2n + l) when ?/i=n,

(iii) for other values of m and n. \addexamplecitation{Peterhouse, 1907.}

9. Shew that

sin /' (sin;= I ( - )*", / ',, cos'- gP (cos ).

\addexamplecitation{Math. Trip. 1907.}

10. Shew, by evaluating / " P (cos 6) dd \hardsectionref{15}{1} example 2), and then
integrating by

/ I '" :,., a.3...(?i-2)] parts, that | /' (/x) arc sni /x . c/ x is
zero when n is even and is equal to tt < (nA-'W i

when n is odd. \addexamplecitation{Clare, 1903.}

11. If wi and be positive integer.s, and m n, shew by induction that

P, P, V lm-r r -r /2?t + 2 l - 4/-+ 1 \

,, 1.3.5...(2m-l)

where 1,,,= - r- 

m !

(Adams, Proc. Royal Soc. xxvil.)

12. By expanding in ascending powers of u shew that

I -\ n ffn,

where 2( is to be replaced by (1-2 ) after the diflFerentiation has
been performed.

13. Shew that P (z) can be expressed as a constant multiple of a
determinant in which all elements parallel to the auxiliary diagonal
are equal (i.e. all elements are equal for which the sum of the
row-index and column-index is the same); the determinant containing n
rows, and its elements being

\ 1 1 11 1

3' 3 ' '5' 5 '""2 -l'*

(Heun, Gott. Nach. 1881.)

14. Shew that, if the path of integration passes above i=l,

dt. \addexamplecitation{Silva.}

p (,.1 [- zJl-J) 2Hl-z ) - " "rrt jo " (l-f2)" i

15. By writing cot ' = cot d - k cosec d and expanding sin ff in
powers of A by Taylor's theorem, shew that

P (cos 6) = - cosec" " ' 6 V i . \addexamplecitation{Math. Trip. 1893.}

' ft ! d (cot 6)

%
% 332
%

16. By considering S /i"P (0), shew that

(Glaisher, Proc. London Math. Soc. vi.)

17. The equation of a nearly spherical surfixce of revolution is

TODO, where a is
small; shew that if a be neglected the radius of curvature of the
meridian is

1 + a 2 n(4j +3)-(m + l)(8/H + 3) P2m + i(cos<9).

\addexamplecitation{Math. Trip. 1894.}

18. The equation of a nearly spherical surface of revolution is

r = a l+ePn i os 6), where e is small.

Shew that if e be neglected, its area is

47ra2 ji + 1,2 l . \addexamplecitation{Trinity, 1894.}

19. Shew that, if k is an integer and

(l-2As + /i2)- = 2 anPni ),

then

where x and ?/ are to be replaced by unity after the differentiations
have been performed.

(Routh, Proc. London Math. Soc. xxvi.) 20. Shew that

' J ' w "-1 ( ) - ' "-1 ( ) ('y> - ' - '

/

= - 1. \addexamplecitation{Catalan.}

21. Let x +9/' +z =r-, z = fir, the numbers involved being real, so
that -l</x<l. Shew that

/ \ n j.n + 1 gn. /J~>

A(m) =

  ! C2"

where r is to be treated as a function of the independent variables
a;,, z in performing the differentiations.

22. With the notation of the preceding example (of. p. 319, footnote
*), shew that

( +i)p (m)+/-A/(/x)= -,, a7>(,-3J-

23. Shew that, if | A | and | 2 | are sufficiently small,

1- 2

(1-2 2 + 2) =0

2 (2 + l)/i"P (2).

%
% 333
%

24. Prove that

\addexamplecitation{Math. Trip. 1894.}

25. If the arbitrary function /(.r) can be expanded in the series

X

f x)= 2 anPn x), n=0

converging uniformly in a domain which inchides the point .y = l,
sliew that the expansion of the integral of this function is

fmd.= -a,-\ a, V -; --- ) /..(...), \addexamplecitation{Bauer.}

26. Determine the coefficients in Neumann's expansion of e" in a
series of Legendre polynomials. \addexamplecitation{Bauer, Journal fur Math, lvi.}

27. Deduce from example 25 that

TT '- (1 .3. 5... (2n-l)) 2,,, ., arcsn.3-=- 2 ....g. .,, ] / . . (
")- A - (.)

\addexamplecitation{Catalan.}

28. Shew that

Qn z) = \ \ 0g( . P z)-\ Pa-x z) P,(Z) + \ P..., Z) P, Z)

+ \ Pn-z P-> z)+-- \ P. Pn-l z) .

\addexamplecitation{\Schlafli; Hermite.}

29. Shew that

Pn.ve also that y (i) = i />. W log J - \, (2),

= f ..( -.) (';r> Ci-X'-.-'-o ''<"--'A<s-'->>-t ( -i-7]

\

. //. \ 1 \ 1 \ 1\ ( -1)( -2)( / + l)(/i + 2)(n+ 3) / IV, * " 2 3/
122232 V 2 J "  "

where - = 1 + + + ... + - . \addexamplecitation{Math. Trip. 1898.}

2 3

30. Shew that the complete solution of Legendre's differential
equation is

the path of integration being the straight line which when produced
backwards passes through the point =0.

* The first of these expressions for / \ i(2) was given by
Christoffel, Journal fiir Math. lv. (1858).

%
% 334
%

31. Shew that s + (s2- 1) =2 A \$2m-a-i (2),

where

32. Shew that, when R )i+1)>0,

27r TO!r(m-a + l) '

[chap. XV

(SchlaBi.

Qn Z)= . -, dh,

and

33. Shew that

  %hz + h?f

 2r-(z--'-l)5 kn

dh

(l-2As + 2)5 r (? + 1 ) /" °° cosh mtt

where the real part of (/i + 1) is greater than m.

du,

\addexamplecitation{Hobson.}

34. Obtain the expansion of /* z) when | arg 2 | < tt as a series of
powers of I/2, when n is not an integer, nameh'

IT

T n + ) T \ y V 2 ' 2' 2 ''' zy

[This is most easily obtained by the method of i 14'51.]

35. Shew that the differential equation for the associated Legendre
function /V" (2) is defined by the schemes*

00

1

\

00 1

hn m -\ n 1 - I, P\ -\ n \ m ~

T y

(Olbricht. 36. Shew that the difi'erential equation for C,," £) is
defined by the scheme

 -1 X 1 \

\ - v n + 2v i- v s,- . -n J

37. Prove that, if

,, (2? + l)(2n + 3)... (27 + 2.-1) . 2\ iv

 ' n n -l) n -4) ... ?i - s-iy n + s) ' dz '

2(2 + l) 271 + 3 then i*,...- - P + 2; P -.

3(2n + 3) . 3J27i\ +5) (2v + 3) (271 + 5)

y3- + 3 2 \ ™+i+ 2 -\ 3 fu-x (271- 1) (271-3) "-"'

and find the general formula.

\addexamplecitation{Math. Trip. 1896.}

See also\hardsectionref{15}{5} example.

%
% 335
%

38. Shew that

p m /cos ) = - r(?t+wi + l) f cos ii + i) <9 - TT + |??t7r V - Am cos
(?t + |)<9-|7r+ 7ft7r Vtt r(n + |) L (2 sin 0)5 2 (2% + 3) (2sin0)

(12 -4m2) (32-4 2 ) cos (n + f)0 f7r + i7r "1 ' 2.4.(2 + 3)(2n + 5)
(2sin0) ' J'

obtaining the ranges of values of m, n and 6 for which it is valid.

\addexamplecitation{Math. Trip. 1901.}

39. Shew that the values of n, for which Pn~"' (cos 6) vanishes,
decrease as 6 increases from to TT when m is positive; and that the
number of real zeros of /* ""' (cos 6) for values of d between -w and
tt is the greatest integer less than n - m+l.

(Macdonald, Proc. London Math. Soc. xxxi, xxxiv.)

40. Obtain the formula

1 rn- \,

- I [1 - 2A cosa) cos + sin o) sin cos(0'- 0)! +/i-] dd= 2 /i"P (cos
\&>)/' (cos 0). 2ir J -TT n=o

\addexamplecitation{Legendre.}

41. H f(x) = x x O) Aud f(x)=-x x<0), shew that, if f(x) can be
expanded into a uniformly convergent series of Legendre pol aiomials
in the range (-1, 1), the expansion is

\addexamplecitation{Trinity, 1893.} (1-2A + AT " :..

42. If 7 T- -TTTTT, = 2 h" 0/ (Z)

shew that

C,," xxi - (a-2 \ i)h (a-j2 - 1) cos 0

  r(2v-i) n 4 r(w-X4-i) r(i/+x) 2(2i/+2X-i) r(wp;,io ~)"" r(n + 2,. +
X)

X x - 1)* (.r,2- i)i c;+;; ( ) (7;+;; (a;,) c[-' (cos< ).

(Gegenbauer, IFiewe?* Sitsimg/sberichte, cil. (1893), p. 942.)

43. If o-,. (2) = P' (  ' -3tz + l)-h t' dt,

where cj is the least root of fi - Ztz + 1 = 0, shew that

(2 + l)o- +,-3(2?i-l).'o- \, + 2( -l)o- \ . = 0, and

4 (4;3 \ 1) an" + 1442V " - : (1 2?i-' - 24? - 291 ) a-,,' - (n - 3)
(2 - 7) 2n + 5) o- - 0, where (r = " -, etc.

(Pincherle, Rendiconti Lincei (4), vii. (1891), p. 74.)

44. If (P-3/i2-fl)- = 2 Rn z)h'\

71=0

shew that 2 ( + l) /2 + i ~ 3(2n+l)2 + (2?i- 1) i? \ 2-0,

 /? +/?' \ ., -2i? ' = 0, and

4 (4 3-1) R "' + 9Gz'iR,:'-z \ 2n + 24n-9l) R,,' - i 2n + 3) 2n + 9)R
=0,

,, d R

where Rj," = -j-, etc.

(Pincherle, Mem. 1st. Bologna (5), i. (1889), p. 337.)

%
% 336
%

45. If (-)=2 r;nWr) ' "'-')" "-')>'

obtain the recurrence-formula

( + l)(2H-l).4 (.r)- (47i2\ l) +l,, \ l(.r)-|-( -l)(2? + l) \ 2( ')
= 0.

\addexamplecitation{Schendel, Journal fur Math, lxxx.}

46. If 11 is not negative and m is a positive integer, shew that the
equation

( 2 - 1 ) + (2'/i + 2) = m ( ( + 2/i + 1 ) 3/ has the two solutions

when X is not a real number such that - 1 .?; 1.

47. Prove that

fClare, 1901.)

48. If F,, x)= I ~; x-\

m=0 ™ 

shew that "'" '"-' £ (e +* ')l =e P (, a),

where P (*', a) is a polynomial of degree 7i in .r; and deduce that

d dx

\addexamplecitation{Trinity, 1905.}

49. If Fn (x) be the coefficient of s" iu the expansion of

2kz

gflZ g ~ ''3

in ascending powers of z, so that

3 .2 \ 2

Fo x) = 1, Fi x) = X, F2 (x) = -g -, etc.,

shew that

(1) F,i x) is a homogeneous polynomial of degree n in x and h,

(2) ' i =Fn-r -) n n

Pn + 1 C-, ) = ( ' + <*) n ( *'5 ) +*-'' TT, A (=' ) )

(3)

f i (.r)<;.r = C i l),

(4) If y = a( F( x) + aiFi (x) + a2F2 x) + ..., where Uo, !, aa? ...
*'i'e real constants, then the mean value of -j~, in the interval from
x= -h to x= +A is a . \addexamplecitation{Leaute.}

50. If Fn (* ) be defined as in the preceding example, shew that,
when -h <x <h,

 2, W = (-0-2- . (cos - -2,- cos - - + cos- + ...j,

\addexamplecitation{Appell.}
\chapter{The Confluent Hypergeometric Function}

\Section{16}{1}{The confluence of two singularities of Riemanns equation}

We have seen (§ 10'8) that the linear differential equation with two
regular singularities only can be integrated in terms of elementary
functions ; while the solution of the linear differential equation
with three regular singularities is substantially the topic of Chapter
xiv. As the next type in order of complexity, we shall consider a
modified form of the differential equation which is obtained from
Riemann's equation by the confluence of two of the singularities. This
confluence gives an equation with an irregular singularity
(corresponding to the confluent singularities of Riemann's equation)
and a regular singularity corresponding to the third singularity of
Riemann's equation.

The confluent equation is obtained by making c -* oo in the equation
defined by the scheme

X c

-.--m —c c — k



P



1;



2

.7 — ni k



The equation in question is readily found to be

We modify this equation by writing TODO and obtain as the equation*
for Wk\^mi\^)



The reader will verify that the singularities of this equation are at
6 and X , the former being regular and the latter irregular ; and when
2m is not an integer, two integrals of equation (B) which are regular
near and valid for all finite values of z are given by the series



* This equation was given by Whittaker, Bulletin American Math. Soc.
x. (1904), pp. 125-134. W. M. A. 22



338 THE TRANSCENDENTAL FUNCTIONS [CHAP. XVI



These series obviously form a fundamental system of solutions.

[Note. Series of the type in \{ \} have been considered by Kummer* and
more recently by Jticobsthalt and Barnes J; the special series in
which k = had been investigated by Lagrange in 1762-1765 (Oeuvres, I.
p. 480). In the notation of Kummer, modified by Barnes, they would be
written iF-i \{h±ni — k; ± 2m + 1 ; 2\} ; the reason for discussing

solutions of equation (B) rather than those of the equation 2

which iFi (a ; p; z) is a solution, is the greater appearance of
symmetry in the formulae, together with a simplicity in the equations
giving various functions of Applied Mathe- matics (see § 16"2) in
terms "of solutions of equation (B).]

16"11. Kummer s formulae.

(I) We shall now shew that, if '2'm is not a negative integer, then

\^-\^--il/,,„(\^) = (-\^)-i\~''W\_,,„,(-\^), that is to say,

., (I + m - k) (f + m - k)

\^ "" -\^"\^ 2 ! (2w + 1) (27\^1 + 2)



h


+ m —


k


1!


\{2m +


1)


i


+ m +


k



\^ \_ .\^ (\^ + w + k) (f + m + k) ., \_

l!(2m + l)\^ 2 ! (2m + 1) (2m +2) \^' '■**

For, replacing TODO by its expansion in powers of z, the coefficient
of TODO in the product of absolutely convergent series on the left is


by § 14"11, and this is the coefficient of \^" on the right§; we have
thus obtained the required result.

This will be called Kummej-'s first formula.

(II) The equation



valid when 2m is not a negative integer, will be called Kummer s
second formula.

To prove it we observe that the coefficient of £"+"'+? in the product
-w-i-i e - *2 1 F, \{m + \^ ; 2wi + 1 ; 2),

* Journal fur Math. xv. (1836), p. 139. t Math. Ann. lvi. (1903), pp.
129-154. X Trans. Camb. Phil. Soc. xx. (1908), pp. 253-279.

§ The result is still true when ?h + \^ + k is a negative integer, by
a slight modification of the analysis of § 14-11.

%
% 339
%

of which the second and third factors possess absolutely convergent
expansions, is (§ 3'73)






nl\{2m+l)\{2m + 2) ...\{27n+7i) ; - -

by Kummer's relation*

F\{2a,2\^; a+/3 + l; A-) = F\{a, \^; a+\^ + h; is\{l-x)\},

valid when \^a'\^ i ; and so the coefficient of 2"+'""'"\^ (by §
14-11) is

(| + w)(§ + m) ... \{n-m--\^ ) T \{-n + h-m)r \{\^ ) n (2i+l)(2m +
2)... (2m + 7i) T \{-m-hi)T\{-n)

n (2m + 1 ) \{2m -- 2) . . . \{2m + n) r (\^ - w - hi) r (\^ - 1?0 '
and when n is odd this vanishes ; for even values of n\{ = 2p) it is

r(i-/)(-\^)(-|)...(i-p)



l.S...\{2p-l) 1



2pl 23p(m + l)(m + 2)...(m+p) 2*P.p \{m + 1) \{m + 2) ... \{m+p)'

16'12. Definition f of the function Wic,m (2) •

The solutions -il/t.imC\^') of equation (B) of § 16"1 are not,
however, the most convenient to take as the standard solutions, on
account of the disappearance of one of them when 2m is an integer.

The integral obtained by confluence from that of § 14*6, when
multiplied by a constant multiple of TODO is TODO:

It is supposed that arg\^ has its principal value and that the contour
is so chosen that the point t = — z is outside it. The integrand is
rendered one- valued by taking ; arg (— \^) | \^ tt and taking that
value of arg (1 + t/z) which tends to zero as i -*- by a path lying
inside the contour.

Under these circumstances it follows from § 5"32 that the integral is
an analytic function of z. To shew that it satisfies equation (B),
write



* See Chapter xiv, examples 12 and 13, p. 298.

+ The function TODO was defined by means of an integral in this manner
by Whittaker, ioc. cit. p. 125.

* A suitable contour has been chosen and the variable t of § 14*6
replaced by - 1.

22—2



340 THE TRANSCENDENTAL FUNCTIONS [cHAP. XVI

and we have without difficulty*

TODO

since the expression in \{ \} tends to zero as i -* + oo ; and this is
the condition that TODO should satisfy (B).

Accordingly the function Wk, m \{z) defined by the integral

is a solution of the differential equation (B).

The formula for TfA;,m (\^) becomes nugatory when \^' - 2 - w is a
negative integer. To overcome this difficulty, we observe that
whenever

and k — \^— m is not an integer, we may transform the contour integral
into an infinite integral, after the manner of § 12-22 ; and so, when

Rik -i-:\^— 7n \^0,

This formula suffices to define Wi\^miz) in the critical cases when m
+ 2->t is a positive integer, and so Wk,m\{z) is defined for all
values of k and m and all values of z except negative real values f.
Example. Solve the equation

d\^u ( b c \_

in terms of functions of the type Tft\_,„ (2), where a, 6, c are any
constants.

\Section{16}{2}{Expression of various functions by functions of the
  type Wk\^,n(z)}

It has been shewn:!: that various functions employed in Applied Mathe-
matics are expressible by means of the function Wk,m (2) ; the
following are a few examples :

* The differentiations under the sign of integration are legitimate by
§ 4-44 corollary.

t When z is real and negative, TODO may be defined to be either TODO
or TODO whichever is more convenient.

+ Whittaker, Bulletin American Math. Soc. x. ; this paper contains a
more complete account than is given here.

%
% 341
%
(I) The Error function* which occurs in connexion with the theories of
Probability, Errors of Observation, Refraction and Conduction of Heat
is defined by the equation

Erfc(\^)=[ e-''dt, where a; is real.

Writing t = x\^\{w\^ — l) and then iu = s/x in the integral for TT.j.
.\^(a\^), we get

Ji = TODO

J X

and so the error function is given by the formula

Erfc (a;) = TODO

Other integrals which occur in connexion with the theory of Conduction
rb of Heat, e.g. TODO, can be expressed in terms of error functions,
and

J a

SO in terms of TODO functions.

Exaviple. Shew that the formula for the eri'or function is true for
complex values of x.

(II) The Incomplete Gamma function, studied by Legendre and othersf,
is defined by the equation

y\{n, x)= j f'-'e-\^dt. Jo

By writing t = s — x in the integral for Wi.,,\^\_i in(x), the reader
will

verify that

TODO

(III) The Logarithmic-integral function, which has been discussed by
Euler and others :|:, is defined, when | arg \{— log \^■j < tt, by the
equation

* This name is also applied to the function

Erf(x)= I ''e-«'d( = 7r\^-Erfc(j-).

•\{• Legendre, Exercices, i. p. 339 ; Hocevar, Zeitschrift fiir Math,
und Phys. xxi. (1876), p. 449; Schlomilch; Zeitschrift fur Math, und
Phys. xvi. (1871), p. 261 ; Prym, Journal filr Math, lxxxii. (1877),
p. 165.

X Euler, Inst. Calc. Int. i. ; Soldner, Monatliche- Correspondenz, von
Zach (1811), p. 182; Briefwechsel zwischen Gauss und Bessel (1880),
pp. 114-120; Bessel, Konigsherger Archiv, i. (1812), pp. 369-405;
Laguerre, Bulletin de la Soc.Math.de France, vii. (1879), p. 72;
Stieltjes, Ann. de VEcole norm. sup. (3), iii. (1886). The
logarithmic-integral function is of considerable importance in the
higher parts of the Theory of Prime Numbers. See Landau, Primzahlen,
p. 11.



342 THE TRANSCENDENTAL FUNCTIONS [cHAP. XVI

On writing 5 — log z = u and then u = — log t in the integral for

it may be verified that

TODO

It will appear later that Weber's Parabolic Cylinder functions (§
16*5) and Bessel's Circular Cylinder functions (Chapter xvii) are
particular cases of the Wk\^rn function. Other functions of like
nature are given in the Miscellaneous Examples at the end of this
chapter.

[Note. The error function has been tabulated by Encke, Berliner ast.
Jahrbuch, 1834, pp. 248-304, and Burgess, Tram. Roy. Soc. Edin. xxxix.
(1900), p. 257. The logarithmic- integral function has been tabulated
by Bessel and by Soldner. Jahnke und Emde, Fiinktionentafeln (Leipzig,
1909), and Glaisher, Factor Tables (London, 1883), should also be
consulted.]

\Section{16}{3}{The asymptotic expansion of TT'\^a;,w \{z), when z is large}
From the contour integral by which Wk,rn,\{z) was defined, it is
possible to obtain an asymptotic expansion for W]c\^m\{z) valid when
|arg\^| < tt.

For this purpose, we employ the result given in Chap, v, example 6,
that

1+-) =1+\^- + ...+ — - + Rn (t, Z),

z) TODO

Substituting this in the formula of § 16'12, and integrating
term-by-term, it follows from the result of § 12-22 that

yyjc,m\{2) = \^ -z ji + YY\^ +— 21\^2 + •••

[m\^ - (k - \^y-] \{m' -(k- \^f] . . . (m\^ -\{k-n-if \^f]



where



n ! z



I ««



provided that n be taken so large that R \{n — k—-\^-- m j > 0.

Now, if I arg 2 | \^ tt — a and z > 1 , then

1\^1(1 + \^/0)1\^1+\^ R\{z)\^Q I (1 + tjz) 1 \^ sin a R (z) \^ O] '

and so*

TODO

* It is supposed that is real ; the inequality has to be slightly
modified for complex values of

%
% 343
%

Therefore Rn\{t,z)

TODO

since 1 + ;< < 1 + t

Therefore, when j \^ | > 1,

= TODO

= TODO

since the integral converges. The constant implied in the symbol is
independent of arg2\^, but depends on a, and tends to infinity as a—
0.

That is to say, the asymptotic expansion of TODO is given by tlie
formula

TODO

for large values ofz when arg 2; | \^ tt - a < tt.

16*31. The second solution of the equation for TODO

The differential equation (B) of § 16"1 satisfied by Wk,m\{z) is
unaltered if the signs of z and k are changed throughout.

Hence, if TODO is a solution of the equation.

Since, when $absval{\arg z} < Tr$,

TODO

whereas, when | arg ( - 2) | < tt,

W.k,.n\{-z) = e\^'H-z)-\^[l + 0\{z-%

the ratio W]c\^m\{z)jW-]c\^jn\{— z) cannot be a constant, and so
Wk\^m\{z) and \^-k,m\{— z) form a fundamental system of solutions of
the differential equation.

\Section{16}{4}{Contour integrals of Barnes type for Wc\^,n(z)}

Consider now

TODO

where | arg\^; ] < - tt, and neither of the numbers k + m + 5 is a
positive integer



344 THE TRANSCENDENTAL FUNCTIONS [CHAP. XVI

or zero*; the contour has loops if necessary so that the poles of T
(s) and those of r ( — s — k—m + 5 ) F (— s — k + m + \^j are on
opposite sides of it.

It is easily verified, by § 13'6, that, as s—>cc on the contour,

TODO

and so the integi-al represents a function of 2 which is analytic at
all points f in the domain j arg z TODO

Now choose N so that the poles oi T (— s — k — m + \^j T (— s — k+ m +
\^j

are on the right of the line R(s) = — N-\^; and consider the integral
taken

round the rectangle whose corners are ± \^i, —N—\^± \^, where \^ is
positive J and large.

The reader will verify that, when 1 arg z <x, the integrals TODO

tend to zero as \^— > 00 ; and so, by Cauchy's theorem,

TODO

where Rn is the residue of the integrand at s = — n.

Write s = — N — I + it, and the modulus of the last integrand is

where the constant implied in the symbol is independent of z.

Since TODO converges, we find that

TODO

* In these cases the series of § 16-3 terminates and TODO is a
combination of elementary' functions.

t The integral is rendered one- valued when J? (z) <0 by specifying
arg z.

X The line joining ±\^j may have loops to avoid poles of the integrand
as explained above.


%
% 345
%
But, on calculating the residue Rn, we get

TODO

and so TODO has the same asymptotic expansion as TODO.

Further / satisfies the differential equation for Wk, m (\^) ! for, on

substituting 1 r\{s) F (- s - k - m + \^j F (- s - k + m + \^j z\^ds
for v in

the expression (given in § 16*12)

,dH \^, dv /, i /, l „ dv

we get

p' r(s)T(-s-k-m. + l)r(-s-k + m + |) z'ds

-j"\r(s + i)r(-s-k-m + l]r(-s-k + m + iy'+'ds

= (r' - r\^"'\^ r(s) rf-s-k-m + f] rf-s-k + m + l\^z'ds.

Since there are no poles of the last integrand between the contours,
and since the integrand tends to zero as | * j — > oo , s being
between the contours, the expression under consideration vanishes, by
Cauchy's theorem ; and so / satisfies the equation for TTjt, m \{2).

Therefore TODO,

where A and B are constants. Making $\absval{TODO} \rightarrow \infty$
when R(z)>0 we see, from the asymptotic expansions obtained for / and
W±k,m\{± \^). that

\^ = 1, 5 = 0.

Accordingly, by the theory of analytic continuation, the equality

I=W,,,n(z) persists for all values of z such that arg\^r|<7r; and, for
values* of arg\^' such that TT < I arg \^ I < I TT, Wk,m (z) may be
defined to be the expression /.

Example 1. Shew that

taken along a suitable contour.

* It would have been possible, by modifying the path of integration in
§ 16-3, to have shewn that that integral could be made to define an
analytic function when $\arg z < TODO$. But the reader will see that
it is unnecessary to do so, as Barnes' integral affords a simpler
definition of the function.



%
% 346
%



Example 2. Obtain Barnes' integral for TODO by writing for TODO in the
integral of TODO and changing the order of integration.



2Trl J \_ ,



16"41. Relations betiveen Wk,m\{z) '-\^nd Mk,±m\{z)- If we take the
expression

TODO

which occurs in Barnes' integral for TODO. and write it in the form

TODO

r(s + k + m + \^)r\{s + k — m + \^)cos\{s + k + m) ir cos \{s--k - m)
it ' we see, by § 13'6, that, when B (s) \^ 0, we have, as , 5 | — > x
,



F\{s) = exp|C-s-\^-2\^'jlog5 +



sec \{s + k -- m) tt sec \{s + k — m) ir.



Hence, if | arg 2\^ | < \^ tt, jF\{s)z--ds, taken round a semicircle
on the

right of the imaginary axis, tends to zero as the radius of the
semicircle tends to infinity, provided the lower bound of the distance
of the serai- circle from the poles of the integrand is positive (not
zero).

Therefore Tf ,,..(.) = - r(\_\^\_,,, +\^)r(-yfc + m + f) '

where SR' denotes the sum of the residues of F(s) at its poles on the
right of the contour (cf. § 14"5) which occurs in equation (C) of §
16*4.

Evaluating these residues we find without difficulty that, when

I arg 2 : < I TT, and 2m is not an integer*,

„, , , T(-2m) ,. , , r(2m) ,\^ , ,

Example 1. Shew that, when | arg ( -s) | <|7r and 2m is not an
integer,

(Earnest.) Example 2. AVhen - -stt < arg s < f tt and - f tt < arg ( -
\^) < \^tt, shew that

* When 1m is an integer some of the poles are generally double poles,
and their residues involve logarithms of z. The result has not been
proved when fe- |i 7h. is a positive integer or zero, but may be
obtained for such values of k and m by comparing the terminating
series for '\^)t,m (2) with the series for Mj.\^±„, (2).

t Barnes' results are given in the notation explained in § 16-1.


%
% 347
%

Example 3. Obtain Kummer's first formula (§ 16'11) from the result

TODO (Barnes.)

Inl J -xi

\Section{16}{5}{The parabolic cylinder functions. Weber's equation}

Consider the differential equation satisfied by TODO it is

TODO

this reduces to TODO

Therefore the function satisfies the differential equation

Accordingly Dn(z) is one of the functions associated with the
parabolic cylinder in harmonic analysis*; the equation satisfied by it
will be called Weber's equation.

From § 16'41, it follows that

Q

when I arg z < -tt.



But



z



4 -i






and these are one-valued analytic functions of z throughout the
TODO-plane. Accordingly Dn (z) is a one-valued function of z
throughout the TODO-plane ; and,

by § 16"4, its asymptotic expansion when arg \^ < - tt is

TODO

16'51. The second solution of Weber's equation.

Since Weber's equation is unaltered if we simultaneously replace n and
z by — n — 1 and + iz respectively, it follows that D\_n-i (iz) and
D-n-i (— iz) are solutions of Weber's equation, as is also Dn (— z).

* Weber, Math. Ann. i. (1869), pp. 1-36; Whittaker, Proc, London Math.
Soc. xxxv. (1903), pp. 417-427.



348 THE TRANSCENDENTAL FUNCTIONS [CHAP. XVI

It is obvious from the asymptotic expansions of Dn\{z) and
Z)\_„\_i(2e\^'\^*), valid in the range — \^ tt < arg z < -ir, that the
ratio of these two solutions is not a constant.

16'511. The relation between the functions Dn\{z), D\^n\_\^ (+ iz).

From the theory of linear diiSerential equations, a relation of the
form Dn \{z) = aD\_n-i \{iz) + h D\_„\_i (- iz) must hold when the
ratio of the functions on the right is not a constant.

To obtain this relation, we observe that if the functions involved be
expanded in ascending powers of z, the expansions are

H i=nr\^ — 1 — \^ — z + ... ■



Comparing the first two terms we get

a = (27r) - * r (w + 1) e\^'''' h = (27r) " \^ T (w + 1) e " and so

r(n + l)



I>n\{z) =



TODO

16'52. The general asymptotic expansion of Dn \{z). So far the
asymptotic expansion of D\^ (z) for large values of z has only

been given (§ ] 6*5) in the sector arg z < jTt. To obtain its form for
values

of arg z not comprised in this range we write — iz for z and —n — 1
for 7i in the formula of the preceding section, and get

TODO

Now, if TODO, we can assign to TODO and TODO arguments between

TODO + J TT ; and arg (— z) = arg z — ir, arg (— iz) = arg z — \^tt;
and then, applying the asymptotic expansion of § 16"5 to Dn\{— z) and
D\^n-i\{—iz), we see that, TODO

%
% 349
%

This formula is not inconsistent with that of § 16-5 since in their
common range of validity, viz. TODO for all positive values of w.

To obtain a formula valid in the range TODO, we use the formula

and we get an asymptotic expansion which differs from that which has
just been obtained only in containing e""'\^' in place of e'"\^'.

Since Dn(z) is one-valued and one or other of the expansions obtained
is valid for all values of arg z in the range — tt \^ arg z \^tt, the
complete asymptotic expansion of i)„ (z) has been obtained.

\Section{16}{6}{A contour integral for Dn(z)}

no+) \_ ,1,2 Consider TODO, where TODO; it represents a one-valued

analytic function of z throughout the \^-plane (§ 5-32) and further

the differentiations under the sign of integration being easily
justified ; accordingly the integral satisfies the differential
equation satisfied by e \~ i\^ Z)„ (2) ; and therefore

e-\^'- e-'f-\^i-t)-'\^-'dt = aB„\{z) + bD\_„\_i\{iz),

where a and b are constants.

Now, if the expression on the right be called E\^ (2)5 we have

En\{0)= e-¥\^\{-t)-\^-\^dt, En'\{0)= e-¥-t)-\^dt.

To evaluate these integrals, which are analytic functions of n, we
suppose first that R \{n) <0 ; then, deforming the paths of
integration, we get

TODO

Similarly TODO.

Both sides of these equations being analytic functions of TODO, the
equations are true for all values of n ; and therefore

TODO

Therefore TODO.



350 THE TRANSCENDENTAL FUNCTIONS [CHAP. XVI

16*61. Recurrence formulae for Dn \{z). From the equation

after using § 16*6, we see that

Dn+, (z) - z Dn (z) + n Dn-, \{z) = 0. Further, by differentiating the
integral of §16"6, it follows that

D\^ \{z) + zDn \{z) - nDn-, \{z) = 0. Example. Obtain these results
from the ascendiug power series of § 16"5.

\Section{16}{7}{Properties of Dn (z) when n is an integer}

When n is an integer, we may write the integral of § 16"6 in the form

TODO

If now we write t = v — z, we get

TODO

a result due to Hermite*.

Also, if m and n be unequal integers, we see from the differential
equations that

Dn \{z) Dm" (z) - D„, \{z) Dn" (z) + (m - n) Dm (z) Dn (z) = 0, and so



Dn\{z)DJ\{z)-Dm(z)Dn'\{z)



\{m - 71) I D„i (z) Dn (z) dz =

J —ex

= 0,

by the expansion of § 16"5 in descending powers of z (which terminates
and is valid for all values of arg z when n is a positive integer).

Therefore if m and n are unequal positive integers

D„,iz)Dn\{z)dz=0.

■>

Comptes Rendus, lviii. (1864), pp. 266-273.


%
% 351
%

On the other hand, when 7n = n, we have

J — cc

= D, (Z) Dn+, (Z)] + I \l zDn iz) D,,\^, \{Z) - Dn+\^ (z) D\^' (z) dz

=r [D,\^,\{z)Ydz,

J —X

on using the recurrence formula, integrating by parts and then using
the recurrence formula again.

It follows by induction that

f \{Dr,(z)Y-dz = nir [D,\{z)Ydz

J —CO

= (27r)\^n!, by § 12-14 corollary 1 and § 12-2.

It follows at once that if, for a function /(\^), an expansion of the
form

/(z) = aoDo (z) + a, D,\{z)+... + a\^Dn \{z)+ ...

exists, and if it is legitimate to integrate term-by-term between the
limits — oc and oo , then



TODO



REFERENCES.

W. Jacobsthal, Math. Ann. LVi. (1903), pp. 129-154.

E. W. Barnes, Trans. Camb. Phil. Soe. xx. (1908), pp. 253-279.

E. T. Whittaker, Bulletin American Math. Soc. x. (1904), pp. 125-134.

H. Weber, Math. Ann. i. (1869), pp. 1-36.

A. Adamoff, Ann. de VInstitut Polytechnique de St Petershourg, v.
(1906), pp. 127-143.

E. T. Whittaker, Proc. London Math. Soc. xxxv. (1903), pp. 417-427.

G. N. Watson, Proc. London Math. Soc. (2), viii. (1910), pp. 393-421;
xvn. (1919), pp. 116-148.

H. E. J. CuRZON, Proc. London Math. Soc. (2), xii. (1913), pp.
236-259.

A. Milne, Proc. Edinburgh Math. Soc. xxxii. (1914), pp. 2-14; xxxill.
(1915), pp. 48-64.

N. Nielsen, Meddelelser K. Danske Videnskabernes Selskab, i. (1918),
no. 6.



352 the transcendental functions [chap. xvi

Miscellaneous Examples.

1. Shew that, if the integral is convergent, then

TODO

2. Shew that TODO

3. Obtain the recurrence formulae

4. Prove that Ht ,,1(2) is the integral of an elementary function when
either of the numbers k-h + m is a negative integer,

5. Shew that, by a suitable change of variables, the equation can be
brought to the form



\{a.2 + b2.v)\^ + \{ai + biX)-£ + \{ao + box)i/ =



derive this equation from the equation for F\{a, b; c; x) by writing x
= \^lb and making 6-*-« .

6. Shew that the cosine integral of Schlomilch and Besso \{Oiornale di
Matematiche, VI.), defined by the equation

TODO

is equal to TODO

Shew also that Schlomilch's function, defined \{Zeitschrift filr Math,
und Physik, iv. (1859), p. 390) by the equations

S\{v,z)=j \{l + t)-''e-'\^dt = z''-\^e' I \^du,

is equal to \^i" - 1 \^i\^ W\^ \_ 1 \^ 1 \_ 1 \^ (2).

7. Express in terms of W,c\^„i functions the two functions

TODO

Jot J z t

8. Shew that Sonine's polynomial, defined \{Math. Ann. xvi. p. 41) by
the equation TODO

is equal to TODO

%
% 353
%

9. Shew that the function TODO defined by Lagrange in 1762-1765
\{Oeuv)-es, i. p. 520) and by Abel (Oeuvres, 1881, p. 284) as the
coefficient of /('*' in the expansion of (1 - A)-i e-''Mi-ft) is eqnal
to

10*. Shew that the Pearson-Cunningham function \{Proc. Royal Soc.
Lxxxi. p. 310), <>>n,m (■')) defined as



T\{n






n- hm



is equal to \^ "\^, " ,, . " \^ ("\^ + 1) e " \^MF„ \^ , , \^ \{z).

11. Shew that, if | arg z < \^n, and | arg (1 + \^) | < tt,

(Whittaker.)

12. Shew that, if n be not a positive integer and if ! arg z < frr,
then

TODO

and that this result holds for all values of args if the integral be /
, the contours enclosing the poles of r ( - but not those of r \{\^t
-n).

13. Shew that, if j arg a | < it,

J tc

= \^-\^ r, n\^(-*". hn-¥i<;hn-hn + - l-ia-).

r(-w)r(\^m-\^i + l)aH'+i)

14. Deduce from example 13 that, if the integral is convergent, then

|J e - i'' z""' A«+i (2) c/2 = (v/2)-i-'" r (i + 1) sin (\^ - \{m) n.

(Watson.)

15. Shew that, if n be a positive integer, and if

E, \{x) = T J- \^'' \{z - .v) - 1 n,, (z) dz,

then TODO.

the upper or lower signs being taken according as the imaginary part
of x is positive or negative. (Watson.)

16. Shew that, if n be a positive integer,

TODO

Jo sm

where fi is \^n or J(n- 1), whichever is an integer, and the cosine or
sine is taken as n is even or odd. (AdamoflF.)

* The results of examples 8, 9, 10 were communicated to us by Mr
Bateman. W. M. A. 23



354 THE TRANSCENDENTAL FUNCTIONS [CHAP. XVI

17. Shew that, if n be a positive integer,

where TODO

TODO \addexamplecitation{Adamoff}

18. With the notation of the preceding examples, shew that, when x is
real,

TODO

while Ja satisfies both the inequalities

TODO

Shew also that as v increases from to 1, o- (/') decreases from to a
minimum at- TODO and then increases to at i'=l ; and as v increases
from 1 to $\infty$, o-(v) increases to a maximum at l + /?2 and then
decreases, its limit being zero ; where

TODO \addexamplecitation{Adamoff.}

19. By employing the second mean value theorem when necessary, shew
that



A.(.\^') = V2.(v'0 e



cos(.r«2 —:\^-)iTr) + '



s'n J'



where co,j(.r) satisfies both the inequalities

I X I \^'TT D

when X is real and n is an integer greater than 2. (AdamofF.)



3-35... ia;2 , ,\^, , .1 \_i



20. Shew that, if n be positive but otherwise unrestricted, and if m
be a positive integer (or zero), then the equation in z

has m positive roots when TODO. \addexamplecitation{Milne.}
%
% 355
%
\chapter{Bessel Functions}
\Section{17}{1}{The Bessel coefficients}
In this chapter we shall consider a class of functions known as  \emph{Bessel functions}
or \emph{cylindrical functions}
which have many analogies with the Legendre functions of  Chapter XV(TODO:ref).
Just as the Legendre functions proved to be particular forms of the
hypergeometric function with three regular singularities,  so the
Bessel functions are particular forms of the  confluent
hypergeometric function with one regular and one irregular
singularity. As is the case of the Legendre functions, we first
introduce\footnote{This procedure is due to TODO \Schlomilch}
a certain set of the Bessel functions as coefficients in an
expansion.

For all values of $z$ and $t$ ($t=0$ excepted), the function
$$
e^{ \frac{1}{2} z \left( t - \frac{1}{t} \right)}
$$
can be expanded by Laurent's theorem in a series of positive and
negative powers of $t$. If the coefficient of $t^{n}$, where $n$ is any
integer positive or negative, be denoted by
$\besJ_{n}(z)$\index{Bessel coefficients [$\besJ_{n}(z)$]}, it follows, from
\hardsectionref{5}{6}, that
$$
\besJ_{n}(z) = \frac{1}{2\pi i} \int^{(0+)} u^{-n-1} e^{\half z \theparen{u
    - \frac{1}{u}}} \dmeasure u.
$$

To express $\besJ_{n}(z)$ as a power series in $z$, write $u = 2t/z$; then
$$
\besJ_{n}(z) = \frac{1}{2\pi i} \theparen{\half z }^{n} \int^{(0+)} t^{-n-1}
\exp \thebrace{ t - \frac{z^{2}}{4t}  } \dmeasure t
$$
since the contour is any one which encircles the origin once
counter-clockwise, we may take it to be the circle
$\absval{t}=1$; as the integrand can be expanded in a series of powers
of $z$ uniformly convergent on this contour, it follows from
\hardsectionref{4}{7} that
$$
\besJ_{n}(z)
=
\frac{1}{2\pi i}
\sum_{r=0}^{\infty} \theparen{\half z}^{n+2r}
\int^{(0+)} t^{-n-r-1} e^{t} \dmeasure t.
$$

Now the residue of the integrand at $t=0$ is
$\thebrace{ (n+r)!  }^{-1}$ by
\hardsectionref{6}{1}, when $n+r$ is a positive
integer or zero; when $n+r$ is a negative integer
the residue is zero.

Therefore, if $n$ is a positive integer or zero,
\begin{align*}
  \besJ_{n}(z) =& \sum_{r=0}^{\infty} \frac{ (-)^{r} (\half z)^{n+2r} }{
    r!(n+r)! } \\
  =& \frac{ z^{n} }{ 2^{n} n! }
  \thebrace{ 1 - \frac{z^{2}}{2^{2}.1(n+1)} + \frac{z^{4}}{2^{4}.1.2(n+1)(n+2)}
    - \cdots };
\end{align*}
%
% 356
%
whereas, when $n$ is a negative integer equal to $-m$,
$$
\besJ_{n}(z)
= \sum_{r=m}^{\infty} \frac{ (-)^{r} (\half z)^{2r-m} }{ r!(r-m)!  }
= \sum_{s=0}^{\infty} \frac{ (-)^{m+s} (\half z)^{m+2s}  }{ (m+s)! s!  },
$$
and so $\besJ_{n}(z) = (-)^{m} \besJ_{m}(z)$.

The function $\besJ_{n}(z)$, which has now been defined for all integral
values of $n$, positive and negative, is called the
\emph{Bessel coefficient} of order $n$; the series defining it
converges for all values of $z$.

% \begin{smalltext}
We shall see later (\hardsectionref{17}{2}) that Bessel
coefficients are a particular case of a class of functions known as
\emph{Bessel functions}.

The series by which $\besJ_{n}(z)$ is defined occurs in a memoir by Euler,
on the vibrations of a stretch circular membrane, TODO,
an investigation dealt with below in \hardsubsectionref{18}{5}{1};
is also occurs in a memoir by Lagrange on elliptic motion, TODO.

The earliest systematic study of the functions was made in 1824 by
Bessel in his TODO; special cases of Bessel coefficients had, however,
appeared in researches published before 1769; the earliest of these is
in a latter, dated Oct. 3, 1703, from Jakob Bernolli to
Leibniz\footnote{TODO}, in which occurs a series which is now
described as a Bessel function of order $\half$; the Bessel
coefficient of order zero occurs in 1732 in Daniel Bernoulli's memoir
on the oscillations of heavy chains, TODO.

In reading some of the earlier papers on the subject, it should be
remembered that the notation has changed, what was formerly called
$\besJ_{n}(z)$ being now written $\besJ_{n}(2z)$.
% \end{smalltext}
\begin{wandwexample}
  Prove that if
  $$
  \frac{ 2b(1+\theta^{2})  }{ (1-2a\theta-\theta^{2})^{2} + 4b^{2}\theta^{2}  }
  =
  A_{1} + A_{2} \theta + A_{3} \theta^{2} + \cdots,
  $$
  then
  $$
  e^{az} \sin bz
  =
  A_{1} \besJ_{1}(z) + A_{2} \besJ_{2}(z) + A_{3} \besJ_{3}(z) + \cdots.
  $$
  \addexamplecitation{Math. Trip. 1896.}
\end{wandwexample}

[For, if the contour $D$ in the $u$-plane be a circle with centre
$u=0$ and radius large enough to include the zeros of the denominator,
we have
$$
e^{\half z \theparen{u - \frac{1}{u}}}
\frac{ 2b \theparen{\frac{1}{u^{2}} + \frac{1}{u^{4}} }  }{ \theparen{1 -
    \frac{2a}{u} - \frac{1}{u^{2}}}^{2}
  + \frac{4 b^{2}}{u^{2}}  }
=
\sum_{n=1}^{\infty}
e^{\half z \theparen{u - \frac{1}{u}}} A_{n} u^{-n-1},
$$
the series on the right converging uniformly on the contour; and so,
using \hardsectionref{4}{7} and replacing the integrals by Bessel
coefficients, we have
\begin{align*}
  \frac{1}{2\pi i}
  \int_{D}
  e^{\half z \theparen{u - \frac{1}{u}}}
  \frac{ 2b \theparen{\frac{1}{u^{2}} + \frac{1}{u^{4}} }  }{ \theparen{1 -
      \frac{2a}{u} - \frac{1}{u^{2}}}^{2}
    + \frac{4 b^{2}}{u^{2}}  }
  \dmeasure u
  =& \frac{1}{2\pi i}
  \int_{D}
  e^{\half z \theparen{u - \frac{1}{u}}}
  \theparen{ \frac{A_{1}}{u^{2}} + \frac{A_{2}}{u^{3}} + \frac{A_{3}}{u^{4}} +
    \cdots  }
  \dmeasure u \\
  =& A_{1} \besJ_{1}(z) + A_{2} \besJ_{2}(z) + A_{3} \besJ_{3}(z) + \cdots .
\end{align*}

%
% 357
%
In the integral on the left write $\half (u - u^{-1}) - a = t$, so that as
$u$ describes a circle of radius $e^{\beta}$, $t$ describes an ellipse
with
semiaxes $\cosh\beta$ and $\sinh\beta$ with foci at
$-a \pm i$; then we have
$$
\sum_{n=1}^{\infty}
A_{n} \besJ_{n}(z)
=
\frac{1}{2\pi i}
\int
\frac{ e^{z(t+a)}b \dmeasure t  }{ t^{2} + b^{2}  },
$$
the contour being the ellipse just specified, which contains the zeros
of $t^{2} + b^{2}$. Evaluating the integral by
\hardsectionref{6}{1}, we have the required result.]
\begin{wandwexample}
  Shew that, when $n$ is an integer,
  $$
  \besJ_{n}(y+z)
  =
  \sum_{m = -\infty}^{\infty} \besJ_{m}(y) \besJ_{n-m}(z)
  $$
  \addexamplecitation{K. Neumann and \Schlafli}
\end{wandwexample}
[Consider the expansion of each side of the equation
$$
\exp \thebrace{ \half (y+z) \theparen{t - \frac{1}{t}}  }
=
\exp \thebrace{ \half y \theparen{t - \frac{1}{t}}  }
\cdot
\exp \thebrace{ \half z \theparen{t - \frac{1}{t}}  }.]
$$
\begin{wandwexample}
  Shew that
  $$
  e^{i z \cos \phi}
  =
  \besJ_{0}(z)
  + 2 i \cos\phi \besJ_{1}(z)
  + 2 i^{2} \cos 2\phi \besJ_{2}(z)
  + \cdots.
  $$
\end{wandwexample}
\begin{wandwexample}
  Shew that if $r^{2} = x^{2} + y^{2}$
  $$
  \besJ_{0}(r)
  =
  \besJ_{0}(x)\besJ_{0}(y)
  - 2 \besJ_{2}(x) \besJ_{2}(y)
  + 2 \besJ_{4}(x) \besJ_{4}(y)
  - \cdots .
  $$
  \addexamplecitation{K. Neumann and Lommel.}
\end{wandwexample}

\Subsection{Bessel's differential equation}
We have seen that, when $n$ is an integer, the Bessel coefficient of
order $n$ is given by the formula
$$
\besJ_{n}(z)
=
\frac{1}{2\pi i}
\theparen{\half z}^{n}
\int^{(0+)}
t^{-n-1}
\exp\theparen{ t - \frac{z^{2}}{4t}  }
\dmeasure t.
$$

From this formula we shall now shew that $\besJ_{n}(z)$ is a solution of the
linear differential equation
$$
\frac{\dd^{2} y}{\dd z^{2}}
+ \frac{1}{z} \frac{\dd y}{\dd z}
+ \theparen{ 1 - \frac{n^{2}}{z^{2}}  } y
= 0,
$$
which is called Bessel's equation for functions of order $n$.

For we find on performing the differentiations
(\hardsectionref{4}{2}) that
\begin{align*}
  \frac{\dd^{2} \besJ_{n}(z)}{\dd z^{2}}
  +& \frac{1}{z} \frac{\dd \besJ_{n}(z)}{\dd z}
  + \theparen{ 1 - \frac{n^{2}}{z^{2}}  } \besJ_{n}(z)
  \\
  %
  =&
  \frac{1}{2\pi i}
  \theparen{ \half z  }^{n}
  \int^{(0+)}
  t^{-n-1}
  \thebrace{ 1 - \frac{n+1}{t} + \frac{z^{2}}{4t^{2}}  }
  \exp\theparen{ t - \frac{z^{2}}{4t}  }
  \dmeasure t \\
  %
  =&
  -\frac{1}{2\pi i}
  \theparen{ \half z  }
  \int^{(0+)}
  \frac{\dd}{\dd t} \thebrace{ t^{-n-1} \exp\theparen{ t - \frac{z^{2}}{4t}  }
  }
  \dmeasure t \\
  %
  =& 0,
\end{align*}
since $t^{-n-1} \exp\theparen{ t - z^{2}/4t  }$ is one-valued.
\emph{Thus we have proved that }
$$
\frac{\dd^{2} \besJ_{n}(z)}{\dd z^{2}}
+ \frac{1}{z} \frac{\dd \besJ_{n}(z)}{\dd z}
+ \theparen{ 1 - \frac{n^{2}}{z^{2}}  } \besJ_{n}(z)
= 0.
$$
The reader will observe that $z=0$ is a regular point and
$z = \infty$ an irregular point, all other points being ordinary
points of this equation.
%
% 358
%
\begin{wandwexample}
  By differentiating the expansion
  $$
  e^{\half z (t - \frac{1}{t})} = \sum_{n=-\infty}^{\infty} t^{n} \besJ_{n}(z)
  $$
  with regard to $z$ and with regard to $t$, shew that the Bessel
  coefficients satisfy Bessel's equation.
  \addexamplecitation{St John's, 1899.}
\end{wandwexample}
\begin{wandwexample}
  The function $P_{n}^{m}\theparen{ 1 - \frac{z^{2}}{2n^{2}}  }$ satisfies the
  equation defined by the scheme
  $$
  TODO
  $$
  shew that $\besJ_{m}(z)$ satisfies the confluent form of this equation
  obtained by making $n \rightarrow \infty$.
\end{wandwexample}
\Section{17}{2}{The solution of Bessel's equation when $n$ is not necessarily
  an integer.}
We now proceed, after the manner of \hardsectionref{15}{2}, to
extend the definition of $\besJ_{n}(z)$ to the case when $n$ is any number,
real or complex. It appears by methods similar to those of
\hardsubsectionref{17}{1}{1} that, for all values of $n$, the
equation
$$
\frac{\dd^{2} y}{\dd z^{2}}
+ \frac{1}{z} \frac{\dd y}{\dd z}
+ \theparen{ 1 - \frac{n^{2}}{z^{2}}  } y
= 0
$$
is satisfied by an integral of the form
$$
y
=
z^{n}
\int_{C}
t^{-n-1}
\exp \theparen{ t - \frac{z^{2}}{4t}  }
\dmeasure t
$$
provided that $t^{-n-1}\exp(t - z^{2}/4t)$ resumes its initial value
after describing $C$ and that differentiations under the sign of
integration is justified.

Accordingly, we define $\besJ_{n}(z)$ by the equation
$$
\besJ_{n}(z)
=
\frac{z^{n}}{2^{n+1}\pi i}
\int_{-\infty}^{(0+)}
t^{-n-1}
\exp\theparen{ t - \frac{z^{2}}{4t}  }
\dmeasure t,
$$
the expression being rendered precise by giving
$\arg z$ its principal value and taking
$\absval{ \arg t } \leq \pi$ on the contour.

To express this integral as a power series, we observe that it is an
analytic function of $z$; and we may obtain the coefficients in the
Taylor's series in powers of $z$ by differentiating under the sign of
integration
(\hardsectionref{5}{32} and \hardsubsectionref{4}{4}{4}
TODO:print has two section symbols together).
Hence we deduce that
\begin{align*}
  \besJ_{n}(z)
  =& \frac{z^{n}}{2^{n+1}\pi i}
  \sum_{r=0}^{\infty} \frac{ (-)^{r} z^{2r}  }{ 2^{2r} r!  }
  \int_{-\infty}^{(0+)}
  e^{t}
  t^{-n-r-1}
  \dmeasure t
  \\
  %
  =&
  \sum_{r=0}^{\infty}
  \frac{ (-)^{r} z^{n+2r}  }{ 2^{n+2r} r! \Gamma(n+r+1)  },
\end{align*}
by \hardsubsectionref{12}{2}{2}. This is the expansion in question.

%
% 359
%
\emph{Accordingly, for general values of $n$, we define the
  \emph{Bessel function} $\besJ_{n}(z)$ by the equations}
\begin{align*}
  \besJ_{n}(z)
  =& \frac{1}{2\pi i} \theparen{\half z}^{n}
  \int_{-\infty}^{(0+)}
  t^{-n-1}
  \exp\theparen{ t - \frac{z^{2}}{4t}  }
  \dmeasure t
  \\
  %
  =&
  \sum_{r=0}^{\infty}
  \frac{ (-)^{r} z^{n+2r}  }{ 2^{n+2r} r! \Gamma(n+r+1)  }.
\end{align*}

This function reduces to a Bessel coefficient when $n$ is an integer;
it is sometimes called a Bessel function \emph{of the first kind}.

The reader will observe that since Bessel's equation is unaltered by
writing $-n$ for $n$, fundamental solutions are $\besJ_{n}(z)$, $\besJ_{-n}(z)$,
except when $n$ is an integer, in which case the solutions are not
independent. With this exception, \emph{the general solution of
  Bessel's equation is
  $$
  \alpha \besJ_{n}(z) + \beta \besJ_{-n}(z),
  $$
  where $\alpha$ and $\beta$ are arbitrary constants.}

A second solution of Bessel's equation when $n$ is an integer will be
given later (\hardsectionref{17}{6}).

\Subsection{The recurrence formulae for the Bessel functions.}
As the Bessel function satisfies a confluent form of the
hypergeometric equation, it is to be expected that recurrence formulae
will exist, corresponding to the relations between contiguous
hypergeometric functions indicated in \hardsectionref{14}{7}.

To establish these relations for general values of $n$, real or
complex, we have recourse to the result of \hardsectionref{17}{2}.
On writing the equation
$$
0 = \int_{-\infty}^{(0+)} \frac{\dd}{\dd t} \thebrace{ t^{-n}
  \exp\theparen{ t - \frac{z^{2}}{4t}  }  } \dmeasure t
$$
at length, we have
\begin{align*}
  0 =& \int_{-\infty}^{(0+)}
  \theparen{ t^{-n} + \frac{1}{4}z^{2} t^{-n-2} - nt^{-n-1}  }
  \exp\theparen{ t - \frac{z^{2}}{4t}  }  \dmeasure t \\
  %
  =& 2\pi i
  \thebrace{ (2z^{-1})^{n-1} \besJ_{n-1}(z)
    + \frac{1}{4} (2z^{-1})^{n+1} \besJ_{n+1}(z)
    - n (2z^{-1})^{n} \besJ_{n}(z)},
\end{align*}
and so
\begin{equation}
  \besJ_{n-1}(z) + \besJ_{n+1}(z) = \frac{2n}{z} \besJ_{n}(z)
  \label{eq:bessel:recur}
  % TODO: fill with dots?
\end{equation}

Next we have, by \hardsubsectionref{4}{4}{4},
\begin{align*}
  \frac{\dd}{\dd z} \thebrace{ z^{-n} \besJ_{n}(z)  }
  =& \frac{1}{2^{n+1}\pi i}
  \frac{\dd}{\dd z}
  \int_{-\infty}^{(0+)}
  t^{-n-1}
  \exp\theparen{ t - \frac{z^{2}}{4t} }
  \dmeasure t
  \\
  %
  =& - \frac{z}{2^{n+2}\pi i}
  \int_{-\infty}^{(0+)}
  t^{-n-2}
  \exp\theparen{ t - \frac{z^{2}}{4t} }
  \dmeasure t
  \\
  %
  =& -z^{-n} \besJ_{n+1}(z),
\end{align*}
%
% 360
%
and consequently, if primes denote differentiations with regard to
$z$,
\begin{equation}
  \besJ'_{n}(z) = \frac{n}{z} \besJ_{n}(z) - \besJ_{n+1}(z)
  \label{eq:bessel:deriv}
\end{equation}

From \eqref{eq:bessel:recur} and \eqref{eq:bessel:deriv} it is easy to
derive the other recurrence formulae
\begin{equation}
  \besJ'_{n}(z) = \half \thebrace{ \besJ_{n-1}(z) - \besJ_{n+1}(z)  },
\end{equation}
and
\begin{equation}
  \besJ'_{n}(z) = \besJ_{n-1}(z) - \frac{n}{z} \besJ_{n}{z}.
\end{equation}
\begin{wandwexample}
  Obtain these results from the power series for $\besJ_{n}(z)$.
\end{wandwexample}
\begin{wandwexample}
  Shew that
  $$
  \frac{\dd}{\dd z} \thebrace{ z^{n} \besJ_{n}(z) } = z^{n} \besJ_{n-1}(z).
  $$
\end{wandwexample}
\begin{wandwexample}
  Shew that
  $$
  \besJ'_{0}(z) = -\besJ_{1}(z).
  $$
\end{wandwexample}
\begin{wandwexample}
  Shew that
  $$
  TODO = \besJ_{n-4}(z) - 4 \besJ_{n-2}(z) + 6 \besJ_{n}(z) - 4 \besJ_{n+2}(z) + \besJ_{n+4}(z).
  $$
\end{wandwexample}
\begin{wandwexample}
  Shew that
  $$
  \besJ_{2}(z) - \besJ_{0}(z) = 2 \besJ''_{0}(z).
  $$
\end{wandwexample}
\begin{wandwexample}
  Shew that
  $$
  \besJ_{2}(z) = \besJ''_{0}(z) - z^{-1} \besJ'_{0}(z).
  $$
\end{wandwexample}

\Subsubsection{Relation between two Bessel functions whose orders
  differ by an integer.}
From the last article can be deduced an equation connecting any two
Bessel functions whose orders differ by an integer, namely
$$
z^{-n-r} \besJ_{n+r}(z) TODO: verify subscript
=
(-)^{r}
\frac{\dd^{r}}{ (z \dd z)^{r} }
\thebrace{ z^{-n} \besJ_{n}(z)  },
$$
where $n$ is unrestricted and $r$ is any positive integer. This result
follows at once by induction from formula \eqref{eq:bessel:deriv},
when it is written in the form
$$
z^{-n-1} \besJ_{n+1}(z)
=
- \frac{\dd}{z \dd z}
\thebrace{ z^{-n} \besJ_{n}(z)  }.
$$
\Subsubsection{The connexion between $\besJ_{n}(z)$ and $W_{k,m}$
  functions.}
The reader will verify without difficulty that, if in Bessel's
equation we write $y = z^{-\half} v$ and then write $z = x/2i$, we get
$$
\frac{\dd^{2} v}{\dd x^{2}}
+
\theparen{ -\frac{1}{4} + \frac{\frac{1}{4} - n^{2}}{x^{2}}  } v
=
0,
$$
which is the equation satisfied by $W_{0,n}(x)$; it follows that
$$
\besJ_{n}(z) = A z^{-\half} M_{0,n}(2iz) + B z^{-\half} M_{0,-n}(2iz).
$$
Comparing the coefficients of $z^{\pm n}$ on each side we see that
$$
\besJ_{n}(z) = \frac{z^{-\half}}{2^{2n+\half} i^{n+\half} \Gamma(n+1)} M_{0,n}(2iz),
$$
%
% 361
%
except in the critical cases when $2n$ is a negative integer;
when $n$ is half of a negative integer, the result follows from
Kummer's second formula (\hardsubsectionref{16}{1}{1}).
\Subsection{The zeros of Bessel functions whose order $n$ is real.}
The relations of \hardsubsectionref{17}{2}{1} enable us to deduce the
interesting theorem that \emph{between any two consecutive real zeros
  of $z^{-n}\besJ_{n}(z)$, there lies one and only one zero\footnote{TODO}
  of $z^{-n}\besJ_{n+1}(z)$.}

For, from relation TODO:addref when written in the form
$$
z^{-n} \besJ_{n+1}(z)
=
- \frac{\dd}{\dd z} \thebrace{ z^{-n} \besJ_{n}(z) },
$$
it follows from Rolle's theorem\footnote{TODO} that between each
consecutive pair of zeros of $z^{-n}\besJ_{n}(z)$ there is at least one zero
of $z^{-n} \besJ_{n+1}(z)$.

Similarly, from relation TODO:addref when written in the form
$$
z^{n+1} \besJ_{n}(z)
=
\frac{\dd}{\dd z} \thebrace{ z^{n+1} \besJ_{n+1}(z) },
$$
it follows that between each consecutive pair of zeros of
$z^{n+1}\besJ_{n+1}(z)$ there is at least one zero of
$z^{n+1}\besJ_{n}(z)$.

Further $z^{-n}\besJ_{n}(z)$ and
$\frac{\dd}{\dd z} \thebrace{ z^{-n} \besJ_{n}(z)  }$ have no common zeros; for
the former function satisfies the equation
$$
z \frac{\dd^{2} y}{\dd z^{2}} + (2n+1) \frac{\dd y}{\dd z} + zy = 0,
$$
and it is easily verified by induction on differentiating this
equation that if both $y$ and $\frac{\dd y}{\dd z}$ vanish for any value of
$z$, all differential coefficients of $y$ vanish, and $y$ is zero by
\hardsectionref{5}{4}.

The theorem required is now obvious except for the numerically
smallest zeros $\pm \xi$ of $z^{-n}\besJ_{n}(z)$, since (except for $z=0$),
$z^{-n}\besJ_{n}(z)$ and $z^{n+1}\besJ_{n}(z)$ have the same zeros. But $z=0$ is a
zero of $z^{-n}\besJ_{n+1}(z)$, and if there were any other positive zero
of $z^{-n}\besJ_{n+1}(z)$, say $\xi_{1}$, which was less than $\xi$, then
$z^{n+1}\besJ_{n}(z)$ would have a zero between $0$ and $\xi_{1}$, which
contradicts the hypothesis that there were no zeros of
$z^{n+1}\besJ_{n}(z)$ between $0$ and $\xi$.

The theorem is therefore proved.

% \begin{smalltext}
[See also \hardsectionref{17}{3} examples TODO and TODO, and example
TODO at the end of the chapter.]
% \end{smalltext}
%
% 362
%
\Subsection{Bessel's integral for the Bessel coefficients.}
We shall next obtain an integral first given by Bessel in the
particular case of the Bessel functions for which $n$ is a positive
integer; in some respects the result resembles Laplace's integrals
given in \hardsubsectionref{15}{2}{3} and \hardsubsectionref{15}{3}{3}
for the Legendre functions.

In the integral of \hardsectionref{17}{1}, viz,
$$
\besJ_{n}(z)
=
\frac{1}{2\pi i}
\int^{(0+)}
u^{-n-1}
e^{\half z \theparen{ u - \frac{1}{u} }}
\dmeasure u,
$$
take the contour to be the circle $\absval{u} = 1$ and write
$u = e^{i\theta}$, so that
$$
\besJ_{n}(z)
=
\frac{1}{2\pi}
\int_{-\pi}^{\pi}
e^{-ni\theta + iz\sin\theta}
\dmeasure \theta
$$

Bisect the range of integration and in the former part write
$-\theta$ for $\theta$; we get
$$
\besJ_{n}(z)
=
\frac{1}{2\pi}
\int_{0}^{\pi}
e^{n i \theta - i z \sin\theta}
\dmeasure \theta
+
\frac{1}{2\pi}
\int_{0}^{\pi}
e^{-n i \theta + i z \sin\theta}
\dmeasure \theta,
$$
and so
$$
\besJ_{n}(z)
=
\frac{1}{\pi}
\int_{0}^{\pi}
\cos (n \theta - z \sin \theta)
\dmeasure \theta,
$$
which is the formula in question.
\begin{wandwexample}
  Shew that, when $z$ is real and $n$ is an integer,
  $$
  \absval{ \besJ_{n}(z) } \leq 1.
  $$
\end{wandwexample}
\begin{wandwexample}
  Shew that, for all values of $n$ (real or complex), the integral
  $$
  y
  =
  \frac{1}{\pi}
  \int_{0}^{\pi}
  \cos (n \theta - z \sin \theta)
  \dmeasure \theta
  $$
  satisfies
  $$
  \frac{\dd^{2} y}{\dd z^{2}}
  +
  \frac{1}{z}
  \frac{\dd y}{\dd z}
  +
  \theparen{ 1 - \frac{n^{2}}{z^{2}} } y
  =
  \frac{ \sin n\pi }{\pi}
  \theparen{ \frac{1}{z} - \frac{n}{z^{2}}  },
  $$
  which reduces to Bessel's equation when $n$ is an integer.

  [It is easy to shew, by differentiating under the integral sign, that
  the expression on the left is equal to
  $$
  -
  \frac{1}{\pi}
  \int_{0}^{\pi}
  \frac{\dd}{\dd\theta}
  \thebrace{ \theparen{ \frac{n}{z^{2}} + \frac{\cos\theta}{z}  }
    \sin(n\theta - z \sin\theta)}
  \dmeasure \theta
  .]
  $$
\end{wandwexample}
%
\Subsubsection{The modification of Bessel's integral when $n$ is not
  an integer.}
We shall now shew that\footnote{TODO}, for general values of $n$,
\begin{equation}
  \besJ_{n}(z)
  =
  \frac{1}{\pi}
  \int_{0}^{\pi} \cos(n\theta - z\sin\theta) \dmeasure\theta
  -
  \frac{\sin n\pi}{\pi}
  \int_{0}^{\infty} e^{-n\theta - z\sinh\theta} \dmeasure\theta,
\end{equation}
when $\Re(z) > 0$. This obviously reduces to the result of
\hardsubsectionref{17}{2}{3} when $n$ is an integer.

Taking the integral of \hardsectionref{17}{2}, viz,
$$
\besJ_{n}(z)
=
\frac{z^{n}}{2^{n+1}\pi i}
\int_{-\infty}^{(0+)}
t^{-n-1}
\exp\theparen{ t - \frac{z^{2}}{4t}  }
\dmeasure t,
$$
%
% 363
%
and supposing that $z$ is positive, we have, on writing
$t = \half u z$,
$$
\besJ_{n}(z)
=
\frac{1}{2\pi i}
\int_{-\infty}^{(0+)}
u^{-n-1}
\exp \thebrace{ \half z \theparen{ u - \frac{1}{u}  }  }
\dmeasure u.
$$

But, if the contour be taken to be that of the figure consisting of
the real axis from $-1$ to $-\infty$ taken twice and the circle
$\absval{u} = 1$, this integral represents an analytic function of $z$
when $\Re(zu)$ is negative as
$\absval{u} \rightarrow \infty$ on the path, \emph{i.e.} when
$\absval{ \arg z } < \half \pi$; and so, by the theory of analytic
continuation, the formula (which has been proved by a direct
transformation for \emph{positive} values of $z$) is true whenever
$\Re(z) > 0$.

Hence
$$
\besJ_{n}(z)
=
\frac{1}{2\pi i}
\thebrace{
  \int_{-\infty}^{-1}
  +
  \int_{C}
  +
  \int_{-1}^{-\infty}
}
u^{-n-1}
\exp\thebrace{ \half z \theparen{ u - \frac{1}{u}  }  }
\dmeasure u,
$$
where $C$ denotes the circle $\absval{u} = 1$, and
$\arg u = -\pi$ on the first path of integration while
$\arg u = +\pi$ on the third path.

TODO:missingfigure

Writing $u = t e^{\mp \pi i}$ in the first and third integrals
respectively (so that in each case $\arg t = 0$), and
$u = e^{i\theta}$ in the second, we have
$$
\besJ_{n}(z)
=
\frac{1}{2\pi}
\int_{-\pi}^{\pi}
e^{-ni\theta + iz\sin\theta}
\dmeasure \theta
+
\thebrace{
  \frac{e^{(n+1)\pi i}}{2\pi i}
  -
  \frac{e^{-(n+1)\pi i}}{2\pi i}
}
\int_{1}^{\infty}
t^{-n-1}
e^{\half z \theparen{-t + \frac{1}{t}}}
\dmeasure t.
$$

Modifying the former of these integrals as in
\hardsubsectionref{17}{2}{3} and writing
$e^{\theta}$ for $t$ in the latter, we have at once
$$
\besJ_{n}(z)
=
\frac{1}{\pi}
\int_{0}^{\pi}
\cos(n\theta - z\sin\theta)
\dmeasure\theta
+
\frac{\sin(n+1)z}{\pi}
\int_{0}^{\infty}
e^{-n\theta - z\sinh\theta}
\dmeasure \theta,
$$
which is the required result, when
$\absval{\arg z} < \half \pi$.

% \begin{smalltext}
When $\absval{ \arg z }$ lies between $\half \pi$ and $\pi$, since
$\besJ_{n}(z) = e^{\pm n\pi i} \besJ_{n}(-z)$, we have
\begin{equation}
  \besJ_{n}(z)
  =
  \frac{e^{\pm n \pi i}}{\pi}
  \thebrace{
    \int_{0}^{\pi} \cos(n\theta + z\sin\theta) \dmeasure\theta
    -
    \sin n\pi \int_{0}^{\infty} e^{-n\theta + z\sinh\theta} \dmeasure\theta
  },
\end{equation}
% \end{smalltext}
the upper or lower sign taken as
$\arg z > \half \pi$ or $< -\half \pi$.

When $n$ is an integer TODO:addref reduces at once to Bessels'
integral, and TODO:addref does so when we make use of the equation
$\besJ_{n}(z) = (-)^{n} \besJ_{-n}(z)$, which is true for integer values of $n$.

%
% 364
%
Equation TODO:addref, as already stated, is due to TODO:addcitation,
and equation TODO:addref was given by TODO:addcitation.

These trigonometric integrals for the Bessel functions may be regarded
as corresponding to Laplace's integrals for the Legendre functions.
For (\hardsubsectionref{17}{1}{1} example TODO:addref)
$\besJ_{m}(z)$ satisfies the confluent form (obtained by making
$n \rightarrow \infty$ of the equation for
$P_{n}^{m}(1-z^{2}/2n^{2})$.

But Laplace's integral for this function is a multiple of
\begin{align*}
\int_{0}^{\pi}
\thebracket{
  1
  -
  \frac{z^{2}}{2n^{2}}
  +
  \thebrace{
    \theparen{
      1 - \frac{z^{2}}{2n^{2}}
    }^{2}
    -
    1
  }^{\half}
  \cos \phi
}^{n} \cos m\phi \dmeasure \phi
\\
=
\int_{0}^{\pi}
\thebrace{
  1
  + \frac{iz}{n} \cos\phi
  + O(n^{-2})
}^{n}
\cos m\phi \dmeasure \phi.
\end{align*}

The limit of the integrand as $n \rightarrow \infty$ is
$e^{iz\cos\phi}\cos m\phi$, and this exhibits the similarity of
Laplace's integral for $P_{n}^{m}(z)$ to the Bessel-\Schlafli\ integral for
$\besJ_{m}(z)$.
\begin{wandwexample}
  From the integral
  $\besJ_{0}(x) = \int_{-\pi}^{\pi} e^{-ix\cos\phi}\dmeasure\phi$, by a
  change of order of integration, shew that, when $n$ is a positive
  integer and $\cos\theta > 0$,
  $$
  P_{n}(\cos\theta)
  =
  \frac{1}{\Gamma(n+1)}
  \int_{0}^{\infty}
  e^{-x\cos\theta}
  \besJ_{0}(x\sin\theta)
  x^{n}
  \dmeasure x.
  $$
  \addexamplecitation{TODO}
\end{wandwexample}
\begin{wandwexample}
  Shew that, with Ferrers' definition of $P_{n}^{m}(\cos\theta)$,
  $$
  P_{n}^{m}(\cos\theta)
  =
  \frac{1}{\Gamma(n-m+1)}
  \int_{0}^{\infty}
  e^{-x\cos\theta}
  \besJ_{m}(x\sin\theta)
  x^{n}
  \dmeasure x
  $$
  when $n$ and $m$ are positive integers and
  $\cos\theta > 0$.
  \addexamplecitation{TODO}
\end{wandwexample}
\Subsection{Bessel functions whose order is half an odd integer.}
We have seen (\hardsectionref{17}{2}) that when the order $n$ of a
Bessel function $\besJ_{n}(z)$ is half an odd integer, the difference of the
roots of the indicial equation at $z=0$ is $2n$, which is an integer.
We now shew that, in such cases, $\besJ_{n}(z)$ is expressible in terms of
elementary functions.

For
$$
\besJ_{\half}(z)
=
\frac{2^{\half} z^{\half}}{\pi^{\half}}
\thebrace{ 1
  - \frac{z^{2}}{2.3}
  + \frac{z^{4}}{2.3.4.5}
  - \cdots
}
= \theparen{ \frac{2}{\pi z}  }^{\half}
\sin z,
$$
and therefore (\hardsubsubsectionref{17}{2}{1}{1}) if $k$ is a
positive integer
$$
\besJ_{k+\half}(z)
=
\frac{(-)^{k} (2z)^{k+\half}}{\pi^{\half}}
\frac{\dd^{k}}{\dd (z^{2})^{k}}
\theparen{ \frac{\sin z}{z}  }
$$
On differentiating out the expression on the right, we obtain the
result that
$$
\besJ_{k+\half}(z) = P_{k} \sin z + Q_{k} \cos z,
$$
where $P_{k}$, $Q_{k}$ are polynomials in $z^{-\half}$.
\begin{wandwexample}
  Shew that
  $$
  \besJ_{-\half}(z) = \theparen{ \frac{2}{\pi z}}^{\half} \cos z
  $$
\end{wandwexample}
%
% 365
%
\begin{wandwexample}
  Prove by induction that if $k$ be an integer and $n = k + \half$,
  then
  \begin{align*}
    \besJ_{n}(z)
    =
    \theparen{\frac{2}{\pi z}}^{\half}
    \left[
      \cos(z - \half n \pi - \frac{1}{4}\pi)
      \thebrace{
        1
        +
        \sum_{r=1}
        \frac{(-)^{r} (4n^{2}-1^{2})(4n^{2}-3^{2})\cdots(4n^{2}-(4r-1)^{2})}{(2r)! 2^{6r} z^{2r}}
      }
    \right.
    \\
    % &
    \left.
      +
      \sin(z - \half n \pi - \frac{1}{4}\pi)
      \sum_{r=1}
      \frac{(-)^{r} (4n^{2}-1^{2})(4n^{2}-3^{2})\cdots(4n^{2}-(4r-3)^{2})}{(2r-1)!
        2^{6r-3} z^{2r-1}}
    \right],
  \end{align*}
  the summations being continued as far as the terms with the
  vanishing factors in the numerators.
\end{wandwexample}
\begin{wandwexample}
  Shew that
  $$
  z^{k+\half} \frac{\dd^{k}}{\dd (z^{2})^{k}} \theparen{\frac{\cos z}{z}}
  $$
  is a solution of Bessel's equation for $\besJ_{k+\half}(z)$.
\end{wandwexample}
\begin{wandwexample}
  Shew that the solution of
  $
  z^{m+\half} \frac{\dd^{2m+1} y}{\dd z^{2m+1}} + y = 0
  $
  is
  $$
  y
  =
  z^{\half m + \frac{1}{4}}
  \sum_{p=0}^{2m}
  c_{p}
  \thebrace{ \besJ_{-m-\half}(2a_{p}z^{\half}) + i\besJ_{m+\half}(2a_{p}z^{\half})},
  $$
  where
  $c_{0},c_{1},\ldots,c_{2m}$ are arbitrary and
  $a_{0},a_{1},\ldots,a_{2m}$ are the roots of
  $$a^{2m+1}=i.
  $$
  \addexamplecitation{Lommel.}
\end{wandwexample}
%TODO:get the Section output working correctly
\Section{Hankel's contour integral for $\besJ_{n}(z)$.}{17}{3}{Hankel's contour integral\footnote{TODO} for $\besJ_{n}(z)$.}
Consider the integral
$$
y = z^{n} \int_{A}^{(1+,-1-)} (t^{2}-1)^{n-\half} \cos(zt) \dmeasure t,
$$
where $A$ is a point on the right of the point $t=1$, and
$$
\arg(t-1) = \arg(t+1) = 0
$$
at $A$; the contour may conveniently be regarded as being in the shape
of a figure of eight.

We shall shew that this integral is a constant multiple of $\besJ_{n}(z)$.
It is easily seen that the integrand returns to its initial value
after $t$ has described the path of integration; for
$(t-1)^{n-\half}$ is multiplied by a factor
$e^{(2n-1)\pi i}$ after the circuit $(1+)$ has been described, and
$(t+1)^{n-\half}$ is multiplied by the factor
$e^{-(2n-1)\pi i}$ after the circuit $(-1-)$ has been described.

Since
$$
\sum_{r=0}^{\infty}
\frac{(-)^{r} (zt)^{2r}}{(2r)!}
(t^{2}-1)^{n-\half}
$$
converges uniformly on the contour, we have (\hardsectionref{4}{7})
$$
y
=
\sum_{r=0}^{\infty}
\frac{(-)^{r} z^{n+2r}}{(2r)!}
\int_{A}^{(1+,-1-)}
t^{2r}
(t^{2}-1)^{n-\half}
\dmeasure t.
$$

To evaluate these integrals, we observe firstly that they are analytic
functions of $n$ for all values of $n$, and secondly that, when
$\Re\theparen{n+\half} > 0$, we may deform the contour into the
circles $\absval{t-1}=\delta$, $\absval{t+1}=\delta$ and the real axis
joining the points $t = \pm (1-\delta)$ taken twice, and then we may
make $\delta \rightarrow 0$; the integrals round the circle tend to
zero and, assigning to $t-1$
%
% 366
%
and $t+1$ their appropriate arguments on the modified path of
integration, we get, if $\arg (1-t^{2}) = 0$ and $t^{2} = u$,
\begin{align*}
  \int_{A}^{(1+,-1-)}
  t^{2r} (t^{2}-1)^{n-\half} \dmeasure t
  \\
  =&
  e^{(n-\half) \pi i}
  \int_{1}^{-1} t^{2r} (1-t^{2})^{n-\half} \dmeasure t
  + e^{-(n-\half) \pi i}
  \int_{-1}^{1} t^{2r} (1-t^{2})^{n-\half} \dmeasure t
  \\
  =&
  -4i
  \sin\theparen{n-\half}
  \pi
  \int_{0}^{1} t^{2r} (1-t^{2})^{n-\half} \dmeasure t
  \\
  =&
  -2i \sin\theparen{n-\half}
  \pi
  \int_{0}^{1} u^{r-\half} (1-u)^{n-\half} \dmeasure u
  \\
  =&
  2i \sin\theparen{ n + \half }
  \pi
  \left.
    \Gamma \theparen{r + \half}
    \Gamma \theparen{ n + \half  }
  \right/
  \Gamma \theparen{ n + r + 1  }.
\end{align*}

Since the initial and final expressions are analytic functions of $n$
for all values of $n$, it follows from \hardsectionref{5}{5} that this
equation, proved when
$$
\Re \theparen{ n + \half } > 0,
$$
is true for all values of $n$.

Accordingly
\begin{align*}
  y
  =&
  \sum_{r=0}^{\infty}
  \frac{(-)^{r} z^{n+2r} 2i \sin(n+\half) \pi \Gamma(r+\half)
    \Gamma(n+\half)}{ (2r)! \Gamma(n+r+1)}
  \\
  =&
  2^{n+1}
  i
  \sin\theparen{n+\half}
  \pi
  \Gamma\theparen{n+\half}
  \Gamma(\half)
  \besJ_{n}(z),
\end{align*}
on reduction.

\emph{Accordingly, when
  $
  \thebrace{\Gamma\theparen{\half-n}}^{-1} \neq 0,
  $
  we have
}
$$
\besJ_{n}(z)
=
\frac{ \Gamma(\half-n) (\half z)^{n} }{ 2\pi i \Gamma(\half)}
$$

\corollary. When $\Re(n+\half) > 0$, we may deform the path of
integration, and obtain the result
\begin{align*}
  \besJ_{n}(z)
  =&
  \frac{ (\half z)^{2} }{ \Gamma(n+\half) \Gamma(\half) }
  \int_{-1}^{1} (1-t^{2})^{n-\half} \cos(zt) \dmeasure t
  \\
  =&
  \frac{2 . (\half z)^{n} }{ \Gamma(n+\half) \Gamma(\half) }
  \int_{0}^{\half \pi} \sin^{2n}\phi \cos(z \cos\phi) \dmeasure\phi.
\end{align*}
\begin{wandwexample}
  Shew that, when $\Re(n+\half) > 0$,
  $$
  \besJ_{n}(z)
  =
  \frac{ (\half z)^{n} }{ \Gamma(n+\half) \Gamma(\half) }
  \int_{0}^{\pi} e^{\pm iz\cos\phi} \sin^{2n}\phi \dmeasure\phi.
  $$
\end{wandwexample}
\begin{wandwexample}
  Obtain the result
  $$
  \besJ_{n}(z)
  =
  \frac{ (\half z)^{n} }{ \Gamma(n+\half) \Gamma(\half) }
  \int_{0}^{\pi} \cos(z\cos\phi) \sin^{2n}\phi \dmeasure\phi,
  $$
  when $\Re(n) > 0$, by expanding in powers of $z$ and
  integrating
  (\hardsectionref{4}{7}) term-by-term.
\end{wandwexample}
%
% 367
%
\begin{wandwexample}
  Shew that when $-\half < n < \half$, $\besJ_{n}(z)$ has an infinite number
  of real zeros. [Let $z = (m+\half)\pi$ where $m$ is zero or a
  positive integer; then by the corollary above
  $$
  \besJ_{n}(m\pi + \half\pi)
  =
  \frac{z^{n}}{2^{n-1} \Gamma(n+\half) \Gamma(\half)}
  \thebrace{ \half u_{0} - u_{1} + u_{2} - \cdots + (-)^{m} u_{m}},
  $$
  where
  \begin{align*}
    u_{r}
    =&
    \absval{
      \int_{\frac{2r-1}{2m+1}}^{\frac{2r+1}{2m+1}}
      (1-t^{2})^{n-\half} \cos \thebrace{(m+\half)\pi t}
      \dmeasure t
    }
    \\
    =&
    \int_{0}^{1/(m+\half)}
    \thebrace{
      1
      -
      \theparen{t + \frac{2r-1}{2m+1}}^{2}
    }^{n-\half}
    \sin\thebrace{ (m+\half)\pi t }
    \dmeasure t,
  \end{align*}
  so, since $n - \half < 0$, $u_{m} > u_{m-1} > u_{m-2} > \ldots$, and
  hence $\besJ_{n}(m\pi + \half\pi)$ has the sign of $(-)^{m}$.
  This method of proof for $n=0$ is due to Bessel.]
\end{wandwexample}
\begin{wandwexample}
  Shew that if $n$ be real, $\besJ_{n}(z)$ has an infinite number of real
  zeros; and find an upper limit to the numerically smallest of them.
  [Use example TODO combined with \hardsubsectionref{17}{2}{2}.]
\end{wandwexample}

\Section{17}{4}{Connexion between Bessel coefficients and Legendre
  functions.}
We shall now establish a result due to Heine\footnote{TODO} which
renders precise the statement of \hardsubsectionref{17}{1}{1} example
TODO, concerning the expression of Bessel coefficients as limiting
forms of hypergeometric functions.

When $\absval{\arg(1\pm z)} < \pi$, $n$ is unrestricted and $m$ is a
positive integer, it follows by differentiating the formula of
\hardsubsectionref{15}{2}{2} that, with Ferrers' definition of
$P_{n}^{m}(z)$,
$$
P_{n}^{m}(z)
=
\frac{\Gamma(n+m+1)}{2^{m} . m! \Gamma(n-m+1)}
(1-z)^{\half m}
(1+z)^{\half m}
F(-n+m,n+1+m; m+1; \half - \half z),
$$
and so, if $\absval{\arg z} < \half\pi$,
$\absval{\arg (1 - \frac{1}{4} z^{2}/n^{2})} < \pi$, we have
$$
P_{n}^{m}\theparen{1 - \frac{z^{2}}{2n^{2}}}
=
\frac{\Gamma(n+m+1) z^{m} n^{-m}}{2^{m} . m! \Gamma(n-m+1)}
\theparen{1 - \frac{z^{2}}{4n^{2}}}^{\half m}
F(-n+m,n+1+m;m+1; \frac{1}{4} z^{2} n^{-2}).
$$

Now make $n \rightarrow +\infty$ ($n$ being positive, but not
necessarily integral), so that, if $\delta = n^{-1}$,
$\delta \rightarrow 0$ continuously through positive values.

Then
$ \frac{\Gamma(n+m+1)n^{-m}}{\Gamma(n-m+1) n^{m}} \rightarrow 1$,
by \hardsectionref{13}{6}, and
$ \theparen{1 - \frac{z^{2}}{4n^{2}}}^{\half m} \rightarrow 1$.

Further, the $(r+1)$th term of the hypergeometric series is
$$
\frac{
  (-)^{r}
  (1-m\delta)
  (1+\delta+m\delta+r\delta)
  \thebrace{ 1 - (m+1)^{2} \delta^{2} }
  \thebrace{ 1 - (m+2)^{2} \delta^{2} }
  \cdots
  \thebrace{ 1 - (m+r)^{2} \delta^{2} }
}{ (m+1)(m+2)\cdots(m+r).r!}
(\half z)^{2r};
$$
this is a continuous function of $\delta$ and the series of which this
is the $(r+1)$th term is easily seen to converge uniformly in a range
of values of $\delta$ including the point $\delta=0$; so, by
\hardsubsectionref{3}{3}{2}, we have
\begin{align*}
  \lim_{n \rightarrow \infty}
  \thebracket{ n^{-m} P_{n}^{m}\theparen{1 - \frac{z^{2}}{2n^{2}}}}
  =& \frac{z^{m}}{2^{m}.m!}
  \sum_{r=0}^{\infty} \frac{(-)^{r} (\half
    z)^{r}}{(m+1)(m+2)\cdots(m+r)r!} \\
  =& \besJ_{m}(z),
\end{align*}
which is the relation required.
\begin{wandwexample}
  Shew that\footnote{TODO}
  $$
  \lim_{n\rightarrow\infty}
  \thebracket{ n^{-m} P_{n}^{m}\theparen{\cos \frac{z}{n}}} = \besJ_{m}(z).
  $$
\end{wandwexample}
%
% 368
%
\begin{wandwexample}
  Shew that Bessel's equation is the confluent form of the equations
  defined by the schemes
  $$
  TODO
  $$
  the confluence being obtained by making $c \rightarrow \infty$.
\end{wandwexample}

\Section{17}{5}{Asymptotic series for $\besJ_{n}(z)$ when $\absval{z}$ is large.}
When have seen (\hardsubsubsectionref{17}{2}{1}{2}) that
$$
\besJ_{n}(z)
=
\frac{z^{-\half}}{2^{2n+\half} e^{\half(n+\half)\pi TODO} \Gamma(n+1)}
M_{0,n}(2iz),
$$
where it is supposed that
$\absval{ \arg z } < \pi$,
$-\half\pi < \arg (2iz) < \frac{3}{2} \pi$.

But for this range of values of $z$
$$
M_{0,n}(2iz)
=
\frac{\Gamma(2n+1)}{\Gamma(\half + n)}
e^{(n+\half)\pi i}
W_{0,n}(2iz)
+
\frac{\Gamma(2n+1)}{\Gamma(\half+n)}
W_{0,n}(-2iz)
$$
by \hardsubsectionref{16}{4}{1} example TODO,
if
$-\frac{3}{2} \pi < \arg (-2iz) < \half \pi$; and so, when
$\absval{\arg z} < \pi$,
$$
\besJ_{n}(z)
=
\frac{1}{(2\pi z)^{\half}}
\thebrace{
  e^{\half (n+\half)\pi i} W_{0,n}(2iz)
  +
  e^{-\half (n+\half)\pi i} W_{0,n}(-2iz).
}
$$

But, for the values of $z$ under consideration, the asymptotic
expansion of $W_{0,n}(\pm 2 i z)$ is
\begin{align*}
  e^{\mp i z}
  \left\{
    1
    \pm \frac{(4n^{2} - 1^{2})}{8iz}
    + \frac{ (4n^{2}-1^{2})(4n^{2}-3^{2})}{2! (8iz)^{2}}
    \pm \cdots
  \right.
  \\
  \left.
    + \frac{ (\pm 1)^{r} \thebrace{4n^{1}-1^{2}} \thebrace{4n^{2}-3^{2}} \cdots
      \thebrace{4n^{2} - (2r-1)^{2}}  }{r! (8iz)^{r}}
    + O(z^{-r})
  \right\},
\end{align*}
and therefore, combining the series, the asymptotic expansion of
$\besJ_{n}(z)$, when $\absval{z}$ is large and $\absval{\arg z} < \pi$, is
\begin{align*}
  TODO
\end{align*}
where $U_{n}(z)$, $-V_{n}(z)$ have been written in place of the series.

The reader will observe that if $n$ is half an odd integer these
series terminate and give the result of \hardsubsectionref{17}{2}{4}
example TODO.

%
% 369
%
Even when $z$ is not very large, the value of $\besJ_{n}(z)$ can be
computed with great accuracy from this formula. Thus, for all
positive values of $z$ greater than $8$, the first three terms of
the asymptotic expansion give the value of $\besJ_{0}(z)$ and $\besJ_{1}(z)$ to
six places of decimals.

This asymptotic expansion was given by Poisson\footnote{TODO} (for
$n=0$) and by Jacobi\footnote{TODO} (for general integral values of
$n$) for real values of $z$.
Complex values of $z$ were considered by Hankel\footnote{TODO} and
several subsequent writers. The method of obtaining the expansion
here given is due to Barnes\footnote{TODO}.

Asymptotic expansions for $\besJ_{n}(z)$ when the order $n$ is large have
been given by Debye TODO and Nicholson TODO.

An approximate formula for $\besJ_{n}(nx)$ when $n$ is large and $0<x<1$,
namely
$$
\frac{ x^{n} \exp\thebrace{n \sqrt{1-x^{2}}}}{ (2\pi n)^{\half} (1-x^{2})^{\frac{1}{4}} \thebrace{1 +
    \sqrt{1 - x^{2}}}^{n}},
$$
was obtained by Carlini in 1817 in a memoir reprinted in Jacobi's
TODO.
The formula was also investigated by Laplace in 1827 in his TODO on
the hypothesis that $x$ is purely imaginary.

A more extended account of researches on Bessel functions of large
order is given by TODO.
\begin{wandwexample}
  By suitably modifying Hankel's contour integral
  (\hardsectionref{17}{3}), shew that, when
  $\absval{\arg z} < \half \pi$ and
  $\Re(n+\half) > 0$,
  \begin{align*}
    \besJ_{n}(z)
    =
    \frac{1}{\Gamma(n+\half) (2\pi z)^{\half}}
    \left[
      e^{i(z - \half n \pi - \frac{1}{4} \pi)}
      \int_{0}^{\infty}
      e^{-u}
      u^{n-\half}
      \theparen{1 + \frac{iu}{2z}}^{n-\half}
      \dmeasure u
    \right.
    \\
    \left.
      +
      e^{-i(z - \half n \pi - \frac{1}{4}\pi)}
      \int_{0}^{\infty}
      e^{-u}
      u^{n-\half}
      \theparen{1 - \frac{iu}{2z}}^{n-\half}
      \dmeasure d
    \right];
  \end{align*}
  and deduce the asymptotic expansion of $\besJ_{n}(z)$ when
  $\absval{z}$ is large and
  $\absval{\arg z} < \half \pi$.

  [Take the contour to be the rectangle whose corners are
  $\pm 1$, $\pm 1 + iN$ the rectangle being indented at $\pm 1$,
  and make $N \rightarrow \infty$; the integrand being
  $(1-t^{2})^{n-\half} e^{izt}$.]
\end{wandwexample}
\begin{wandwexample}
  Shew that, when $\absval{\arg z} < \half \pi$ and
  $\Re(n+\half) > 0$,
  $$
  \besJ_{n}(z)
  =
  \frac{2^{n+1} z^{n}}{\Gamma(n+\half) \pi^{\half}}
  \int_{0}^{\half\pi}
  e^{-2z\cot\phi}
  \cos^{n-\half}\phi
  \cosec^{2n+1}\phi
  \sin\thebrace{z - (n-\half)\phi}
  \dmeasure \phi.
  $$
  [Write $u = 2z\cot\phi$ in the preceding example.]
\end{wandwexample}
\begin{wandwexample}
  Shew that, if $\absval{\arg z} < \half \pi$ and
  $\Re(n+\half) > 0$, then
  $$
  A e^{iz} z^{n}
  \int_{0}^{\infty}
  v^{n-\half}
  (1+iv)^{n-\half}
  e^{-2vz}
  \dmeasure v
  +
  B e^{-iz} z^{n}
  \int_{0}^{\infty}
  e^{n-\half}
  (1-iv)^{n-\half}
  e^{-2vz}
  \dmeasure v
  $$
  is a solution of Bessel's equation.

  Further, determine $A$ and $B$ so that this may represent
  $\besJ_{n}(z)$.
  \addexamplecitation{TODO}
\end{wandwexample}

\Section{17}{6}{The second solution of Bessel's equation when the order
  is an integer.}
We have seen in \hardsectionref{17}{2} that, when the order $n$ of
Bessel's differential equation is not an integer, the general
solution of the equation is
$$
\alpha \besJ_{n}(z) + \beta \besJ_{-n}(z),
$$
where $\alpha$ and $\beta$ are arbitrary constants.

%
% 370
%
When, however, $n$ is an integer, we have seen that
$$
\besJ_{n}(z) = (-)^{n} \besJ_{-n}(z),
$$
and consequently the two solutions $\besJ_{n}(z)$ and $\besJ_{-n}(z)$ are
not really distinct. We therefore require in this case to find
another particular solution of the differential equation, distinct
from $\besJ_{n}(z)$ in order to have the general solution.

We shall now consider the function
$$
\hardY_{n}(z)
=
2 \pi e^{n \pi i}
\frac{\besJ_{n}(z)\cos n\pi - \besJ_{-n}(z)}{\sin 2n\pi},
$$
which is a solution of Bessel's equation when $2n$ is not an
integer.
The introduction of this function $\hardY_{n}(z)$ is due to
Hankel\footnote{TODO}.

When $n$ is an integer, $\hardY_{n}(z)$ is defined by the limiting
form of this equation, namely
\begin{align*}
  \hardY_{n}(z)
  =& \lim_{\eps\rightarrow 0}
  2\pi e^{(n+\eps)\pi i} \frac{ \besJ_{n+\eps}(z) \cos(n\pi+\eps\pi) -
    \besJ_{-n-\eps}(z)}{\sin 2(n+\eps)\pi}
  \\
  =& \lim_{\eps\rightarrow 0}
  \frac{2\pi e^{n\pi i}}{\sin 2\eps\pi}
  \thebrace{ \besJ_{n+\eps}(z) \cdot(-)^{n} - \besJ_{-n-\eps}(z)}
  \\
  =&
  \lim_{\eps\rightarrow 0}
  \eps^{-1}
  \thebrace{ \besJ_{n+\eps}(z) - (-)^{n} \besJ_{-n-\eps}(z)}.
\end{align*}

To express $\hardY_{n}(z)$ in terms of $W_{k,m}$ functions, we have
recourse to the result of \hardsectionref{17}{5}, which gives
\begin{align*}
  \hardY_{n}(z)
  = \lim_{\eps\rightarrow 0}
  \frac{\eps^{-1}}{ (2\pi z)^{\half}}
  \left[
    \thebrace{
      e^{\half(n+\eps+\half)\pi i} W_{0,n+\eps}(2iz)
      + e^{-\half(n+\eps+\half)\pi i} W_{0,n+\eps}(-2iz)
    }
  \right.
  \\
  \left.
    - (-)^{n}
    \thebrace{
      e^{\half(-n-\eps+\half)\pi i} W_{0,n+\eps}(2iz)
      + e^{-\half (-n-\eps+\half)\pi i} W_{0,n+\eps}(-2iz)
    }
  \right],
\end{align*}
remembering that $W_{k,m} = W_{k,-m}$.

Hence, since\footnote{TODO}
$\lim_{\eps\rightarrow 0} W_{0,n+\eps}(2iz) = W_{0,n}(2iz)$, we
have
$$
\hardY_{n}(z)
=
\theparen{ \frac{\pi}{2z} }^{\half}
\thebrace{
  e^{(\half n + \frac{3}{4}) \pi i} W_{0,n}(2iz)
  +
  e^{-(\half n + \frac{3}{4})\pi i} W_{0,n}(-2iz)
}.
$$

This function ($n$ being an integer) is obviously a solution of
Bessel's equation; it is called a \emph{Bessel function of the
  second kind}.

Another function (also called a function of the second kind) was
first used by TODO and by TODO; it is defined by the equation
$$
\besY_{n}(z) = \frac{\besJ_{n}(z) \cos n\pi - \besJ_{-n}(z)}{\sin n\pi}
= \frac{\hardY_{n}(z) \cos n\pi}{\pi e^{n\pi i}},
$$
%
% 371
%
or by the limits of these expressions when $n$ is an integer. This
function which exists for \emph{all} values of $n$ is taken as the
canonical function of the second kind by TODO, and formulae
involving it are generally but not always) simpler than the
corresponding formulae involving Hankel's function.

The asymptotic expansion for $\besY_{n}(z)$, corresponding to that of
\hardsectionref{17}{5} for $\besJ_{n}(z)$, is that, when
$\absval{\arg z} < \pi$ and $n$ is an integer,
$$
\besY_{n}(z)
\sim
\theparen{ \frac{2}{\pi z} }^{\half}
\thebracket{ \sin\theparen{z - \half n \pi - \frac{1}{4} \pi}
  \cdot U_{n}(z)
  + \cos\theparen{z - \half n \pi - \frac{1}{4}} \cdot V_{n}(z)
},
$$
where $U_{n}(z)$ and $V_{n}(z)$ are the asymptotic expansions defined
in \hardsectionref{17}{5}, their leading terms being $1$ and
$(4n^{2}-1)/8z$ respectively.
\begin{wandwexample}
  Prove that
  $$
  \hardY_{n}(z)
  =
  \frac{\dd\besJ_{n}(z)}{dn}
  -
  (-)^{n} \frac{\dd\besJ_{-n}(z)}{dn},
  $$
  where $n$ is made an integer after differentiation.
  \addexamplecitation{Hankel.}
\end{wandwexample}
\begin{wandwexample}
  Shew that if $\hardY_{n}(z)$ be defined by the equation of example
  TODO, it is a solution of Bessel's equation when $n$ is an integer.
\end{wandwexample}
\Subsection{The ascending series for $\hardY_{n}(z)$.}
The series of \hardsectionref{17}{6} is convenient for calculating
$\hardY_{n}(z)$ when $\absval{z}$ is large. To obtain a convenient
series for small values of $\absval{z}$, we observe that, since
the ascending series for $\besJ_{\pm (n+\eps)}(z)$ are uniformly
convergent series of analytic functions\footnote{TODO} of $\eps$,
each term may be expanded in powers of $\eps$ and this double
series may then be arranged in powers of
$\eps$ (TODO).

Accordingly, to obtain $\hardY_{n}(z)$, we have to sum the
coefficients of the first power of $\eps$ in the terms of the
series
$$
\sum_{r=0}^{\infty}
\frac{(-)^{r} (\half z)^{n+2r+\eps}}{r! \Gamma(n+\eps+r+1}
-
(-)^{n}
\sum_{r=0}^{\infty}
\frac{(-)^{r} (\half z)^{-n+2r -\eps}}{r! \Gamma(-n-\eps+r+1}
$$

Now, if $s$ be a positive integer or zero and $t$ a negative
integer, the following expansions in powers of $\eps$ are valid:
\begin{align*}
  \theparen{\half z}^{n+\eps+2r}
  =& \theparen{\half z}^{n+2r}
  \thebrace{1 + \eps\log\theparen{\half z} + \cdots},
  \\
  \frac{1}{\Gamma(s+\eps+1}
  =&
  \frac{1}{\Gamma(s+1)}
  \thebrace{
    1 - \eps \frac{\Gamma'(s+1)}{\Gamma(s+1)} + \cdots
  }
  \\
  =&
  \frac{1}{\Gamma(s+1)}
  \thebrace{
    1 - \eps\theparen{-\gamma + \sum_{m=1}^{s} m^{-1}} + \cdots
  },
  \\
  \frac{1}{\Gamma(t+\eps+1}
  =&
  - \frac{\sin(t+\eps)\pi}{\pi}
  \Gamma(-t-\eps)
  =
  (-)^{t+1} \eps \Gamma(-t) + \cdots,
\end{align*}
where $\gamma$ is Euler's constant (\hardsectionref{12}{1}).

%
% 372
%
Accordingly, picking out the coefficient of $\eps$, we see that
\begin{align*}
  \hardY_{n}(z)
  =&
  \log\theparen{\half z}
  \thebracket{
    \sum_{r=0}^{\infty}
    \frac{(-)^{r} (\half z)^{n+2r}}{r! \Gamma(n+r+1)}
    + (-)^{n} \sum_{r=n}^{\infty} \frac{(-)^{r} (\half z)^{-n+2r}}{r! \Gamma(-n+r+1)}
  }
  \\
  &
  + \sum_{r=0}^{\infty}
  \frac{(-)^{r} (\half z)^{n+2r}}{r! \Gamma(n+r+1)}
  \theparen{\gamma - \sum_{m=1}^{n+r} m^{-1}}
  \\
  &
  + (-)^{n}
  \sum_{r=n}^{\infty}
  \frac{(-)^{r} (\half z)^{-n+2r}}{r! \Gamma(-n+r+1)}
  \theparen{\gamma - \sum_{m=1}^{r-n} m^{-1}}
  \\
  &
  + (-)^{n}
  \sum_{r=0}^{n-1}
  \frac{ (-)^{r} (\half z)^{-n+2r}}{r!}
  (-)^{r-n+1} \Gamma(n-r),
\end{align*}
and so
\begin{align*}
  \hardY_{n}(z)
  =&
  \sum_{r=0}^{\infty}
  \frac{(-)^{r} (\half z)^{n+2r}}{r!(n+r)!}
  \thebrace{
    2 \log\theparen{\half z}
    + 2\gamma
    - \sum_{m=1}^{n+r} m^{-1}
    - \sum_{m=1}^{r} m^{-1}
  }
  \\
  & - \sum_{r=0}^{n-1} \frac{(\half z)^{-n+2r} (n-r-1)!}{r!}.
\end{align*}
When $n$ is an integer, fundamental solutions\footnote{TODO} of
Bessel's equations, regular near $z=0$, are $\besJ_{n}(z)$ and $\besY_{n}(z)$ or
$\hardY_{n}(z)$.

Karl Neumann\footnote{TODO} took as the second solution the function
$\besY^{(n)}(z)$ defined by the equation
$$
\besY^{(n)}(z)
=
\half \hardY_{n}(z) + \besJ_{n}(z) \cdot (\log 2 - \gamma);
$$
but $\besY_{n}(z)$ and $\hardY_{n}(z)$ are more useful for physical
applications.
\begin{wandwexample}
  Shew that the function $\besY_{n}(z)$ satisfies the recurrence formula
  \begin{align*}
    n \besY_{n}(z)
    =& \half z \thebrace{ \besY_{n+1}(z) + \besY_{n-1}(z)},
    \\
    \besY_{n}'(z)
    =& \half \thebrace{ \besY_{n-1}(z) - \besY_{n+1}(z)  }.
  \end{align*}
  Shew also that Hankel's function $\hardY_{n}(z)$ and Neumann's
  function $\besY^{(n)}(z)$ satisfy the same recurrence formulae.

  [These are the same as the recurrence formulae satisfied by $\besJ_{n}(z)$.]
\end{wandwexample}
\begin{wandwexample}
  Shew that, when $\absval{\arg z} < \half \pi$,
  $$
  \pi \besY_{n}(z)
  =
  \int_{0}^{\pi}
  \sin (z \sin\theta - n\theta) \dmeasure \theta
  -
  \int_{0}^{\infty}
  e^{-z \sinh\theta} \thebrace{e^{n\theta} + (-)^{n} e^{-n\theta}} \dmeasure\theta.
  $$
  \addexamplecitation{TODO}
\end{wandwexample}
\begin{wandwexample}
  Shew that
  $$
  \besY^{(0)}(z)
  =
  \besJ_{0}(z) \log z
  + 2 \thebrace{ \besJ_{2}(z) - \half \besJ_{4}(z) + \frac{1}{3} \besJ_{6}(z) - \cdots  }.
  $$
\end{wandwexample}

\Section{17}{7}{Bessel functions with purely imaginary argument.}
The function\footnote{TODO}
$$
\besI_{n}(z)
=
i^{-n}
\besJ_{n}(iz)
=
\sum_{r=0}^{\infty}
\frac{(\half z)^{n+2r}}{r!(n+r)!}
$$
%
% 373
%
is of frequent occurrence in various branches of applied
mathematics; in these applications $z$ is usually positive.

The reader should have no difficulty in obtaining the following
formulae:
\begin{enumerate}
\item $\besI_{n-1}(z) - \besI_{n+1}(z) = \frac{2n}{z} \besI_{n}(z)$.
\item $\frac{\dd}{\dd z} \thebrace{z^{n} \besI_{n}(z)} = z^{n} \besI_{n-1}(z)$.
\item $\frac{\dd}{\dd z} \thebrace{z^{-n} \besI_{n}(z)} = z^{-n} \besI_{n+1}(z)$.
\item $\frac{\dd^{2} \besI_{n}(z)}{\dd z^{2}} + \frac{1}{z} \frac{\dd
    \besI_{n}(z)}{\dd z} -
  \theparen{1 + \frac{n^{2}}{z^{2}}} \besI_{n}(z) = 0$.
\item When $\Re \theparen{n + \half} > 0$,
  $$
  \besI_{n}(z)
  =
  \frac{z^{n}}{2^{n} \Gamma(\half) \Gamma(n+\half)}
  \int_{0}^{\pi}
  \cosh (z\cos\phi) \sin^{2n} \phi \dmeasure \phi.
  $$
\item When $-\frac{3}{2}\pi < \arg z < \half \pi$, the asymptotic
  expansion of $\besI_{n}(z)$ is
  \begin{align*}
    \besI_{n}(z)
    \sim
    \frac{ e^{z}}{ (2\pi z)^{\half}}
    \thebracket{
      1
      +
      \sum_{r=1}^{\infty} (-)^{r}
      \frac{[4n^{2}-1^{2}] [4n^{2} - 3^{2}] \cdots [4n^{2} - (2r-1)^{2}]}{r! 2^{3r} z^{r}}
    }
    \\
    +
    \frac{ e^{-(n+\half) \pi i} e^{-z}}{ (2\pi z)^{\half} }
    \thebracket{
      1
      +
      \sum_{r=1}^{\infty}
      \frac{[4n^{2}-1^{2}] [4n^{2} - 3^{2}] \cdots [4n^{2} - (2r-1)^{2}]}{r! 2^{3r} z^{r}}
    },
  \end{align*}
  the second series being negligible when $\absval{\arg z} < \half
  \pi$. The result is easily seen to be valid over the extended
  range $-\frac{3}{2} \pi < \arg z < \frac{3}{2} \pi$ is we write
  $e^{\pm (n+\half)\pi i}$ for $e^{-(n+\half)\pi i}$, the upper or
  lower sign being taken according as $\arg z$ is positive or negative.
\end{enumerate}

\Subsection{Modified Bessel functions of the second kind.}
When $n$ is a positive integer or zero, $\besI_{n}(z) = \besI_{n}(z)$;
to obtain a second solution of the modified Bessel equation TODO
of \hardsectionref{17}{7}, we define\footnote{TODO} the function
$\besK_{n}(z)$ for all values of $n$ by the equation
$$
\besK_{n}(z)
=
\theparen{\frac{\pi}{2z}}^{\half}
\cos n\pi
W_{0,n}(2z),
$$
so that
$$
\besK_{n}(z) = \half \pi \thebrace{\besI_{-n}(z) - \besI_{n}(z)} \cot n\pi.
$$

%
% 374
%
\emph{Whether $n$ be an integer or not}, this function is a
solution of the modified Bessel equation, and when
$\absval{\arg z} < \frac{3}{2} \pi$ it possesses the asymptotic
expansion
$$
\besK_{n}(z)
\sim
\theparen{ \frac{\pi}{2z} }^{\half}
e^{-z}
\cos (n\pi)
\thebracket{
  1
  +
  \sum_{r=1}^{\infty}
  \frac{ [4n^{2}-1^{2}] [4n^{2}-3^{2}] \cdots [4n^{2} - (2r-1)^{2}]}{r! 2^{3r} z^{r}}
}
$$
for large values of $\absval{z}$.

When $n$ is an integer, $\besK_{n}(z)$ is defined by the equation
$$
\besK_{n}(z)
=
\lim_{\eps\rightarrow 0}
\half \pi
\thebrace{ \besI_{-n-\eps}(z) - \besI_{n+\eps}(z)} \cot \pi\eps,
$$
which gives (\emph{cf}. \hardsubsectionref{17}{6}{1})
\begin{align*}
  \besK_{n}(z)
  =&
  -\sum_{r=0}^{\infty}
  \frac{(\half z)^{n+2r}}{r! (n+r)!}
  \thebrace{
    \log \half z + \gamma
    - \half \sum_{m=1}^{n+r} m^{-1}
    - \half \sum_{m=1}^{r} m^{-1}
  }
  \\
  &
  + \half \sum_{r=0}^{n-1}
  \theparen{\half z}^{-n+2r}
  \frac{(-)^{n-r} (n-r-1)!}{r!}
\end{align*}
as an ascending series.
\begin{wandwexample}
  Shew that $\besK_{n}(z)$ satisfies the same recurrence formula as
  $\besI_{n}(z)$.
\end{wandwexample}

% TODO: get this working correctly
\Section{TMP}{17}{7p9}{Temporary name}
\Section[Neumann's expansion of an analytic function in a series of
  Bessel coefficients]{17}{8}{Neumann's
  expansion\footnote{TODO} of an analytic function in a series
  of Bessel coefficients}
We shall now consider the expansion of an arbitrary function
$f(z)$, analytic in a domain including the origin, in a series
of Bessel coefficients, in the form
$$
f(z) = \alpha_{0} \besJ_{0}(z) + \alpha_{1} \besJ_{1}(z) + \alpha_{2}
\besJ_{2}(z) + \cdots,
$$
where $\alpha_{0}, \alpha_{1}, \alpha_{2},\ldots$ aref independent of
$z$.

%\begin{smallfont}
Assuming the possibility of expansions of this type, let us first
consider the expansion of $1/(t-z)$; let it be
$$
\frac{1}{t-z}
=
O_{0}(t) \besJ_{0}(z)
+ 2 O_{1}(t) \besJ_{1}(z)
+ 2 O_{2}(t) \besJ_{2}(z)
+ \cdots,
$$
where the functions $O_{n}(t)$ are independent of $z$.

We shall now determine conditions which $O_{n}(t)$ must satisfy if the
series on the right is to be a uniformly convergent series of analytic
functions; by these conditions $O_{n}(t)$ will be determined, and it
will then be shewn that, if $O_{n}(t)$ is so determined, then the series
on the right actually converges to the sum $1/(t-z)$ when
$\absval{z} < \absval{t}$.
% \end{smallfont}

Since
$$
\theparen{
  \frac{\partial}{\partial t}
  +
  \frac{\partial}{\partial z}
}
\frac{1}{t-z}
=
0,
$$
we have
$$
O'_{0}(t) \besJ_{0}(z)
+
2 \sum_{n=1}^{\infty}
O'_{n}(t) \besJ_{n}(z)
+
O_{0}(t) \besJ'_{0}(z)
+
2 \sum_{n=1}^{\infty}
O_{n}(t) \besJ'_{n}(z)
\equiv
0,
$$
so that, on replacing $2\besJ'_{n}(z)$ by
$\besJ_{n-1}(z) - \besJ_{n+1}(z)$, we find
$$
\thebrace{ O'_{0}(t) + O_{1}(t) } \besJ_{0}(z)
+
\sum_{n=1}^{\infty}
\thebrace{ 2 O'_{n}(t) + O_{n+1}(t) - O_{n-1}(t)  }
\besJ_{n}(z)
\equiv
0.
$$

%
% 375
%
\emph{Accordingly the successive functions $O_{1}(t), O_{2}(t), O_{3}(t),
  \ldots$ are determined by the recurrence formulae}
$$
O_{1}(t) = -O'_{0}(t),
O_{n+1}(t) = O_{n-1}(t) - 2O'_{n}(t),
$$
and, putting $z=0$ in the original expansion, we see that $O_{0}(t)$ is
to be defined by the equation
$$
O_{0}(t) = 1/t.
$$

These formulae shew without difficulty that $O_{n}(t)$ is a polynomial
of degree $n$ in $1/t$.

We shall next prove by induction that $O_{n}(t)$, so defined, is equal
to
$$
\half
\int_{0}^{\infty}
e^{-tu}
\thebracket{
  \thebrace{u + \sqrt{u^{2}+1}}^{n}
  +
  \thebrace{u - \sqrt{u^{2}+1}}^{n}
}
\dmeasure u
$$
when $\Re(t) > 0$.
For the expression is obviously equal to $O_{0}(t)$ and $O_{1}(t)$ when
$n$ is equal to $0$ or $1$ respectively; and
\begin{align*}
  &
  \half
  \int_{0}^{\infty}
  e^{-tu}
  \thebrace{u \pm \sqrt{u^{2}+1}}^{n-1}
  \dmeasure u
  -
  \frac{\dd}{\dd t}
  \int_{0}^{\infty}
  e^{-tu}
  \thebrace{u \pm \sqrt{u^{2}+1}}^{n}
  \dmeasure u
  \\
  =
  &
  \half
  \int_{0}^{\infty}
  e^{-tu}
  \thebrace{u \pm \sqrt{u^{2}+1}}^{n-1}
  \thebrace{1 + 2u^{2} \pm 2u \sqrt{u^{2}+1}}
  \dmeasure u
  \\
  =
  &
  \half
  \int_{0}^{\infty}
  e^{-tu}
  \thebrace{u \pm \sqrt{u^{2}+1}}^{n+1}
  \dmeasure u,
\end{align*}
whence the induction is obvious.

Writing $u = \sinh \theta$, we see that, according as $n$ is even or
odd\footnote{TODO},
\begin{align*}
  TODO
\end{align*}
and hence, when $\Re(t) > 0$, we have on integration,
$$
O_{n}(t)
=
\frac{2^{n-1} n!}{t^{n+1}}
\thebrace{
  1
  + \frac{t^{2}}{2 (2n-2)}
  + \frac{t^{4}}{ 2.4 (2n-2)(2n-4)}
  + \cdots
},
$$
the series terminating with the term in $t^{n}$ or $t^{n-1}$; now,
whether $\Re(t)$ be positive or not,
$O_{n}(t)$ is defined as a polynomiral in $1/t$; and so the expansion
obtained for $O_{n}(t)$ is the value of $O_{n}(t)$ for \emph{all} values
of $t$.
\begin{wandwexample}
  Shew that, for all values of $t$,
  $$
  O_{n}(t)
  =
  \frac{1}{2 t^{n+1}}
  \int_{0}^{\infty}
  e^{-x}
  \thebracket{
    \thebrace{x + \sqrt{x^{2} + t^{2}}}^{n}
    +
    \thebrace{x - \sqrt{x^{2} + t^{2}}}^{n}
  }
  \dmeasure x,
  $$
  and verify that the expression on the right satisfies the recurrence
  formulae for $O_{n}(t)$.
\end{wandwexample}
\Subsection{Proof of Neumann's expansion.}
The method of \hardsectionref{17}{8} merely determined the
coefficients in Neumann's expansion of $1/(t-z)$, on the hypothesis
that the expansion existed and that the rearrangements were
legitimate.

To obtain a proof of the validity of the expansion, we observe that
$$
\besJ_{n}(z)
=
\frac{(\half z)^{n}}{n!}
\thebrace{1 + \theta_{n}},
\quad
O_{n}(t)
=
\frac{2^{n-1} n!}{t^{n+1}}
\thebrace{ 1 + \phi_{n}},
$$
%
% 376
%
where $\theta_{n} \rightarrow 0$, $\phi_{n} \rightarrow 0$ as $n
\rightarrow \infty$, when $z$ and $t$ are fixed. Hence the series
$$
O_{0}(t) \besJ_{n}(z)
+
2
\sum_{n=1}^{\infty}
O_{n}(t) \besJ_{n}(z)
\equiv
F(z,t)
$$
is comparable with the geometrical progression whose general term is
$z^{n} / t^{n+1}$, and this progression is absolutely convergenet when
$\absval{z} < \absval{t}$, and so the expansion for $F(z,t)$ is
absolutely convergent (\hardsubsectionref{2}{3}{4}) in the same
circumstances.

Again if $\absval{z} \leq r$, $\absval{t} \geq R$, where $r < R$, the
series is comparable with the geometrical progression whose general
term is $r^{n} / R^{n+1}$, and so the expansion for $F(z,t)$ converges
uniformly throughout the domains $\absval{z} \leq r$ and $\absval{t}
\geq R$ by \hardsubsectionref{3}{3}{4}. Hence, by
\hardsectionref{5}{3}, term-by-term differentiations are permissible,
and so
\begin{align*}
  \theparen{ \frac{\partial}{\partial t} + \frac{\partial}{\partial z}
  }
  F(z,t)
  =&
  O_{0}'(t) \besJ_{0}(z)
  + 2 \sum_{n=1}^{\infty} O_{n}'(t) \besJ_{n}(z)
  \\
  &
  + O_{0}(t) \besJ'_{0}(z)
  + 2 \sum_{n=1}^{\infty}
  O_{n}(t) \besJ'_{n}(z)
  \\
  =&
  \thebrace{ O_{0}'(t) + O_{1}(t)} \besJ_{0}(z)
  + \sum_{n=1}^{\infty} \thebrace{
    2 O_{n}'(t) + O_{n+1}(t) - O_{n-1}(t)
  }
  \besJ_{n}(z)
  \\
  =&
  0,
\end{align*}
by the recurrence formulae.

Since
$$
\theparen{ \frac{\partial}{\partial t} + \frac{\partial}{\partial z}
}
F(z,t)
=
0,
$$
it follows that $F(z,t)$ is expressible as a function of $t-z$; and
since
$$
F(0,t) = O_{0}(t) = 1/t,
$$
it is clear that $F(z,t) = 1/(t-z)$.

It is therefore proved that
$$
\frac{1}{t-z}
=
O_{0}(t) \besJ_{0}(z)
+ 2 \sum_{n=1}^{\infty}
O_{n}(t) \besJ_{n}(z),
$$
provided that $\absval{z} < \absval{t}$.

Hence, if $f(z)$ be analytic when $\absval{z} \leq r$, we have, when
$\absval{z} < r$,
\begin{align*}
  f(z)
  =&
  \frac{1}{2\pi i} \int \frac{f(t)}{t-z} \dmeasure t
  \\
  =&
  \frac{1}{2\pi i}
  \int f(t)
  \thebrace{
    O_{0}(t) \besJ_{0}(z) + 2 \sum_{n=1}^{\infty} O_{n}(t) \besJ_{n}(z)
  }
  \dmeasure t
  \\
  =&
  \besJ_{0}(z) f(0)
  +
  \sum_{n=1}^{\infty} \frac{\besJ_{n}(z)}{\pi i}
  \int O_{n}(t)f(t) \dmeasure t,
\end{align*}
by \hardsectionref{4}{7}, the paths of integration being the circle
$\absval{t}=r$; and this establishes the validity of Neumann's
expansion when $\absval{z} < r$ and $f(z)$ is analytic when
$\absval{z}\leq r$.

%
% 377
%
\begin{wandwexample}
  Shew that
  \begin{align*}
    \cos z =& \besJ_{0}(z) - 2\besJ_{2}(z) + 2\besJ_{4}(z) - \cdots, \\
    \sin z =& 2\besJ_{1}(z) - 2\besJ_{3}(z) + 2\besJ_{5}(z) - \cdots.
  \end{align*}
  \addexamplecitation{K. Neumann.}
\end{wandwexample}
\begin{wandwexample}
  Shew that
  $$
  (\half z)^{n}
  =
  \sum_{r=0}^{\infty}
  \frac{(n+2r)(n+r-1)!}{r!}
  \besJ_{n+2r}(z).
  $$
  \addexamplecitation{K. Neumann.}
\end{wandwexample}
\begin{wandwexample}
  Shew that, when $\absval{z}<\absval{t}$,
  \begin{align*}
    O_{0}(t) \besJ_{0}(z)
    +
    2 \sum_{n=1}^{\infty} O_{n}(t) \besJ_{n}(z)
    =&
    \sum_{n=-\infty}^{\infty}
    \besJ_{n}(z)
    \int_{0}^{\infty}
    t^{-n-1}
    e^{-x}
    \thebrace{
      x + \sqrt{x^{2} + t^{2}}
    }^{n}
    \dmeasure x
    \\
    =&
    \int_{0}^{\infty}
    \frac{e^{-x}}{t}
    \sum_{n=-\infty}^{\infty}
    \besJ_{n}(z)
    \thebrace{
      x + \sqrt{x^{2} + t^{2}}
    }^{n}
    \dmeasure x
    \\
    =&
    \frac{1}{t}
    \int_{0}^{\infty} \exp \theparen{ \frac{zx}{t} - x } \dmeasure x
    \\
    =&
    \frac{1}{t-z}.
  \end{align*}
  \addexamplecitation{Kapteyn.}
\end{wandwexample}

\Subsection{\Schlomilch's expansion of an arbitrary function in a
  series of Bessel coefficients of order zero.}
\Schlomilch\footnote{TODO} has given an expansion of quite a different
character from that of Neumann. His result may be stated thus:

\emph{
  Any function $f(x)$, which has a continuous differential coefficient
  with limited total fluctuation for all values of $x$ in the closed
  range $(0,\pi)$, may be expanded in the series
  $$
  f(x)
  =
  a_{0} + a_{1} \besJ_{0}(x) + a_{2} \besJ_{0}(2x) + a_{3} \besJ_{0}(3x) + \cdots,
  $$
  valid in this range; where
  \begin{align*}
    a_{0} =& f(0)
    +
    \frac{1}{\pi}
    \int_{0}^{\pi} u
    \int_{0}^{\half\pi} f'(u\sin\theta)
    \dmeasure\theta \dmeasure u,
    \\
    a_{n} =&
    \frac{2}{\pi}
    \int_{0}^{\pi} u\cos nu
    \int_{0}^{\half\pi} f'(u\sin\theta)
    \dmeasure\theta \dmeasure u
    \qquad (n>0).
  \end{align*}
}
\Schlomilch's proof is substantially as follows:

Let $F(x)$ be the continuous solution of the integral equation
$$
f(x) = \frac{2}{\pi} \int_{0}^{\half \pi} F(x \sin\phi) \dmeasure \phi.
$$
Then (\hardsubsectionref{11}{8}{1})
$$
F(x) = f(0) + x \int_{0}^{\half\pi} f'(x\sin\theta) \dmeasure\theta.
$$

In order to obtain \Schlomilch's expansion, it is merely necessary to
apply Fourier's theorem to the function $F(x\sin\phi)$. We thus have
\begin{align*}
  f(x)
  =&
  \frac{2}{\pi}
  \int_{0}^{\half\pi}
  \dmeasure\phi
  \thebrace{
    \frac{1}{\pi}
    \int_{0}^{\pi} F(u) \dmeasure u
    +
    \frac{2}{\pi}
    \sum_{n=1}^{\infty}
    \int_{0}^{\pi} \cos nu \cos(nx\sin\phi) F(u) \dmeasure u
  }
  \\
  =&
  \frac{1}{\pi}
  \int_{0}^{\pi}
  F(u) \dmeasure u
  +
  \frac{2}{\pi}
  \sum_{n=1}^{\infty}
  \int_{0}^{\pi} \cos nu F(u) \besJ_{0}(nx) \dmeasure u,
\end{align*}
the interchange of summation and integration being permissible by
TODO.
%
% 378
%

In this equation, replace $F(u)$ by its value in terms of $f(u)$. Thus
we have
\begin{align*}
  f(x)
  =
  \frac{1}{\pi}
  \int_{0}^{\pi}
  \thebrace{
    f(0) + u \int_{0}^{\half\pi} f'(u\sin\theta) \dmeasure\theta
  }
  \dmeasure u
  \\
  +
  \frac{2}{\pi}
  \sum_{n=1}^{\infty}
  \besJ_{0}(nx)
  \int_{0}^{\pi}
  \cos nu
  \thebrace{
    f(0) + u \int_{0}^{\half\pi} f'(u\sin\theta) \dmeasure\theta
  }
  \dmeasure u,
\end{align*}
which gives \Schlomilch's expansion.
\begin{wandwexample}
  Shew that, if $0 \leq x \leq \pi$, the expression
  $$
  \frac{\pi^{2}}{4}
  -
  2
  \thebrace{
    \besJ_{0}(x)
    + \frac{1}{9} \besJ_{0}(3x)
    + \frac{1}{25} \besJ_{0}(5x)
    + \cdots
  }
  $$
  is equal to $x$; but that, if $\pi \leq x \leq 2\pi$, its value is
  $$
  x
  +
  2\pi \arccos (\pi x^{-1})
  -
  2 \sqrt{x^{2} - \pi^{2}},
  $$
  where $\arccos(\pi x^{-1})$ is taken between $0$ and
  $\frac{\pi}{3}$.

  Find the value of the expression when $x$ lies between $2\pi$ and
  $3\pi$.
  \addexamplecitation{Math. Trip. 1895.}
\end{wandwexample}

\Section{17}{9}{Tabulation of Bessel functions.}
Hansen used the asymptotic expansion (\hardsectionref{17}{5}) to
calculate tables of $\besJ_{n}(x)$ which are given in TODOl
Meissel tabulated $\besJ_{0}(x)$ and $\besJ_{1}(x)$ to 12 places of
decimals from $x=0$ to $x = 15.5$ (TODO), while the TODO gives tables
by which $\besJ_{n}(x)$ and $\besY_{n}(x)$ may be calculated when $x >
10$.

Tables of $\besJ_{\frac{1}{3}}(x)$, $\besJ_{\frac{2}{3}}(x)$,
$\besJ_{-\frac{1}{3}}(x)$, $\besJ_{TODO}(x)$ are given by TODO.

Tables of the second solution of Bessel's equation have been given by
the following writers: TODO.

The functions $\besI_{n}(x)$ have been tabulated in TODO; by TODO; and
by TODO.

Tables of $\besJ_{n}(x \sqrt{i})$, a function employed in the theory of
alternating currents in wires, have been given in the TODO; by TODO;
by TODO; and by TODO.

Formulae for computing the zeros of $\besJ_{0}(z)$ were given by TODO
and the 40 smallest zeros were tabulated by TODO. The roots of an
equation involving Bessel functions were computed by TODO.

A number of tables with Bessel functions are given in TODO, and also
by TODO.

%
% 379
%
\miscexamples
\begin{enumerate}
\item Shew that
  \begin{align*}
    \cos(z \sin\theta)
    =& \besJ_{0}(z) + 2 \besJ_{2}(z) \cos 2\theta + 2 \besJ_{4}(z)
    \cos 4\theta + \cdots,
    \\
    \sin(z\sin\theta)
    =& 2\besJ_{1}(z) \sin\theta + 2 \besJ_{3}(z)\sin 3\theta + 2
    \besJ_{5}(z) \sin 5\theta + \cdots
    \addexamplecitation{K. Neumann.}
  \end{align*}
\item
  By expanding each side of the equations of example TODO in powers of
  $\sin\theta$, express $z^{n}$ as a series of Bessel coefficients.
\item
  By multiplying the expansions for
  $\exp{ \thebrace{\half z\theparen{t - \frac{1}{t}}} }$
  and
  $\exp{ \thebrace{-\half z\theparen{t - \frac{1}{t}}} }$
  and considering the terms independent of $t$, shew that
  $$
  \thebrace{ \besJ_{0}(z) }^{2}
  + 2 \thebrace{ \besJ_{1}(z) }^{2}
  + 2 \thebrace{ \besJ_{2}(z) }^{2}
  + 2 \thebrace{ \besJ_{3}(z) }^{2}
  +
  \cdots
  =
  1.
  $$
  Deduce that, for the Bessel coefficients,
  $$
  \absval{\besJ_{0}(z)} \leq 1,
  \quad
  \absval{\besJ_{n}(z)} \leq 2^{-\half},
  \quad
  (n \geq 1)
  $$
  when $z$ is real.
\item
  If
  $$
  J_{m}^{\ k}(z)
  =
  \frac{1}{\pi}
  \int_{0}^{\pi}
  2^{k}
  \cos^{k} u
  \cos (m u - z\sin u)
  \dmeasure u
  $$
  (this function reduces to a Bessel coefficient when $k$ is zero and
  $m$ an integer), shew that
  TODO
\item
  If $v$ and $M$ are connected by the equations
  $$
  M = E - e \sin E,
  \quad
  \cos v = \frac{\cos E - e}{1 - e\cos E},
  \quad
  \text{where } \absval{e} < 1,
  $$
  shew that
  $$
  v
  =
  M
  +
  2 (1-e^{2})^{\half}
  \sum_{m=1}^{\infty}
  \sum_{k=0}^{\infty}
  (\half e)^{k}
  J_{m}^{\ k}(me)
  \frac{1}{m}
  \sin mM,
  $$
  where $J_{m}^{\ k]}(z)$ is defined as in example TODO.
  \addexamplecitation{Bourget.}
%
% 380
%
\item
  Prove that, if $m$ and $n$ are integer,
  $$
  P_{n}^{\ m}(\cos\theta)
  =
  \frac{c_{n}^{\ m}}{r^{n}}
  \besJ_{m}
  \theparen{ \sqrt{x^{2}+y^{2}} \frac{\partial}{\partial z}}
  z^{n},
  $$
  where $z = r\cos\theta$, $x^{2} + y^{2} = r^{2} \sin^{2}\theta$, and
  $c_{n}^{\ m}$ is independent of z.
  \addexamplecitation{Math. Trip. 1893}
\item
  Shew that the solution of the differential equation
  $$
  \frac{\dd^{2} y}{\dd z^{2}}
  -
  \frac{\phi'}{\phi} \frac{\dd y}{\dd z}
  +
  \thebrace{
    \frac{1}{4}
    \theparen{\frac{\phi'}{\phi}}^{2}
    - \frac{1}{2} \frac{\dd}{\dd z} \theparen{\frac{\phi'}{\phi}}
    - \frac{1}{4} \theparen{\frac{\psi''}{\psi}}^{2}
    + \frac{1}{2} \frac{\dd}{\dd z} \theparen{\frac{\psi''}{\psi'}}
    +
    \theparen{\psi^{2} - \frac{4\nu^{2}-1}{4}}
    \theparen{\frac{\psi'}{\psi}}^{2}
  }
  y
  =
  0,
  $$
  where $\phi$ and $\psi$ are arbitrary functions of $z$, is
  $$
  y = \theparen{\frac{\phi\psi}{\psi'}}^{\half}
  \thebrace{A \besJ_{\nu}(\psi) + B\besJ_{-\nu}(\psi)}.
  $$
\item
  Shew that
  $$
  \besJ_{1}(x)
  +
  \besJ_{3}(x)
  +
  \besJ_{5}(x)
  +
  \cdots
  =
  \half
  \thebracket{
    \besJ_{0}(x)
    +
    \int_{0}^{x}
    \thebrace{
      \besJ_{0}(t) + \besJ_{1}(t)
    }
    \dmeasure t
    -
    1
    }.
  $$
  \addexamplecitation{Trinity, 1908.}
\item
  Shew that
  $$
  \besJ_{\mu}(z) \besJ_{\nu}(z)
  =
  \sum_{n=0}^{\infty}
  \frac{(-)^{n} \Gamma(\mu+\nu+2n+1) (\half z)^{\mu+\nu+2n}}{
    n! \Gamma(\mu+n+1) \Gamma(\nu+n+1) \Gamma(\mu+\nu+n+1)}
  $$
  for all values of $\mu$ and $\nu$.
  TODO:addattribution
\item
  Shew that, if $n$ is a positive integer and $m+2n+1$ is positive,
  $$
  (m-1)
  \int_{0}^{x}
  x^{m} \besJ_{n+1}(x) \besJ_{n-1}(x)
  \dmeasure x
  =
  x^{m+1}
  \thebrace{
    \besJ_{n+1}(x)\besJ_{n-1}(x) - \besJ_{n}^{2}(x)
  }
  +
  (m+1)
  \int_{0}^{x} x^{m} \besJ_{n}^{2}(x)
  \dmeasure x.
  $$
  \addexamplecitation{Math. Trip. 1899.}
\item
  Shew that
  $$
  \besJ_{3}(z)
  +
  3 \frac{\dd \besJ_{0}(z)}{\dd z}
  +
  4 \frac{\dd^{3} \besJ_{0}(z)}{\dd z^{3}}
  =
  0.
  $$
\item
  Shew that
  $$
  \frac{\besJ_{n+1}(z)}{\besJ_{n}(z)}
  =
  \frac{ \half z }{n+1-}
  \frac{ (\half z)^{2} }{n+2-}
  \frac{ (\half z)^{2} }{n+3-}
  \cdots.
  $$
\item
  Shew that
  $$
  \besJ_{-n}(z) \besJ_{n-1}(z)
  +
  \besJ_{-n-1}(z) \besJ_{n}(z)
  =
  \frac{2 \sin n\pi}{\pi z}
  $$
  \addexamplecitation{Lommel.}
\item
  If $\frac{\besJ_{n+1}(z)}{z \besJ_{n}(z)}$ be denoted by $Q_{n}(z)$,
  shew that
  $$
  \frac{ \dd Q_{n}(z) }{\dd z}
  =
  \frac{1}{z}
  -
  \frac{2(n+1)}{z}
  Q_{n}(z)
  +
  z \thebrace{ Q_{n}(z) }^{2}
  $$
\item
  Shew that, if
  $R^{2} = r^{2} + r_{1}^{2} - 2rr_{1}\cos\theta$
  and
  $r_{1} > r > 0$,
  \begin{align*}
    \besJ_{0}(R)
    =&
    \besJ_{0}(r) \besJ_{0}(r_{1})
    +
    2 \sum_{n=1}^{\infty} \besJ_{n}(r) \besJ_{n}(r_{1}) \cos n\theta,
    \\
    \besY_{0}(R)
    =&
    \besJ_{0}(r) \besY_{0}(r_{1})
    +
    2 \sum_{n=1}^{\infty} \besJ_{n}(r) \besY_{n}(r_{1}) \cos n\theta.
  \end{align*}
  \addexamplecitation{K. Neumann.}
\item
  Shew that, if $\Re(n+\half) > 0$,
  $$
  \int_{0}^{\half\pi}
  \besJ_{2^{n}}(2z\cos\theta) \dmeasure \theta
  =
  \half\pi \thebrace{ \besJ_{n}(z) }^{2}.
  \addexamplecitation{K. Neumann.}
  $$
%
% 381
%
\item
  Shew how to express $z^{2n} \besJ_{2n}(z)$ in the form
  $A \besJ_{2}(z) + B \besJ_{0}(z)$, where $A$, $B$ are polynomials in
  $z$; and prove that
  $$
  \besJ_{4}(\sqrt{6}) + 3 \besJ_{0}(\sqrt{6}) = 0,
  \quad
  3 \besJ_{6}(\sqrt{30}) + 5 \besJ_{2}(\sqrt{30}) = 0.
  \addexamplecitation{Math. Trip. 1896.}
  $$
\item
  Shew that, if $\alpha \neq \beta$ and $n > -1$,
  \begin{align*}
    (\alpha^{2}-\beta^{2})
    \int_{0}^{x} x \besJ_{n}(\alpha x) \besJ_{n}(\beta x) \dmeasure x
    =&
    x \thebrace{
      \besJ_{n}(\alpha x) \frac{\dd}{\dd x} \besJ_{n}(\beta x)
      - \besJ_{n}(\beta x) \frac{\dd}{\dd x} \besJ_{n}(\alpha x)
    },
    \\
    2 \alpha^{2}
    \int_{0}^{x}
    x \thebrace{ \besJ_{n}(\alpha x) }^{2} \dmeasure x
    =&
    (\alpha^{2} x^{2} - n^{2})
    \thebrace{ \besJ_{n}(\alpha x) }^{2}
    +
    \thebrace{
      x \frac{\dd}{\dd x} \besJ_{n}(\alpha x)
    }.
  \end{align*}
\item
  Prove that, if $n > -1$, and $\besJ_{n}(\alpha) = \besJ_{n}(\beta) =
  0$ while $\alpha \neq \beta$,
  $$
  \int_{0}^{1} x \besJ_{n}(\alpha x) \besJ_{n}(\beta x) \dmeasure = 0,
  \quad
  \textrm{and}
  \quad
  \int_{0}^{1} x \thebrace{\besJ_{n}(\alpha x}^{2} \dmeasure x
  =
  \half \thebrace{ \besJ_{n+1}(\alpha)}^{2}.
  $$
  Hence prove that, when $n > -1$, the roots of $\besJ_{n}(x)=0$, other
  than zero, are all real and unequal.

  [If $\alpha$ could be complex, take $\beta$ to be the conjugate
  complex.]
  \addexamplecitation{TODO}
\item
  Let $x^{\half} f(x)$ have an absolutely convergent integral in the
  range $0 \leq x \leq 1$; let $H$ be a real constant and let $n \geq
  0$. Then, if $k_{1}, k_{2}, \ldots$ denote the positive roots of
  the equation
  $$
  k^{-n}
  \thebrace{ k \besJ'_{n}(k) + H \besJ_{n}(k)}
  =
  0,
  $$
  shew that, at any point $x$ for which $0 < x < 1$ and $f(x)$
  satisfies one of the conditions of \hardsubsectionref{9}{4}{3},
  $f(x)$ can be expanded in the form
  $$
  f(x) = \sum_{r=1}^{\infty} A_{r} \besJ_{n}(k_{r} x),
  $$
  where
  $$
  A_{r}
  =
  \thebracket{ \int_{0}^{1} x \thebrace{\besJ_{n}(k_{r}x)}^{2} \dmeasure x}^{-1}
  \int_{0}^{1} x f(x) \besJ_{n}(k_{r}x) \dmeasure x.
  $$

  In the special case when $H = -n$, $k_{1}$ is to be taken to be
  zero, the equation determining $k_{1}, k_{2}, \ldots$ being
  $\besJ_{n+1}(k) = 0$, and the first term of the expansion is $A_{0}
  x^{n}$ where
  $$
  A_{0} = (2n+2) \int_{0}^{1} x^{n+1} f(x) \dmeasure x.
  $$

  Discuss, in particular, the case when $H$ is infinite, so that
  $\besJ_{n}(k)=0$, shewing that
  $$
  A_{r}
  =
  2 \thebrace{\besJ_{n+1}(k_{r})}^{-2}
  \int_{0}^{1}
  x f(x) \besJ_{n}(k_{r}x)
  \dmeasure x.
  $$

  [This result is due to TODO; see also TODO. The formal expansion was
  fiven with $H$ infinite (when $n=0$) by Fourier and (for general
  values of $n$ by Lommel; proofs were given by Hankel and \Schlafli.
  The formula when $H = -n$ was given incorrectly by TODO, the term
  $A_{0}x^{n}$ being printed as $A_{0}$, and this error was not
  corrected by Nielsen. See TODO. The expansion is usually called the
  \emph{Fourier-Bessel expansion}.]
\item
  Prove that, if the expansion
  $$
  a^{2} - x^{2}
  =
  A_{1} \besJ_{0}(\lambda_{1} x)
  + A_{2} \besJ_{0}(\lambda_{2} x)
  + \cdots
  $$
  exists as a uniformly convergent series when
  $ -a \leq x \leq a$, where $\lambda_{1}, \lambda_{2}, \ldots$ are
  the positive roots of $\besJ_{0}(\lambda a)=0$, then
  $$
  A_{n} = 8 \thebrace{ a \lambda_{n}^{3} \besJ_{1}(\lambda_{n} a)}^{-1}
  $$
  \addexamplecitation{Clare, 1900.}
%
% 382
%
\item
  If $k_{1}, k_{2}, \ldots$ are the positive roots of
  $\besJ_{n}(k\alpha) = 0$, and if
  $$
  x^{n+2} = \sum_{r=1}^{\infty} A_{r} \besJ(k_{r}x),
  $$
  this series converging uniformly when $0 \leq x \leq \alpha$, then
  $$
  A_{r}
  =
  \frac{2 \alpha^{n-1}}{k_{r}^{2}}
  \left.
    (4n + 4 - \alpha^{2}k_{r}^{2})
  \right|
  \frac{\dd \besJ_{n}(k_{r}\alpha)}{\dd \alpha}.
  $$
  \addexamplecitation{Math. Trip. 1906.}
\item
  Shew that
  $$
  \besJ_{n}(x)
  =
  \frac{x^{n-m}}{2^{n-m-1}\Gamma(n-m)}
  \int_{0}^{\half \pi}
  \besJ_{m}(x\sin\theta)
  \cos^{2n-2m-1}\theta
  \sin^{m+1}\theta
  \dmeasure \theta
  $$
  when $n>m>-1$.
  \addexamplecitation{TODO}
\item
  Shew that, if $\sigma>0$,
  $$
  \int_{0}^{\infty}
  \cos(t^{3} - \sigma t) \dmeasure t
  =
  \frac{\pi \sigma^{\half}}{3\sqrt{e}}
  \thebrace{
    \besJ_{\frac{1}{3}} \theparen{\frac{2 \sigma^{3/2}}{3^{3/2}}}
    +
    \besJ_{-\frac{1}{3}} \theparen{\frac{2 \sigma^{3/2}}{3^{3/2}}}.
  }
  $$
  \addexamplecitation{TODO}
\item
  If $m$ be a positive integer and $u>0$, deduce from Bessel's
  integral formula that
  $$
  \int_{0}^{\infty}
  e^{-x \sinh u} \besJ_{m}(x) \dmeasure x
  =
  e^{-m u} \sech u.
  $$
  \addexamplecitation{Math. Trip. 1904.}
\item
  Prove that, when $x>0$,
  $$
  \besJ_{0}(x) = \frac{2}{\pi} \int_{0}^{\infty} \sin(x \cosh t)
  \dmeasure t,
  \quad
  \besY_{0}(x) = -\frac{2}{\pi}
  \int_{0}^{\infty} \cos(x \cosh t)
  \dmeasure t.
  $$
  [Take the contour of \hardsectionref{17}{1} to be the imaginary axis
  indented at the origin and a semicircle on the left of this line.]
  \addexamplecitation{TODO}
\item
  Shew that
  $$
  \int_{0}^{\infty} x^{-1} \besJ_{0}(xt) \sin x \dmeasure x
  =
  \begin{cases}
    \half\pi & 0 < t < 1
    \\
    \arccosec t & t > 1
  \end{cases}
  $$
  and that
  $$
  \int_{0}^{\infty} x^{-1} \besJ_{1}(xt) \sin x \dmeasure x
  =
  \begin{cases}
    t^{-1} \thebrace{ 1 - (1-t^{2})^{\half}  } & 0 < t < 1
    \\
    t^{-1} & t > 1
  \end{cases}.
  $$
  \addexamplecitation{TODO}
\item
  Shew that
  $$
  u = \int_{0}^{\pi} e^{nr\cos\theta} \thebrace{A + B \log(r\sin^{2}\theta} \dmeasure \theta
  $$
  is a solution of
  $$
  \frac{\dd u}{\dd r^{2}} + \frac{1}{r} \frac{\dd u}{\dd r} - n^{2} u
  = 0.
  $$
  \addexamplecitation{TODO}
\item
  Prove that no relation of the form
  $$
  \sum_{s=0}^{k} N_{s} \besJ_{n+s}(x) = 0
  $$
  can exist for rational values of $N_{s}$, $n$, and $x$ except
  relations which are satisfied when the Bessel functions are replaced
  by arbitrary solutions of the recurrence formula of TODO.
  \addexamplecitation{Math. Trip. 1901.}

  [Express the left-hand side in terms of $\besJ_{n}(x)$ and
  $\besJ_{n+1}(x)$, and shew by example TODO that
  $\besJ_{n+1}(x)/\besJ_{n}(x)$ is irrational when $n$ and $x$ are
  rational.]
%
% 383
%
\item
  Prove that, when $\Re(n) > -\half$,
  \begin{align*}
    \besJ_{n}
    =&
    \frac{z^{n}}{2^{n-1}\Gamma(n+\half) \Gamma(\half)}
    \theparen{1 + \frac{\dd^{2}}{\dd z^{2}}}^{n-\half}
    \theparen{ \frac{\sin z}{z} },
    \\
    -\besY_{n}(z)
    =&
    \frac{z^{n}}{2^{n-1}\Gamma(n+\half) \Gamma(\half)}
    \theparen{1 + \frac{\dd^{2}}{\dd z^{2}}}^{n-\half}
    \theparen{ \frac{\cos z}{z} }.
  \end{align*}
  [
  $\theparen{1 + \frac{\dd^{2}}{\dd z^{2}}}^{n-\half}$
  means
  $ 1
  + \frac{n-\half}{1!} \frac{\dd^{2}}{\dd z^{2}}
  + \frac{ (n-\half) (n- \frac{3}{2}}{2!} \frac{\dd^{4}}{\dd z^{4}}
  + \cdots .
  $
  Write
  $ \frac{e^{iz}}{z} = \int_{\infty i}^{1} i e^{izt} \dmeasure t. $
  ]
  \addexamplecitation{TODO}
\item
  Shew that, when $\Re(m+\half)>0$,
  $$
  \theparen{
    \frac{2}{\pi}
  }^{\half}
  \int_{0}^{\half\pi}
  \besJ_{m}(z \sin\theta)
  \sin^{m+1}\theta
  \dmeasure \theta
  =
  z^{-\half} \besJ_{m+\half}(z).
  $$
  \addexamplecitation{Hobson.}
\item
  Shew that, if $sn+1 > m > -1$,
  $$
  \int_{0} x^{-n+m} \besJ_{n}(\alpha x) \dmeasure x
  =
  2^{-n+m} \alpha^{n-m-1}
  \frac{\Gamma(\half m + \half}{n - \half m + \half}.
  $$
  \addexamplecitation{TODO}
\item
  Shew that
  $$
  \frac{z}{\pi}
  =
  \sum_{p=0}^{\infty} \frac{2p+1}{2} \thebrace{
    \besJ_{p+\half}(z)}^{2}.
  $$
  \addexamplecitation{Lommel.}
\item
  In the equation
  $$
  \frac{\dd^{2} y}{\dd z^{2}}
  + \frac{1}{z} \frac{\dd y}{\dd z}
  + \theparen{ 1 + \frac{n^{2}}{z^{2}}} y
  = 0,
  $$
  $n$ is real; shew that a solution is given by
  $$
  \cos (n \log z)
  -
  \sum_{m=1}^{\infty}
  \frac{
    (-)^{m} z^{2m} \cos(u_{m} - n\log z)}{
    2^{2m} m! (1+n^{2})^{\half}
    (4 + n^{2})^{\half} \cdots (m^{2}+n^{2})^{\half}}
  $$
  where $u_{m}$ denotes $\sum_{r=1}^{m} \arctan (n/r)$.
  \addexamplecitation{Math. Trip. 1894.}
\item
  Shew that, when $n$ is large and positive,
  $$
  \besJ_{n}(n)
  =
  2^{-2/3} 3^{-1/6} \pi^{-1} \Gamma(1/3) n^{-1/3} + \littleo(n^{-1}).
  $$
  \addexamplecitation{TODO}
\item
  Shew that
  $$
  \besK_{0}(x)
  =
  \int_{0}^{\infty} \frac{t \besJ_{0}(tx)}{1+t^{2}} \dmeasure t.
  $$
  \addexamplecitation{TODO}
\item
  Shew that
  $$
  e^{\lambda\cos\theta}
  =
  2^{n-1} \Gamma(n)
  \sum_{k=0}^{\infty} (n+k) C_{k}^{n}(\cos\theta) \lambda^{-n} \besI_{n+k}(\lambda).
  $$
  \addexamplecitation{Math. Trip. 1900.}
\item
  Shew that, if
  $$
  W
  =
  \int_{0}^{\infty} \besJ_{m}(a x) \besJ_{m}(bx) \besJ_{m}(cx) x^{1-m}
  \dmeasure x,
  $$
  $a,b,c$ being positive, and $m$ is a positive integer or zero, then
  $$
  W =
  \begin{cases}
    0 & (a-b)^{2} > c^{2}, \\
    \frac{a^{-m} b^{-m} c^{-m}}{2^{3m-1} \pi^{\half} \Gamma(m+\half)}
    \thebrace{2 \sum b^{2} c^{2} - \sum a^{4}}^{m-\half} & (a+b)^{2} >
    c^{2} > (a-b)^{2}, \\
    0 & (a+b)^{2} > c^{2}.
  \end{cases}
  $$
  \addexamplecitation{TODO}
%
% 384
%
\item
  Shew that, if $n > -1$, $m > -\half$ and
  $$
  W
  =
  \int_{0}^{\infty}
  \besJ_{n}(ax) \besJ_{n}(bx) \besJ_{m}(cx) x^{1-m} \dmeasure x,
  $$
  $a,b,c$ being positive, then
  $$
  W =
  \begin{cases}
    0
    &
    (a-b)^{2} > c^{2},
    \\
    (2\pi)^{-\half}
    a^{m-1} b^{m-1} c^{-m}
    (1-\mu^{2})^{\frac{1}{4} (2m-1)}
    P_{n-\half}^{\half-m}(\mu)
    &
    (a+b)^{2} > c^{2} > (a-b)^{2},
    \\
    (\half\pi)^{-\half}
    a^{m-1} b^{m-1} c^{-m}
    \frac{\sin (m-n)\pi}{\pi}
    e^{(m-\half)\pi i}
    (\mu_{1}^{2} -1)^{\frac{1}{4} (2m-1)}
    Q_{n-\half}^{\half-m}(\mu_{1})
    &
    c^{2} > (a+b)^{2},
  \end{cases}
  $$
  where
  $$
  \mu=(a^{2} + b^{2} - c^{2})/2ab, \quad \mu_{1} = -\mu.
  $$
  \addexamplecitation{TODO}
\item
  TODO(n40)
\item
  Shew that $O_{n}(z)$ satisfies the differential equation
  $$
  \frac{\dd^{2} O_{n}(z)}{\dd z^{2}}
  +
  \frac{3}{z} \frac{\dd O_{n}(z)}{\dd z}
  +
  \thebrace{1 - \frac{n^{2}-1}{z^{2}}} O_{n}(z)
  =
  g_{n},
  $$
  where
  $$
  g_{n} = z^{-1} \textrm{ ($n$ even)},
  \quad
  g_{n} = n z^{-2} \textrm{ ($n$ odd)}.
  $$
  \addexamplecitation{K. Neumann.}
\item
  If $f(z)$ be analytic throughout the ring-shaped region bounded by
  the circles $c,C$ whose centres are at the origin, establish the
  expansion
  \begin{align*}
    f(z)
    =&
    \half \alpha_{0} \besJ_{0}(z) + \alpha_{1} \besJ_{1}(z)
    + \alpha_{2} \besJ_{2}(z) + \dots
    \\
    &
    + \half \beta O_{0}(z) + \beta_{1} O_{1}(z) + \beta_{2} O_{2}(z) + \dots,
  \end{align*}
  where
  $$
  \alpha_{n}
  =
  \frac{1}{\pi i}
  \int_{C} f(t) O_{n}(t) \dmeasure t,
  \quad
  \beta_{n}
  =
  \frac{1}{\pi}
  \int_{c} f(t) \besJ_{n}(t) \dmeasure t.
  $$
  \addexamplecitation{K. Neumann.}
\item
  Shew that, if $x$ and $y$ are positive,
  $$
  \int_{0}^{\infty}
  \frac{e^{-\beta x}}{\beta}
  \besJ_{0}(ky)
  k \dmeasure k
  =
  \frac{e^{-ir}}{r},
  $$
  where $r = \sqrt{x^{2}+y^{2}}$ and
  $\beta = +\sqrt{k^{2} - 1}$
  or
  $i \sqrt{1 - k^{2}}$
  according as $k>1$ or $k<1$.
  \addexamplecitation{Math. Trip. 1905.}
%
% 385
%
\item
  Shew that, with suitable restrictions on $n$ and on the form of the
  function $f(x)$,
  $$
  f(x)
  =
  \int_{0}^{\infty} \besJ_{n}(tx) t
  \thebrace{
    \int_{0}^{\infty}
    f(x') \besJ_{n}(tx') x' \dmeasure x'
  }
  \dmeasure t.
  $$
  [A proof with an historical account of this important theorem is
  given by TODO. It is due to Hankel, but (in view of the result of
  \hardsectionref{9}{7}) it is often called the \emph{Fourier-Bessel
    integral}.]
\item
  If $C$ be any closed contour, and $m$ and $n$ are integers, shew
  that
  $$
  \int_{C} \besJ_{m}(z) \besJ_{n}(z) \dmeasure z
  =
  \int_{C} O_{m}(z) O_{n}(z) \dmeasure z
  =
  \int_{C} \besJ_{m}(z) O_{n}(z) \dmeasure z
  =
  0,
  $$
  unless $C$ contains the origin and $m=n$; in which case the first
  two integrals are still zero, but the third is equal to $\pi i$ (or
  $2\pi i$ if $m=0$) if $C$ encircles the origin once
  counterclockwise.
  \addexamplecitation{K. Neumann.}
\item
  Shew that, if
  $$
  \frac{(-)^{p}}{p!q!} = a_{p,q},
  $$
  and if $n$ be a positive integer, then
  $$
  z^{-2n}
  =
  \sum_{m=1}^{n} a_{n-m,n+m-1} O_{2m-1}(z),
  $$
  while
  $$
  z^{1-2n}
  =
  a_{n-1,n-1} O_{0}(z)
  +
  2 \sum_{m=1}^{n-1} a_{n-m-1,n+m-1} O_{2m}(z).
  $$
  \addexamplecitation{K. Neumann.}
\item
  If
  $$
  \Omega_{n}(y)
  =
  \sum_{m=0}^{n}
  \frac{2^{2m} (m!)^{2}}{2m!}
  \frac{n^{2} (n^{2}-1^{2}) (n^{2}-2^{2}) \cdots (n^{2}-(m-1)^{2})}{y^{2m+2}},
  $$
  shew that
  $$
  (y^{2}-x^{2})^{-1}
  =
  \Omega_{0}(y)
  \thebrace{\besJ_{0}(x)}^{2}
  +
  2 \sum_{n=1}^{\infty} \Omega_{n}(y) \thebrace{\besJ_{n}(x)}^{2}
  $$
  when the series on the right converges.
  \addexamplecitation{K. Neumann TODO}
\item
  Shew that, if $c>0$, $\Re(n) > -1$ and
  $\Re(a \pm b)^{2} > 0$, then
  $$
  \besJ_{n}(a) \besJ_{n}(b)
  =
  \frac{1}{2\pi i}
  \int_{c-\infty i}^{c+\infty i}
  t^{-1} \exp \thebrace{ (t^{2}-a^{2}-b^{2})/(2t)}
  \besI_{n}(ab/t) \dmeasure t.
  $$
  \addexamplecitation{TODO}
\item
  Deduce from example TODO, or otherwise prove, that
  \begin{align*}
    (a^{2}+b^{2}-2ab \cos\theta)^{-\half n}
    \besJ_{n}\thebrace{ (a^{2}+b^{2}-2ab\cos\theta)^{\half}}
    \\
    =
    2^{n} \Gamma(n)
    \sum_{m=0}^{\infty}
    (m+n) a^{-n} b^{-n} \besJ_{m+n}(a) \besJ_{m+n}(b)
    C_{m}^{\ n}(\cos\theta).
  \end{align*}
  \addexamplecitation{TODO}
\item
  Shew that
  $$
  y = \int_{C} \besJ_{m}(t) \besJ_{n}(t z^{\half}) t^{k-1} \dmeasure t
  $$
  satisfies the equation
  $$
  \frac{\dd^{2} y}{\dd z^{2}}
  +
  \theparen{\frac{1}{z} + \frac{k}{z-1}} \frac{\dd y}{\dd z}
  +
  \theparen{k^{2}-m^{2}+\frac{n^{2}}{z}} \frac{y}{4z(z-1)}
  =
  0
  $$
  if
  $$
  k t^{k} \besJ_{m}(t) \besJ_{n}(t z^{\half})
  -
  t^{k+1} \besJ'_{m}(t) \besJ'_{n}(t z^{\half})
  +
  z^{\half} t^{k+1} \besJ_{m}(t) \besJ'_{n}(t z^{\half})
  $$
  resumes its initial value after describing the contour.

  Deduce that, when $0<z<1$,
  $$
  \int_{0}^{\infty}
  \besJ_{\alpha-\beta}(t)
  \besJ_{\gamma-1}(t z^{\half})
  t^{1+\beta+\gamma}
  \dmeasure t
  =
  \frac{\Gamma(\alpha) z^{\half(\gamma-1)}}{ 2^{\gamma-\alpha-\beta}
    \Gamma(1-\beta) \Gamma(\gamma)}
  F(\alpha,\beta;\gamma;z).
  $$
  \addexamplecitation{TODO; Math. Trip. 1903.}
\end{enumerate} % First pass complete; missing references
\chapter{The Equations of Mathematical Physics} 

18"1. The differential equations of mathematical pliysics. 

The functions which have been introduced in the preceding chapters are 
of importance in the applications of mathematics to physical investigations. 
Such applications are outside the province of this book ; but most of them 
depend essentially on the fact that, by means of these functions, it is possible 
to construct solutions of certain partial differential equations, of which the 
following are among the most important : 

(I) Laplace s equation 

dx- dy'  dz  
which was originally introduced in a memoir* on Saturn's rings. 

If  x, y, z) be the rectangular coordinates of any point in space, this equation is 
satisfied by the following functions which occur in various branches of mathematical 
physics : 

(i) The gravitational potential in regions not occupied by attractii:ig matter. 

(ii) The electrostatic potential in a uniform dielectric, in the theory of electro- 
statics. 

(iii) The magnetic potential in free aether, in the theory of magnetostatics. 

(iv) The electric potential, in the theory of the steady flow of electric currents in 
solid conductoi's. 

(v) The temperature, in the theory of thermal equilibrium in solids. 

(vi) The velocity potential at points of a homogeneous liquid moving irrotationally, 
in hydrodynamical problems, 

Notwithstanding the physical diflferences of these theories, the mathematical investi- 
gations are much the same for all of them : thus, the problem of thermal equilibrium in a 
solid when the points of its surface are maintained at given temperatures is mathe- 
matically iden.tical with the problem of determining the electric intensity in a region 
when the points of its boundary are maintained at given potentials. 

(II) The equation of ivave motions 

dx  dy'  dz  C' dt- 
This equation is of general occurrence in investigations of undidatory disturbances 
propagated with velocity c independent of the wave length ; for example, in the theory of 
electric waves and the electro-magnetic theory of light, it is the equation satisfied by each 
component of the electric or magnetic vector; in the theory of elastic vibrations, it 
is the equati<Mi satisfied by each component of the displacement ; and in the theory 
of sound, it is the equation satisfied by the velocity potential in a perfect gas. 

* Mem. de FAcad. des Sciences, 1787 (published 1789), p. 252. 



18*1, 18 "2] THE EQUATIONS OF MATHEMATICAL PHYSICS 387 

(III) The equation of conduction of heat 

dfV dfV dfV\ ldV 
dec- dy  dz- k dt 

This is the equation satisfied by the temperature at a point of a homogeneous isotropic 
body ; the constant k is proportional to the heat conductivity of the body and inversely 
proportional to its specific heat and density. 

(IV) A particular case of the preceding equation (II), when the variable 
z is absent, is 

dx- dy- c- dt'  

This is the equation satisfied by the displacement in the theory of transverse vibrations 
of a membrane ; the equation also occurs in the theory of wave motion in two dimensions. 

(V) The equation of telegraphy 

This is the equation satisfied by the potential in a telegraph cable when the inductance 
X, the capacity K, and the resistance It per unit length are taken into account. 

It would not be possible, within the limits of this chapter, to attempt 
an exhaustive account of the theories of these and the other differential 
equations of mathematical physics; but, by considering selected 'typical 
cases, we shall expound some of the principal methods employed, with 
special reference to the uses of the transcendental functions. 

18'2. Boundary conditions. 

A problem which arises very frequently is the determination, for one of the 
 equations of § 18*1, of a solution which is subject to certain boundary con- 
ditions ; thus we may desire to find the temperature at any point inside a 
homogeneous isotropic conducting solid in thermal equilibrium when the 
points of its outer surface are maintained at given temperatures. This 
amounts to finding a solution of Laplace's equation at points inside a given 
surface, when the value of the solution at points on the surface is given. 

A more complicated problem of a similar nature occurs in discussing 
small oscillations of a liquid in a basin, the liquid being exposed to the 
atmosphere ; in this problem we are given, effectively, the velocity potential 
at points of the free surface and the normal derivate of the velocity potential 
where the liquid is in contact with the basin. 

The nature of the boundary conditions, necessary to determine a solution 
uniquely, varies very much with the form of differential equation considered, 
even in the case of equations which, at first sight, seem very much alike. 
Thus a solution of the equation 



dx- dy'  



25—2 



388 THE TRANSCENDENTAL FUNCTIONS [CHAP. XVIII 

(which occurs in the problem of thermal equilibrium in a conducting 
cylinder) is uniquely determined at points inside a closed curve in the 
x7/-ip\ a,ne by a knowledge of the value of V at points on the curve ; but 
in the case of the equation 

da;  c- dt- 

(which effectively only differs from the former in a change of sign), occurring 

in connexion with transverse vibrations of a stretched string, where V 

denotes the displacement at time t at distance x from the end of the 

string, it is physically evident that a solution is determined uniquely only if 

dV 
both V and -  are given for all values of x such that  x l, when   = 

(where I denotes the length of the string). 

Physical intuitions will usually indicate the nature of the boundary 
conditions which are necessary to determine a solution of a differential 
equation uniquely ; but the existence theorems which are necessary from 
the point of view of the pure mathematician are usually very tedious and 
difficult*. 

18"3. A general solution of Laplace s equation . 

It is possible to construct a general solution of Laplace's equation in the 
form of a definite integral. This solution can be employed to solve various 
problems involving boundary conditions. 

Let V  x, y, z) be a solution of Laplace's equation which can be expanded 
into a power series in three variables valid for points of  x, y, z) sufficiently 
near a given point  Xf , yo, z ). Accordingly we write 

x = Xq + X, y=zy  + Y, z = Zo+ Z; 

and we assume the expansion 

V = c(o + a X + biY + CiZ + a.. X- + b. Y- + c  Z  

+ 2cLYZ+2e,ZX + 2f,XY+..., 

it being supposed that this series is absolutely convergent whenever 

\ X'f+\ Yr + \ Z'  a, 

where a is some positive constant :]:. If this expansion exists, V is said to 
be analytic at (xo, yo, •S'o). It can be proved by the methods of §§ 3*7, 4'7 

* See e.g. Forsyth, Theory of Functions (1918), §§ 216-220, where an apparently simple 
problem is discussed. 

t Whittaker, Math. Ann. lvii. (1902), p. 333. 

* The functions of applied mathematics satisfy this condition. 



18-3] THE EQUATIONS OF MATHEMATICAL PHYSICS 389 

that the series converges uniformly throughout the domain indicated and 
may be differentiated term-by-term with regard to X, Y or Z any number of 
times at points inside the domain. 

If we substitute the expansion in Laplace's equation, which may be 
written 

d'V a F d V   
dX-' dY-' dZ-' ' 

and equate to zero (§ 373) the coefficients of the various powers of X, Y 
and Z, we get an infinite set of linear relations between the coefficients, 
of which 

a.2 4- 62 -I- c. = 
may be taken as typical. 

There are  n(n — l) of these relations* between the   n + 2) n + l) 
coefficients of the terms of degree n in the expansion of V, so that there 
are only  (n+2)(n + 1) - n (n - 1) = 2n -f 1 independent coefficients in 
the terms of degree n in V. Hence the terms of degree n in V must be 
a linear combination of 2n+l linearly independent particular solutions of 
Laplace's equation, these solutions being each of degree n in X, Y and Z. 

To find a set of such solutions, consider (Z + iX cos u 4- iFsin m)" ; it is 
a solution of Laplace's equation which may be expanded in a series of sines 
and cosines of multiples of u, thus : 

n n 

2 gm X, Y, Z) COS mu+ 2 h,, (X, Y, Z) sin. mu, 

m = 111 = 1 

the functions g iX, Y, Z) and h   X, Y, Z) being independent of u. The 
highest power of Z in gm  X, Y, Z) and A,,  (X, F, Z) is Z"'"  and the former 
function is an even function of Y, the latter an odd function; hence 
the functions are linearly independent. They therefore form a set of 
2?i -f- 1 functions of the type sought. 

Now by Fourier's rulef (§ 9'1'2) 

Trgm  X, Y, Z)= j  Z + iX cos u + i Fsin u)" cos mudu, 

irhyn  X, Y, Z)= \  Z + iX cos u -H iFsin uY sin mudu, 

J —77 

* If a ,s,j (where r + s + t = n) be the coefficient of A''T Z' in V, and if the terms of degree 
  . d-V d V dW   , . ., 

" ~ WV "*" Wr- "*" dZ  arranged primarily in powers of X and secondarily in powers of F, 

the coefficient fl .s.t <loes not occur in any term after Z'— -r Z< (or ZT*-   if r = or 1), and 
hence the relations are all linearly independent. 

t 27r must be written for tt in the coefficient of g   X, Y, Z). 



390 



THE TRANSCENDENTAL FUNCTIONS [CHAP. XVIII 



and so any linear combination of the 2/; + 1 solutions can be written in the 
form 



/: 



 Z + iX COS i( + iY sin i()' fn  u) du, 

where / (w) is a rational function of e*". 

Now it is readily verified that, if the terms of degree n in the expression 
assumed for V be written in this form, the series of terms under the integral 
sign converges uniformly if \ X\ \   + \ Y Jr\ Z  be sufficiently small, and so 
(§ 4*7) w e may write 

V =\    Z+iX cos u + iY sin ?()"/  (w) du. 

• -TT n-i) 

But any expression of this form may be written 

F = / F Z - iX cos u + i Y sin u, u) du, 

J —n 

where i  is a function such that differentiations with regard to X, Y or Z 
under the sign of integration are permissible. And, conversely, if F be any 
function of this type, V is a solution of Laplace's equation. 

This result may be written 



J —TT 



on absorbing the terms — 2 ,1 — * o cos m — I'yo sin m into the second variable; 
and, if differentiations under the sign of integi-ation are permissible, this 
gives a general solution of Laplace's equation ; that is to say, every solution 
of Laplace's equation which is analytic throughout the interior of some 
sphere is expressible by an integral of the form given. 

This result is the three-dimensional analogue of the theorem that 

V=f x- iy)+g x-ii/) 
is the general solution of 

ox  oy- 

[NoTE. A distinction has to be drawn between the primitive of an ordinary diflFerential 
equation and general integrals of a i)artial differential equation of order higher than the 
first*. 

Two apparently distinct primitives are always directly transformable into one another 

by means of suitable relations between the constants; thus in the case of ;T"f +y = 0, we 

can obtain the primitive Csin  x + t) from A cos .r + 5 sin .r by defining C and e by the 
equations Csin e = A, C'cos f = B. On the other hand, every solution of Laplace's equation 
is expressible in each of the forms 



/: 



f x cos < +// sin   + iz, t) dt, I g (y cos ?< +    sin u + ix,  ) du ; 

r ' ' — TT 



* For a discussion of general integrals of such equations, see Forsyth, Theory of Differential 
Equations, vi. (1906), Ch. xii. 



18 "31] THE EQUATIONS OF MATHEMATICAL PHYSICS 391 

but if these are known to be the same solution, there appears to be no general analytical 
relation, connecting the functions / and g, which will directly transform one form of 
the solution into the other.] 

Example 1. Shew that the potential of a particle of unit mass at (a, b, c) is 
1 [ " du 



2tt J -TT  z — c) + i  x - a) cos u + i y — b) sin u 
at all points for which z> c. 

Example 2. Shew that a general solution of Laplace's equation of zero degree in 
X, y, z is 

/ log (.rcos <+j/sin il + jj)  (<)(:/i, if I g t)dt = Q. 

Express the solutions -  — ; and log-; — \   in this form, where r- = x' +y' - z . 

Example 3. Shew that, in the case of the equation 

p  + g;  = x+y 

( where jo =  , q =  ), integrals of Charpit's subsidiary equations (see Forsyth, Differential 

Equations, Chap, ix.) are 

(i) p --x=y-q - = a, 

(ii) p = q + a?. 

Deduce that the corresponding general integrals are derived from 

(i) z =  , :c + af+l y-af+F a)\ 
Q =  x + af- y-af + F'  a) j' 

(ii) Az = \ (. +y)3 + 2a2  x-y)-a   x -y)-  + G a)\ 
= Aa x-y)-Aa  x->ry)-'  + G' [a) j' 

and thence obtain a diflferential equation determining the function O (a) in terms of the 
function E a) when the two general integrals are the same. 

18'31. Solutions of Laplace s equation involving Leg endive functions. 
If an expansion for V, of the form assumed in § 18'3, exists when 

Xq = y  Zq    ) , 
we have seen that we can express F as a series of expressions of the type 

I (z + ix cos u + iy sin u)"' cos mudu, (z + ix cos u + iy sin uy  sin inudu, 

J —TT J -TT 

where n and m are integers such that  m n. 

We shall now examine these expressions more closely. 
If we take polar coordinates, defined by the equations 

x= r sin 6 cos (f), y = r sin 6 sin (/>, z = r cos 6, 



392 THE TRANSCENDENTAL FUNCTIONS [CHAP. XVIII 

we have 

I  z + ix cos u + iy sin w)" cos mu du 

J — ]T 

= ?•" I  cos   + i sin 6 cos (  —  )j" cos mudu 

rn-<t> 
= ?'" (cos + i sin   cos yjr]" cos 7?l (</) + ylr) rf-v/r 

•Z — TT — I  

= ?'"  cos   + I sin   cos -v  " cos m  (p +  fr) dy\ r 

J —It 

= ?•" COS m(f>  cos d + i sin cos i  "  cos m'yjrd'yjr, 

J -TT 

since the integrand is a periodic function of -  and 

(cos   + z sin 6 cos i/r)" sin  ii/r 

is an odd function of yjr. Therefore (§ 15"61), with Ferrers' definition of the 
associated Legendre function, 

rir 27ri"' . n ! 

1 (z + ix cos u + iy sin  )" cos ??m c?ii = 7  -;' r Pn'  (cos  ) cos md). 

j -  c / (-. j + m) ! 

Similarly 



/: 



(  + iv cos i/ + 1?/ sin k)" sin ?/it<c?w = 7  -! r Fn '  (cos  ) sin md>. 

'  (71 + m) I 



Thei efore 7'' P,i"  (cos 6) cos 7?i  aiic? r ''Pn'  (cos  ) sin 7/i</> are polynomials 
in X, y, z and are particular solutions of Laplace's equation. Further, by 
§ 18"3, every solution of Laplace's equation, which is analytic near the origin, 
can be expr essed in the form 

F = i r'  \ AnFn (cos (9) + I (yl " ' cos m<  + Bn '''  sin m ) P "  (cos 6)1 . 

Any expression of the form 

AnPn (cos  ) + i ( n<"" cos ? </) + 5 <'   sin m(f>) P,r (cos  ), 

m = \ 

where w is a positive integer, is called a surface harmonic of degree 7i ; 
a surface harmonic of degi-ee n multiplied by ?'" is called a solid harmonic 
(or a spherical Jiarmonic) of degree n. 

The curves on a unit sphere (with centre at the origin) on which P  (cos 6) vanishes 
are n parallels of latitude which divide the surface of the sphere into zones, and so P  (cos d) 

is called (see S 15*1) a zonal harmonic ; and the curves on which . mtb . Pn"  (cos 6) vanishes 

are n—7n parallels of latitude and 2m meridians, which divide the surface of the sphere 
into quadrangles wh(> e angles are right angles, and so these functions are called tesseral 
harmonics. 



18 '4] THE EQUATIONS OF MATHEMATICAL PHYSICS 393 

A solid harmonic of degree n is .evidently a homogeneous polynomial of degree n in 
X, y, z and it satisfies Laplace's equation. 

It is evident that, if a change of rectangular coordinates* is made l;)y rotating the axes 
about the origin, a solid harmonic (or a surface harmonic) of degree n transforms into 
a solid harmonic (or a suz'face harmonic) of degree n in the new coordinates. 

Spherical harmonics were investigated with the aid of Cartesian coordinates by 
W. Thomson in 1862, see Phil. Trans. (1863), pp. 573-582, and Thomson and Tait, 
Treatise on Natural Philosophy i. (1879), jip. 171-218 ; they were also investigated 
independently in the same manner at about the same time by Clebsch, Journal fur Math. 
LXi. (1863), pp. 195-262. 

Example. If coordinates r,  ,   are defined by the equations 

1 1 

a; = rcos , ?/ = (/'— 1)- sin   cos 0, 3 = (/'2- 1)- sin  sin( , 

shew that P,/" (r) P '" (cos 6) cos wi0 satisfies Laplace's equation. 

18'4. The solution of Laplace's equation which satisfies assigned boundary 
conditions at the surface of a sphere. 

We have seen (§ 18"31) that any solution of Laplace's equation which 
is analytic near the origin can be expanded in the form 



w = ( 

+ i (  <" ' cos m(f) + 5 " ' sin m(f>) P,,"* (cos 6) \ 

m = l J 

and, from § 3*7, it is evident that if it converges for a given value of r, 
say a, for all values of 6 and (f) such that O  tt, — tt  tt, it converges 
absolutely and uniformly when r < a. 

To determine the constants, we must know the boundary conditions 
which V must satisfy. A boundary condition of frequent occurrence is 
that F is a given bounded integrable function of 6 and (f), say f(0, (f>), on 
the surface of a given sphere, which we take to have radius a, and V is 
analytic at points inside this sphere. 

We then have to determine the coefficients An, -4 <'"', 5,i""* from the 
equation 



f d,( )= S a'M Pn (cos  )+ 2 (  ' ' cos ?n</) + £ " ' sin m</))P,;'  (cos 6')   

rt = i m = l J 

Assuming that this series converges uniformly f throughout the domain 

multiplying by 

P '  (cos  )  °%i<i, 
sm   

* Laplace's operator  — , + ;r— , + ; -5 is invariant for changes of rectangular axes. 
ox  dij- oz- 

t This is usually the case in physical problems. 



394 THE TRANSCENDENTAL FUNCTIONS [CHAP. XVIII 

integrating term-by-term (§ 4'7) and using the results of §§ 15'14, 1551 on 
the integral properties of Legendre functions, we find that 

fid', 4>') P,r (cos 6') cos m<f>' sin e'dd'dcf ' = 7ra  -   . )   " ', 

Cn fir 9 (yx -1- 171, "  t 

A 6*', 6') P '  (cos d') sin wt<i)' sin O'dO'd ' = -jra"   =- . ) ( ; 5 <' ', 

f f V( '' <l>') Pn (cos  ') sin e'dB'dcf)' = lira"  5-?—.   . 
Therefore, when r < a, 
F(r, e,4>)=l ?  f-)" f  f7( ', f )]p,(cos )P (cos ') 

+ 2 i (!i:i ; p  m (cos 6) Pn'  (cos  ') COS m (6 - d)')  sin O'dO'd )'. 

The series which is here integrated term-by-term converges uniformly 
when r <a, since the expression under the integral sign is a bounded 
function of 6, 0', (f), (f>', and so (§ 4*7 ) 

47rF(r, 6*, 0) = T 1 /(0', cf>') I (2n -h 1) f-VlPn (cos  ) P  (cos  ) 

+ 2 i  y' —  Pn'"" (cos d)Pn''' (cos d') cos 7n(4>-(f>')\ sin e'de'd(f>'. 
m=i(n + m)l J 

Now suppose that we take the line (6, (f>) as a new polar axis and let 
( 1'.  1') be the new coordinates of the line whose old coordinates were (6', < ') ; 
we consequently have to replace Pn (cos 6) by 1 and P, !  (cos 6) by zero ; and 
so we get 

47rF(r, e, (f>)=r t  f(0', </)') t (2n + l)(-Y Pn(cos0,')sme,'de,'d<f>,' 

J -TT J W=0 \ \  V 

= r rf(0''<f>') i (2n+l)(-Y Pn(cose,')sme'dO'd(f>'. 

If, in this formula, we make use of the result of example 23 of Chapter xv 
(p. 332), we get 

 ' '  J-nJo \ r'-2arcos6,'+a-)  
and so 

47r F(r, e, </)) 

, r f" f(0', <i>') sin e'dd'dd)' 

= a(a- — r')\ I  - — —n- 

J -nJ [r- — 2ar  cos 6 cos  ' -f sin 6 sin  ' cos ((  — </>)  + a"]  

In this compact formula the Legendre functions have ceased to appear 
explicitly. 



18 '5] THE EQUATIONS OF MATHEMATICAL PHYSICS 395 

The last formula can be obtained by the theory of Green's functions. For properties 
of such functions the reader is refei'red to Thomson and Tait, Natural Philosophy., 
§§ 499-519. 

[Note. From the integrals for V (r, 6, cf)) involving Legendre functions of cos  j' and 
of cos , cos ' respectively, we can obtain a new proof of the addition theorem for the 
Legendre polynomial. 

For let 

Xn  \&', (t>') =Pn (COS d ) - |P  (cos 6) P  (cOS 6')  

+ 2 2  — -f; P - (COS 6) P,;" (cos 6') cos ra (0 - < ') , 
and we get, on comparing the two formulae for V  r, 6, (f)), 

0= r f'/iff, < ') 2 (2 +i) (-Xxn (d\ 4>') siD e'd\&d ,'. 

J -TT J n=0 \ / 

If we take/( ', 0') to be a surface harmonic of degree n, the term involving r" is the only 
one which occurs in the integrated series ; and in particular, if we take/( ', 4>') = Xn i 'i 0')> 
we get 

' Xni6',fp')   'id'de'd(t)' = 0. 



-n J 

Since the integrand is continuous and is not negative it must be zero ; and so 
Xn ff, (f)') = 0; that is to say we have proved the formula 

Pn (cos  i') = Pn (cos 6) Pn (COS 6') + 2 2 ) '- . Pn'" (COS 6) Pn"" (COS 6') COS 711  (j>- (f)'), 

m-1  n + M) ! 
wherein it is obvious that 

cos  ]' = cos 6 cos 6' + sin 6 sin 0' cos (0 - < '), 
from geometrical considerations. 

We have thus obtained a physical proof of a theorem proved elsewhere* (§ I5"7) by 
purely analytical reasoning.] 

Example 1. Find the solution of Laplace's equation analytic inside the sphere /•=1 
which has the value sin 36 cos <  at the surface of the sphere. 

[ s r Pgi (cos 6) cos (f) - irPji (cos 0) cos 0.] 

Example 2. Let fii r, 6, (f)) be equal to a homogeneous polynomial of degree n 
in  , y, z. Shew that 

l" I fnla, G, (f)) Pn  cos 6 con d' + siudamd' cos  (f) -4>')  a'  sin 6 d0dcf> 
J -  J o' 

- ; /.( . '< )- 

[Take the direction (6', < ') as a new polar axis.] 

18*5. Solutions of Laplace's equation luhich involve Bessel coeffi,cients. 
A particular case of the result of § 18'3 is that 



Qk(z+ixcosu+iysmu, gQg f)iuclu 

is a solution of Laplace's equation, k being any constant and m being any 

integer. 

* The absence of the factor ( - )'"  which occurs in § 1-5-7 is due to the fact that the functions 
now employed are Ferrers' associated functions. 



396 THE TRANSCENDENTAL FUNCTIONS [CHAP. XVIII 

Taking cylindrical-polar coordinates  p, (f), z) defined by the equations 
.   = p cos  , y = p sin 4>, 
the above solution becomes 

 kz I gikpcos iu-4>) cos  (   dii = gkz i gikpcos V g g .   (v + (f)) . clv 

J —IT J — JT 

= 2e*  I 6**'" °  " cos ?HV cos m<pdv 

Jo 

= 2e*  cos (mcj)) j e'*'" °  " cos mvdv, 

Jo 
and so, using § 17*1 example 3, we see that 'Itti '  e''  cos (m(f)) . Jm(f 'p) is a 
solution of Laplace's equation analytic near the origin. 

Similarly, from the expression 

T 

where m is an integer, ive deduce that 27rz'" e*  sin (??i</)) .    (kp) is a solution 
of Laplace's equation. 

18 "SI. The periods of vibration of a uniform membrane*. 

The equation satisfied by the displacement V at time t of a point  x, y) of a uniform 
plane membrane vibrating harmonically is 

'dx  cy- c- Ct- ' 

where c is a constant depending on the tension and density of the membrane. The 
equation can be reduced to Laplace's equation by the change of variable given by z = cti. 
It follows, from § 18'5, that expressions of the form 

'  sm   sin 
satisfy the equation of motion of the membrane. 

Take as a particular case a drum, that is to say a membrane with a fixed circular 
boundary of radius R. 

Then one possible type of vibration is given by the equation 

r=t    kp) cos m<\ i cos ckt, 
provided that F=0 when p = R ; so that we have to choose k to satisfy the equation 

J,  kR)=0. 
This equation to determine /  has an infinite number of real roots (§ 17'3 example 3), 
 1,  2j  3j ••• S3,y. A possible type of vibration is then given by 

r=/ , (f-'rp) cosmcf) cos ckrt  r= 1, 2, 3, ,..). 

This is a periodic motion with period 1iTJ ckj.) ; and so the calculation of the periods 
depends essentially on calculating the zeros of Bessel coefficients (see   17'9). 

• Euler, Novi Covim. Acad. Petrop. x. (1764) [published 1766], pp. 243-260; Poisson, Mem. 
de I'Academie, viii. (1829), pp. 357-570; Bourget, Ann. de I'Ecole nvrm.sup. iii. (1866), pp. 55-95. 
For a detailed discussion of vibrations of membranes, see also Rayleigh, Theory of Sound, 
Chapter ix. 



18 "5 1-1 8 '61] THE EQUATIONS OF MATHEMATICAL PHYSICS 397 

Example. The equation of motion of air in a circular cylinder vibrating per- 
pendicularly to the axis OZ of the cylinder is 

dx  dy'  c' dt- ' 
V denoting the velocity potential. If the cylinder have radius R, the boundary condition 

dV 
is that n— =0 when p = R. Shew that the determination of the free periods depends on 

finding the zeros of J,,/ (C) == 0- 

18'6. A general solution of the equation of wave motions. 
It may be shewn* by the methods of § 183 that a general solution of 
the equation of wave motions 

dx  dy'' dz- c- di? 



IS 



F = I I f(x sin u cos v + y sin u sin v + z cos n + ct, u, v) dudv, 



where /is a function (of three variables) of the type considered in § 18"3. 

Regarding an integral as a limit of a sum, we see that a physical 
interpretation of this equation is that the velocity potential V is produced 
by a number of plane waves, the disturbance represented by the element 

f(x sin u cos v + y sin u sin v + z cos u + ct, u, v) 8u Bv 
being propagated in the direction (sin u cos v, sin u sin v, cos u) with velocity c. 
The solution therefore represents an aggregate of plane waves travelling in 
all directions with velocity c. 

18'61. Solutions of the equation of wave motions which involve Bessel 
functions. 

We shall now obtain a class of particular solutions of the equation of 
wave motions, useful for the solution of certain special problems. 

In physical investigations, it is desirable to have the time occurring by 
means of a factor sin ckt or cos ckt, where k is constant. This suggests that 
we should consider solutions of the type 

/•tt rn 

y  =, \ I giA;(a;sinttcost;+2/sinMsiuv+2cosM+cti f (u i)\ dudv 

J -TT J 

Physically this means that we consider motions in which all the elementary waves 
have the same period. 

Now let the polar coordinates of (x, y, z) be (r, 6, (f>) and let (o), yjr) be the 
polar coordinates of the direction (u, v) referred to new axes such that 
the polar axis is the direction (ff, (f)), and the plane -v/r = passes through 
OZ; so that 

cos CO = cos 6 cos u + sin d sin u cos ((f) — v), 

sin II sin ((f) — v) = sin w sin yjr. 

* See the paper previously cited, Math. Ann. lvii. (1902), pp. 342-345, or Messenger of Mathe- 
matics, XXXVI. (1907), pp. 98-106. 



398 THE TRANSCENDENTAL FUNCTIONS [CHAP. XVIII 

Also, take the arbitrary function /(w, y) to be ;S,j (w, y) sin i  where Sn 
denotes a surface harmonic in u, v of degree ?i ; so that we may write 

Sn(u, V) = Sn  0, 4>] CO, yjr), 

where (§ 18*31) Sn is a surface harmonic in co, -sjr of degree n.   
We thus get 

V = e ' ' f r I ' e''""°"" Sn  6, < \ w, y\ r) sin a> dw df. 

Now we may write (§ 18 "SI) 

Sn (d, (i>; 0 ,y r) = An e,(f>). Pn (COS Oj) 

+ S    "'" (0, <f>) cos myfr + Bn'"'   0, (f)) sin myfr] P "' (cos co), 

m = l 

where An  6, </>),  n'"'*  6, <p) and 5 ""'  6, 0) are independent of yjr and co. 
Performing the integration with respect to -yjr, we get 

F= 27re'' '   An  0, 4>) f"e '*"° "P  (cos to) sin codco 

Jo 

= 27re' '-  A n  0,(ii)j e''"-'  P,, (/i) djM 

= 27r. - '* An (  , ( ) J'  e'* '  -  (f.  - l)n d , 

by Rodrigues' formula (§ loll); on integrating by parts ti times and using 
Hankel's integral (§ 17 '3 corollary), we obtain the equation 

 = o;r '"" n  0, 4>)  ikrY I e'>"-  (1 - /x )" c yu 

Zi • 'It \ J 1 

= (27r)  iV'* '   kr) - -J  +   (kr) An (6, c >), 
and so F is a constant multiple of e' ' h-'   Jj \  \ i(kr) An(0, (f>). 

Now the equation of wave motions is unaffected if we multiply x, y, z 
and t by the same constant factor, i.e. if we multiply r and t by the same 
constant factor leaving Q and   unaltered ; so that An B, <f)) may be taken 
to be independent of the arbitrary constant k which multiplies r and t. 

Hence lim e**-''' r ~ H' ~ " ~   /  , i (kr) An  0, 9) is a solution of the equation 

A:-*0   

of wave motions; and therefore r' An(0, <f>) is a solution (independent of t) 
of the equation of wave motions, and is consequently a solution of Laplace's 
equation ; it is, accordingly, permissible to take An (0, <f>) to be any surface 
harmonic of degree n ; and so we obtain the result that 



r -   Jn+r ih-) Pn''' (cos 0)   md>   ckt 
 2   sm   sm 

is a particular solution of the equation of wave motions. 



I8"61lj THE EQUATIONS OF MATHEMATICAL PHYSICS 399 

18*611. Application of § 18-61 to a physical problem. 

The solution just obtained for the equation of wave motions may be used in the 
following manner to determine the periods of free vibration of air contained in a rigid 
sphere. 

The velocity potential T' satisfies the equation of wave motions and the boundary 

dV 
condition is that - =- =0 when ?  =  , where a is the radius of the sphere. Hence 

or   

~\ -, ,i.r, , /iv COS , COS , 

Y-r *J  ,.(/-/•)  ,™ cos<9) . wd) . ckt 
"+-   " sm   sin 

gives a possible motion if Z* is so chosen that 

This equation determines k; on using § 17'24, we see that it may be written in 
the form 

tan ka =    ka)  

where f  (ka) is a rational function of ka. 

In particular the radial vibrations, in which V is independent of 6 and < , are given by 
taking n = 0; then the equation to determine k becomes simply 

tan ka = ka ; 
and the pitches of the fundamental radial vibrations correspond to the roots of this 
equation. 

REFERENCES. 

J. Fourier, La theorie analytique de la Chaleur. (Translated by A. Freeman.) 

W. Thomson and P. G. Tait, Natural Philosophy. (1879.) 

Lord Rayleigh, Theory of Sound. (London, 1894-1896.) 

F. PoCKELS, Uber die partielle Diferentialgleichung /\ u + khi. = 0. (Leipzig, 1891.) 

H. BuRKHARDT, Eiitivickelungen nach oscillirenden Funktionen. (Leipzig, 1908.) 

H. Bateman, Electrical and Optical Wave-motion. (1915.) 

E. T. Whittaker, History of the Theories of Aether and Electricity. (Dublin, 1910.) 

A. E. H. Love, Proc. London Math. Soc. xxx. (1899), pp. 308-321. 

H. Bateman, Proc. London Math. Soc. (2), i. (1904), pp. 451-458. 

L. N. G. FiLON, Philosophical Magazine (6), vi. (1903), pp. 193-213. 

H. Bateman, Proc. London Math. Soc. (2), vii. (1909), pp. 70-89. 

Miscellaneous Examples. 

1. If V be a solution of Laplace's equation which is symmetrical with respect to OZ, 
and if y=f z) on OZ, shew that if f  \ be a function which is analytic in a domain of 
values (which contains the origin) of the complex variable f, then 



'=i r -   + i(.r2+j/2)5cos0 rf(  
  / 



at any point of a certain three-dimensional region. 

Deduce that the potential of a uniform circular ring of radius c and of mass M lying in 
the plane XO T with its centre at the origin is 



  r [  +    +   ( ' + '/)  cos (/) 2] - I d<t>. 

7!" y 



400 THE TRANSCENDENTAL FUNCTIONS [CHAP. XVIII 

2. If r be a solution of Laplace's equation, which is of the form e""*jP(p, 2), where 

(p, (f>, z) are cylindrical coordinates, and if this solution is approximately equal to 

p"'e""*/(i) near the axis of z, where /(f) is of the character described in example 1, 
shew that 



 = r ( i+ 1 ) r (*) /  ' "   ''''  ' '"""  ' - (DougaU.) 



3. If ?  be determined as a function of .r, y and z by means of the equation 

A.v+By + Cz=\, 
where A, B, C are functions of ?< such that 

shew that (subject to certain general conditions) any function of   is a solution of 

Laplace's equation. 

(Forsyth, Messenger, xxvii. (1898), pp. 99-118.) 

4. A, B are two points outside a sphere whose centre is C. A layer of attracting 
matter on the surface of the sphere is such that its surface density (Tp at P is given by 
the formula 

apCc AP.BP)-\ 

Shew that the total quantity of matter is unaffected by varying A and B so long as 

CA . CB and ACB are unaltered ; and prove that this result is equivalent to the theorem 

that the surface integral of two harmonics of different degrees taken over the sphere 

IS zero. 

(Sylvester, Phil. Mag. (5), il. (1876), pp. 291-307.) 

5. Let V (.r, y, z) be the potential function defined analytically as due to particles 
of masses X + iy., X - z/x at the points (a + ia', b + ib', c + ic') and (a — ia', b - ib', c — ic') 
respectively. Shew that V  x, y, z) is infinite at all points of a certain real circle, and 
if the point (.r, y, z) describes a circuit intertwined once with this circle the initial 
and final values of V x,y, z) are numerically equal, but opposite in sign. 

(Appell, Math. Ann. xxx. (1887), pp. 155-156.) 

6. Find the solution of Laplace's equation analytic in the region for which a<r<iA, 
it being given that on the spheres /•=a and r = A the solution reduces to 

2 c P (cos(9), 2 C;P (cos<9), 
respectively. 

7. Let 0' have coordinates (0, 0, c), and let 

Pdz=6, P0'Z=6', PO = r, PO' = r'. 
Shew that 

P  cos6') \  P  cosd ) . .s cPn i(cos ) jn + 1) Qi + 2) c'' P  +    cos d) 
( Y" f   I (n I 1) - 1 (  " ) 1  n + l) n + 2) r 'P, oo,d) 1 

"  ~  / Un + l V" / pn + 2 '  2! C""  T...|, 

according as r>c or r<c. 

Obtain a similar expansion for /"P ' (cos  ). (Trinity, 1893.) 

8. At a point (r, 6, (p) outside a uniform oblate spheroid whose semi-axes are a, b and 
whose density is p, shew that the potential is 



)a-b 



;\ \  \  m  Pj (cos 0) m  P  (cos 6) 



-...]. 



3/- 3.5 r  b.l r° 

where ni  — a--b- and ?•>? .. Obtain the i otential at points for which r<m. 

(St John's, 1899.) 



THE EQUATIONS OF MATHEMATICAL PHYSICS 401 

9. Shew that 

eirco,e=(  )i I in [271 + 1) r-lP   cos 6) J +i(r). 

n =   

(Bauer, Journal fiir Math, lvi.) 

10*. Shew that if x ±iy = h cosh (| + irj), the equation of two-dimensional wave motions 
in the coordinates   and  ? is 

a|7 + 5  =   (   h- 1 - cos2 rj) -  . (Lame.) 

IL Let X —  c + r cos 6) cos (p, y =  c + r cos 6) sin cp, 2 = /-sin ; 

shew that the surfaces for which /•, d, (f) respectively are constant form an orthogonal 
.system; and shew that Laplace's equation in the coordinates / , \&, is 

— - r(c + '/-cos )  +-;  (c + ?-cos(9)-; [+—  7., =0. 

vr \   a- j r 06 \ j- ' cd \ c + rcosdd(f>- 

(W. D. Xiven, J/essenger, x.) 

12. Let P have Cartesian coordinates  x, ?  z) and polar coordinates (;•, 6, (f)). Let 
the plane POZ meet the circle x'  + t/' = k-, s = in the points a, y; and let 

aPy = (o, log (Pa/Py) = a. 
Shew that Laplace's equation in the coordinates o-, a>, (p is 

8 f sinho- bV] d  j sinho- \  871 1  -0- 

da- [cosh o- — cos a da j ca (cosh a — cos o) 8w j sinh o- (cosh cr — cos w) 8( 2 ' 
and shew that a solution is 

V= (cosh o" — cos 0))  cos /iQ) cos ??i0 P (cosh o-). 

(Hicks, Phil. Trans. CLXXii. p. 617 et seq.) 

13. Shew that 

00  -hn i r-j- r  

 R  + p -2Rpcoscl) + c-)-h= 2 I dk I e-"" J,   I- p)e' ' ''°"' cos rauclu, 

m=0   y J -T 

and deduce an expression for the potential of a particle in terms of Bessel functions. 

14. Shew that if a, b, c are constants and X, /x, v are confocal coordinates, defined as 
the roots of the equation in e 

a2 + e' 6-  + e'  6-2 + 6 ' 
then Laplace's equation may be written 

Ax(M-) xf  + M(-X)| A/J +A.(X-M)  A. ]=0, 

where A;, = V (aHX) (i'  + X) (c-  + X) . 

(Lame.) 

* Examples 10, 11, 12 and 14 are most easily proved by using Lame's result (Journal de 
VEcole Polyt. xiv. cahier 23 (1834), pp. 191-288) that if (X, /ul, v) be orthogonal coordinates for 
which the line-element is given by the formula  dx)'' + (8yf +  5zy =(H d\ y  + (H2diM)'- +  H 5v)' , 
Laplace's equation in these coordinates is 

d fH.H cV\ 8 / H. H, dV\ 8 / H,H., dV\ \   



'Hi- 



d\ V Hi a,\ J dfjL\ H., d/jiJ di'\ Hs dv 
A simple method (due to W. Thomson, Camh. Math. Journal, iv. (1845), pp. 83-42) of proving 
this result, by means of arguments of a physical character, is reproduced by Lamb, Hydro- 
dynamics (1916), § 111. Analytical proofs, based on Lame's proof, are given by Bertrand, 
Traite de Calcul Differentielle (1864), pp. 181-187, and Goursat, Coiirs d'Analyse, i. (1910), 
pp. 155-159, the last proof being appreciably the simplest. Another proof is given by Heine, 
Theorie der Kugelfunctionen, i. (1878), pp. 303-306. 

W. M. A. 26 



402 THE TRANSCENDENTAL FUNCTIONS [CHAP. XVIII 

15. Shew that a general solution of the equation of wave motions is 

r= F  .v coH 6 +   iiin \$ + iz, >/ + iz siu 6 + ct cos B, B)dd. 

J —IT 

(Bateman, Proc. London Math. Soc. (2) l. (1904), p. 457.) 

16. If r=/(.r, y, z, t) be a solution of 

d  ct cx' cy'  dz- ' 
prove that another solution of the equation is 

17. Shew that a general solution of the equation of wave motions, when the motion is 
independent of < , is 

/ / (2 + ip cos e, ct + p sin 6) dO 

f' /""   , / a -   z + ct cos d\  ,  , ,. , 

+ arc suih —  r—   ) F (a, 6) dBda, 

J (1 .' -TT \ p sin   / ' ' 

where p, (f>, z are cylindrical coordinates and a, h are arbitrary constants. 

(Bateman, Proc. London Math. Soc. (2) i. (1904), p. 458.) 

"  18. If r=/(.r, y, z) is a solution of Laplace's equation, shew that 

Y  1  / r -a  r +gg az \ 

~( \ , )i-' V2(.'--i '  i x-iyy x-iy) 

is another solution.   

(Bateman, Proc. London Math. Soc. (2) vii. (1909), p. 77.) 

19. If U=f x, y, z, t) is a solution of the equation of wave motions, shew that 
another solution is 

rj  1 .( X y \ r2-l rHl \ 

z-ct-'Xz-ct'' z-ct"   z-cty 2c z-ct))' 

(Bateman, Proc. London Math. Soc. (2) vii. (1909), p. 77.) 

20. If l=x-iy, m = z + hv, n = . : - + y'  + z - + iv% 

\ = j:+iy, fji = z-uv, p= — \, 
so that l\ + 7nfi + nv=0, 

shew that any homogeneous solution, of degree zero, of 

cHT cH[ dH/ c U Q 
dx'  dy  cz  dw  

satisfies  +Z£+ =0- 

dldX dmdfi dndv ' 

and obtain a solution of this equation in the form 

j' , b, c \ 

U',  ', y' ) 

where   = ( -c) (f-a),. ? /Li = (c-a) (f-6), np = (a- b) ((-c). 

(Bateman, Proc. London Math. Soc. (2) vii. (1909), pp. 78-82.) 



THE EQUATIONS OF MATHEMATICAL PHYSICS 403 

21*. If (r, 6, (f)) are spheroidal coordinates, defined by the equations 

jr = c (r  + iy sin   cos< , y=c (r2-|-l) sin  sin( , z=crcosd, 

where x, y, z are rectangular coordinates and c is a constant, shew that, when n and m are 
integers, 

('"   /.  cos  + y sin -f i2\ cos ,   (n — in)\ \   ,.     ,   cos 

Pn\  —]   mtdt = 27ry—  —  P,r tr)Pn'"  cos 6) . md). 

j -n- \ c /Sin  n+m)l     "   sin   

(Blades, Proc. Edinburgh Math. Soc. xxxiii.) 
22. With the notation of example 21, shew that, if 2 =t= 0, 

/    /cPcos  + y sin  +i'2\ cos , ,  (n — m)l  , . s -r, /   cos 
Qn  - — )   intdt = 277 )--— -fj § '   ir) P '  cos e) . m4>. 
-r \ c J Sin (n+m) i  "   -' "   ' sin   

(Jeffery, Proc. Edinburgh Math. Soc. xxxiil.) 

* The functions introduced in examples 21 and 22 are known as internal and external 
spheroidal harmonics respectively. 



26—2 


%
% 404
%
\chapter{Mathieu Functions}
\Section{19}{1}{The differential equation of Mathieu.}

The preceding five chapters have been occupied with the discussion of
functions which belong to what may be generally described as the
hyper-geometric type, and many simple properties of these functions
are now well known.

In the present chapter we enter upon a region of Analysis which lies
beyond this, and which is, as yet, only very imperfectly explored.

The functions which occur in Mathematical Physics and which come next
in order of complication to functions of hypergeometric type are
called Mathieu functions; these functions are also known as the
functions associated ivith the elliptic cylinder. They arise from the
equation of two- dimensional wave motion, namely

dx dy c- dt'

This partial difterential equation occurs in the theory of the
propagation of electro- magnetic waves; if the electric vector in the
wave-front is parallel to OZ and if E denotes the electric force,
while Hx, ffy, 0) are the components of magnetic force, Maxwell's
fundamental equations are

lo \ 8 \ a c\ H \ \ dE dHy\ dE <? dt ~ dx ly ' at ~ dij ct ~ dx '

c denoting the velocity of light; and these equations give at once

c ct cx' cy' '

In the case of the scattering of waves, propagated parallel to OX,
incident on an elliptic cylinder for which OX and OY are axes of a
pi-incipal section, the boundary condition is that E should vanish at
the surface of the cylinder.

The same partial differential equation occurs in connexion with the
vibrations of a uniform plane membrane, the dependent variable being
the displacement perpendiculai- to the membrane; if the membrane be
in the shape of an ellipse with a rigid boundary, the boundary
condition is the same as in the electromagnetic problem just
discussed.

The differential equation was discussed by Mathieu* in 1868 in
connexion with the problem of vibrations of an elliptic membrane in
the following manner:

* Journal de Math. (2), xiii. (1868), p. 137.

%
% 405
%

Suppose that the membrane, which is in the plane XOY when it is

in equilibrium, is vibrating with frequency p. Then, if we write

V= u (cc, y) cos pt + e),

the equation becomes

d' u d u p- -, + -, + li = 0. cw dy c-

Let the foci of the elliptic membrane be (+ h, 0, 0), and introduce
new

real variables*, i] defined by the complex equation

cc + iy = h cosh ( + irj),

so that x = h cosh cos t, y = h sinh | sin r].

The curves, on which or r] is constant, are evidently ellipses or
hyperbolas confocal with the boundary; if we take and - tt < ?? \$
tt, to each point (w, y, 0) of the plane corresponds one and only onef
value of (, ??).

The differential equation for u transforms into;]:

af-2 + a + - (cosh- f - cos- 7?) u = 0.

If we assume a solution of this equation of the form

u = F )G v), where the factors are functions of only and of t) only
respectively, we see that

1 d'J(g) Ay )\ ( 1 d'GM i jf

Since the left-hand side contains but not ? while the right-hand side
contains ri but not, F ) and G (tj) must be such that each side is a
constant, A, say, since | and tj are independent variables.

We thus arrive at the equations

 ) + ('i!£%osl.f- )l-(f) = 0,

By a slight change of independent variable in the former equation, we
see that both of these equations are linear differential equations, of
the second order, of the form

-T-; -f (a -1- 1 65 cos 2z) u = 0,

* The iutroduction of these variables is due to Lame, who called the
thermometric parameter. They are more usually known as confocal
coordinates. See Lame, Sur les fonctions inverses des transcendantes,
1'"'' Le(?on.

+ This may be seen most easily by considering the ellipses obtained by
giving f various positive values. If the ellipse be drawn through a
definite point (t, v) ot the plane, r? is the eccentric angle of that
point on the ellipse.

t A proof of this result, due to Lame, is given in numerous text-books
; see p. 401, footnote.

%
% 406
%

where a and q are constants*. It is obvious that every point (infinity
excepted) is a regular point of this equation.

This is the equation which is known as Jfathieu's equation and, in
certain circumstances \hardsectionref{19}{2}), particular solutions of it are called
Mathieu functions.

1911. The form of the solution of Mathieu s equation.

In the physical problems which suggested Mathieu's equation, the
constant a is not given a priori, and we have to consider how it is to
be determined. It is obvious from physical considerations in the
problem of the membrane that u (x, y) is a one-valued function of j
osition, and is consequently unaltered by increasing 77 by 27r; and
the condition f G (r) + lif) = G rj) is sufficient to determine a set
of values of a, in terms of q. And it will appear later (§§ 19'4,
19*41) that, when a has not one of these values, the equation

G 'n + 2'rr)=G (v) is no longer true.

When a is thus determined, q (and thence p) is determined by the fact
that F ) = on the boundary; and so the periods of the free vibrations
of the membrane are obtained.

Other problems of Mathematical Phy.sics which involve Mathieu
functions in their solution are (i) Tidal waves in a cylindrical
vessel with an elliptic boundary, (ii) Certain forms of steady vortex
motion in an elliptic cylinder, (iii) The decay of magnetic force in a
metal cylinder . The equation also occurs in a problem of Rigid
Dynamics which is of general interest g.

\Subsection{19}{1}{2}{Hill's equation.}

A differential equation, similar to Mathieu's but of a more general
nature, arises in G. W. HiU's]] method of determining the motion of
the Lunar Perigee, and in Adams' determination of the motion of the
Lunar Node. Hill's equation is

 ' + k + 2 i cos2 ) = 0.

The theory of Hill's equation is very similar to that of Mathieu's (in
spite of the increase in generality due to the presence of the
infinite series), so the two equations will, to some extent, be
considered together.

* Their actual values are a = A - h-p-l 2c-), q = h-2)-l(S2c-); the
factor 16 is inserted to avoid powers of 2 in the solution.

t An elementary analogue of this result is that a solution of -j- +aii
= has period 2v if,

and only if, a is the square of an integer.

* K. C. Maclaurin, Trans. Camb. Phil. Soc. xvii. p. 41.

§ A. W. Young, Proc. Edinburgh Math. Soc. xxxii. p. 81.

II Acta Math. viii. (1S86). Hill's memoir was originally published in
1877 at Cambridge, U.S.A.

H Monthly Notices R.A.S. xxxviii. p. 43.

%
% 407
%

In the astronomical applications o> i>  are known constants, so the
problem of choosing them in such a way that the solution may be
periodic does not arise. The solution of Hill's equation in the Lunar
Theory is, in fact, not periodic.

\Section{19}{2}{Periodic solutions of Mathieu's equation.}

We have seen that in physical (as distinguished from astronomical)
problems the constant a in Mathieu's equation has to be chosen to be
such a function of q that the equation possesses a periodic solution.

Let this solution be G (z); then G z), in addition to being periodic,
is an integral function of z. Three possibilities arise as to the
nature of G z): (i) G (z) may be an even function of z, (ii) G z) may
be an odd function of z, (iii) G (z) may be neither even nor odd.

In case (iii), G (z) + G - z)]

is an even periodic solution and

  G z)-G -z)]

is an odd periodic solution of Mathieu's equation, these two solutions
forming a fundamental system. It is therefore sufficient to confine
our attention to periodic solutions of Mathieu's equation which are
either even or odd. These solutions, and these only, will be called 31
athieu functions.

It will be observed that, suice the roots of the indicial equation at
z = are and 1, two even (or two odd) periodic solutions of Mathieu's
equation cannot form a fundamental system. But, so far, there seems to
be no reason why Mathieu's equation, for special values of a and q,
should not have one even and one odd periodic solution; for com-
paratively small values of 1 5' ' it can be seen \hardsectionref{19}{3} example 2,
(ii) and (iii)] that Mathieu's equation has two periodic solutions
only in the trivial case in which q = 0; but for larger values o \ q\
there may be pairs of periodic solutions, though no such pairs have,
as yet, been discovered.

\Subsection{19}{2}{1}{An integral equation satisfied by even Mathieu functions*.}

It will now be shewn that, if G r ) is any even Mathieu function, then
G(r)) satisfies the homogeneous integral equation

G(v)=-\ [" e'''°' '''' G(e)dO,

where k = \/(32 ). This result is suggested by the solution of
Laplace's equation given in \hardsubsectionref{1}{8}{3}.

* This integral equation and the expansions of \hardsectionref{19}{3} were published
by Wbittaker, Proc. Int. Congress of Math. 1912. The integral equation
was known to him as early as 1904; see Trans. Camb. Phil. Soc. xxi.
(1912), p. 193.

%
% 408
%

For, if A' + i/ = h cosh, i +iri) and if F ) and G (v) are solutions
of the differential equations

 d - ( + f cosh I) F( ) = 0, ( '\ M + (A + nvJf- cos- 7;) G v) = 0,

then, by § 191, F ) G rj) e" is a particular solution of Laplace's
equation. If this solution is a special case of the general solution

/(// cosh cos 77 cos + h sinh sin 7; sin 6 + iz, 6) dO,

given in \hardsectionref{18}{3}, it is natural to expect that*

f(v,e) F 0)e' '<f>(e), where <f> (6) is a function of 6 to be
determined. Thus

F )G (v) e""' =1 F(0)(f) (6) exp mh cosh cos v cos 6

. -IT

+ mh sinh sin rjsmO + miz] dd.

Since and tj are independent, we may put = 0; and we are thus led to
consider the possibility of Mathieu's equation possessing a solution
of the form

J -77

\Subsection{19}{2}{2}{Proof that the even Mathieu functions satisfy the integral equation.}

It is readily verified \hardsubsectionref{5}{3}{1}) that, if (f> (6) be analytic in the
range (- tt, tt) and if 6 (7;) be defined by the equation

G (77) = I " e'"'' cosncose ( ) 0

then G (77) is an even periodic integral function of 7; and

- j- + (A + m-h- cos- 7;) G (7;) drf-

= [" ?u-/(2 (sin- 77 cos- + cos 77) - mh cos 77 cos + g'"'' cos >,
cose ( q

J - TV

= - n H/i sin e cos 77</) ( ) + </)' 6)] e' ''cos cos6

 < " (6') + ( + ni'h'' cos' ) (/) ( ) e"'''cos.,cosfl /

on integrating by parts.

* The constant F (0) is inserted to simplify the algebra.

%
% 409
%

But if(f) 6) be a ])eriodic function ivith period 27r) such that (f)"
(6) + A+ m h- cos 6) </> 6) = 0,

both the integral and the integrated part vanish; that is to say, G
(rj), defined by the integral, is a periodic sohition of Mathieu's
equation.

Consequently G (tj) is an even periodic solution of Mathieu's equation
if <f) (0) is a periodic solution of Mathieu's equation formed with
the same constants; and therefore (ft (6) is a constant multiple of
G (6); let it be \ G(6).

[In the case when the Mathieu equation has two periodic sokitions, if
this case exist, we have (p d) = XG (d) + Gi 6) where 6-'i 6) is an
odd periodic function; but

 mh cosv cos eg g

r

vanishes, so the subsequent work is unaffected.]

If we take a and q as the parameters of the Mathieu equation instead
of A and mh, it is obvious that mh = \/ S2q) = k.

We have thus proved that, if G 'r]) be an even periodic solution of
Mathieu's equation, then

G r]) = \ r e''-' ' °' G(0)de,

which is the result stated in \hardsubsectionref{19}{2}{1}.

From § 11 "23, it is known that this integral equation has a solution
only when X has one of the ' characteristic values.' It will be shewn
in \hardsectionref{19}{3} that for such values of \, the integral equation affords a
simple means of constructing the even Mathieu functions.

Example 1. Shew that the odd Mathieu functions satisfy the integral
equation

G i]) = \ j sin (/ sin r] sin 6) G (d) dd.

Example 2. Shew that both the even and the odd Mathieu functions
satisfy the integral equation

G,j) = x[" e ''''''' G d)de.

Example 3. Shew that when the eccentricity of the fundamental ellipse
tends to zero, the confluent form of the integral equation for the
even Mathieu functions is

J,,(.r) = /" e'-*' cose cos

ZTTi" j -

d6.

\Section{19}{3}{The construction of Mathieu functions.}

We shall now make use of the integral equation of \hardsubsectionref{19}{2}{1} to construct
Mathieu functions; the canonical form of Mathieu's equation will be
taken as

-T 4- (o + 1 6g cos z) u = 0.

%
% 410
%

In the special case when q is zero, the periodic sohitions are
obtained by taking a= n-, where n is any integer; the solutions are
then

1, cos ', cos 22, ...,

sin z, sin z,

The Mathieu functions, which reduce to these when q- 0, will be called
ceo z, q), cei (z, q), ce. z, q), . . ., se?! z, q), seo (z, q), ....
To make the functions precise, we take the coefficients of cos nz and
sin nz in the respective Fourier series for ce z, q) and sen z, q) to
be unity. The functions cen z, q), sen i, Q) will be called Mathie it
functions of oixler n. Let us now construct ce (z, q).

Since ceo(3', 0)=1, we see that A,-*.(27r)~ as (y --* 0. Accordingly
we suppose that, for general values of q, the characteristic value of
X which gives rise to ce z, q) can be expanded in the form

(27r ) -i = 1 + ttig + Oioq- + . . ., and that ce (z, q) = l+ q i z)
+ q- o ( ) + . . .,

where Oj, Oo, ... are numerical constants and x z), 13.2 (z), ... are
periodic functions of z which are independent of q and which contain
no constant term.

On substituting in the integral equation, we find that

(1 + oc,q+a,q- + ...) l+q/3, (z) -i-q', z) + ...

1 T" = - / (1 + \/(32g) . cos cos + 16r/cos- 'cos- + ...

Equating coefficients of successive powers of q in this result and
making use of the fact that i(z), /SjC ),,. .. contain no constant
term, we find in succession

Oj = 4, /3i (z) = 4 cos 2z,

oTo = 14, /Sa (z) = 2 cos 4,

and we thus obtain the following expansion:

/ -77 29 \ / 160 N

cco (z, q) = l + Uq - 28q + "- q' - ...jcos2z + i2q-- ~ q + ...j cos
4iz

+ U'f-' 5' +  ) cos 6z + ( - r/ - . . . j cos 8

the terms not written down being (q ) as -* 0.

210 99 The value of a is -S2q- + 224>q' l f/+0(r/); it will be
observed

that the coefficient of cos 2z in the series for C6?o(2, q) is
-ai(8q).

%
% 411
%

The Mathieu functions of higher order may be obtained in a similar
manner from the same integral equation and from the integral equation
of \hardsubsectionref{19}{2}{2} example 1. The consideration of the convergence of the
series thus obtained is postponed to \hardsubsectionref{19}{6}{1}.

Example 1. Obtain the following expansions*:

(i) ce, (., s, \ 1 +, J\ -. ~ \ y i ~ + (,/  .) CO, 2..,

oc I" gV ' r+l qr + l

(n) cei(2, o) = cos5+ 2 \ \ - r ~ /, i m / "ttv. r=i i(?'+l)!?*! (r +
1)! (r+1)!

(ni) TODO

(iv) TODO

where, in each case, the constant implied in the symbol depends on r
but not on z.

\addexamplecitation{Whittaker.}

Example 2. Shew that the values of a associated with (i) ceo(5, </),
(ii) cei z, q), (iii) sfij (2, 5'), (iv) ce2 z, q) are respectively:

210 2Q (i) -32(?2 + 224(?4 \ \, 6 + ((? ),

(ii) l-8q- + -\ qi + 0 q%

(iii) l + 8j-8j2\ 823- <?* + 0(j5

(iv) 4 + j2- g* + 0(? ). \addexamplecitation{Mathieu.}

Example 3. Shew that, if n be an integer,

\Subsection{19}{3}{1}{The integral formulae for the Mathieu functions.}

Since all the Mathieu functions satisfy a homogeneous integral
equation with a symmetrical nucleus \hardsubsectionref{19}{2}{2} example 3), it follows (§
11'61) that

cem z, q) cen (z, q)dz = (m n),

.' - jr

sem (z, q) sen (z, q)dz = m i= n)

T

cem z, q) sen (z, q)dz = 0.

T

* The leading terms of these series, as given in example 4 at the end
of the chapter (p. 427), were obtained by Mathieu.

%
% 412
%

Example 1. Obtain expansions of the form:

(ii) cos (/ sin z sin ) = 2 B ce (z, q) ce 6, q),

?l=0

(iii) sin (/ sin z sin 6)= 2 C se (2, q) se 6, q), where i=, 32q).

Example 2. Obtain the expansion

)! = - 3C

as a confluent form of expansions (ii) and (iii) of example 1.

\Section{19}{4}{The nature of the solution of Mathieu s general equation; Floquet's theory.}

We shall now discuss the nature of the solution of Mathieu's equation
when the parameter a is no longer restricted so as to give rise to
periodic solutions; this is the case which is of importance in
astronomical problems, as distinguished from other ph -sical
applications of the theory.

The method is applicable to any linear equation with periodic
coefficients which are one-valued functions of the independent
variable; the nature of the general solution of particular equations
of this type has long been perceived by astronomers, by inference
from the circumstances in which the equations arise. These inferences
have been confirmed by the following analytical investigation which
was published in 1883 by Floquet*.

Let g z), h (z) be a fundamental system of solutions of Mathieu's
equation (or, indeed, of any linear equation in which the coefficients
have period 27r); then, if F(z) be any other integral of such an
equation, we must have

F z)=Ag(z)+Bh(z), where A and B are definite constants.

Since g z+ 27r), h (z -f 27r) are obviously solutions of the
equationf, they can be expressed in terms of the continuations of g
(z) and Ji (z) by equations of the type

g(z + 27r) = a g (z) + a,h (z), h z Itt) =,g z) + 0,h (z), where ttj,
a.>, /S, /?.\ > are definite constants; and then

F z + 27r) = (Aa + B,) g z) + Aoi., + B/3,) h (z).

* Ann. de VEcole norm. sitj). (2), xii. (1883), p. 47. Floquet's
analysis is a natural sequel to Picard's theory of differential
equations with doubly-periodic coefficients \hardsectionref{20}{1}), and to the
theory of the fundamental equation due to Fuchs and Hamburger.

t These solutions may not be identical with (j(z), h(z) respectively,
as the solution of an equation with periodic coefficients is not
necessarily periodic. To take a simple case, H = e sin z

is a solution of -r - (1 + cot ) 1/ = 0.

dz '

%
% 413
%

Consequently F z + 27r)= kF(z), where k is a constant*, if A and B are
chosen so that

A a, + BI3, = hA, A a, + B/3o = kB.

These equations will have a solution, other than A = B = 0, if, and
only if,

oc,-k, A =0;

ofo, /5o - A.-

and i k be taken to be either root of this equation, the function F(2)
can be constructed so as to be a solution of the differential equation
such that

F(z+27r) = kF(z).

Defining fi by the equation k= e-'" and writing ( z) for e~' F z), we
see that

(f>(z + 27r) = €-''''+-''> F z+-2tt)=( (z).

Hence the differential equation has a particular solution of the form
e' (f) z), where (f)(z) is a periodic function with period 27r.

We have seen that in physical problems, the jjarameters involved in
the differential equation have to be so chosen that k = l is a root of
the quadratic, and a solution is periodic. In general, however, in
astronomical problems, in which the parameters are given, A- 1 and
there is no periodic solution.

In the particular case of Mathieu's general equation or Hill's
equation, a fundamental system of solutions f is then e' -(f)(z), e~>
-z), since the equation is unaltered by writing - r for; so that the
complete solution of Mathieu's general equation is then

u = Cie'*-(f) z) + Coe~i ( (- z),

where Ci, c, are arbitrary constants, and /i is a definite function of
a and q.

Example. Shew that the roots of the equation

a,-k,,3i =0 ao, )io - k are independent of the particular pair of
solutions, g z) and h (z), chosen.

\Subsection{19}{4}{1}{Hill's method of solution.}

Now that the general functional character of the solution of equations
with periodic coefficients has been found by Floquet's theory, it
might be expected that the determination of an explicit expression for
the solutions of Mathieu's and Hill's equations would be a
comparatively easy matter; this however is not the case. For example,
in the particular case of Mathieu's general equation, a solution has
to be obtained in the form

y = e (f) (z),

* The symbol k is used in this particular sense only in this section.
It must not be confused with the constant A; of \hardsubsectionref{19}{2}{1}, which was
associated with the parameter q of Mathieu's equation. t The ratio of
these solutions is not even periodic; still less is it a constant.

%
% 414
%

where <f) (z) is periodic and /i is a function of the parameters a and
q. The crux of the problem is to determine /j.; when this is done,
the determination of (f) (2) presents comparatively little difficulty.

The first successful method of attacking the problem was published by
Hill in the memoir cited in \hardsubsectionref{19}{1}{2}; since the method for Hill's
equation is no more difficult than for the special case of Mathieu's
general equation, we shall discuss the case of Hill's equation, viz.

where J (z) is an even function of z with period tt. Two cases are of
interest, the analysis being the same in each:

(I) The astronomical case when z is real and, for real values of z, J
(z) can be expanded in the form

J(z) = 00 + 2 1 cos 2z + 202 cos 4>z + 26 cos 6 + . . .; the
coefficients 6n are known constants and S 6n converges absolutely.

n=0

(II) The case when is a complex variable and J (2) is analytic in a
strip of the plane (containing the real axis), whose sides are
parallel to the

real axis. The expansion of J(z) in the Fourier series 6 + 2 "Z 6n cos
2nz

H = l

is then valid \hardsubsectionref{9}{1}{1}) throughout the interior of the strip, and, as
before,

00

2 On converges absolutely. =o

Defining \ to be equal to 6n, we assume

00

71= -OC

as a solution of Hill's equation.

[In case (II) this is the solution analytic in the strip (§> 10-2,
19'4); in case (I) it will have to be shewn ultimately (see the note
at the end of \hardsubsectionref{19}{4}{2}) that the values of 6

which will be determined are such as to make 2 n-bn absolutely
convergent, in order to

n= -

justify the processes which we shall now carry out.] On substitution
in the equation, we find

M=- 00 \ n=-x / j=- 00 /

Multiplying out the absolutely convergent series and equating
coefficients of powers of e ' to zero (§§ 9"6-9"632), we obtain the
system of equations

(,i + 2niyb + i e X\,, = (n = ..., -2,-1,0,1,2, ...).

%
% 415
%

If we eliminate the coefficients bn determinantally (after dividing
the typical equation by 6o - 4 n" to secure convergence) we obtain*
Hill's determinantal equation:

( >+4) -

io

- >

-00

-0, 42- 0

-0,

- 4 -0,

4- 0

4- -00

42- 0 "'

-e.

0> + 2)2-, 2- -00

-0,

 2' -Bo

-02

2- -00

- 3

'" 22-,,

2- -00 -

-Oo

-6,

W-do

02 -do

-0, 0- -00

-do

  02 -. 0

0- - 6 0

02 -do -

-03

-6-2 22- 0

-0

22 - o

(z>-2)2- 22- 0

00

-01

" 2 -60

22- do -

-0,

- 3

42-,,

-02

i--0o

- 1 42- 0

 ±

-4)2 -do 42 -do -

=0.

We write A ifx) for the determinant, so the equation determining yu.
is

A (i = 0.

\Subsection{19}{4}{2}{The evaluation of Hill's determinant.}

We shall now obtain an extremely simple expression for Hill's deter-
minant, namely

A iix) = A (0) - sin- ( tti/x) cosec- (|-7r V o)-

Adopting the notation of \hardsectionref{2}{8}, we write

A(i =[,,,J, (t'/i - 2m)- - Oq

where,, j =

4?7i- - dr,

A m-n

4m2 - 0Q

(m n).

The determinant [,,ii] is only conditionally convergent, since the
product of the principal diagonal elements does not converge
absolutely (§§ 2"81, 2'7). We can, however, obtain an absolutely
convergent determinant, Aj (i/x), by dividing the linear equations of \hardsubsectionref{19}{4}{1} by 6q- (i/x- 2n)- instead of dividing by (, - 4/1-. We write
this determinant Ai(2ju,) in the form [5, \ ], where

  m,ra - -) - j/i, n - /.

-Or.

(m n).

 2in - i/x)-- Hf)

The absolute convergence of S 6,1 secures the convergence of the
deter-

minant [-S,, ], except when /x has such a value that the denominator
of one of the expressions B n vanishes.

* Since the coefl\&cients 6,j are not all zero, we may obtain the
infinite determinant as the eliminant of the system of linear
equations by multiplying these equations by suitably chosen cofactors
and adding up.

%
% 416
%

From the definition of an infinite determinant \hardsectionref{2}{8}) it follows that

sin TT (?> - V o) sin tt (i> + \/Oo) and so A (i ) = - A, (i ) -
smM*7rV ) '

Now, if the determinant A, i/j,) be written out in full, it is easy to
see (i) that Ai (ifi) is an even periodic function of /x with period
2tV(ii) that Aj / ) is an analytic function (cf. §§ 2'81, 8"34, 5'3)
of yu, (except at its obvious simple poles), which tends to unity as
the real part of /j, tends to ± oo .

If now we choose the constant K so that the function D (/a), defined
by the equation

D (ijl) = Ai (i"/u.) - K [cot ir ifi, + \/0(,) - cot i TT (ifj, - V o)
,

has no pole at the point ij, = i 6, then, since D /j,) is an even
periodic function of yu., it follows that D (/x) has no pole at any of
the points

2ni ± i V o) where n is any integer.

The function D (/j.) is therefore a periodic function of /n (with
period 2 ) which has no poles, and which is obviously bounded as R
(//.) + x . The conditions postulated in Liouville's theorem \hardsubsectionref{5}{6}{3})
are satisfied, and so D (fi) is a constant; making / - + go, we see
that this constant is unity.

Therefore

Ai (ifi) = 1+K cot Itt ifi + \/d,) - cot -h TT ifi - V(9o), and so

sinl7r(t>- V o)sin 7r(i>+\/ o), o7- wi ia\ A rfi) = sinM -V o) " - '
'"' " " -

To determine K, put /a =; then

A(0) = l + 2/i:cot(i7rv/ o)- Hence, on subtraction,

A(, = MO)-| li

which is the result stated.

The roots of Hill's determinantal equation are therefore the roots of
the equation

sin -nifi) = A (0) . sin ( tt V o)-

When fj, has thus been determined, the coefficients bn can be
determined in terms of b, and cofactors of A ifi); and the solution
of Hill's differential equation is complete.

%
% 417
%

[In case (I) of \hardsubsectionref{19}{4}{1}, the convergence of 2 | 6 | follows from the
rearrangement theorem of \hardsubsectionref{2}{8}{2}; for 2 2 1 6 | is equal to | 6o | 2 |
C j, o I - i o, o I where C, n is the cofactor of B,

ni= - x>

in Ai (ifi.)', and 2 | C i,o I is the determinant obtained by
replacing the elements of the row through the origin by numbers whose
moduli are bounded.]

It was shewn by Hill that, for the purposes of his astronomical
problem, a remarkably good approximation to the value of fj. could be
obtained by considering only the three central rows and columns of his
determinant.

\Section{19}{5}{The Lindemann-Stieltjes' theory of Mathieu's general equation.}

Up to the present, Mathieu's equation has been treated as a linear
differential equation with periodic coefficients. Some extremely
interesting properties of the equation have been obtained by
Lindemann* by the substitution =cos, Avhich transforms the equation
into an equation with rational coefficients, namely

4 (1 - O, + 2 (1 - 20 + (a -I6q + 32 0 = 0.

This equation, though it somewhat resembles the hypergeometric
equation, is of higher type than the equations dealt with in Chapters
xiv and xvi, inasmuch as it has two regular singularities at and 1 and
an irregular singularity at x; whereas the three singularities of the
hypergeometric equation are all regular, while tlie equation for TFj.\
(3) has one irregular singularity and only one regular singularity.

We shall now give a short account of Linderaann's analysis, with some
modifications due to Stieltjesf.

\Subsection{19}{5}{1}{Lindemann' s form of Floquet's theorem.}

Since Mathieu's equation (in Lindemann's form) has singularities at =
and = 1, the exponents at each being 0, \, there exist solutions of
the form

W=0 M=0

2 o = i an (1 - y\ u = (1 - 0* i n (1 - KT;

>i = M =

the first two series converge when ] j < 1, the last two when 1 1 - |
< 1.

When the -plane is cut along the real axis from 1 to + x and from to -
00, the four functions defined by these series are one-valued in the
cut plane; and so relations of the form

Vw = ay 00 + Voi, Vn = 73/00 + i/oi will exist throughout the cut
plane.

Now suppose that describes a closed circuit round the origin, so that
the circuit crosses the cut from - oo to; the analytic continuation
of 3/10 is

* Math. Ann. xxii. (1883), p. 117.

t Astr. Nach. cix. (1884), cols. 145-152, 261-266. The analysis is
very similar to that employed by Hermite in his lectures at the Ecole
Polytechnique in 1872-1873 [Oeuvres, iii. (Paris, 1912), pp. 118-122]
in connexion with Lame's equation. See \hardsectionref{23}{7}.

W. M. A. 27

%
% 418
%

oj/oo - /3yoi (since l/ is unaffected by the description of the
circuit, but ?/oi changes sign) and the continuation of j/u is 7 00 -
j/oi; "c? so Ay - + By - will he unaffected by the description of the
circuit if

A ay,o + ySyoi)' + B yy + SyoO' s A ay - /3?/oi)' + B (73/00 - S oi)-,

i.e. if Aa + ByS = 0.

Also Ay f-h Byii- obviously has not a branch-point at f=l, and so, if
Aal3 + By8 - 0, this function has no branch-points at or 1, and, as it
has no other possible singularities in the finite part of the plane,
it must be an integral function of .

The two expressions

A y,o + iB -yn, -j/io - iB 'l/u are consequently two solutions of
Mathieu's equation whose product is an integral function of .

[This amounts to the fact \hardsectionref{19}{4}) that the product of ef" ( > (z) and
e~' ~ (- z) is a periodic integral function of z.']

\Subsection{19}{5}{2}{The determination of the integral function
  associated with Mathieu's general equation.}

The integral function F(z) = Ay o" + By, just introduced, can be
deter- mined without difficulty; for, if jo and y are any solutions
of

:+p(n|+Q(r) =o,

their squares (and consequently any linear combination of their
squares) satisfy the equation*

 ! + 3P (D + [P' (0 + 4Q (0 + 2 [P (or ]

in the case under consideration, this result reduces to

+ (a-l-l6q + S2q ) J - 16qF (f) = 0.

X

Let the Maclaurin series for F ! ) be 1 c,i "; on substitution, we
easily obtain the recurrence formula for the coefficients c, namely

where

(n + I) (n - ly - a + 16q] \ \ n (n -h l)(2n + l)

""" ieq(2n + l) ' ''" S2q 2n-1) "

* Appell, Comptes Rendus, xci. (1880), pp. 211-214; cf. example 10,
p. 298 supra.

%
% 419
%

At first sight, it appears from the recurrence formula that Co and Ci
can be chosen arbitrarily, and the remaining coefficients C2, C3, ...
calculated in terms of them; but the third order equation has a
singularity at t= 1) nd the series thus obtained would have only unit
radius of convergence. It is necessary to choose the value of the
ratio Ci/Cq so that the series may con- verge for all values of .

The recurrence formula, when written in the form

suggests the consideration of the infinite continued firaction

Uu+'

V,

W2

n+i 1" W j o +

lim 1*, +

nn+i + ...+

The continued fraction on the right can be -sNTitten* w,i/r (n, n +
m)IK (n + 1, n + m),

where K (n, n + m) =

1

- u

-1

- Uni->, 1

The limit of this, as ??i - x, is a convergent determinant of von
Koch's type (by the example of \hardsubsectionref{2}{8}{2}); and since

Vr+i

llrUr+l

as n - 00,

it is easily seen that K (n, x ) 1 as 7i -* x .

Cn Un K n, X )

Therefore, if

Cn+1 K n + 1, co)' then Cn satisfies the recurrence formula and, since
Cn i/Cn - as ?i - x, the resulting series for F ( ) is an integral
function. From the recurrence formula it is obvious that all the
coefficients c are finite, since they are finite when n is
sufficiently large. The construction of the integral function F ( )
has therefore been effected.

\Subsection{19}{5}{3}{The solution of Mathieus equation in terms of F( ).}
If Wi and
Wo be two particular solutions of

g+P(r)|+(3(f)"=o,

thenf

W Wi - W1IV2

r=cexp|-j P(r)fzr >

* Sylvester, Phil. Mag. (4), v. (1S53), p. 446 [Math. Papers, i. p.
609].

t Abel, Journal fiir Math. ii. (1827), p. 22. Primes denote
differentiations with regard to f.

97 9

%
% 420
%

where is a definite constant. Taking iv and w to be those two
solutions of Mathieu's general equation whose product is -P( ), we
have w w. C w/ w,' r( )

W, Wo f (l- t)*i (0' '"'' '2 iO' the latter following at once from the
equation tv iu. Fi ).

Solving these equations for iv lic, and tuJ/wo, and then integrating,
we at once get

where 71, y.. are constants of integration; obviously no real
generality is lost by taking Cq = 71 = 72 = 1-

From the former result we have, for small values of | |,

while, in the notation of § 1 "51, we have aJao = - a+ Sq.

Hence C = I69 - a - c .

This equation determines C in terms of a, q and Cj, the value of Ci
being

K(l, cc) uoK(0, x) .

Example 1. If the solutions of Mathieu's equation be e' ' (p ±z),
where </> (s) is periodic, shew that

Example 2. Shew that the zeros oi F C) are all simple, unless (7=0.

\addexamplecitation{Stieltjes.}

[If F () could have a re jeated zero, v and ivo would then have an
essential singularity.]

\Section{19}{6}{A second method of constructing the Mathieu function.}

So far, it has been assumed that all the various series of \hardsectionref{19}{3}
involved in the expressions for cey(2, q) and sey(z, q) are
convergent. It will noiv be sheton that ce z, q) and scy z, q) are
integral functions of z and that the coefficients in their expansions
as Fourier series are power series in q which converge absolutely when
\ q\ is sufficiently small*.

To obtain this result for the functions ce2f z, q), we shall shew how
to determine a particular integral of the equation

- + (a + \ Qq cos 2z) u = y\ r a, q) cos Nz

* The essential part of this theorem is the proof of the convergence
of the series which occur in the coefficients; it is already known §§
10'2, 10-21) that solutions of Mathieu's equation are integral
functions of z, and (in the case of periodic solutions) the existence
of the Fourier expansion follows from \hardsubsectionref{9}{1}{1}.

%
% 421
%

in the form of a Fourier series converging over the whole 2 -plane,
where yjr (a, q) is a function of the parameters a and q. The equation
-v/r (a, q) = then determines a relation between a and q which gives
rise to a Mathieu function. The reader who is acquainted with the
method of Frobenius* as applied to the solution of linear differential
equations in power series will recognise the resemblance of the
following analysis to his work.

Write a = iY + 8p, where JSf is zero or a positive or negative
integer.

Mathieu's equation becomes

 \ +:N'Hi = -S (p + 2q cos 2z) 11.

If jj and q are neglected, a solution of this equation is u - cos Nz=
Uo )> say.

To obtain a closer approximation, write -8(p + 2q cos 2z) ITq (z) as a
sum of cosines, i.e. in the form

- 8 cos X-2)z+p cos Nz + q cos (N + 2) z] = Fj z), say.

Then, instead of solving -r-- + X'-u = V z), suppress the terms f in V
z)

which involve cos Nz; i.e. consider the function W z) wherej

If, ( )=F,( ) + 8; cos iY . A particular integral of

,+NHl=W, z)

18

11 = 9.

iro iT) ' ' ( - 2) + roTiT) '°' ' + 1 = ' ' ' '

Now express -S(p + 2q cos 2z) L\ (z) as a sum of cosines; calling
this sum Vo (z), choose a. to be such a function of p and q that V (z)
+ a . cos Nz contains no term in cos Nz; and let V., z) + a., cos Nz
= W. z).

cP u Solve the equation -r-j + N- u = Wo z),

and continue the process. Three sets of functions Um z), Vm z), Wm z)
are thus obtained, such that U,n z) and W,n z) contain no term in cos
Nz when m 0, and

W z) = F, z) + a, cos Nz, F, ( ) = - 8 p + 2q cos 2z) U,n-i (z),

 J + NL ( z) = W, z), where, is a function of p and q hut not of z.

* Journal fiir Math, lxxvi. (1873), pp. 214-224.

, d-u -, t The reason for this suppression is that the particular
integral of + A'- = cosA

contains non-periodic terms.

+ Unless N = \, in which case \ \ \ \ {z)-1\ \ {z) + 9 i) + q) co%z.

%
% 422
%

It follows that

\ az ) j=o /=i

n-i / n \

= - 8 (;9 + 25 COS 2ir) S [7',rt\ i ( ) + S a, ) cos Nz.

Therefore, if U z)= S f/', ( ' be a uniformly convergent series of
analytic

m=0

functions throughout a two-dimensional region in the -plane, we have

\hardsectionref{5}{3})

d?U(z

-7-2 + (' + 9. cos 'Iz) U (z) = ylr (a, q) cos Nz,

oc

Avhere fr a,q)= 0 .

It is obvious that, if a be so chosen that yjf (a, q) = 0, then U z)
reduces to cey z).

A similar process can obviously be carried out for the functions 5e y
z, q) by making use of sines of multiples of z.

\Subsection{19}{6}{1}{The convergence of the series defining Mathieu functions.}

We shall now examine the expansion of \hardsectionref{19}{6} more closely, with a view
to investigating the convergence of the series involved.

When n" 1, we may obviously write

/! n

U-a\ Z)= 2 */3,,.cos(iy-2r)£-|- 2 a rCOfi N->r'2.r)z, r=l r=l

the asterisk denoting that the first summation ceases at the greatest
value of r for which r N.

 (12 1

-jpi+ \ n+i (-) = an + 1 cos Nz -8 p + 2q cos 22) £/" (2),

it follows on equating coefficients of cos (iV + 2r) z on each side of
the equation t that

0.1 + 1 = !? ("h,1+3,i),

/(>- + .y)a +,,, = 2 /?a,, + j(a, r-i + a,r + i) (''=1,2, ...),

These formulae hold universally with the following conventions %:

(i) Vo = 3 .o = ( = 1,2,...); a,. =, = (;> ),

(ii) j, . j = iv-i li6i' - i* 'en and r=|i, (i") n.H-V+D ' n.H V-i)
en .V is odd and r=h N- ) . t When A'=0 or 1 these equations must be
modified by the suppression of all the coefficients * The conventions
(ii) and (iii) are due to the fact that cos2; = cos (-2), cos 22 = cos
(- 22).

%
% 423
%

The reader will easily obtain the following special formulae:

(I) a = 8p, (iV= l); ai = 8ip + q), (iV=l),

(III) a,y and n,r homogeneous polynomials of degree n in p and q.

we have >/ (a, j) = 8jt? + 8y (Ji+5,) (iV- D,

rir + ]V)A,=2ipAr+q(Ar-, + A, i)] (A),

r(r-.V) B, = 2 pB,+q B,\, + Br i) (B),

where Ao=B = 1 and B,. is subject to conventions due to (ii) and (iii)
above. Now write w,.= -q r r + ]V)-2p -\ '/= -q r (r-y)-2p -\ The
result of eliminating Ai, Ao, ... Aj.-, A + i, ... from the set of
equations (A) is

where A,, is the infinite determinant of von Koch's type \hardsubsectionref{2}{8}{2})

A,.= 1, il'r+u 0,0,....

, Wr+3, 1, W +3, ...

The determinant converges absolutely \hardsubsectionref{2}{8}{2} example) if no
denominator vanishes; and Ar-*-l as r- -cc (cf.\hardsubsectionref{19}{5}{2}). If p and q
be given such values that Ao fcO, 2p r r- N), where r = l, 2, 3, ...,
the series

2 - yiVx%o.2...Wj.Ar (r cos(iV+2r) z

r=\

represents an integral function of z.

In like manner B,.D(,= -Y Wi tc ...w ' J),., where D,. is the finite
determinant

1, w'r + i,, ... J,

w'r + 2, 1, w'r + 2,  !

the last row being 0, 0, ... 0, 2w\ \, 1 or 0, 0, ... 0, <''i(jvr-i))
l + '''i(,v\ i) according as N is even or odd.

The series 2 Un (z) is therefore

?t=0

COS V + Ao" 2 (-)'"?<?i?i'2...?<'rArCOS(iV+2r)2 r = l

+ D- 1 2 ( - )' iv; 10,' . . . w,' I), cos (iY- 2;-) z,

these series converging uniformly in any bounded domain of values of
z, so that term-by- term differentiations are permissible.

Further, the condition yj/ (, g') = is equivalent to

If we multiply by

pAo o-q w- Ai Do + Wi'Di Aq) = 0.

%
% 424
%

the expression on the left becomes an integral function of both p and
q, (a, q), h-a\; the terms of (a, q\ which are of lowest degrees.in p
and q, are respectively p and

 ow expand - -. . ..o, o c - - 5 - - - ap

in ascending powers of q (cf.\hardsubsectionref{7}{3}{1}), the contour being a small
cii'cle in the p-plane, with centre at the origin, and | q I being so
small that (iV + Sjo, q) has only one zero inside the contour. Then it
follows, just as in \hardsubsectionref{7}{3}{1}, that, for sufficiently small values of \
q\, we may expand p as a power series in q commencing* with a term in
q; and if | q be sufficiently small D and Aq will not vanish, since
both are equal to 1 when =0.

On substituting for p in terms of q throughout the series for U (z),
we see that the series involved in cex (s, q) are absolutely
convergent when | | is sufficiently small.

The series involved in se (2, q) may obviously be investigated in a
similar manner.

\Section{19}{7}{The method of change of parameter.}

The methods of Hill and of Lindemann-Stieltjes are effective in
determining, but only after elaborate analysis. Such analysis is
inevitable, as is by no means a simple function of q; this may be
seen by giving q an assigned real value and making a vary from - C30
to + 00; then /x alternates between real and complex values, the
changes taking place when, with the Hill-]\ Iathieu notation, A (0)
sin- [hir s, a) passes through the values and 1; the complicated
nature of this condition is due to the fact that A (0) is an elaborate
expression involving both a and q.

It is, however, possible to express fj. and a in terms of q and of a
new parameter tr, and

the results are very well adapted for purposes of numerical
computation when | 5' | is small J.

The introduction of the parameter a- is suggested by the series for
cei z, q) and sey z, q)

given in \hardsectionref{19}{3} example 1; a consideration of the.se series leads us
to investigate the

potentialities of a solution of Mathieu's general equation in the form
y=e' ' 0(s), where

  if) = sin z-(r) + a-i cos (82 - o-) + 63 sin (82 - cr) + 05 cos hz -
cr) + 65 sin (52 - tr) + . . ., the parameter o- being rendered
definite by the fact that no term in cos z - a) is to appear in (j)
z); the special functions 5 1(2, q), cei z, q) are the cases of this
solution in which o- is or \ ir.

On substituting this expression in Mathieu's equation, the reader will
have no difficulty in obtaining the following approximations, valid
for § small viilues of q and real values of cr:

  =4 ' sin 2o-- 12 -" sin 2o-- 12j*sin 4o- + (2" ),

a = 1+ 8y cos 2(7 + ( - 16 + 8 cos 4o-) 2 \ i cos 2o- + ( f -- 88 cos
4o-) q + O q% a3=Sq sin 2a- + '3q sin 4:a + - sin 2a- + 9 sin 6a) q +
q% h =q + q cos 2o- + ( - J f + 5 cos 4(r) j + ( \ IJ. cos 2o- + 7 cos
60-) </* + q% a- = Y? sin 2o-4-|f ? sin 4o- + q% h=W + j q cos 2(7 + (
- Vr + f f cos Aa) q +0 q->), 7 = f'ifs 9* sin 2a + (q ), 67 = q + (/*
cos 2<t+0 iq% a, = 0 q% h, = l,,q + 0 q% the constants involved in the
various functions 0 q ) depending on a-.

* If A = l this result has to be modified, since there is an
additional term q on the right and the term q jiN - 1) does not
appear. .

t Wbittaker, Proc. Edinburgh Math. Soc. xxxn. (1914), pp. 75-80.

X They have been applied to Hill's problem by luce, Monthly Notices of
the R. A. S. lxxv. (1915), pp. 436-448.

§ The parameters q and a are to be regarded as fundamental in this
analysis, instead of a and q as hitherto.

%
% 425
%

The domains of values of q and o- for which these series converge have
not yet been determined*.

If the sokition thus obtained be called A z, a, q), then A (z, cr, q)
and A z, - o", q) form a fundamental system of solutions of Mathieu's
general equation if /x= 0.

Example 1. Shew that, if o- = / x 0'5 and = 0-01, then

a = M24,841,4..., /x = ?x 0-046,993,5 ...; shew also that, if (r = i
and 2' = 0'01, then

rt = l-,321, 169,3..., / = ix 0-145,027,6.... Example 2. Obtain the
equations

/x = 4 sin 2a- - 4 ja3, rt - 1 + 83' cos 2(r - /it- - 8363,
expressing n and a in finite terms as functions of q, a, a and 63.
Example 3. Obtain the recurrence formulae -4:n n+l) +
8qcos2a--8qb3±8qi 2n+l) as-sin2(r) z.2n + i + 8q z2 \ i+Z2 3) = 0,

where 22jh-i denotes bon+i + ict-in + i oi" -in + i - i 2)i + i>
according as the upper or lower sign is taken.

\Section{19}{8}{The asymptotic solution of Mathieu's equation.}

If in Mathieu's equation

d' v. / 1,., \

-5- + a + - A,- cos 22 I M =

dz- \ 2 J

we write k sin 2 =, we get

where i/ = + P.

This equation has an irregular singularity at infinity. From its
resemblance to Bessel's equation, we are led to write u - e' |~- v,
and substitute

V=l+ a,/ ) + a,!e) + ... in the resulting equation for v; we then
find that

ai = - i (i - 3P + F), a., = - Hi - + ') (f - - H F) + IF, the
general coefficient being given by the recurrence formula

2i(r+l)a, + i = J-J/2 + F + /-(r+l) + (2?--l)zFa \ l-(/-2-2/- + |)Fa,\
2. The two series

e'U~-(l+ + jl+-], e-''r'(l- +

|,p,...), .-.,-.,. . \

are formal solutions of Mathieu's equation, reducing to the well-known
asymptotic solutions of Bessel's equation \hardsectionref{17}{5}) when -- 0. The
complete formulae which connect them with the solutions e ' (f)(±z)
have not yet been published, though some steps towards obtaining them
have been made by Dougall, F7'oc. Edinburgh Math. Soc. xxxiv. (1916),
pp. 176-196.

* It seems highly probable that, if | g | is sufficiently small, the
series converge for all real values of a, and also for complex values
of cr for which |I((r) | is sufficiently small. It may be noticed
that, when q is real, real and purely imaginary values of cr
correspond respectively to real and purely imaginary values of fi.

%
% 426
%

KEFEREXCES.

E. L. Mathiec, Journal de Math. (2), xiii. (1868), pp. 137-203.

G. W. Hill, Acta Mathematica, viii. (1886), pp. 1-36.

G. Floquet, Ann. de VEcole norm. sup. (2), xii. (1883), pp. 47-88.

C. L. F. LiNDEMANN, Mcith. Ann. xxii. (1883), pp. 117-123.

T. J. Stieltjes, Astr. Naeh. cix. (1884), cols. 145-152, 261-266.

A. LiXDSTEDT, Astr. Nach. cm. (1882), cols. 211-220, 257-268; Civ.
(1883), cols. 145-150; cv. (1883), cols. 97-112.

H. Bruns, Astr. Xach. cvi. (1883), cols. 193-204; cvii. (1884), cols.
129-132.

R. C. Maclaurin, Trans. Camb. Phil. Soc. xvii. (1899), pp. 41-108.

K. AiCHi, Proc. Tokyo Math, and Phys. Soc. (2), iv. (1908), pp.
266-278.

E. T. Whittaker, Proc. International Congress of Mathematicians,
Cambridge, 1912, I. pp. 366-371.

E. T. Whittaker, Proc. Edinburgh Math. Soc. xxxii. (1914), pp. 75-80.

G. N, Watson, Proc. Edinburgh Math. Soc. xxxiii. (1915), pp. 25-30.

A. W. Young, Proc. Edinburgh Math. Soc. xxxii. (1914), pp. 81-90.

E. Lindsay Inge, Proc. Edinburgh Math. Soc. xxxiii. (1915), pp. 2-15.

J. Dougall, Proc. Edinburgh Math. Soc. xxxiv. (1916), pp. 176-196.

Miscellaneous Examples.

1. Shew that, if k= l Z \

2wce(, s, q) = cco 0,q) j cos k sin z sin 6) ccq (0, q) d6. J -It

2. Shew that the even Mathieu functions satisfy the integral equation

G (2)=X j Jo [ik (cos z + cos 6) G 6) d6.

3. Shew that the equation

(a22 + c) +2a5 + (X%2+ 0 =

(where o, c, X, m are constants) is satisfied by

u = \ \ v s)ds tAken round an appropriate contour, provided that v (s)
satisfies

 as + c) + 2as - X cs - + m)v 3) = 0,

which is the same as the equation for u.

Derive the integral equations satisfied by the Mathieu functions as
particular cases of this result.

%
% 427
%

4. Shew that, if powers of q above the fourth are neglected, then

ce?! (s, q) = cos 2 + J cos 32 + q (J cos hz - cos 3s)

+ ( (i 8 cos 7s - f cos 5s + J cos 3s)

+ ?* (rl < ~ I's 0* ''' + H cos 5s + cos 3s),

sei (s, j) = sin s + g sin 3s + §'2 (i sin 5s + sin 3s)

+ <f ( jij sin 7s + f sin 5s + A sin 3s)

+ q (yIcj sin 9s + Jj sin 7s + sin 5s - - sin 3s),

C(?2 (s, g-) = cos 2s + g- (cos 4s - 2) + g cog g

+ ? (-/s cos 8s + If cos 4s + -* )

+ ?* (sTO cos 10s + f|§ cos 6s).

\addexamplecitation{Mathieu.}

5. Shew that

663(0, 2') = cos 32 + 2'( - coss+l cos5s)

+ j2 (cos s + J(5 COS 7s) + j3 ( \ I COS s + - COS 5s + Jq cos 9s) +
q*),

and that, in the case of this function

a = 9 + 4q -8q +0 q ).

\addexamplecitation{Mathieu.}

6. Shew that, if 1/ (s) be a Mathieu function, then a second sokition
of the corresponding differential equation is

Shew that a second solution * of the equation for ce (s, q) is zceQ
(z, q) - 4:qsiB 2s- Sq- sin 4s- ....

7. If ?/ (2) be a solution of Mathieu's general equation, shew that

 y(s + 2 )+3/(s-2:r) /j/(2) is constant.

8. Express the Mathieu functions as series of Bessel functions in
which the coefficients are multiples of the coefficients in the
Fourier series for the Mathieu functions.

[Substitute the Fourier series under the integral sign in the integral
equations of \hardsubsectionref{19}{2}{2}.]

9. Shew that the confluent form of the equations for ce (s, q) and se
(2, q), when the eccentricity of the fundamental ellipse tends to
zero, is, in each case, the equation satisfied by J,, ii- eoH z).

10. Obtain the parabolic cylinder functions of Chapter xvi as
confluent forms of the Mathieu functions, by making the eccentricity
of the fundamental ellipse tend to unity.

11. Shew that ce (s, q) can be expanded in series of the form

2 J,cos2'"3 or 2 5,,,cos2' + i2,

 n=0 m=0

according as % is even or odd; and that these series converge when j
coss [ < 1.

* This solution is called in (z, q); the second sohitions of the
equations satisfied by Mathieu functions have been investigated by
Ince, Proc. Edinburgh Math. Soc. xxxiii. (1915), pp. 2-15. See also §
19-2.

%
% 428
%

12. With the notation of example 11, shew that, if

ce z, q) = X I e*cos3oo60 e,, ((9, q) dd, then A is given by one or
other of the series

provided that these series converge.

13. Shew that the differential equation satisfied by the product of
any two solutions of Bessel's equation for functions of order n is

S S-2n) S+21l)u + -I. S + l) u = 0,

where 3 denotes z -j- . dz

Shew that one solution of this equation is an integral function of 2;
and thence, by the methods of 5; \$ 19'5-19'53, obtain the Bessel
functions, discussing particularly the case in which a is an integer.

14. Shew that an approximate solution of the equation

- -ir(A+k-sm\ i-z)u=Q dz

is \ i = C (cosech s) - sin k cosh z + e),

where C and e are constants of integration; it is to be assumed that
k is large, A is not very large and z is not small.

%
% 429
%
\chapter{Elliptic Functions. General Theorems and the
Weierstrassian Functions}

\Section{20}{1}{Doubly-periodic functions.}

A most important property of the circular functions sin, cos, tan,
... is that, if/( ) denote any one of them,

f z + 2 )=f z),

and hfmce f z- '2mr)=f z), for all integer values of n. It is on
account of this property that the circular functions are frequently
described as periodic functions with period 27r. To distinguish them
from the functions which will be discussed in this and the two
following chapters, they are called singly-periodic functions.

Let Q)i, 0)2 be any two numbers (real or complex) whose ratio* is not
purely real. A function which satisfies the equations

f(z + 2a,0 =f z), f(z + 2a,,) =f(z),

for all values of z for which $f(z)$ exists, is called a doubly-periodic
function of z, with periods 2\&)i, 2co2. A doubly-periodic function
which is analytic (except at poles), and which has no singularities
other than poles in the finite part of the plane, is called an
elliptic function.

[Note. What is now known as an elliptic integral occurs in the
researches of Jakob Bernoulli on the Elastica. Maclaurin, Fagnano,
Legendre, and others considered such integrals in connexion with the
problem of rectifying an arc of an ellipse; the idea of 'inverting'
an elliptic integral (§ 21 '7) to obtain an elliptic function is due
to Abel, Jacobi and Gauss.]

The periods 2( i, 2\&,o play much the same part in the theory of
elliptic functions as is played by the single period in the case of
the circular functions.

Before actually constructing any elliptic functions, and, indeed,
before establishing the existence of such functions, it is convenient
to prove some general theorems (\hardsubsectionref{20}{1}{1}-20'14) concerning properties
common to all elliptic functions; this procedure, though not strictly
logical, is convenient

* If w.,/wj is real, the parallelograms defined in \hardsubsectionref{20}{1}{1} collapse,
and the function reduces to a singly-periodic function when Wg/wj is
rational; and when w /wj is irrational, it has been shewn by Jacobi,
Journal fiir Math. xiii. (183.5), pp. 55-56 [Ges. Werke, 11. (1882),
pp. 25-26] that the function reduces to a constant.

t A brief discussion of elliptic integrals will be found in §§
22-7-22*741.

%
% 430
%

because a large number of the properties of particular elliptic
functions can be obtained at once by an appeal to these theorems.

Example. The diflferential coefficient of an elliptic function is
itself an elliptic function.

2011. Pei-iod-parallelograms.

The study of elliptic functions is much facilitated by the geometrical
representation afforded by the Argaud diagi-am.

Suppose that in the plane of the variable z we mark the points 0,
2(Wi, 2(02, 2a)i + 2\&)2, and, generally, all the points whose complex
coordinates are of the form 2mo)i + 2h\&j.,, where m and n are
integers.

Join in succession consecutive points of the set 0, 2(Oi, 2(Oi + 2o
.2, 2ei).,, 0, and we obtain a parallelogram. If there is no point a
inside or on the boundary of this parallelogram (the vertices
excepted) such that

f z + <o)=f z)

for all values of z, this parallelogram is called \& fundamental
period-parallelo- gram for an elliptic function with periods 2a)i, 2(0
.

It is clear that the -plane may be covered with a network of
parallelo- grams equal to the fundamental period-parallelogram and
similarly situated, each of the points 2?/ia)i + 2na), being a vertex
of four parallelograms.

These parallelograms are called peHod-parallelograms, or meshes; for
all values of z, the points z, z- 2w, ... z +2m(o +2nw. ...
manifestly occupy corresponding positions in the meshes; any pair of
such points are said to be congruent to one another. The congruence of
two points z, z is expressed by the notation / = (mod. 2\&)i, 2w.2).

From the fundamental property of elliptic functions, it follows that
an elliptic function assumes the same value at every one of a set of
congruent points; and so its values in any mesh are a mere repetition
of its values in any other mesh.

For purposes of integration it is not convenient to deal with the
actual meshes if they have singularities of the integrand on their
boundaries; on account of the periodic properties of elliptic
functions nothing is lost by taking as a contour, not an actual mesh,
but a parallelogram obtained by translating a mesh (without rotation)
in such a way that none of the poles of the integrands considered are
on the sides of the parallelogi-am. Such a parallelogram is called a
cell. Obviously the values assumed by an elliptic function in a cell
are a mere repetition of its values in any mesh.

A set of poles (or zeros) of an elliptic function in any given cell is
called an irreducible set; all other poles (or zeros) of the function
are congruent to one or other of them.

%
% 431
%

\Subsection{20}{1}{2}{Simple properties of elliptic functions.}

(I) The number of poles of an elliptic function in any cell is finite.

For, if not, the poles would have a limit point, by the
two-dimensional analogue of \hardsubsectionref{2}{2}{1}. This point is \hardsubsectionref{5}{6}{1}) an
essential singularity of the function; and so, by definition, the
function is not an elliptic function.

(II) The number of zeros of an elliptic function in any cell is
finite.

For, if not, the reciprocal of the function would have an infinite
number of poles in the cell, and would therefore have an essential
singularity; and this point would be an essential singularity of the
original function, which would therefore not be an elliptic function,
[This argument presupposes that the function is not identically zero.]

(III) The sum of the residues of an elliptic function, f z), at its
poles in any cell is zero.

Let C be the contour formed by the edges of the cell, and let the
corners of the cell be, + 2\&ji, t + 2(yi + 2\&).,, t + 2\&)2.

[Note. In future, the periods of an elliptic function will not be
called 2ci)i, 2(B2 indifferently; but that one will be called 2wi
which makes the ratio o),/©! have a positive imaginary part; and
then, if C be described in the sense indicated by the order of the
corners given above, the description of C is counter-clockioise.

Throughout the chapter, we shall denote by the symbol C the contour
formed by the edges of a cell.]

The sum of the residues oif z) at its poles inside C is

 .\ f z)dz=-~\ \ . + + + \ f z)dz.

In the second and third integrals write z+2a)i, z + 2(Oo respectively
for 2, and the right-hand side becomes

 . /( ) -/ + 2a>.)) dz - - f(z) -f z + 20,01 dz,

and each of these integrals vanishes in virtue of the periodic
properties of f z); and so I f(z) dz = 0, and the theorem is
established.

(IV) Liouvilles theorem*. An elliptic function, f z), luith no poles
in a cell is merely a constant.

For if f(z) has no poles inside the cell, it is analytic (and
consequently bounded) inside and on the boundary of the cell \hardsubsectionref{3}{6}{1}
corollary ii); that is to say, there is a number K such that f(z), <
K when z is inside or on the boundary of the cell. From the periodic
properties oi f(z) it follows that

* This modification of the theorem of \hardsubsectionref{5}{6}{3} is the result on which
Liouville based his lectures on elliptic functions.

%
% 432
%

f z) is analytic and \ f(z) \ < K for all values of z; and so, by §
5-63, $f(z)$ is a constant.

It will be seen later that a very large number of theorems concerning
elliptic functions can be proved by the aid of this result.

\Subsection{20}{1}{3}{The order of an elliptic function.}

It will now be shewn that, \ \ f(z) be an elliptic function and c be
any constant, the number of roots of the equation

f( ) = c

which lie in any cell depends only on f z), and not on c; this number
is called the order of the elliptic function, and is equal to the
number of poles off(z) in the cell.

By \hardsubsectionref{6}{3}{1}, the difference between the number of zeros and the number
of poles o f(z) - c which lie in the cell G is

1 ' /'< > d..

27ri J r f(z) - c

Since /' z + 2\&)i) =/' z + 2(Wo) = /" z), by dividing the contour
into four parts, precisely as in \hardsubsectionref{20}{1}{2}(III), we find that this
integral is zero.

Therefore the number of zeros of /'( )- c is equal to the number of
poles of $f(z)$ - c\ but any pole oif z) - c is obviously a pole of f(z)
and conversely; hence the number of zeros of f(z)- c is equal to the
number of poles of $f(z)$, which is independent of $c$; the required
result is therefore established.

[Note. In determining the order of an elliptic function by counting
the number of its irreducible poles, it is obvious, from \hardsubsectionref{6}{3}{1}, that
each pole has to be reckoned according to its multiplicity.]

The order of an elliptic function is never less than 2; for an
elliptic function of order 1 would have a single irreducible pole;
and if this point actually were a pole (and not an ordinary point) the
residue there would not be zero, which is contrary to the result of §
201 2 (III).

So far as singularities are concerned, the simplest elliptic functions
are those of order 2. Such functions may be divided into two classes,
(i) those which have a single irreducible double pole, at which the
residue is zero in accordance with \hardsubsectionref{20}{1}{2} (III); (ii) those which
have two simple poles at which, by § 20"! 2 (III), the residues are
numerically equal but opposite in sign.

Functions belonging to these respective classes will be discussed in
this chapter and in Chapter xxii under the names of Weierstrassian and
Jacobian elliptic functions respectively; and it will be shewn that
any elliptic function is expressible in terms of functions of either
of these types.

%
% 433
%

\Subsection{20}{1}{4}{Relation hetiueen the zeros and poles of an elliptic function.}

We shall now shew that the sum of the affixes of a set of irreducible
zeros of an elliptic function is congruent to the sum of the affixes
of a set of irreducible poles.

For, with the notation previously employed, it follows, from \hardsubsectionref{6}{3}{1},
that the difference between the sums in question is

27ri J C f 2) TTl [J t J +2<o, J f+2a,,+2a,2 J t+2ojJ J 2:)

  j rt+2. j ) \ z+2oy,)f'(z + 2co. ] ~27ri]t \ f ) f z + 2ay,) \ '

27rij, \ f(z) f z+2c.,) r

27ri [ Jt f (z) J t f z)

J.j-2.,[log/(.)

t+iu

+ 2(0,

log/( )

t

on making use of the substitutions used in § 2012 (III) and of the
periodic properties off z) and f' z).

Now $f(z)$ has the same values at the points t + 2(Oi, t 4-20), as at t,
so the values of \ ogf(z) at these points can only differ from the
value of $f(z)$ at $t$ by integer multiples of 27ri, say - 2n7ri,
2///'7rt; then we have

2'mJc fiz)

and so the sum of the affixes of the zeros minus the suin of the
affixes of the poles is a period; and this is the result which had to
be established.

\Section{20}{2}{The construction of an elliptic function. Definition of z).}

It was seen in | 20'1 that elliptic functions may be expected to have
some properties analogous to those of the circular functions. It is
therefore natural to introduce elliptic functions into analysis by
some definition analogous to one of the definitions which may be made
the foundation of the theory of circular functions.

One mode of developing the theory of the circular functions is to
start

from the series S z-m7r)~-; calling this series (sin ')~-, it is
possible

JW= - CO

to deduce all the known properties of sin z; the method of doing so
is briefly indicated in \hardsubsubsectionref{20}{2}{2}{2}.

W. M. A. 28

%
% 434
%

The analogous method of founding- the theory of elliptic functions is
to define the function (,.> ( ) by the equation*

 " m.n \ \ {2; - 2vio)i - '2n(02y (2? \&)i+2?ieu2)-j ' Avhere (o, Wo
satisfy the conditions laid down in §§ 0*1, 20"12(III); the summation
extends over all integer values (positive, negative and zero) of m and
n, simultaneous zero values of ni and n excepted.

For brevity, we write Qm,n in place of 'Iinw + 'Inwo, so that

m, n

When m and n are such that \ \ m,n\ is large, the general terra of the
series defining ip (z) is 0(\ flm,n\~% and so \hardsectionref{3}{4}) the series
converges absolutely and uniformly (with regard to z) except near its
poles, namely the points Clm,n-

Therefore \hardsectionref{5}{3}), J (z) is analytic throughout the whole 2 -plane
except at the points n,\, where it has double poles.

The introduction of this function p z) is due to Weierstrassf; we now
proceed to discuss properties of (z), and in the course of the
investigation it will appear that j z) is an elliptic function with
penods 2\&)i, 2a)2.

For purposes of numerical computation the series for p (z) is useless
on account of the slowness of its convergence. Elliptic functions free
from this defect will be obtained in Chapter xxi.

Example. Prove that

P(2)=U- I - + 2 cosec2 =7r - 2 cosec- - "tt .

\Subsection{20}{2}{1}{Periodicity and other properties of z).}

Since the series for z) is a uniformly convergent series of analytic
functions, term-by-terra differentiation is legitimate \hardsectionref{5}{3}), and so

ij' z) =,j z)=-2 t

  m, n \ \ - '>, n)

The function ' z) is an odd functio7i of z\ for, from the definition
of y( ), we at once get

 '(- ) = 2 ( 4-n,, )- .

m, n * Throughout the chapter 2 will be written to denote a summation
over all integer values of m and n, a prime being inserted (2') when
the term for which j =:n = has to be omitted

m, n

from the summation. It is also customary to write ' z) for the
derivate of 4> z). The use of the prime in two senses will not cause
confusion.

t Werke, ii. (1895), pp. 245-2.55. The subject-matter of the greater
part of this chapter is due to Weierstrass, and is contained in his
lectures, of which an account has been published by Schwarz, Formeln
und Lehrsatze zum Gebrauche der elUptischen Fiinktionen, Xach
Vorle.tnngen und Aufzeichnungen des Herrn Prof. K. Weierstrass
(Berlin, 1893). See also Cayley, Journal de Math. X. (1845), pp.
385-420 [Math. Papers, i. pp. 156-182], and Eisenstein, Journal fUr
Math. XXXV. (1847), pp. 137-184, 18-5-274.

%
% 435
%

But the set of points - D, (\ is the same as the set Qm, n and so the
terms of ' (- z) are just the same as those of - ' (z), but in a
different order. But, the series for j' (z) being absolutely
convergent \hardsectionref{3}{4}), the derangement of the terms does not affect its
sum, and therefore

 y -z) = - y (z).

In like manner, the terms of the absolutely convergent series

ni,n

are the terms of the series

y Hz - n - n~ i

m, n

in a different order, and hence

i (- z) = i U); that is to say, (z) is an even functipn\ oj[j.

Further, j' z + 2co,) = - 2 1 ( - D, + 2a,r;

m, n

but the set of points n, - 2\&ji is the same as the set Q,,, so the
series for y (z + 'Zcoi) is a derangement of the series for ' (z). The
series being absolutely convergent, we have

 y (z + 2ft)i) = y (z);

that is to say, '/ (z) has the period 2fOi; in like manner it has the
period 2ft)o.

Since y' (z) is analytic except at its poles, it follows from this
result that y (z) is an elliptic function.

If now we integrate the equation ' z + '2(o ) = < ' z), we get

<p z + 2oy,) = < i z) + A,

where A is constant. Putting z = - w and using the fact that < z) is
an even function, we get = 0, so that

iO z + 2ft)i) = J (z); in like manner j z + 'Icoo) = (jp (z).

Since p (z) has no singularities but poles, it follows from these two
results that J(2) is an elliptic function.

There are other methods of introducing both the circular and elliptic
functions into analysis; for the circular functions the following may
be noticed:

(1) The geometrical definition in which sin z is the ratio of the side
opposite the angle 2 to the hypotenuse in a right-angled triangle of
which one angle is z. This is the definition given in elementary
text-books on Trigonometry; from our point of view it has various
disadvantages, some of which are stated in the Appendix.

(2) The definition by the power series

Z 2

sm2 = 2--, + -

28-2

%
% 436
%

(3) The definition by the product

(4) The definition by 'inversion' of an integral

fBim J

The periodicity properties may be obtained easily from (4) by taking
suitable paths of integration (of. Forsyth, Theory of Functions,
(1918), 104), but it is extremely difficult to prove that sin z
defined in this way is an analytic function.

The reader will 'see later (§§ 22-82, 22-1, 20-42, 20-22 and \hardsubsectionref{20}{5}{3}
example 4) that elliptic functions may be defined by definitions
analogous to each of these, with corre- sponding disadvantages in the
cases of the first and fourth.

Example. Deduce the periodicity of .> (z) directly from its definition
as a double series. [It is not difficult to justify the necessary
derangement.]

\Subsection{20}{2}{2}{The differential equation satisfied by j z).}

We shall now obtain an equation satisfied by < z), which will prove to
be of great importance in the theory of the function.

The function j (2) -,2" which is equal to S' (2 - n,,,i)~2 - H, is

m, n

analytic in a region of which the origin is an internal point, and it
is an even function of s. Consequently, by Taylor's theorem, we have
an expansion of the form

valid for sufficiently small values of | |. It is easy to see that

m, n ni, n

Thus (z) = Z-' + 20 g z' + 28 3 + ( ') '

differentiating this result, we have

 y (z) = - 2z-' + g,z 1 g,z + (z ). Cubing and squaring these
respectively, we get

f z) = z-' + .l~ g,z- + l g..+ 0(z ),

Hence ' (z) - 4 =* (z) = - g.z" -gs + z'),

and so '- z) - 4f (z) + g (z) + gs=0 (z').

That is to say, the function z) - 4 (z) + g (z) + g-s, which is
obviously an elliptic function, is analytic at the origin, and
consequently it is also analytic at all congruent points. But such
points are the only possible singularities of the function, and, so it
is an elliptic function luith no singularities; it is therefore a
constant \hardsubsectionref{20}{1}{2}, lY).

On making z- 0, we see that this constant is zero.

%
% 437
%

Thus, finally, the function ( z) satisfies the differential equation

where 2 and g (called the invariants) are given by the equations

g, = m T n f, 5r3 = i4o 1' n-.

), n m, n

Conversely, given the equation if numbers \&)i, (o can he determined*
such that

m, n i, n

then the general solution of the differential equation is

y = iO ±z + a), where a is the constant of integration. This may be
seen by taking a new dependent variable u defined by the equationf y =
< (u), when the differential

equation reduces to [ \ =1.

Since g? z) is an even function of z, we have y = 0 z ± a), and so the
solution of the equation can be written in the form

y = j z + a) without loss of generality.

Example. Deduce from the differential equation that, if

)i = l

then C2=5r2/22 . 5, Ci=g l . 7, CQ=g ij2i.' . 52,

 ~2*.5.7.11' "~25..3.5M.3" 2*.72.13' '' 2 . 3 . 52 . 7 . 11 "

\Subsubsection{20}{2}{2}{1}{The integral formula for < z).}
 Consider the equation

z=\ (U'-g,t-g,)- dt,

  i

determining z in terms of if; the path of integration may be any
curve which does not pass through a zero of M - g. t - g . On
differentiation, we get

(i/ = 4?'-< f-...

and so =Sf> ( + oc),

where a is a constant.

* The difficult problem of establishing the existence of such numbers
w and Wj when g and g are given is solved in \hardsubsectionref{21}{7}{3}.

t This equation in it always has solutions, by \hardsubsectionref{20}{1}{3}.

%
% 438
%

Make - > x; then - >0, since the integral converges, and so a is a
pole of the function; i.e., a is of the form n,,i, and so = j(2 + n,
n) = (z).

The result that the equation z=l 4:t' - gd - (/ i)~-dt is equivalent
to

u

the equation = z) is sometimes written in the form

= f,

\Subsubsection{20}{2}{2}{2}{An illustration from the theory of the circular functions.}

The theorems obtained in §>; 20'2-20'221 may be illustrated by the
corresponding results in the theory of the circular functions. Thus we
may deduce the properties

of the function cosec- z from the series 2 z- mTT)~ in the following
manner:

m=-a)

Denote the series by/ (2); the series converges absolutely and
uniformly* (with regard to z) except near the points mn at which it
obviously has double poles. Except at these points, /(j) is analytic.
The effect of adding any multiple of tt to is to give a series whose
terms are the same as tho.se occurring in the original series; since
the series converges absolutely, the sum of the series is unaffected,
and 80/(2) is a periodic function of z icith period n.

Now consider the behaviour oi $f(z)$ in the .strip for which -\ 7r R z)
\ n. From the periodicity of f z), the value off z) at any point in
the plane is equal to its value at the corresponding point of the
strip. In the strip/ (2) has one .singularity, namely 2 =; and /(2)
is bounded as 2-*-oc in the strip, becau.se the terms of the series
for/ (2 are

ac

small compared with the corresponding terms of the comparison series
2' m~ .

m= - 00

In a domain including the point z=0, f z)-z~ is analytic, and is an
even function; and consequently there is a Maclaurin expansion

/(2)-2-2= 2 a, z valid when ' z' <n. It is easily seen that

a2 = 27r-- (2 + l) 2 jn-'"~-,

m = l

and so o= a2 = 67r~* 2 in'* .

m=l

Hence, for small values of | 2 |,

f z) = z- + + \ z + 0(z*). Differentiating this result twice, and also
squaring it, we have

f"(z)=ez- + +0 z%

 Hz)=z-*+p- +n+o z ).

It follows that /" (2) -e/'-i (2) + 4/(2)= (z ).

That is to say, the function /" (2) - 6/ (2) + 4/" (2) is analytic at
the origin and it is obviously periodic. Since its only possible
singularities are at the points mn, it follows from the periodic
property of the function that it is an integral function.

* By comparison with the series 2' m~' .

m = - 30

%
% 439
%

Furthei*, it is bounded as z-s-qc in the strip - ir R z) 7r, since f
z) is bounded and so is* f" (z). Hence/" (2) - 6/ (3) + 4/(2) is
bounded in the strip, and therefore from its periodicity it is bounded
everywhere. By Liouville's theorem \hardsubsectionref{5}{6}{3}) it is therefore a
constant. By making z ~0, we see that the constant is zero. Hence the
function cosec z satisfies the equation

f"(z) = 6fHz)-4f z).

Multiplying by 2/' (s) and integrating, we get

f z) = 4P z) f(z)-l +c,

where c is a constant, which is easily seen to be zero on making use
of the power series

for/' (2) and/ (2).

We thence deduce that 2s = I f' t-l)~i dt,

J fiz)

when an appropriate path of integration is chosen.

Example 1. If j/ = (2) and primes denote difierentiations with regard
to 2, shew that

4 -|j3=tV (y- i)-H0/-.'.2)-H(3/-.3)- -|y(3/- i)-'(y- 2)-V.y- 3)--\

where ej, e, e are the roots of the equation Afi - got - gz = - [AVe
have y" = 4f-go,y-g

Differentiating logarithmically and dividing by y\ we have

r=\

Differentiating again, we have

2y"' 4?/"2 3,,

y y r=i '

Adding this equation multiplied by j to the square of the preceding
equation, multiplied by, we readily obtain the desired result.

It should be noted that the left-hand side of the equation is half the
Schwarziaii derivative t of z with respect to y; and so z is the
quotient of two solutions of the equation

c + Ire,.!, (//-O -gy n (.-..) j .=0.]

Example 2. Obtain the 'properties of homogeneity ' of the function
(2); namely that

 H>''')= '' f 'l'"')'; ~'92, X-V3) = X-'\&?(--; <72,5'3), \ 1 A 2/ \
i CO2/

where (s M denotes the function formed with periods 2q)i, 2a)2 and (2;
g, g )

denotes the function formed with invariants g-n g -

[The former is a direct consequence of the definition of z) by a
double series; the latter may then be derived from the double series
defining the g invariants.]

* The series for /" (s) may be compared with 2' t~'*.

m= -

t Cayley, Gamh. Phil. Tram. xiii. (1883), p. 5 [Math. Papers, xi. p.
148].

%
% 440
%

\Section{20}{3}{The addition-theorem for the function TODO.}

The function z) possesses what is known as an addition-theorem; that
is to say, there exists a formula expressing J z + y) as an algebraic
function oi z) and (y) for general values* <) z and ?/.

Consider the equations

i ' z) = Af z) + B, <,j' y) Af j) + B,

which determine A and B in terms of and y unless z) = < y), i.e.
unlessf z = ±y (mod. 2wi, 2\&)2),

Now consider \&>' (0 - iP (t) - >

 i/a function of f. It has a triple pole at = and consequently it has
three, and only three, irreducible zeros, by \hardsubsectionref{20}{1}{3}; the sum of these
is a period, by \hardsubsectionref{20}{1}{4}, and as =z, =y are two zeros, the third
irreducible zero must be congruent to - - y. Hence - z - y h a. zero
of ' ( ) - Af ( ) - B,

and so

 o' -z-y)=A< -z-y) + B.

Eliminating A and B from this equation and the equations by which A
and B were defined, we have

  z) f' z) 1 =0.

i iz y) - ' z + y) 1

Since the derived functions occurring in this result can be expressed
algebraically in terms of z), i> y), <fP (z + y) respectively (§
20"22), this result really expresses .> z + y) algebraically in terms
of i z) and < (y). It is therefore an addition-theorem.

Other methods of obtaining the addition-theorem are indicated in §
20-311 examples 1 and 2, and \hardsubsubsectionref{20}{3}{1}{2}.

A symmetrical form of the addition-theorem may be noticed, namely
that, if u + V -I- w = 0, then

! ij(u) i iu) 1 =0.

\&(v) '(v) 1 (lu) ' (lu) 1

\Subsection{20}{3}{1}{Another form of the addition-theorem.}

Retaining the notation of \hardsectionref{20}{3}, we see that the values of, which
make ' ( ) - A ( ) - B vanish, are congruent to one of the points z,
y,-z - y.

* It is, of course, unnecessary to consider the special cases when y,
or z, or ij +z is n period.

t The function z)- (y), qua function of z, has double poles at points
congruent to 2 = 0, and no other singularities; it therefore (§
20-13) has only two irreducible zeros; and the points congruent to z=
y therefore give all the zeros of (z) - J (y).

%
% 441
%

Hence ' (0- l- fr (O + B]- vanishes when t is congruent to any of the
points z, y, - z - y. And so

4j.nr) - AY K) - AB + g. f 0 - (B-' + 9s)

vanishes when j ( ) is equal to any one of (z), y), j (z + y).

For general values of z and y, j (z), o (y) and z + y) are unequal and
so they are all the roots of the equation

4>Z - A-Z-- - -lAB + g )Z- (B- +g,) = 0.

Consequently, by the ordinary formula for the sum of the roots of a
cubic equation,

 ( ) + iHy) + i (2 + y) = lA%

and so i. + y)=li - -,). o(y),

on solving the equations by which A and B were defined.

This result expresses j (z + y) explicitly in terms of functions of z
and of 2/.

\Subsubsection{20}{3}{1}{1}{The duplication formula for < z).}

The forms of the addition-theorem which have been obtained are both
nugatory when y = z. But the result of \hardsubsectionref{20}{3}{1} is true, in the case of
any given value of z, for general values of y. Taking the limiting
form of the result when y approaches z, we have

From this equation, we see that, if 22- is not a period, we have

/ox Ir W (z) - i ' + h) o / X g (2z) = -J hm V- -x - -7 ~-y - 2p (z)

= -. 11m

on applying Taylor's theorem to J z + h), j' z + h)\ and so

 < >=i |<f- ( >'

unless 2z is a period. This result is called the duplicatioih formula.
Example 1. Prove that

qua function of j, has no singularities at points congruent with 2 =
0, ±i/; and, by making use of Liouville's theorem, deduce the
addition-theorem.

%
% 442
%

Example 2. Apply the process indicated in example 1 to the function

I (2/) ¥iy) 1 I,

and deduce the addition-theorem. Example 3. Shew that

  z .y) + z-y) = W z)-<p:y) -'-[ 2 z)ip y)-\ g ] <§> z)- y) -g,\ [By
the addition-theorem we have

Replacing ' z) and p (y) by 4 ( )\ (2)\ 3 and <\& y) - g y) - 9
respec- tively, and reducing, we obtain the required result.] Example
4. Shew, by Liouville's theorem, that

j W z-a) z-h)] = <p a-h) <p' z-a) + ' ' z-h) -iy a-b) z-a)- z-h)].

\addexamplecitation{Trinity, 1905.} 20"312. Abel's* method of proving the
addition-theorem for p z).

The following outline of a method of establishing the addition-theorem
for p (z) is instructive, though a completely rigorous proof would be
long and tedious.

Let the invariants of p z) be go, g ', take rectangular axes OX, OY in
a plane, and consider the intersections of the cubic curve

y - = Ax -g.2X-g with a variable line y = mx- n.

If any point ( i, y ) be taken on the cubic, the equation in z

P z)-x, = Q has two solutions +2i, - i \hardsubsectionref{20}{1}{3}) and all other
solutions are congruent to these two.

Since P' z) = ip z)-g2P z)-gz, we have P"- z)=yi; choose z- to be the
solution for which p' (2i)= -Hyi, not -y .

A number Zi thus chosen will be called the parameter of ( i, j/i) on
the cubic.

Now the abscissae i, x.2, x. of the intersections of the cubic with
the variable line are the roots of

< ix) ~ 4x gi -gz- (" - +nY= 0,

and so ( x)=.A x- x ) (x - x, (x - x ).

The variation bx in one of these abscissae due to the variation in
position of the line consequent on small changes hn, 8n in the
coefficients m, n is given by the equation

(b' (Xr) S*v+ - 8m + ? 8?t = 0, Cm en

and so * (f)' (Xr) 8 r = 2 (mx + n) x tm - 8n),

TODO

whence 2 - =2 2

=1 mjCr + n r=l 0' (- r) '

provided that Xi, x, x are unequal, so that < ' Xr)+0.

* Journal fiir Math. ii. (1827), pp. 101-181; iii. (1828), pp. 160-190
[Oeuvres, i. (Christiania, 1839), pp. 141-2.52].

%
% 443
%

Now, if we put X (x 8)n + 8n)/4> (s), qua function of .r, into partial
fractions, the result is

3 r=\

where,.= lim .r ( Sm + Sw) -p

= x x . 8m + 8n) lim (x - av)/ ( )

by Taylor's theoi-em.

:=.c, x,8m + 8n)l(f)' x \

3 3

Putting x=0, we get 2 S.r,./j . = 0, i.e. 2 62,. = 0.

r=l r=l

That is to say, the sum of the parameters of the points of
intersection is a constant independent of the position of the line.

Vary the line so that all the points of intersection move off to
infinity (no two points coinciding during this process), and it is
evident that 21 + 22 + 23 is equal to the sum of the parameters when
the line is the line at infinity; but when the line is at infinity,
each parameter is a period of p (2) and therefore 21 + 22 + 23 is a
period of (2).

Hence the sum of the parameters of three collinear points on the cubic
is congruent to zero. This result having been obtained, the
determinantal form of the addition-theorem follows as in \hardsectionref{20}{3}.

2032. The constants e-, 62, 63.

It will now be shewn that J (coi), j((Oo), i ifOs), (where 0)3= - co -
coo), are all unequal; and, if their values be e, e, e, then e-,
e.2, e-, are the roots of the equation 4 * - g t - g O.

First consider < ' (\&)i). Since ' z) is an odd periodic function, w e
have \&>' i i) = ~ i ' (- i) = - i ' (2<yi - i) = - < ' (ft)i), and so
fr''(\&>i) = 0.

Similarly < ' w. = ( ' (0)3) = 0.

Since < ' z) is an elliptic function whose only singularities are
triple poles at points congruent to the origin, < ' z) has three, and
only three \hardsubsectionref{20}{1}{3}), irreducible zeros. Therefore the only zeros of
f' z) are points congruent to

COj, \&),, 6)3.

Next consider i z) - ei. This vanishes at (o and, since < ' wy) = 0,
it has a double zero at (o . Since i z) has only two irreducible
poles, it follows from \hardsubsectionref{20}{1}{3} that the only zeros of ( z) - ei are
congruent to coj. In like manner, the only. zeros of z)- e-., J(z) -
es are double zeros at points con- gruent to (1)2, ois respectively.

Hence i 62 = s- For if gj = e, then (z) - e has a zero at w., which
is a point not congruent to Wj.

Also, since ' (z) = 4 (z) - gz (z) - g and since '' (z) vanishes at
coj, to,, 6)3, it follows that 4 =* (z) - g2 (z) - g vanishes when p
(z) = e, 62 or e .

That is to say, e, 62, e are the roots of the equation

4>f-g2t-gs = 0.

%
% 444
%

From the well-known formulae connecting roots of equations with their

coefficients, it follows that

e, + e. + 3 = 0,

Example 1. When g and rg are real and the discriminant g.? - %1g- is
positive, shew that ex, e2 s are all real; choosing them so that e-
>eo> 63, shew that

. 00 0,1=1 t -g t-g- ~ dt,

and 0)3 = - 1 / ' g:i+g.it - 4fi) ~ dt,

so that 0)1 is real and 003 a pure imaginary.

Example 2. Shew that, in the circumstances of example 1, p z) is real
on the peri- meter of the rectangle whose corners are 0, 0)3, wi +
cos, coi.

\Subsection{20}{3}{3}{The addition of a half-period to the argument of p (z). }
From the form of the addition-theorem given in \hardsubsectionref{20}{3}{1}, we have

,(,)., +,K)=H|-l;!- r.

3

and so, since J'-(z) = 4 n P z) - er,

r=l

we have (-'+o,.) = ' <";!l' i''"'" - W- .

 J Z) - Bi

3 on using the result 2 6 =0;

r=l

this formula expresses (s + coi) in terms of (2).

Example 1.' Shew that

  (|6,i)=ei± (61-62) (61-63) -

Example 2. From the formula for z + at. combined with the result of
example 1, shew that

  ( CO, + Q).,) = 61 + (61 - 62) (61 - 63) .

\addexamplecitation{Math. Trip. 1913.}

Example 3. Shew that the value of '(2) '(z-t-wi) ' (s-l-wo) (2 + W3)
is equal to the discriminant of the equation 4t - got - g3=0.

[Differentiating the result of \hardsectionref{20}{3}3, we have

r (2 -f l) = - (61 - 62) (61 - 63) ' (2) if> (2) - 6, -2;

from this and analogous results, we have

f (2) ' (2 + Wi) g)' (2 + CO2) p' (2 -I- CO3)

= (61 -62)2 (62 -63)- (63-61)2 (2) n p(2)-6, -2

r=l = 16(6l-62)M 2- 3)M 3-e,)

which is the discriminant g - Tig- in question.]

%
% 445
%

Example 4. Shew that, with appropriate interpretations of the
radicals,

 ' (icoi)= -2 e,-e.,) e,-e )]i l ei-e ) + e -e )h .

\addexamplecitation{Math. Trip. 1913.}

Example 5. Shew that, with appropriate interpretations of the
radicals, p 2z) - e p 2z) - 63 * + W (22) - 3 * P (22) - e

+ iP m~e, i ip 2z)-e.2 = p z)-p 2z).-

\Section{20}{4}{Quasi-periodic functions. The function* TODO.}

We shall next introduce the function (z) defined by the equation

dz - ' ' coupled with the condition lim [ z) - z~ ]=0.

Since the series for p z) - z~' converges uniformly throughout any
domain from which the neighbourhoods of the points f 'm,n are
excluded, we may integrate term-by-term \hardsectionref{4}{7}) and get

l;iz)-z- = -n z)-z-,dz J

= -S' r z-n,,nr'- 7n%]dz, 711, n J

andso ( )= +N !\ \ + +

The reader will easily see that the general term of this series is

0( n i,;-= ) as in,,,, - co;

and hence (cf \hardsectionref{20}{2}), z) is an analytic function of z over the Avhole
sr-plane except at simple poles (the residue at each pole being -H 1)
at all the points of the set i m, ti- lt is evident that

Z m.n \ Z -T i-m,n iii,n I2, j,j)

and, since this series consists of the terms of the series for (z),
deranged in the same way as in the corresponding series of \hardsubsectionref{20}{2}{1}, we
have, by \hardsubsectionref{2}{5}{2},

  -z) = - (z),

that is to say, z) is an odd function of z.

* This function should not, of course, be confused with the
Zeta-funetion of Eieraann, discussed in Chapter xni.

t The symbol il', is used to denote all the points fl,,,i with the
exception of the origin (cf.\hardsectionref{20}{2}).

%
% 446
%

Following up the analogy of J 20-222, we may compare z) with the
function cot 2 defined by the series 2~'+ 2' (2- i7r)~' + (m7r)~i,
the equation -7; cot 2 = - cosec 2

cori-e ponding to f (2) = - (2).

\Subsection{20}{4}{1}{The quasi-periodicity of the function i z).}

The heading of \hardsectionref{20}{4} was an anticipation of the result, which will
now be proved, that i z) is not a doubly-periodic function .of z; and
the effect on z) of increasing z by 2( i or by 2w.. will bo
considered. It is evident from \hardsubsectionref{20}{1}{2} (III) that z) cannot be an
elliptic function, in view of the fact that the residue of z) at every
pole is + 1.

If now we integrate the equation

ip z- 2\&)i) = (j) z\

weget ( +2a,,)=r(~0 + 277i,

where 27 1 is the constant introduced by integration; putting z= - ai
, and taking account of the fact that z) is an odd function, we have

In like manner, z + 2\&)o)= z)- 2??2,

where 1-1 = K ( 2)-

Example 1. Prove by Liouville's theorem that, if A--i-?/-|-2 = 0, then

(Frobenius u. Stiekelberger, Journal fur Math, lxxxviii.)

[This result is a pseudo-addition theorem. It is not a true
addition-theorem since

C i- ), C y\ C (2) ai'e not algebraic functions of ( (x), f (.?/), C
(2)-]

Example 2. Prove by Liouville's theorem that

2 I 1 X) p (x)

' 1 PLy) PHy)

1 (2) (2)

1 p x) p' x)\ \ = ax+y + z)-i x)-( jj)-C .z).

1 (y) F(i )

1 (2) F'(2) I Obtain a generalisation of this theorem involving n
variables.

\addexamplecitation{Math. Trip. 1894.}

\Subsubsection{20}{4}{1}{1}{Tlie relation hetiueen ij and rj.,.}

We shall now shew that

1 .

To obtain this result consider i z)dz taken round the boundary of a
cell. There is one pole of i z) inside the cell, the residue there
being -I- 1.

Hence 1: (z) dz = 27ri.

J r

%
% 447
%

Modifying the contour integral in the manner of \hardsubsectionref{20}{1}{2}, we get

 -Ki = K z) - r( + 2a),)i dz - ( ) - ( + 2 o (

= - 27?2 rf + 27/1 fZt

and so 27rt = - iVi i + ' Vi -z,

which is the required result.

\Subsection{20}{4}{2}{The function a z).}

We shall next introduce the function o z), defined by the equation

j- \ og(7 z)=l; z)

coupled with the condition lim [a z)!z] = 1.

On account of the uniformity of convergence of the series for t, z),
except near the poles of z), we may integrate the series term-by-term.
Doing so, and taking the exponential of each side of the resulting
equation, we get

a(z) zn'\ [ l- ' ' ( z z

( ) = n' l- -)exp

+

--m.n 2S2T.

the constant of integration has been adjusted in accordance with the
condition stated.

By the methods employed in §§ 202, 20-21, 20-4, the reader will easily
obtain the following results:

(I) The product for a (z) converges absolutely and uniformly in any
bounded domain of values of z.

(II) The function a(z) is an odd integral function of with simple
zeros at all the points flm,n'

The function a (z) may be compared with the function sin z defined by
the product

the relation -j- log sin z = cot z corresponding to -i- log o- (z) =
(z).

\Subsubsection{20}{4}{2}{1}{The quasi-periodicity of the function cr z).}

If we integrate the equation

  z + 2(o,) = !: z +2v we get a (z + 2(Wi) = ce- ' -a (z),

where c is the constant of integration; to determine c, we put z = -
coi, and

then

a (coi) = - ce~-''''">o- (coi).

z

J = - CO (\

%
% 448
%

Consequently c = - e-'''"',

and o- (z + 2\&)j) = - e2''i( +'-i) a z).

In like manner (t(z + 2<i)o) = - g i 'z-i-'xj) a (z).

These results exhibit the behaviour of (t z) when z is increased by a
period of (z).

If, as in \hardsubsectionref{20}{3}{2}, we wTite ais = - co - ay.,, then three other
Sigma-functions are defined by the equations

a, z) = e-'i'-'o- (z + \&),.) V (wr) (? = !, 2, 3).

The four Siguia-functions are analogous to the four Theta-functions
dis- cussed in Chapter XXI (see \hardsectionref{21}{9}).

Example 1 . Shew that, if m and n are any integers,

<r z + 277i(Oi + 2na>2) = ( - )"' + " a- (z) exp 27nrii + 2nr).>) z +
2m-r) <3>i + 4? ?i;j,a)2 + 2n' r\ < ia>, and deduce that r]xi>ii-
r]-i<Ji\ is an integer multiple of \ ni.

Example 2. Shew that, if 5' = exp iriu)-!! o>i)i so that y < 1, and if

then / (2) is an integi-al function with the same zeros as (t z) and
also F z)la- z) is a doubly-periodic function of 2 with periods 2a)i,
2a).2-

Example 3. Deduce from example 2, by using Liouville's theorem, that

Example 4. Obtain the result of example 3 by expressing each factor on
the right as a singty infinite product.

\Section{20}{5}{Formulae expressing any elliptic function in terms of Weie7'strassian functions ivith the same periods.TODO}

There are various formulae analogous to the expression of any rational
fraction as (I) a quotient of two sets of products of linear factors,
(II) a sum of partial fractions; of the first type there are two
formulae involving Sigma- functions and Weierstrassian elliptic
functions respectively; of the second type there is a formula
involving derivates of Zeta-functions. These formulae will now be
obtained.

\Subsection{20}{5}{1}{The expression of any elliptic function in terms of z) and ( >'z).}

Let $f(z)$ be any elliptic function, and let i z) be the Weierstrassian
elliptic function formed with the same periods 2\&)i, 2\&)o.

We first ite

fiz) = I Uiz) +/(- z)] + \ [ f(z) -f - z)\ Wizfr] ' z).

%
% 449
%

The functions

f z) +/(- z\ [f z) -/(- )] Wi )]-' are both even functions, and they
are obviously elliptic functions when/( ) is an elliptic function.

The solution of the problem before us is therefore effected if we can
expi ess any even elliptic function < (2), say, in terms of i ) (z).

Let a be a zero of (f> (z) in any cell; then the point in the cell
congruent to - a will also be a zero. The irreducible zeros of (z) may
therefore be arranged in two sets, say a, a, ...an and certain
points congruent to - a-,

 2 >    - Cln

In like manner, the irreducible poles may be arranged in two sets, say
hi, b.., ... bn, and certain points congruent to -61, - 60, ... - 6 .
Consider now the function*

1 fi H')( )-iP("r)

d> z)-, = i iiO Z) - ) br)

It is an elliptic function of z, and clearly it has no poles; for the
zeros of (b (z) are zeros f of the numerator of the product, and the
zeros of the denominator oi" the product are polesf of 4>(z).
Consequently by Liouville's theorem it is a constant, A, say.

Therefore < (.) = A H | ? 1,

that is to say, <f) (z) has been expressed as a rational function of J
(z).

Carrying out this process with each of the functions

f z) +fi-z), f(z) -f -z) ip\ z) -\

we obtain the theorem that any elliptic function f (z) can be
expressed in terms, of the Weierstrassian elliptic functions (z) and
p' z) luith the same periods., the expression being rational in z) and
linear in < ' z).

\Subsection{20}{5}{2}{The expression of any elliptic function as a
  linear combination of Zeta functions and their derivatives.}

Let $f(z)$ be any elliptic function with periods 2coi, 2(iJo: Let a set
of irreducible poles of $f(z)$ be i, c/o, ... a, and let the principal
part \hardsubsectionref{5}{6}{1}) off z) near the pole a be

 k,i . Ck,2 Ck, rje

z-ajc (z- a f '" z- ttkY '

* If any one of the points a,, or h is congruent to the origin, we
omit the corresponding factor ii> (2) - .' (rt,.) or J ( ) - (\&,.).
The zero (or pole) of the product and the zero (or pole) of (j) [z) at
the origin are then of the same order of multiplicit.y. In this
product, and in that of \hardsectionref{20}{5}o, factors corresponding to multiple
zeros and poles have to be repeated the appropriate number of times.

f Of the same order of multiplifity.

W. M. A, 29

%
% 450
%

Then we can shew that

f z) = .4, + i \ c,,, 2 - ( ) -Ct,,r ( - "X") + . c - l (.

k:

where A., is a constant, and (z) denotes -r- z).

Denoting the summation on the right by F z), we see that F z +
-2co,)-F z)= I 27J,Ck,,

k = l

by \hardsubsectionref{20}{4}{1}, since all the derivates of the Zeta-functions are
periodic.

n

But S C/c i is the sum of the residues of y'(2 ) at all of its poles
in a cell,

k = l

and is consequently \hardsubsectionref{20}{1}{2}) zero.

Therefore F(z) has period 2\&)i, and similarly it has period 2coo; and
so f(z) - F (z) is an elliptic function.

Moreover $F(z)$ has been so constructed that $f(z) - F(z)$ has no poles at
the points nj, o, ... a; and hence it has no poles in a certain cell.
It is consequently a constant, A.,, by Liouville's theorem.

Thus the function $f(z)$ can be expanded in the form
A=is=i(s - i; !
This result is of importance in the problem of
integrating an elliptic function $f(z)$ when the principal part of its
expansion at each of its poles is known; for we obviously have

f(z)dz = A.z+ 2 k=i

Ck,i\ og a (z - ak)

where C is a constant of integration.

Example. Shew by the method of this article that

and deduce that

where C is a constant of integration.

\Subsection{20}{5}{3}{The expression of any elliptic function as a
  quotient of Sigma-functions.}

Let $f(z)$ be any elliptic function, with periods 2\&)i and 2a)2, and
let a set of irreducible zeros of /( ) be a,, a, ... an. Then
\hardsubsectionref{20}{1}{4}) we can choose a

%
% 451
%

set of poles bi,bo, ... hn such that all poles 0 /(2) are congruent to
one or other of them andf

a + a2+ ... + an = bi + b.2 + ... +bn.

Consider now the function

  (r(z- ar)

This product obviously has the same poles and zeros as f z); also the
effect of increasing z by 2(0j is to multiply the function by

  exp [2771 z - g,.),.=1 exp 277i ( - br)] The function therefore has
period 2\&)i (and in like manner it has period 2\&)2), and so the
quotient

is an elliptic function with no zeros or poles. By Liouville's
theorem, it must be a constant, A- say.

Thus the function/' (2 ) can be expressed in the form

r=l(T Z-b,)

An elliptic function is consequently determinate (save for a
multiplicative constant) when its periods and a set of irreducible
zeros and poles are known. Example 1. Shew that

Example 2. Deduce by difFerentiation, from example 1, that

and by further differentiation obtain the addition-theorem for (z).

II

2 b,., shew that

=1

I (r ar-bi)(r(ar-b. ...(T ar-bJ \,.=1 o- (a,. - ai) o- (a,. - a J . .
. . . . o- (a - a ) ' the * denoting that the vanishing factor o- (a -
a ) is to be omitted. Example 4. Shew that

  z)-e,. = a/ z)la' z) (r=l,2, 3).

[It is customary to define g? z) - e to mean o-,. (2)/(r (2), not -
o-,. (s)/'o- (2).] Example 5. Establish, by example I, the '
three-term equation,' namely,

d z + a) a- z - a) a- b + c) a- b - c) + <T z + b) a- z - b) a- c + a)
a c - a)

+ 0- z + c)(t z-c) (r a + b)a a-b) = 0.

t Multiple zeros or poles are, of course, to be reckoned according to
their degree of multi- plicity; to determine b, h-i, ...b, we
choose 6i, bo,, ... b, \ i, 6,/ to be the set of poles in the cell
in which ai, a-i, ...a lie, and then choose \&, congruent to 6, in
such a way that the required equation is satisfied.

29-2

Example 3. If 2 a,.= 2 6,., shew that

r-. r=]

%
% 452
%

[This result is due to Weierstrass; see p. 47 of the edition of his
lectures by Schwarz.] The equation is characteristic of the
Sigma-function; it has been proved by Halphen, Fonctions ElUptiques,
I. (Paris, 1886), p. 187, that no function essentially diflferent from
the Sigma-function satisfies an equation of this type. See p. 461,
example 38.

\Subsection{20}{5}{4}{The connexion between ani/ hvo elliptic functions with the same periods.}

We shall now prove the important result that an algebraic relation
exists bettveen any ttuo elliptic functions, f(z) and <f)(z), with the
same periods.

For, by \hardsubsectionref{20}{5}{1}, we can express $f(z)$ and (z) as
rational functions of
the Weierstrassian functions (z) and ' (z) with the same periods, so
that

f(z) = R, [p (z), io' (z)], < (z) = R, (z), ' (z)], where Ri and jB,
denote rational functions of two variables.

Eliminating (z) and ' (z) algebraically from these two equations and
'Uz) = 4>f' z)-go\ \& z)-g

we obtain an algebraic relation connecting f(z) and (f> (z); and the
theorem is proved.

A particular case of the proposition is that every elliptic function
is con- nected with its derivate by an algebraic relation.

If now we take the orders of the elliptic functions $f(z)$ and 4> z) to
be $m$ and $n$ respectively, then, corresponding to any given value of/( )
there is (§ 2018) a set of m iiTeducible values of z, and consequently
there are m values (in general distinct) of cf) (z). So, corresponding
to each value off, there are ni values of cf) and, similarly, to each
value of (f) correspond n values of /.

The relation between f(z) and cf) (z) is therefore (in general) of
degree m in (f) and n in f

The relation may be of lower degree. Thus, iff(z) = p (z), of order 2,
and (j) (2) = 2 of order 4, the relation is/- = cf).

As an illustration of the general result take f(z) = o z), of order 2,
and (j> (z) = y (z), of order 8. The relation should be of degree 2 in
(/> and of degree 3 in f; this is, in f;ict, the case, for the
relation is < - = 4/'-' - .2/- 3.

Example. If u, v, ic are three elliptic functions of their argument of
the second order with the same periods, shew that, in general, there
exist two distinct relations which are linear in each of l, v, w,
namely

A uvtv+Bvw + Cicu + Btiv + E u + F r + G iv+ H =0, A'uvw + B'vw +
C'lvti + D'u v + E'u + F'v + G'w + H' = 0, where 1, B, . . ., IT'
are constants.

\Section{20}{6}{On the integration of TODO.}

It will now be shewn that certain problems of integration, which are
insoluble by means of elementary functions only, can be solved by the
intro- duction of the function < z).

%
% 453
%

Let ttoX + a af + Qa. x- + a x + a =f(x) be any quartic polynomial
which has no repeated factors; and let its invariants* be

g = ciocti - 4aia3 + 3a.

gs = a aoCti + 2aia2a3 - ai - a a - a-ca .

/'* -1

Let z = /(t)] 'dt, where a-'o is any root of the equation/(a;) =;
then,

if the function z) be construe tedf with the invariants g and g, it
is possible to express x as a rational function of <,p z; go, g ).

[Note. The reason for assuming tbat/(.r) has no repeated factors is
that, when/(.r) has a repeated factor, the integration can be eftected
with the aid of circular or logarithmic functions only. For the same
reason, the case in which aQ = ai = need not be considered.]

By Taylor's theorem, we have

f t) = 4 3 t - X,) + QA. (t - x,y + 4>A, (t - x,y + 0 ( - 'oY,

(since / (xq) = 0), where

A 2 = UoXq + 2aiXQ + a.,, As = o V'' + ' UiXq- + a.2Xo + ttj. On
writing (t - Xa)~ = t, (x - Xq)~'- =, we have

2 = 1 4.43T= + QA,T + 4A,T + Ao] ~ irfr.

' s

To remove the second term in the cubic involved, write

r = Ar(cr-h . =A,- (s-iA,), and we get

r z ['ia'- SA. - A,A,)a- 2A,A.,A,-A, -AoAs')]~ -d(T.

. s

The reader will verify, without difficulty, that

3i4.;- - 4 1 3 and iA- A. A - A - A A

are respectively equal to g. and g-, the invariants of the original
quartic, and so

s=io z;go,g.;).

Now X = Xq + Az \ s - A. ~',

and hence x = Xo + \ f' ( o) z ', 92, Qi) - -hf" ( 'o) ~S

so that X has been expressed as a rational function of ( (z; g.,, g .

* Burnside and Pauton, Tlieonj of Equations, ii. p. 113. + See \hardsubsectionref{21}{7}{3}.

J This substitution is legitimate since A3 + O; for the equation -13 =
involves /(x) = having x - Xq as a repeated root.

%
% 454
%

This formula for cc is to be regarded as the integral equivalent of
the relation

z

Example 1. With the notation of this article, shew that Example 2.
Shew that, if

I a

where a is ani/ constant, not necessarily a zero of f x), and / (x) is
a quartic polynomial with no repeated factors, then

,,, /( ) F( )+i/'W ( -)- V/"( ) +A/( )/'"( )

the function p (z) being formed with the invariants of the quartic
/(:c).

\addexamplecitation{Weierstrass.}

[This result was first published in 1865, in an Inaugural-dissertation
at Berlin by Biermann, who ascribed it to Weierstrass. An alternative
result, due to Mordell, Messengery XLIV. (1915), pp. 138-141 is that,
if

\ C' v y dx - X dy

where /(:f, y) is a homogeneous quartic whose Hessian is h x, y), then
we may take

x=ap' z) ff+ip z)f,+ h

y-hp- z) lf-\ p z)f,-Ua, where /and h stand for /(a, 6) and A (a, 6),
and suffixes denote partial dMferentiations.] Example 3. Shew that,
with the notation of example 2,

(. M /w/( ) +/( ) I /'( ),rw

 'x-af 4(.r-a) 24 '

and F( )= - 1, 3 - rr l /( ) - j/- 3 +/ U /(-) *

* ' x-af A x-a)-y-' ' x-af A x-ay] ' '

\Section{20}{7}{The uniformisation* of curves of genus unity.}
The theorem of §
20*6 may be stated somewhat differently thus: If the variables x and
y are connected by an equation of tlie form y"- = a id + a-i a? + a x
+ a x + a,

then they can be expressed as one-valued functions of a variable z by
the

equations, .,,, . .

  x=x,+\ r x,)\ p z) - j-j" x,yr

y = -U' o)p' z) p z)-i-J" xo)]

where f(x) = a x + 4aia + Qa.,x + 'ia x -t- a, Xq is any zero of f
x), and the function < z) is formed with the invariants of the quartic
; and z is such that

z=r [f t)]- dt.

* This term employs the word uniform in the sense one-valued. To
prevent coufusion with the idea of uniformity as explained in Chapter
in, tliioughout the present work we have used the phrase 'one-valued
function' as being preferable to 'uniform function.'

!07]

%
% 455
%

It is obvious that y is a two-valued function of x and a; is a
four-valued function of y; and the fact, that x and y can be
expi'essed as one-valued functions of the variable z, makes this
variable z of considerable importance in the theory of algebraic
equations of the type considered; z is called the uniformising vm
iahle of the equation

y" = a x* -f- 4ai.'C -f a. x" a x + a .

The reader who is acquainted with the theory of algebraic plane curves
will be aware that they are classified according to their deficiency/
or genus*, a number whose geometrical significance is that it is the
difterence between the number of double points possessed by the curve
and the maximum number of double points which can be possessed by a
curve of the same degree as the given curve.

Curves whose deficiency is zero are called tmicursal curves. If/ (x,
y) = is the equation of a unicursal curve, it is well known t that x
and ?/ can be expressed as rational functions of a 'parameter. Since
rational functions are one-valued, this parameter is a uniformising
variable for the curve in question.

Next consider curves of genus unity; let /(.r, ?/) = be such a curve;
then it has been shewn by CIebsch| that x and y can be expressed as
rational functions of and ?; where ry' is a polynomial in | of degree
three or four. Hence, by \hardsectionref{20}{6}, and r\ can be expressed as rational
functions of (2) and ' z\ (these functions being formed with suitable
invariants), and so x and y can be expressed as one-valued (elliptic)
functions of z which is therefore a uniformising variable for the
equation under consideration.

When the genus of the algebraic curve /(.r, 3/) = is greater than
unity, the uniformi- sation can be effected by means of what are known
as automorphic functions. Two classes of such functions of genus
greater than unity have been constructed, the first by Weber,
Oottinger Nach. (1886), pp. 359-370, the other by Whittaker, Phil.
Trans, cxcii. (1898), pp. 1-32. The analogue of the
period-parallelogram is known as the 'fundamental polygon.' In the
case of Weber's functions this polygon is ' mviltiply-connected,' i.e.
it consists of a. region containing islands which have to be regarded
as not belonging to it; whereas in the case of the second class of
functions, the polygon is ' simply-connected,' i.e. it contains no
such islands. The latter class of functions may therefore be regarded
as a more immediate generalisation of elliptic functions. Cf. Ford,
Introduction to theory of Auto- morphic Functions., Edinburgh Math.
Tracts, No. 6 (1915).

REFERENCES.

K. Weierstrass, Werke, i. (1894), pp. 1-49, 11. (1895), pp. 245-255,
257-309.

C. Briot et J. C. Bouquet, Theorie des fonctions elliptiques. (Paris,
1875.)

H. A. ScHWARZ, Formeln und Lehrsdtze zuni Gehrauche der elliptischen
Funktionen. Nach

Vorlesungen und Aufzeichnungen des Herrn Prof. K. Weierstrass.
(Berlin, 1893.) A. L. Daniels, 'Notes on Weierstrass' methods,'
American Journal of Math. vi. (1884),

pp. 177-182, 253-269; vn. (1885), pp. 82-99. J. LiouviLLE (Lectures
published by C. W. Borchardt), Journal fiir Math. Lxxxviii.

(1880), pp. 277-310. A. Enneper, ElliptiS'-he Funktionen. (Zweite
Auflage, von F. Miiller, Halle, 1890.) J. Tannery et J. Molk,
Fonctions Elliptiques. (Paris, 1893-1902.)

* French genre, German Geschlecht.

t See Salmon, Higher Plane Curves (Dublin, 1873), Ch. 11.

X Journal f fir Math. lxiv. (1865), pp. 210-270. A proof of the result
of Clebsch is given by Forsyth, Theory of Functions (1918), § 248. See
also Cayley, Proc. London Math. Sac. iv. (1873), pp. 347-352 [Math.
Papers, vni. pp. 181-187].

%
% 456
%

Miscellaneous Examples.

1. Shew that

9 'ry)- z-y)=-\&' z) ' y) ip z)- y)]-\

2. Prove that

where, on the right-hand side, the subject of diflferentiation is
symmetrical in 2, y, and w.

\addexamplecitation{Math. Trip. 1897.} ti. Shew that

n -y) r\ y- ' r'o - )

\ 1

55 2

r"(2-y) r"(3/-"') r'( '-2)

  (-- y) (y- O (w'- ) 1 1 1

\addexamplecitation{Trinity, 1898.}, dy

4. If y= (2)-ei, .y'=

shew that y is one of the vahies of

if \ d' \ i 1 i

|/(y-4 2logyj +(ei-e2)(ei-e3)|

\addexamplecitation{Math. Trip. 1897.}

5. Prove that

2 W (2) - (P (y) - P ( ') ' IP (y + ) - 4* (P (y - ' ) - )* = O'

where the sign of summation refers to the three arguments 2, y, v.;
and e is any one of the

roots Ci, Co, So.

\addexamplecitation{Math. Trip. 1896.}

6. Shew that

P' z + <oi)\ (P (i i)-£( i)r

P'( ) ........

\addexamplecitation{Math. Trip. 1894.}

7. Prove that

P (22) - p (a,,) = IP' (2) -2 [ (2) - (|a ) 2 (2) - (a>2 +*o,i) .

\addexamplecitation{Math. Trip. 1894.}

8. Shew that

p u + v)iO(u-v) =

iP (u) p (v) + ig l+giilMllMl

 p u)-p v)

,,.V2

\addexamplecitation{Trinity, 1908.}

9. If p(u) have primitive periods 2a)i, 2( 2 and f(u) = p u) - p co- '
, while g:)i (?0 And/i (m) are similarly constructed with periods
2a)i/yi and 2a)2, prove that

Pi ii) = P ti)+"'2 p u + '2ma>iln) -p 2mcoi/7i),

m=l

n-1 n f u + 2ma)iin)

 and /i (w) = " i

n f 2mailn)

m=r

(Math. Trip. 1914; the first of the formulae is due to Kiepert,
Journal fur Math. Lxxvi. (1873), p. 39.)

%
% 457
%

10. If .r = p u + a), y = <p u-a),

where a is constant, shew that the curve on which (.r, y) lies is

 xy + ex + cy + g. f = 4 ( +y + c) [cxy - lg \ where c = p(2a).

\addexamplecitation{Burnside, Messenger xxi.}

11. Shew that

2 "3 (.0 - Zg.£" u) +gi = 21 ' u)+g,Y.

12. If z=r xi + 6cx + e -)-idx,

verify that x=,

the elliptic function being formed with the roots -c, c + e), (c - e).

\addexamplecitation{Trinity, 1909.}

\addexamplecitation{Trinity, 1905.}

F(i/)i z)-<P y) " '' (2)- (y)

13. If m be any constant, prove that

1 / e' (2)-S'(2')fp(3)(;2, e MS'(2)-S'(y) y

-I? IT

 P( )- r P(y)-er '

where the summation refers to the values 1, 2, 3 of;; and the
integrals are indefinite.

\addexamplecitation{Math. Trip. 1897.}

1 4. Let R x) = Ax + Bx" + Cx"- + Dx + ',

and let |= (.r) be the function defined by the equation

. '=| ( ) - /, where the lower limit of the integral is arbitrary.
Shew that

20' ( ) \ <j>' a- y) + ' ia ) 0' (a-y)+< '(a) \ < '(a+y)-< '( )

( (A'+2/)-0(a) < (o+ )-0(a) (i> a-y)-<f> a) (]i a+y)-(j) x)

0'( -y)-0'(- ')

c ) a-y)-cl) x) '

[Hermite, Proc. J/ / Congress (Chicago, 1896), p. 105. This formula is
an addition-formula which is satisfied by every elliptic function of
order 2.]

15. Shew that, when the change of variables

is applied to the equations

r + .(l+i> ) + l = 0, a.- - = 0, they transform into the similar
equations

Shew that the result of performing this change of variables three
times in succession is a retiirn to the original variables, r; and
hence prove that, if and r) be denoted as functions of by -Ei u) and F
u) respectively, then

where A is one-third of a period of the functions E u) and F u). Shew
that E (u) = -P it; g2, g ),

where 92 = '2p + - p\ 93= -'i-- qP - P -

(De Brun, Ofversigt af K. Vet. Akad., Stockholm, Liv.)

%
% 458
%

16. Shew that

[chap. XX

aud

\&' =

F ( )=

2o- (2 + Wj) (T (2 + <t>2) <T (Z- <i>i - 0)9) O- (s) C (ft)i) (T (0)2)
O" (©i +0)2)

60- (z + a) a z-a)(r z + c) a z - c)

(r*(2)<r2(a)o-2(c)

where

17. Prove that

+ \&'(a-h) C z-a)-a'-b) + C <')-ah)] IS. Shew that

\addexamplecitation{Math. Trip. 1913.}

\addexamplecitation{Math. Trip. 1895.}

2l (M)- (t ) gJ(i;)- (M')J ' ''

\addexamplecitation{Math. Trip. 1910.}

19. Shew that

C( i) + C,W2) + C("3)-r(W] + *2 + 3)

2 (? i) - g> (M2) ( 2) - 9 n )] P ( 3) - P ( <l)

F ( l) W ( 2) - (Ws) + ' ( -2) P ( 3) - ( l) + i;-'' (%) ( l) - ( 2)

\addexamplecitation{Math. Trip. 1912.}

20. Shew that

a x+y+z) o- x-y) <r 1/ - z) a z - x) 1 I 1 P (-0 ' i- )
a3(.r)cr3(y)cr3(2) 2,

1 p z) p' z) Obtain the addition-theorem for the function p z) from
this result.

21. Shew by induction, or otherwise, that

1 ( i) '( i)... <"- )(2i)

\ /\ xiM(n-l)j I 2 ! ... ?i:

, O- (20 + 2 i + . . . + 2n) no- (2; - Z )

'<T z )...a"*' z )

' 1 PizJ P'(2 )...p-1)(2 )

where the product is taken for pairs of all integral values of X and
/i from to 71, such

that X < /I.

(Frobenius u. Stickelberger*, Journal fiir Math, lxxxiii. (1877), p.
179.)

22. Express

1 p x) P( ) p' x) I

1 Piy) F(y) r(3') I 1 (2) P(2) F( ) I 1 p u) p( ) f (w);

as a fraction whose numerator and denominator are products of
Sigma-functions.

* See also Kiepert, Journal filr Math. L.xs.yi. (1873), pp. 21-33;
Hermite, Journal fiir Math. Lxxxn. (1877), p. 346.

%
% 459
%

Deducethat if a = p(.r), 8 = p y), y=p z), 8 = p (ii), where x +
9/+z+u = 0, then ( 2 - 63) (a - ei) (/3 - ei) (y - e ) 8 - ej) + (es -
ei) (a - e ) (/3 - e,) (7 - e ) (S - 63) *

+ ( 1 - ea) (a - 63) (/3 - e ) (y - 63) (8 - 63) * = e - 63) ( 3 - ei)
(ej - eg)-

\addexamplecitation{Math. Trip. 1911.}
23. Shew that

2C(2 )-4aiO = |J' ''

3t(32i)-9C(20 =

\addexamplecitation{Math. Trip. 1905.}

24. Shew that

and prove that a- nu)j cr ( ) "' is a doubly-periodic function of ?6.

\addexamplecitation{Math. Trip. 1912.}

25. Prove that

a- (z-2a + b) a- (z -2b + a)

  z-a)-( z-b)-C(a-b) + C 2a-2h)-

a- (26 - 2a) o- (2 - a) o- ( - 6) '

\addexamplecitation{Math. Trip. 1895.}

26. Shew that, if Sj + S2 + 3 + i = 0, then

 2C (2,.) = 3 2t (2,.) SP (,.) + 2 ' (.-.), the summations being
taken for r = l, 2, 3, 4. \addexamplecitation{Math. Trip. 1897.}

27. Shew that every elliptic function of order n can be expressed as
the quotient of two ex2)ressions of the form

aiPiz + b) + a,p' z + b) + ...+a,,p( -')(z + b),

where b, rtj, 02, ... a are constants. (Painleve, Bulletin de la Soc.
Math, xxvii.)

28. Taking e >e2>e;, p a>) = ei, p(co') = es, consider the values
assumed by

C( )-Mf ( ')/ '

as u passes along the perimeter of the rectangle whose corners are -
co,, w + w', -a> + w

\addexamplecitation{Math. Trip. 1914.}

29. Obtain an integral of the equation

1 d w,, ., - -Ty = 6 (2) + 36

in the form

dzla z)a c) ' \ b-2p c) ' fj'

where c is defined by the equation

(62- 39 2) (0) = 3 (63 + 3).

Also, obtain another integral in the form

 f j exp -sf(ai)-2aa2),

where ( i) + P ( 2) = \&, F ( i) + P ( 2) = 0,

and neither ai + a2 nor ai-a-i is congruent to a period.
\addexamplecitation{Math. Trip. 1912.}

%
% 460
%

30. Prove tliat

, .\ a z + Zi) (r z + Z 2) a- z + Z3) a- (z + Zj)

  '~ ' r 22 + i(2i + 22 + £3 + 24)

is a doubly-periodic function of z, such that

  (2) +5r (2 + coj) + (/ (2+ o).,) + <7 (2+ 0)1 + C02)

= - 20- | (20 + 23 - 2i - 24) or \ (23 + 2i - Zo - 24) (T J (21 + 22 -
23 - 24) .

\addexamplecitation{Math. Trip. 1893.}

31. If $f(z)$ be a doubly-periodic function of the third order, with
poles at 2 = Cj, 2 = 02, 2=C3, and if (2). be a doubly-periodic
function of the second order with the same periods and poles at 2 = 0,
: = 3, its value in the neighbourhood of 2 = being

( z) = + \ \ z-a) + <o z-af - ...,

z - a

prove that

iX-' /" (a) -/" (3) - \ /' ( ) +/' m 2</) ( 1) + / (a) -fm sXXi + 2ct>
c,) cj> (C3)| = 0.

\addexamplecitation{Math. Trip. 1894.}

32. If X (2) be an elliptic function with two poles aj, a, and if z,
z-j,, ... 22n be 2n constants subject only to the condition

Zl + Z2 + ...+Z. = 7l ai + Cto),

shew that the determinant whose ith row is

1, \ \ {Zi), X2(2,.), ... \ \ Zi), X,(2,), X(2i)X,(2i), X2 (2,) Xi
(2,), . .. X -M t) l (2f)

[where Xj (zi) denotes the result of writing 2 for 2 in the derivate
of X (2)], vanishes identically. \addexamplecitation{Math. Trip. 1893.}

33. Deduce from example 21 by a limiting process, or otherwise prove,
that

\ F(z) P" z) ...p-')(2) =(-)"-Ml! 2!... ( -l)! 2cr(;m)/ a-(?0 ' .

p" z) r'(2) -P' H ):

\ 2(01 "'1/ <"! ©i

P-1)(2) p)(2)...F""'K2) I

\addexamplecitation{Kiepert, Journal fur Math, lxxvi.}

34. Shew that, provided certain conditions of inequality are
satisfied,

  z)(T y) ' 2coi

where the summation applies to all positive integer values of m and n,
and j = exp (7ria)2/< i)

\addexamplecitation{Math. Trip. 1895.}

35. Assuming the formula

>)i s? 1 - 2o- cos - + q*" 2a), 2<x)i . nz ° tO] a-(2) = e ' . - sm
r- n 7:r-r,,

prove that

(P (2)= - - + U- cosec2 2 - 2, o cos

when 2 satisfies the inequalities

-2li( )<R( )<2R(?A, \ la>l/ \ i(Oi/ V(Oi/

\addexamplecitation{Math. Trip. 1896.}

%
% 461
%

36. Shew that if 2 is- be any expression of the form 2ma)i + 2n(02 and
if

then X is a root of the sextic

,r - 55'2a;*-405'3A-3 - 5g. x - Sg.2gzX-bg = 0, and obtain all the
roots of the sextic. \addexamplecitation{Trinity, 1898.}

37. Shew that

where

/ . - )(. -.)ri..=-iiog:-|-;-f; i.og:jj >,

(Dolbnia, Darhoux' Bulletin (2), xix.) 38. Prove that every analytic
function (3) which satisfies the three-term equation

2 /(2 + a)/(2-a)/(6+c)/(6-c) = 0,

for general values of, 6, c and, is expressible as a finite
combination of elementary functions, together with a Sigma-function
(including a circular function or an algebraic

function as degenerate cases).

(Hermite, Fonctions elliptiqnes, i. p. 187.)

[Put 0=a = 6 = c=O, and then/(0) = 0; put 6 = c, and then /( -
6)+/(6-(x) = 0, so that/ (2) is an odd function.

If F [z) is the logarithmic derivate of /'(2), the result of
differentiating the relation with respect to 6, and then putting 6 =
c, is

Differentiate with respect to 6, and put 6 = 0; then /(. + a)/( -a) /'
(0)F \

 /( )/(a)F " -

If/' (0) "were zei'o, / ' z) would be a constant and, by integration,/
(2) would be of the form A exp (Bz+Cz ), and this is an odd function
only in the trivial case when it is zero.

If /' (0) 0, and we write F' (s)= - 4> (z), it is found that the
coefficient of a* in the expansion of

l2f z+a)f z-a)/ fiz)Y

is 6 \$ (z) - " (j), and the coefficient of a* in 12 /( ) * (a) - *
(2) is a linear function of * (2). Hence 4>" (2) is a quadratic
function of \$(2); and when we multiply this function by \$' (2) and
integrate we find that

 <!>' (2) 2 = 4 * (2) 3+ 12J * (2)j-'+ 12 \$ (2)4-46',

where A, B, C are constants. If the cubic on the right has no
re2:)eated factors, then, by \hardsectionref{20}{6}, <I> z) = z + a) + A, where a is
constant, and on integration

f z) = (r z + a) exp - Az'-Kz-L),

where K and L are constants; since/ (2) is an odd function a = K=0,
and

/(2) = (r(2)exp -iJ22-Z .

If the cubic has a repeated factor, the Sigma-function is to be
replaced (cf.\hardsubsubsectionref{20}{2}{2}{2}) by the sine of a multiple of z, and if the
cubic is a perfect cube the Sigma-function is to be replaced by a
multiple of 2.]

\chapter{The Theta Functions} 

211. The definition of a Tli eta-function. 

When it is desired to obtain definite numerical results in problems 
involving Elliptic functions, the calculations are most simply performed 
with the aid of certain auxiliary functions known as Thetafunctions. These 
functions are of considerable intrinsic interest, apart from their connexion 
with Elliptic functions, and we shall now give an account of their funda- 
mental properties. 

The Theta-functions were first systematically studied by Jacobi*, who 
obtained their properties by purely algebraical methods ; and his analysis 
was so complete that practically all the results contained in this chapter 
(with the exception of the discussion of the problem of inversion in §§ 21-7 
et seq.) are to be found in his works. In accordance with the general scheme 
of this book, we shall not employ the methods of Jacobi, but the more 
powerful methods based on the use of Cauchy's theorem. These methods 
Avere first employed in the theory of Elliptic and allied functions by Liouville 
in his lectures and have since been given in several treatises on Elliptic 
functions, the earliest of these works being that by Briot and Bouquet. 

[Note. The first function of the Theta-function type to appear in Analysis was the 

Partition function  U (l-.*'" )"  of Euler, Introductio in Anaiysin Infinitorum, i. 

(Lausanne, 1748), § 304; by means of the results given in 5  21-3, it is easy to express 
Theta-functions in terms of Partition functions. Euler also obtained properties of products 
of the type 

n (i±A-"), n (i±a;2 ), n (i± -2 -i). 

n=l n = l n=\ 

The associated series 2 ?n "('*+ \ 2 m-" ""*   and 2 m ' had previously occurred in the 

>i=0 n=0 M=0 

posthumous work of Jakob Bei'nouUi, Ars Conjectandi (1713), p. 55. 

* Fundamenta Nova Theoriae Fimctionum Ellipticarum (Konigsberg, 1829), and Ges. Werke, 
I. pp. 497-538. 

t The Partition function and as-sociated functions have been studied by Gauss, Comm. Soc. 
reg. sci. Gottinnensis rec. i. (1811), pp. 7-l'2 [Werke, ii. pp. 16-21] and Werke, iii. pp. 433-480 and 
Cauchy, Coviptes liendus, x. (1840), pp. 178-181. For a discussion of properties of various functions 
involving what are known as Basic  lumbers (which are closely connected with Partition functions) 
see Jackson, Froc. Roijal Soc. Lxxiv. (1905), pp. 64-72, Froc. London Math. Soc. (1) xxviii. (1897), 
pp. 475-486 and (2) i. (1904), pp. 63-88, ii. (1904), pp. 192-220; and Watson, Camb. Fhil. Trails. 
XXI. (1912), pp. 281-299. A fundamental formula in the tlieory of Basic numbers was given by 
Heine, Kugelfunktionen (Berlin, 1878), i. p. 107. 



2ri, 2ril] THE THETA FUNCTIONS 463 

Theta-functions also occur in Fourier's La Theorie Analytique de la Chaleur (Paris, 
1822), cf. p. 265 of Freeman's translation (Cambridge, 1878). 

The theory of Theta-functions was developed from the theory of elliptic functions 
by Jacobi in his Fundameata Nova Theoriae Functionum Ellipticarum (1829), reprinted 
in his Ges. Werke, i. pp. 49-239; the notation there employed is explained in § 21-62. 
In his subsequent lectures, he introduced the functions discussed in this chapter ; an 
account of these lectures (1838) is given by Borchardt in Jacobi's Ges. Werl-e, i. pp. 497-538. 
The most important results contained in them seem to have been discovered in 1835, 
cf. Kronecker, Sitzungsherichte der ALad. zu Berlin (1891), pp. 653-659.] 

Let T be a (constant) complex number whose imaginary part is positive ; 
and write q = e' '", so that q\ < l. 

Consider the function   z, q), defined by the series 

00 

qua function of the variable z. 

If A be any positive constant, then, when \ 2\ \  A, we have 

I qII' Q-k2.niz i < I (7 i '*'" QinA 

n being a positive integer. 

00 

Now d'Alembert's ratio (§ 2'36) for the series S | q ["'e ji  is j q |2n+ig2  

  = — 00 

which tends to zero as ?i   x . The series for    z, q) is therefore a series of 
analytic functions, uniformly convergent (§ 334) in any bounded domain of 
values of vS", and so it is an integral function (§§ 5"3, 5*64). 

It is evident that 

 70 

  ( , 5) - 1 + 2 S  -y'q" ' COS 2nz, 





 =i 


and that 


 (  + 7r,g) =  (ir, g); 


further 


'  z + 'TTT,q)= S (\ ) 5"- 2ng2nw 

 - - 00 



-- \  Q—1 Q—2iz y (\ \ n+iQ(n+i)-Q'i n+ ) iz 

and SO    z + ttt, q) = ~ q~  e~ '  '  (z, q). 

In consequence of these results,   z, q) is called a quasi doubly -periodic 
function of z. The effect of increasing 2  by tt or ttt is the same as the effect 
of multiplying    z, q) by 1 or — q~ e~' , and accordingly 1 and — q~ e~'   are 
called the multipliers or periodicity factors associated with the periods ir and 
TTT respectively. 

21'11. The four types of Theta functions. 

It is customary to write  4  z, q) in place of    z, q) ; the other three 
types of Theta-functions are then defined as follows : 



464 THE TRANSCENDENTAL FUNCTIONS [CHAP. XXI 

The function  3(2',  ) is defined by the equation 

/ 1 \ "  , 

 3 ( , g) =  4 (   + 2 TT, 9 1 = 1 + 2 1 7"- COS 2)12. 

Next, %(z, q) is defined in terms of  4(2 , q) by the equation 

 i = — 00 

and hence* % 2, q) = 2 t (-)"(/(" + *)' sin (2 n + l)z. 

71 = 

Lastly,  2 ( . q) is defined by the equation 

' ., z,q  = %(z + l7r,q) = 2 t q'<" +  - ' cos 2n+  ) z. 

V   / H=0 

Writing down the series at length, we have 

 1 ( ) q) =  q* sin z - 2q* sin 3  + 25"*"" sin bz — ..., 
 2  z, q) = 2q* cos z + 2g' cos Sz + 2q' '' cos 5  + . . . , 
 3 (z, fy) = 1 + 2 ' cos 2z + 2q* cos 4  + 29  cos 62 + ... , 
 4 ( >  ) = 1 ~ 2g cos 22 + 2q* cos 4  — S *" cos Qz + — 

It is obvious that  j (2, q) is an odd function of 2 and that the other 
Theta-functions are even functions of z. 

The notation which has now been introduced is a modified form of 
that employed in the treatise of Tannery and Molk ; the only difference 
between it and Jacobi's notation is that  4 (2, q) is written where Jacobi 
would have written   (2, q). There are, unfortunately, several notations in 
use ; a scheme, giving the connexions between them, will be found in § 2r9. 

For brevity, the parameter q will usually not be specified, so that  1 (2), ... 
will be written for  1 (2, q), .... When it is desired to exhibit the dependence 
of a Theta-function on the parameter r, it will be written   (2 j t). Also 
 2(0),  3(0), ' 4(0) will be replaced by  2>  3, ' 4 respectively; and  / will 
denote the result of making 2 equal to zero in the derivate of  1(2). 
Example 1. Shew that 

3i z, q) = 9s 2z, q )-h 2z, q*). 
Example 2. 01)tain the results 

Bi z)= -B.2 z + i ) =-iMB-i z +  n + UT)=-iMSi z +  !TT\ 

\$2(2)= MS3 z + Ut)= mi z +  \ n +  nr)= S Z + hTr), 

 3(2)= 3i z+U) = .l/ ,(3 + W+i7rr)= m, z + UT), 

3i (2) = - iJfSi (2 +  777") = im.2  z+U+ i rr) =  3 (2 + U), 
where M=q* e". 

* Throughout the chapter, the many-valued function q  is to be interpreted to mean 
exp (Xttit). 



21-12] 



THE THETA FUNCTIONS 



465 



Example 3. Shew that the multipliers of the Theta-functions associated with the 
periods tt, ttt are given by the scheme 





S,iz) 


 2 Z) 


h z) 


3, z) 


n 


-1 


- 1 

N 


1 


1 


TTT 


-N- 


iV" . 


-N 



where iV = q~ e '  . 

Example 4. If -9 (2) be any one of the four Theta-functions and d' (2) its derivate with 
respect to z, shew that 



 9'(2+7r) \ .9'(2) 
 (0 + ,r) ~T £)' 






21'12. TAe  reros of the Theta-functions. 

From the quasi-periodic properties of the Theta-functions it is obvious 
that if    z) be any one of them, and if  o be any zero of    z), then 

z  - - riiTT + niTT 

is also a zero of    z), for all integral values of in and n. 

It will now be shewn that if G be a cell with corners i, t -f- vr,   -I- vr + ttt, 
t -  TTT, then    z) has one and only one zero inside G. 

Since    z) is analytic throughout the finite part of the  -plane, it follows, 
from § 6'31, that the number of its zeros inside G is 



 ' z) 



dz. 



27riJc  ( ) 
Treating the contour after the manner of § 2012, we see that 
1 f " '(z) 



Inriir   ( ) 



dz 



1 /• +-| ( )  '(  + 7 rT)) \ J\  
27rtj, l ( )  (s-f-TTT)  " "  27n 






dz 



t + T 



2idz, 



27rij t 
by § 21*11, example 4. Therefore 

1 r y( ) 



27ri J c   ( ) 



C?  = 1, 



that is to say,    z) has one simple zero only inside G ; this is the theorem 
stated. 



W. M. A. 



30 



466 THE TRANSCENDENTAL FUNCTIONS [cHAP. XXI 

Since one zero of  i  z) is obviously z = 0, it follows that the zeros of 
 i( ), %(z), %, z),  4( ) are the points congruent respectively to 0,  , 

Itt-I-Ittt, Ittt. The reader will observe that these four points form the 
corners of a parallelogram described counter-clockwise. 

21-2. The relations between the squares uf the Theta-f unctions. 

It is evident that, if the Thcsta-functions be regarded as functions of a 
single variable z, this variable can be eliminated from the equations defining 
any pair of Theta-functions, the result being a relation* between the functions 
which might be expected, on general grounds, to be non-algebraic; there 
are, however, extremely simple relations connecting any three of the Theta- 
functions ; these relations will now be obtained. 

Each of the four functions  i"  z), "   z),  3-  z), V  z) is analytic for all 
values of z and has periodicity factors 1, q-" e-*  associated with the periods 
IT, TTT ; and each has a double zero (and no other zeros) in any cell. 

From these considerations it is obvious that, if a, b, a' and b' are suitably 
chosen constants, each of the functions 

aV(g)-h6V ( ) a'% '(z)+b'X'(z) 
V( ) ' V( ) 

is a doubly-periodic function (with periods tt, ttt) having at most only a 
simple pole in each cell. By § 20-13, such a function is merely a constant; 
and obviously we can adjust a, b, a, b' so as to make the constants, in each 
of the cases under consideration, equal to unity. 

There exist, therefore, relations of the form 

     z) = a%' (z) + 6V (z), %' (z) = a%' (z) + b'X' (z). 

To determine a, b, a, b' , give z the special values   ttt and ; since 

we have ' 3- = — (( 4-,  2" =  4"; %- = — a" , ' 3- = 6" 4-. 

Consequently, we have obtained the relations 

X'  z)  4' =  4   z) X' - X' (z) %', %  (z) %' = X'  z) %' - %'  z)  ,1 
If we write z + - ir for z, we get the additional relations 

%' (z) V = %' (z) %' -  .  (z)  3% V (z) V = %' (z) V - X' (z) V. 
By means of these results it is possible to express any Theta-function in 
terms of any other pair of Theta-functions. 

* The analoo;ou.s relation for the fuuctions sinz and cos 2 is, of course, (sin2)"''+(cos,:)2= 1. 



21-2 — 21-22] THE THETA FUNCTIONS 467 

Corollary. Writing z=0 in the last relation, we have 
that is to say 

21*21. The addition-formulae for the Theta functions. 

The results just obtained are particular cases of formulae containing two 
variables ; these formulae are not addition-theorems in the strict sense, as 
they do not express Theta-functions of 2 + y algebraically in terms of Theta- 
functions of z and y, but all involve Theta-functions of z — y as well as of 
z - -y, z and y. 

To obtain one of these formulae, consider  3  z -h y) ' 3  z — y) qua function 
of z. The periodicity factors of this function associated with the periods tt 
and TTT are 1 and (f  e-2''2+!/) . q-  Q-iiKz-y) = q-2  -Hz  

But the function a j-  z) + 6 1-  z) has the same periodicity factors, and 
we can obviously choose the ratio a:b so that the doubly -periodic function 

a%' z) +  fh Hz) 
% z + y)% z-y) 
has no poles at the zeros of  3  z — y) ; it then has, at most, a single simple 
pole in any cell, namely the zero of  3(2:4- y) in that cell, and consequently 
(§ 20'13) it is a constant, i.e. independent of z ; and, as only the ratio a : 6 is 
so far fixed, we may choose a and b so that the constant is unity. 
We then have to determine a and b from the identity in z, 

a%' (z) + b -' (z) =%(z-h y) X (z - y). 
To do this, put z in turn equal to and - ir + -  ttt, and we get 

aX' =  Hy\ h  '( 7r +  rTT  = %  ir + \ 7rr + y)%  'rr +  rrT-y 

and so a =  3-  y)l . , b = " i   y)l 3\ 

We have therefore obtained an addition-formula, namely 

 3 (  + y) %  z - y)  / = V  y) V  z) +  i  iy) X'  z). 

The set of formulae, of which this is typical, will be found in examples 1 
and 2 at the end of this chapter. 

21 '22. Jacobi's fundameMal formulae *. 

The addition-formulae just obtained are particular cases of a set of identities first given 
by Jacobi, who obtained them by purely algebraical methods ; each identity involves as 
many as four independent variables, w, x, y, z. 

Let iv', x\ y\ z' be defined in terms of lu, x, y, z by the set of equations 

2w' = —w- x- y- z  
2x' = iv — x+y + z, 
2?/' = w+x — y + z, 
22' = tv- x- y — z. 

* Ges. Werke, i. p. 505. 

30—2 



463 



THE TRANSCENDENTAL FUNCTIONS 



[chap. XXI 



The re<\ der will easily verify that the connexion between w, x, y, z and iv\ x\ y\ z' is a 
reciprocal one*. 

For brevity t, write [? ] for 5, (w) 5  (.r) 5   y) 3   z) and [r]' for S, (to') \$, (x') \$, (if) S, (z'). 

Consider [3], [1]', [2]', [3]', [4J qua functions of z. The effect of increasing   by tt or nr 
is to transform the functions in the first row of the following table into those in the second 
or third row respectively. 





[3] 


[1]' 


[2]' 


[3]' 


[4]' 


in) 


[3] 


-[ J 


-[I]' 


[4]' 


[3]' 


(rrr) 


 [3J 


-iy[4]' 


iV[3]' 


iV[2]' 


-.V[l]' 



For brevity, iV has been written in place of q~  e~- . 

Hence both -[l]' + [2]' + [3]'-|-[4]' and [3] have periodicity factors 1 and N, and so 
their quotient is a doubly-periodic function with, at most, a single simple pole in any cell, 
namely the zero of  3 (z) in that cell. 

By § 20-13, this quotient is merely a constant, i.e. independent of 2; and considerations 
of symmetry shew that it is also independent of w, x and y. 

We have thus obtained the result 

J[3]=-[l]' + [2]' + [3]' + [4]', 

where A is independent of w, x, y, z; to determine A put w=x=y=z=  and we get 

J g - .-' + V +  i  
and so, by § 21-2 corollary, we see that  4 = 2. 

Therefore 2 [3]= -[l]' + [2]' + [3]' + [4j' (i). 

This is one of Jacobi's formulae ; to obtain another, increase if, .r, y, z (and therefore 
also tp', x\ y', z) by  w ; and we get 

2[4] = [lJ-[2]' + :3]' + [4]' (ii). 

Increasing all the variables in (i) and (ii) by htrr, we obtain the further results 

2[2] = [l]' + [2]' + [3]'-[4]' (iii), 

2[l] = [l]' + [2]'-[3]' + [4j (iv). 

[Note. There are 256 expressions of the form dp (ic) 3g (x) S  (y) 3  (2) which can be 
obtained from  3 (w)  3 (x)  3 (y)  3 (2) by incre;ising w, x, y, z by suitable half-period.s, but 
only those in which the suffixes p, q, r, s are either equal in pairs or all different give rise 
to formulae not containing quarter-periods on the right-hand side.] 

Example 1. Shew that 

[1] + [2] = [!]' + [2]', [2]-h[3] = [2]' + [.3]', [l]-h[4]=.[l]'-f [4]', [3] -h [4] = [3]' -h [4]', 

[l] + [3] = [2]' + [4]', [2]-f[4] = [l]'-h[3]'. 

In Jacobi's work the signs of u-, .r', y', z' are changed throughout so that the complete 
symmetry of the relations is destroyed ; tlie symmetrical forms just given are due to H. J. S. Smith, 
Proc. London Math. Soc. 1. (May 21, 1860, pjx 1-12). 

t The idea of this abridged notation is to be traced in H. J. S. Smith's memoir. It seems, 
however, not to have been used before Kronecker, Journal j'ilr Math. cii. (1887), pp. 260-272. 



21-3] THE THETA FUNCTIONS 469 

Example 2. By writing tv +  n, x + \ tv for iv, x (and consequently  /- \ i , s' +  tt 
for y\ /), shew that 

[3344] + [2211] = [4433]' + [1122j, 

where [3344] means  3  w) S3 (x) S   y)  4 (z), etc. 
Example 3. Shew that 

2[1234] = [3412]'+[2143]'-[1234]' + [4321]'. 
Example 4. Shew that 

21"3. Jacobi's expressions for the Theta-f unctions as infinite products*. 
We shall now establish the result 

M = l 

(where G is independent of z), and three similar formulae. 
Let fi2)= n (1 -  n-i e-- '>) n  l-q->'- e--''); 

each of the two products converges absolutely and uniformly in any bounded 

domain of values of z, by § 3'341, on account of the absolute convergence of 
00 
V  2n-i. hence / (2:) is analytic throughout the finite part of the 2 -plane, 

and so it is an integral function. 

The zeros of/(2') are simple zeros at the points where 

g2iz = g(2H+l) T  (w=..., -2, - 1,0, 1,2, ...) 

i.e. where 2iz = (2?i + 1) ttit + 2ni7ri; so that f(z) and  4 (z) have the same 
zeros; consequently the quotient ' i z)/f z) has neither zeros nor poles in 
the finite part of the plane. 

Now, obviously / (2  + tt) =f z) ; 

GO QO 

and f z+'77r)= IT (1 - 5- +ie-'' ) IT (1 - 52 -3 g-2fe) 

  = 1 n = \ 

=f z) l-q-U- )!  qe  ) 

= -q- e'- f z). 
That is to say f z) arid  i z) have the same periodicity factors (§ 2 I'll 
example 3). Therefore  i z)/f(z) is a doubly-periodic function with no 
zeros or poles, and so (| 20*12) it is a constant G, say; consequently 

 4 (z)=G U  1- 2q"'"-' cos 22 + 5 "--). 
11=1 

00 
[It will appear in § 2142 that G= U (1 - q-'').] 

n = l 

Write z +  TT for 2  in this result, and we get 

%,(z)=G n (1 + 25 ' -! cos 2z + q' '-'). 

n=l 
* Cf. Fundamenta Nova, p. 145. 



470 THE TRANSCENDENTAL FUNCTIONS [CHAP. XXI 

Also  1 (z) = - iq  e"  ,   +   ttt) 

00 3C 

= \  igi e'z G IJ (1 - 9=" e-'') TI (1 - q'''~"- e -'''') 

n=l   = 1 

= 26 5* sin   11 (1 - f"e-") U  I -  e'-'O, 

w = 1   = 1 

and so  , (z) = 2Gq  sin   ft ( I - 25-" cos 2z + q' ) 

n = \ 

while ' , z)=' Jz + l7r] 

= 2Gq  cosz U  1 + 2q-'' cos 2z +  "). 

n = l 
Example. Shew that* 

( cc 18 (-00 ISfoc 18 

J n (i-?2 -i)i +16?- n (i+?2 )l = ' n (i+92 -i)  . 

bi = l J 'n=l J ln = l J 

(Jacobi.) 

21"4. TZ/e differential equation satisfied hy the Theta-functions. 

We may regard  s(z t) as a function of two independent variables z 
and t; and it is permissible to differentiate the series for  3(2 |t) any 
number of times with regard to z or r, on account of the uniformity of 
convergence of the resulting series (§ 4*7 corollary) ; in particular 

—   '   = — 4 2 n- exip n-7nT + 2mz) 

OZ ft = — 00 

Consequently, the function ' 3 (2 \ t) satisfies the partial differential equation 

1 .d y dy   

The reader will readily prove that the other three Theta-functions also 
satisfy this equation. 

21*41. A relation between Theta-functions of zero argument. 

The remarkable result that 

V(0) =  2 (0) 3 (0) 4(0) 

will now be established f. It is first necessary to obtain some formulae for 
differential coefficients of all the Theta-functions. 

* Jacobi describes this result (Fund. Nova, p. 90) as 'aequatio identica satis abstrusa.' 

t Several proofs of this important proposition have been given, but none are simple. 

Jacobi's original proof (Gfis. Werke, i. pp. 515-517), though somewhat more difficult than the 

proof given here, is well worth study. 



21-4, 21-41] 



THE THETA FUNCTIONS 



471 



Since the resulting series converge uniformly, except near the zeros of 
the respective Theta-functions, we may differentiate the formulae for the 
logarithms of Theta-functions, obtainable from § 21"3, as many times as we 
please. 

Denoting differentiations with regard to z by primes, we thus get 



%' z) = X z) 



1—1 a—'iiZ 






L =i (1 + q e ) n=\ 

Making 2r   0, we get 



a%  (>2n— 1 g-2i3 



=1 (1 + g "-i e-- )- 



V (0) = 0,  3" (0) = - 8 3 (0) J  (3 - .  



In like manner. 



V(0)-0, V(0) = 8 4(0) 2 



 2' (0) = 0,  2" (0) =  2 (0) 



..=i(] -cf- r 



-1-8 S 



 =i(l + <? '?J 



and, if we write  1  z) = sin z .   (z), we get 

</)'(0) = 0, f (0) = 8(/>(0) i  '" 

If, however, we differentiate the equation  1 (z) = sin z . cj) (z) three times, 
we get 



 / (0) = (/> (0),  /" (0) = S<p" (0) - </> (0). 



Therefore 



V"(0) 
V(0) 



= 24 2 



=1 (1 - q' y 



-1; 



and 



V(0) v:iO) , V:(0) 

" ,(0)   Sf3(0)  4(0) 



 2?l 00  ,2/1—1 CO  2W— 1 



8-2   2   + 2   

L  =i (1 + q' 'f .=1 (1 + q' '-'r .=1 (1 - ?''"-'/ 



= 8 



\  V 



+ 2 



- 2 



.=1 (1 + q 'T- .=1 (1 - q y n=i (1 - q y 
on combining the first two series and writing the third as the difference of 
two series. If we add corresponding terms of the first two series in the last 
line, we get at once 

V(0) 



 V(0) V10)\  Vi0) 2  



= 1 + 



 ,(0) %(0) %iO) n=i l~q'''y V(0) 



472 THE TRANSCENDENTAL FUNCTIONS [cHAP. XXI 

Utilising the differential equations of § 21-4, this may be written 

1 d%' (0 I t) 
V(0|t) dr 

\  1 d%(0\ T) 1 d% 0\ r) 1 d%(0\ r) 

~ 2(0|t) dr " 3(0|t) dr %(0\ t) dr 

Integrating with regard to r, we get 

V (0, q) = C% (0, q) % (0, q) % (0, q), 

where C is a constant (independent of q). To determine C, make q- 0; since 

\ imq- X = % \ imq-- % = 2 lim 3 = l, lim 4 = l, 

q O q O q -O (/ -O 

we see that = 1; and so 

which is the result stated. 

21-42. The value of the constant G. 

From the result just obtained, we can at once deduce the value of the 
constant G which was introduced in § 21-3. 
For, by the formulae of that section, 

 / = 0(0)= 2q  GU 1- q' % % = 2qiGU l + f  )  

M=l n=l 

 3 = G  n (1 + r''-')\ x = GU i- q ' -'f, 

and so, by | 21-41, we have 

00 O) OO CO 

n (1 - (f y = G  n (1 + q;"')' n (i + q '- y n (i - q  -y. 

Now all the products converge absolutely, since \ q\ < l, and so the 
following rearrangements are permissible : 

I n (1 - g "-') n (1 - ? 'ol • I fi (1 + (t-') n (1 +  'ol 
= n (i-9'O n (i + g* ) 

M=l W=l 

= n (1 - g * ), 

M = l 

the first step following from the consideration that all positive integers are 
comprised under the forms 2n — 1 and 2n. 
Hence the equation determining G is 

n (1 - (f'J = G\ 



n=\ 



andso G=+ n (1 - r/  



21*42 — 21-5] THE THETA FUNCTIONS 473 

To determine the ambiguity in sign, we observe that G is an analytic 
function of q (and consequently one-valued) throughout the domain \ q\ < \; 
and from the product for  3(2 ), we see that G- 1 as q—>0. Hence the 
plus sign must always be taken ; and so we have established the result 

G=Yl (l- 'O- 

Example 1. Shew that  1=22*0 . 

Example 2. Shew that 

Example 3. Shew that 

1 + 2 i y -= n  (i-(72 )(i+g2'i-i)2). 

)! = 1 n = l 

21  S. Connexion of the Sigma-f unction with the Theta- functions. 

It has been seen    20-421 example 3) that the function a- [z \ wx, M2), formed with 
the periods 2a)i, 2co2, is expressible in the form 

where (/=exp (ttiojo/wi). 

If we compare this result with the product of § 21 '4 for  1 [z \ t), we see at once that 

.(.) =  exp( Y ,-in(l-./ )-3i( |- ). 
TT \ 2coi/ 2  =i   \ 2coi I COj/ 

To express j i in terms of Theta-functious, take logarithms and differentiate twice, 
so that 



 < )=:i-(0— =(e,)-( .T 






.4>' 

where v =  7rzja)  and the function cf) is that defined in § 21"41. 

Expanding in ascending powers of z and equating the terms independent of z in this 
result, we get 



a>i 3 \ 2(oiJ \ 2(Ui/ (f) (0) 

and SO ,;=\ —— -— . 

Lzooi  i 

Consequently a-  z \ wi, wo) can be expressed in terms of Theta-functions by the 
formula 

'031 / I'-. l \ o / I  "2  



mi 



a( lo, co,)=- ,exp( --g J5,( v 

where v hivzlwi. 

Example. Prove that 

/n'-a-iBi" 7n'\ 

21"5. The expression of elliptic functions hy means of Theta-functions. 

It has just been seen that Theta-functions are substantially equivalent 
to Sigma-functions, and so, corresponding to the formulae of §§ 20'5-20"58, 
there will exist expressions for elliptic functions in terms of Theta-functions. 



474 THE TRANSCENDENTAL FUNCTIONS [CHAP. XXI 

From the theoretical point of view, the formulae of §§ 20"5-20"53 are the 
more important on account of their S3'mmetry in the periods, but in practice 
the Theta-function formulae have two advantages, (i) that Theta-functions 
are more readily computed than Sigma-functions, (ii) that the Theta- 
functions have a specially simple behaviour with respect to the real period, 
which is generally the significant period in applications of elliptic functions 
in Applied Mathematics. 

Let f z) be an elliptic function with periods 2wi, 2\&J2; let a fundamental 
set of zeros (aj, Oa, ... an) and poles (/3j, /3..., ... /3 ) be chosen, so that 



as in § 20o3. 

Then, by the methods of § 20"53, the reader will at once verify that 

TTZ — TTCUr 1 W.jX TTZ — TTySr t  Wg 

r=\ (.   

where A  is a constant ; and if 



f z) = A,\ \ \ X[— - -'~' \ \  %  .  

' V Iw, (jdJ \ 2\&), CO 



1)1). 

m = l 

be the principal part oi f z) at its pole  , then, by the methods of §20-')2, 

r=i (, =.! (m-1)! dz"" " V 2\&)i \&)i/J 
where J. 2 is a constant. 

This formula is important in connexion with the integration of elliptic 
functions. An example of an application of the formula to a dynamical 
problem will be found in § 22741. 

Example. Shew that 

  3  (2)\  \  \ \     - l' ( ) , - 3 - 3" 

5i2 (2) '    dz .9i (2)  i'3 ' 
and deduce that 

21'51. Jacohis imaginary transformation. 

If an elliptic function be constructed with periods 2\&)i, 2w2, such that 

/ (( 2/a)i) > 0, 
it might be convenient to regard the periods as being 'Iw , — 2\&)i : for these 
numbers are periods and, if I (co-i/coi) >0, then also /(— oji/wo)> 0. In the 
case of the elliptic functions which have been considered up to this point, 
the periods have appeared in a symmetrical manner and nothing is gained 
by this point of view. But in the case of the Theta-functions, which are 
only quasi-periodic, the behaviour of the function with respect to the real 
period tt is quite different from its behaviour with respect to the complex 
period ttt. Consequently, in view of the result of § 21"43, we may expect to 



21-51] THE THETA FUNCTIONS 475 

obtain transformations of Theta-functions in which the period-ratios of the 
two Theta-functions involved are respectively t and — l/r. 

The transformations of the four Theta-functions were first obtained by 
Jacobi*, who obtained them from the theory of elliptic functions ; but Poissonf 
had previously obtained a formida identical with one of the transformations 
and the other three transformations can be obtained from this one by ele- 
mentary algebra. A direct proof of the transformations is due to Landsberg, 
who used the methods of contour integration ij:. The investigation of Jacobi's 
formulae, which we shall now give, is based on Liouville's theorem ; the precise 
formula which we shall establish is 

where (- ir)'   is to be interpreted by the convention arg(- iV) <  '  

For brevity, we shall write - 1 /t = r', q — exp  ttW). 

The only zeros of ' 3  z \ t) and ' 3  t z j t) are simple zeros at the points 
at which 

1 1 / / ,,,1,1/ 

z = mir -  UTTT + 2 '"' + o '""''' TZ = mir + tiTTT 4-   vr 4- .3 ttt 

respectively, where m, n, m, n take all integer values; taking m =- )i — \, 
n = m, we see that the quotient 

is an integral function with no zeros. 

A 1 . / X . X (IZTTT + 7r-T-\ \  \  . , 

Also   z + ttt)  ylf z) = exp ( -. j  q  e -'  = 1, 

while yjr (z - 'tt) -  yjr ( z) = exp ( ; j x q~ e~-' '' = 1. 

Consequently -v/r (z) is a doubly-periodic function with no zeros or poles ; 
and so (§ 20'12) i/ ( ) must be a constant, A (independent of 2). 

Thus  3 (z ! t) = exp (tW/7r)  3 ( r \ r') ; 

and writing z +  -tt,   +   ttt,   -t- - tt -i-   ttt in turn for z, we easily get 
 , (z : t) = exp  irz-'/Tr) % (zr I r), 

A% (z\ t)= exp (itZ /tt) % (ZT I t), 

J.' i (z\ t)= — i exp  Wz-j-n) % (zt \ t'). 

* Journal fur Math. in. (1828), pp. 403-404 [Ges. Werke, i. (1881), pp. 264-265]. 

t Mem. dc VAcad. des Sci. vi. (1827), p. 592; the special case of the formula in which z-0 
had been given earlier by Poisson, Journal de VEcole polyteclmique, xii. (cahier xix), (1823), 
p. 420. 

+ This method is indicated in example 17 of Chapter vi, p. 124. See Landsberg, Journal fiir 
Math. CXI. (1893), pp. 234-253. 



47 G THE TRANSCENDENTAL FUNCTIONS [CHAP. XXI 

We still have to prove that A = — it)- ; to do so, differentiate the last 
equation and then put 2 = 0; we get 

 /(OJT) = -iVV(0 t'). 
But %' (0 t) =  , (0 j t)  3 (0 i t)  4 (0 1 t) 

and  / (0 : t') =  , (0 : r')  3 (0 \ r)  , (0 t') ; 

on dividing these results and substituting, we at once get A~'- = — W, and so 

A = ± -iT) -. 
To determine the ambiguity in sign, we observe that 
 3(0 t)= 3(0|t'), 

both the Theta-functions being analytic functions of t when 7 (t) > ; 
thus A is analytic and one-valued in the upper half r-plane. Since the 
Theta-functions are both positive when t is a pure imaginary, the plus sign 
must then be taken. Hence, by the theory of analytic continuation, we 
always have 

A = +  - ir)  ; 

this gives the transformation stated. 
It has thus been shewn that 

X 1 °  

5" on-Trir+tniz \  'V As -mr]- 1  1:17) 



7j= -oe 

Example 1. Shew that 

when TT = -.  

Example 2. Shew that 



B, Q\ t) \  %AO\ t') 
53(0|r) 53(0|r') 



MOPr + 1)  i  2iOJjr) 
53(0 I r + 1) 54(0|r)' 



Example 3. Shew that 

and shew that the plus sign should be taken. 

21"52. Landens type of transformation. 

A transformation of elliptic integrals (§ 227), which is of historical 
interest, is due to Landen (§ 22-42); this transformation follows at once 
from a transformation connecting Theta-functions with parameters t and 2t, 

namely 

X  z\ t)X  z I t)  3(0 It)  4(0 JT) 
 4(22|2t)  4(0|2t) 

which we shall now prove. 

The zeros of   z r) i z\ r) are simple zeros at the points where 

z = [m + -\ IT -  \ n + -A TTT and where z = imr +in + -Airr, where m and n 



21-52, 21-6] THE THETA FUNCTIONS 477 

take all integral values ; these are the points where 1z = mir + Ui Ar - ir .'Ir, 
which are the zeros of  4 (2  j 2t). Hence the quotient 

 4 (2  I 2t) 
has no zeros or poles. Moreover, associated with the periods it and ttt, it 
has multipliers 1 and  cf-  e"-'' )   - q-  e''  ) -   - q~~"' e- " ) = \ \ it is therefore 
a doubly-periodic function, and is consequen-tly (§ 20"12) a constant. The 
value of this constant may be obtained by putting z —   and we then have 
the result stated. 

If we write \ \   +;- TTT for z  we get a corresponding result for the other 

Theta-functions, namely 

 ,( |t) i( |t)  3 (0 1 )>4 (OK) 

 i(2 |2r) '"" 4(0i2T) 

21-6. The differential equations satisfied hy quotients of Theta-functions. 
From § 21-11 example 3, it is obvious that the function 

has periodicity factors - 1, + 1 associated with the periods tt, ttt respectively; 
and consequently its derivative 

[X  z)  4  z) - %' ( )  1 ( )l - V ( ) 
has the same periodicity factors. 

But it is easy to verify that  .(z) ,, z)/ J  (z) has periodicity f;xctors - 1, 
+ 1 ; and consequently, if <  (z) be defined as the quotient 

 X (z) % (z) - X ( )  1 i )] - [% (z) % (z) , 
then (f) (z) is doubly-periodic with periods tt and ttt ; and the only possible 
poles of (f) (z) are simple poles at points congruent to   tt and   vr -h .  ttt. 



Now consider cf) iz + I ttt] ; from the relations of § 21-11, namely 

% Z + l7rT =iq-h-''X z\ ' ,( z + l7rT'j = iq-ie-''%(z), 

%\ z+l7rT =q-ie'''X(z), x z- l'rrT =q~ e-''% z), 
we easily see that 

(  (  -i-   ttt) =  - X (z)  1 (z) + X ( )  4 ( )l - [% (z) % (z) . 
Hence   (z) is doubly-periodic with periods tt and -  ttt ; and, relative to 



these periods, the only possible poles of   z) are simple poles at points 
1 
2 



congruent to   vr 



478 THE TRANSCENDENTAL FUNCTIONS [CHAP. XXI 

Therefore ( 20-12), </)( ) is a constant; and making z- 0, we see that 
the value of this constant is [ i' ' 4  -  1 2' a  =  Z- 

We have therefore established the important result that 

writing   = * i ( )/ 4  2) and making use of the results of § 21 2, we see that 

(§)' =  -  ~  ' ''  ' "  '  '' ' 

This differential equation possesses the solution  i( )/ 4(2). It is not 
difficult to see that the general solution is ±% z + a)/%(z + a) where a 
is the constant of integration ; since this quotient changes sign when a is 
increased by tt, the negative sign may be suppressed without affecting the 
generality of the solution. 

Example 1. Shew that 

dz [Si  z;j 2 S4 (2)  4 (z) • 

Example 2. Shew thcat • 

lPiii)l=\ q2 ii?) Mi) 

21*61. The genesis of the Jacohian Ellip.tic function* snu. 
The differential equation 

(S)'   '' "  ' '  '' "  ' '' ' 
which was obtained in § 21-6, may be brought to a canonical form by a slight 
change of variable. 

Writet  %/% = y,  V =   ; 

then, if A:- be written in place of  2/ 3, the equation determining y in terms 
of M is 

(|y = a-/)(i-A.y). 

This differential equation has the particular solution 

The function of u on the right has multipliers -1,4-1 associated with 
the periods 7r%", irT -,;-; it is therefore a doubly-periodic function with 
periods 27r 3  ttt -. In any cell, it has two simple poles at the points 
congruent to hirr a- and tt -J  + iTTT j- ; and, on account of the nature of the 
quasi-periodicity of y, the residues at these points are equal and opj3osite in 
sign ; the zeros of the function are the points congruent to and tt sI 

* Jacobi and other early writers used the notation sin am in phice of sn. 

t Notice, from the formulae of § 21-3, that  2 + 0,  3 + when \ q\ < l, except when q = 0, in 
which case the Theta-functions degenerate; the substitutions are therefore legitimate. 



21-61, 21*62] THE THETA FUNCTIONS 479 

It is customary to regard y as depending on k rather than on q ; and to 
exhibit y as a function of u and k, we write 

2/ = sn  u, k), 
or simply y = sn u. 

It is now evident that sn (u, k) is an elliptic function of the second 
of the types described in § 20'13 ; when g— >0 (so that  '— >0), it is easy to see 
that sn(w, A")— >sin v. 

The constant k is called the modulus; if  ''  =  4/ 3, so that k  + k'' =l, 
k' is called the complementary modulus. The quasi-periods ir , ttt .  are 
usually written 2K, 2iK', so that sn (u, k) has periods 4iK, 2 K'. 

From § 21-51, we see that 2K' = 7r' . (0 \ r'), so that K' is the same 
function of t as K is of t, when tt' = — 1. 

Example 1. Shew that 

dzS, z) -"  S,(z) S, z)' 
and deduce that, if ?y = -   " ~. , and u = z\$ -, theu 

• •J - 4 ( ) 

Example 2. Shew that 

rf2 3. 



4(2) ''•' S, z)3, z)' 



d 3 ' (z) 
and deduce that, if j/ = -r  6"7 \ >  '  u zS , then 

 3  4 [z) 

Example 3. Obtain the following results : 

[These results are convenient for calculating /•, k\ A", A'' when q is given.] 

21'62. Jacohi's earlier notation*. The Theta-f unction ©(?<) and the 
Eta-fanction H ( ). 

The presence of the factors  3"- in the expression for sn  u, k) renders it 
sometimes desirable to use the notation which Jacobi employed in the 
Fvndamenta Nova, and subsequently discarded. The function which is of 
primary importance with this notation is © (u), defined by the equation 

Ch) (u) =  4  u%-' i t), 
so that the periods associated with © (u) are 2K and 2iK'. 

* This is the notation employed throughout the Fundamenta Nova. 



480 THE TRANSCENDENTAL FUNCTIONS [CHAP. XXI 

The function \& u + K) then replaces  3 (' ) ; and in place of  i(-?) we 
have the function H (u) defined by the equation 

H ( ) = - iq - ie'"" " <-' '  e (u + i7v") =    u%-'- , t), 
and  .  z) is replaced by H (?/ + A"). 

The reader will have no difficulty in translating the analysis of this 
chapter into Jacobi's earlier notation. 

Example 1. If e'  u)=- , , shew that the singularities of -t-t-t are simple poles 

at the points congruent to I'K' (mod 2A', -liK') ; and the residue at each singularity is 1. 

Example 2. Shew that 

H' (0) = W A'-i H ( A') e (0) e (A'). 

21"7. The problem of Inversion. 

Up to the present, the Jacobian elliptic function sn (i  k) has been 
implicitly regarded as depending on the parameter q rather than on the 
modulus k ; and it has been shewn that it satisfies the differential equation 

I — -  — I = (1 - sn- u) (1 — k- sn- u), 

where A:  = V (0, ?)/ V (0, 5). 

But, in those problems of Applied Mathematics in which elliptic functions 
occur, we have to deal with the solution of the differential equation 



 I)-('- '> '-'y-> 



in which the modulus k is given, and we have no a priori knowledge of the 
value of q\ and, to prove the existence of an analytic function sn(M, k) 
which satisfies this equation, we have to shew that a number t exists* such 
that 

When this number t has been shewn to exist, the function sn(w, k) can 
be constructed as a quotient of Theta-functions, satisfying the differential 
equation and possessing the properties of being doubly-periodic and analytic 
except at simple poles ; and also 

lim sn u, k)/i( = 1. 

That is to say, we can invert the integral 

\  [y dt 

 ~Jo  l-(')  l-k'i ) ' 
so as to obtain the equation y= sn (u, k). 

* The existence of a number r, for which / (t) > 0, involves t)ie existence of a number q such 
that I g I < 1. An alternative procedure would be to discuss tlie differential equation directly, 
after the manner of Chapter x. 



217, 2r7l] THE THETA FUNCTIONS 481 

The difficulty, of course, arises in shewing that the equation 

c=V(0|tW(0|t), 

(where c has been written for L-). has a solution. 

When* < c < 1, it is easy to shew that a solution exists. From the 
identity given in §21*2 corollary, it is evident that it is sufficient to prove 
the existence of a solution of the equation 

1-c = V(0|t)/V(0!t), 

CO /  \  Q2n-1\ 8 

which may be written 1 - c = 11  - ) . 

Now, as q increases from to 1, the product on the right is continuous 
and steadily decreases from 1 to ; and so (§ 3'63) it passes through the 
value 1 — c once and only once. Consequently a solution of the equation 
in T exists and the problem of inversion may be regarded as solved. 

21 "71. The problem of inversion for complex values of c. The modular functions 

f r),g T),h T). 

The i roblem of inversion may be regarded as a problem of Integral Calcuhis, and it 
may be j roved, by somewhat lengthy algebraical investigations involving a discussion of 

the behaviour of I (I - ;!-) ~ 2 (1 — k fi) ~ 2 dt, when y lies on a 'Eiemann surface,' that the 

J 
problem of inversion possesses a solution. For an exhaustive discussion of this aspect of 
the problem, the reader is referred to Hancock, Elliptic Functions, i. (New York, 1910). 

It is, however, more in accordance with the sjjirit of this work to prove by Cauchy's 
method (§ 6-.31) that the equation =  2* (  I ' V- s* (  1 '')    one root lying in a certain 
domain of the T-j)lane and that (subject to certain limitations) this root is an analytic 
function of c, when c is regarded as variable. It has been seen that the existence of this 
root yields the solution of the inversion problem, so that the existence of the Jacobian 
elliptic function with given modulus k will have been demonstrated. 

The method just indicated has the advantage of exhibiting the potentialities of what 
are known as modular /mictions. The general theory of these functions (which are of 
great importance in connexion with the Theories of Transformation of Elliptic Functions) 
has been considered in a treatise by Klein and Fricket. 

:Mnir ,s S./ 0\ t) 



Let   / (r) = We ir n \ —  , — r  \ 



IgV""   '" ' J  3 (0 

" =53-'(0 



..  \ g(2n-l) T 8 54*(0ir) 



h r)=-f r)lg r). 
Then, if tt = — 1, the functions just introduced possess the following properties : 
/(r + 2)=/(r), 5r(r + 2)=5r(r), f r)+g r) = \, 

f r +  )  h (r), / (r') =g (r), g (r') =/(r), 

by §§ 21 '2 corollary, 2r51 example 1. 

* This is the case which is of practical importance. 

t F. Klein, Vorlesungen uber die Theorie der clliptischen Modnlfunktionen (ausgearbeitet und 
vervoUstandigt von E. Fricke). (Leipzig, 1890.) 

W. M. A. 31 



482 



THE TRANSCENDENTAL FUNCTIONS [CHAP. XXI 



It is easy|\ to see that as /(r)- - + QO , the functions iV<*~""/('") = /i W and g (t) tend to 
unity, uniformly with resi)ect to R (r), when - 1   (r)   1 ; and the derivates of these two 
functions (with regard to r) tend uniformly to zero* in the same circumstances. 

21 'Til. The principal solution off (t) — r = 0. 

It has been seen in § 6*31 that, if /(t) is analytic inside and on any contour, iiri times 
the number of roots of the equation /(t) — c = inside the contour is equal to 

/• 1 df r) 
Ifir c ir ' 

taken ]round the contour in question. 

Take the contour ABCDEFE' D'C B' A shewn in the figure, it being supposed 
temporarily! that /(t) — c has no zero actually on the contour. 

E' . F E 




-1 1 

The contour is constructed in the following manner : 

FE is drawn parallel to the real axis, at a large distance from it. 

AB la the inverse of FE with respect to the circle | t | = 1. 

BC is the inverse of ED with respect to [ r | = 1, Z) being chosen so that D\=AO. 

By elementary geometry, it follows that, since C and D are inverse points and 1 is its 
own inverse, the circle on D\ as diameter passes through C ; and so the arc CD of this 
circle is the reflexion of the arc AB in the line R (r) =  . 

The left-hand half of the figure is the reflexion of the right-hand half in the line 
R t) = 0. 

* This follows from the expressions for the Tlieta-functions as power series in q, it being 
observed that [ 9 [ -  as I (t)  - -|- oo . 

t The values of/ T) at points on the contour are discussed in § 21'712. 



2r71l] THE THETA FUNCTIONS * 483 

It will now be shewn that, unless* c  1 or c O, the equation /(r) — c=0 has one, and 
only one, root inside the contour, provided that FE is sufficiently distant from the real 
axis. This root will be called the principal root of the equation. 

To establish the existence of this root, consider / -rr\ — --y  dr taken along; the 
various portions of the contour. 

Since/(r + 2)=/(r), we have , 

I j BE J ED- ) f (t) -C dr 

Also, as T describes BC and B'C", r'(= — l/r) describes E'D' and ED respectively; 
and so 

   BC j C-B'] f T)-C dr \ j BC Jc-B']g T)-C dr 

 ] ED' J DE) g r)-C dr 
= 0, 
because g (•r' + 2)= (r'), and consequently corresponding elements of the integrals cancel. 
Since / (r ± 1 ) = A (r), we have   

[j D'C jCD]f r)-C dr jB:ABh T)-c dr 

but, as T describes B'AB, r describes EE\ and so the integral round the complete contoui* 
reduces to 

/" f\ ] df r)   1 dh r')   1 dfiyiXdr 

jEE'\ f -r)-c dr h(T') — c dr f 'r') — c dr ] 

 i    dfjr) 1 dh r) 1 dlMidr 

]EE'\ f T)-C dr h r)\ \ -c.h r)] dr '  g  r) - C dr j ' 

Now as EE' moves off to infinityt, /(t) — c-*- -c=t=0,  (t)-c- -1 — c4=0, and so the 
limit of the integral is 

- lim f  —   -   log h (r)  dr 

J EE' l-C.A(r) dr  °   '  

 X m \ 1 r . logACr) \  rflog (r) ]   

.' EE  C.h r)\ dr dr j ' 

But 1- c.A(t) 1, fi(r)~ \,g, r)- l,   - 0,  - 0, and so the limit of the 

dr clr 

integral is 



I nidr = 2-i 

J E'E 



Now, if we choose EE' to be initially so far from the real axis that / (r) — c,  - c.h (r), 
g  r) — e have no zeros when r is above EE', then the contour will pass over no zeros 
of /(r) — c as EE' moves off to infinity and the radii of the arcs CD, D'C, B'AB diminish 
to zero ; and then the integral will not change as the contour is modified, and so the 
original contour integral will be Sttz, and the number of zeros oif r) — c inside the original 
contour will be precisely one. 

* It is shewn in § 21'71'2 that, if c l or c O, then/(T) -c has a zero on the contour. 
t It has been supposed temporarily that c=|=0 and 4=1. 

31—2 



484 THE TRANSCENDENTAL FUNCTIONS [CHAP. XXI 

21 '712. The values of the modular function f  t) on the contoitr considered. 

We now have to discuss the point mentioned at the beginning of  5 21-711, concerning 
the zeros of f r) — c on the lines* joining +1 to ±l + x t and on the semicircles of 
05C1,  - ) C'B'0. 

As T goes from 1 to 1 + x t or from — 1 to — 1 + x i, /(t) goes from - x to through 
real negative values. So, if c is negative, we make an indentation in DE and a corre- 
sponding indentation in D' E' ; and the integrals along the indentations cancel in virtue of 
the relation /(t + 2) +/(r). 

As r describes the semicircle 0 C1,t' goes from - 1 + x I'to — 1, and/(r) = 5r(T') = l — /(r), 
and goes from 1 to +x through real values ; it would be possible to make indentations in 
BC and B'C to avoid this difficulty, but we do not do so for the following reason : the 
ettect of changing the sign of the imaginary part of the number '• is to change the sign of the 
real part of r. Now, if < 7  (c) < 1 and I (c) be small, this merely make-s t cross OF by a 
short i>ath ; if R (c) < 0, t goes from DE to D' E' (or vice versa) and the value of q alters 
only slightly ; but if R c) > 1, r goes from BC to B'C, and so q is not a one-valued function 
of c so far as circuits round c = +1 are concerned ; to make q a one-valued function of c, 
we cut the c-plane from -l-l to -fx ; and then for values of c in the cut plane, q is 
determined as a one-valued analytic function of c, say q (c), by the formula q (c) = e' '"' ''' 

where 

. , 1 /• r df r), 
2771 j/(r)-C dr 

as may be seen from § 6'3, by using the method of § 5-22. 

If c describes a circuit not siu-rounding the point c=l, q c) is one-valued, but t c) is 
one-valued only if, in addition, the circuit does not surround the point c = 0. 

21 '72. The periods, regarded as functions of the modidus. 

Since K=\ 7rB:  (0, q) we see from 5  21-712 that K is a one-valued analytic function of 
c =k ) when a cut from 1 to -l-x is made in the c-plane; but since K'= -irK, we see 
that K' is not a one-valued function of c unless an additional cut is made from to — x ; 
it will appear later (§ 22-32) that the cut fi'om 1 to -fx which was necessary so far as 
K is concerned is not necessary c\ s regards K'. 

2173. The inversion-problem associated iv it h Weierstrassiari elliptic functions. 

It will now be shewn that, when invariants g.  and g  are given, such that g %lg , it 
is possible to construct the Weierstrassian elliptic function with these invariants ; that is 
to say, we shall shew that it is possible to construct \ periods 2(i)i, 2u>.2 such that the function 
fp  z ( oi, do) has invariants g  and g . 

The problem is solved if we can obtain a solution of the differential equation 



m 



   if -9- -93 

of the form   =   (2 i wi, <t>2)- 

"We proceed to effect the solution of the equation with the aid of Theta-functions. 
Let v = Az, where .4 is a constant to be determined presently. 

* We have seen that EE ' can be so chosen that f (t) -c lias uo zeros either on EE ' or on 
the small circular arcs. 

t On the actual calculation of the periods, see E. T. A. Innes, Proc. Edinburgh Royal Sue. 
XXVII. (1907), pp. 357-368. 



21*7 12-21-8] THE THETA FUNCTIONS 485 

By the methods of § 21 -e, it is easily seeu that 

and hence, using the results of § 21 "2, we have 

Now let ei, e. , e  be the roots of the equation ' y' —g-iy—gz =  j chosen in such an order 
that ( 1 — e'2)l ei — 63) is not* a real number greater than unity or negative. 

In these circumstances the equation 

61-63  3'(U|r) 

possesses a solution (§ 21 "712) such that /(r)>0; this eqvxation determines the parameter 
T of the Theta-functions, which has, up till now, been at our disposal. 

Choosing t in this manner, let A be next chosen so thatt 



Then the function 
satisfies the equation 



.V =  ' |! j 53MO I r) V (0 I r) + ei 
(   ) = 4 (y - ei)  y - 62)  y - 63). 



The periods of y, qua function of z, are ttJ, tttJA ; calling these 2(bi, 2W2 we have 

/(co2/<ai)>0. 
The function   z | wi, oa.  may be constructed with these periods, and it is easily 

seen that  (3)- %\ f-f-J  ,2(0 | t) V(0 | r)-ei is an elliptic function with no pole at 
the origin | ; it is therefore a constant, C, say. 

If 6*2 , 6*3 be the invariants of   (2 | wi , a>. , we have 

Aip ( z)-G.  z)-G, = f  z) =   ip z)-C-e,]   z)-C-e. l iJ z)-C-e, , 
and so, comparing coefiicients of powers of <  (z), we have 

= 12C, G2=g-2-l2C% G3=g3-g,C+iC  
Hence (7=0, G.2=g2, Gz g ; 

and so the function   z \ co , wo) with the required invariants has been constructed. 

21'8. The numerical Computation of elliptic functions. 

The series proceeding in ascending powers of q are convenient for 
calculating Theta-functions generally, even when |   | is as large as 0"9. But 
it usually happens in practice that the modulus k is given and the calculation 

* If  \ iZ: >i thenO<   <:l; andif   <0, then 1-  >1, and 

ei-e  ei-i'j e -e  Ci-e  

 j -ej   ri\  lZ 1" <l. 

The values 0, 1, qo of (ej - <'2)/( i -  3) a-i'e excluded since (12  4= 27(/3 . 

t The sign attached to   is a matter of indifference, since we deal exclusively with even 
functions of v and -. 

t The terms in z'"  cancel, and there is no term in z~i because the function is even. 



486 THE TRANSCENDENTAL FUNCTIONS [CHAP. XXI 

of K, K' and q is necessary. It  v ll be seen later (§§ 22'801, 22'32) that 
K, K' are expressible in terms of hypergeometric functions, by the equations 

but these series converge slowly except when | k \ and j k' \ respectively are 
quite small ; so that the series are never simultaneously suitable for numerical 
calculations. 

To obtain more convenient series for numerical work, we first calculate q 
as a root of the equation k =  o" (0, q)l 'i' (0, q), and then obtain K from the 

formula K= r ir' i (0, q) and K' from the formula 

K'=7r- K\ og, l!q). 

The equation k = V (0, ?)/ V (0, q) 

is equivalent to*  Jk' =  4 (0, 5)/ 3 (0, q). 

-i 111 
Writing 2e = —j, , (so that < e < ;j when 0<k< 1), we get 

\ %i O,q)-' AO, q)  %(0,q') 
 ' % 0,q) + % 0,q) % 0,q )- 

We have seen (§§ 21*71-21-712) that this equation in g  possesses a 

solution which is an analytic function of e* when I e < - ; and so q will be 

expansible in a Maclaurin series in powers of e in this domainf. 

It remains to determine the coefficients in this expansion from the 
equation 

g + (/ + r + ... 
  ~ 1 + 2 * + 25" + . . . ' 
which may be written 

q=6 + 2qU-q''+2q' €-q-'+ ...; 

the reader will easily verify by continually substituting e + 2q*e — q  + ... 
for q w herever q occurs on the right that the first two termsj are given by 

q = e + 2e' + 15e  + loOe'  + U''). 

It has just been seen that this series converges when j e , <   . 

[Note. The first two terms of this expansion usually suffice ; thus, even if k be as 
large as  (0-8704) = 0-933..., e = |, 2€-  = 0-0000609, 15f  = 0-0000002.] 

Example. Given k = k' = \ IJ2, calculate q, K, A" by means of the expansion just 
obtained, and also by observing that t=i, so that q = e~  . 

[y = 0-04.32 139,  = A" = 1-854075.] 

* In numerical work < A; < 1, and so q is positive and <  k' < 1. 

t The Theta-functions do not vanish when |5|<1 except at '  = 0, so this gives the only 
possible branch point. 

X This expansion was given by Weierstrass, Werke, ii. (1895), p. '276. 



21-9] 



THE THETA FUNCTIONS 



487 



21 'Q. The notations employed for the Theta-f unctions. 

The following scheme indicates the principal systems of notation which have been 
employed by various writers ; the symbols in any one column all denote the same 
function. 



 i(tj) 


9,(nz) 


9s (nz) 


S (nz) 


Jacobi 


 l( ) 


 2(2) 


h  ) 


Si z) 


Tannery and Molk 


(9i (0)2) 


62 (os) 


03  (CZ) 


6 m) 


Briot and Bouquet 


e, z) 


02   ) 


e iz) 


 0(2) 


Weierstrass, Halphen, Hancock 


6 z) 


6i z) 


Osiz) 


BoM 


Jordan, Harkness and Morley 



The notation employed by Hermite, H. J. S. Smith and some other mathematicians is 
expressed by the equation 



v=0, 1) 



6 , , x)= 2 ( - T" q  '+i )- e'""  " >  '  ; (/x = 0, 1 
tt = -   

with this notation the results of § 21*11 example 3 take the very concise form 
e , ,(a;+a) =  -y 6 , , (.r),  , , (.   + ar) = (-)"?-   e" ' *"* '  6 , , (.r). 

Cayley employs Jacobi's earlier notation (§ 21 •62). The advantage of the Weierstrassian 
notation is that unity (instead of tt) is the real period of 63  z) and  0 (2). 

Jordan's notation exhibits the analogy between the Theta-functions and the three 
Sigraa-functions defined in § 20'421. The reader will easily obtain relations, similar 
to that of § 21 •43, connecting 6  ( ) with 0-,.  iwxz) when r= 1, 2, 3. 

REFERENCES. 
L. EuLER, Opera Omnia, (1), xx. (Leipzig, 1912). 
C G. J. Jacobi, Fundamenta Nova* (Konigsberg, 1829) ; Ges. Math. Werke, i. 

pp. 497-538. 
C. Hermite, Oeuvres Mathematiqiies. (Paris, 1905-1917.) 
F. Klein, Vorlesungen iiher die Theorie der elliptischen Modulfunktionen (Ausgear- 

beitet und vervoUstandigt von R. Fricke). (Leipzig, 1890.) 
H. Weber, EUiptische Funktionen und algebraische Zahlen. (Brunswick, 1891.) 
J. Tannery et J. Molk, Fonctions Elliptiques. (Paris, 1893-1902.) 

Miscellaneous Examples. 

1. Obtain the addition-formulae 

9, y+z)9, y-z)9, =hHy)9o  z)-9.  2/)h  z) = S, y)9  z)-9  y)9,  z\ 

 2 (y + 2)   2  y -  ) V = 9  iy) s,' (z) - s,-  (y) V (2) = S-2' (.y)  4  (2) -  3  (y)  1' (2), 

 3 (y +  )  3 (y - 2) V =  4  (y )  3  (2) - 5 2 (y) V (,)   532 (3/) 5 2 (,) \   2 (3,)  2 (,)  

 4 (y + -')  4 iy - z) 9C  =  3  (i/)  3   z) - 5/ (y )  2   z) = V [y) 9i' (z) - 9,  (y) 5i2 (3). 

(Jacobi.) 
* Reprinted in his Ges. Math. Werke, i. (1881), pp. 49-239. 



488 



THE TRANSCENDENTAL FUNCTIONS 



[chap. XXI 



2. Obtain the addition-formulae 

•94(y+2)54Cy- )V=V(y) 2M2)+V(y)V(2)=V(3/)V(2)+V(y) 3'( ). 

and, by increasing  / by half periods, obtain the corresponding formulae for 

5r (i/ + z) Sr Ly - --)  2  and Br y- -z)Br 2f-z) V. 
where r=l, 2, 3. (Jacobi.) 

3. Obtain the formulae 

5l 0/ ± --)  2 (y +  ) h -94 = -9, Q/) \$2 (i/)  3 (2)  4 (2) ±  3 (y)  4 C )  1 C )  2  z), 
Sl(]/±z)Ss(l/ + z)  2 4 = • ! Cy)  3 (y)  2 (2)  4 C ) ±  2 i )  4 (y)  1 (z)  3 i l 
h l/±z) -34 (y + ') 5.2-93 = 5, Cy)  4 (y)  2 (2)  3 C ) ±  o (y)  3 (y) Si  z) Si (z), 

So i/±z)SsQ/ + z) 3. S  = So (y) S3 (v/) S., (2)  3 (z) + S   y) S   1/) S  (z) 3  (z), 
S, y±z) Si i/ + z) S.,Si=S, (j/) Si (y)  2  z) Si  z) + Si (j/)  3 (y)  1 (2)  3 (2), 
S3 i/±z)S, y + z)SsSi = S, y)Si 9/)S3 z)Siiz) + S,iy)S. i/)S, z)S2 z). 

(Jacobi.) 

4. Obtain the duplication-fomiulae 

5,  2y) S,Si  = 5,2 (y) V (3/) - 3i2 (3,) S 2 (\ y)  

53 (2y) 53 V =  3  (y) 54  0/) - - i' (y) -92'  (. ), 
Si  2y) Si' = S3* (1/) - S,* (y ) = Si* y)- S,* (y). 



Obtain the dupHcation-formula 

S, (2y) 52 3 4 = 2 1 (y) -9, (y)  3 (y)  4 (y). 

Obtain dupUcation-fomuilae from the results indicated in example 2. 

Shew that, with the notation of § 21 '22, 

[l]-[2] = [4]'-[3]', [l]-[3] = [l]'-[3]', [l]-[4] = [2]'-[3]', 
[2]-[3] = [l]'-[4]', [2]-[4] = [2]-[4]', [3]-[4] = [2]'-[l]'. 

Shew that 

2 [11 22] = [11 22]' + [2211]' -[4433]' + [3344]', 

2[1133] = [1133]' + [3311]'-[4422]' + [2244]', 
2[1144] = [1144]'+[4411]'-[3322]' + [2233]', 
2 [2233] = [2233]' + [3322]' - [441 1]' + [1 1 44]', 
2[2244] = [2244]' + [4422]'-[3311]' + [1133]', 
2 [3344] = [3344]' + [4433]' - [221 1 ]' + [1 1 22]'. 

Obtain the formulae 

2n- fa =-2qi U  (1 -?2")2 (1 -y2n-l)-2| . 



(Jacobi.) 
(Jacobi.) 



(Jacobi.) 



k k' 



i = 2fyi n  (l+?2H)2(l-f/ -i)- 



10. Deduce the following i-esults from example 9 : 
n (1 - j2n- 1)6 = 2 i > •'/(•->, 



n  l-f' f =27v-'q- M'K\ 



n (l+?2 -i) =22 (M')-i, 
1=1 

n (i+92 )  =lq -kk'- , 



n (i-j )  

n=l 



= An- q- k k'-'K\ 



n (1+9" 

n = l 



Iq- k k'-K 



(Jacobi.) 



THE THETA FUNCTIONS 489 

11. By considering I .* /* e "" dz taken along the contour formed by the parallelogram 

J °i (2) 

whose corners are —  tt, Itt,  Tr + Trr, —  tt + ttt, shew that 
and deduce that, when | I z) \ < I ttt), 

12. Obtain the following expansions : 

S3' z)\  * ( - )'   sin 2 3 

each expansion being valid in the strip of the s-plane in which the series involved is 

absolutely convergent. 

(Jacobi.) 

13. Shew that, if \ I y)\ < l (ttt-) and 1 / (2) | < 7 (ttt), then 

  ± = cot y + cot 2 + 4 i 2 q'"'  sin  2my + 2nz). 
 1 (y)  1 (2) m=l  =i 

(Math. Trip. 1908.) 

14. Shew that, if | /(s) j < \ I (ttt), then 

TT  ,(S)"2' 

where a  = 2 2 j('  +  )(2' +' + ). 

(Math. Trip. 1903.) 

[Obtain a reduction formula for    by considering    Si(z) -U-'' 'dz taken round the 
contour of example 11.] 

15. Shew that 

cos 22 + g*" J 



 --rT=i o+ 2 a cos2?i2, 



 1(2) L  =il-2j2ncos: 

is a doublv-periodic function of z with no singularities, and deduce that it is zero. 

Prove similarly that 

 2( )\  . A% g-" sin 22 

.92 (z)  !=i 1 + 2? " cos 22 + 2 " 

 3 (2) " ~  lil + 2g''' -icos22 + j4 -2' 

 4'.(g) . I g2n-l sii, 2g 

  4(2)  =i l-2g ''-icos22 + \$ "-2' 
16. Obtain the values of k, k', K, K' correct to six places of decimals when q= . 
[/ .•= 0-895769, F = 0-444518, 
 =2-262700, Zr' = 1-658414.] 



490 THE TRANSCENDENTAL FUNCTIONS [CHAP. XXI 

17. Shew that, if w+x + i/ + 2=0, then, with the notation of § 21-22, 

[3] + [l] = [2] + [4], 
[1234] + [3412] + [2143] + [4321] =0. 

18. Shew that 

 4(y)  4(2) hilZ + z) ' ' i(y)Si  )h(2/ +  y 

19. By putting x=y = z, w=3a; in Jacobi's fundamental formulae, obtain the following 

results : 

Si3 ( )  j (3 .)  .  3 ( .)   (ar) =  3 (2.1,.) 5  , 

 33  x) S3 (ar) - 3i  (x) \$i (ar) = B   2x) 5-2, 
So3 (.f) 3. (Sx) + 3 3 ( ) 5  (3.1;) =  33 (2 )  3 . 

20. Deduce from example 19 that 

 Si  (x) Si  3x) Si  + Si   x) Si  Sx) Si     +  Ss  (x) S3  3x) S  - V ( )  4 (3 )  2    

=  52=* ('*-•)  2 (3.f) S3- +  4  (x) Si  3x) .932   • 
(Trinity, 1882.) 



%
% 491
%
\chapter{The Jacobian Elliptic Functions}

\Section{22}{1}{Elliptic functions with two simple poles.}

In the course of pr6ving general theorems concerning elliptic
functions at the beginning of Chapter xx, it was shewn that two
classes of elliptic functions were simpler than any others so far as
their singularities were concerned, namely the elliptic functions of
order 2. The first class consists of those with a single double pole
(with zero residue) in each cell, the second consists of those with
two simple poles in each cell, the sum of the residues at these poles
being zero.

An example of the first class, namely z), was discussed at length in
Chapter xx; in the present chapter we shall discuss various examples
of the second class, known as Jacobian elliptic functions*.

It will be seen \hardsubsubsectionref{22}{1}{2}{2}, note) that, in certain circumstances, the
Jacobian functions degenerate into the ordinary circular functions;
accordingly, a notation (invented by Jacobi and modified by Gudermann
and Glaisher) will be employed which emphasizes an analogy between the
Jacobian functions and the circular functions.

From the theoretical aspect, it is most simple to regard the Jacobian
functions as quotients of Theta-functions \hardsubsectionref{21}{6}{1}). But as many of
their fundamental properties can be obtained by quite elementary
methods, without appealing to the theory of Theta-functions, we shall
discuss the functions without making use of Chapter xxi except when it
is desirable to do so for the sake of brevity or simplicity.

\Subsection{22}{1}{1}{The Jacobian elliptic functions, TODO, TODO, TODO.}
It was shewn in
§ 21 "61 that if

y '\$,,% ul%'y the Theta-functions being formed with parameter t, then

where k = o (0 ! t)/ 3 (0 | r). Conversely, if the constant h (called
the modulus-f) be given, then, unless k- 1 or /c- O, a value of r can
be found

* These functions were introduced by Jacobi, but many of their
properties were obtained independently by Abel, who used a different
notation. See the note on p. 512.

t If 0</c<l, and is the acute angle such that sin d = k, 6 is called
the modular angle.

%
% 492
%

(§§ 21-7-21-712) for which V (0 t) V (0 t) = A--, so that the sokition
of the differential equation

;|y=(i-yn(i-%')

subject to the condition ( -y ) = 1 is

the Theta-functions being formed with the parameter t which has been
determined.

The differential equation may be written

to=i\ l-t-)- - l-k-H')- -dt, Jo

and, by the methods of\hardsubsectionref{21}{7}{3}, it may be shewn that, if y and a are
con- nected by this integral formula, y may be expressed in terms of u
as the quotient of two Theta-functions, in the form already given.

Thus, if

Jo y may be regarded as the function of u defined by the quotient of
the Theta- functions, so that y is an analytic function of u except at
its singularities, which are all simple poles; to denote this
functional dependence, we write

y = sn(? k), or simply y= sn u, when it is unnecessary to emphasize
the modulus*. The function sn u is known as a Jacobian elliptic
function of ii, and

 °"=a;§:w5?) -

[Unless the theory of the Theta-functions is assumed, it is
exceedingly difficult to shew that the integral formula defines y as a
function of u which is analytic except at simple poles. Cf. Hancock,
Elliptic Functions, i. (New York, 1910).]

 ow write cnOa-) =;(;y (B),

''"(" > = a, aTWV) * >-

Then, from the relation of\hardsectionref{21}{6}, we have

 - sn i( = en M dn (/ (I),

du

* The modulus will alwaj's be iuserted when it is not k.

%
% 493
%

and from the relations of § 21 '2, we have

sn- u + cn u = 1 (II),

k' sn- u + dn- u = l (Ill),

and, obviously, en -= dn = 1 (I )-

"We shall now discuss the properties of the functions sn u, en u, dn u
as defined bj' the equations (A), (B), (C) by using the four relations
(I), (II), (III), (IV); these four relations are sufficient to make
sn u, on u, dn u determinate functions of u. It will be assumed, when
necessary, that sn u, en m, dn u are one-valued functions of ti,
analytic except at their poles; it will also be assumed that they are
one-valued analytic functions of k' when cuts are made in the plane of
the complex variable k from 1 to -f oo and from to - xi .

\Subsection{22}{1}{2}{Simple properties of TODO, TODO, TODO.}

From the integral u= j (1 - t-) (1 - k'-t-) 2 df it is evident, on
writing

Jo

- t for t, that, if the sign of y be changed, the sign of u is also
changed.

Hence sn u is an odd function of u.

Since sn (- ii) = - sn u, it follows from (II) that en (- u) = + en u;
on account of the one-valuedness of en u, by the theory of analytic
continuation it follows that either the upper sign, or else the lower
sign, must always be taken. In the special case u = 0, the upper sign
has to be taken, and so it has to be taken always; hence cn -u) = cnu,
and cnu is cm even function of u. In like manner, dn u is an even
function of u.

These results are also obvious from the definitions (A), (B) and (C)
of §2211.

Next, let us differentiate the equation sn- u + en- u = l; on using
equation (I), we get

d en a T

- z - = - sn u dn u; du

in like manner, from equations (III) and (I) we have

d dn u, - -, - = - A;- sn u en u. du

\Subsubsection{22}{1}{2}{1}{Tlie complementary TODO modulus.}

If k- + k'- = 1 and ' -f 1 as k 0, k' is known as the complementary
modulus. On account of the cut in the --plane from 1 to -f x, k' is a
one- valued function of k.

[With the aid of the Theta-functions, we can make k' one-valued, by
defining it to be

Example. Shew that, if

J V then ?/ = en u, k).

%
% 494
%

Also, shew that, if = [ \ l - -') - J ( 2 \ /.'2) - I dt

then 3/ = dn (, /().

[These results are sometimes written in the form

J en ti J An It

22122. Glaiskers notation* for quotients.

A short and convenient notation has been invented by Glaisher to
express

reciprocals and quotients of the Jacobian elliptic functions; the
reciprocals

are denoted by reversing the order of the letters which express the
function,

thus

ns u = 1/sn u, nc u = l /en u, nd = 1 /dn u;

while quotients are denoted by writing in order the first letters of
the

numerator and denominator functions, thus

sc u = sn u/cn u, sd u - sn ? /dn u, cd u = en w/dn w, cs w = en w/sn
w, ds u = dn ujsn u, dc u = dn u/cn u.

[Note. Jacobi's notation for the functions sn rt, en u, dn u was sinam
u, cosam m, Aamw, the abbreviations now in use being due to
Gudermannt, who also wrote tuu, as an abbreviation for tanam u, in
place of what is now written sc u.

The reason for Jacobi's notation was that he regarded the inverse of
the integral

u= |*(l-Fsin2 )- rf

as fundamental, and wrote J ( = am w; he also wrote A = (l-/?-- sin c
)- for -j- .] Example. Obtain the following results :

J ./ cs u

= r' '\ i-k'-'fi)- ii+kH )-- dt= r f'-f )- t''+k r dt

Jo 7 ds H

= /' t )- l-kH )- dt =[' t''-l)- t -k )- dt

J cd i( J dc u

= / t- -l)- t -k )~ dt = ["'" t''- ) - - FH'-+k- )~ -dt

r Ud M 1 1

= 1 fi-i)-i i-r-f )- dt.

\Section{22}{2}{The addition-theorem for the function TODO.}

We shall now shew how to express sn (u + v) in terms of the Jacobian
elliptic functions of u and v; the result will be the
addition-theorem for the function sn u; it will be an
addition-theorem in the strict sense, as it can be written in the form
of an algebraic relation connecting sn u, sn v, sn u + v).

* Messenger of Mathematics, xi. (1882), p. 86. t Journal fiir Math,
xviii. (1838), pp. 12, 20. J Fundamenta Nova, p. 30. As k-*-Q, am )(-
-(/.

%
% 495
%

[There are numerous methods of establishing the result; the one given
is essentially due to Euler*, who was the first to obtain (in 1756,
1757) the integral of

in the form of an algebraic relation between x and, when X denotes a
quartic function of X and Y is the same quartic function of y.

Three t other methods are given as examples, at the end of this
section.]

Suppose that xi and v vary while u- v remains constant and equal to a,
say, so that

dv \ die

Now introduce, as new variables, Sj and Sg defined by the equations

Si - sn %i, Sn = sn v,

so thatj .V = (1 - s ) (1 -]es \

and 4- = (1 - s.f) (1 - I:? si), since xr = .

Differentiating with regard to u and dividing by 2si and 24
respectively, we find that, for general values § of u and v,

Si = - ( 1 + A;-) Si + IhH, if. = - (1 + A; 52 + Ih s-f.

Hence, by some easy algebra,

S\ Sn - S Sx Zk S1S2 \ Si - 2 )

Si's.:r-S./S,' ~ si-Sx ) l-k'Sx si) '

and so

(s,s, - s,s,)-' ~ (s,s, - s,s;) = (1 - k's.W)-' (1 - k's,%');

on integrating this equation we have

S1S2 s Si

1 - k 81 82'

where C is the constant of integration.

Replacing the expressions on the left by their values in terms of u
and v we get \

en ?i dn M sn y + en w dn w sn u 1 -k sn u sn v

* Acta Petropolitana, vi. (1761), pp. 35-57. Euler had obtaiued some
special cases of this result a few years earlier.

t Another method is given by Legendre, Fonctions Elliptiques, i.
(Paris, 1825), p. 20, and an interesting geometrical proof was given
bj- Jacobi, Journal fur Math. iii. (1828), p. 376. X For brevity, we
shall denote differential coefficients with regard to u by dots, thus

dv .. d v du dii

8 I.e. those values for which en u dn u and cnv nv do not vanish.

%
% 496
%

That is to say, we have two integrals of the equation du + dv = 0,
namely ) u + v= a and (ii)

sn u en V dn V + sn ?; en M dn u

1 - k sn" u sn v

= G,

each integral involving an arbitrary constant. By the general theory
of differential equations of the first order, these integrals cannot
be functionally independent, and so

sn u en v dn v + sn v en u dn u 1 - A;- sn ii sn"-* v

is expressible as a function oi u + v; call this function f(u + v).

On putting i; = 0, we see that /(u)=snw; and so the function / is the
sn function.

We have thus demonstrated the result that

sn tt en V dn V 4- sn v en u dn u

sn u + v) = j- -,

1 - '- sn- u sn- V

which is the addition-theorem.

Using an obvious notation*, wc may write

. s.cdo + SoCidi

Example 1. Obtain the addition-theorem for sin ic by using the results

Example 2. Prove from first principles that

3 d\ SiC2d2 + s.2Cidi\

\ cv cuj

ycv cuJ 1 - k'-s -s-r

and deduce the addition-theorem for sn u.

(Abel, Journal fUr Math. n. (1827), p. 105.) Example 3. Shew that

s -8 \ S iCid + s Cidi SidiC +SjdiCi s c d - s Cidi CiC2 + Sidj Sod2
did2+khiS2CiC2'

(Cayley, Elliptic Functions (1876), p. 63.)

Example 4. Oljtain the addition-theorem for sn u from the results

 i(y-H ) 4(y- ) 253=-9,(y)54(y) 2( ) 3( )+- 2(y) 3(3/)5i( )54(4 4 y+z)
Si(2/-2) V= V (y) i' (z) - 1- ii/) l' (z),

given in Chapter xxi, Miscellaneous Examples 1 and 3 (pp. 487, 488).
\addexamplecitation{Jacobi.}

Example 5. Assuming that the coordinates of any point on the curve

?/- = (l-;f2)(l-y5;2a;2)

can be expressed in the form (sn u, en u dn u), obtain the
addition-theorem for sn u by Abel's method \hardsubsubsectionref{20}{3}{1}{2}).

* This uotation is due to Glaisher, Messenger, x. (1881), pj). 9'2,
124.

I

%
% 497
%

[Consider the intersections of the given curve with the variable curve
y = I + mx + n.v'; one is (0, 1); let the others have parameters u,
Uz, M3, of which u . xii may be chosen arbitrarily Vjy suitable choice
of m and n. Shew that U1 + U2 + U3 is constant, by the method of §
20-312, and deduce that this constant is zero by taking

 i = 0, = -|(1+F).

Observe also that, by reason of the relations

we have

 3(1 -k' Xi x. ) = X3- +,2\ 2 ) mx x x - mxix.,- nxxX2 ( 1 + 2+ 3)

= xi+x-2- -x - nxi x-2 X3) - Xi + X2) - 2mxi Xo - nx Xo xi + 2)

\Subsection{22}{2}{1}{The addition-theorems for en u and dn u.}
We shall now establish
the results

en M en w - sn w sn V dn w dn w

en (z< + v) = = Y, - 2 2 .

- ., dn dn V - k- sn w sn v en u en v dn (u + v) = rw- - .;

the most simple method of obtaining them is from the formula for sn (u
+ v).

Using the notation introdueed at the end of\hardsectionref{22}{2}, we have (1 - k's
's.-y en- (u + v) = (l - k s s Y 1 - sn- (u + v)]

= (1 - k' S ' S y - (SjCaC o + SoCidiY

= 1 - 'Ik'-si's + k*s,*s.2' - sf (1 - si) (1 - k si)

- si (1 - si) (1 - k-si) - 2sxS.2CiC.2did2 = (1 - si) (1 - si) + sisi
(1 - k'si) (1 - k'si)

- 2siS2CiC.,dxd.2 = (ciCo - SySod d y

and so en (u + v) = + - - r ~ -

 1 - k-sisi

But both of these expressions are one- valued funetions of u, analytic
exeept at isolated poles and zeros, and it is ineonsistent with the
theory of analytic continuation that their ratio should be + 1 for
some values of u, and - 1 for other values, so the ambiguous sign is
really definite; putting li = 0, we see that the plus sign has to be
taken. The first formula is consequently proved.

The formula for dn u + v) follows in like manner from the identity (1
- k' sisiy - k (s CoJz + s.Cxdn y

= (1 - k'-si) (1 - fi'si) + k*sisi (1 - si) (1 - si) -
2k's,SoC,C2d,d2,

the proof of which is left to the reader.

w. ii. A. 32

%
% 498
%

Example 1. Shew that

du u + v) dn u-v) = -j j i '

\addexamplecitation{Jacobi.} [A set of 33 formulae of this nature connecting functions of
u + v and of - v is given in the Fundamenta Nova, pp. 32-34.]

Example 2. Shew that

8 cnu + cnv \ 9 cnw+cnv

cu sn M dn V + sn V dn ?< ov sn i dn v + sn i; dn u '

so that (en r(+cn v)/(snMdn?' + sn vdnii) is a function of i + v only;
and deduce that it is

equal to 1 +cn u + v) lsvi u + v).

Obtain a corresponding result for the function ( iC9 +
S2Ci)/(c?j+(a?2).

(Cayley, Messenger, xiv. (1885), pp. 56-61.) Example 3. Shew that

1 - Fsn2 u + v) sn2 u-v) = Fsn* u) l- F sn* v) (1 - Fsn2 ?< sn2 y) -2,
'- + >fc2cn2(?< + v) (in u~v) = k" + k-cn*u) (/ '2+Fcn' r) (1
-Fsn2%sn2w)-2.

\addexamplecitation{Jacobi and Glaisher.}

Example 4. Obtain the addition-theorems for cn(-w + r), dn(w-|-'i?) by
the method of\hardsectionref{22}{2} example 4.

Example 5. Using Glaisher's abridged notation Messenger, x. (1881), p.
105), namely

s, c, d=s\ \ u, en u, dn u, and >S', C, Z) = sn 2i, en 2i(, dn 2,
prove that

2gCt l-2g2 .2g4 1 - 2/t2s2 + y[;y

l->;254' - l\ y[,.254 ' - 1 L y[ s4- '

(i+ )4\ (i\ i

(l + -?:,S') + (l-Z-,S') ' Example 6. With the notation of example 5,
shew that

1-C D D-BC-k- D-G

   =

l+Z>~F(l + (7) k D-G) k' '+D-BC \ D + C \ D + k- C-k ' \ /;'2(i-Z)) \
/j:'2(i + (7) ~l-|-i>~ k: l + G)' ~ k- D -C)~ k"- + D- PC

k' +D + k' G \ D+C \ k' C) k' \ + D ) \ + D "l + C" D~C ~k' + D-k- G'

\addexamplecitation{Glaisher.}

\Section{22}{3}{The constant K.}
We have seen that, if

u=\ \ l- )- ~ l-kH'')- ~dt,

J then y = sn u, k).

If we take the upper limit to be unity (the path of integration being
a straight line), it is customary to denote the value of the integral
by the symbol K, so that sn K, k)= 1.

[It will be seen in\hardsubsubsectionref{22}{3}{0}{2} that this detinition of A' is equivalent
to the definition as TT a in\hardsubsectionref{21}{6}{1}.]

%
% 499
%

It is obvious that en = and dn K= ± k' : to fix the ambiguity in sign,
suppose < A' < 1, and trace the change in (1 - k'H-) as t increases
from to 1; since this expression is initially unity and as neither of
its branch points (at t = ± k~ ) is encountered, the final value of
the expression is positive, and so it is + A '; and therefore, since
dn is a continuous function of k, its value is always + k'.

The elliptic functions of K are thus given by the formulae sniT-l, cn
= 0, dn K = k'.

\Subsubsection{22}{3}{0}{1}{The expression of K in terms of k.}

In the integral defining K, write t = sin <, and we have at once

K \ (l-A-2sin-(/))- cZ(/>.

J

When \ k\ < \, the integi-and may be expanded in a series of powers of
k, the series converging uniformly with regard to (by\hardsubsectionref{3}{3}{4}, since
sin-" ( 1 ); integrating term-by-term \hardsectionref{4}{7}), we at once get

K=\ \ F \, \; 1; k =\ \ Fl\, 1;

where c = k" . By the theory of analytic continuation, this result
holds for all values of c when a cut is made from 1 to -l- oo in the
c-plane, since both the integrand and the hypergeometric function are
one-valued and analytic in the cut plane. Example. Shew that

(Legendre, Fonctions Elliptiques, i. (1825), p. 62.)

\Subsubsection{22}{3}{0}{2}{The equivalence of the definitions of K-}

Taking = 3- in\hardsubsectionref{21}{6}{1}, we see at once that \&n hTrd ) = \ and so
cn(|7r53-) = 0. Consequently, 1 - sn m has a double zero at \ 7r i .
Therefore, since the number of poles of sn w in the cell with corners
0, nS, n t- ) B, tt (r - 1) .93'- is two, it follows from\hardsubsectionref{20}{1}{3}
that the only zeros of 1 - sum are at the points = Jn-
(4?n-l-l-|-2?ir) 3, where m and n are integers. Therefore, with the
definition of\hardsectionref{22}{3},

A'=|7r(4m-l-l+2 r)532.

Now take t to be a pure imaginary, so that < /( < 1, and K is real;
and we have ?i = 0, so that

\ TT Am + l)B = j ''(l->t2sin2 0)~2(/,

where m is a positive integer or zero; it is obviously not a negative
integer.

If m is a positive integer, since / (1 - k' sin 4>)~ i * continuous
function of a and

J so passes through all values between and K as a increases from to
tt, we can find a value of a less than tt, such that

A7(4?ra -I- 1 ) = hrrS = f " ( 1 - P sin2 cj))- dc;

and so sn ( TrSs ) = sin a < 1,

which is untrue, since sn (-177 32) = 1.

32-2

%
% 500
%

Therefore m must he zero, that is to say we have

But both K and n - are analytic functions of / when tlie o-plane is
cut from 1 to + 00, and so, by the theory of analytic continuation,
this result, proved when 0< -<l, persists throughout the cut plane.

The equivalence of the definitions of K has therefore been
established. Example 1. By considering the integral

J shew that sn 2K= 0.

Example 2. Prove that

sn A'=(l +/;')" S cnhK=M \ + k')', dn|A' = /CA

[Notice that when u = \ K, cn2M = 0. The simplest way of determining
the signs to be attached to the various radicals is to make --a-O,
X'-a-l, and then sn ?<, en w, dn u degenerate into sin u, cos u, 1.]

Example 3. Prove, by means of the theory of Theta-functions, that

cs iA'=dn \ K=k' .

\Subsection{22}{3}{1}{The periodic proper ties (associated with K) of the Jacobian elliptic functions.}

The intimate connexion of K with periodic properties of the functions
snu, en 11, dnu, which may be anticipated from the periodic properties
of

Theta-functions associated with - tt, will now be demonstrated
directly from

the addition-theorem.

By\hardsectionref{22}{2}, Ave have

, snucnK dnK + iiKcn.udnu,

sn(u + K)-; - y ., -, --77 = cd u.

  \ - k' sn- a sn K

In like manner, from\hardsectionref{22}{2}],

en (it + K) = - ' sd u, dn (a + ii ) = ' nd u.

TT /, Tj x cn((t + / ) 'sdw

Hence sn ( (i 4- 2it ) =, - 7; = - .-> - ~ = - sn u,

  dn(u-l-ir) A;ndM

and, similarly, en u -f IK) = - en u, dn it + 'IK) = dn u.

Finally, sn u -h K) = - sn u -f 2K) = sn u, en (?( -I- 4 A") = en u.

Thus 4iK is a period of each of the functions sn u, en ( while dn u
has the smaller period 2K.

Example 1. Obtain the results

sn ( u + A') = cd u, en (m + K) = - k' sd u, dn ( u + K) = k' nd m,
directly from the definitions of sn ?t, en ?<, dn u as quotients of
Theta-functions. Example 2. Shew that cs u cs ( K - u) = k'.

%
% 501
%

\Subsection{22}{3}{2}{The constant K'.}
We shall denote the integi'al

Jo

by the symbol K', so that K' is the same function of k'- (= c) as K is
of k (= c); and so

K'= rrFi -, \; 1; k",

when the c'-plane is cut from 1 to + oo, i.e. when the c-plane is cut
from

to - X .

To shew that this definition of K' is equivalent to the definition of
§ 21 "61, we observe that if T7-'= - 1, is the one-valued function of
P, in the cut plane, defined by the equations

K= M - (0 1 r), F = 5, (0 ! r) 3* (0 I r),

while, with the definition of\hardsubsectionref{21}{5}{1},

 ' = 1 32(010,; :'2 = 52*(01r') 3'*(0|r'),

so that K' must be the same function of kf as K is of k"; and this is
consistent with the integral definition of K' as

Jo It will now be shewn that, if the c-plane be cut from to - 00 and
from

1 to + X, then, in the cut plane, K' may be defined by the equation

K' =r\ s'-l)-- l- k's') - ds.

J 1

First suppose that 0< '<1, so that < ' < 1, and then the integrals
concerned are real. In the integral

f\ l -t )-iO--k'H')- dt .'0

make the substitution

s = l-k'H-)-',

which gives

(s' - 1)4 = k't (1 - I -f) -i, (1 - k's'-)i = k' (1 - i?f (1 - k'H-) '
*, ds \ k'H dt Xl-k'H-'f

it being understood that the positive value of each radical is to be
taken. On substitution, we at once get the result stated, namely that

K' = I ' (6- - 1) - i (1 - k's') - i ds,

provided that < k <\; the result has next to be extended to complex
values of .

%
% 502
%

Consider T' ' l-t )~ l -kH"-) " dt.

the path of integration passing above the point 1, and not crossing
the imaginary axis*. The path may be taken to be the straight lines
joining to 1 - 8 and 1 + 8 to k~ together with a semicircle of (small)
radius S above the real axis. If (l-t-y and ( k' t-)

reduce to + 1 at = the value of the former at 1 + S is e" "" S (2 + 8)
= - i (t - Vf, where each radical is positive; while the value of the
latter at <=1 is +/' when k is real, and hence by the theory of
analytic continuation it is always +/;'.

Make 8-*-0, and the integral round the semicircle tends to zero like
8-; and so

Now f-)- - k'H y -clt=\ (F-?('-i)"2(i\,,2)-2o;?(,

which t is analytic throughout the cut plane, while K is analytic
throughout the cut plane.

Hence \ \ ' i: -\ y hH y dt

is analytic thi'oughout the cut plane, and as it is equal to the
analytic function K' when < X-'< 1, the equality persists throughout
the cut plane; that is to say

/ l/A; 1 1

when the c-plane is cut from to - qo and from 1 to + oo,

Since

K + iK'=\ l-t-)-- il-hH')- ~dt,

Jo

we have sn (K + iK') = i/k, dn (K + iK') =;

while the value of en (K + iK') is the value of (1 - P) when t has
followed the prescribed path to the point 1/A-, and so its value is
-ik'/k, not +ik'lk.

Example 1. Shew that

I f \ t l-t) l-Pt) ~- dt = l r t t-' ) kH- ) ~ dt=K,

- / -t l-t) kH) - dt = ( - 1) (1 - FOl- ~ 'dt = A".

Example 2. Shew that K' satisfies the same linear differential
equation as K \hardsubsubsectionref{22}{3}{0}{1} example).

\Subsection{22}{3}{3}{The periodic propertiesX associated with K + iK') of the Jacohian elliptic functions.}

If we make use of the three equations

sn K + iK') = k-\ en (K + iK') = - ik'jk, dn K + iK') = 0,

* II (A:) > because | arg c I < tt.

t The path of integration passes above the point u = k.

J The double periodicity of snw may be inferred from dynamical
considerations. See Whittaker, Analytical Dynamics (1917), § 44.

%
% 503
%

we get at once, from the addition-theorems for sn u, en u, dn u, the
following results :

, ., sn u en K + iK') dn (K + iK') + sn (Z H- iK') cnudnw

sn ( 4- ii + *A ) = z Tz - 2/ ir, ir'\ '

  1 - k sn u sn K +%K )

= k~ do (t, and similarly en ' u + K + iK') = - ik'k~ nc u,

dn (zt + if + iK') = ik' sc w. By repeated applications of these
formulae we have

f sn (u + 2K+ 2iK') = - sn u, ( sn (a + 4>K + UK') = sn u, cn(u+2K-
2iK') en t<, -. en (u + 4 " + 4tX') =cn m, [dn (m + 2Z + 2iK') = - dn
w, [dn u 4<K + UK') = dn u. Hence the functions sn u and dn u have
period 4iK + UK', ivhile en u has the smaller period 2K + 2iK'.

\Subsection{22}{3}{4}{The periodic proper-ties (associated with iK') of the Jacobian elliptic functions.}

By the addition-theorem we have

sn (u + iK') = Sn(u-K + K+ iK') = k-' dc (u - K) = k~ ns u. Similarly
we find the equations

en (u + iK') - - ik~ ds u, dn u + iK') = - ics u. By repeated
applications of these formulae we have

' Bn u- 2iK') = sn u, ( sn (ii + UK') = sn u, - en (u + 2iK') = - en
u, - en u + UK) = en u,,dn (u + 2iK') = - dn u, [dn u + UK') = dn u.
Hence the functions en u and dn u have period UK', luhile sn u has the
smaller period 2iK' .

Example. Obtain the formulae

sn iu + 2mK+ 2mK') = ( - )'" sn ti, en (u + 27nK+ 2niK') = ( - )™ + "
en i, dn ( + 2mK+2niK') = ( - )" dn u.

\Subsubsection{22}{3}{4}{1}{The behaviour of the Jacobian elliptic functions near the origin and near iK'.}

We have

d d

- sn u = en M dn u, -i- sn u = 4 gn u en u dn m - en m dn u (dn u + k
- cn u). du du

%
% 504
%

Hence, by Maclaurin's theorem, we have, for small values of | w|, sn u
=u-~(l + k-) u' + (u'), on using the fact that sn u is an odd
function.

In like manner

dn u = l-l khr + (u'). It follows that

sn (u + iK') = k~ ns u

ku [ b )

1 1 + '-, 3,

=,- + -7rr- u+0(u'); ku bk

- i 2k- - 1 and similarly en (u + iK') = tt; +, iu + (u'),

dn (u + iK') = - - + / ' ill + (u').

u b

It follows that at the point iK' the functions sn r, cnv, dnv have
simple poles with residues k~, - ik~, - i ixspectively.

Example. Obtain the residues of snw, cnw, dnw at iK' by the theory of
Theta- fiinctions.

2235. General description of the functiotis sn u, en u, dn u.

The foregoing investigations of the functions sn u, en u and dn u may
be summarised in the following terms :

(I) The function sn m is a doubly-periodic function of u with period:?
K, 2iK'. It is analytic except at the points congruent to iK' or to 2K
+ iK' (mod. 4jfir, 2iK'); these points are simple poles, the residues
at the first set all being k~ and the residues at the second set all
being - k~; and the function has a simple zero at all points
congruent to (mod. 2K, 2iK').

It may be observed that sn u is the only function of u satisfying this
description; for if (m) were another such function, sn m - (m) would
have no singularities and would be a doubly-periodic function; hence
\hardsubsectionref{20}{1}{2}) it would be a constant, and this constant vanishes, as may
be seen by putting u = 0; so that (tt) = sn u.

When 0< A; < 1, it is obvious that K and K' are real, and sn u is real
for real values of u and is a pure imaginary when a is a pure
imaginary.

(II) The function cm* is a doubly-periodic function of u with periods
K and 2K -+ 2iK'. It is analytic except at points congruent to iK' or
to 2ir-f- iK' (mod. 4/1", 2K - 2iK'); these points are simple poles,
the residues

%
% 505
%

at the first set being - ik~\ and the residues at the second set being
ik~; and the function has a simple zero at all points congruent to K
(mod. 2K, 2iK'). (Ill) The function dn u is a doubly-periodic function
of u with periods 2,K and 4iiK'. It is analytic except at points
congruent to iK' or to ZiK' (mod. ''2K, 4tiK'); these points are
simple poles, the residues at the first set being - i, and the
residues at the second set being i; and the function has a simple
zero at all points congruent to K + iK' (mod. 2K, 2iK').

[To see that the fvinctions have no zeros or poles other than those
just specified, recourse must be had to their definitions in terms of
Theta-functions.]

\Subsubsection{22}{3}{5}{1}{The connexion between Weierstrassian and Jacobian elliptic functions.}
If ej, 62) 3 be any three distinct numbers whose sum is
zero, and if we write

61-63

  = 3 +

sn2 (Xm, k) '

we have (;7 ) (' 'i ~" z)' " "' ** '

= 4 (ei ~e'i)-\ \ ns Xzi (ns Xw - 1) (ns Xw- ) = 4X2 (ci - 63) ~ V / -
3) ( / - ei) y - F e - 63) - 63 . Hence, if X2 = 6i - 63 and >?'2 =
(e-. - 63)/(6i - 63), then y satisfies the equation*

and so e3 + (ei-63) ns Jm (61-63)2, A/ zyr = § (' +; Qi, 9z\

where a is a constant. Making u - 0, we see that a is a period, and so

  (u; g2, 93) = 63 + ei - 63) ns2 u (ci - 63)% the Jacobian elhptic
function having its modulus given by the eqviation

1-63

61-63
\Section{22}{4}{Jacobi's imaginary transformation TODO.}

The result of | 21-51, which gave a transformation from
Theta-functions with parameter t to Theta-functions with parameter t'
= - l/r, naturally produces a transformation of Jacobian elliptic
functions; this transformation is expressed by the equations

sn (iu, k) = i sc (u, k'), en iu, k) nc (u, k'), dn in, k) = dc (a,
¥). Suppose, for simplicity, that < c < 1 and y >; let

' 'l-t')' l- kH-) ' dt = iu,

f

Jo

so that iy = sn (iu, k);

take the path of integration to be a straight line, and we have en
(iu, k) = (1 + y' ), dn iu, A;) = (1 -1- k-y-) .

* The values of 2 '' 93, usual, - J2e2 3 and eie e-s.

f Fundamenta Nova, pp. 34, 35. Abel Journal fUr Math. 11. (1827), p.
104) derives the double periodicity of elliptic functions from this
result. Cf. a letter of Jan. 12, 1828, from Jacobi to Legendre
[Jacobi, Ges. Werke, i. (1881), p. 402].

%
% 506
%

Now put V = 77/(1 - ?;-)-, where < 77 < 1, so that the range of values
of t is from to 177/(1-77-)-, and hence, if t = iUl \ - t ), the
range of values of t is from to 77.

Then dt = i t )- -idU, (1 - r-) = (1 -,-)-*,

1 - kH' = (1 - kH ) ~ - l- /r) - i

and we have in=\ (1 - fj-) " (1 - h'-ti-) ' idti,

Jo

so that 77 = sn (u, k')

and therefore tj = sc u, k').

We have thus obtained the result that sn hi, k) = i sc u, k').

Also en in, k) = (1 -f- y'-)- -- (1 - 77-) ~ = nc u, k'),

and dn iu, k) = l- ¥if) = ( 1 - k'-'n'') ( 1 - 77-) ~ * = dc a, k').

Now sn iu, k) and isc (w, k') are one-valued functions of u and A; (in
the cut c-plane) with isolated poles. Hence by the theory of analytic
continuation the results proved for real values of u and k hold for
general complex values of u and k.

\Subsection{22}{4}{1}{Proof of Jacohi's imagimiry transformation by the aid of Theta-functions.}

The results just obtained may be proved very simply by the aid of
Theta-functions. Thus, from \hardsubsectionref{21}{6}{1},

sndu M\ 3(0|T),(i iT) '' '''' -% Oit)% J t)'

where = m/ 3- (0 | t),

and so, by\hardsubsectionref{21}{5}{1}, sn tu, k) = |4 |4> . "f '"'

= - isc (v, k'),

where v = izr" (0 j t) = izr' . (- ir) 3- (0 j t) = - w,

so that, finally, sn iu, k) = i sc u, k').

Example 1. Prove that en iu, l-) = nc(u, k', dn (in, i-) = do ri, k')
by the aid of Theta- functions.

Example . Shew that

sn hiK', k) = iiic K', k') = ik~-,

en i iK', k) = l+k) k-, dn iiK', k) = l+k)K

[There is great difficulty in determining the signs of sn iK', ci\ j
iE', dnit'A'', if any method other than Jaeobi's transformation is
used.]

%
% 507
%

Example 3. Shew that

Example 4. If < Z- < 1 and if be the modular angle, shew that

sn \ K + ?:/ :') = e'' *' " *' V(cosec ), en h K + iK') = e " i' '
/(cot 6),

\addexamplecitation{Glaisher.} 22"42. Landens transformation* . We shall now obtain the
formula

f ' (1 - l- sin- d,) ~ dd, = l- k') f (1 - -' sin- 6) " *c, Jo Jo

where sin j = (1 + k') sin ( cos < (1 - kr sin- ) ~

and k, = l-k')l l + k').

This formula, of which Landen was the discoverer, may be expressed by-
means of Jacobian elliptic functions in the form

sn (1 + k') u, k \ = \ ->r k') sn u, k) cd u, k), on writing (f) = am
u, (f) - am u .

To obtain this result, we make use of the equations of | 21-52, namely
% z\ r), z\ r ) % z\ t); z\ t ) 3 (0 | t) 4 (0 I t) 4(2 12t) -
(22I2t) 4(0|2t)

Writef Ti = 2t, and let k-, A, A' be the modulus and quarter-periods
formed with parameter Tj; then the equation

% z\ r)X z\ r) \ ' 'lz\ n;) % z\ r), z\ r),(20iTO

may obviously be written

k sn 2Kz/7r, k) cd (2 A /tt, k) = k, sn (4 A tt, k,) (A).

To determine k in terms of k, put z = jTt, and we immediately get

  /(l + k') = k, which gives, on squaring, k = (1 - '')/(l + k'), as
stated above.

To determine A, divide equation (A) by z, and then make - >0; and we
get

2Kk = 4 'l* A,

so that A=~ [+k')K.

* Phil. Trans, of the Royal Sac. lxv. (1775), p. 285.

t It will be supposed that \ R(t)\ < :, to avoid difficulties of sign
which arise if R (tj) does not lie between ±1. This condition is
satisfied when 0<k<l, for r is then a pure imaginary.

%
% 508
%

Hence, writing ii in place of Kzjir, we at once get from (A) (1 + A;')
sn u, k) cd u, k) = sn (1 + k') u, ki], since 4<Az/7r = 2Au/K =(1 +
k')ii;

so that Landen's result has been completely proved.

Example 1. Shew that JA = 2K'jK, and thence that A' = (l +1-') K'.
Example 2. Shew that

en (!+/')", /i = l -(!+>(') sn2(, H') nd n, k), dn (l+ ')w, i] = W
+ (l - k') cn (u, X-) nd(M, k). Example 3. Shew that

dn u, k) = l-k')cn l+k')u, ki] + I + i-') dn I + k') u, l\ \ },

where X-=2 -ii/(l+ 'i).

\Subsubsection{22}{4}{2}{1}{Transformations of elliptic functions.}

The formula of Landen is a particular case of what is known as a
transformation of elliptic functions; a transformation consists in
the expression of elliptic functions with parameter t in terms of
those with parameter a + bT)j c + dT, where a, b, c, d are integers.
"We have had another transformation in which = - 1, 6 = 0, c = 0,
c?=l, namely Jacobi's imaginary transformation. For the general theory
of transformations, which is out- side the range of this book, the
reader is referred to Jacobi, Fundamenta JYova, to Klein, Vorlesungen
iiher die Theorie der elliptischen Modulfunktionen (edited by Fricke),
and to Cayley, Elliptic Functions (London, 1895).

Example. By considering the transformation ro = r+l, shew, by the
method of\hardsubsectionref{22}{4}{2}, that

sn k'u, k2)=k' sd u, k),

where -0= ± ik/k', and the upper or lower sign is taken according as R
t)<0 or R (r) >; and obtain formulae for en k'u, 2) and dn k' l, k' .

\Section{22}{5}{Infinite products for the Jacohian elliptic functions*.TODO}

The products for the Theta-functions, obtained in\hardsectionref{21}{3}, at once
yield products for the Jacobian elliptic functions; writing a =
KxJtt, we obviously have, from\hardsubsectionref{22}{1}{1}, formulae (A), (B) and (C),

  i, - It  ( 1 - 2m cos 2a; + o " ]

   =i 1 1 - 25-'*-! cos 1x + (7 "-- J

g i;47 -4 n 1 1 + 29 " COS 2 + " [

en II = 2q*k -k - cos x H \ ?r- r~~, t;;- i .

,j i (1 - 2 2"-> cos 2x + 5'*"-2)

- lA u (l + 2g - cos2a; + g -'

''1 [1 - 2cf' ''' COS 2x + q

From these results the products for the nine reciprocals and quotients
can

be written down.

There are twenty-four other formulae which may be obtained in the
following manner :

From the duplication-formulae \hardsubsectionref{22}{2}{1} example 5) we have

1-cntt 1,1 l-HduM,1 1 dn?i-|-cnM 1,1

= sn - udc - u, =as -u uc -,  = en ? as - u.

Huu 2 2 sn 2 2 sn m 2 2

* Fundamenta Nova, pp. 84-115.

%
% 509
%

Take the first of these, and use the products for sn u, en |i(, dn u;
we get

7+q j '

l - cnu l-cos. ' ° fl -2 ( -o)" cos j;-

snu ~ sin =i (1 + 2 (-g )"cosa;+g'2 on combining the various products.

Write u + K for u, x+h-rr for x, and ve have

dnw + snw l + sin.-?; * fl + 2 (-g)" sin.- g + g " ! cnu " cos j; =i
[1 - 2 ( - g)" sin a; + ( - j '

Writing u + iK' for in these formulae we have

. - ri + 2  ( - ) o" - sin .v - <f'' " ] k sn if + dn ?< = I n - =
rr-. - r; - -. - - t,

and the expression for cd ?i + ik' nd u is obtained by writing cos x
for sin .t? in this product.

From the identities I - cmi) (I + cmi) = an ti, (ksmi + idnu)
l-sm(-idmi) l, etc., we at once get four other formuhxe, making eight
in all; the other sixteen follow in the same way from the expressions
for ds-|Mnc M and cn tids ic. The reader may obtain these as an
example, noting specially the following :

Example 1. Shew that

  >-i ((i-ij '-'Xi+t '""')!

Example 2. Deduce fron\ example 1 and from § 22 '41 example 4, that,
if 6 be the modular angle, then

" tl+(-)" . +*/'

and thence, by taking logarithms, obtain Jacobi's result
$$
TODO
$$
' quae inter formulas elegantissimas censeri debet.' Fund. Nova, p.
108.) Example 3. By expanding each term in the equation log sn M = log
(2 J') - i log /t + log sin x + 2 log ( 1 - j " e * )

+ log (1 - j2ne-2ia;)\ log(l-j2n-l e' ) -\ og (1 - J n-l g -2ia:-)j

in powers of e ', and rearranging the resulting double series, shew
that

, K 1, 7 1  ** 2o'" cos 2mA'

logsnw=log(2g')-|log/ - + logsm.r+ 2;,,,

when l/(2)l<-|7r/(r).

Obtain similar series for log en u, log dn i.

(Jacobi, Fundamenta Nova, j). 99.)

Example 4. Deduce from example 3 that

 K

log sn udu= - irK' - \ K log k.

(Glaisher, Proc. Royal Soc. xxix.)

/:

%
% 510
%

\Section{22}{6}{Fourier series for the Jacobian elliptic functions*.TODO}

If u = Rxjir, sn u is an odd periodic function of x (with period 27r),
which obviously satisfies Dirichlet's conditions \hardsectionref{9}{2}) for real
values of x; and therefore \hardsubsectionref{9}{2}{2}) we may expand sn w as a Fourier
sine-series in sines of multiples of x, thus

sn a - hn sin nx,

n = \

the expansion being valid for all real values of x. It is easily seen
that the coefficients 6 are given by the formula

TTibn - I sn u . exp nix) dx.

J -77

To evaluate this integi'al, consider I snu. exp nix) dx taken round
the parallelogram whose corners are - tt, tt, ttt, - 27r + ttt.

Tttt

From the periodic properties of sn w and exp nix), we see that 1
cancels

r -It ' IT

1; and so, since -tt + ttt and ttt are the only poles of the
integrand

 qua function of x) inside the contour, with residuesf

- ~ ( 2 tt/K ] exp ( - niir + - niriT j

and k- Q Tr/iTJ exp Q nirir)

respectively, we have

\ \ - [ sn li . exp nix)dx = - g-" 1 - (-)" .

(J -TT J -2;7 + 7rTj -"- "

Writing a; - tt + ttt for x in the second integral, we get

[1 + (-)"?" j sn I* . exp (mic) rfa; = -| 3 " 1 - (-)~ .

Hence, when /i is even, hn =; but when n is odd Consequently

sn ti =

27r J5'- sin a; g sin Sx q sin oa; ]

when X is real; but the right-hand side of this equation is analytic
when q " exTp nix) and q exp -nix) both tend to zero as w- >x, and the
left- hand side is analytic except at the poles of sn u.

* These results are substantially due to Jacobi, Fundamenta Nova, p.
101. t The factor irlK has to be inserted because we are dealing with
sn (2KxJTr).

%
% 511
%

Hence both sides are analytic in the strip (in the plane of the
complex variable x) which is defined by the inequality [ I(oc) : < irl
r).

And so, by the theory of analj tic continuation, we have the result sn

\ 27r I g" sin(2;? + ) x (where u = KxJtt), valid throughout the strip
[ / (a;) | < - tt/ (r)

Example 1. Shew that, if ?i = 2A'.iY7r, then

CU M =

27r 5 cos(2?;.+ l) A- \ tt Stt °° §'" cos 2??.r

A n=o l + ? -i ' '''''*"2Z + T !i l+j- '

I " J . 7., 29" sin 2?i ./o,1=1 %(l+j2 )

these results being valid when | /( ) | <c\ tt I r).

Example 2. By writing x-k-\ iT for x in results ah'eady obtained, shew
that, if u= Kxl-K and \ I x)\ < \ ivI t\

then cd - 1 (-rg--+ cos(2 + l ) sd..-- 1 (- ?"+ sin (2n + l).: then
cdw-, 2 l\ j2 + i ' '" '-A'M'io T+ i '

, TT 27r "" ( - )" o" COS 2 a;

nd =-.>,, + -FT, 2 - - - 5 .

2AX-' Kk' =i l+j2n

\Subsection{22}{6}{1}{Fourier series for reciprocals of Jacobian elliptic functions.}

In the result of\hardsectionref{22}{6}, write u + iK' for u and consequentl a; + ttt
for a;;

then we see that, ifO>/(a7)> - tt/ (t),

and so (§ 22-:34)

ns li = (- iirjK) S " + * [5" + ie(2 +i) \ - - ie-(2n+i) ix|/( i \ 2
+i

w =

X

= (- iV/iT) S [2t5-"+i sin (2n + 1) a; + (1 - q--' - ) e-(- +i '
a'J/(i \ 2n+i =o

  27r - r+' sin (27 + 1) a : \ iV,,, .

That is to say

TT 27r o-'*+i sin (2/i 4- 1) a; ns =j eosee + yJ\ L\ \ .

But, apart from isolated poles at the points x = wrr, each side of
this equation is an analytic function of x \ r\ the strip in which

IT I (r) >I x)>-'TrI (r) : - a strip double the width of that in which
the equation has been proved to be true; and so, by the theory of
analytic continuation, this expansion for ns u is valid throughout the
wider strip, except at the points x = iiir.

%
% 512
%

Example. Obtain the following expansions, valid throughout the strip '
I x)\ < itI t) except at the poles of the first term on the right-hand
sides of the respective expansions :

TT 27r == a2n + isin(2?i + l) ds u = cosec..- 2 i: .T- >

TT, 27r " o- " sin 2nx

-' cot y;r 2

2K K =i 1+?= " '

TT 27r " (-) g2n + lcos(2?l + l).r

 . secA- + 2 l-g- -i '

nc = 2ZX,sec.r -, 2 -,,,

TT, 27r " ( - )" o " sin 2?U7

2 A A: Kk =i l+y n

\Section{22}{7}{Elliptic integrals.}

An integi-al of the form IR(w, x)dx, where R denotes a rational
function

of w and x, and w is a QUARTIG, or CUBIC function of x (without
repeated factors), is called an elliptic integral*.

[Note. Elliptic integrals are of considerable historical importance,
owing to the fact that a very large number of important properties of
such integrals were discovered by Euler and Legendre before it was
realised that the inverses of certain standard types of such
integrals, rather than the integrals themselves, should be regarded as
fundamental functions of analjsis.

The first mathematician to deal with elliptic functions as opposed to
elliptic integrals was Gauss (§ 22 "S), but the first results
published were by Abelt and Jacobil.

The results obtained by Abel were brought to the notice of Legendre by
Jacobi immediately after the publication by Legendre of the Traite des
fonciions elliptiques. In the supplement (tome in. (1828), p. 1),
Legendre comments on their discoveries in the following terms : "A
peine mon ouvrage avait-il vu le jour, a peine son titre pouvait-il
§tre connu des .savans etrangers, que j'appris, avec autant
d'etonnement que de satisfaction, que deux jeunes geometres, IM.
Jacobi (C.-G.-J.) de Koeuigsberg et Abel de Christiania, avaient
reussi, par leurs travaux particuliers, a perfectionner
considerablement la theorie des fonctions elliptiques dans ses points
les plus eleves."

An interesting correspondence between Legendre and Jacobi was printed
in Journal fur Math. Lxxx. (1875), pp. 20.5-279; in one of the letters
Legendre refers to the claim of Gauss to have made in 1809 many of the
discoveries published by Jacobi and Abel. The validity of this claim
was established by Schering (see Gauss, Werke, in. (1876), pp. 493,
494), though the researches of Gauss ( Werke, in. pp. 404-460)
remained unpubli-shed until after his death.]

We shall now give a brief outline of the important theorem that every
elliptic integral can be evaluated by the aid of Theta-functions,
combined

* Strictly speaking, it is only called an elliptic integral when it
cannot be integrated by means of the elementary functions, and
consequently involves one of the three kinds of elliptic integrals
introduced in § 22 "72.

t Journal fur Math. ii. (1827), pp. 101-196.

i Jacobi announced his discovery in two letters (dated June 13, 1827
and August 2, 1827) to Schumacher, who published extracts from them in
Astr. Nach. vi. (No. 123) in September 1827 - the month in which
Abel's memoir appeared. .

%
% 513
%

with the elementary functions of analysis; it has already been seen
\hardsectionref{20}{6}) that this process can be carried out in the special case of
jiv~ dx, since

the Weierstrassian elliptic functions can easily be expressed in terms
of Theta-functions and their derivates \hardsubsectionref{21}{7}{3}).

[The most important case practically is that in which R is a real
function of x and w, which are themselves real on the path of
integration; it will be shewn how, in such circumstances, the
integral may be expressed in a real form.]

Since R (lu, x) is a rational function of w and x we may write

R (w, x) = P (w, x)IQ w, x),

where P and Q denote polynomials in w and x; then we have

R(w x)= ' ~ ' ~ wQ (w, x) Q (- w, x) '

Now Q (w, x) Q (- w, x) is a rational function of w- and x, since it
is unaffected by changing the sign of lu; it is therefore expressible
as a rational function of x.

If now we multiply out wP w, x) Q (- w;, x) and substitute for w- in
terms of X wherever it occurs in the expression, we ultimately reduce
it to a poly- nomial in X and w, the polynomial being linear in iv. We
thus have an identity of the form

R (w, x) = Ri (x) + tvRz x)]lw,

by reason of the expression for w- as a quartic in x; where jRj and
R2 denote

rational functions of x.

Now \ Ro x) dx can be evaluated by means of elementary functions
only*;

so the problem is reduced to that of evaluating jw~ Ri (x) dx. To
carry out

this process it is necessary to obtain a canonical expression for w'-,
which we now proceed to do.

\Subsection{22}{7}{1}{The expression of a quartic as the product of sums of squares.}

It will now be shewn that any quartic (or cubicf) in x (with no
repeated factors) can be expressed in the form

 A,(x- a) + BJx- Y] A, (x - af + B, (x - f],

where, if the coefficients in the quartic are real, A, B, A.., B,
a, /3 are all real.

* The integratiou of rational functions of one variable is discussed
in text-books on Integral Calculus.

t In the following analysis, a cubic may be regarded as a quartic in
which the eoefiScient of X* vanishes.

W. M. A. 33

%
% 514
%

To obtain this result, we observe that any quartic can be expressed yi
the form S S.. where Si, S2 are quadratic in x, say*

<S'i = a x + 2biX + Ci, S.2 = aox- + 2h..x + c.,.

Now, X being a constant, 1 - \ S. will be a perfect square in x if

(tti - Xftj) (ci - Xco) - (61 - \ h - = 0.

Let the roots of this equation be X, X ', then, by hypothesis,
numbers a, yQ exist such that

Si - \ 1S.2 = (o-i - Xitto) (x - a)-. Si - X S.. (a I - Xstu) x - f;

on solving these as equations in j, S.,, we obviously get results of
the form

Si = Ai x- a)- + Bi x- /3)-, S. s A (x - a)- + Bo x - /S)

and the required reduction of the quartic has been effected.

[Note. If the quartic is real and has two or four complex factors, let
Si have com- plex factors; then Xi and Xi are real and distinct since

( ! - Xa2) (c'l - XC2)- (61- X62)

is positive when X = and negative! when X = ai/ 2.

When Si and S-i have real factors, say x - i) -v - i), - 2) - 2), the
condition that Xi and X2 should be real is easily found to be

( 1 - 2) (6' - 2) ( 1 - 2') ( 1' - f/) > 0,

a condition which is satisfied when the zeros of Si and those of \& do
not interlace; this was, of course, the reason for choosing the
factors S"! and So of the quartic in such a way that their zeros do
not interlace.]

\Subsection{22}{7}{2}{The three kinds of elliptic integrals.}

Let a, /3 be determined by the rule just obtained in\hardsubsectionref{22}{7}{1}, and, in
the integral w~ Ri x) dx, take a new variable t defined by the
equation;!:

t=(x-a)l x- );

,, dx (a-B)~ dt we then have - = + .

    Aif' + Bi)(Ad' + B.2)\ i

* If the coefficients in the quartic are real, the factorisation can
be carried out so that the coefficients in .Sj and So are real. In the
special case of the quartic having four real linear factors, these
factors should be associated in pairs (to give Sj and S2) in such a
waj* that the roots of one pair do not interlace the roots of the
other pair; tlie reason for this will be seen in the note at the end
of the section.

t Unless ttj; 02 = 1 : \&2i i" which case

Si ai x-a) + Bi, S., = ao x - a) + Bo.

t It is rather remarkable that Jacobi did not realise the existence of
this homographic substitution; in his reduction he employed a
quadratic substitution, equivalent to the result of applying a Landen
transformation to the elliptic functions which we shall introduce.

%
% 515
%

If we write R x) in the form ± (a - /3) R (t), where R is rational, we
get

fRi (x) dx \ r R., (t) dt

-' ~f (A,t' + B,)(A,t'+B,)]i' Now R, (t) + R, (- t) ~ 2R, tr), R, t) -
R, (- t) = 2tR, t% where R and R are rational functions of t-, and so

Ro (t) = R, (f ) + tR, (t ).

But l (A,t;' + B,)(A.J' + Bo) tR, t")dt

can be evaluated in terms of elementary functions by taking t" as a
new variable*; so that, if we put Ri t') into partial fractions, the
problem of

integrating I R (tv, x) dx has been reduced to the integration of
integrals of the following types :

 f A,P + B,) A,t' + B,)] - dt,

[(1 + m y A,t' + B,) A,t' + B,)] - dt;

in the former of these m is an integer, in the latter m is a positive
integer andi\ \ 0.

By differentiating expressions of the form

t? -' [ A,t + B,) A.J- + i?,)]*, t (1 + m y- [ A,i:' + B,) A,t' +
5,);i,

it is easy to obtain reduction formulae by means of which the above
integrals can be expressed in terms of one of the three canonical
forms :

(i) [ A,t + B,) A,P + B.;)]-idt, (ii) 1 [(.4, + B,) A,t- + B,)] -Ut,

(iii) [(1 + m )-' (A,t' + B,) A,r- + 5,) -idt.

These integrals were called by Legendref elliptic integrals of the
first, second and thii'd kinds, respectively.

The elliptic integral of the first kind presents no difficulty, as it
can be integrated at once by a substitution based on the integral
formulae of §§22121, 22-122; thus, if A B A., B., are all positive and
A B,>A,B., we write

A, t = Bi cs (u, k). [k'-' = (A,B,)/(A,B,).]

* See, e.g., Hardy, Integration of Functions of a single Variable
(Camb. Math. Tracts, No. 2). t Exercices de Calcul Integral, i.
(Paris, 1811), p. 19.

. 33-2

%
% 516
%

Example 1. Verify that, in the case of real integi'als, the following
scheme gives all possible essentially diflerent arrangements of sign,
and determine the appropriate substitutions necessary to evaluate the
corresponding integrals.

 1

+

+

-

+

+

-

A

+

-

+

-

-

+

A,

+

+

+

+

-

-

1

+

+

+

-

+

+ :

1- cd u

Example 2. Shew that

I sn \ i du = -rr log r-,,,

I dmidu = a.m u,

I en u du = k~ arc tan k sd u),

1, dn?i + X-'

/

,, 1, 1 - en ?

as udu = - log -,

2 °l+cnM'

I sc udu=z, log '-. J-,,

I A 7 1 1 1 + sn ?i

I dc 2( du - ~ log,,

] 2 ° 1 - sn M '

and obtain six similar formulae by writing u- K for u.

\addexamplecitation{Glaisher.}

Example 3. Prove, by differentiation, the equivalence of the following
twelve expressions :

n - Ifi \ \&vr n du, J dn- M du,

// tf + dn u%cu - l<!''-\ vi.o''udu, dn tt sc - 2 J gc2 j (j 2(

M + P sn M cd - F Jcd u du,

k"- It - dn ?; cs M - Jds- u du,

Example 4. Shew that

'!" = n (n - 1) sn --' u - n- (\ + k-) sn" u + n (n + 1) F- sn + 2 du-

and obtain eleven similar formulae for the second differential
coefficients of en" ?i,

dn" ? ... nd"M. What is the connexion between these formulae and the
reduction

formula for \ t'' Ait' - Bi) A2fi + Bo)]~ dt]

\addexamplecitation{Jacobi; and Glaisher, Messenger, xi.}

Example 5. By means of § 206 shew that, if a and;3 are positive.

k'- u + k- I en- u du,

u - dmi cs u - j ns- u du,

k sn % cd it + kf Jnd xi dti,

k" u + k SQU cdu + P k" j sd - z< du,

- dn u cs ? - J cs'' u du,

u + dn w sc M - I" dc u du.

i:

 i

-a ./ <>,

where e is the real root of the cubic and

92 = .2 a - 'f-a' ', 93= - a'- ') a - 2)2 -36a2/32 /216; and prove
that, if 2 = 0, then a and are given by the equations

a- - - = - 3 (2 3)4, d' + /a2 = 2 /3 . I 2g3 |* .

%
% 517
%

Example 6. Deduce from example 5, combined with the integral formula
for en m, that, if g-i is positive,

where a2 = (V3-f) (2 3)*, /32 = (v/3 + f) (2 3)*, and the modulus is
a(aH )" -

\Subsection{22}{7}{3}{The elliptic integral of the second kind. The function* E(u).}
To reduce an integral of the type

jf' (A,t' + B,) (A,t + B,)l - idt,

we employ the same elliptic function substitution as in the case of
that elliptic integral of the first kind which has the same expression
under the radical. We are thus led to one of the twelve integrals

jsn udu, icn udu, ... jnd' udu.

By\hardsubsectionref{22}{7}{2} example 3, these are all expressible in terms of u,
elliptic functions of u and Jdn-udu; it is convenient to regard

fu

E(ii) = dn- udu

Jo

as the fundamental elliptic integral of the second kind, in terms of
which all others can be expressed; when the modulus has to be
emphasized, we write E u, k) in place oi E u).

We observe that

dE(it)

du

= dn-u, E 0) = 0.

Further, since dn u is an even function with double poles at the
points 2mK + (2n + l)iK, the residue at each pole being zero, it is
easy to see that E(u) is an odd one-valuedf function of u with simple
poles at the poles of dn u.

It will now be shewn that E () may be expressed in terms of Theta-
functions; the most convenient type to employ is the function (u).

 °'''''' '' m %U\ '

it is a doubly-periodic function of u with double poles at the zeros
of © (u), i.e. at the poles of dn u, and so, if A be a suitably chosen
constant,

dn-u - A - -,; ' du (©(z<.)

* This notation was introduced by Jacobi, Journal fiir Math. iv.
(1829), p. 373 [Ges. Werke, I. (1881), p. 299J. In the Fundamenta
Nova, he wrote E (am u) where we write E (u).

+ Since the residues of dn m are zero, the integral defining E u) is
independent of the path chosen \hardsectionref{6}{1}).

%
% 518
%

is a doubly-periodic function of u, with periods 2K, 2 K', with only a
single

simple pole in any cell. It is therefore a constant; this constant is
usually

written in the form JE/K. To determine the constant A, we observe that

the principal part of dn- u at iK' is - (u - iK')~, by\hardsubsubsectionref{22}{3}{4}{1}; and
the

residue of \&' (u)IS (u) at this pole is unity, so the principal part
of

d 10'(u)) . \ .,

-y- <,, ' y IS - n - iK ) -. Hence yi = 1, so

du [yd u))

Integrating and observing that H' (0) = 0, we get E (w) = ©' (;f )/0
(w) + tiEjK. Since ©' K) = 0, we have E (K) = E; hence

E=( dn- ucZw=r (l-k'sm"-0dcf> = l7rF(-l, ~; 1; kA .

It is usual (cf.\hardsectionref{22}{3}) to call K and E the complete elliptic
integrals of the first and second kinds. Tables of them qua functions
of the modular angle are given by Legendre, Fonctions Elliptiques, ii.

Example 1. Shew that E u + 2nK)= E ti) + 2nE, where n is any integer.

Example 2. By expressing e u) in terms of the function 9 ( irulK), and
expanding about the point u = iK', shew that

 = (2-P-V7W5i') ir.

\Subsubsection{22}{7}{3}{1}{The Zeta-f unction Z (w).}

The function E (u) is not periodic in either 2K or in 2iK', but,
associated with these periods, it has additive constants 2E, 2iK'E -
7ri]/K; it is convenient to have a function of the same general type
as E u) which is singly-periodic, and such a function is

Z (u) = ©' (m)/© (u); from this definition, we have*

Z (u) = E (ti) - uE/K, © (m) = © (0) exp I j" Z t) dt\ .

\Subsubsection{22}{7}{3}{2}{The addition-formulae for E u) and Z u).}

Consider the expression

e'(u + v) ©'(m) W(v),,,

-- ' / - T TT x - /-v / + f ' sn u sn V sn u + v)

@(u + v) ©(m) S v)

* The integral in the expression for (u) is not one-valued as Z (f)
has residue 1 at its poles; bat the difference of the integrals taken
along any two paths with the same end points is 2/i7r/ where n is the
number of poles enclosed, and the exponential of the integral is
therefore one- valued, as it should be, since 0(m) is one-valued.

%
% 519
%

qua function of m. It is doubly-periodic* (periods 2if and 2iK') with
simple poles congruent to iK' and to iK' - v; the residue of the
first two terms at iK' is - 1, and the residue of sn u sn v sn (u + v)
is k~ sn v sn (iK' + v) = k~ .

Hence the function is doubly-periodic and has no poles at points
congruent to iK' or (similarly) at points congruent to iK' - v. By
Liouville's theorem, it is therefore a constant, and, putting u = 0,
we see that the constant is zero.

Hence we have the addition-formulae

Z (w) -f- Z (v) - Z(u+v) = k sn M sn V sn (?< + v),

E(u) + E (v) - Eiu 4- v) = k sn u sn v sn u + v).

[Note. Since Z u) and E (u) are not doubly- periodic, it is possible
to prove that no algebraic relation can exist connecting them with sn
u, en u and dn u, so these are not addition-theoreras in the strict
sense t.]

\Subsubsection{22}{7}{3}{3}{Jacobi's imaginary transformation\ of7i u).}

From\hardsubsectionref{21}{5}{1} it is fairly evident that there must be a transformation
of Jacobi's type for the function Z ii). To obtain it, we translate
the formula

 2 ix i t) = (- ir) exp (- iTX /ir) . 4 (ixr \ r) into Jacobi's
earlier notation, when it becomes

H (iu + K, k) = (- ir)i exp (\ (, k'),

and hence

/ Tru-" \ 4 (0 I t) © (u, k')

en (in, k) = (- V)* exp ( j

Taking the logarithmic ditferential of each side, we get, on making
use of\hardsectionref{22}{4},

Z iu, k) = i dn (u, k') sc u, k') - iTj u, k') - 7riu/(2KK').

22 734. Jacobi's imaginary transformation of E(u).

It is convenient to obtain the transformation of E (u) directly from
the integral definition; we have

E (in, k) = I "dn- t, k) dt=\ dn- (it', k) idt'

Jo Jo

= i dc- (t', k') dt', ]

on writing t - it' and making use of § 22 "4.

* 2iK' is a j eriod since the additive constants for the first two
terms cancel.

t A theorem due to Weierstrass states that an analytic function,/ (2),
possessing an addition- theorem in the strict sense must be either

(i) an algebraic function of z, or (ii) an algebraic function of exp
(Trizjw),

or (iiij an algebraic function of (2 | wi, W2) J

where w, wj, W2 are suitably chosen constants. See Forsyth, Theory of
Functions (1918), Ch. xiii.

J Fundamenta Nova, p. 161.

%
% 520
%

Hence, from\hardsubsectionref{22}{7}{2} example 3, we have

E hi, k) = i \ u + dn u, k') sc (, k') - \ dn= t', k') dt' [,

and so E iu, k) = iu + i dn u, k') sc u, k') - iE u, k').

This is the transformation stated.

It is convenient to write E' to denote the same function of k' as E is
of k, i.e. E' = E(K', k'X so that

E 2 K',k) = 2i(K' -E').

\Subsubsection{22}{7}{3}{5}{Legendre's relation*.}

From the transformations of E u) and Z u) just obtained, it is
possible to derive a remarkable relation connecting the two kinds of
complete elliptic integrals, namely

EK' + E'K-EK' = \ ir.

For we have, by the transformations of §§ 22"733, 22*734,

E iu, k) - Z iu, k) = iu - i [E u, k') - Z u, k')\ + '7riu/ 2KK'),

and on making use of the connexion between the functions E u, k) and Z
u, k), this gives

iuE/K = iu - i [uE'jK'] + 'Triu/ 2KK'). Since we may take m =| 0, the
result stated follows at once ft-om this equation; it is the analogue
of the relation rj coo - Vzf i = 9 tJ"* which arose in the
Weierstrassian theory (§ 2041 1). Example 1. Shew that

E u + K)-F ii) = E-k-mucdu. Example 2. Shew that

E(2u + 2iK') E (2u) + 2i K' - E'). Example 3. Deduce from example 2
that

E ti + iK') = E 2u + 2iK') + W sn ( u + iK') sn (2m + 2iK') = E (u)
-fen M ds M + i (A" - £"). Example 4. Shew that

E u + K+ iK') = E (u) - sn u dc u + E+i E' - E'). Example 5. Obtain
the expansions, vaHd when | I x) l< 7r/(T),

/7 r'NO o tt;,'T-, o ' iw" COS 2ii.r . -, °° Q' s\ Vi2nx n=i q " n=\
1-?''"

\addexamplecitation{Jacobi.}

* Exercices de Calcul Integral, i. (1811), p. 01. For a geometrical
proof see Glaisher, Messenger, iv. (1874), pp. 95-96,

%
% 521
%

\Subsubsection{22}{7}{3}{6}{Properties of the complete elliptic integrals, regarded as functions of the modulus.}

If, in the formulae B=\ (1 - k sin (f)y d<f>, we differentiate under
the

Jo

L \hardsectionref{4}{2}), we have

fh . . \ i E-

= - I k sm- (f) (1 - k" sm (f)) -d(J3 = - j-

J K

Drmula for K in the same manner, we ha'

= ' sin- < ( 1 -  ' sm" (j>) d(f) = k \ sd u di

J a Jo

sign of integration \hardsectionref{4}{2} ), we have

dE fh, . . . .., . ..\ ! .. E-K

dk

Treating the formula for K in the same manner, we have

dJ dk

K

jLj/ dn'u*,.-

k!''-u

by\hardsubsectionref{22}{7}{2} example 3; so that

dk kk'- k If we write k- = c, k'-= c', these results assume the forms
dE E-K dK E-Kc' dc c dc cc

Example 1. Shew that

 dE K'~E' 2 = cK'-E'

dc c' ' dc cc'

Example 2. Shew, by difterentiation with regard to c, that EK' + E'K -
KK' is constant.

Example 3. Shew that K and K' are solutions of

d dk

and that E and E' - K' are sohitions of

|m' | |= ..,

 '" ~M- \ \ 'M ' ~ ' (Legendre. )

\Subsubsection{22}{7}{3}{7}{The values of the complete elliptic integrals for small values of k. }
From the integral definitions of E and K it is easy to see, by
expanding in powers of k, that

lim K = lim E l-TT, lim (K - E)lk' = tt.

In like manner, lim E' = I cos (f)d(j) = 1.

k- -o J

It is not possible to determine lim/i' in the same way because

fc O

(1 - '-sin ( )" 2 is discontinuous at </> = 0, k =; but it follows
from example 21 of Chapter xiv (p. 299) that, when | argA- 1 < tt,

lim i '-log(4/ )l=0.

%
% 522
%

This result is also deducible from the formulae 2iK' = -!TT i, k .
jBi, by making q-*-0; or it may be proved for real values of k by the
following elementary method :

By\hardsubsectionref{22}{3}{2}. A" = / f- - k-) ' l-t')' dt; now, when / < < Jk, l-t-)
lies between

1 and 1-/-; and, when s,'k<t<\ \ fi-k )/t lies between 1 and 1 ~k.
Therefore A'' lies between

Jk i /A-

and (i-/.)-*]/" ( 2\ .2)-ij;+ / t-- l-f- )-hdt\;

and therefore

A =(1 - 6k) ijlog log 1-- /(i l.

= (1 - Ok) - h [2 log 1 + v'(l - k) - log /], where 0 1.

Xow lira [2 (1 - Ok) ' i log 1 + v/(l - k) - log 4] = 0,

\ \ m I - I - 6k) - log k = 0,

and therefore lim A' - log (4/),-) = 0,

which is the required result.

Example. Deduce Legendre's relation from\hardsubsubsectionref{22}{7}{3}{6} example 2, by making
k- 0.

\Subsection{22}{7}{4}{The elliptic integral of the third kind*.}
To evaluate an
integral of the type

f(l + Nt')-' A,P + B,) A,t'+B. ] - dt

/<

in terms of known functions, we make the substitution made in the
corre- sponding integrals of the first and second kinds (§§ 22*72,
22"73). The integral is thereby reduced to

I -- du = ait + (p - av) - du,

where or, yS, v are constants; if i = 0, - 1, x or - the integral
can be expressed in terais of integrals of the first and second kinds
; for other values of V we determine the parameter a by the equation v
= - k sn- a, and then it is evidentl ' permissible to take as the
fundamental integral of the third kind

, [ k-snacn a dn a sn' u,

n (u, a) = - j~. - du.

  Jo 1 - k sn a sn u

To express this in terms of Theta-functions, we observe that the inte-
grand may be written in the form

I k- sn u sn a sn (u + a) -1- sn (u - ) = ( (" - ) - (" + a) + 2Z (a)
,

* Legendre, Exercices de Calcul Integral, i. (1811), p. 17; Fonctions
EUiptiques, i. (1825), pp. 14-18, 74, 75; Jacobi, Fundamenta Nova
(1829), pp. 137-172; we employ Jacobi's notation, not Legendre's.

%
% 523
%

by the addition-theorem for the Zeta-fimction; making use of the
formula Z (i() = B'(m)/ (u), we at once get

a result which shews that U(u, a) is a many- valued function of u with
logarithmic singularities at the zeros of @ (u ± a).

Example 1. Obtain the addition-formula*

e (u+v + a) e ( tt-g) e (v -a)

n u, a)+n v, a)-niu+v, ) = iloge( +.,-a)e( -ha)e( + )

j 1 - Psnasn - sn-?;sn (tc + v - a) ~2 °"l + Psn asnwsnvsn (w + v +
a)'

\addexamplecitation{Legendre.}

(Take a- : y : s : tf = m : w : ± a : i( -h r ± a in Jacobi's
fundamental formula

[4] + [l] = [4]'-r[l]'.)

-Example 2. Shew that

n u, a) - n (a, m) = uZ (a) - aZ (tt).

\addexamplecitation{Legendre and Jacobi.}

[This is known as the formula for interchange of argument and
parameter.]

Example 3. Shew that

l-PsnaHnbsnuHn a + b-u) n (iL a) + n hi, h) - n (u, a + b) = i log \,
- j r -. -, i,,,\

+ uk sn a sn 6 sn (a 4- 6).

\addexamplecitation{Jacobi.} [This is known as the formula for addition of parameters.]

Example 4. Shew that

IT iio, ia + /i, k) = Il u,a + K\ k'). \addexamplecitation{Jacobi.}

Example 5. Shew that

n (m + v, a-irb) + n (u-v, a-b)-2n u, a)-2U v, b)

,,,, l-Fsn2(2<-a)sn2(i;-6) = - F sn a sn 6 . (u + v) sn (a -f b) -
(w - y) sn ( - 6) -H i log i + 2 . 2 (;,,) s ( + 6) '

and obtain special forms of this result by putting v or h equal to
zero. \addexamplecitation{Jacobi.}

22 741. A dynamical application of the elliptic integral of the third
kind. It is evident from the expression for n (m, a) in terms of
Theta- functions that if u, a, k are real, the average rate of
increase of n (m, a) as u increases is Z (a), since 9 ii±a) is
periodic with resjject to the real period 2K.

This result determines the mean precession about the invariable line
in the motion of a rigid body relative to its centre of gravity under
forces whose resultant passes through its centre of gravity. It is
evident that, for purposes of computation, a result of this nature is
preferable to the corresponding result in terms of Sigma-functions and
Weierstrassian Zeta-functions, for the reasons that the
Theta-functious have a specially simple behaviour with respect to
their real period - the period which is of importance in Applied
Mathe- matics - and that the -series are much better adapted for
computation than the product by which the Sigma-function is most
simply defined.

* No fewer than 96 forms have been obtained for the expression on the
right. See Glaisher, Messenger, x. (1881), p. 124.

%
% 524
%

\Section{22}{8}{The lemniscate functions.}

The integral (1 - ) dt occurs in the problem of rectifying the arc of

Jo

the lemniscate*; if the integral be denoted by, we shall express the
relation between and x by writingi* x - sin lemn < .

In like manner, if

J X - .'

we write

x = cos lemn j, and we have the relation

sin lemn < = cos lemn ( tn- - ( j .

These lemniscate functions, which were the first functions:]: defined
by the inversion of an integral, can easily be expressed in terms of
elliptic functions with modulus l/VS; for, from the formula \hardsubsubsectionref{22}{1}{2}{2}
example)

r sd u u=\ [ l-k' f) l+l: f)]-Uy,

.

it is easy to see (on writing y = t \/2) that

sin lemn = 2 " sd (< \/2, l/V ); similarly, cos lemn = en (</> V2, 1/
2).

Further, is the smallest positive value of for which

cn(< V2, 1/V2) = 0, so that OT=V2 o,

the suffix attached to the complete elliptic integral denoting that it
is formed with the particular modulus l/\/2.

This result renders it possible to express Kq in terms of
Gamma-functions,

thus

K,= 2 [ P)- -dt='2---\ \ u - l-u)- -di Jo Jo

a result first obtained by Legendre§.

Since k = // when k = l/\/2, it follows that 7 = 7iV. and so o = e ".

* The equation of the lemniscate being r- = a' cos 26, it is easj' to
derive the equation

 %)' = ar~* '' '" ' ' ' " " tj= 1 + (1-)'

t Gauss wrote si and cl for siu lemn and cos lemn, Werke, iii. (1876),
p. 493.

t Gauss, Werke, iii. (187(5), p. 40i. The idea of investigating the
functions occurred to Gauss on January 8, 1797.

§ Exercices de Calcul Integral, i. (Paris, 1811), p. 209. The value of
Kq is 1-85407468... while 07 = 2-62205756....

%
% 525
%

Example . Express A'o in terms of Gamma-fuuctions by usiug Kummer's
formula (see Chapter xiv, example 12, p. 298).

Example 2. By writing t = \ -m ) j the formula

shew that 2 o = / ( 1 - ?**) ' du+T u l- ic*) ~ i du,

and deduce that 2 o - J o = Stt r ( j) ~ 2,

Example 3. Deduce Legendre's relation \hardsubsubsectionref{22}{7}{3}{5}) from example 2
combined with\hardsubsubsectionref{22}{7}{3}{6} example 2.

Example 4. Shew that

.,, 1 - cos lemn (b

sm Iemn2d) = - --t- .

H-coslemn''9

\Subsection{22}{8}{1}{The values of K and K' for special values of k.}

It has been seen that, when k=ljJ2, K can be evaluated in terms of
Gamma-functions, and K=E'; this is a special case of a general
theorem* that, whenever

E' \ a + hjn E c + djn'' where, b, c, d, n are integers, k is a root
of an algebraic equation with integral coefficients.

This theorem is based on the theory of the transformation of elliptic
functions and is beyond the scope of this book; but there are three
distinct cases in which k, 7i, E' all have fairly simj le values,
namely

(I) >(- = V2-l, E' = EJ2,

(II) k = Hm 7r, E' = E %

(III) k=tsin 7r, E' 2E. Of these we shall give a brief investigation
t.

(I) The quarter-periods with the modulus /2 - 1.

Landen's transformation gives a relation between elliptic functions
with any modulus k and those with modulus ki = l - k')/ l+k'); and the
quarter-periods A, A' associated with the modulus k satisfy the
relation A'/a = 2E'/E.

If we choose k so that ky = k', then A = /i'' and k = k so that i. ' =
E; and the relation A7A=2A'7A'gives A'2=2a2.

Therefore the quarter-periods A, A' associated with the modulus k-
given by the equation i = (l - i)/(l-|-X-i) are such that A'=±Av/2;
i.e. \ i ki = l2-\, then A' = Av/2 (since A, A' obviously are both
positive).

(II) The quarter-periods associated ivith the modidus sin y tt.

The case of k = s,\ n -rrr was discussed by Legendre|; he obtained the
remarkable result that, with this value of k,

E' = Ey/3.

* Abel, Journal fiir Math. in. p. 184 [Oeuvres, i. (1881), p. 377].

t For some similar formulae of a less simple nature, see Kronecker,
Berliner Sitzungsherichte, 1857, 1862.

X Exercices de Calcul Integral, i. (1811), pp. 59, 210; Fonctions
Elliptiques, i. (1825), pp. 59, 60.

%
% 526
%

This result follows from the relation between detinite integi-als

To obtain this relation, consider l l-z )~ dz taken round the contour
formed by the part of the real axis (indented at s=l by an arc of
radius R' ) joining the points and B the line joining e*"' to and the
arc of radius B joining the points R and Re ""; as R a:, the
integral round the arc tends to zero, as does the integral round the
indentation, and so, by Cauchy's theorem,

on writing .v and e*'"* respectively for z on the two straight lines.
Writing [\ l-x )- -dx = Ii, j\ x- -l)- dx = L, j\ l+.f )-- ds=j\ l-x
)- dx I

we have /j + iV. = 4 (1 + is/S) h;

so, equating real and imaginary parts,

A = 5- 3 5 - 2 = 5- 3X 3,

and therefore h-V I - Lis'' = \ h + h-¥i=

which is the relation stated*. Now, by\hardsubsectionref{22}{7}{2} example 6,

/2 = 4(a2 + 2)-*£', / + /3 = 4(a2 + /32)-4 ',

where the modulus is a d' + ) ~ and

a2=2v3-3, /32 = 2V3 + 3, so that, > :2 = i(2-V3) = sin2Js7r.

We therefore have

3-i. 2/1 = 3-2. 2 ' = /2 = 3 /i

= 3- |V"s(i- )-ic =i,r r(J)/r(|),

when the modulus k is sin jV -

(III) The quarter-periods iiith the modtdus tan -i-Tr.

If, in Landen's transformation \hardsubsectionref{22}{4}{2}), we take l::=i; '2, we have
A'/A=2K'/K=2;

now this value of k gives

7 V -1 . "1

and the corresponding quarter-periods \, A' are |(1 + 2 ) A'o and (1 +
2 ) Kq.

Example 1. Discuss the quarter-i)eriods when k has the values
(2;,y2-2), sin f 7r, and 2 ( 2-1).

* Another method of obtaining the relation is to express Ij, I, h in
terms of Gamma- functions by writing t, t~, ( -' - 1)* respectively
for x in the integrals by which Ii, I2, I3 are defined.

%
% 527
%

Example 2. Shew that

?l = )i = l

\addexamplecitation{Glaisher, Messenger, v.}
Example 3. Express the coordinates of any
point on the curve y =ofi- 1 in the form

1 3

3 (1- en u) \ 2.3 snMdnM

 ~ l+cn?< ' ' ~ (l + cnM)2 '

where the moduhis of the eUiptic functions is sin jV"", and shew that
-=- = ~ iy.

By considering / y~ dx = 3~* I die, evahiate K in terms of
Gamma-functions when

Example 4. Shew that, when y' =x -l,

r y-Hx-lYdx=[- y- l-x-' fl +' T x~ y- -x-h/- )dx;

7 1 L ' Ji ' y 1

and thence, by using example 3 and expressing the last integral in
terms of Gamma- functions by the substitution x = t~, obtain the
formula of Legendre Calcul Integral, p. 60) connecting the first and
second complete elliptic integrals with modulus sin jJjtt:

Example 5. By expressing the coordinates of any point on the curve Y'
= X in the form

 \ 32(l-cnv) \ 2. 3*sn vdn??

1+cni? ' (1-l-cn ') '

in which the modulus of the elliptic functions is sin y tt, and
evaluating

 />/

Y- XfdX

in terms of Gamma-functions, obtain Legendre's result that*, when k sm
w,

\Subsection{22}{8}{2}{A geometrical illustration of tlie functions sn u, en ii, dn u.}

A geometrical representation of Jacobian elliptic functions with
k=ljJ2 is afforded by the arc of the lemniscate, as has been seen in §
22"8; to represent the Jacobian functions with any modulus k 0 <k < )
, we may make use of a curve described on a sphere, known as Seifferfs
spherical spiral .

Take a sphere of radius unity with centre at the origin, and let the
cylindrical polar coordinates of any point on it l e (p,, z), so that
the arc of a curve traced on the sphere is given by the formula \

 dsr =pHdcf>y+ i-p )- dpy.

* It is interesting to observe that, when Legendre had proved by
differentiation that EK' + E'K- KK' is constant, he used the results
of examples 4 and 5 to determine the constant, before using the
methods of\hardsectionref{22}{8} example 3 and of\hardsubsubsectionref{22}{7}{3}{7}.

t Seiffert, Ueher eine neue geometrische Eiiifilhrung in die Theorie
der elliptisclien Funktiunen (Charlottenburg, 1896).

X This is an obvious transformation of the formula ds)"= dp)' + p'
(d(p)~ + dz) when p and z are connected by the relation p' + z'- - l.

%
% 528
%

Seiffert's spiral is defined by the equation

(f) = ks,

where s is the arc measured from the pole of the sphere (i.e. the
point where the axis of s meets the sphere) and k is a positive
constant, less than unity*.

For this curve we have

and so, since s and p vanish together,

p = sn (s, k).

The cylindrical polar coordinates of any point on the curve expressed
in terms of the arc measured from the pole are therefore

 p, cf), z) = (sn s, /ts, en s);

and dn s is easily seen to be the cosine of the angle at which the
curve cuts the meridian. Hence it may be seen that, if K be the arc of
the curve from the pole to the equator, then sn s and cu s have period
AK, while dn 6* has period 2K.

REFERENCES.

A. M. Legendre, Traite des Fonctions Elliptiques (Paris, 1825-1828).

C. G. J. Jacobi, F\ hiidamenta Nova Theoriae Functionv.m Ellipticanim
(Konigsberg, 1829).

J. Tannery et J. Molk, Fonctions Elliptiques (Paris, 1893-1902).

A. Cayley, Elliptic Functions (London, 1895).

P. F. Verhulst, Traite eUmentaire des fonctions elliptiques (Brussels,
1841).

A. Enneper, Elliptische Funktionen, Zweite Auflage von F. Miiller
(Halle, 1890).

Miscellaneous Examples.

1. Shew that one of the values of

11 1 1

dnM + cnwV- /dn w-cn wV'j U l-sn% / l + sni* \ \ '\

l+cn?t / \ - cnn J ) \ \ dinu - k'su.u) \ 6.nu + k' snu) J

is 2 (1 +k'). \addexamplecitation{Math. Trip. 1904.}

2. li x + iy = sn" ( u + iv) and x - iy - sn (% - iv), shew that

 ( -l)2+2/- - = ( -+y ) dn22( + cn2w.

\addexamplecitation{Math. Trip. 1911.}

3. Shew that

 l±cn( + .) l±cn( -i;) = /'-'?/= . 1 - '= sn w sn- V

4. Shew that

, cn u + cn V

1 +cn (w + v) en u-v) = - - -p, - 5- .

  1 - A: sn-M sn V

\addexamplecitation{Jacobi.} * If fc>l, the curve is imaginary.

%
% 529
%

p \ + cn(ti + v)cn(u-v) r . c o, i

5. Express - \ (-; - 7 ( as a function or sn-'w + sn v.

   + an u + v)axi(u - v)

6. Shew that

7. Shew that

sn u - v) dn u + v) =

\addexamplecitation{Math. Trip. 1909.}
sn M dn -M en V - sn v dn i? en u

1 - 2 sn" u su" r

\addexamplecitation{Jacobi.}

    + k')s,r).us,n u- E)) l~ k') sn m sn u + K)] = su ( . + ) - sn uf.

\addexamplecitation{Math. Trip. 1914.}

8. Shew that

9. Shew that

sn (i< + |iA''') =

(1 -/(:')

\ 1 (1 + ) sn M + en w dni< 1 + sn M

.,,,,,, 2sn?icnMdnv

sm am i,, + v) + - m u-v) j-j,~,

,, ., ., cn V - sn V dn m

cos am u + v)- m ( - )H 2 3 2,,s 2

\addexamplecitation{Jacobi.}

10. Shew that and hence express

dn u+v) dn u - v) =

ds2?ids- v + F '2

ns ?i ns V - -2 '

(u + v) - €2 iP u - vj - e-

as a rational function of ip u) and v). \addexamplecitation{Trinity, 1903.}

11. From the formulae for cn(2 - ) and diW 2K-u) combined with the
formulae for 1 +cn 2u and l + dnSw, shew that

(l-cn|ir)(H-dn§ ) = l. \addexamplecitation{Trinity, 1906.}

12. With notation similar to that of\hardsectionref{22}{2}, shew that

Ci d-i - c<idx en ( i + u-i) - dn ( Mj + 2) Sj-sa sn( i4- 2)

and deduce that, if Mi + (/o f-% + '4 = 2A', then

(Ci 0?9 - C2 0?i) (C3 c/4 - C4 ( 3) = 2 (sj \ s.,) (S3 - S4).

\addexamplecitation{Trinity, 1906.}

13. Shew that, if u + i + ?r = 0, then

1 - dn XL - dn v - dn-  + 2 dn m dn v dn u'=k* sn ?< sn v sn m;.

\addexamplecitation{Math. Trip. 1907.}

14. By Liouville's theorem or otherwise, shew that

dn?< dn u + w)-dn <;dn v + w) = k sn yen u sn (v + w) en (u + w)

- snMcn vsn u + xo) en (y +;) .

\addexamplecitation{Math. Trip. 1910.}

15. Shew that

2 en Uo en u sn (?<2 - %) dn Mj + sn ( 2 - M3) sn (W3 - ?<i) sn (i<i -
Mo) dn ?ii dn Mo dn % = 0, the summation applying to the suffices 1,
2, 3. \addexamplecitation{Math. Trip. 1894.}

W. M. A. 34

%
% 530
%

16. Obtain the formulae

sn 3 = AID, cii 3?< = BjD, dn 'Mi = CLD, where = 3s - 4 ( 1 + /f- ) s
+ Qk' s - k* s%

B = c l-4:S + 6k s*-4kU' + k s,

C=d l-4k s + 6/5-2 4 \ 4X.2 c + ia gsx

D = l- 6 -2 S-* + 4:k- ( 1 + /?;2) §0 - Zk s% and s = sn, c' = cni(,
c/=dn;<.

17. Shew that

1 - dn 3u \ / 1 - dn ?t\ /I + a dn u + 02 dn u + c'3 dn ?< + a dn 1 +
dn 'iu \ l + dn tcj \ l - aj dn ?< + 2 dn 14 - ag dn % + a dn* uj
where Ui, ao, 3, a are constants to be determined. \addexamplecitation{Trinity, 1912.}

18. If

shew that

  . ., 1 + dn 3i( W + dn ?<

P u) + F tt+2iK')

sn 2u en u en 2m sn u '

F u)-F u + 2iK')

Determine the poles and zeros of P (u) and the fii'st term in the
expansion of the function about each pole and zero.

\addexamplecitation{Math. Trip. 1908.}

19. Shew that

sn ui + U-2 + U3) = A/D, en ui + U2 + ii ) = BjD, dn ui + M2 + %) =
C'/i), where

A=SiSoSi - 1 --?-2 + 2F2V-('<'-' + -'') 252 532+ 2ytisi2s22 532)

+ 2 s c.c dod i l + 2k' s.isi-B s. si)],

B =CiC2Cs 1 - k 2s.2 S3 +2k*S S2 S-/

+ 2 CiS2.?3O?20?3(- 1 +2Ps2 S32 + 2Fsi2-F25225g2))

C = d,d2d3 l-P2s./ss' + 2F-s, s.2H:i

+ F2 diS SsCoC-i ( - 1 +2/ -2s22s32 + 2Si2-F2S22532), D=l - 2P2S2'-
S32 + 4 (F + F) Si2 2 3- " 2k*Si S2-S3~2Si- + B2s.*S3*,

and the summations refer to the suffices 1, 2, 3. (Glaisher,
Messenger, xi.)

20. Shew that

sn (ui + W2 + '-s) = - '/D', en ui + U2 + its) = '/- 'j dn (u + U2 +
u. ) = G'jD', where J' = 2siC2C3C 2 3--5iS2S3(l+ - 2si2+/
;4si2s22s32),

5' = Ci C2C3 (1 - * Si2s22 32) - di d d 'Ss s Ci di,

C" = 0?i C?2 0?3 ( 1 - F Si2 §22 S32) - F Cj C2 C3 2 2 S3 <?! C l,

Z)' = l - F 25-22 S32 + (F + *) Si2s22s32-FsjS2S32SiC2C3C 2' 3-

((/'ayley, Journal fur Matli. XLi.)

21. By applying Abel's method \hardsectionref{20}{3} 12) to the intersections of the
twisted curve;j72\ |.y2\ i 22 + F\ .2\ .i itii the variable plane Ix
+ my + nz-l, shew that, if

'Mi + M2 + % + 4 = 0j

then Si Ci d 1 =0.

52 C2 2 1

53 C3 0?3 1

54 C4 0 4 1

OVjtain this result also from the equation

( 2 - i) ( 3 4 - 40 3) + ( 4 - s which may be proved by the method of
example 12.

 cid2~C2di) = 0,

\addexamplecitation{Cay ley, Messenger, xiv.}

%
% 531
%

22. Shew that

by expressing each side in terms of Si, §2, s, s; and deduce from
example 21 that, if

TODO

\addexamplecitation{Forsyth, Messenger, xiv.}

23. Deduce from Jacobi's fundamental Theta-function formulae that, if

Ui + U.2 + M3 + M4 = 0,

then k'- - k k"'si So S3 S4 4- k Ci c, c c - di d.2 d d = 0.

\addexamplecitation{Gudermann, Journal fur Math, xviii.}

24. Deduce from Jacobi's fundamental Theta-function formulae that, if

Ul + U.2 + lt3 + Ui = 0,

then F (si S2 C3 C4 - Ci 6'2 S3 S4) - c ia?2 + < 3 0 4= 0,

k' S1S2- 838 ) +did2C3Ci - CiC2d3di = 0,

.SiS2C 3( 4 - did2S Si + C3C4 - Ci C2 = 0.

(H. J. S. Smith, Proc. London Math. Soc. (1), x.)

25. If Ui + Uo + n3-)rUi = 0, shew that the cross-ratio of sn Ui, sn
Uo, sn M3, sn ?<4 is equal to the cross-ratio of sn (u + K)., sn
u-i+K), sn 11,3 + K\ sn Ui + K).

\addexamplecitation{Math. Trip. 1905.}

26. Shew that

iin ii,+v) sn (ji-f-y) sn (w- <;) fiTi? u-v), Sk"' SiS2 CiC did

cn2 (w -I- v) en (m -f v) en ( - v) en- (w - v) (, - 'Si S2 J

dn' ( -|-i ) dn (m -t- 2 ) dn (?( - y) dn u - v)

\addexamplecitation{Math. Trip. 1913.}

27. Find all systems of values of u and f for which iivi u + iv) is
real when u and y are real and 0<k-<. \addexamplecitation{Math. Trip. 1901.}

28. If k' = J a ~ - a)'-, where < a < 1, shew that

' 1 r- 4a3

 "(l-f-a2)(l + 2a-a- )'

and that sn'- fiT is obtained by writing -a~ for a in this expression.

\addexamplecitation{Math. Trip. 1902.}

29. If the values of en z, which are such that en 3z = a, are Cj, C2,
... Cg, shew that

9 9

3k* n Cr + k'* 2 c,. = 0.

\addexamplecitation{Math. Trip. 1899.}

y,. g-l-sn (M-l-y) \ 6-|-cn( f-f y) \ c + dn u + v)

a + su u - v) b + cn u - v) c + dii it - v)'

and if none of snv, cnu, dnu, 1 - -' sn wsn y vanishes, shew that ti
is given by the equation

X-2 (t' d' + b''- c ) sn- u = k'-+k' b'' - c\

\addexamplecitation{King's, 1900.}

34- :i

%
% 532
%

31. Shew that

\addexamplecitation{Math. Trip. 1912.}

32. Shew that

l-sn(2 .r/7r)

"Jl-2j2n-lgin-j; + j4n-2

 dn 2KxlTr) - /;' sn (2A'j

\addexamplecitation{Math. Trip. 1904.}

33. Shew that if k be so small that k may be neglected, then

sn M = sin u - k-cosic. (t< - sin iccos u), for small values of ? .
\addexamplecitation{Trinity, 1904.}

34. Shew that, if | / x) \ < nl (r), then

4j" sin nx

log en (2 A'.r/Tr) = log cos .r - 2 -

 =iw l + (-?)"

\addexamplecitation{Math. Trip. 1907.} [Integrate the Fourier series for sn (2A'' /7r)dc
(2A'r/7r).]

35. Shew that

cn 71 dn'= m

\addexamplecitation{Math. Trip. 1906.} [Express the integrand in terms of functions of
2u.]

36. Shew that

/cnvdu \ . 5i (|a' + -| - i7r) i(| .r + 5y-|7r - -iTrr) i iZ + hirr)
snv-snT<~ Ml- - y) h il/ - i r) '* Ij y+i" ) '

where 2Kx = Tru, 2Ky=irv. \addexamplecitation{Math. Trip. 1912.}

37. Shew that

  jo l+cuu)dn u

\addexamplecitation{Math. Trip. 1903.}

,,' +,, l+X-snasn/3

k I sn ?(a; = io2;

38. Shew that

/ I v'r -i/ /-/)/ - Irvnr

1 - /;sn asn j3'

\addexamplecitation{St John's, 1914.}

39. By integrating je ' dmicsudz round a rectangle whose corners are
±i7r, ± ir + cci (where 2Kz = itii) and then integrating by parts,
shew that, if </:- < 1, then

\ cos (iru/K) logsn t(,dii = K timh. ( irir). Jo'

\addexamplecitation{Math. Trip. 1902.}

40. Shew that K and K' satisfy the equation

c(l- ) + (l-2c) -i = 0,

where c = 2. j j-kJ deduce that they satisfy Legendre's equation for
functions of degree - with argument 1 - 2k' .

%
% 533
%

41. Express the coordinates of any point on the curve x +y =l in the
form 2.3 snMdnM-(l-cn u) \ 2 cos 3 tt (1 - on w) 1 + tan jLtt en u)

2.3isn%dntt + (l-cn w)2 2. 3 snwdn m + (1 -cnw)'

the modulus of the elliptic functions being sin yV 't; and shew that

J X Jo

Shew further that the sum of the parameters of three coUinear points
on the cubic is a period.

[See Richelot, Journal fiir Math. ix. (1832), pp. 407-408 and Cayley,
Proc. Camb. Phil. Soc. IV. (1883), pp. 106-109. A uniformising
variable for the general cubic in the canonical form X + F + Z +
6rnXYZ=0 has been obtained by Bobek, Einleitung in die Theorie dex
elliptischen Funktionen (Leipzig, 1884), p. 251. Dixon Quarterly
Journal, xxiv. (1890), pp. 167-233) has developed the theory of
elliptic functions by taking the equivalent curve a +y' - cuicy=\ as
fundamental, instead of the curve

y2=(l-. 2)(1\ .2 .2).]

42. Express I 2x-x-) (4 -2 + 9) ~ " dx in terms of a complete elliptic
integral of the first kind with a real modulus. \addexamplecitation{Math. Trip. 1911.}

43. If u=l t + ) t: + t + ) ]- dt,

express x in terms of Jacobian elliptic functions of u with a real
modulus.

\addexamplecitation{Math. Trip. 1899.}

44. If i= P (1+ 2 \ 2 4)- i;

express x in terms of by means of either Jacobian or Weierstrassian
elliptic functions.

\addexamplecitation{Math. Trip. 1914.}

45. Shew that

2' 7r

\addexamplecitation{Trinity, 1881.}

46. When a>x> >y, reduce the integrals

\ \ a-t) t- ) t-y)]- dt, j-" (a-t) t- )(t-y) -idt

by the substitutions

x-y = a- y) dn w, x - y = (p-y) nd- ??

respectively, where k' = a - )/ a - y).

Deduce that, ii u + v = K, then

1-sn u - sn v + k'sn uiin v = 0.

By the substitution y = a - t) t - j3)l t - y) applied to the above
integral taken between the limits j3 and a, obtain the Gaussian form
of Landen's transformation,

I a cos d + bi' Hm-d)~ de= I a- eos 6 + b' sin (9) ~ a dd,

where ai, bi are the arithmetic and geometric means between a and b.

(Gauss, Werke, ill. p. 352; Math. Trip. 1895.)

%
% 534
%

47. Shew that

sc ?f = - k' - 1 C u - A') - C (u - K - 2iK') - C 2iK% where the
Zeta-functions are formed with periods 2a)i, 2co2 = 2K, AiE'.

\addexamplecitation{Math. Trip. 1903.}

48. Shew that E - k" K sati.sfies the equation

where c=lfi, and obtain the primitive of this equation.
\addexamplecitation{Math. Trip. 1911.}

49. Shew that ni k K' dk= n- ) ( /('' - E' dk,

(71 + 2) j t'E'dk= n + l) I k' K'dk. \addexamplecitation{Trinity, 1906.}

50. If u W"" t t) ct) ~ dt,

  J

shew that o(c-l) +(2c-l) + 4. = | 3

51. SHew that the primitive of

du u k \

dk' J' T '

A E-K) + A'E'

\addexamplecitation{Trinity, 1896.}

'""'AE+A'iE'-K'y

where J, A' are constants. \addexamplecitation{Math. Trip. 1906.}

52. Deduce from the addition-formula for E u) that, if

Ui + U2 + U + Ui = 0,

then (sn ?<! sn l<l - sn u sn Ui) sn (uy + %i

is unaltered by any permutation of suffices. \addexamplecitation{Math. Trip. 1910.}

53. Shew that

\addexamplecitation{Math. Trip. 1913.}

54. Shew that

U- i' u cd udu = 2K (2 + P) A'- 2 (1 -l-F) E). [Write = A''+r.] '
\addexamplecitation{Math. Trip. 1904.}

55. By considering the curves ?/2 = (i \ .) (1 \ .2,) y = l- mx +
n.v', shew that, if u 1 + Uo + Us + Ui = 0, then

E %lx)+E U2) + E ll3) + E Ui) = k\ 2 .V + 2CiC2C3C4-2SiS2S3S4-2r.

\addexamplecitation{Math. Trip. 1908.}

56. By the method of example 21, obtain the following seven
expressions for

E ui) + E u2) + E(u3) + E tii) when Ui + U2 + U3 + Ui = 0:

l + k s s Si r=i ' ' " k'-' + d d dsdi r=i '' '' ' k cc c c, - X-'2
,=1 ' ' '' ' k SiS2S3Sidid 2d3di * c Ks d ) - k CiC2 C3Cidi d2d3di * g
u d)

Pkf SiS2S3Si-did2d3di r=l d d2d3di-lrk' CiC2C3Ci r=l

k'SiS2S3Si + CiC2C3Ci *

ClC2C3C4 + A:2siS253S4,.=i

4

- 2;(gj52S3S4)-l + (CiC2C3C4)-l + X' (a?lC/2C 3<: 4)~' ~ 2 ll s,Crdr).

r=l

\addexamplecitation{Forsyth, Messenger, xv.}

%
% 535
%

57. Shew that

when I I x) | <7r/(r); and, by differentiation, deduce that

6 f y ns* [- ") = 6 cosec* x + i (1 + F) ( y - 1 cosec x

TT \ TT /

 -, * f 79S /2/r\ 2 "] o2 COS 2 A'

Shew also that, when | /( ) | < |7r/(r),

n.M + 2

2Kx\ \ - fl-hf-2 \ 2n + lf fjn yi 27rg"- - sin (2ot + 1) j -,!o 1 2P
' 2F - \ 2k) j (1 - ? - 1)

\addexamplecitation{Jacobi.}

58. Shew that, if a be the semi-major axis of an ellipse whose
eccentricity is sin j-Vtt, the perimeter of the ellipse is

\addexamplecitation{Ramanujan, Quarterly Journal, XLV.}

59. Deduce from examjjle 19 of Chapter xxi that

TO o -k" + dn udn3u,,\ k' + k cn u en Su

F cn3 2u = - - T-, - 5 r-, dn 2u = r- 71 - 5 i~

l+k sn usn3u l + Fsn wsnSM

\addexamplecitation{Trinity, 1882.}

60. From the formula sd i.u, k) = i sd (n, k') deduce that

rq '+i . f n + i)7ru\ \ 1 - (-)'*?i' "';. f 7i + i)nu

K'

1 1 ( rv i.br ' i V- i - g ""'

where q = exp ( - ttK'/K), q = exp ( - irKjK'),

and u lies inside the parallelogram whose vertices are

±iK±K'. By integrating from u to K\ from to u and again from ti to K',
prove that

[A formula which may be derived from this by writing 2i = \$ + iri,
where | and rj are real, and equating imaginary parts on either side
of the equation was obtained by Thomson and Tait, Natural Philosopki
ii. (1883), p. 249, but they failed to observe that their formula was
nothing but a consequence of Jacobi's imaginary transformation. The
formula was suggested to Thomson and Tait by the solution of a problem
in the theory of Elasticity.]


\chapter{Ellipsoidal Harmonics and \Lame's Equation}

231. The definition of ellipsoidal harmonics.

It has been seen earlier in this work \hardsectionref{18}{4}) that solutions of
Laplace's equation, which are analytic near the origin and which are
appropriate for the discussion of physical problems connected with a
sphere, may be con- veniently expressed as linear combinations of
functions of the type

cos r' P (cos ), r' Pn'" (cos 6) . md),

where n and m are positive integers (zero included).

When Pn (cos 6) is resolved into a product of factors which are linear
in cos 6 (multiplied by cos 6 when n is odd), we see that, if cos 9 is
replaced by zjr, then the zonal harmonic r"P (cos 6) is expressible as
a product of factors which are linear in a, y and z, the whole being
multiplied by z when n is odd. The tesseral harmonics are similarly
resoluble into factors which are linear in a, y and z- multiplied by
one of the eight products 1, cc, y, z, yz, zx, xy, xyz.

The surfaces on which any given zonal or tesseral harmonic vanishes
are surfaces on which either 6 or (f) has some constant value, so that
they are circular cones or planes, the coordinate planes being
included in certain cases.

When we deal with physical problems connected with ellipsoids, the
structure of spheres, cones and planes associated with polar
coordinates is replaced by a structure of confocal quadrics. The
property of spherical harmonics which has just been explained suggests
the construction of a set of harmonics which shall vanish on certain
members of the confocal system.

Such harmonics are known as ellipsoidal harmonics; they were studied
by Lame* in the early part of the nineteenth century by means of
confocal coordinates. The expressions for ellipsoidal harmonics in
terms of Cartesian coordinates were obtained many years later by W. D.
Niven-f-, and the following account of their construction is based on
his researches.

The fundamental ellipsoid is taken to be

x ir z, a 0 c

and any confocal quadric is

* Journal de Math. iv. (1839), pp. 100-125, 126-163. t Phil. Trans.
182 a (1892), pp. 231-278.

%
% 537
%

where is a constant. It will be necessary to consider sets of such
quadrics, and it conduces to brevity to write

U - I 1 = (h)

The equation of any member of the set is then

Qp = 0.

The analysis is made more definite by taking the a;-axis as the
longest axis of the fundamental ellipsoid and the -axis as the
shortest, so that a>b> c.

23'2. The foiir species of ellipsoidal hat monics.

A consideration of the expressions for spherical harmonics in factors
indicates that there are four possible species of ellipsoidal
harmonics to be investigated. These are included in the scheme

1, y, zx, xyz r% %. ...%,

X,

yz,

2/'

zx,

z,

 y,

where one or other of the expressions in is to multiply the product If
we write for brevity

©,(H),...e, = n((H)),

any harmonic of the form IT (0) will be called an ellipsoidal harmonic
of the first species. A harmonic of any of the three forms* 11 (©),
yYi (©), 11 (©) will be called an ellipsoidal harmonic of the second
species. A harmonic of any of the three forms* yzXl (0), '.rll (®),
xyYi (©) will be called an ellipsoidal harmonic of the third species.
And a harmonic of the form xyzYl (0) will be called an ellipsoidal
harmonic of the fourth species.

The terms of highest degree in these species of harmonics are of
degrees 27n, 2m + 1, 2m + 2, 2m + 3 respectively. It will appear
subsequently \hardsubsectionref{23}{2}{6}) that 2/1 + 1 linearly independent harmonics of
degree n can be constructed, and hence that the terms of degree n in
these harmonics form a fundamental system (§ 18"o) of harmonics of
degree n.

We now proceed to explain in detail how to construct harmonics of the
first species and to give a general account of the construction of
harmonics of the other three species. The reader should have no
difficulty in filling up the lacunae in this account with the aid of
the corresponding analysis given in the case of functions of the first
species.

* The three forms will be distinguished by being described as
different tyj es of the species.

%
% 538
%

23'21. Tlie construction of ellipsoidal harmonics of the first
species.

As a simple case let us first consider the harmonics of the first
species which are of the second degree. Such a hal-monic must be
simply of the form @i.

Now the effect of applying Laplace's operator, namely

9- d 9- x y- z"

9 ' 9p' 9? a d,' ¥'+T, ¥Ver

2 2 2

ci" + e, b'- + 0, c' + e, '

and so @i isa harmonic if 6 is a root of the quadratic equation

(0 + ¥) 6 + cO + ((9 + d") (6 + a') + (f) + a') (6 + b') = 0.

This quadratic has one root between - c and - b~ and another between -
6 and - a-. Its roots are therefore unequal, and, by giving 6 the
value of each root in turn, we obtain two* ellipsoidal harmonics of
the first species of the second degree.

Next consider the general product @i02 ... @; this product will be
denoted by n (0) and it will be supposed that it has no repeated
factors - a supposi- tion which will be justified later \hardsubsectionref{23}{4}{3}).

If we temporarily regard Bj, @, ... (S) j as a set of auxiliary
variables, the ordinary formula of partial differentiation gives

9n (0) \ ' dU ((H)) 90 \ dUi® 2cc dx pZi 9@p dx p i d p ' a- + dp'

and, if we differentiate again,

9 n(@)\ 9n(0) 2 . dm(®) Sx

dx' p=i 9@ ' a'+0p' j,, dBpdB, " (a' + dp) a" + 6 ) '

where the last summation extends over all unequal pairs of the
integers 1, 2, ... m. The terms for which p = q may be omitted because
none of the expressions 0,, @o, ... 0 enters into fl (0) to a degree
higher than the first.

It follows that the result of applying Laplace's operator to 11 (0) is

5 i5II(©) j 2 2 2 )

V S'n (0 ) (\ Sx' 8y' 8f ]

  jZ Wpm, t(a + 0p) (a- + e~,) "*" (6 + 0p) (¥ + dg) (c + Op) (c +
d,)\ '

j\ ow = - 

(.ry,A ci'+0p)(O''+f g) 0g-0p' \ a,b,c )

* The complete set of 5 ellipsoidal harmonics of the second degree is
composed of these two together with the three harmonics yz, zx, xy,
which are of the third species.

%
% 539
%

and 9n (S)/dSp consists of the product 11 (0) with the factor ®p
omitted, while d U (S)/d®pdSq consists of the product IT (0) with the
factors 0 and 0 omitted. That is to say

3 n(0) an(0) a n(0) 8n(©)

If we make these substitutions, we see that

n(0)

9 9- 3'

 o n

may be written in the form

g 9 n (0) f 2 2 2 V' \ \ ?\

the prime indicating that the term for which q = p has to be omitted
from the summation.

If n (0) is to be a harmonic it is annihilated by Laplace's operator;
and it will certainly be so annihilated if it is possible to choose 0
, O., ... 0,n so that each of the equations

a' + e ' b' + ep c'+dp Op-e

is satisfied, where p takes the values 1, 2, ... m.

Now let i9 be a variable and let Aj (6) denote the polynomial of
degree m in 6

m

n d-e,).

q = \

If A/( ) denotes cZAj (6)/cW, then, by direct differentiation, it is
seen that A/ (6) is equal to the sum of all products of - i, 6 - 6.,,
... 6 - 6,ni - l at a time, and A/' (6) is twice the sum of all
products of the same expressions, m - 2 at a time.

Hence, if 6 be given the special value 6p, the quotient A " (6p)IA
(6p) becomes equal to twice the sum of the reciprocals of 6p-6y, 6p -
6.2, ... 6p - 6, (the expression 6p - 6p being omitted).

Consequently the set of equations derived from the hypothesis that

n (0J,) is a harmonic shews that the expression

'' ' 1 1 1 2A/'( J

a' + 6' ¥+6' c'+d A,' (6)

vanishes whenever 6 has any of the special values 6, 60, ... 6j,.

Hence the expression

(a' + 6) (b' + 6) (c + 6) A," 6) + l\ 1 (¥ + 6) (c' + 6),- A/ 6)

" ( a, h, c '

%
% 540
%

is a polynomial in 6 which vanishes when 6 has any of the values 
TODO, and so it has TODO as factors. Now this
polynomial is of degree m + 1 in and the coefficient of '"+ is m (m +
|), Since m of the factors are known, the remaining factor must be of
the form

m (m + ) 6 + IG,

where is a constant which will be determined subsequently.

We have therefore shewn that

( . + 0) f2 + d) (c + 6) A/' 6) + \ \ 1 h - + 6) \& + )l A/ 6)

= |m(m + *)6' + i-C' Ai(6').

That is to say, any ellipsoidal harmonic of the first species of
(even) degree n is expressible in the form

where 0, 6.,, ..., i are the zeros of a pol Tiomial Ai( ) of degree
n; and this polynomial must be a solution of a differential equation
of the type

4V (a + )(6 + )(c + ) |

 \ \ {a + e) ¥ + e) c - e)] '>

= [n n + l)e + C].\, e).

This equation is known as Lame's differential equation. It will be in-
vestigated in considerable detail in §§ 23"4-23*81, and in the course
of the investigation it will be shewn that (I) there are precisely n +
1 different real values of G for which the equation has a solution
which is a polynomial in d of degree \ n, and (II) these polynomials
have no repeated factors.

The analysis of this section may then be reversed step by step to
establish the existence of -J + 1 ellipsoidal harmonics of the first
species of (even) degree n, and the elementary theory of the harmonics
of the first species will then be complete.

The corresponding results for harmonics of the second, third and
fourth species will now be indicated briefly, the notation already
introduced being adhered to so far as possible.

23'22. Ellipsoidal liarmonics of the second species.

in

We take x II (0 ) as a typical harmonic of the second species of
degree

2m + 1. The result of applying Laplace's operator to it is

r: 311(0) I 6 2 2 I

 Ip i dSp \ a' + dp' b' + Op' c' + dp]

   X dSpde ( a'- + dp) (a + 6, ) (6' + dp) (b' + 6, ) " (c= + 0p) (c +
6,)\ ] '

%
% 541
%

and this has to vanish. Consequently, if

3 = 1

we find, by the reasoning of\hardsubsectionref{23}{2}{1}, that A 0) is a solution of the
differential equation

(cc- + d) b'- + e) c' + e)A./'(e)

+ 3 (6 + 0) ic' + e) + (c ~ + d) (a' + 0) + ( + 0) ib' + 0) A/ (0)

= m(m + f) + ia Ao(6'), where Cn is a constant to be determined.

If now we write Ag (0) = A ( )/\/(a- + 0), we find that A (0) is a
solution of the differential equation

4> (a + 0)(h - + 0) c +0)

V((a + )(6- + )(c + )]. '

d0

= (2m + 1) 2ru + 2) + C A 0), where C = 6*2 + 6 + c-.

It will be observed that the last differential equation is of the same
type as the equation derived in\hardsubsectionref{23}{2}{1}, the constant n being still
equal to the degree of the harmonic, which, in the case now under
consideration, is 2m + 1.

Hence the discussion of harmonics of the second species is reduced to
the discussion of solutions of Lame's differential equation. In the
case of harmonics of the first type the solutions are required to be
polynomials in multiplied by \/ a + 0); the corresponding factors for
harmonics of the second and third types are \/(b- -+- 0) and (c" + 0)
respectively. It will be shewn subsequently that precisely m + 1
values of C can be associated with each of the three types, so that,
in all, 8???, + 3 harmonics of the second species of degree 2m 4- 1
are obtained.

23*23. Ellipsoidal harmonics of the third species.

m

We take yz U (@ ) as a typical harmonic of the third species of degree
2m + 2. The result of applying Laplace's operator to it is

r ' 31] (Ch)) [ 2 6 6

 Ip' i % ( ' + 0p h + 0p c'- -0,

  dm (@) f 8 %if 8 n

+ j q d%d% ((a + 0p) a + 0q) (¥ + dp) (6 +,) " (c + 0 ) (c' + 0q) i I
' and this has to vanish. Consequently, if

A3( )= n 0-0,;),

3=1

%
% 542
%

we find, by the reasoning of\hardsubsectionref{23}{2}{1}, that A3 6) is a solution of the
differential equation

(a' + 6) (b' + 6) c' + 6) A," \&)

+ h K '-' + d) (c- + 6') + 3 (c + 6) tr + ) + 3 te + 6) b' + 6)] A,' 0

= [m(vi + )e + iC,]As(e), where Oj is a constant to be determined.

If now we write A3 (6) = A 0)l [ b' + 6) (c-' + 6)], we find that A 6)
is a solution of the differential equation

4 Vl(a- + 0) ¥ + 6) (c + )| ~ s/ ce + 6) ¥ + 6) (c' + 6)

dA(e)'

cie

= (2m -f 2) (2m + 3) + C] A (0),

where C = C3 -F 4 (1 + b- + c-.

It will be observed that the last equation is of the same type as the
equation derived in\hardsubsectionref{23}{2}{1}, the constant n being still equal to the
degree of the harmonic, which, in the case now under consideration, is
2m + 2.

Hence the discussion of harmonics of the third species is reduced to
the discussion of solutions of Lame's differential equation. In the
case of harmonics of the first type, the solutions are required to be
j olynomials in multiplied by VK " + ) ( + )j ' ® corresponding
factors for harmonics of the second and third types are J (c" + 6) (a-
+ 6)] and \ J[ a- + 6) (b- + 6)] respectively. It will be shewn
subsequently that precisely m + 1 values of C can be associated with
each of the three types, so that, in all, Sni + 3 harmonics of the
third species of degree 2m + 2 are obtained.

23'24. Ellipsoidal harmonics of the fourth species.

The harmonic of the fourth species of degree m + 8 is expressible in
the

m

form xyz 11 (@p). The result of applying Laplace's operator to it is

r an 0) [ 6 i>\ \ . \ \ L

"l3i % t ' + P ' ' + ' P c' + p

  d'U (B) ( 8a,- \ %'

",i ae aB, [(a- + e ) a? + e,) " (ft + e j¥ + e ) " (c + e,;) (c +
e,,)\ \ '

and this has to vanish. Consequently, if

Hi

A,( )=ll( -,),

7 = 1

we find by the reasoning of\hardsubsectionref{23}{2}{1} that A4( ) is a solution of the
equation

( + )(6' + )(c- + )a;'( )+ I :s (6 + <9)(c + 6 )|a;( )

= m(m + |) + iC, A,( ), where C4 is a constant to be determined.

 'ce~ + e) h - + e)(c' + 0) ~

%
% 543
%

If now we write

A, 6) = A 0)/ (a + 6) (¥ + d) (c' + 0), we find that A (0) is a
solution of the differential equation

= (2m + 3) 2m + 4) + C A (6),

where C = C + 4< (a- + h- + c'-).

It will be observed that the last equation is of the same type as the
equation derived in\hardsubsectionref{23}{2}{1}, the constant n being still equal to the
degree of the harmonic which, in the case now under consideration, is
2m + 3.

Hence the discussion of harmonics of the fourth species is reduced to
the discussion of solutions of Lame's differential equation. The
solutions are required to be polynomials in 6 multij)lied by \/ a" +
6) b"- + 6) (c + n)]. It will be shewn subsequently that precisely m +
1 values of C can be associated with solutions of this type, so that m
+ 1 harmonics of the fourth species of degree 2rii -f 3 are obtained.

23*25. Nivens expressions fur ellipsoidal harmonics in terms of homo-
geneous harmonics.

If Gn (x, y, z) denotes any of the harmonics of degree n which have
just been tentatively constructed, then Gn ( ", y. z) consists of a
finite number of terms of degrees n, n - 2, w - 4, ... in x, y, z. If
H x, y, z) denotes the aggregate of terms of degree n, it follows from
the homogeneity of Laplace's operator that Hn (, y, z) is itself a
solution of Laplace's equation, and it may obviously be obtained from
Gn x, y, z) by replacing the factors (h), which occur in the
expression of Gn oc, y, z) as a product, by the factors Kp.

It has been shewn by Niven loc. cit., pp. 243-245) that Gn (x, y, z)
may be derived from Hn oc, y, z) by applying to the latter function
the differential operator

2(2?i-l)"*'2.4.(2w-l)(2w-3) 2.4.6(2 1 - l)(27i-3)(27i-5) " " '

where D'- stands for

92 B 92

a - + b -- + c ~ 

da dy dz '

and terms containing powers of D higher than the nth may be omitted
from the operator.

We shall now give a proof of this result for any harmonic of the first
species*.

* The proofs for harmonies of the other three species are left to the
reader as examples. A proof applicable to fuiictious of all four
species has been given by Hobson, Proc. London Math. Soc. XXIV.
(1893), pp. (30-64. In constructing the proof given in the text,
several modifi- cations have been made in Niven's proof.

%
% 544
%

For such harmonics the degree is even and we write

p=i p=\

where Sn, Sn-2, Sn-i, ... are homogeneous functions of degrees n, w
- 2, w - 4, . . ., respectively, and

in Sn = Hn (x, y, Z)=U Kp.

p = l

The function Sn-2r is evidently the sum of the products of K, K2, ...
Ki, taken |- n - ? at a time.

If K, K2, ... /iTi n be regarded as an auxiliary system of variables,
then, by the ordinary formula of partial differentiation

a>s,

doc p=-i dKp dx

in dS 9r

p=i dKp a~ + Bp and, if we differentiate again,

= t - - + t

dx pLx dKp a-'+dp p dKpdKq c(r + dp) a + 6 )'

The terms in d-Sn- rl Kp can be omitted because each of the functions
Kp does not occur in Sn\ 2r to a degree higher than the first.

It follows that

p=i dKp \ a + p 6 + <9p \& + dp\

;, BKpdK, ( a' + Op) (a + 6,) ¥ + Op) b' + 6,) (c + dp) (c + 6,)] It
will now be shewn that the expression on the right is a constant
multiple

OI >Oji\ 2r- 2-

We first observe that

a?x OpKp - dqKq

\ a, b, c)

and that, by the differential equation of\hardsubsectionref{23}{2}{1},

a,T, ( " + 0p a, b, c a- + Op

= 6 - Up 2,,

q = \ tTp - Vq

%
% 545
%

D' n-or

p=i oKp p=i

+ s t e,

dSn-.

, in

bK

1 6p - 0q

pjpq dKpdKq

d"Sn-2r dpKp- UqKq

% - ( q

Now dSn-or/dKp is the sum of the products of the expressions K, Ko, .
. . iTijj (Kp being omitted) taken n - ?' - 1 at a time; and Kqd
Sn-2rldKpdKq consists of those terms of this sum which contain Kg as a
factor.

Hence

C'0,i\ 2 '

dK.,

~ - K,

is equal to the sum of the products of the expressions K, Ko, ...
-K"i; (Kp and Kq both being omitted) taken hn - r-1 at a time; and
therefore, by sym- metry, we have

dKr.

K,

''dKpdKq

so that

dKpdkq " 1 ~dKr

dK,

dSn-

-K

dK.

P dKpdKq'

\ Kq-Kp).

,p w-i q

On substituting by this formula for the second differential
coefficients, it is found that

UpKp- UqKq

hi \ i i

  q = ( p - Gq q = \ \ {Gp- q) Kp- Kq)\

hi

6-8 i'

p i dKp

= 4:11 - I) Z

K

2 = 1 Kp - Kq\

\ Kp-Kq).

p = \ " -Q-p p <i

Now we may write Sn-2r in the form

 n-ir "I" "- p n--2r-'2 "l -' ' Ji- 2r- 2 "T -"- - r/' Jl- 2;'- 4>

where 5io, denotes the sum of the products of the expressions K, K.,
... /fj,, (Kp and Kq both being omitted) taken in at a time; and we
then see that

dSn-

" dK

 Kp-Kq)Sn-,r-2.

Hence

D'-8n-,r = (4n -2) t

p-l cIVp p q

Now it is clear that the expression on the right is a homogeneous sym-
metric function of K, K.2, Kx, of degree n - r-\, and it contains
no power of any of the expressions K, K, ... Ki to a degree higher
than the first. It is therefore a multiple of Sn- r-z- To determine
the multiple we w. M. A. 35

%
% 546
%

observe that when Sn- - is written out at length it contains jt C;.+i
terms while the number of terms in

is hn (4n - 2) ., \,C, - 8 . a . i \,C,-i-

The multiple is consequently

and this is equal to (2r + 2) (2?i - 2r - 1). It has consequently been
proved that

D- Sn-.r = (2r + 2) (2n - 2r - 1) \ . \,. It follows at once by
induction that

 n-2r -

2.4... 2/- . 2n - 1) (271 - 3) ... (2n - 2r + 1)' and the formula

Gn (x, y, z) =

I (-YD"-'-

Hn x, y, z)

,.=o2.4...2r.(2n-l)(2n-:3)...(2n-2r + l) is now obvious when Gn x, y,
z) is an ellipsoidal harmonic of the first species.

Example 1. Prove Niven's fonmila when G (.r, y, z) is an ellipsoidal
harmonic of the second, third or fourth species.

Example 2. Obtain the symbolic formula

G x, y, 2) = r(i- ). (iZ))"+ /\ \ (Z)). (.r, y, z).

2326. Ellipsoidal harmonics of degree n.

The results obtained and stated in §§ 23"21-23"24 shew that when n is
even, there are n + 1 harmonics of the first species and |/i harmonics
of the third species; when 7i is odd there are |(?i + l) harmonics of
the second species and n -1) harmonics of the fourth species, so that,
in either case, there are 2n + 1 harmonics in all. It follows from §
18*3 that, if the terms of degree n in these harmonics are linearly
independent, they form a funda- mental system of harmonics of degree n
; and any homogeneous harmonic of degree n is expressible as a linear
combination of the homogeneous harmonics which are obtained by
selecting the terms of degree n from the 2)i + 1 ellip- soidal
harmonics.

In order to prove the results concerning the number of harmonics of
degree n and to establish their linear independence, it is necessary
to make an intensive study of Lame's equation; but before we pursue
this investigation we shall study the construction of ellipsoidal
harmonics in terms of confocal coordinates.

%
% 547
%

These expressions for ellipsoidal harmonics are of historical
importance in view of Lame's investigations, but the expressions which
have just been obtained by Niven's method are, in some respects, more
suitable for physical applications.

For applications of ellipsoidal harmonics to the investigation of the
Figure of the Earth, and for the reduction of the harmonics to forms
adapted for numerical computation, the reader is referred to the
memoir by G. H. Darwin, Phil. Trans. 197 a (1901), pp. 461-537.

23"3. Confocal coordinates.

If X, Y, Z) denote current coordinates in three-dimensional space, and
if a, b, c are positive (a > 6 > c), the equation

X' Y' Z,

- +7 + = 1 a- 0- c-

represents an ellipsoid; the equation of any confocal quadric is

X2 7-2 2

,+

b' + 0' c + e '

a' + e

and 6 is called the pcuximeter of this quadric.

The quadric passes through a particular point x, y, z) if 6 is chosen
so that

+ -

y-

+

= ].

ar + d ¥ + 6 c- + e Whether 6 satisfies this equation or not, it is
convenient to write

  yl\ z' \ /( )

62 + (9

1-

a--ve ¥ + e c'' + e~ a' + d) (6 + 0) (c- + 6>) '

and, since f d) is a cubic function of 6, it is clear that, in
general, three quadrics of the confocal system pass through any
particular point (x, y, z).

To determine the species of these three quadrics, we construct the
following Table :

e

f \&)

- 00

- 00

-d

-.r2(a2-62)(a2.

-C2)

-62

y2(a2-62)(52.

-C2)

- C

-22(a2-c2) (62-

-C2)

+ 00

+ 00

It is evident from this Table that the equation y*( ) = has three real
roots X,, /i, V, and if they are arranged so that ix>v, then

 -C'>fi> - b->v > - a'-; and also /( ) = ( \ X)( - )((9 - i').

From the values of X, /x, v it is clear that the surfaces, on which
has the respective values X, /z, v, are an ellipsoid, an hyperboloid
of one sheet and an hyperboloid of two sheets.

35-2

%
% 548
%

Now take the identity in 6,

\ a? \ j/ \ 2' e- ) e-ii ) e-v)

ar- e b-'+tl C--+6 a' + e) b"- + d) c- + e)' and multiply it, in turn,
by a- + 6, h- + 0, c- + 6; and after so doing, replace 6 by - (('-, -
h-, - c- respectively. It is thus found that

, \ (a- + ) (a + fj.) (a- + v)

, \ (6 + X) (6 + 6- + lO ' '(a--6-)( '-c-) '

  \ (c- X ) (c- + fj,)(c- + v) '~ (a -c2)(6--c=) *

From these equations it is clear that, if (a-, y, z) be any point of
space and if X, /Lt, V denote the parameters of the quadrics confocal
with

X'- Y-' Z-,

- + 1T + - = 1 a- 0 c-

which pass through the point, then (x-, y", z-) are uniquely
determinate in terms of (X, fj,, v) and vice versa.

The parameters (X, fi, v) are called the confocal coordinates of the
point x, y, z) relative to the fundamental ellipsoid

X' Y- Z'-,

a- b- c- It is easy to shew that confocal coordinates form an
orthogonal system; for consider the direction cosines of the tangent
to the curve of intersection of the surfaces (/a) and (v); these
direction cosines are proportional to

/dx dy dz \ d dX' dx

and smce - -- + 4.\ -- = % \ - -\ - \ =

dx dx dy dy dz dz \ a + v

dXdJi Xd l' dXdJjL aXc (a -6')(a2- c ) it is evident that the
directions

fdx dy dz\ /dx dy dz\

Vax' dx' dx)' \ d ' Yfi' dfx)

are perpendicular; and, similarly, each of these directions is
perpendicular to

/dx dy dz\

\ dv ' dv ' dv) ' It has therefore been shewn that the three systems
of surfaces, on which X, jji, V respectively are constant, form a
triply orthogonal system. Hence the square of the line-element, namely

 hxy- hjr + hz)\

is expressible in the form

%
% 549
%

with similar expressions in /j, and v for H. and H3'. To evaluate Hi-
in terras of (X, u, y), observe that

' " 4 va y " 4 / Ux-y 4 \ dx

 x s; (g + fx) a? + i )

But, if we express

(A, - /Lt) (X, - v)

(a + X) (6 + X) (c" + X) '

qua function of X, as a sum of partial fractions, we see that it is
precisely equal to

  (g + fi) (g" + v)

 .5,c(g' + X)(g2-6 )(a -cO'.

and consequently H - = . -r . il - z-A c 

  ' 4 (a2 + X) (62 + X) (c + X)

The values of H. and Hf are obtained from this expression by cyclical
interchanges of (X, /i, i').

Formulae equivalent to those of this section were obtained by Lame,
Journal de Math. II. (1837), pp. 147-183.

Example 1. With the notation of this section, shew that

.r2+/ + 5- = a2 + 62 + c2 + X + jtx + i'.

Example 2. Shew that

ATTI- - I y" I

   a;' + \ f h + \ f c'' + \ f'

23 '31. Uniformising variables associated ivith confocal coordinates.

It has been seen in\hardsectionref{23}{3} that when the Cartesian coordinates (x, y,
z) are expressed in terms of the confocal coordinates (X, fx, v), the
expressions so obtained are not one-valued functions of (X, /x, v). To
avoid the inconvenience thereby produced, we express (X, /i, v) in
terms of three new variables u, v, w) respectively by writing

HP (u) = X + |(a- + 6 + C-),

 j(v) =/i +!( ' + ' + c'),

'i'w) = v \ \ {d + h'"' + c ),

the invariants g.2 and g of the Weierstrassian elliptic functions
being defined by the identity

4>(a-+X) b' + ) c'+X) f u)-g u)-g,.

%
% 550
%

The discriminant associated with the L-lHptic functions (cf.\hardsubsectionref{20}{3}{3},

example 3) is

16 a- - b-Y b"- - c'Y- (c- - a'Y,

and so it is positive; and, therefore*, of the periods 2a),, 2\&)o
and 2\&)3, 2g)i is positive while 2a)3 is a pure imaginary; and 2a)2
has its real part negative, since Wj + \&).,+ 0)3 =; the imaginary
part of Wo is positive since / ((o.J(o ) > 0.

In these circumstances e, >e.2>e:;, and so we have

3e, = a- + b-- '2c-, Se = c" + a- - 2b-, 8e, = b- + c- - 2a-.

Next we express (x, y, z) in terms of u, v, w); we have
TODO
by\hardsubsectionref{20}{5}{3}, example 4. Therefore, by\hardsubsubsectionref{20}{4}{2}{1}, we have

x.- -. "/ 0-3 (") 0-3 (t') o-s (w) o- (w) o- (t;) o- w)

a-2 (w) (7o (v) cTa (w) and similarly 3/ = ± e- -a-(a,,) (,,) - ( ) )

X -T,co "/ o l(")o-l(y)o l(w) \ 4. g ').'"i cr- ( w ) - -- - --- - - .
- a u) a- (v) a (w)

The effect of increasing each of u, v, w by 2(0 is to change the sign
of the expression given for x while the expressions for y and z remain
unaltered; and simila'r statements hold for increases by 2\&).\ . and
2\&)i; and again each of the three expressions is changed in sign by
changing the signs of u, v, iv.

Hence, if the upper signs be taken in the ambiguities, there is a
unique correspondence between all sets of values of x, y, z), real or
complex, and all the sets of values of (it, v, w) whose three
representative points lie in any given cell.

The uniformisation is consequently effected by taking

., ., o-:, u) 0-3 v) 0-3 (w) cr ii) a v) a (w)

-.u. 0/, o-2 (t/-)o-.,(t;)o-,(w) ' a- a) a v) cr w)

(t u) (T v)cr w)

Formulae which differ from these only by the interchange of the
suffixes- 1 and 3 were given by Halphen, Fonctions Elliptiques, li.
(1888), p. 459.

 Cf.\hardsubsectionref{20}{3}{2}, example 1.

23-32]

ELLIPSOIDAL HARMONICS

551

23'32. Laplace's equation referred to confocal coordinates.

It has been shewn by Lame and by W. Thomson* that Laplace's equation
when referred to any system of orthogonal coordinates (X, /j., v)
assumes the form

0,

H, dv where H, H.2, H- are to be determined from the consideration
that

is to be the square of the line-element. Although W. Thomson's proof
of this result, based on arguments of a physical character, is
extremely simple, all the analytical proofs are extremely long and
cumbrous.

It has, however, been shewn by Lamef that, in the special -case in
which (X, /JL, v) represent confocal coordinates, Laplace's equation
assumes a simple form obtainable without elaborate analysis; when the
uniformising variables (u, V, w) of\hardsubsectionref{23}{3}{1} are adopted as
coordinates, the form of Laplace's equation becomes still simpler.

By straightforward differentiation it may be proved that, when any
three independent functions (X, /n, v) of x, y, z) are taken as
independent variables, then

a F a F a F

transforms into

dx- dy

+

dz-"-

t + 2 S

yax

\ dx

J<1

+

dz. d/Ji dv

dfx dv d/ji dv

dx dx dy dy dz dzj d/jidv

d \ d \ d'\

d\ V dx'

A.M. V \ \ x- dy- dz \ In order to reduce this expression, we 'observe
that X satisfies the equation

X- y z-

oMO. "*" PTX " cM

= 1,

and so, by differentiation with x, y, z as independent variables, 2x x
y  1 ' \ A

a- +X

14a;

,+

+

(a + xy (6- + xy (c + x)- j dx

dx

+ 2

a + X (a + X)- dx a + X)

+

y.

X-

(6 + X)=

y

+

+

(c + X)

+

(a2 + X)- h- + \ y (c + X)*- ] dx

d'X

= 0.

* Cf. the footnote on p. 401.

t Journal de Math. iv. (1839), pp. 133-136.

%
% 552
%

Hence ~- =4<Hi,

a.' + \ da;

a' + X (a-' + \ y //j- 2 1 a' + xy /,7, v (a- + Xf ' d.v' '

\ a, b, c)

with similar equations in fx, v and y, z.

d-V . From equations of the first type it is seen that the coefficient
of - - is

1 d-V . . . .

V5r- and the coefficient of tt-t is zero; and if we add up equations
of the

second type obtained by interchanging x, y, z cyclically, it is found
that

[dx- dy- 0Z-] o, i.c " + with similar equations in [x and v. If, for
brevity, we write

 /[ a + X) h- + ) c' -X)] = K, ' with similar meanings for A and A,
we see that

a;- " a " a - (x - /i) (x - v) 1 a- + x 6- +'x " c + X

4Ax c Ax

(X - /x) (X - I/) rfX ' and so Laplace's equation assumes the form

x.T", >' (' - A*) ( ~ ) L ' dx ax

that is to say

= 0,

(' - ) a-xWI ]+< - ) 'IKa + -'') ' .K'5 = -

The equivalent equation with u, v, lu) as independent variables is
simply or, more briefly,

[i j v) - i w)] 1 + w) - in)] + [iO u) - v)] = 0,

 - >a + < - >-a + - >a = '-

The last three equations will be regarded as canonical forms of
Laplace's equation in the subsequent analysis.

23*33. Ellipsoidal harmonics referred to confocal coordinates. When
Niven's function 0p, defined as

a- + 6, J b'' + 6,, c- +

p p p

%
% 553
%

is expressed in terms of the confocal coordinates \ [x, v) of the
point x, y, z), it assumes the form

  e ) t,-ej) v-dp )

and consequently, when constant factors of the form

- ( + e,,) (¥ + dp) (c + Op)

are omitted, ellij)soidal harmonics assume the form X, yz \

I m m m

1, y, zx, xyz I n (X - ) H (/i-6'p) H v-B ).

I p = l 73 = 1 p = \

z, xy ]

If now we replace x, y, z b ' their values in terms of X, /*, v, w e
see that any ellipsoidal harmonic is expressible in the form of a
constant multiple of AMN, where A is a function of \ only, and M and N
are the same functions of /Li, and V respectively as A is of X.
Further A is a polynomial of degree nn in \ multiplied, in the case of
harmonics of the second, third or fourth species, by one, two or three
of the expressions \/ a' + X), \/ b- + X), \/(c + )-

m

Since the polynomial involved in A is H (X - 6p), it follows from a
con-

i3 = l

4 si [ a- + X) ¥ + X ) (c + X)

V((a- + X)(6 + X)(c + X)

sideration of §§ 23'21-23-24 that A is a solution of Lame's
differential equation

X

= [n n 4-l)X+C' A, where n is the degree of the harmonic in x, y, z).

This result may also be attained from a consideration of solutions of
Laplace's equation which are of the type*

V= AMN, where A, M, N are functions only of X, /x, v respectively.

For if we substitute this expression in Laplace's equation, as
transformed in\hardsubsectionref{23}{3}{2}, on division by V, we find that

  v)-io w) d K ( w) - J u) d'M 0 (u) - ipO ) d'N A dii M dv N " div-

The last two terms, qua functions of u, are linear functions of ip
(u), and

1 d A

so -r- -y 2 must be a linear function of o(u); since it is independent
of the

coordinates v and w, we have

where K and B are constants.

* A harmonic which is the product of three functions, each of which
depends on one coordi- nate only, is sometimes called a normal
solution of Laplace's equation. Thus normal solutions with polar
coordinates are \hardsubsectionref{18}{3}{1}|

r" P ' (cos d) ° m(p. " ' sin

%
% 554
%

If we make this substitution in the differential equation, we get a
linear function of (,) (k) equated (identically) to zero, and so the
coefficients in this linear function must vanish; that is to say

m,. . s) 1 d-M 1 rf'N

and on solving these with the observation that (v) - j J (lu) is not
identically zero, we obtain the three equations

When X. is taken as independent variable, the first equation becomes

4 A. |a. I = [K\ - i? + i/i (a + ¥ + c )] A,

and this is the equation already obtained for A, the degree n of the
harmonic being given by the formula

n (n + l) = K.

We have now progressed so far with the study of ellipsoidal harmonics
as is convenient without making use of properties of Lame's equation.

We now proceed to the detailed consideration of this equation.

234. Various forms of Lame s differential equation.

We have already encountered two forms of Lame's equation, namely

and this may also be written

d \ a" + \ h- + X c- +X d\ ~ 4 (a- + X) b- + X) (c- + X) ' which may
be termed the algebraic form; and

  = n n + l) j(ii) + B]A,

which, since it contains the Weierstrassian elliptic function (u), may
be termed the Weierstrassian form; the constants B and C are
connected by the relation

B + in ( + 1) (a- + b- + c-) = C.

%
% 555
%

If we take j (u) as a new variable, which will be called, we obtain
the slightly modified algebraic form (c£\hardsectionref{10}{6})

This differential equation has singularities at gj, eo, e at which the
exponents are 0, in each case; and a singularity at infinity, at
which the exponents are - n,\ \ {n + ) .

The "Weierstrassian form of the equation has been studied by Halphen,
Fonctions Elliptiques, ir. (Paris, 1888), pp. 457-531.

The algebraic forms have been studied by Stieltjes, Acta Math. vi.
(1885), y>V. 321-326, Klein, Vorlesimgen iiber lineare
Diferentielgleichunge/i lithogr\&iAiGd, Gottingen, 1894), and Bocher,
l/ber die Reihenentioickelungen der Potentialtheorie (Leipzig, 1894).

The more general differential equation with foiu* arbitrary
singularities at which the exponents are arbitrary (save that the sum
of all the exponents at all the singularities is 2) has been discussed
by Heun, Math. Ann. xsxiii. (1889), pp. 161-179; the gain in
generality by taking the singularities arbitrary is only apparent,
because by a homographic change of the independent variable one of
them can be transferred to the point at infinity, and then a change of
origin is sufficient to make the sum of the complex coordinates of the
three finite singularities equal to zero.

Another important form of Lame's equation is obtained by using the
notation of Jacobian elliptic functions; if we write

Zl = U V(ei - 63),

the Weierstrassian form becomes

dzi"

n n + 1) I - ' 1 1 -

4-ns2 iV +

€ > e,-e

A,

and putting 2 = a - iK', w here 2iK' is the imaginary period of sn z-
, we obtain the simple form

- i- = \ n (n + 1) k- sn-a + A] A, doP / >

where A is a constant connected with B by the relation

B + e-iii (n + 1) = A e,- e-,).

The Jacobian form has been studied by Hermits, Sur quelques
applications des fonctions elliptiques, Comptes Rendns, Lxxxv. (1877),
published separately, Paris, 1885.

In studying the properties of Lame's equation, it is best not to use
one form only, but to take the form best fitted for the purpose in
hand. For practical applications the Jacobian form, leading to the
Theta functions, is the most suitable. For obtaining the properties of
the solutions of the equation, the best form to use is, in general,
the second algebraic form, though in some problems analysis is simpler
with the Weierstrassian form.

%
% 556
%

23'41. Solittioiis in series of Lame s equation.

Let us now assume a solution of Lamp's equation, which may be written

in the form

A= I 6.(1-,) "-''.

)-=0

The series on the right, if it is a solution, will converge \hardsubsectionref{10}{3}{1})
for sufficiently small values of \ \ - e.,, but our object will be
not the discussion of the convergence but the choice of B in such a w
ay that the series may terminate, so that considerations of
convergence will be superfluous.

The result of substituting this series for A on the left-hand side of
the differential equation and arranging the result in powers of - go
is minus the series

4 i ( -e.,) ' -r + i[r(n-r-[-hJbr- Se,( n-r + ir--in n+l)e,-lB]br-

r=--0

+ (e, - e,) (e, - e,) ( w - r + 2) (i n - r + f ) 6,-2],

in which the coefficients hj. with negative suffixes are to be taken
to be zero.

Hence, if the series is to be a solution, the relation connecting
successive coefficients is

r (n -r + )br = [Se ( n - r + 1)- - i ?i (n +l)eo-lB\ br-i

- (ei - e ) ( 2 - 63) i n - r + 2) n - r + f) br-.,

and (n - i) 61 = n% -ln n + l)e,- B b,.

If we take \& = 1, as we may do without loss of generality, the
coefficients bj. are seen to be functions of B with the following
properties :

(i) bj. is a polynomial in B of degree r.

(ii) The sign of the coefficient of B'' in br is that of (-)'",
provided that ?' :\$ n; the actual coefficient of B is

i-Y

2.4>...2r(2n-l) 2n-3)... ( 2n - 2r + 1) '

(iii) If 1, e., e-i and B are real and i > 62 > 3, then, if b -i = 0,
the values of 6,. and 6,.\ o are opposite in sign, provided that r < n
+ 'A) and r < n.

Now suppose that n is even and that we choose B in such a way that If
this choice is made, the recurrence formula shews that

hn 2 = 0,

%
% 557
%

by putting r = 7i + 2 in the formula in question; and if both 6i,j\ i
and 1)1 + 2 6 subsequent recurrence formulae are satisfied hy taking

 1/1 + 3 = i + 4 = ... =0.

Hence the condition that Lame's equation should have a solution which
is a polynomial in is that B should be a root of a certain algebraic
equation of degree n + 1, when n is even.

When n is odd, Ave take 6x j i\ to vanish and then 6i/ 3x also
vanishes, and so do the subsequent coefficients; so that the
condition, when n is odd, is that B should be a root of a certain
algebraic equation of degree \ (n + 1).

It is easy to shew that, when e > e., > e-, these algebraic equations
have all their roots real. For the properties (ii) and (iii) shew
that, qua functions of B, the expressions h, b, b, ... b,. form a
set of Sturm's functions* when r < (n + 3), and so the equation

has all its roots realf and unequal.

Hence, when the constants gj, e,, es are real (which is the case of
practical importance, as was seen in\hardsubsectionref{23}{3}{1}), there are ?i + 1 real
and distinct values of B for which Lame's equation has a solution of
the type

tbr( -e -'

Avhen n is even; and there are (n+1) real and distinct values of B for
which Lame's equation has a solution of the type

when n is odd.

When the constants ej, eo, e are not all real, it is possible for the
equation satisfied by B to have equal roots; the solutions of Lame's
equation in such cases have been discussed by Cohn in a Konigsberg
dissertation (1888).

Example 1. Discuss solutions of Lame's equation of the types \ (i) ( -
iF i h; -e. -'--, ' .

r=0

*?l-r-J

(ii) ( -e,) 2 V'd-eo) (iii) ( -e,) -e,)i I b;" -e,) "-'-\

* Mem. pr senUs par les Savans Etrangers, vi. (1835), pp. 271-318.

t This procedure is due to Liouville, Journal de Math. xi. (1846), p.
221.

%
% 558
%

ohtainiug the recurrence relations

(i) r n -r + h) W =- Se.-, hi - r + hf + ie -e ) h u - r + |) - n (n +
1) 63 - i 6',.-i

- (ei - 62) ( 2 - 63) hi -r + %) hi -r + l) 6V-2, (ii) r n-r + h)b," =
3e., hn-r+hf- ei-e2) in-r + )- 7i 7i+l)e.,-iB\ b"r\,

- ei- 62) 62-63) n - r+ ) hi-r + l) b",.\ .i, (iii) r n-r + h)b;" =
3e2 n-r + h)--ke2 n' + n + l)- B b"'r\ i

- ( 1 - eo) (62 - e-i) ih)i-r+l) l n -r + h) b". \ 2 . Example 2. With
the notation of example 1 shew that the numbers of real distinct

values of B for which Lamp's equation is satisfied by terminating
series of the several species are

(i) \ \ {n- ) or \ \ {n-2); (ii) \ \ {n- ) ov h n-9.); (iii) n-2) or h
n-Z).

2342. The definition of Lame functions.

When we collect the results which have been obtained in\hardsubsectionref{23}{4}{1}, it is
clear that, given the equation

n being a positive integer, there are 2n + 1 values of B for which the
equation has a solution of one or other of the four species described
in §§ 23"21-23'24.

If, when such a solution is expanded in descending powers of, the
coefficient of the leading term | " is taken to be unity, as was done
in § 2341, the function so obtained is called a Lame function of
degree n, of the first kind, of the first (second, third or fourth)
species. The 2n + 1 functions so obtained are denoted by the symbol

E, - ); (m = l, 2, ...2 + l).

and, when we have to deal with only one such function, it may be
denoted by the symbol

Tables of the expressions representing Lame functions for ?i = l, 2,
...10 have been compiled by Guerritore, Giornale di Mat. (2) xvi.
(1909), pp. 164-172.

Example 1. Obtain the five Lame functions of degree 2, namely

V(X + i-)v/(X + c2), v/(X + c2) /(X + a2), v'(X + -)V(X + n Example 2.
Obtain the seven Lame functions of degree 3, namely - V (X + a' )(X +
\&'')(X4-c2), and six functions obtained by interchanges of a, b, c
in the expressions

v'(X + a-') . [X + 1 (a2 + 262 + .2c') ± I J a* + ib* + 4c* - 76V-' -
f' a \ a'b ].

23'43. 27te non-repetition of factors in Lame functions.

It will now be shewn that all the rational linear factors of "' (|)
are unequal. This result follows most simply from the differential
equation which - n™ (?) satisfies; for, if | - ?i be any factor of
'"(|), where |i is not one of

%
% 559
%

the numbers gj, eo or 63, then fj is a regular point of the equation
(§ 10"H), and any solution of the equation which, when expanded in
powers of f - 1, does not begin with a term in ( - 1)° or ( - Y must
be identically zero.

Again, if |i were one of the numbers gj, 63 or s. the indicial
equation appropriate to 1 would have the roots and J, and so the
expansion of En"' ( ) in ascending powers of 1 would begin with a term
in (f - j)" or

Hence, in no circumstances has,i'"( ), g'wa function of, a repeated
factor.

The determination of the numbers 0, 6.2, ... 0 introduced in §§
23-21- 23*24 may now be regarded as complete; for it has been seen
that solutions of Lame's equation can be constructed with non-repeated
factors, and the values of 0, 62, ... which correspond to the roots
of En ( ) = satisfy the equations which are requisite to ensure that
Niven's products are solutions of Laplace's equation.

It still remains to be shewn that the '2n + 1 ellipsoidal harmonics
con- structed in this way form a fundamental system of solutions of
degree 71 of Laplace's equation.

23'44. The linear independence of Lame functions.

It will now be shewn that the 2n -i- 1 Lame functions E)/ ) which are
of degree n are linearly independent, that is to say that no linear
relation can exist which connects them identically for general values
of .

In the first j)lace, if such a linear relation existed in which
functions of different species were involved, it is obvious that by
suitable changes of signs of the radicals \/( - i), \/( - 2), \/(f -
s) we could obtain other relations which, on being combined by
addition or subtraction with the original relation, would give rise to
two (or more) linear relations each of which involved functions
restricted not merely to be of the same species but also of the same
type.

Let one of these latter relations, if it exists, be

and let this relation involve r of the functions.

Operate on this identity 7 - 1 times with the operator

The results of the successive operations are

ta,, Bn'-y Err )=0 (5 = 1, 2, ... r - 1), where Bn' is the particular
value of jB which is associated with E "- ).

%
% 560
%

Eliminate a-, iu, ... a,, from the r equations now obtained; and it
is found

that

1, 1, 1, ... 1 =0.

! R 1 R i R 3 Tl r

 Dii, J->n, -'-'71 >  -'-'(!

 B, y-\ B,;r-\ (B/y-'

Now the only factors of the determinant on the left are differences of
the numbers 5 '", and these differences cannot vanish, by\hardsubsectionref{23}{4}{1}.
Hence the determinant cannot vanish and so the postulated relation
does not exist.

The linear independence of the 2m +1 Lame functions of degree n is
therefore established.

2345. The linear independence of ellipsoidal harmonics.

Let Gn', y, z) be the ellipsoidal harmonic of degree n associated
with E, ( ), and let T,/" (cc, y, z) be the corresponding homogeneous
harmonic.

It is now easy to shew that not only are the 2?? + 1 harmonics of the
type Gn" (, y, z) linearly independent, but also the 2?; + 1
harmonics of the type Hn'' x, y, z) are linearly independent.

In the first place, if a linear relation existed between harmonics of
the type Gn'" oc, y, z). then, when we expressed these harmonics in
terms of con- focal coordinates X, /x, v), we should obtain a linear
relation between Lame functions of the type En' i ) where =X+ a- + b-+
c"), and it has been seen that no such relation exists.

Again, if a linear relation existed between homogeneous harmonics of
the type Hn ' x, y, z), by operating on the relation with Niven's
operator

\hardsubsectionref{23}{2}{5}),

  D\

2 (2n - T) " 2.4 (2 - 1 ) 2n- 3) ' " '

we should obtain a linear relation connecting functions of the type
(z,/'' x, y, z), and since it has just been seen that no such relation
exists, it follows that the homogeneous harmonics of degree n are
linearly independent.

2346. Stieltjes' theorem on the zeros of Lame functions.

It has been seen that any Lam6 function of degree n is expressible in
the form

m

(6 + a'Y' (d + h- r (0 + cy .u d- dp),

;> = 1

where i, k, k- are equal to or and the numbers 6, 6., ... 6, are
real and unequal both to each other and to - a, - b, - c-; and n =
ni + Ki + Kn + k . When /cj, /Co, Ks are given the number of Lame
functions of this degi-ee and type is ru + 1.

%
% 561
%

The remai'kable result has been proved by Stieltjes* that these w + 1
functions can be arranged in order in such a way that the rth function
of the set has r - 1 of its zeros f between - a and - h- and the
remaining m - r + 1 of its eros between - 6- and - c and,
incidentally, that, for all the 7?i + 1 functions, 6-, 6.2, ... 0 lie
between - a- and - c-.

To prove this result, let 0j, 2;  4 m be any real variables such
that

(- a- (jip - h, (p - 1,2,...') - 1)

[-h" ( )p - c (p = r,r+l,... m) and consider the product

 ' J- 4-1 a.1

11 = n [| ((/>p +aO T'+i . i (</),. + h"-) r -'i . j (< + c ) i;3+4] n
] (c/> - <,) |.

p=l pdpq

This product is zero when all the variables 0 have their least values
and also when all have their greatest values; when the variables p
are unequal both to each other and to - a-, - If, - c", then IT is
positive and it is obviously a continuous bounded function of the
variables.

Hence there is a set of values of the variables for which 11 attains
its upper bound, which is positive and not zero (cf\hardsubsectionref{3}{6}{2}).

For this set of values of the variables the conditions for a maximum
give

eiogn \ a log n \

that is to say

111

fCi + -r Ko + T ' 3 + T

4 \ 4 4 ', 1 \

<f)p + a <f>p + 6 cf)p + c- fjli <j>p - (j),i

where p assumes in turn the values 1, 2, ... m.

Now this system of equations is precisely the system by which 6, 6,
... 6 are determined (cf \hardsubsectionref{23}{2}{1}-23'24); and so the system of
equations determining Oi, 02, ... dm has a solution for which

j-a'<dp<-b% (p = l,2, ...r-1)

 - b"< dp < - C-. (jo = ?',?+ 1, ... to)

Hence, if r has any of the values 1, 2, ... m + 1, a Lame function
exists with r - 1 of its zeros between - a and - h" and the remaining
vi - ? + 1 zeros between - h- and - c'-.

Since there are m + 1 Lame functions of the specified type, they are
all obtained when r is given in turn the values 1, 2, ... to + 1; and
this is the theorem due to Stieltjes.

* Acta Mathematica, \ i. (1885), pp. 321-326.

t The zeros -a', -b, - c are to be omitted from this enumeration, 6
, d., ..., only being taken into account.

W. M. A. 36

%
% 562
%

An interesting statical interpretation of the theorem was given by
Stieltjcs, namely that if wi +3 particles which attract one another
according to the law of the inverse distance

are placed on a line, and three of these particles, whose masses are
Ki + -r, f2+ > ''s + ti '

fixed at points with coordinates -a-, -b-, -c, the remainder being of
unit mass and free to move on the line, then log n is the
gravitational potential of the system; and the positions of
equilibrium of the system are those in which the coordinates of the
moveable particles are 61,6-2, ... 6, i-e. the values of 6 for which
a certain one of the Lame functions of degree 2 m + ki + K2 + <z)
vanishes.

Example. Discuss the positions of the zeros of polynomials which
satisfy an equation of the type

d6' "* rti e-a, d6 * ',, ~ '

n 6-a,)

s = l

where (f)r-2 6) is a polynomial of degree r - 2 in 6 in which the
coefficient of 6 ~- is

r

- m m + r - 1 - 2 a, .-=1

m being a positive integer, and the remaining coefficients in <f>r-2
6) are determined from

the consideration that the equation has a polynomial solution.

\addexamplecitation{Stieltjes.}

23'47. Lame functions of the second kind.

The functions En ( ), hitherto discussed, are known as Lame functions
of the first kind. It is easy to verify that an independent solution
of Lame's equation

du- '

is the function i,i'" ( ) defined by the equation*

/ .. (f) = (2 H-l) '"(f)/;jj;,

and i,i'" (1 ) is termed a Lam6 function of the second kind. From
this formula it is clear that, near u = 0,

F ( ) = (271 + 1) u- 1 + (lOl I " u-'' 1+0 u)] du = u''+' 1 + (u)],

Jo and we obviously have

E,r ) = u-[l + 0 u)].

It is clear from these results that Fn" (|) can never be a Lame
function of the first kind, and so there is no value of Bn" for luhich
Lame's equation is satisfied by two Lame functions of the first kind
of different species or types.

It is possible to obtain an expression for Fn'\ \ ) which is free from
quadratures, analogous to Christoffel's formula for Qni ), given on p.
333, example 29. We shall give the analysis in the case when En" (|)
is of the first species. The only irreducible poles of l/,j"'( )j qua
function of u, are at a set of points u, u.,, ... Un which are none
of them periods or half periods.

* This definitiou of the function i<' '" (4) is due to Heine, Journal
fUr Math. xxix. (1845), p. 194.

%
% 563
%

Near any one of these points we have an expansion of the form

En'" (I) = /.-i U - Ur) + h (W - Urf + '3 u - U,) + ...,

and, by substitution of this series in the differential equation, it
is found that 2 is zero.

Hence the principal part of l/ £'n"* ) near iir is

1

k]- (u - Uff ' and the residue is zero.

Hence we can find constants Ar such that

 E,r( yr- I Ario(u-u,.)

r = l

has no poles at any points congruent to any of the points iif; it is
therefore a constant A, by Liouville's theorem, since it is a doubly
periodic function of M.

fit dn n

Now the points Uf can be grouped in pairs whose sum is zero, since Ey
(I) is an even function of u.

If we take iin-,- = - '';+i, we have

--~--: = Au- 2 Ar u-Ur) + K u + U>)] Jo i n g)i r = l

= AlC - 2 u) t Ar-t - ff' ''],

r=l r = lip y-)- Ur)

and therefore

? = !

where '?< i,j\ i(| ) is a polj nomial in | of degree n - 1.

Example. Obtain formulae analogous to this expression for i, ™ ) when
E, ( ) is of the second, third or fourth species.

23'5. Lame's equation in association tvith Jacobian elliptic
functions.

All the results \ Yhich have so far been obtained in connexion with
Lame functions of course have their analogues in the notation of
Jacobian elliptic functions, and, in the hands of Hermite (cf.\hardsectionref{23}{7}
1), the use of Jacobian elliptic functions in the discussion of
generalisations of Lame's equation has produced extremely interesting
results.

Unfortunately it is not possible to use Jacobian elliptic functions in
which all the variables involved are real, without a loss of symmetry.

36-2

564

THE TRANSCENDENTAL FUNCTIONS [CHAP. XXIII

and then the formulae of

X =

The symmetrical formulae may be obtained by taking new variables a,
/3, . 7 defined by the equations

fa = iK' + u V(ei - e- ),

1 7 = iK' + w V(ei - e-,), 23"31 are equivalent to A;- \/ a~ - c") .
sn a sn /3 sn 7, y - - (f -/k') V(ft" - C-) . en a en en 7, \ z =
ijk') \/(a- - C-) . dn a dn /8 dn 7, the modulus of the elliptic
functions being

V W-cV-

The equation of the quadric of the confocal system on which a is con-
stant is

X' Y Z-

(a2-62)sn2a (a - b') cn a ~ (a - c-) dn" a ~ " This is an ellipsoid if
a lies between iK' and K + iK'; the quadric on which yS is constant
is an hyperboloid of one sheet if lies between K + iK' and -fir; and
the quadric on which 7 is constant is an hyperboloid of two sheets if
7 lies between and K; and with this determination of (a,, 7) the
point x, y, z) lies in the positive octant.

It has already been seen \hardsectionref{23}{4}) that, with this notation, Lame's
equation assumes the form

--T-v = n + 1 ) A,'- sn- a + A] A,

and the solutions expressible as periodic functions of a will be
called* -£ '" (a). The first species of Lame' function is then a
polynomial in sn a, and generally the species may be defined by a
scheme analogous to that of\hardsectionref{23}{2}, sn a, en a dn a,

1, en or, dnasna, sn a en a dn a dn a, sn a en a,,

n (sn- oc - sn-Qp). J

23'6. The integral equation satisfied by Lame functions of the first
and second species' .

We shall now shew that, if En' (a) is any Lamd function of the first
species (n being even) or of the second species (n being odd) with sn
a as a

* There is no risk of confusing these with the corresponding functions
i-'/' (t).

t This integral equation and the corresponding formulae of § 28 -62
associated with ellipsoidal harmonics were given by Whittaker, Froc.
London Math.- Soc. (2) xiv. (1915), pp. 260-268. Proofs of the
formulae involving functions of the third and fourth species have not
been previously published.

%
% 565
%

factor, then E '"'(a) is a solution of the integral equation

E,r ( ) = X Pn (k sn a sn 6) E,r (0) dd;

J -2K

where \ is one of the 'characteristic numbers' \hardsubsectionref{11}{2}{3}).

To establish this result we need the lemma that P,i (A: sn a sn ) is
annihilated by the partial differential operator

3 2 - 9 - " ( + 1) '' (sn' a - sn d).

To prove the lemma, observe that, when /x is written for brevity in
place of k sn a sn 0, we have

 a|- i ' '-'""'' >

= k- (en- a dn- a sn- 6 - en- 6 dn- 6 sn- a] Pn" (/i)

+ k" sn a sn (sn a - sn 0) P,,' (/x) = k' (sn a - sn 6) [(fj, - - 1 )
P,/' C/.) + 2/iP,/ ( )] = 2 (sn- a - sn 6) n (n + l) P (jj,),

when we use Legendre's differential equation \hardsubsectionref{16}{1}{3}). And the lemma
is established.

The result of applying the operator

 2 - n (n + l)k"sn-oc-A '

to the integi'al

r2K

Pn (k sn a sn 6) E T (0) dO

J ~2K

is now seen to be

\ - -n n + l) k' sn a - .4,,' [ P (k sn a sn 6) E " (6) dO J -2K (ca
)

= \, \ \ .- n + ) k-s,n e-AyvPn k nasne) and when we integrate twice
by parts this becomes

E,r e)dd,

-B PAfcsnasnB) \ dE e)

2K \ -2K

+ j Pn ( n asn d) Uj,- n (n + 1) k' sn' d - A . [ En'"" d) .dd = 0.

Hence it follows that the integral riK

Pn (k sn a sn 6) En"" (0) d9

J -2K *

is annihilated by the operator

d

T- 2 - n n + l)k sn- a - J.,l"

%
% 566
%

and it is evidently a polynomial of degree n in sn- a. Since Lame's
equation has onlj' one integral of this type*, it follows that the
integral is a multiple of En ' (a) if it is not zero; and the result
is established.

It does not appear to have been proved that the only vahies of A, for
which the equation

f a) = \ [" J\ \ {kHnaKn6)f 6)dd

has a solution, are those which make/(o) a solution of Lame's
equation.

Example 1. Shew that the nucleus of an integral equation satisfied by
Lame functions of the first species n being even) or of the second
species n being odd) with en a as a factor, may be taken to be

/  (-p en a en 6

Example 2. Shew that the nucleus of an integral equation satisfied by
Lame functions of the first species (?i being even) or of the second
species n being odd) with dn a as a factor, may be taken to be

P Ypdnadn y

23'61. Tlie integral equation satisfied by Lame functions of the third
and fourth species.

The theorem analogous to that of\hardsectionref{23}{6}, in the case of Lame functions
of the third and fourth species, is that any Lame function of the
fourth species (n being odd) or of the third species (n being even)
with en a dn a as a factor,, satisfies the integral equation r2K E,r
(a) = X en a dn a en 6 dn dPn' k sn a sn 6) E/'' (0) dO.

J -2 A'

The preliminary lemma is that the nucleus

en a dn a en d dn dPn' k sn a sn 6), like the nucleus of\hardsectionref{23}{6}, is
annihilated by the operator

TT-, - -, - n n + l) A'- (sn- a - sn- 6). dot- ctf-

To verify the lemma observe that

| JcnadnaP/(/ snasn )

= k- cn ' a dn'' a sn- PJ'' (/u.) - 3 - sn a en a dn a sn (dn- a + k'
en- a) Pn" (fi) - en a dn a (dn- a + k- en- a - 4k' sn" a) Pn" (fi),
and so

. \ I . (en a dn a en dn 6Pn k sn a sn 0)\

= kcnadnacne dn 6 (sn a - sn' 0) (,x'' - 1 ) PJ " (/i) -f GfxPn" ( ) +
GPn" (/*)]

d = k- en a dn a en dn (sn- a - sn- ) 3 (m-' - 1) Pv/ (a )

= k- n n + 1 ) en a dn o£ en 6 dn 6 (sn- a. - sn'- 6) Pn" (/a),

* The other solution when expanded in descending powers of sn a begins
with a term in (sn a)- - .

%
% 567
%

and the lemma is established. The proof that En (a) satisfies the
integral equation now follows precisely as in the case of the integral
equation of\hardsectionref{23}{6}.

Example 1. Shew that the nucleus of an integral equation which is
satisfied by Lame functions of the fourth species ( being odd) or of
the third species ii being even) with sn a dn a as a factor, may be
taken to be

sn a dn a sn 6 dn 6 P " ( -77 en a en ) .

Example 2. Shew that the nucleus of an integral equation which is
satisfied by Lame functions of the fourth species ii being odd) or of
the third species (/i being even) with sn a en a as a factor, may be
taken ta be

sn a en a sn 6 en 6P,l' j, dn a dn | .

Example 3. Obtain the following three integral equations satisfied by
Lame functions of the fourth species n being odd) and of the third
species (?i being even) :

(i) k" ..'- a E.r ( ) -X c,. dn / P. (i. ., sn ) jjJj '- ] M, (ii) -i
c =,.i.V'W = -.,n d / P,.(fcn c ) - / > o(,

(iii) I.-' in' a Er M = Xr- s a c / i>. Q. dn dn ) J ™ M;

in the case of functions of even ordei-, the functions of the
different types each satisfy one of these equations only.

23'62. Integral formulae fur ellipsoidal harmonics.

The integral equations just considered make it possible to obtain
elegant representations of the ellipsoidal harmonic Gn x, y, z) and of
the corre- sponding homogeneous harmonic H x, y, z) in terms of
definite integrals.

From the general equation formula of\hardsectionref{18}{3}, it is evident that Hn x,
y, z) is expressible in the form

Hn''' (x, y,z)= I (x cost + y sin t + izTf(t) dt,

J -TV

where y( ) is a periodic function to be determined.

Now the result of applying Niven's operator D- to x cos t + y sin t +
t )" is

n (n - 1 ) (a- cos"- 1 + b'- sin t - c'-) (x cos t + y sin t + izy ~,

and so, by Niven's formula \hardsubsectionref{23}{2}{5}) we find that Gn" x, y, s) is
expressible in the form

n n-l)(n-2)(n-S) )

+ 2.4(2n-l)(2u-3) '  / t)at,

%
% 568
%

where * l = xcost + y sin t + is,

  = V ( ' - C-) cos- t + b-- C-) sin t], so that

Now write sin = cd, the modulus of the elliptic functions being, as
usual, given by the equation

, \ tt" - b' a- - C-

The new limits of integration are - K and K, but they may be replaced
by - 'IK and 'IK on account of the periodicity of the integrand.

It is thus found that

 - T / 'a;sn + ycn + >dn \ .

e " ( . y, ) = j\ . i ( :j \ - j > ( ) ''<'.

where ( ( ) is a periodic function of 6, independent of x, y, z, which
is, as yet, to be determined.

If we express the ellipsoidal harmonic as the product of three Lame
functions, with the aid of the formulae of § 235 we find that

rlK

E (a) E,r (/3) E,r l)=G\ Pa (/x) < (0) dO,

J -2A"

where C is a known constant and

/x = k- sn a sn /3 sn 7 sn - k-/k'-) en a en /? en 7 en

- (l/'k'-) dn a dn y8 dn 7 dn 6.

If the ellipsoidal harmonic is of the first species or of the second
species and first type, we now give /3 and 7 the special values

  = K, y = K + iK', and we see that

C [' Pn(ksnasnd)(f3 e)(W

J -2 A'

is a solution of Lame's equation, and so, by\hardsectionref{23}{6}, ( ) is a solution
of Lame's equation which can be no other* than a multiple of E,i'"
(6).

Hence it follows that

l"- T. /'k'x sn0+ u en \$ + iz dn \,,, 7

where X is a constant.

* If (p e) involved the second solution, the integral would not
converge.

23-63]

ELLIPSOIDAL HARMONICS

569

If Gn!" x, y, z) be of the second species and of the second or third
type

we put

/3 = 0, 7 = iiT + iK',

or /3 = 0, 7 = /i

respectively, and we obtain anew the same formula.

It thus follows that if (th'" x, y, z) be any ellipsoidal harmonic of
the first or second species, then

GrT oo, y,z) = \ f "" Pn (/ ) E,i- 6) cie,

J -2 A'

  '(x, y, z) = X. ' r r ( ' sn + 3/ en 6? + iz dn df En''' (0) dd,

where [x = (k'x sn6 + y end + iz dn d)/\/ b- - c").

23'63. I nteg7'al formulae for ellipsoidal harmonics of the third and
fom th species.

In order to obtain integral expressions for harmonics of the third and
fourth species, we turn to the equation of\hardsubsectionref{23}{6}{2}, namely

£,' (a) En'- (/8) En - (7) = C r Pn (yu) cf> (0) d0,

J -iK

where

fi = A; sn a sn /3 sn 7 sn - k-/k'-) en a en /3 en 7 en - (l/k'-) dn a
dn /S dn 7 dn;

this equation is satisfied by harmonics of any species.

Suppose now that £'/' (a) is of the fourth species or of the first
type of the third species so that it has en a dn a as a factor.

We next differentiate the equation with respect to /3 and 7, and then
put = K,y = K + iK'.

It is thus found that d

En' ict)

d/3

En'"( )

i.-=A-L 7

V = A'+iA"

= C'

2A

Now SO that

dPn(f )

\ 87 \

K L ?/3S7 .

(? = K, y = K+iK'

(0) d0.

y = A'+?A''

- ilk') dn a dn /S dn 0Pn (/x),

d dy J =K,y=K+iK

:')

= - en a dn a en dn 0Pn" (k sn a sn 0).

Hence

r

J -2A

en a dn a en dn 0Pn" (k sn a sn ) ( (0) d0

is a solution of Lame's equation with en a dn a as a factor; and so,
by\hardsubsectionref{23}{6}{1}, (ji (0) can he none other than a constant multiple of.E -
(a).

%
% 570
%

We have thus found that the equation

G,r oc, y,z) = \ \ Pn (fi) E,r (0) (le

J -IK

is satisfied by any ellipsoidal harmonic which has en a dn a as a
factor; the corresponding formula for the homogeneous harmonic is

Hn'" (x, XL z) = \ J- I ( /. ' X sn e- xi en 6 + t>dn ) E. iO) dd.

Example. Shew that the equation of this section is satisfied by the
ellipsoidal harmonics which have sn a dn a or sn a en a as a factor.

23"7. Generalisations of Lame s equation.

Two obvious generalisations of Lame's equation at once suggest them-
selves. In the first, the constant B has net one of the characteristic
values Bn, for which a solution is expressible as an algebraic
function of f u); and in the second, the degree n is no longer
supposed to be an integer. The first generalisation has been fully
dealt with by Hermite* and Halphenf, but the only case of the second
which has received any attention is that in which n is half of an odd
integer; this has been discussed by Brioschij, Halphen§ and Crawford
II .

We shall now examine the solution of the equation

 \ = [n n + l) io u) - B] i\,

where B is arbitrary and n is a positive integer, by the method of
Lindemann- Stieltjes already explained in connexion with Mathieu's
equation (§§ 19'5- 19-52).

The product of any pair of solutions of this equation is a solution of

'~ ' ' + " " + S ~ -" ' + '' '' = ' by\hardsubsectionref{19}{5}{2}. The algebraic form of
this equation is

4 ( - e,) (I - e,) ( \ .3) + 3 (6 - i,) '

- 4 ( 2 + - 3) + i l - -In (;i + 1 ) X = 0. If a solution of this in
descending powers of - go be taken to be

* Comptes liendus, lxxxv. (1877), pp. ti89-(;95, 728-732, 821-826.

t Fonctions Elliptiques, 11. (Paris, 1888), pp. 494-502.

:J: Comptes Remlus, i.xxxvi. (1878), pp. 313-315.

§ Fonctions Elliptiques, ii. (Paris, 1888), pp. 471-473.

II Quarterly Journal, xxvii. (1895), pp. 93-98.

%
% 571
%

the recurrence formula for the coefficients c,. is

4r (n, - r + I) (2/1 - ? + 1) Cr

= ( ?i - r + 1) 12e (n - r) (n - r + 2) - 4eo (n- + n - 3) - 4J5 c,\ i
- 2 (n - r + ] ) (n - r + 2) (ei - e ) (e. - 63) (2h - 2r + 3) c,\ 2.

Write r = 71 4- 1, and it is seen that Cn+i =; then write r = n + 2
and Cn+2 =; and the recurrence formulae with r > n + 2 are all
satisfied by taking

Hence Lame's generalised equation always has two solutions tuhose
product is of the form

2c,( -e,)-'-.

r=0

This polynomial may be written in the form

n

n j(w)-\&>( r)l>

where i, a, ... a are, as yet, undetermined as to their signs; and
the two solutions of Lame's equation will be called Aj, A,.

Two cases arise, (I) when A1/A2 is constant, (II) when Aj/A is not
constant.

(I) The first case is easily disposed of; for unless the polynomial

r = l

is a perfect square in, multiplied possibly by expressions of the
type - e, - 2, - 63, then the algebraic form of Lame's equation has
an indicial equation, one of whose roots is |, at one or more of the
points | = i> (a,-); and this is not the case \hardsubsectionref{23}{4}{3}).

Hence the polynomial must be a square multiplied possibly by one or
more of - e, - 62, - e-s, and then Aj is a Lame function, so that B
has one of the characteristic values 5,/"; and this is the case which
has been discussed at length in §§ 23-1-23-47.

(II) In the second case we have \hardsubsectionref{19}{5}{3})

du ' du

where S is a constant which is not zero. Then

d log A2 d log Ai \ 26 du du X

d log A2 d log Ai \ 1 dX du du X du

rf log A, \ 1 dX 6 c logA a 1 dX 6 so that - 2 X ' du 2X du X -

%
% 572
%

On integration, "sve see that Ave may take

A, = VXexp|-(>[~|. Again, if we differentiate the equation

[chap. XXIII

we find that

1 dA,\ 1 dX (i Ai du ~ '2X du X '

1 d'A, 1 dA,\ \ 1 d'X 1 fdXy g dX

A, dir- ( Ai du \ 2X dti? 2X' \ du ) X du '

and hence, with the aid of Lame's equation, we obtain the interesting
formula

If now,. = jf) a,), we find from this formula (when multiplied by
X-), that, \ i u be given the special value cir, then

'dXV

m-

We now fix the signs of aj, an, ... cin by taking 'dX\ 2g

fdX\ \ 2(

( .)

And then, if we put 26/X, g-ita function of |, into partial fractions,
it is seen that

2(J (a' (n \ '

and therefore
TODO
whence it
follows that \hardsubsectionref{20}{5}{3}, example 1)

Ai= n

a- (ar + -m)

;x[exp|-?t 2 (a,)|,

and

r=l lO" (z<) (7 (a,

 2= n I - J-exp|M 2 ?(a,.)h

,.=1 |o-(iOo-(a,) =1 '

The complete solution has therefore been obtained for arbitrary values
of the constant B.

%
% 573
%

23'71. The Jacobian form of tJie generalised Lame equation. We shall
now construct the solution of the equation

-- = [n n + 1) k' sn a + A,

for general values of A, in a form resembling that of\hardsectionref{23}{6}.

The solution which corresponds to that of\hardsectionref{23}{6} is seen to be*

where p, a, a, ...cun are constants to be determined. On
differentiating this equation it is seen that

Ada,-=i|H(a + a,) B\ a)\ \

n

= X Z (a + a, + iK') - Z (a) + p + \ mrilK,

r=l, 1 C A (1 dA, 1 .w -rr i,,

   ' A - JA 1 = .-X "" " + ° + - ' " ""

and therefore, since A is a solution of Lame's equation, the constants
p, a, Oo, ... a,i are to be determined from the consideration that
the equation

2 Z(a + ar+ iK') - Z a ] -f p + niri/K

r = l

n n + 1) k- sn- a - A = X dn- (a + a,- + iK') - dn- a

r=l

+ is to be an identity; that is to say

n

n'k" sn- a + 71 + A + 2 cs-(a + a,-)

2 Z (a + a, + t7f ') - Z (a)] + p + mrilK

r=l

Now both sides of the proposed identity are doubly periodic functions
of a with periods 2K, 2iK', and their singularities are double poles
at points congruent to -iK', - Qi, - a.,, ... - a; the dominant terms
near -iK' and -;. are respectively

n- 1

(a+TKy ' ~ (a + arf

in the case of each of the expressions under consideration.

The residues of the expression on the left are all zero and so, if we
choose p, !, Ko, ... On SO that the residues of the expression on the
right are zero,

* This solution was published in 1872 in Hermite's lithographed notes
of his lectures delivered at the Ecole polytecbnique.

574

THE TRANSCENDENTAL FUNCTIONS [CHAP. XXIII

it will follow from Liouville's theorem that the two expressions
differ by a constant which can be made to vanish by proper choice of
A.

We thus obtain n+ 2 equations connecting p, a, a.., ... a,,, with A,
but these equations are not all independent.

It is easy to prove that, near - o,

2 Z a + a, + iK') - Z a)\ + p + i iri/K

r=l

= - + I' Z (Op - a, + Hi') + nZ (a,) + p + i (n - 1) -nilK + (a +
a,-), where the prime denotes that the term for which jj = ?' is
omitted; and, near

I [Z (a + a, + iK') - Z (a)] + p + i n-ni\ K

r = \

= + i Z(,) + p+0(a+27r).

ft + til >. = !

Hence the residues of

i Z (a + a, + i7i ') - Z (a) + p + | ??.7rt7A" will all vanish if p,
a,, a., ... ft are chosen so that the equations ' i' Z (ft,, - ft, +
iK') + '/iZ (ft,) + p + ( - 1) 7r?:/ii = 0,

I Z(a,) + P = V -=1

are all satisfied.

The last equation merely gives the value of p, namely

- 2 Z (ft,),

and, when we substitute this value in the first system, we find that

2' [Z (ftp - ft, -f- iK') + Z (ft,) - Z (a,,) + t TTzyZ] = 0,

where r = 1, 2, ... n. By\hardsubsubsectionref{22}{7}{3}{5}, example 2, the sum of the
left-hand sides of these equations is zero, so they are equivalent to
n - 1 equations at most; and, when i, a.,, ... a have any values
which satisfy them, the difference

7t-A''- sn- a + ?i + -4 + 51 cs- ( + ft;-)

S |Z (ft + ft, + IK') - Z (ft) - Z (ft,) + \ irijK\

%
% 575
%

is constant. By taking a = 0, it is seen that the constant is zero if

n + A + S cs- Of,. =

r = l

2 [Z (a, + iK') - Z (a,) + Tri/K]

) = !

i.e. if \ S en Of,, ds cif,.[- - 2 ns- ot,. = .

ij-=l j r=l

We now reduce the system of n equations; with the notation of § 22
'2, if functions of Op, a,, be denoted by the suffixes 1 and 2, it is
easy to see that

Z up -ar + iK') + Z (a,) - Z (op) + Tri'/A'

= Z (a - Or + iK') + Z(ar) -Z ap + iK') + Cj 0?i /Si = k' sn (op +
iK') sn a sn (ap + iX' - o ) + Ci ci?i /si

 2

+

Cidi

Sj sn (op- n,.) Si

\ SiCiO?i+g2C2C 2 ~ V- 2

Consequently a solution of Lame's equation

-- -= 71. (n + l) -sn-a+ A] A act- I '

IS

A= n

@( ) exp-aZ(a,)

provided that !, a2>  n be chosen to satisfy the n independent
equations comprised in the system

/, sn ttp en ttp dn ctp + sn of en o dn;. \

I J i sn Op - sn ar

2 en ttr ds a,.

2 ns a,. = A;

r = l

and if this solution of Lame's equation is not doubly periodic, a
second solution is

"H(a-a,)

n

@ (a)

exp [otZ (,.)]

= 0.

The existence of a solution of the system of n - l equations follows
from\hardsectionref{23}{7}.

REFERENCES.

G. Lam, Journal de Math. ii. (1837), pp. 147-188; iv. (1839), pp.
100-125, 126-163, 351- 385; vin. (1843), pp. 397-434. Lecons sur les
fonctions inverses des transcendantes et les surfaces isothermes
(Paris, 1857). Lemons sur les coordonnees curvilignes (Paris, 1859).

E. Heine, Journal fUr Math. xxix. (1845), pp. 185-208. Theorie der
Kugelfunctionen, ii. (Berlin, 1880).

C. Herhite, Comptes Rendus, lxxxv. (1877), pp. 689-695, 728-732,
821-826; Ann. di Mat. (2) IX. (1878), pp. 21-24. Oeuvres
Mathematiques (Paris, 1905-1917).

%
% 576
%

G. H. Halphen, Fonctions Elliptiques, il. (Paris, li

F. LixuEMANN, Math. Ann. xix. (1882), pp. 323-386.

K. Hecn, JIath. Ann. xxxili. (1889), pp. 161-179, 180-196.

L. Crawford, Quarterly Journal, xxvii. (1895), pp. 93-98; xxix.
(1898), pp. 196-201.

W. D. NiVEN, Phil. Trans, of the Royal Society, 182 a (1891), pp.
231-278.

A. Cayley, Phil. Trans, of the Royal Society, 165 (1875), pp. 675-774.

G. H. DAR VIN, Phil. Trans, of the Royal Society, 197 A (1901), pp.
461-557; 198 a (1901),

pp. 301-331.

Miscellaneous Examples.

1. Obtain the formula

Gn X, y, =, /J " Pn (J) e-' dn.Hn x, y, z).

(Niven, Phil. Travis. 182 a (1891), p. 245.)

2. Shew that

rr fl l\ \ (-r. 2n)l Hn x,y,z)

(Hobson, Proc. London Math. Soc. xxiv.)

3. Shew that the 'external ellipsoidal harmonic' F '" (\$) En'" (rj)
E,,''" (C) is a constant multiple of

"Xcx' dy' 82; V 2.(271 + 3) 2. 4 (2yi + 3) (2?i + 5) --J x +y + z )'

(Niven; and Hobson, Proc. London Math. Soc. xxiv.)

4. Discuss the confluent form of Lame's equation when the invariants 2
and g of the

Weierstrassian elliptic function are made to tend to zero; express the
solution in terms of

Bessel functions.

(Haentzschel, Zeitschrift fiir Math, und Phys. xxxi.)

5 If t! denotes - - exp [ X - Z (/x) a], where X and /x are constants,
shew that

G(a)

Lamd's equation has a solution which is expressible as a linear
combination of

dn-l cln-3 dn-b

where X- and sn p. are algebraic functions of the constant A.

\addexamplecitation{Hermite.}

6. Obtain solutions of

- =12Psn22-4(l + F)±5V(I-F + Z-'). w dz-

\addexamplecitation{Stenberg, Acta Math, x.}

7. Discuss the solution of the equation

2(2-l)(2-a) + [(a + + l)s' - n + /3-g + l+(y + S)a i+ay] + (5-j)3/ =

in the form of the series

Gn q) hY

l+a,3 2

i=i !y(y+l)...(y+ )' where Gi q) = q, (?2(?) = a/3?2 + (a + /3-8+ l) +
(-y + 8)a j-ay,

6-' + i (?) = [ (a + /3-8 + n) + (y + S + -l)o + a/3j]6' (g)

-(a + -l)(/3 + ?i-l)(y + n-l)?i .6' \ i(j).

(Heun, Math. Ann. xxxiii.)

%
% 577
%

8. Shew that the exponents at the singularities 0, \, a, cc of Heun's
equation are

(0, 1-y), (0, 1-S), (0,1-0, (, ), where y + 8 + e = a + l3 + l.

(Heun, Math. Ann. xxxiii.)

9. Obtain the following group of variables for Heun's equation,
corresponding to the group

\ I z 2-1

for the hypergeometric equation :

 )

1-z,

1

z '

1

1-2'

z z-l'

z

-1

z '

z

a - z

  

a

z

z-a

a'

a '

a-z'

z-a'

z

z-

-a

z-l

1-

 a

a-

1 z-

 a

z

-1

1-a' a - 1' 2-a' z-V s-1' 2 '

2-a (a-l)2 a(2 - 1) a(2-l) z - a (l-a)2

a (2-1)' rt(2-l)' 2-a ' (a -1)2' (I-o)!;' 2-a '

(Heun, J/a . 4 ?;. xxxiii.)

10. If the series of example 7 be called

F a, q; a, jS, y, 8; 2),

obtain 192 solutions of the differential equation in the form of
powers of 2, 2- 1 and z - a. multiplied by functions of the type F.

[Heun gives 48 of these solutions.]

11. If ic=2v, shew that Lame's equation

r/2 A

- = n (71 + 1) iHu) + B A

may be transformed into

by the substitution

12. If C = P (v), shew that a formal solution of the equation of
example 11 is

provided that (a - 2?i) (a - ?i + 5) =

and that

4:(a-r-2n) a-r-n + h)b,+ [l2e.2ia-r + l) a-r-2n+l) + 'ie2n
2n~l)-4B]br-i

- 4( 1 - 62) (6-2 - 63) (a-r + 2) (a-r - n+l) h,.\ 2 = 0., (Brioschi,
Comptes Bendus, lxxxvi. (1878), pp. 313-3,15 and Halphen.)

13. Shew that, if n is half of an odd positive integer, a solution of
the equation of example 11 expressible in finite form is

i= 6,(C-e2)'"-'-, r=0

W. M. A. 37

%
% 578
%

p o ded that

 ir u-r + )l>,. + [' -2e.> -2n-i- + ) r-l)-4ein '2n-l) + -iB]b,\ i

+ 4(ei-e2)( '2-t'3)(2 -r + 2)( -/-+S)6r-2 = 0,

and B is so determined that,j .j = 0-

\addexamplecitation{Brioschi and Halphen.}

14. Shew that, if n is half of an odd integer, a sohition of tlie
equation of example 11 xpressible in finite form is

Z'="i'V(C-eo)"- -

provided that

4/) (n+jt? + A) bp - [Ue., n -p+ [ +p - *) - 1<'2 (2 - 1) + 45] //p\ i

+ 4 (ei - f 2) ( -.' - 3) in-p + ' ) p-l) ?>',, -2 =

and ?>' 1=0 is the equation which determines B.

\addexamplecitation{Crawford.}

15. "With the notation of examples 13 and 14 shew that, if

V = ( - )" ( 1 - '2)'' e. - es)" c \ \ .j. the equations which
determine Cq, Cj, .< \ j!. ' I'e identical with those which
determine /)o, 61, ... > \ i; and deduce that, if one of the
solutions of Lame's equation (in which n is half of an odd integer) is
expressible as an algebraic function of p v\ so also is the other.

\addexamplecitation{Crawford.}

16. Prove that the values of B determined in example 13 are re;\ l
when e, e and 6-3 are real.

17. Shew that the complete solution of

is A = iO' U)r - Ap iu) + B],

where A and B are arbiti-aiy constants.

(Halphen, Jle'm. par divers savants, xxviii. (i), (1880), p. 105.)

18. Shew that the complete solution of

ig = |Fsn a-i(l4-F)

is \ = sn C-a)cQ C-a)dn C-a)r A + Bsnn.iC-a),

where .-1 and B are arbitrary constants and C=2K+iK'.

\addexamplecitation{Jamet, Comptes Rendus, cxi.}

\appendix
\chapter{Appendix}

THE ELEMENTARY TRANSCENDENTAL FUNCTIONS

A*l. On certain results assumed in Chapters I-IV.

It was convenient, in the first four chapters of this work, to assume
some of the properties of the elementary transcendental functions,
namely the exponential, logarithmic and circular functions; it was
also convenient to make use of a number of results which the reader
would be prepared to accept intuitively by reason of his familiarity
with the geometrical representation of complex numbers by means of
points in a plane.

To take two instances, (i) it was assumed (§ 2"7) that lim (exp i) =
exp (lim * , and (ii) the geometrical concept of an angle in the
Argand diagram made it appear plausible that the argument of a complex
number was a many-valued function, possessing the property that any
two of its values diffei-ed by an integer multiple of ir.

The assumption of results of the first type was clearly illogical ; it
wjis also illogical to base ai'ithmetical results on geometrical
reasoning. For, in order to put the foundations of geometry on a
satisfactory basis, it is not only desirj ble to employ the axioms of
arithmetic, but it is also necessary to utilise a further set of
axioms of a more definitely geometrical character, concerning
properties of points, straight lines and planes*. And, fm-ther, the
arithmetical theory of the logarithm of a complex nimiber appears to
be a necessary preliminary to the development of a logical theory of
angles.

Apart from this, it seems unsatisfcictory to the c esthetic taste of
the mathematician to employ one branch of mathematics as an essential
constituent in the structui-e of another ; particularly when the
former has, to some extent, a material basis whereas the latter is of
a purely abstract nature f.

The reasons for pursuing the somewhat illogical and unaesthetic
procedure, adopted in the earlier part of this work, were, firstly,
that the properties of the elementary transcen- dental functions were
required gradually in the com"se of Chapter n, and it seemed
imdesirable that the coiu e of a general development of the various
infinite processes should be frequently interrupted in oi"der to prove
theoremsf (with which the reader was, in all probabihty, already
famihar concerning a single particuliU* function ; and, secondly, that
(in counexioji with the assumption of results based on geometrical
considei-ations) a pui'ely arithmetical mode of development of
Chapters l-iv, deriving no help or illus- trations from geometrical
processes, would have very greatly inci-eased the difliculties of the
reader unacquainted with the methods and the spirit of the analyst,

* It is not our object to give any account of the foundations of
geometry in this work. They are investigated by various writei-s. such
as Tiitehead. Axioms of Projective Geometry (Cambridge Math. Tracts,
no. i. 1906) and Mathews, Projectile Geometry London, 1914). A perusal
of Chapters i, xx, xxii and xxv of the latter work will convince the
reader that it is even more laborious to develop geometiy in a logical
manner, from the minimum number of axioms, than it is to evolve the
theory of the circular functions by purely analytical methods. A
complete account of the elements both of arithmetic and of geometry
has been given by Whitehead and Bussell, Principia Mathematica
(1910-1913).

t Cf. Merz, History of European Thought in the Nineteenth Century, n.
(London, 1903), pp. 631 (note 2) and 707 (note 1), where a letter from
Weierstrass to Schwarz is quoted. See also Sylvester, P/ii7. Mag. (5),
ii. (1876), p. 307 [Math. Papers, lu. (1909). p. -50].



580 APPENDIX

A'll. Summary of the Appendix.

The general course of the Appendix is as follows :

In v5§ A*2-A-22, the exponential function is defined by a power
series. From this definition, combined with results contained in
Chapter ll, are derived the elementary properties (apart from the
periodic properties) of this function. It is then easy to deduce
corresponding properties of logarithms of positive numbers (§§
A'3-A'33).

Next, the sine and cosine are defined by power series from which
follows the connexion of these functions with the exponential
function. A brief sketch of the manner in which the formulae of
elementary trigonometry may be derived is then given ( § A'4-A-42).

The results thus obtained render it possible to discuss the
periodicity of the exponential and circular functions by purely
arithmetical methods (§§ A"5, A*51).

In §§ A"52-A'522, we consider, substantially, the continuity of the
inverse circular functions. When these functions have been
investigated, the theory of logarithms of complex numbers (§ A'6)
presents no further difficulty.

Finally, in § A"7, it is shewn that an angle, defined in a purely
analytical manner, possesises properties which are consistent with the
ordinary concept of an angle, based on our experience of the material
world.

It will be obvious to the reader that we do not profess to give a
complete account of the elementary transcendental functions, but we
have confined ourselves to a brief sketch of the logical foundations
of the theory*. The developments have been given by writers- of
various treatises, such as Hobson, Plane Trigonometry ; Hardy, A
course of Pure Mathematics ; and Bromwich, Theory of Infinite Series.

A'12. -1 logical order of development of the elements of Analysis.

The reader will find it instructive to read Chapters i-iv and the
Appendix a second time in the following order :

Chapter i (omitting + all of § 1 "5 except the first two paragraphs).

Chapter ii to the end of § 2*61 (omitting the examples inf §,
2-31-2-61).

Chapter iii to the end of § 3-34 and §§ 3-5-3-73.

The Appendix, §§ A-2-A-6 (omitting §§ A-32, A-33).-

Chapter ii, the examples of v § 2-31-2'61.

Chapter in, 5 § 3-341-3-4.

Chapter iv, inserting §§ A"32, A-33, Aw after § 4'13.

Chapter ii, §§ 2-7-2-82.

He should try thus to convince himself that (in that order) it is
possible to elaborate a purely arithmetical development of the
subject, in which the graphic and familiar language of geometry J is
to be regarded as merely conventional.

* In writing the Appendix, frequent reference has been made to the
article on Algebraic Analysis in the Encijklopiidie der Math.
Wissenschaften by Pringsheim and Faber, to the same article translated
and revised by Molk for the Encyclopedic des Sciences Math., and to
Tannery, Introduction h la Theorie dis Fonctions d'uue Variable
(Paris, 1904).

t The properties of the argument (or phase) of a complex number are
not required in the text before Chapter v.

i E.p. 'a point ' for 'an ordered number-pair,' ' the circle of unit
radius with centre at the origin' for 'the set of ordered number-pairs
(.r, y) which satisfy the condition .x'- + y-=l,' 'the points of a
straight line ' for ' the set of ordered number-pairs (x, y) which
satisfy a relation of the type Ax + By + C =0,' and so on.



A-ll-A-21] THE ELEMENTARY TRANSCENDENTAL FUNCTIONS 581

A "2. The exponential function exp z.

The exponential function, of a complex variable 2, is defined bj the
series*

exp.= l + f; + | + | + ... = l+J |;.

This series converges absolutely for all values of z (real and
complex) by D'Alembert's ratio test (§ 2'36) since lim | z\ n) | = 0<1
; so the definition is valid for all values of z.

Further, the series converges uniformly throughout any bounded domain
of values of z ; for, if the domain be such that j 2 j /iJ when z is
in the domain, then

j(sVm!)I R"/nl,

and the uniformity of the convergence is a consequence of the test of
Weierstrass (§ 3"34),

X

by reason of the convergence of the series 1+2 R"i'n !), in which the
terms are indepen-

n = l

dent of z.

Moreover, since, for any fixed value of n, s'V ! is a continuous
function of z, it follows from § 3'32 tha t the exponential function
is continuous for all values of z ; and hence (cf. § 3 '2), if z be a
variable which tends to the limit (, we have

lim exp 2= exp f.

A*21. The addition-theorem for the exponential function, and its
consequences. From Cauchy's theorem on multiplication of absolutely
convergent series (§ 2-53), it follows thatf

(exp 2i)(exp2,) = (l+- V|', + -)(l + + !;' + ..•)

2i + 22 Zi + 2z Z2 + Z-2 ,

= 1 + -Yj-+ 2", +...

= exp (21 + 22),

so that exp (21 + 22) can be expressed in terms of exponential
functions of 21 and of 22 by the formula

exp (21 + 22) = (exp 2i) (exp z. .

This result is known as the addition-theorem for the exponential
function. From it,

we see by induction that

(exp2i) (exp 22) ... (exp2,i) = exp(2i + 22+...+2 ),

and, in particular,

 exp 2 exp ( - 2) =exp 0=1.

From the last equation, it is apparent that there is no value of 2 for
which exp 2 = ; for, if there were such a value of 2, since exp (—2)
would exist for this value of 2, we should have 0=1.

It also follows that, when x is real, exp >0 ; for, from the series
definition, exp.r l when x O ; and, when x 0, exp x= 1/exp ( - .r)>0.



* It was formerly customary to define exp 2 as lim ( 1 + - I , cf.
  Cauchy, Coins cV Analyse, i.

p. 167. Cauchy ibid. pp. 168, 309) also derived the properties of the
function from the series, but his investigation when z is not rational
is incomplete. See also Sclilomilch, Handbuch der alg. Analysis
(1889), pp. 29, 178, 246. Hardy has pointed out (Math. Gazette, in. p.
284) that the limit definition has many disadvantages.

t The reader will at once verify that the general term in the product
series is (2i + Ci2i"- 22 + C22i"-22.\ ,2 + . . . + z ' n ! = ( i + z
)''ln ! .



582 APPENDIX

Further, exp .>• is an increasing function of the real variable ,v ;
for, if X.'>0, exp x + k) — exp x = exp x . exp >l- — 1 > 0, because
exp .r>0 and exp /•>!.

Also, since expA- 1 /A = 1 +(A/2 :) + (/i2/3 !) + ...,

and the series on the right is seen (by the methods of § A"2) to be
continuous for all

values of h, we have

lim exp/i- 1 //; = 1,

rfexpa ,. exp(2+A)-exp2 and so —y = hm — - = exp z.

A'22. Variom properties of the exponential function.

Returning to the formula (exp j) (exp 2.>) ... (exp 2- ) = exp (21+22
+ •••+2n) we see that,

when n is a positive integer,

(exp 2)" = exp (ns),

and (exp z)-" = 1 /(exp 2)" = 1 /exp nz) = exp ( - ?iz).

In particular, taking 2 = 1 and writing e in place of exp 1 =
2-71828..., we see that, when m is an integer, positive or negative,

e' = exp Hi = l + ( i/i:) + (? -/2 !) + ....

Also, if /Li be any rational number =Pi'q, where// and q are integers,
q being positive)

(exp fi)'i = exp fiq = exp p = e'\

so that the th power of exp is et' ; that is to say, expM is a value
of ef'ii=ei , and it is obviously (§ A-21) the real positive value.

If X be an irrational-real number (defined by a section in which Oj
and ao are typical members of the X-class and the ff-class
respectively), the irrational power e is most simply defined as exp x
; we thus have, for all real values of x, rational and irrational,

X x'

e =\ -I 1 1-

1 ! 2! '

an equation first given by Newton*.

It is, therefore, legitimate tg write e for exp x when x is real, and
it is customary to write e for exp 2 when z is complex. The function e
(which, of course, must not be regarded as being a power of e), thus
defined, is subject to the ordinary laws of indices,, viz.

[Note. Tannery, Lecons d'Algehre et d' Analyse (1906), I. p. 45,
practically defines e , when X is irrational, as the only number X
such that e'* X e"-, for every ! and ao. From the definition we have
given it is easily seen that such a unique number exists. For e\ ] )x(
= X) satisfies the inequality, and i( X' X) also did so, then

exj) 2 - exp r i = c'"- — /'' I X' - -V i ,

so that, since the exponential function is continuous, a-i — a cannot
be chosen arbitrarily small, and so ( i, a-2) does not define a
section.]

* De Aitdlysi per aequat. nu)n. term. inf. (written before 16G9, but
not published till 1711) ; it was also given both by Newtou and by
Leibniz in letters to Oldenburg in 1676 ; it was first published by
Wallis in 1685 in his Treatise on Algebra, p. 343. The equation when x
is irrational was explicitly stated by Schlomilch, Handbuch der alg.
Analysis (1889), p. 182.



A-22-A-32] THE ELi IENTARY TRANSCENDENTAL FUNCTIONS 583

A'3. Logarithms of positive numbers*.

It has been seen ( § A"2, A"21) that, when x is real, expA* is a
positive continuous increasing function of x, and obviously exp a- - +
x as x - - + oo , while

exp.'r=l/exp (— ')- >-0 as x- — cc.

If, then, a be any positive number, it follows from § 3-63 that the
equation in x,

exp X = a,

has one real root and only one. This root (which is, of course, a
function of a) will be written t LoggW or simply Log a ; it is called
the Logarithm of the positive number a.

Since a one-one correspondence has been established between i- and a,
and since a is an increasing function of x, x must be an increasing
function of a ; that is to say, the Logarithm is an increasing
function.

Example. Deduce from § A-21 that Log a + Log 6 = Log a6.

A "31. The continuity of the Logarithm.

It will now be shewn that, when a is positive. Log a is a continuous
function of a.

Let Log a = jp, Log (a + A) = .r + k

so that e =a, e'' =a-Vh, l + (hja) = e''.

First suppose that A>0, so that •>0, and then

l + hla) = l+k+U' + ...>l + k, and so 0<k<hja,

that is to say 0<Log( + A) — Loga<A/a.

Hence, h being positive, Log (a + A) — Log can be mjide arbitrarily
small by taking k sufficiently small.

Next, suppose that A<0, so that Z-<0, and then a/ a + h) = e~''. Hence
(taking 0< -h< a, as is obviousl ' permissible) we get

al(a + h) = l + -k) + U' +...>l-k, and so - - < - 1+ a/ (a + A) = - h/
a + h)<- hja.

Therefore, whether h be positive or negative, if e be an arbitrary
positive number and if I A I be taken less than both \ a and |ae, we
have

I Log a + h) — Log a\ <e,

and so the condition for continuity (§ 3-2) is satisfied,

A'32. Diferentiation of the Logarithm.

Retaining the notation of § A-31, we see, from results there proved,
 that, if h- O a being fixed), then also k ~Q. Therefore, when a>0,

(ILosa ,. k 11

Since Log 1=0, we have, by § 4"13 example 3,

Loga= I ~i dt.

* Many mathematicians define the Logarithm by the integral formula
given in § A.-32. The reader should consult a memoir by Hurwitz Math.
Ann. lxx. (1911), pp. 33-47) on the founda- tions of the theory of the
logarithm.

t This is in agreement with the notation of most text-books, in which
Log denotes the principal value (see § A-6) of the logarithm of a
complex number.



584 APPENDIX

A'33. The expansion of Log (1 + ) in powers of a. From § A "32 we have

Log (l + a) =i" I + t)~idt J •>

= r t+f -... + -y>-H'- + -)"(" i+t)- dt

= a - ia- + itf'' -... + (-)"-!- a" + /? ,

n

where / = (\ )" /"% (i +<)-i c .

Now, if - l<a<l, we have

i nl i t" l-\ a ) - dt

= |a|" + i (n+l)a-| i) -' -- as ?t - ac .

feence, when — !< <!, Log(l + a) can be expanded into the convergent
series*

X

Log(l+ ) = o--ia2 + ia3-...= 2 (-)"-! "/?;.

n = l If a=+l,

|/?,J=/ i (H-;)-it (;< I i;"rf = (>i + l)-i- Oas n-9-x, ./ ./

so the expansion is valid when a= + 1 ; it is not valid when = - 1.
Example. Shew that lim (l+-j =e.

[We have Jm. n log (l + 1) = Jiu. ( - , + sTT " -)

= 1,

and the result required follows from the result of § A-2 that lim e- =
e ]

A*4. The definition of the sine and cosine.

The functions t sin z and cos z are defined analytically by means of
power series, thus

23 25 CO /\ .) 22 + l

these series converge absolutely for all values of z (real and
complex) by § 2"36, and so the definitions are valid for all Vcilues
of z.

On comparing these series with the exponential scries, it is ap )arcnt
that the sine and cosine are not essentially new functions, but they
can be expressed in terms of exponential functions by the equations J

2i sin z = exp iz) — exp ( — iz), 2 cos 2 = exp (iz) + exp ( — iz).

* This method of obtaining the Logarithmic expansion is, in effect,
due to Walhs, P]iil. Trans. 11. (1668), p. 754.

t These series were given by Newtou, Dc Anahjsi... (1711), see § A'22
footnote. The other trigouometrical functions are defined in the
manner with which the reader is famiHar, as quotients and reciprocals
of sines and cosines.

X These equations were derived by Euler [they were given in a letter
to Jobann liernoulH in 1740 and pubhshed in the Hist. Acad. Berlin, v.
(1749), p. 279] from the geometrical definitions of the sine and
cosine, npon which the theory of the circular functions was tlien
universally based.



A'33-A"5] THE ELEMENTARY TRANSCENDENTAL FUNCTIONS 585

It is obvious that sin z and cos z are odd and even functions of z
respectively ; that is to say

sin ( — 2)= —sin 2, cos ( — 2) = cos 2.

A*41. The fundaraental properties of sin z and cos z.

It may be proved, just as in the case of the exponential function (§
A"2), that the series for sin 2 and cos 2 conyerge uniformly in any
bounded domain of values of z, and con- sequently that sin 2 and cos 2
are continuous functions of 2 for all values of 2. Further, it may be
proved in a similar manner that the series

, 22 z* 1-3-1 + 5--- defines a continuous function of 2 for all values
of 2, and, in particular, this function is continuous at 2=0, and so
it follows that

lim (2~ sin2) = L

A "42. The additio7i-theorems for sin 2 and cos 2.

By using Euler's equations ( A*4), it is easy to prove from properties
of the exponential function that

sin (2 + 22) = sin z cos 23 + cos 2 sin z and cos z + 22) — cos 21 cos
22 — sin z sin 22 ;

these results are known as the additiorit-theorems for sin 2 and cos
2.

It may also be proved, by using Euler's equations, that

sin2 2 + cos-2 = l.

By means of this result, sin(2j + 22) can be expressed as an algebraic
function of sin2i and sin 22, while cos (21 + 22) can similarly be
expressed as an algebraic function of cos 21 and cos 22 ; so the
addition-formulae may be regarded as addition-theorems in the strict
sense (cf. §§ 20-3, 22-732 note).

By differentiating Euler's equations, it is obvious that

(/sin 2 dcoaz

— J — = cos 2, — 5 — = — sm 2. dz dz

Example. Shew that

sin 2i = 2 sin 2 cos 2, cos 22 = 2 cos- 2 — 1 ;

these results are known as the duplication-formulae.

A'5. The periodicity of the exponential function.

If 2i and 22 are such that exp2j = exp22, then, multiplying both sides
of the equation by exp( — 22), we get exp (21 — 22) = ! ; and writing
y for z — Z2, we see that, for all values of 2 and all integral values
of n,

exp (2 -f- ny) = exp 2 . (exp y)'' = exp 2.

The exponential function is then said to have period y, since the
effect of increasing z by y, or by an integral multiple thereof, does
not affect the value of the function.

It will now be shewn that such numbers y (other than zero) actually
exist, and that all the numbers y, possessing the property just
described, are comprised in the expression

2mvi, ( =±1, ±2, ±3, ...)

where it is a certain jjositive number* which happens to be greater
than 2v'2 and less than 4.

* The fact that tt is an irrational number, whose value is 3'141-59...
, is irrelevant to the present investigation. For an account of
attempts at determining the value of tt, concluding with a proof of
the theorem that it satisfies no algebraic equation with rational
coefficients, see Hobson's monograph Squaring the Circle (1913).



586 APPENDIX

A*51. The sold tion of the equation e\ \ >y=.

Let y = a + ifi, where a and /3 are real ; then the problem of solving
the equation expy=l is identical with that of solving the equation

exp . expi/3=l. Comparing the real and imaginary parts of each side of
this equation, we have

expa.cos =l, expo, sin j3=0. Squaring and adding these equations, and
using the identity cos-/3 + sin2/3= 1, we get

exp2a=l. Xow if a were positive, exp 2a would be greater than 1, and
if a were negative, exp 2a would be less than 1 ; and so the only
possible value for a is zero. It follows that cos /3 = 1 , sin 3 = 0-

Now the equation sin/ii = is a necessary consequence of the equation
cos = l, on account of the identity cos2/3 + sin- i3=l. It is
therefore sufficient to consider solutions (if such solutions exist)
of the equation cos/3 = l.

Instead, however, of considering the equation cos/3 = l, it is more
convenient to consider the equation* cos 07=0.

It will now be shewn that the equation cos.r=0 has one root, and only
one, lying between and 2, and that this root exceeds 2 ; to prove
these statements, we make use of the following considerations :

(I) The function cos.r is certainly continuous in the range O a' 2.

(II) When A- '2, we have +

1- — >0 -\ \ >0 \ \ \ >o

2! ' 4! 6! ' 8! 10!= ' "'

and so, when .i- y'2, cos x > 0. (Ill) The value of cos 2 is

2<' / 4 \ 210 / 4 \

•- + S-72o('-7:Ti)-IO!('-l-Tn-2)--=-i--< -

(IV) WhenO<.r 2,

sin.v .v"\ A''* x \ , x' ,

— = 0-6) + 12oO-6r7) + ->l-6 ' and so, when x 4. 2, sin x \ x.

It follows from (II) and (III) combined with the results of (I) and of
3-63 that the equation cosa'=0 has at least one root in the range v 2
<.i'< 2, and it has no root in the range \$ x J±

Further, there is oiot more than, one root in the range J2<x<2; for,
suppose that there were two, Xi and X2 x.2>Xi) ; then
0<.r2-A'j<2-y'2<l, and sin (a% — Xij — sin .Cg cos x — sin Xi cos x =
0, and this is incompatible with (IV) which shews that sin (0: 2 — -
'i) M' 2~' 'i)-

The equation cos:r=0 therefore has one and onli/ one root lying
betiveen and 2. This root lies between 2 and 2, and it is called \ n ;
and, as stated in the footnote to § A"5, its actual value happens to
be 1 "57079....

* If cos.r = 0, it is an immediate consecjuence of the
duplication-formulae that cos2.r= -1 and tht'uce that co8 4.t = 1, so,
if x is a solution of cos.c = 0, 4.t is a solutiou of cos/3 = l.

t The symbol may be replaced by > except when x = , 2 in the first
place where it occurs, and except when x = in the other places.



A.-51, A-52] THE ELEMENTARY TRANSCENDENTAL FUNCTIONS 587

From the addition-formulae, it may be proved at once by induction that

cos ?;7r = ( — 1)", sin?i7r = 0, where ?i is any integer.

In particular, cos2)i7r = l, where n is any integer.

Moreover, there is no value of /3, other than those values which are
of the form 2?i7r, for which cosj3=l ; for if there were such a value,
it must be real*, and so we can choose the integer ra so that

— TT 27mr —I3<7r.

We then have

sin I WITT - /3 I = ±sin(m -l/3)=±sini =±2~2(l\ cos/3)i = 0,

and this is inconsistent t with sin j wtt — /3 | J | j/itt — i/3 [
unless /3 = 2mTr.

Consequently the numbers 2n7r, (/i = 0, ±1, ±2,...), a/id no others,
hare their cosines equal to unity.

It follows that a positive number n exists such that exps has period
iri and that exp2 has no period fundamentally distinct from liri.

The formulae of elementary trigonometry concerning the periodicity of
the circular functions, with which the reader is already acquainted,
can now be proved by analytical methods without any difficulty.

Example . Shew that sini r is equal to 1,-not to — L

Example 2. Shew that tan.i->.z; when 0<x<\ Tr.

[For cos.i'>0 and



sm.r — orcosA':



-—-A n-'i- 7- 1, l)!i 4>i + lj'



71 = 1 (4

and every term in the series is positive.]

x 77 v 25 x x

Example 3. Shew that 1 - V + ir7 ~ ; i positive when x = ,r , and that
1 - V + Jrr

2 24 / 20 lb z 24

vanishes when .r = (6 — 2 /3)2 = 1-5924... ; and deduce that J

3-125 <7r<3-185.

A'52. The solution of a pair of trigonometncal equations. Let X, /x be
a pair of real numbers such that X + jit l. Then, if X=|= - 1, the
equations

cos X = X, sin X = i have an infinity of solutions of which one and
only one lies between § — tt and tt.

First, let X and fi be not negative ; then (§3-63) the equation cos.r
= X has at least one solution Xi such that O .x-j iTr, since cos = 1,
cos 7r=0. The equation has not two solutions in this range, for if Xi
and x.2, were distinct solutions we could prove (cf. § A-51) that sin
(.t'l — .i"2) = 0, and this would contradict § A-51 (IV), since

0<| 2-- i I <i7r<2. Further, sin.ri= + (1 —coh' X])= -1- /(1 -- ) = M5
so x is a solution of both equations.

* The equation cos/3=l implies that exp i(i — l, and we have seen that
this equation has no complex roots.

+ The inequality is true by (IV) since j irnr - |/3 | : i7r<2.

+ See De Morgan, A Budget of Purado.ves (London, 1872), pp. .S16 et.
sfq., for reasons for proving that 7r>3 .

§ If X=: - 1, ±7r are solutions and there are no others in the range (
- tt, tt).



588 APPENDIX

The equations have no solutions iu the ranges ( - tt, 0) and ( tt, tt)
since, in these ranges, either sin x or cos x is negative. Thus the
equations have one soUition, and only one, in the range ( — rr, tt).

If X or fi (or both) is negative, we may investigate the equations in
a similar manner; the details are left to the rea<ier.

It is obvious that, if x- is a solution of the equations, so also is
Xy + 'iniY, where n is any integer, and therefore the equations have
an infinity of real solutions.

A'521. The principal solution of the trigonometrical equations.

The unique solution of the equations cos.t' = X, sin,t-=/i. (where X +
fx' = ) which lies between - it and n is called the principal
solution*, and any other solution differs from it by an integer
multiple of Stt.

The principal valuei of the argument of a complex number (=1=0) can
now be defined analytically as the principal solution of the equations

I z \ cos ( = / (2), j 2 1 sin = / z), and then, if = 1 2 1 . (cos + /
sin 6),

we must have d = (P + 2mr, and 6 is called a value of the argument of
j, and is written arg2 (cf i 1-5).

A "522. The continuity of the argument of a complex variable.

It will now be shewn that it is possible to choose such a value of the
argument 6 (z), of a complex variable z, that it is a continuous
function of z, provided that z does not pass through the value zero.

Let 2,1 6 given 'alue of 2 and let 6 be any value of its argument ;
then, to prove that 6 (z) is continuous at Zq, it is sufficient to
shew that a number 61 exists such that i=arg2i and that j 1 — 0 1 can
be made less than an arbitrary positive number e by giving 1 21 — 20 1
any value less than some positive number t).

Let fo = •'-"o + Vo > 1 = . 'i + >i •

Also let j 2i - So 1 l e chosen to be so small that the following
inequalities are satisfied J : (I) I .Ti - 0 i < 2 I - 0 1 5 provided
that Xq =t= 0, (II) I ?/i - 2/0 i < 2 ! J/o I , provided that 3/0 =t=
0, (III) |Xi-.%|<Je|2(,|, i?/i-3/o|<ie|2ol. From (I) and (II) it
follows that x Vi and y f/i are not negative, and

.'ToA-i a'o /o i > i.yo so that xo .vi + ?/o2/i J i 20 j 2.

Now let that value of 61 be taken which differs from 0 by less tlian
it ; then, since x , and Xi have not opposite signs and yo f 'id 1/1
have not opposite signs J , it follows from the solution of the
equations of A"52 that and 0 differ by less than n.

Now tan( i- ,.) = ':i ",

* If \= - 1, we take +7r as the principal solution ; of. p. 9.

t The term principal value was introduced in 1845 by Bjorling ; see
the Archiv der Math, und Fhijs. ix. (1847), p. 408.

X (I) or (II) respectively is simply to be suppressed in the case when
(i) .Tq O, or when (ii) 2/0 = 0.

§ The gtometrical interpretation of these conditions is merely that r
and z are not in different quadrants of the plane.



A"521-A7] THE ELEMENTARY TRANSCENDENTAL FUNCTIONS 589 aud so (§ A"51
example 2),

  I o(yi-yo)-y o(' i-- o) I

  2 I 2o |~ I 0 I • I yi- o ! + l 'o ! • I - 'i - 0 ! . But I A'o 1 ko
! and also | yo j | :o | ; therefore

I (9i - 6>o i =\$ 2 I 2o 1~ | yi- ?/o I + 1 1 - . 0 | <e-

Further, if we take | 2i - zq 1 l ss than • - ] J 'o | , (if o + 0)
and i [ yo i ? (if o + 0) and | e j g I , the inequalities (I), (II),
(III) above are satisfied ; so that, if ri be the smallest of the
three numbers * i | a-o | , -I ! yo ! > i ko i j by taking j Sj - q 1
< '?, we have | i - o I < < ; and this is the condition that 6 z)
should be a continuous function of the complex ariable z.

A '6. Logarithms of comflex numhers. The number f is said to be a
logarithm of z if 2 = e

To solve this equation in (, write = + '7? where and rj are real ; and
then we have

z = e (cos Tj + i sin ?;).

Taking the modulus of each side, we see that | 3| = e , so that (§
A-3), = Log \ z\; and

then

z=\ z\ (cos 7/ + 1 sin/;),

so that T] must be a value of argz.

The logarithm of a complex number is consequently a many-valued
function, and it can be expressed in terms of more elementary
functions by the equation

log z = Log j 2 1 4- 1 arg z.

The continuity of logs (when s4=0) follows from § A'31 and § A'522,
since \ z\ is a continuous function of z.

The differential coefficient of any particular branch of logs (§ 5'7)
may be determined as in § A*32 ; and the expansion of § A"33 may be
established for log (1 + ) when | a | < 1.

Corollary. If a be defined to mean e'iog , a is a continuous function
of z aud of a when a=t=0.

A'7. The analytical definition of an angle.

Let 1, 22, 23 be three complex numbers represented by the points Pj,
P. , F3 in the Argand diagram. Then the angle between the lines (§
A"12, footnote) PiPo and P1P3 is defined to be any value of arg (23 —
Zj) - arg (22 — Zi).

It will now be shewn f that the area (defined as an integi'al), which
is bounded by two radii of a given circle and the arc of the circle
terminated by the radii, is proportional to one of the values of the
angle between the radii, so that an angle (in the analytical sense)
jjossesses the propei'ty which is given at the beginning of all
text-books on Trigonometry |.

* If any of these numbers is zero, it is to be omitted.

t The proof here given applies only to acute angles ; the reader
should have no difficulty in extending the result to angles greater
than lir, and to the case when OX is not one of the bounding radii.

X Euclid's definition of an angle does not, in itself, afford a
measure of an angle ; it is shewn in treatises on Trigonometry (cf.
Hobson, Plane Trigonometry (1918), Ch. i) that an angle is measured by
twice the area of the sector which the angle cuts off from a unit
circle whose centre is at the vertex of the angle.



590 APPENDIX

Let (o-'i, yi) be any point (both of whose coordinates are positive)
of the circle x' - y- = a? a>0). Let 6 be the principal value of arg(
i + iyi), so that 0<6 <Itt. Then the area bounded by OX and the line
joining (0, 0) to x\, i/y) and the arc of the

circle joining .Vi, y ) to (o, 0) is / f x)dx, where*

./ '

/( )=a;tan (0 .r a cos ),

/ x) = a — x )- (a cos d x a), if an area be defined as meaning a
suitably chosen integral (cf. p. 61).

It remains to be proved that I f x) dx is proportional to 6.

J "

fa /"a cose fa ,

Now I Z f x) dx = I X tan 6 dx + (a - x-)i dx

J I) Jo J (I cose

 ha- smecose + hr L a --x ')- + x a -x ) \ dx

f" - 1

= a-j a -x-) - dx

J o COS Q

= |a2 1 r (1 - -) 'id(- p' (1 - f') - - dt\

on writing x = at and using the example worked out on p. 64.

That is to say, the area of the sector is proportional to the angle of
the sector. To this extent, we have shewn that the popular conception
of an angle is consistent with the analytical definition.

* The reader will easily see the geometrical interpretation of the
integral by drawing a figure.

% Index entries remaining to be added
\chapter{General Index: Unindexed}

[TVie numbers refer to the pages. References to theorems contained in a few of
the more important examples are given by numbers in italics]

Abel's discovery of elliptic functions, 429, 512; inequality, 16; integral equation, 211, 229, 230;
method of establishing addition theorems, 442, 496, 497, 530, 534; special form,, (2), of
the confluent hypergeometric function, 353; test for convergence, 17; theorem on continuity
of power series, 57; theorem on multiplication of convergent series, 58, 59

Abridged notation for products of Theta-f unctions, 468, 469; for quotients and reciprocals of
elHptic functions, 494, 498

Absolute convergence, 18, 28; Cauchy's test for, 21; D'Alembert's ratio test for, 22; De
 Morgan's test for, 23

Absolute value, see Modulus

Absolutely convergent double series, 28; infinite products, 32; series, 18, (fundamental
property of) 25, (multiplication of) 29

Addition formula for Bessel functions, 357, 380; for Gegenbauer's function, 335; for Legendre
polynomials, 326, 395; for Legendre functions, 328; for the Sigma-function, 451; for
Theta-functions, 467; for the Jacobian Zeta-function and for E[u), 518, 534; for the
third kind of elliptic integral, 523; for the Weierstrassian Zeta-function, 446

Addition formulae, distinguished from addition theorems, 519

Addition theorem for circular functions, 535; for the exponential function, 531; for Jacobian
elliptic functions, 494, 497, 530; for the Weierstrassian elliptic function, 440, 457; proofs
of, by Abel's method, 442, 496, 497, 530, 534

Affix, 9

Air in a sphere, vibrations of, 399

Amplitude, 9

Analytic continuation, 96, (not always possible) 98; and Borel's integi-al, 141; of the hyper-"
geometric function, 288. See also Asymptotic expansions

Analytic functions, 82-110 (Chapter v); defined, 83; derivates of, 89, (inequality satisfied by) 91;
clistinguislied from monogenic functions, 99; represented by integi-als, 92; Eiemann's
equations connected with, 84; values of, at points inside a contour, 88; uniformly convergent
series of, 91

Angle, analytical definition of, 589; and popular conception of an angle, 589, 590

Angle, modular, 492

Area represented by an integi-al, 61, 589

Argand diagram, 9

Argument, 9, 588; principal value of, 9, 588; continuity of, 588

Associated function of Borel, 141; of Eiemann, 183; of Legendre [P "' (z) and Q '" (z)], 323-326

Asymptotic expansions, 150-159 (Chapter viii); differentiation of, 153; integration of, 153;
multiplication of, 152; of Bessel functions, 368, 369, 371, 373, 374; of confluent hyper-
geometric functions, 342, 343; of Gamma-functions, 251, 276; of parabolic cylinder functions,
347, 348; uniqueness of, 153, 154

Asymptotic inequality for parabolic cylinder functions of large order, 354

Asymptotic solutions of Mathieu's equation, 425

Auto-functions, 226 *

Automorphic functions, 455

Axioms of arithmetic and geometry, 579

Barnes' contour integrals for the hypergeometric function, 286, 289; for the confluent hyper-
geometric function, 343-345

Barnes' G-f unction, 264, 278

Barnes' Lemma, 289

Basic numbers, 462

Bernoullian numbers, 125; polynomials, 126, 127

Bertrand's test for convergence of infinite integrals, 71

Bessel coefficients [$\besJ_{n}(z)$], 101; addition formulae for, 357; Bessel's integral for, 362;
differential equation satisfied by, 357; expansion of, as power series, 355; expansion of

?..S- 2

%
% 596
%
functions in series of (by Neumann), 374, 375, 3S4, (by Schlomilch), 377; expansion of
 t-z)~  in series of, 374. 375, 376; expressible as a confluent fomi of Legendre functions,
367; expressible as confluent hypergeometric functions, 358; inequality satisfied by, 879;
Neumann's function 0, (z) connected with, sec Neumann's function; order of, 356; recur-
rence formulae for, 359; special case of confluent hypergeometric functions, 358. See also
Bessel functions
Bessel functions, 355-385 (Chapter xvn), Jn z) defined, 358-360; addition formulae for, 380;
asymptotic expansion of, 368, 369, 371, 373, 374; expansion of, as an ascending series, 358,
371; expansion of functions in series of, 374, 375, 377, 5.S'i; first kind of, 359; Hankel's
integi'al for, 365; integi'al connecting Legendre functions with, 364, 401; integral properties
of, 380, 381, 384, 385; integi-als involving products of, 380, 383, 385; notations for, 356,

372, 373; order of, 356; products of, 379, 360, 383, 385, 428; recurrence formulae for, 359,

373, 374; relations between, 360, 371, 372; relation between Gegenbauer's function and,
378; Schliifli's form of Bessel's integral for, 362, 372; second kind of, Y (2) (Hankel), 370;
yC' (-) (Neumann), 372; 1',, (z) (Weber-Schliifli), 370; second kind of modified, K  z), 373;
solution of Laplace's equation by, 395; solution of the wave-motion equation by, 397;
tabulation of, 378; whose order is large, 368, 383; whose order is half an odd integer, 364;
with imaginary argument. I  z), K  z), 372, 373, 384; zeros of, 361, 367, 378,381. See
aho Bessel coefficients (ind Bessel's equation

Bessel's equation, 204, 357, 373; fundamental system of solutions of (when )i is not an integer),
359. 372; second solution when n is an integer, 370, 373. See also Bessel functions

Binefs integrals for log V (z), 248-251

B6clier's theorem on linear differential equations with five singularities, 203

Bolzano's theorem on limit points, 12

Bonnet's foi-m of the second mean value theorem. 66

Borel's associated function, 141; integral, 140; integral and analytic continuation, 141; method
of ' summing ' series, 154; theorem (the modified Heine-Borel theorem), 58

Boundary, 44
 ' Boundary conditions, 387; and Laplace's equation, 393

Bounds of continuous functions, 55

Branch of a function, 106

Branch -point. 106

Biirmanns theorem, 128; extended by Teixeira, 131

Cantor's Lemma, 183

Cauchy's condition for the existence of a limit, 13; discontinuous factor, 123; formula for the

remainder in Taylor's series, 96; inequality for derivatives of an analytic function, 91;

integral, 119; integi'al representing V [z), 243; numbers, 372; tests for convergence of series

and integrals, 21, 71
Cauchy's theorem, 85; extension to curves on a cone, 87; Morera's converse of, 87, 110
CeU, 430

Cesaro's method of ' summing ' series, 155; generalised, 156

Change of order of terms in a series, 25; in an infinite deteiTninant, 37; in an infinite product, 33
Change of parameter (method of solution of Mathieu's equation), 424
Characteristic functions, 226; numbers, 219; numbers associated with symmetric nuclei are

real. 226
Chartier's test for convergence of infinite integrals, 72
Circle, iirea of sector of, 589; limiting, 98; of convergence, 30
Circular functions, 435, 584; addition theorems for, 585; continuity of, 585; differentiation

of, 585; duplication fonnulae, 585; periodicity of, 587; relation with Gamma-functions,

239
Circular membrane, vibrations of, 356, 396
Class, left (L), 4; right  li), 4
Closed. 44
Cluster point, 13
CoefiBcients, equating, 59; in Fourier series, nature of, 167, 174; in trigonometrical series, values

ot, 163, 165
Coefficients of Bessel. sec Bessel coefficients

Comparison theorem for convergence of integrals, 71; for convergence of series, 20
Complementary moduli, 479, 493; elliptic integrals with, 479, 501, 520

Complete elliptic integrals [K, K, K', 7v"] (first and second kinds), 498, 499, 518; Legendre's re-
lation lietween, 520; properties of  qua functions of the modulus), 484, 498, 499, 501, 521;

 A

 %
 % 597
 %

series for, 299; tables of, 518; the Gaussian transformation, 533; values for small values
of \ k\, 521; values (as Gamma-functions) for special values of k, 524-527; with comple-
mentary moduli, 479, 501, 520
Complex integrals, 77; upper limit to value of, 78
Complex integration, fundamental theorem of, 78

Complex numbers, 3-10 (Chapter i), defined, 6; amplitude of, 9; argument of, 9, 588; dependence
of one on another, 41; imaginary part of (I), 9; logarithm of, 589; modulus of, 8; real part
oi\ R), 9; representative point of, 9
Complex variable, continuous function of a, 44

Computation of elliptic functions, 485; of solutions of integi'al equations, 211
Conditional convergence of series, 18; of infinite determinants, 415. See also Convergence and

Absolute convergence
Condition of integrability (Eiemann's), 63
Conditions, Dirichlet's, 161, 163, 164, 176
Conduction of Heat, equation of, 387
Confluence, 202, 337
Confluent form, 203, 337

Confluent hypergeometric function  Wj  to ( )]' 337-354 (Chapter xv); equation for, 337; general
asymptotic expansion of, 342, 345; integral defining, 339; integrals of Barnes' type for,
343-345; Kummer's formulae for, 338; recurrence formulae for, 352; relations with Bessel
functions, 360; the functions W    (z) and xl/j.\,  (z), 337-339; the relations between functions
of these types, 346; various functions expressed in terms of JFj,,,; (z), 340, 352, 353, 360. See
also Bessel functions mid Parabolic cylinder functions
Confocal coordinates, 405, 547; form a triply orthogonal system, 548; in association with
ellipsoidal harmonics, 552; Laplace's equation referred to, 551; uniformising variables
associated with, 549
Congruence of points in the Argand diagram, 430
Constant, Euler's or Mascheroni's, [7], 235, 246, 248
Constants ci, e., e, 443; E, E', 518, 520; of Fourier, 164; 771, r/.,, 446, (relation between -qi

and 7?,,) 446"; G, 469, 472; A', 484, 498, 499; K', 484, .501, 503 "
Construction of elliptic functions, 438, 478, 492; of Mathieu functions, 409, (second method)

420
'Contiguous hypergeometric functions, 294
Continua, 43
Continuants, 36

Continuation, analytic, 96, (not always possible) 98; and Borel's integral, 141; of the hyper-
geometric function, 288. See also Asymptotic expansions
Continuity, 41; of power series, 57, (Abel's theorem) 57; of the argument of a complex variable,
588; of the circular functions, 585; of the exponential function, 581; of the logarithmic
function, 583, 589; uniformity of, 54
Continuous functions, 41-60 (Chapter in), defined, 41; bounds of, 55; integrability of, 63; of a

complex variable, 44; of two variables, 67
Contour, 85; roots of an equation in the interior of a, 119, 123
Contour integrals, 85; evaluation of definite integrals by, 112-124; the Mellin-Barnes type of,

286, 343; see also under the special function represented by the integral
Convergence, 11-40 (Chapter 11), defined, 13, 15; circle of, 30; conditional, 18; of a double
series, 27; of an infinite determinant, 36; of an infinite product, 32; of an infinite integral,
70, (tests for) 71, 72; of a series 15, (Abel's test for) 17, (Dirichlet's test for) 17; of Fourier
series, 174-179; of the geometric series, 19; of the hypergeometric series, 24; of the series
2n~*', 19; of the series occurring in Mathieu functions, 422; of trigonometrical series, 161;
principle of, 13; radius of, 30; theorem on (Hardy's), 156. See also Absolute convergence.
Non-uniform convergence and Uniformity of convergence
Coordinates, confocal, 405, 547; orthogonal, 401, 548
Cosecant, series for, 135
Cosine, see Circular functions

Cosine- integral [Ci (z)], 352; -series (Fourier series), 165
Cotangents, expansion of a function in series of, 139
Cubic function, integration problem connected with, 452, 512
Cunningham's function [w,  j (z)], 353
Curve, simple, 43; on a cone, extension of Cauchy's theorem to, 87; on a sphere (Seiffert's

spiral), 527
Cut, 281
Cylindrical functions, 355. See Bessel functions

%
% 598
%
D'Alemberfs ratio test for convergence of series, 22

Darboux" formula. 125

Decreasing- sequence, 12

Dedekind's theory of irrational numbers, 4

Deficiency of a plane curve, 455

Definite integrals, evaluation of, 111-124 (Chapter vi)

Degree of Legendre functions, 302, 307, 324

De la Vallee Poussins test for uniformity of convergence of an infinite integral, 72

De Morgan's test for convergence of series, 23

Dependence of one complex number on another, 41

Derangement of convergent series, 25; of double series, 28; of infinite determinants, 37; of

intinite products, 33, 34
Derivates of an analytic function, fi9; Cauchy's inequality for, 91; integrals for, 89
Derivates of elliptic functions, 4 0
Determinant. Hadamard's, 212
Determinants, infinite, 36; convergence of, 36, (conditional) 415; discussed by Hill, 36, 415;

evaluated by Hill in a particular case, 415; rearrangement of, 37
Difiference equation satisfied by the Gamma-function, 237
Diflferential equations satisfied by elliptic functions and quotients of Theta-f unctions, 436, 477,

492; (partial) satisfied by Theta-functions, 470; Weierstrass' theorem on Gamma-functions

and, 236. See also Linear diflferential equations and Partial diflferential equations
DiflFerentiation of an asymptotic expansion, 153; of a Fourier series, 168; of an infinite

integral, 74; of an integral, 67; of a series, 79, 91; of elliptic functions, 480, 493; of the

circular functions, 585; of the exponential function, 582; of the logarithmic function, 583,

589
Dirichlet's conditions, 161, 163, 164, 176; form of Fourier's theorem, 161, 163, 176; formula

connecting repeated integrals, 75, 76, 77; integral, 252; integral for f (z), 247; integral for

Legendre functions, 314; test for convergence, 17
Discontinuities, 42; and non-uniform convergence, 47; of Fourier series, 167, 169; ordinary, 42;

regular distribution of, 212; removable, 42
Discontinuous factor, Cauchy's, 123

Discriminant associated with Weierstrassian elliptic functions, 444, 550
Divergence of a series, 15; of infinite products, 33
Domain, 44

Double circuit integrals, 256, 293
Double integrals, 68, 254
Double series, 26; absolute convergence of, 28; convergence of (Stolz' condition), 27; methods

of summing, 27; a particular form of, 51; rearrangement of, 28
Doubly periodic functions, 429-535. See alt > Jacobian elliptic functions, Theta-functions and

Weierstrassian elliptic functions
Duplication formula for the circular functions, 585; for the Gamma-function, 240; for the

Jacobian ellij)tic functions, 498; for the Sigma-function, 459, 460; for the Theta-functions,

48S; for the Weierstrassian elliptic function, 441; for the Weierstrassian Zeta-function, 459

Electromagnetic waves, equations for, 404

Elementary functions, S2

Elementary transcendental functions, 579-590 (Appendix). Sec (dm Circular functions.
Exponential function and Logarithm

Ellipsoidal harmonics, 536-578 (Chapter xxiii); associated with confocal coordinates, 552
derived from Lame's equation, 538-543, 552-554; external, 576; integi'al equations con
nected with, 567; linear independence of, 560; number of, when the degi-ee is given, 546.
physical applications of, 547; .species of, 537; types of, 537. See also Lamp's equation
and \Lame\ functions

Elliptic cylinder functions, see Mathieu functions

Elliptic functions, 429-535 (Chapters xx-xxii); computation of, 485; construction of, 433, 478
derivate of, 480; discovery of, by Abel, Gauss and Jacobi, 429, 512, 524; expressed by
means of Theta-functions, 473; expressed by means of Weierstrassian functions, 448-451
general addition formula, 457; number of zeros (or poles) in a cell, 431, 432; order of
432; periodicity of, 429, 479, 500, 502, 503; period parallelogram of, 430; relation be
tween zeros and poles of, 433; residues of, 431, 504; transformations of, 508; with no
poles (are constant), 431; with one double pole, 432, 434; with the same periods (relations
between), 452; with two simple poles, 432, 491. See also Jacobian elliptic functions,
Theta-functions and Weierstrassian elliptic functions

%
% 599
%

Elliptic integrals, 429, 512; first kind of, 515; function E (it) and, 517; function Z ( ) and,
518; inversion of, 429, 452, 454, 480, 484, 512, 524; second kind of, 517, (addition formulae
for) 518, 519, 534, (imaginary transformation of) 519; third kind of, 522, 523, (dynamical
application of) 523, (parameter of) 522; three kinds of, 514. See also Complete elliptic
integrals

Elliptic membrane, vibrations of, 404

Equating coefiQcients, 59, 186

Equations, indicial, 198; number of roots inside a contour, 119, 123; of Mathematical Physics,
203, 386-403; with periodic coefficients, 412. See also Difference equation, Integral
equations, Linear differential equations, and under the names of special equations

Equivalence of curvilinear integrals, 83

Error- function [Erf (,() and Erfc (,r)], 341

Essential singularity, 102; at infinity, 104

Eta-function [H ( )], 479, 480

Eulerian integrals, first kind of [B (m, n)], 253; expressed by Gamma-functions, 254; extended
by Pochhammer, 256

Eulerian integrals, second kind of, 241; .see Gamma-function

Euler's constant [7], 235, 246, 24S; expansion (Maclaurin's), 127; method of 'summing' series,
155; product for the Gamma-function, 237; product for the Zeta-function of Riemann, 271

Evaluation of definite integrals and of infinite integrals, 111-124 (Chapter vi)

Evaluation of Hill's infinite determinant, 415

Even functions, 165; of Mathieu [cc,  z, q)], 407

Existence of derivatives of analytic function, 89; -theorems, 888

Expansions of functions, 125-149 (Chapter vii); by Biirmann, 128, 131; by Darboux, 125; by
Euler and Maclaurin, 127; by Fourier, see Fourier Series; by Fourier (the Fourier-Bessel
expansion), 381; by Lagrange, 132, 149; by Laurent, 100; by Maclaurin, 94; by Pincherle,
149; by Plana, 145; by Taylor, 93; by Wronski, 147; in infinite products, 136; in series of
Bessel coefficients or Bessel functions, 374, 375, 381, 384; in series of cotangents, 139; in
series of inverse factorials, 142; in series of Legendre polynomials or Legendre functions,
310, 322, 330, 331, 335; in series of Neumann functions, 374, 375, 384; in series of parabolic
cylinder functions, 351; in series of rational functions, 134. See also Asymptotic expansions.
Series, and under the itaines of special functions

Exponential function, 581; addition theorem for, 581; continuity of, 581; differentiation of,
582; periodicity of, 585

Exponential-integrral [Ei (z)], 352

Exponents at a regular point of a linear differential equation, 198

Exterior, 44

External harmonics, (ellipsoidal) 576, (spheroidal) 403

Factor, Cauchy's discontinuous, 123; periodicity-, 463

Factorials, expansion in a series of inverse, 142

Factor-theorem of Weierstrass, 137

Fej r's theorem on the summability of Fourier series, 169, 178

Ferrers' associated Legendre functions [P '" (z) and Q "  (z)], 323

First kind, Bessel functions of, 359; elliptic integi-als of, 515, (complete) 518, (integration of)
515; Eulerian integral of, 253, (expressed by Gamma-functions) 254; integral equation of,
221; Legendre functions of, 307

First mean-value theorem, 65, 96

First species of ellipsoidal harmonic, 537, (construction of) 538

Floquet's solution of differential equations with periodic coefficients, 412

Fluctuation, 56; total, 57

Foundations of arithmetic and geometry, 579

Fourier-Bessel expansion, 381; integral, 385

Fourier series, 160-193 (Chapter ix); coefficients in, 167, 174; convergence of, 174-179; differ-
entiation of, 168; discontinuities of, 167, 169; distinction between any trigonometrical
series and, 160, 163; expansions of a function in, 163, 165, 175, 176; expansions of Jacobian
elliptic functions in, 510, 511; expansion of Mathieu functions in, 409, 411, 414, 420; Fejer's
theorem on, 169; Hurwitz-Liapounotf theorem on, 180; Parseval's theorem on, 182; series
of sines and series of cosines, 165; summability of, 169, 178; uniformity of convergence of,
168, 179. See also Trigonometrical series

Fourier's theorem, Dirichlet's statement of, 161, 163, 176

%
% 600
%
Fourier's theorem on integrals. 188, 211

Fourtb species of ellipsoidal harmonic, 537, (construction of) 542

Fredliolm's integral equation, 213-217, 228

Functionality, concept of, 41

Functions, branches of, 106; identity of two, 98; limits of, 42; principal parts of, 102; without
essential singularities, 105; which cannot be continued, 98. See also inidcr the mnnes of
special functions! or special types offitncti(yns, e.g. Legendre functions. Analytic functions

Fundamental formulae of Jacobi connecting Theta-functions, 467, 4S8

Fundamental period parallelogram, 430; polygon (of automorphic functions), 455

Fundamental system of solutions of a linear differential equation, 197, 200, 389, 559. .S (r also
iDulcr the names of special equations

Gamma-function [r( )], 235-264 (Chapter xii); asymptotic expansion of, 251, 276; circular
functions and, 239; complete elliptic integrals and, 524-527, 535; contour integi-al (Hankel's)
for, 244; difference equation satisfied by. 237; differential equations and, 236; duplication
formula, 240; Euler's integral of the first kind and, 254; Euler's integi-al of the second
kind, 241. (modified by Cauchy and Saalschiitz) 243, (modified by Hankel) 244; Euler's
product, 237; incomplete form of, 341; integrals for, (Binet's) 248-251, (Euler's) 241;
minimum value of, 253; multiplication formula, 240; series, (Rummer's) : 50, (Stirling's)
251; tabulation of. 253; trigonometrical integrals and, 256; Weierstrassian product, 235,
236. See also Eulerian integrals and Logaritlimic derivate of the Gamma-function

Gauss' discovery of elliptic functions, 429, 512, 524; integral for r'(~)/r(r), 246; lemniscate
functions, see Lemniscate functions; transformation of elliptic integrals, 533

Gegenbauer's function [C,/ (z)], 329; addition formula, 335; differential equation for, 329;
recuiTence formulae, 330; relation with Legendre functions, 329; relation involving Bessel
functions and, 56-5; Rodrigues' formula (analogue), 329; Schliifii's integral (analogue), 329

Genus of a plane curve, 455

Geometric series, 19

Glaisher's notation for quotients and reciprocals of elliptic functions, 494, 498

Greatest of the limits, 13

Green's functions, 395

Hadamard's lemma, 212

Half-periods of Weierstrassian elliptic functions, 444

Hankel's Bessel function of the second kind, Y,j( ), 370; contour integral for T (z), 244; integral

for J ( ), 365
Hardy's convergence theorem, 156; test for uniform convergence, 50
Harmonics, solid and surface, 392; spheroidal, 403; tesseral, 392, 536; zonal, 302, 392, 536;

Sylvester's theorem concerning integi-als of, 400. See also Ellipsoidal harmonics
Heat, equation of conduction of, 387
Heine-Borel theorem (modified), 53

Heine's expansion of (t - z)-  in series of Legendre polynomials, 321
Hermite's equation, 204, 209, 342, 347. See also Parabolic cylinder functions
Hermite's formula for the generalised Zeta-function f (s,  ), 269
Hermite's solution of Lame's equation, 573-575
Heun's equation, 576, 577

Hill's equation, 406. 413-417; Hill's method of solution, 413
Hill's infinite determinant, 36, 40, 415; evaluation of, 415
Hobson's associated Legendre functions, 325
Holomorphic, 83

Homogeneity of Weierstrassian elliptic functions, 439
Homogeneous harmonics (associated with ellipsoid), 543, 57G\ ellipsoidal harmonics derived

from (Nivcn's formula), 543; linear independence of, 560
Homogeneous integral equations, 217, 219
Hurwitz' definition of the generalised Zeta-function j'(x,  ), 265; formula for f (.<,(/), 268;

theorem concerning Foiu'ier constants, 180
Hypergeometric equation, see Hypergeometric functions
Hypergeometric functions, 281-301 (Chapter xiv); Barnes' integrals, 286, 289; contiguous, 294;

continuation of, 2H,S; contour integrals for, 291; differential equation for, 202, 207, 283;

functions expressed in terms of, 281, 311; of two variables (Appell's), 300; relations between

twenty-four expressions involving, 284, 285, 290; liiemann's P-equation and, 208, 283;

series for (convergence of), 24, 281 squares and products of,  96'; value of F(a, l>; c; 1),

%
% 601
%

281, 393 \ values of special forms of bypergeometric functions, 298, 301. See also Bessel
functions, Confluent hypergeometric functions and Legendre functions

Hypergeometric series, >iee Hypergeometric functions

Hypothesis of Riemann on zeros of f(.s), 272, 280

Identically vanisMng power series, 58

Identity of two functions, 98

Imaginary argument, Bessel functions with [I,j(2) and K  z)'], 372, 373, 384

Imaginary part (/) of a complex number, 9

Imaginary transformation (Jacobi's) of elliptic functions, 505, 506, 5-3.5; of Theta-functions, 124,
474; of E[u) and Z(h), 519

Improper integrals, 75

Incomplete Gamma-functions \ \  y n, c)], 341

Increasing sequence, 12

Indicial equation, 198

Inequality (Abel's), 16; (Hadamard's), 212; satisfied by Bessel coefficients, 379; satisfied by
Legendre polynomials, 303; satisfied by parabolic cylinder functions, 354; satisfied by
j-(.s', a), 274, 275

Infinite determinants, see Determinants

Infinite integrals, 69; convergence of, 70, 71, 72; differentiation of, 74; evaluation of, 111-124;
functions represented by, sec under the names of special functions; representing analytic
functions, 92; theorems concerning, 73; uniform convergence of, 70, 72, 73. See also
Integrals and Integration

Infinite products, 32; absolute convergence of, 32; convergence of, 32; divergence to zero, 33;
expansions of functions as, 136, 137  see also under the names of special functions); expressed
by means of Theta-functions, 473, 488; uniform convergence of, 49

Infinite series, see Series

Infinity, 11, 103; essential singularity at, 104; point at, 103; pole at, 104; zero at, 104

Integers, positive, 3; signless, 3

Integrability of continuous functions, 63; Eiemann's condition of, 63

Integral, Borel's, 140; and analytic continuation, 141

Integral, Cauchy's, 119

Integral, Dirichlet's, 258

Integral equations, 211-231 (Chapter xi); Abel's, 211, 229, 330; Fredholm's, 213-217, 228;
homogeneous, 217, 219; kernel of, 213; Liouville-Neumann method of solution of, 221;
nucleus of, 213; numbers (characteristic) associated with, 219; numerical solutions of, 211;
of the first and second kinds, 213, 221; satisfied by \Lame\ functions, 564-567; satisfied by
Mathieu functions, 407; satisfied by parabolic cylinder functions, 331; Schlomilch's, 229;
solutions in series, 228; Volterra's, 221; with variable upper limit, 213, 221

Integral formulae for ellipsoidal harmonics, 567; for the Jacobian elliptic functions, 492, 494;
for the Weierstrassian elliptic function, 437

Integral functions, 106; and Lame's equation, 571; and Mathieu's equation, 418

Integral properties of Bessel functions, 380, 381, 385; of Legendre functions, 325, 305, 324; of
Mathieu functions, 411; of Neumann's function, 385; of parabolic cylinder functions, 350

Integrals, 61-81 (Chapter iv); along curves (equivalence of), 87; complex, 77, 78; differentiation
of, 67; double, 68, 255; double-circuit, 256, 293; evaluation of, 111-124; for derivates of an
analytic function, 89; functions represented by, see under the juinies of the special functions;
imjjroper, 75; lower, 61; of harmonics (Sylvester's theorem), 400; of irrational functions,
452, 512; principal values of, 75; regular, 201; repeated,
68, 75; representing analytic functions, 92; representing areas, 61, 589; round a contour,
85; upper, 61. See also Elliptic integrals. Infinite integrals, and Integration

Integral theorem, Fourier's, 188, 211; of Fourier-Bessel, 385

Integration, 61; complex, 77; contour-, 77; general theorem on, 63; general theorem on
complex, 78; of asymptotic expansions, 153; of integi'als, 68, 74, 75; of series, 78; pro-
blem connected with cubics or quartics and elliptic functions, 452, 512. See also Infinite
integrals and Integrals

Interior, 44

Internal spheroidal harmonics, 403

Invariants of Weierstrassian elliptic functions, 437

Inverse factorials, expansions in series of, 142

Inversion of elliptic integrals, 429, 452, 454, 480, 484, 512, 524

Irrational functions, integration of, 452, 512

Irrational-real numbers, 5

%
% 602
%

Irreducible set of zeros or poles, 430

Irregular points i singularities) of differential equations, 197, 202

Iterated functions. 222

Jacobian elliptic functions [sn /  en ii, dn;(], 432, 478, 491-535 (Chapter xxii); addition theorems
for, 494, 497, 530, 535; connexion with Weierstrassian functions, 505; definitions of am,
A<p, sn M (sin am  ), en u, dn u, 478, 492, 494; differential equations satisfied by, 477, 492;
differentiation of, 493; duplication formulae for, 498; Fourier series for, 510, 511, 55.5;
geometrical illustration of, 524, 527; general description of, 504; Glaisher's notation for
quotients and reciprocals of, 494; infinite products for, 508, 53:; integral formulae for, 492,
494; Jacobi's imaginary transformation of, 505, 506; \Lame\ functions expressed in terms of,
564, 573; Landen's transfonnation of, 507; modular angle of, 492; modulus of, 479, 492,
(complementary) 479, 493; parametric representation of points on curves by, 524, 537, 527,
555; periodicity of, 479, 500, 502, 503; poles of, 432, 503, 504; quarter periods. A', iK', of,
479, 498, 499, 501; relations between, 492; residues of, 504; Seiffert's spherical spiral and,
527; triplication formulae, 530. 534, 535; values of, when ii is hK, kiK' or h  K riK'), 500,
506. 5(17; values of. when the modulus is small, 533. See ahit Elliptic functions. Elliptic
integrals. Lemniscate functions, Tbeta- functions, and Weierstrassian elliptic functions

Jacobi"s discovery of elliptic functions, 429, 512; earlier notation for Theta-functions, 479;
fundamental Theta-function fomiulae, 467, 4H8; imaginary transformations, 124, 474, 505,
506, 519, 535; Zeta-function, t ee under Zeta-function of Jacobi

Kernel. 213

Klein's theorem on linear differential equations with five singularities, 203

Kummer's formulae for confluent hypergeometric functions, 338; series for logF (z), 250

Lacunary function. 98

Lagrange's expansion, 132, 149; form for the remainder in Taylor's series, 96

Lame functions, defined, 558; expressed as algebraic functions, 556; 577; expressed by Jacobian
elliptic functions, 573-575; expressed by Weierstrassian elliptic functions, 570-572; integi'al
equations satisfied by, 564-567; linear independence of, 559; reality and distinctness of
zeros of, 557, 558, 578; second kind of, 562; values of, 558; zeros of (Stieltjes' theorem),
560. See alsi) Lames equation and Ellipsoidal harmonics

Lame's equation, 204, 536-578 (Chapter xxiii); derived from theory of ellipsoidal harmonics,
538-543, 552-554; different forms of. 554, 573; generalised,' 204, 570, 573, 576, 577;
series solutions of, 556, 577, 578; solutions expressed in finite fomi, 459, 556, 576, 577, 578;
solutions of a generalised equation in finite foim, 570, 573. See alao Lam6 functions and
Ellipsoidal harmonics

Landens transformation of Jacobian elliptic functions, 476, 507, 533

Laplace's equation, 386; its general solution, 388; normal solutions of, 553; solutions involving
functions of Legendre and Bessel, 391, 395; solution with given boundary conditions, 393;
synnnetrical solution of, 399; transformations of, 401, 407, 551, 553

Laplace's integrals for Legendre polynomials and functions, 312, 313, 314, 319, 326, 337

Laurent's expansion, 100

Least of limits. 13

Lebesgue's lemma, 172

Left (L-) class. 4

Legendre's equation, 204, 304; for associated functions, 324; second solution of, 316. See aho
Legendre functions and Legendre polynomials

Legendre functions, 302-336 (Chapter xv); P  z), Q  z), P "Uz), Q '" z) defined, 306, 316, 323,
325; addition formulae for, 328, 395; Bessel functions and, 364, 367, 401; degree of, 307,
324; differential equation for, 204, 306, 324; distinguished from Legendre polynomials,
306; expansions in a.scending series, 311, 326; expansions in descending series, 302, 317,
326, 334; expansion of a function as a series of, 334; expressed by Murphy as hypergeometric
functions, 311, 312; expression of Qn z) in terms of Legendre polynomials. 319, 320, 333;
Ferrers' functions associated with, 323. 324; first kind of, 307; Gegenbauer's function,
Cn" (z), associated with, xee Gegenbauer's function; Heine's expansion of  t - e)~i as a series
of, 321; Hobson's functions associated with, 325; integral connecting Bessel functions with,
364; integi-al properties of, 324; Laplace's integials for, 312, 313, 319, 326, 334; Mehler-
Dirichlet integral for, 314; order of, 326; recuiTence fomiulae for, 307, 318; Schlafli's
integral for, 304, 306; second kind of, 316-320, 325, 320; summation of ::ii"P (:) and
Z/f" Q  (z), 302, 321; zeros of, 303, 316, 335. See also Legendre polynomials and Legendre's
equation

Legendre polynomials [P   z)], 95, 302; addition foraiula for, 326, 387; degi-ee of, 302; differ-
ential equation for, 204, 304; expansion in ascending series, 311; expansion in descending

%
% 603
%

series, 302, 334; expansion of a function as a series of, 310, 322, 330, 331, 332, 335;
expressed by Mui-phy as a hypergeometric function, 311, 312; Heine's expansion of (t - z)~'
as a series of, 321; integi-al connecting Bessel functions with, 364; integi'al properties of,
225, 305; Laplace's equation and, 391; Laplace's integrals for, 312, 314; Mehler-Dirichlet
integi-al for, 314; Neumann's expansion in series of, 322; numerical inequality satisfied by,
303; recurrence fonnulae for, 307, 309; Eodrigues' fonnula for, 225, 303; Schlafli's integral
for, 303, 304; summation of 2/;" P  (z), 302; zeros of, 303, 316. See aim Legendre functions

Legendre's relation between complete elliptic integi-als, 520

Lemniscate functions [sin lemn and cos lemn 0], 524

LiapounofF's theorem concerning Fourier constants, 180

Limit, condition for existence of, 13

Limit of a function, 42; of a sequence, 11, 12; -point (the Bolzano -Weierstrass theorem), 12

Limiting circle, 98

Limits, greatest of and least of, 13

Limit to the value of a complex integi'al, 78

Lindemann's theory of Mathieu's equation, 417; the similar theory of Lame's equation, 570

Linear differential equations, 194-210 (Chapter x), 386-403 (Chapter xviii); exponents of, 198;
fundamental system of solutions of, 197, 200; iixegular singularities of, 197, 202; ordinary
point of, 194; regular integi-al of, 201; regular point of, 197; singular points of, 194, 197,
(confluence of) 202; solution of, 194, 197, (uniqueness of) 196; special types of equations :
- Bessel's for circular cylinder functions, 204. 342, 357, 358, 373; Gauss' for hypergeo-
metric functions, 202, 207, 283; Gegenbauer's, 329; Hermite's, 204, 209, 342, 3 7; Hill's,
406, 413; .Tacobi's for Theta-functions, 463; Lame's, 204, 540-543, .554-558, 570-575;
Laplace's, 386, 388, 536, 551; Legendre's for zonal and surface harmonics, 204, 304, 324;
Mathieu's for elliptic cylinder functions, 204, 406; Neumann's, 3\&5; Riemann's for
P-functions, 206, 283, 291, 294; Stokes', 204; Weber's for parabolic cylinder functions,
204, 209, 342, 347; Whittaker's for confluent hypergeometric functions, 337; equation for
conduction of Heat, 387; equation of Telegraphy, 387; equation of wave motions, 386, 397,
402; equations with five singularities (the Klein-Bocher theorem), 203; equations with three
singularities, 206; equations with two singularities, 208; equations with r singularities,
209; equation of the third order with regular integrals, 210

Liouville's method of solving integral equations, 221

Liouville's theorem, 105, 431

Logarithm, 583; continuity of, 583, 589; differentiation of, 586, 589; expansion of, 584, 589;
of complex numbers, 589

Logarithmic derivate of the Gamma-function [i (z)], 240, 241; Binet's integrals for, 248-251;
circular functions and, 240; Dirichlet's integi-al for, 247; Gauss' integi-al for, 246

Logarithmic derivate of the Riemann Zeta-function, 279

Logarithmic-integral function [Liz], 341

Lower integral, 61

Lunar perigee and node, motions of, 406

Maclaurin's (and Euler's) expansion, 127; test for convergence of infinite integrals, 71; series,
94, (failure of) 104, 110

Many-valued functions, 106

Mascheroni's constant [7], 235, 246, 248

Mathematical Physics, equations of, 203, 386-403 (Chapter x\ an). See aho under Linear dif-
ferential equations and the names of special equations

Mathieu functions [oc (£, q), .sp (z, q), in,  z, q)], 404-428 (Chapter xix); construction of, 409,
420; convergence of series in, 422; even and odd, 407; expansions as Fourier series, 409,
411, 420; integral equations satisfied by, 407, 409; integral formulae, 411; order of, 410;
second kind of, 427

Mathieu's equation, 204, 404-428 (Chapter xix); general form, solutions by Floquet, 412, by
Lindemann and Stieltjes, 417, by the method of change of parameter, 424; second solution
of, 413, 420, 427; solutions in" asymptotic series, 425; solutions which are periodic, .tee
Mathieu functions; the integial function associated with, 418. See also Hill's equation

Mean-value theorems, 65, 66, 96

Mehler's integral for Legendre functions, 314

Mellin's (and Barnes') type of contour integi-al, 286, 343

Membranes, vibrations of, 356, 396, 404, 405

Mesh, 430

Methods of ' summing ' series, 154-156

Minding's formula, 119

Minimum value of T (.r), 253

%
% 604
%

Modified Heine-Borel theorem, -53

Modular aiiplo, 49'i; function, 481, (equation connected with) 482; -surf.ice, 41

Modulus, 430; of a complex number, 8; of Jacobian elliptic functions, 479, 492, (complementary)

479. 493; periods of elliptic functions regarded as functions of the, 484, 498, 499, 501, 521
Monogenic, 83; distinguished from analytic, 99
Monotonic, 57

Morera's theorem (converse of Cauchy's theorem), 87, 110
Motions (if lunar perigee and node, 406
M-test for uniformity of convergence, 49

Multiplication formula for T [z), 240; for the Sigma-function, 460
Multiplication of absolutely convergent series, 29; of asymptotic expansions, 152; of convergent

series (Abel's theorem), 58, 59
Multipliers of Thcta-functions, 463
Murphy's formulae for Legendre functions and polynomials, 311, 312

Neumann's definition of Bessel functions of the second kind, 372; expansions in series of

Legendre and Bessel functions, 322, 374; (F. E. Neumann's) integral for the Legendre

function of the second kind, 320; method of solving integral equations, 221
Neumann's function [0,j (z)], 374; differential equation satisfied by, 385; expansion of, 374;

expansion of functions in series of, 376, 384; integral for, 375; integral properties of,

56.5; recurrence formulae for, 375
Non-uniform convergence. 44; and discontinuity, 47
Normal functions, 224

Normal solutions of Laplace's equation, 553
Notations, for Bessel functions, 356, 372, 373; for Legendre functions, 325, 326; for quotients

and reciprocals of elliptic functions, 494, 498; for Theta-functions, 464, 479, 487
Nucleus of an integi-al equation, 213; symmetric, 223, 228
Numbers, 3-10 (Chapter i); basic, 462; Bernoulli's, 125; Cauchy's, 379; characteristic, 219,

(reality of) 226; complex, 6; irrational, 6; irrational-real, 5; pairs of, 6; rational, 3, 4;

rational-real, 5; real, 5

Odd functions, 166; of Mathieu, [.s(' (, 7)], 407

Open. 44

Order [O and o), 11; of Bernoullian polynomials, 126; of Bessel functions, 356; of elliptic
functions, 432; of Legendre functions, 324; of Mathieu functions, 410; of poles of a
function, 102; of terms in a series, 25; of the factors of a product, 33; of zeros of a
function, 94

Ordinary discontinuity, 42

Ordinary point of a linear differential equation, 194

Orthogonal coordinates, 394; functions, 224

Oscillation, 11

Parabolic cylinder functions [/>  (2:)], 347; contour integi-al for, 349; differential equation for,
204,;i()9, 347; expansion in a power series, 347; expansion of a function as a series of, 351;
general asymptotic expansion of, 348; inequalities satisfied by, 354; integral equation
satisfied by, 231; integral properties, 350; integrals involving, 353; integrals representing,
353; properties when n is an integer, 350, 353, 354; recurrence formulae, 350; relations
between different kinds of [D,  z) and D-n-ii - i )]  348; zeros of, 354. See also Weber's
equation

Parallelogram of periods, 430

Parameter, change of (method of solving Mathieu's equation), 424; connected with Theta-
functions, 463, 464; of a point on a curve, 442, 496, 497, 527, 530, 533; of members of
confocal systems of quadrics, 547; of third kind of elliptic integral, 522; thermometric, 405

Parse val's theorem, 182

Partial differential equations, property of, 390, 391. See (iIko Linear differential equations

Partition function. 462

Parts, real an<l imaginary, 9

Pearson's function [w,, (z)], 353

P-equation, Riemann's, 206, 337; connexion with the hypergeometric equation, 208, 283; solu-
tions of, 2S3, 291, (relations between) 294; transformations of, 207

Periodic coefficients, equations with (Floquet's theory of), 412

Periodic functions, integrals involving, 256. See also Fourier series and Doubly periodic
functions

%
% 605
%
Periodicity factors, 463

Periodicity of circular and exponential functions, 585-587; of elliptic functions, 429, 434, 479,

500, 502, 503; of Theta-funetions, 463
Periodic solutions of Mathieu's equation, 407
Period-parallelogram, 430; fundamental, 430

Periods of elliptic functions, 429; qua functions of the modulus, 484, 498, 499, 501, 521
Phase, 9

Pincherle's functions (modified Legendre functions), 335
Plana's expansion, 145

Pochhammer's extension of Eulerian integrals, 256

Point, at infinity, 103; limit-, 12; representative, 9; singular, 194, 202
Poles of a function, 102; at infinity, 104; irreducible set of, 430; number in a cell, 431; relations

between zeros of elliptic functions and, 433; residues at, 432, 504; simple, 102
Polygon, (fundamental) of automoi-phic functions, 455
Polynomials, expi'essed as series of Legendre polynomials, 310; of Abel, 333; of Bernoulli, 126,

127; of Legendre, xce Legendre polynomials; of Sonine, 352
Popular conception of an angle, 589; of continuity, 41
Positive integers, 3

Power series, 29; circle of convergence of, 30; continuity of, 57, (Abel's theorem) 57; expan-
sions of functions in, xee under the name  of special functiomi; identically vanishing, 58;
Maclaurin's expansion in, 94; radius of convergence of, 30, 32; series derived from, 31;
Taylor's expansion in, 93; uniformity of convergence of, 57

Principal part of a function, 102; solution of a certain equation,
482; value of an integral, 75;  value of the argument of a complex
number, 9, 588

Principle of convergence, 13

Pringsheim's theorem on summation of double series, 28
Products of Bessel functions, 379, 380, 383, 385, 428; of hypergeometric functions, 298. See

also Infinite products

Quarter periods K, iK', 479, 498, 499, 501. See also Elliptic integrals

Quartic, canonical form of, 513; integi'ation problem connected with, 452, 512

Quasi-periodicity, 445, 447, 463

Quotients of elliptic functions (Glai her's notation), 494, 511; of Theta-f unctions, 477

Radius of convergence of power series, 30, 32

Rational functions, 105; expansions in series of, 134

Rational numbers, 3, 4; -real numbers, 5

Real functions of real variables, 56

Reality of characteristic numbers, 226

Real numbers, rational and irrational, 5

Real part (li) of a complex number, 9

Rearrangement of convergent series, 25; of double series, 28; of infinite determinants, 37; of
infinite products, 33

Reciprocal functions, Volterra's, 218

Reciprocals of elliptic functions (Glaisher's notation), 494, 511

Recurrence formulae, for Bessel functions, 359, 373, 374; for confluent hypergeometric functions,
352; for Gegenbauer's function, 330; for Legendre functions, 307, 309, 318; for Neumann's
function, 375; for parabolic cylinder functions, 350. See also Contiguous hypergeometric
functions

Region, 44

Regvilar, 83; distribution of discontinuities, 212; integrals of linear differential equations, 201,
(of the third order) 210; points (singularities) of linear differential equations, 197

Relations between Bessel functions, 360, 371; between confluent hypergeometric functions
Tr\; . jjj (± ) and M/ .    z, 346; between contiguous hypergeometric functions, 294; be-
tween elliptic functions, 452; between parabolic cylinder functions D,j ( ± z) and D\  \ i ( ± iz),
348; between poles and zeros of elliptic functions, 433; between Riemann Zeta-f unctions
f (s) and f (1 - s), 269. See also Recurrence formulae

Remainder after *; terms of a series, 15; in Taylor's series, 95

Removable discontinuity, 42

Repeated integrals, 68, 75

Representative point, 9

Residues, 111-124 (Chapter vi); of elliptic functions,
425, 497

%
% 606
%
Riemann's associated function, 183, 184, 185; condition of integrability, 63; equations satisfied
by analytic functions, S4; hypothesis concerning f(j,), 272, 280; lemmas, 172, 184, 185;
/'-equation, 206, 283, 291, 294, (transformation of) 207, (and the hypergeometric equation)
208, nee (tho Hypergeometric functions; theory of trigonometiical series, 182-188; Zeta-
function, xer Zeta-function (of Riemann)

Riesz' method of ' summing ' series, 156

Right  U-) class, 4

Rodrigues' formula for Legendre polynomials, 303; modified, for Gegenbauer's function, 329

Roots of an equation, number of, (inside a contour) 123; of Weierstrassian elliptic
functions (<'i . eo, e ), 443

Saalschiitz' integral for the Gamma-function, 243

Schlafli'-s Bessel function of the second kind, [r  (2)], 870

Sclilafli's integral for Bessel functions, 362, 372; for Legendre polynomials and functions, 303,

304, 306; modified, for Gegenbauer's function, 329
Schlomilch's expansion in series of Bessel coefficients, 377; function, 352; integi-al equation, 229
Schmidt's theorem, 223
Schwarz" lemma, 186

Second kind. Bessel function of, (Hankel's) 370, (Neumann's) 372, (Weber-Schliifli), 370,
(modified) 373; elliptic integral of [-E (m), Z ((/)], 517, (complete) 518; Eulerian integral of,
241, (extended) 244; integi-al equation of, 213, 221; \Lame\ functions of, 562; Legendre
functions of, 316-320, 325, 326
Second mean-value theorem, 66

Second solution of Bessel's equation, 370, 372, (modified) 373; of Legendre's equation, 316; of
Mathieu's equation, 413, 427; of the hypergeometric equation, 286, (confluent form) 343; of
Weber's equation, 347
Second species of ellipsoidal harmonics, 537, (construction of) 540
Section, 4

Seififerfs spherical spiral, 527
Sequences, 11; decreasing, 12; increasing, 12

Series (infinite series), 15: absolutely convergent, 18; change of order of terms in, 25; con-
ditionally convergent, 18; convergence of, 15; differentiation of, 31, 79, 92; divergence of,
15  geometric, 19; integration of, 32, 78; methods of summing, 154-156; multiplication
of, 29, 58, 59; of analytic functions, 91; of cosines, 185; of cotangents, 139; of inverse
factorials, 142; of powers, see Power series; of rational functions, 134; of sines, 166; of
variable terms, 44 (see also Uniformity of convergence); order of terms in, 25; remainder of,
15; representing particular functions, see ii) iler tlie name of the fitiietioit; solutions of
differential and integi'al equations in, 194-202, 228; Taylor's, 93. Sec also Asymptotic
expansions. Convergence, Expansions, Foiirier series. Trigonometrical series and Uniformity
of convergence
Set, Irreducible (of zeros or poles), 430

Sigma-functions of Weierstrass [(t(z), <ri z), 0-2(2), 0-3(2)], 447, 448; addition formula for, 451,
458, 460; analogy -with circular functions, 447; duplication formulae, 459, 460; four
types of, 448; expression of elliptic functions by, 450; quasi-periodic properties, 447;
singly infinite product for, 448; three-term equation involving, 451, 461; Theta-functions
connected with, 448, 473, 487; triplication formula, 459
Signless integers, 3
Simple curve, 43; pole, 102; zero, 94
Simply-connected region, 455

Sine, product for, 137. See also Circular functions
Sine-integi-al [Si (2)], 352; -series (Fourier scries), 166 .
Singly-periodic functions, 429. See also Circular functions

Singularities, 83, 84, 102, 194, 197, 202; at infinity, 104; confluence of, 203, 337; equations
with five, 203; equations with three, 206, 210; equations with two, 208; equations with r,
209; essential, 102, 104; irregular, 197, 202; regular, 197
Singular points (singularities) of linear differential equations, 194, 202
Solid harmonics. 392

Solution of Riemann's P-equation by hypergeometric functions, 283, 288
Solutions of differential equations, see Chapters x, xviii, xxiii, and under the names of special

iiiuiitions
Solutions of integral equations, see Chapter xi
Sonine s polynomial ['/'," (2)], 352
Species (various) of ellipsoidal harmonics, 537

%
% 607
%

Spherical harmonics, see Harmonics

Spherical spiral, Seiffert's, 527

Spheroidal harmonics, 403

Squares of Bessel functions, 379, 380; of liypergeometric functions, 298; of Jacobian elliptic

functions (relations between), 492; of Theta-f unctions (relations between), 466
Statement of Fourier's theorem, Dirichlet's, 164, 176
Steadily tending to zero, 17
Stieltjes' theorem on zeros of \Lame\ functions, 560, (generalised) 562; theory of Mathieu's

equation, 417
Stirling's series for the Gamma-function, 251
Stokes' equation, 204

Stolz' condition for convergence of double series, 27
Successive substitutions, method of, 221
Sum-formula of Euler and Maclaurin, 127

Summability, methods of, 154-156; of Fourier series, 169; uniform, 156
Surface harmonic, 392
Surface, modular, 41
Surfaces, nearly spherical, 332

Sylvesters theorem concerning integi-als of harmonics, 400
Symmetric nucleus, 223, 228

Tabulation of Bessel functions, 378; of complete elliptic integi-als, 518; of Gamma-functions, 253

Taylor's series, 93; remainder in, 95; failure of, 100, 104, 110

Teixeira's extension of Biirmann's theorem, 131

Telegraphy, equation of, 387

Tesseral harmonics, 392; factorisation of, 536

Tests for convergence, see Infinite integrals, Infinite products and Series

Thermometric parameter, 405

Theta-functions [ i (;), S-o (z), \&3  z),  4  z) or   (,-), 9 ( )], 462-490 (Chapter xxi); abridged nota-
tion for products, 468, 469; addition formulae, 467; connexion with Sigma- functions, 448,
473, 487; duplication formulae, 488; expression of elliptic functions by, 473; four types
of, 463; fundamental formulae (Jacobi's), 467, 488; infinite products for, 469, 473, 488;
Jacobi's first notation, G ( ) and H (;/), 479; multipliers, 463; notations, 464, 479, 487;
parameters q, t, 463; jmrtial differential equation satisfied by, 470; periodicity factors,
463; periods, 463; quotients of, 477; quotients yielding Jacobian elliptic functions, 478;
relation S-i' = a-jS 3  4, 470; squares of (relations between), 466; transformation of, (Jacobi's
imaginary) 124, 474, (Landen's) 476; triplication formulae for, 490; with zero argument
( 9,  :j,  4,  1'), 464; zeros of, 465

Third kind of elliptic integral, IT (u, a), 522; a dynamical application of, 523

Third order, linear differential equations of, 210, 298, 418, 428

Third species of ellipsoidal harmonics, 537, (construction of) 541

Three kinds of elliptic integi-als, 514

Three-term equation involving Sigma-f unctions, 451, 461

Total fluctuation, 57

Transcendental functions, see under the names of special functions

Transformations of elliptic functions and Theta-functions, 508; Jacobi's imaginary, 474, 505,
506, 519; Landen's, 476, 507; of Eiemann's P-equation, 207

Trigonometrical equations, 587, 588

Trigonometrical integrals, 263; and Gamma-functions, 256

Trigonometrical series, 160-193 (Chapter ix); convergence of, 161; values of coefficients in, 163;
Eiemann's theory of, 182-188; which are not Fourier series, 160, 163. See also Fourier series

Triplication formulae for Jacobian elliptic functions and E  u), 530,534; for Sigma-functions,
459; for Theta-functions, 490; for Zeta-functions, 459

Twenty-four solutions of the hypergeometric equation, 284; relations between, 285, 288, 290

Two-dimensional continuum, 43

Two variables, continuous functions of, 67; hypergeometric functions (Appell's) of, 300

Types of ellipsoidal hannonics, 587

Unicursal, 455
Uniformisation, 454

%
% 608
%

Uniformising variables, 455; associated with confocal coordinates, 549

Uniformity, concept of, 52

Uniformity of continuity, 54; of sumniability, 156

Uniformity of convergence. 41-60 (Chapter iii), defined, 44; of Fourier series, 172, 179, 180; of

intiuite intei;nils. 70. 72, 73; of infinite products, 49; of power series, 57; of series, 44,

(condition for) 45, (Hardy's test for) 50, (Weierstrass'ili-test for) 49
Uniformly convergent infinite integrals, properties of, 73; series of analytic functions, 91,

(differentiation of) 92
Uniqueness of an asymptotic expansion, 153; of solutions of linear differential equations, 196
Upper bound, 55; integi-al, 61
Upper limit, integral equation with variable, 213, 221; to the value of a complex integi-al, 78, 91

Value, absolute, i ee Modulus; of the argument of a complex number, 9, 588; of the coefficients
in Fourier series and trigonometrical series, 163, 165, 167, 174; of particular hypergeometric
functions, 281, :i93, 298, 301; of Jacobian elliptic functions of  A',  iK', h(K+iK'), 500,
506, 507; of K, K' for special values of A", 521, 524, 525; of  (a) for special values of . *,
267, 269

Vanishing of power series, 58

Variable, uniformising, 455; terms (series of), see Uniformity of convergence; upper limit,
integral equation with, 213, 221

Vibrations of air in a sphere, 399; of circular membranes, 396; of elliptic membranes, 404, 405

Volterra's integral equation, 221; reciprocal functions, 218

Wave motions, equation of, 386; general solution, 397, 402; solution involving Bessel functions,
3S)7

Weber's Bessel function of the second kind [r, ( )], 370

Weber's equation, 204. 209, 342, 347. See also Parabolic cylinder functions

Weierstrass' factor theorem, 137; il/-test for uniform convergence, 49; product for the Gamma-
function, 235; theorem on limit points, 12

Weierstrassian elliptic function [  >( )], 429-461 (Chapter xx), defined and constructed, 432,

433; addition theorem for, 440, (Abel's method) 442; analogy with circular functions,
438; definition of \   [z) - e ], 451; differential equation for, 436; discriminant of, 444;
duplication formula, 441; expression of elliptic functions by, 448; expression of ip (z) -   (y)
by Sigma-functions, 451; half-periods, 444; homogeneity properties, 439; integral formula
for, 437; integi-ation of in-ational functions by, 452; invariants of, 437; inversion problem
for, 484; Jacobian elliptic functions and, 505; periodicity, 434; roots fj, e.,, e., 443. See
also Sigma-fimctions and Zeta-f unction (of Weierstrass)

Whittaker's function TFj., (£), see Confluent hypergeometric functions

Wronskis expansion, 147

Zero argument, Theta-f unctions with, 464; relation between, 470

Zero of a function, 94; at infinity, 104; simple, 94

Zeros of a function and poles (relation between), 438; connected with zeros of its derivative,
123; in-educible set of, 430; number of, in a cell, 431; order of, 94

Zeros of fimctions, (Bessel's) 361, 367, 378, 381, (Lame's) 557, 558, 560, 578, (Legendre's) 303,
316, 335, (parabolic cylinder) 354, (Eiemann's Zeta-) 268, 269, 272, 280, (Theta-) 465

Zeta-function, Z("). (of Jacobi), 518; addition formula for, 518; connexion with E u), 518;
Fourier series for, 520; Jacobi's imaginary transformation of, 519. See also Jacobian
elliptic functions

Zeta-fimction, j'(*), i'(s,a), (of Riemann) 265-280 (Chapter xiii), (generalised by Hurwitz) 265;
Euler's product for, 271; Hermite's integral for, 269; Hurwitz' integral for, 268; in-
equalities satisfied by, 274, 275; logarithmic derivate of, 279; Eiemann's hypothesis
concerning, 272, 280; Eiemann's integrals for, 266, 273; Eiemann's relation connecting f (s)
and f (1 - * ), 269; values of, for special values of .s, 267, 269; zeros of, 268, 269, 272, 280

Zeta-function, f(2), (of Weierstrass), 445; addition formula, 446; analogy with circular
functions, 446; constants t/j, t)., connected with, 446; duplication formulae for, 459; ex-
pression of elliptic functions by, 449; quasi-periodicity, 445; triplication formulae, 459.
See also Weierstrassian elliptic functions

Zonal harmonics, 302, 392; factorisation of, 536
\printindex

\end{document}